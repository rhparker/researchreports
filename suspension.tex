\documentclass[12pt]{article}
\usepackage[pdfborder={0 0 0.5 [3 2]}]{hyperref}%
\usepackage[left=1in,right=1in,top=1in,bottom=1in]{geometry}%
\usepackage[shortalphabetic]{amsrefs}%
\usepackage{amsmath}
\usepackage{enumerate}
% \usepackage{enumitem}
\usepackage{amssymb}                
\usepackage{amsmath}                
\usepackage{amsfonts}
\usepackage{amsthm}
\usepackage{bbm}
\usepackage[table,xcdraw]{xcolor}
\usepackage{tikz}
\usepackage{float}
\usepackage{booktabs}
\usepackage{svg}
\usepackage{mathtools}
\usepackage{cool}
\usepackage{url}
\usepackage{graphicx,epsfig}
\usepackage{makecell}
\usepackage{array}

\def\noi{\noindent}
\def\T{{\mathbb T}}
\def\R{{\mathbb R}}
\def\N{{\mathbb N}}
\def\C{{\mathbb C}}
\def\Z{{\mathbb Z}}
\def\P{{\mathbb P}}
\def\E{{\mathbb E}}
\def\Q{\mathbb{Q}}
\def\ind{{\mathbb I}}

\DeclareMathOperator{\spn}{span}
\DeclareMathOperator{\ran}{range}

\graphicspath{ {suspension/} }

\newtheorem{lemma}{Lemma}
\newtheorem{theorem}{Theorem}
\newtheorem{corollary}{Corollary}
\newtheorem{definition}{Definition}
\newtheorem{assumption}{Assumption}
\newtheorem{hypothesis}{Hypothesis}

\newtheorem{notation}{Notation}

\begin{document}

\section{Suspension Bridge Equation}

\subsection{Background}

We look at the suspended beam equation from Chen97. The idea is to look at an equation which has a second order time derivative which also possesses multimodal solutions. This is suggested on p. 2095 of SanStrut and p. 351 of Chen97.\\

The equation we will look at is

\begin{equation}\label{susp}
u_{tt} + u_{xxxx} + e^{u - 1} - 1 = 0
\end{equation}

Note that the equation used for most of the analysis in Chen97 is 

\begin{equation}\label{susp2}
u_{tt} + u_{xxxx} + u^+ - 1 = 0
\end{equation}

which has a cusp at $u = 0$. On p. 351 of Chen97, Chen and McKenna note that they have proved existence of solutions only to \eqref{susp2}, not to \eqref{susp}, but there is strong numerical evidence that similar solutions exist for \eqref{susp}. At some point, we will look and see if this has been addressed. For now, we will say that the numerical evidence is ``good enough''.\\

We are interested in traveling wave solutions, so we take a modified form of the usual traveling wave ansatz. Letting $\xi = x - ct$, we take

\begin{equation}
u(x, t) = z(\xi, t) + 1 = z(x - ct, t) + 1
\end{equation}

Substituting this into \eqref{susp} gives us

\begin{equation*}
z_{tt} - c z_{\xi t} + c^2 z_{\xi \xi} + z_{\xi \xi \xi \xi} + e^{z} - 1 = 0
\end{equation*}

Since it's annoying to use $z$ and $\xi$, we rewrite this using $u$ and $x$ to get our traveling frame suspended beam equation

\begin{equation}
u_{tt} - c u_{x t} +  + c^2 u_{xx} + e^{u} - 1 = 0
\end{equation}

For an equilibrium solution (such as a homoclinic orbit), all time derivatives are zero, so any equilibrium solution must satisfy the ODE

\begin{equation}
u_{xxxx} + c^2 u_{xx} + e^{u} - 1 = 0
\end{equation}

which is (46) on p. 342 of Chen97, with $\tilde{f}(u) = e^u - 1$.



\end{document}