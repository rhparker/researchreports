% \documentclass{book}

\documentclass[12pt]{article}
\usepackage[pdfborder={0 0 0.5 [3 2]}]{hyperref}%
\usepackage[left=1in,right=1in,top=1in,bottom=1in]{geometry}%
\usepackage[shortalphabetic]{amsrefs}%
\usepackage{amsmath}
\usepackage{enumerate}
\usepackage{enumitem}
\usepackage{amssymb}                
\usepackage{amsmath}                
\usepackage{amsfonts}
\usepackage{amsthm}
\usepackage{bbm}
\usepackage[table,xcdraw]{xcolor}
\usepackage{tikz}
\usepackage{float}
\usepackage{booktabs}
\usepackage{svg}
\usepackage{mathtools}
\usepackage{cool}
\usepackage{url}
\usepackage{graphicx,epsfig}
\usepackage{makecell}
\usepackage{array}

\def\noi{\noindent}
\def\T{{\mathbb T}}
\def\R{{\mathbb R}}
\def\N{{\mathbb N}}
\def\C{{\mathbb C}}
\def\Z{{\mathbb Z}}
\def\P{{\mathbb P}}
\def\E{{\mathbb E}}
\def\Q{\mathbb{Q}}
\def\ind{{\mathbb I}}

\graphicspath{ {images17/} }

\newtheorem{lemma}{Lemma}
\newtheorem{definition}{Definition}

\begin{document}

\section*{24 August 2017}

\subsection*{Eigenfunction symmetry}

In \texttt{KdV11} we looked at even/odd symmetry of eigenfunctions for the case where we have a purely imaginary eigenvalue. Here we look at the general case. The linearized 5th order problem about the stationary solution $u^*(x)$ is $Lv = \partial_x Hv = \lambda v$, where $L = \partial_x( \partial_x^4 - \partial_x^2 + c - 2u^*)$. We want to consider the eigenvalue problem $Lv(x) = \lambda v(x)$. \\

Suppose we have an eigenfunction $v(x)$ of $L$ with corresponding eigenvalue $\lambda$. The following have already been shown.
\begin{enumerate}
	\item $\bar{v}(x)$ is an eigenfunction of $L$ corresponding to eigenvalue $\bar{\lambda}$:
	\item Assuming that the stationary solution $u^*(x)$ is an even function (which it is for our case), $v(-x)$ is an eigenfunction of $L$ corresponding to eigenvalue $-\lambda$.
\end{enumerate}
Write the eigenfunction as $v(x) = u(x) + i w(x)$, where $u(x)$ and $w(x)$ are both real. Also write our eigenvalue as $\lambda = \alpha + i \beta$. If we substitute these into the eigenvalue problem and equate real/imaginary parts, we get
\begin{align*}
Lu &= \alpha u - \beta w \\
Lw &= \alpha w + \beta u
\end{align*}
We considered before the special case where $\alpha = 0$. The real and imaginary parts of $v$ are related. One relationship is given by
\[
w = -\frac{1}{\beta}(Lu - \alpha u)
\]





\subsection*{Integrated Eigenfunction Construction (revised yet again)}
For the 5th order KdV equation (written in traveling frame), assume for a specific value of $c$ (which we know is greater than 1/4) we have constructed a 2-pulse $q_2(x)$. The 1-pulse is given by $q(x)$. Then we can find a real number $L$ so that we can write $q_2(x)$ piecewise as:
\begin{equation}\label{q2piecewise}
\begin{cases}
q(x) + r_1(x) & \text{on } (-\infty, L] \\
q(x) + r_2(x) & \text{on } [-L, \infty)
\end{cases}
\end{equation}

where the two pieces are spliced together one after the other so that $\pm L$ corresponds to 0 and the resulting function (and all derivatives) are continuous at the splice point. Note that this a special case of Theorem 1 in Sandstede (1998), since we are not varying a parameter to break a homoclinic orbit, but rather are perturbing that homoclinic orbit directly. This has the advantage that we only have one splice point, rather than the case in Sandsted e (1998) where we break the homoclinic orbit into two pieces $q^-$ and $q^+$ by altering a parameter $\mu$.
\\

We have the single matching condition

\begin{align*}
q(L) + r_1(L) &= q(-L) + r_2(-L)
\end{align*}

In our case, we know the single pulse is an even function, so $q(L) = q(-L)$. Thus the matching condition reduces to $r_1(L) = r_2(-L)$. The matching condition holds for all derivatives as well. \\

Consider the eigenvalue problem as before. Linearizing about the solution $q_2(x)$ we get the eigenvalue problem $\partial_x H v = \lambda v$, where 
\begin{equation}\label{hamiltonian}
H = \partial_x^4 - \partial_x^2 + c - 2 q_2(x)
\end{equation}

Now integrate both sides of the eigenvalue problem from $a$ to $x$. This is the integrated eigenvalue problem. The value of $a$ will be determined later. We will assume that any solution $v$ is localized, i.e. we will seek solutions $v(x)$ such that the function and all its derivatives decay to 0 exponentially as $x \rightarrow \pm \infty$. There's probably a nice function space with this property, but we won't worry about that for now. Thus the integrated eigenvalue problem becomes

\begin{equation}\label{inteigproblem}
(Hv)(x) - (Hv)(a) = \lambda \int_{a}^x v(y) dy
\end{equation}

Asssuming that $v(x)$ has the decay properties we mentioned above, if we take $a = -\infty$ and let $x \rightarrow \infty$ (using the DCT on the integral on the RHS, since $v$ is integrable), we get

\[
\lambda \int_{-\infty}^\infty v(y) dy = 0
\]
So the eigenfunction has mean 0 if the eigenvalue is nonzero. As per our discussion in the periodic case, I don't think this gets us anything new.\\

We want to write this as a first-order system, but before we do that, we will split up the eigenfunction $v(x)$ into pieces similar to what we did with $q_2(x)$. The idea here is that we can choose a different lower limit of integration for each piece and want to do this in a way that minimizes the notational mess. We will use the same piecewise domain as we did with $q_2(x)$. The derivative of the integrals will be correct on each piece, and if we have the appropriate matching condition, we will be all set. \\

We split up $v(x)$ as follows:

\begin{equation}
\begin{cases}
v_1(x) & \text{on } (-\infty, L]  \\
v_2(x) & \text{on } [-L, \infty)
\end{cases}
\end{equation}

where we have a matching condition $v_1(L) = v_2(-L)$. Again, this matching condition must hold for all derivaties which exist. \\

Now we write the integrated eigenvalue problem for each piece. Since we seek solutions which decay at $\pm \infty$, we will use those as the endpoints $a$ for our integration, so $(Hv)(a) = 0$ in those cases.

\begin{equation}
\begin{cases}
(Hv_1)(x) = \lambda \int_{-\infty}^x v_1(y) dy & x \in (-\infty, L] \\
(Hv_2)(x) = \lambda \int_{\infty}^x v_2(y) dy & x \in [-L, \infty) \\ 
\end{cases}
\end{equation}

Now we can write these piecewise equations as a first-order system:
\[
\begin{pmatrix}v_i\\(v_i)_x\\(v_i)_{xx}\\(v_i)_{xxx}\end{pmatrix}_x = 
\begin{pmatrix}0 & 1 & 0 & 0 \\ 0 & 0 & 1 & 0 \\ 0 & 0 & 0 & 1 \\ 2q_2 - c & 0 & 1 & 0\end{pmatrix}
\begin{pmatrix}v_i\\(v_i)_x\\(v_i)_{xx}\\(v_i)_{xxx}\end{pmatrix} + \lambda
\begin{pmatrix}0\\0\\0\\\int_{a}^x v_i(y) dy\end{pmatrix}
\]
Write our integrated eigenvalue problem in matrix form as
\[
V_i' = A(q_2) V_i + \lambda B K_i V_i
\]
where $A(q_2)$ is the first matrix on the RHS above, and $B$ is the matrix
\[
\begin{pmatrix}0 & 0 & 0 & 0 \\0 & 0 & 0 & 0 \\0 & 0 & 0 & 0 \\1 & 0 & 0 & 0 \end{pmatrix}
\]

which moves the first component to fourth and gets rid of everything else. The $K_i$ are integration operators, which are given by:

\begin{align*}
(K_1 U)(x) &= \int_{-\infty}^x U(y) dy && x \in (-\infty, L] \\
(K_2 U)(x) &= \int_{\infty}^x U(y) dy && x \in [-L, \infty) \\
\end{align*}

We have the matching condition as before: $V_1(L) = V_2(-L)$, which matches $v_1(L)$ and $v_2(-L)$ up to the third derivative. Assuming we have done that, the condition to match the fourth derivative is the familiar mean-zero condition (recall how the splicing is done, i.e. the integrals below are the left and the right piece of the integrated eigenfunction):

\[
(v_1)_{xxxx}(L) - (v_2)_{xxxx}(-L) = \lambda \left( \int_{-\infty}^L v_1(y) dy + \int_{-L}^\infty v_2(y)dy \right)
\]

We recall that the derivative of the double pulse $Q_2'(x)$ is an eigenfunction with eigenvalue 0, so for small $\lambda$, our eigenfunction should be a small perturbation of this. As in (3.5) in Sandstede (1998), we write

\begin{equation}
V_i(x) = d_i Q_2'(x) + W_i(x) = d_i (Q'(x) + R_i'(x)) + W_i(x)
\end{equation}

Now we plug this into the integrated eigenvalue problem to get equations for the $W_i$. We use the fact that $Q_2'$ solves the original eigenvalue problem with $\lambda = 0$. We have removed the dependence on $x$ for convenience

\begin{align*}
(d_i Q_2' + W_i)' &= A(Q_2) (d_i Q_2' + W_i) + \lambda B K_i (d_i Q_2' + W_i) \\
W_i' &= A(Q_2) W_i + \lambda d_i B K_i Q_2' + \lambda B K_i W_i \\
&= A(Q + R_i) W_i + \lambda d_i B K_i Q_2' + \lambda B K_i W_i \\
&= A(Q) W_i + A(Q + R_i) W_i  - A(Q) W_i + \lambda d_i B K_i Q_2' + \lambda B K_i W_i 
\end{align*}

This is basically in the format we want, although we can integrate the derivative $Q_2'$ to make things simpler. By the fundamental theorem of calculus and the fact that $q_2$ decays at $\pm \infty$ we have

\begin{align*}
(K_1 Q_2')(x) &= Q_2(x) \\
(K_2 Q_2')(x) &= Q_2(x)
\end{align*}

Substituting these into the above, we get

\begin{align*}
W_i' &= A(Q) W_i + G_i W_i + \lambda d_i B Q_2 + \lambda B K_i W_i
\end{align*}

where

\[
G_i(x) = A(Q(x) + R_i(x))  - A(Q(x)) \\
\]
For the matching condition, we have

\begin{align*}
W_1(L) - W_2(-L) = D_1 d
\end{align*}

where
\begin{align*}
D_1 d &= d_2 Q_2'(-L) - d_1 Q_2'(L)\\
&= d_2 [ Q'(-L) + R_2'(-L)] - d_1 [ Q'(L) + R_1'(L) ]
\end{align*}

Putting all this together, here is our system.

\begin{align}\label{system}
W_i'(x) &= A(Q(x)) W_i(x) + G_i(x) W_i(x) + \lambda B K_i(x) W_i(x) + \lambda d_i B Q_2(x)  \\
W_1(L) - W_2(-L) &= D_1 d
\end{align}

where
\begin{align}
G_i(x) &= A(Q(x) + R_i(x)) - A(Q(x)) \\
D_1 d &= d_2 [ Q'(-L) + R_2'(-L)] - d_1 [ Q'(L) + R_1'(L) ]
\end{align}

Note that this is very similar to the system (3.7), (3.8) in Sandstde (1998). The main differences are (i) a much simpler system, since only two equations and one join; and (ii) the presence of an integration operator.\\

In order to keep going, we will need some estimates like in Lemma 3.1 in Sandstede (1998)

\begin{lemma}We have the estimates
\begin{align*}
|G_i(x)| &\leq C|R_i(x)| \leq C \sup_{|x| \geq L} |Q(x)| \\
| B Q_2(x) - B Q(x) | & \leq C |R_i(x)| \leq C \sup_{|x| \geq L} |Q(x)| \\
D_1 d &= (Q'(L) + Q'(-L))(d_2 - d_1) +\mathcal{O}\left( e^{-\alpha L} |d| \sup_{|x| \geq L} |Q(x)| \right)
\end{align*}
where $\alpha > 0$ is defined as on pages 432 and 434 of Sandstede (1998).
\begin{proof}
The first estimate is the same as in Sandstede (1998) with $Q$ replacing $Q^+$ and $Q^-$, and follows from the smoothness of $A$ together with (2.6)(i) in Sandstede (1998). The second estimate follows from (2.6)(i) in Sandstede (1998) and the expansion of $Q^2$ as $Q + R_i$. The third estimate is as in Lemma 3.1 of Sandstede (1998).
\end{proof}
\end{lemma}



\end{document}