\documentclass[12pt]{article}
\usepackage[pdfborder={0 0 0.5 [3 2]}]{hyperref}%
\usepackage[left=1in,right=1in,top=1in,bottom=1in]{geometry}%
\usepackage[shortalphabetic]{amsrefs}%
\usepackage{amsmath}
\usepackage{enumerate}
% \usepackage{enumitem}
\usepackage{amssymb}                
\usepackage{amsmath}                
\usepackage{amsfonts}
\usepackage{amsthm}
\usepackage{bbm}
\usepackage[table,xcdraw]{xcolor}
\usepackage{tikz}
\usepackage{float}
\usepackage{booktabs}
\usepackage{svg}
\usepackage{mathtools}
\usepackage{cool}
\usepackage{url}
\usepackage{graphicx,epsfig}
\usepackage{makecell}
\usepackage{array}

\def\noi{\noindent}
\def\T{{\mathbb T}}
\def\R{{\mathbb R}}
\def\N{{\mathbb N}}
\def\C{{\mathbb C}}
\def\Z{{\mathbb Z}}
\def\P{{\mathbb P}}
\def\E{{\mathbb E}}
\def\Q{\mathbb{Q}}
\def\ind{{\mathbb I}}

\DeclareMathOperator{\spn}{span}
\DeclareMathOperator{\ran}{ran}

\newtheorem{lemma}{Lemma}
\newtheorem{theorem}{Theorem}
\newtheorem{corollary}{Corollary}
\newtheorem{definition}{Definition}
\newtheorem{assumption}{Assumption}
\newtheorem{hypothesis}{Hypothesis}

\newtheorem{notation}{Notation}

\graphicspath{ {discrete/} }

\begin{document}

\section{Multipulses in Discrete Systems}

\subsection{Existence for dNLS}

This is perhaps the simplest case, so we do it first. We are interested in real-valued multi-pulse equilibrium solutions to dNLS. These will satisfy the difference equation

\[
\epsilon(u_{n+1} - 2 u_n + u_{n-1}) - \omega u_n + u_n^3 = 0
\]

where $u_n$ is real. Letting $\tilde{u} = u_{n-1}$, this becomes

\begin{align*}
\begin{pmatrix}u \\ \tilde{u} \end{pmatrix}_{n+1} =
\begin{pmatrix}2 u - \tilde{u} + \dfrac{\omega u - u^3}{\epsilon} \\ u \end{pmatrix}_n
\end{align*}

Suppose we have a symmetric, homoclinic orbit solution $q(n)$. Then we know by the symmetry of dNLS that $-q(n)$ is also a symmetric, homoclinic orbit solution. Linearizing about this, we obtain the variational equation

\begin{align*}
\begin{pmatrix}v \\ \tilde{v}\end{pmatrix}_{n+1} =
\begin{pmatrix}2 + \dfrac{\omega}{\epsilon} - \dfrac{3q^2}{\epsilon} & -1  \\
1 & 0 \end{pmatrix}
\begin{pmatrix}v \\ \tilde{v}\end{pmatrix}_n
\end{align*}

and the adjoint variational equation

\begin{align*}
\begin{pmatrix}w \\ \tilde{w} \end{pmatrix}_n =
\begin{pmatrix}2 + \dfrac{\omega}{\epsilon} - \dfrac{3q^2}{\epsilon} & 1  \\
-1 & 0 \end{pmatrix}
\begin{pmatrix}w \\ \tilde{w} \end{pmatrix}_{n+1}
\end{align*}

In this case, I don't think the variational equation has any bounded solution, so the stable and unstable manifolds of 0 intersect transversely.\\

The problem we want to solve is of the form 

\begin{equation}\label{diffeq}
U(n+1) = F(U(n))
\end{equation}

where $U \in \R^2$. Instead of looking at this, we will look at a more general form for which this will be a specific case.

\subsection{Existence, transverse intersection version}

\subsubsection{Setup and Main Theorem}

Consider the lattice PDE

\begin{equation}\label{latticePDE}
\dot{u}(n) = f(u(n))
\end{equation}

where $u(n) \in \R$ and $f$ involves a finite number (greater than 1) of indices near $n$. An equilibrium solution satisfies $f(u_n) = 0$. We can rewrite the stationary equation in system form as

\begin{equation}\label{diffeq}
U(n+1) = F(u(n))
\end{equation}

where $U \in R^d$ for some $d > 1$ which depends on how many indices near $n$ are used by $f$. For this case, we take the following assumptions.

\begin{hypothesis}\label{initialhyp}
We assume the following regarding \eqref{diffeq}.
\begin{enumerate}[(i)]
\item $F$ is a diffeomorphism.
\item 0 is a hyperbolic equilibrium for $F$, i.e. $F(0) = 0$ and $DF(0)$ has no eigenvalues on the unit circle. Thus we can find a radius $r > 1$ such that for all eigenvalues $\nu$ of $DF(0)$ we have $|\nu| \leq 1/r$ or $|\nu| > r$.
\item Let $E^s, E^u$ be the stable and unstable eigenspaces of $DF(0)$. We will assume $\dim E^s, \dim E^u \geq 1$.
\item There exists a symmetric homoclinic orbit (primary pulse) solution $Q(n)$ to \eqref{diffeq} which connects the equilibrium at 0 to itself.
\item $-Q(n)$ is also a solution to $\eqref{diffeq}$. 
\item The stable and unstable manifolds $W^s(0)$ and $W^u(0)$ intersect transversely. In particular, we have $Y^+ \oplus Y^- = \R^d$, where $Y^+ = T_{Q(0)} W^s(0)$ and $Y^- = T_{Q(0)} W^u(0)$.
\end{enumerate}
\end{hypothesis}

We wish to prove the existence of multi-pulse solutions to \eqref{diffeq}, which are solutions that resemble multiple, well-separated copies of the primary pulse $Q(n)$.\\

Choose an integer $m > 1$ (the number of pulses), and let $N_i$ ($i = 1, \dots, m-1$) be the desired distances (in lattice points) between the $m$ copies of $Q(n)$. Let 

\begin{align*}
N_i^+ &= \lfloor \frac{N_i}{2} \rfloor \\
N_i^- &= N_i - N_i^+
\end{align*}

and note that either $N_i^+ = N_i^-$ (if $N_i$ is even) or $N_i^+ = N_i^- - 1$ (if $N_i$ is odd). Let $N = \min\{ N_i^\pm \}$. In addition, let $N_0^- = -\infty$ and $N_m^+ = \infty$. We will look for a solution which is a piecewise perturbation of $n$ copies of $Q(n)$, i.e. a solution of the form

\begin{align*}
U_i^-(n) &= c_i Q(n) + W_i^-(n) && n \in [-N_{i-1}^-, 0] \\
U_i^+(n) &= c_i Q(n) + W_i^+(n) && n \in [0, N_i^+]
\end{align*}

where $c_i = \pm 1$. \\

Then we have the following theorem, which shows the existence of a unique multi-pulse solution to \eqref{diffeq}. It will be proved in a series of lemmas to follow the statement of the theorem.

\begin{theorem}\label{transversemulti}
Consider the difference equation
\begin{equation}\label{diffeq3}
U(n+1) = F(u(n))
\end{equation}
where $U \in R^d$ for $d > 1$. Assume Hypothesis \ref{initialhyp}. Let $Q(n)$ be the primary pulse solution to \eqref{diffeq3}. Choose $m > 1$ and positive integers $N_1, \dots, N_{m-1}$. Let 

\begin{align*}
N_i^+ &= \lfloor \frac{N_i}{2} \rfloor \\
N_i^- &= N_i - N_i^+
\end{align*}

and let $N = \min\{ N_i^\pm : i = 1, \dots, m-1 \}$. Then, for sufficiently large $N$, there exists a unique $m-$pulse solution $Q_m$ to \eqref{diffeq}. This solution resembles $m$ well-separated copies of the primary pulse solution and can be written piecewise as 

\begin{align}
U_i^-(n) &= c_i Q(n) + W_i^-(n) && n \in [-N_{i-1}^-, 0] \\
U_i^+(n) &= c_i Q(n) + W_i^+(n) && n \in [0, N_i^+]
\end{align}

where $N_0 = N_m = \infty$. For the remainder terms $W_i^\pm(n)$ we have the following estimates

\begin{align}
||W_i^\pm|| &\leq C r^{-N} \\
W_i^+(N_i^+) &= c_{i+1} Q(-N_i^-) + \mathcal{O}(r^{-2N}) \\
W_{i+1}^-(-N_i^-) &= c_i Q(N_i^+) + \mathcal{O}(r^{-2N})
\end{align}

\end{theorem}

To prove the theorem, we note that the $U_i^\pm$ must satisfy

\begin{align}
(U_i^\pm)(n+1) - F(U_i^\pm(n)) &= 0 \\
U_i^+(N_i^+) - U_{i+1}^-(-N_i^-) &= 0 \\
U_i^+(0) - U_i^-(0) &= 0
\end{align}

for $i = 1, \dots, m$. Expand $F$ in a Taylor series about $Q(n)$ to get

\begin{align*}
F(U_i^\pm(n)) &= F(c_i Q(n) + W_i^\pm(n)) \\
&= F(c_i Q(n)) + DF_{U}(c_i Q(n)) W_i^\pm + G(W_i^\pm(n)) \\
&= A(n; c_i) W_i^\pm + G(W_i^\pm(n))
\end{align*}

where

\begin{align*}
A(n; c_i) &= DF_{U}(c_i Q(n))\\
G(W_i^\pm(n)) &= \mathcal{O}(|W_i^\pm|^2)
\end{align*}

with $G(0) = DG(0) = 0$. Finally, let 

\begin{equation}
d_i = c_{i+1} Q(-N_i^-) - c_i Q(N_i^+)
\end{equation}

where $d_i \mathcal{O}(r^{-N_i/2})$. Substituting this in, we get the system

\begin{align}
(W_i^\pm)(n+1) &= A(n; c_i) W_i^\pm + G(W_i^\pm(n)) \\
W_i^+(N_i^+) - W_{i+1}^-(-N_i^-) &= d_i \\
W_i^+(N_i^+) - W_{i+1}^-(-N_i^-) &= 0 \\
\end{align}

\subsubsection{Exponential Dichotomy}

The variational equation associated with $A(n; c_i)$ is

\begin{align}
V(n+1; c_i) &= A(n; c_i) V(n; c_in) \label{vareq} \\
\end{align}

Since we are assuming that the stable and unstable manifolds intersect transversely, this has no bounded solution, and we do not have to consider the adjoint variational equation. Note that our system is a piecewise, nonlinear perturbation of the variational equation.

Let $\Phi(m, n; c_i)$ be the discrete evolution operators for the variational equation. Then (following something like Beyn97) since $A(n; c_i)$ is exponentially asymptotic to the hyperbolic matrix $DF(0)$, we can decompose $\Phi(m, n; c_i)$ in an exponential dichotomy on $\Z^-$ and $\Z^+$. In other words, we can find projections $P_\pm^s(n; c_i)$ and $P_\pm^u(n; c_i)$ defined on $\Z^\pm$ such that

\begin{equation}\label{projcommute}
P_\pm^{s/u}(m; c_i) \Phi(m, n; c_i) =  \Phi(m, n; c_i) P_\pm^{s/u}(n; c_i)
\end{equation}

and

\begin{align*}
\dim \text{range }P_\pm^s(n; c_i) &= \dim E^s \\
\dim \text{range }P_\pm^u(n; c_i) &= \dim E^u
\end{align*}

Let $\Phi_\pm^{s/u}(m, n; c_i) = \Phi(m, n; c_i) P_\pm^{s/u}(n; c_i)$ for $m, n \geq 0$ and $m, n \leq 0$ (respectively). Then we have the estimates

\begin{align*}
|\Phi_+^s(m, n; c_i)| &\leq C r^{-(m - n)} && 0 \leq n \leq m \\
|\Phi_+^u(m, n; c_i)| &\leq C r^{n-m} && 0 \leq m \leq n \\
|\Phi_-^s(m, n; c_i)| &\leq C r^{-(m - n)} && n \leq m \leq 0 \\
|\Phi_-^u(m, n; c_i)| &\leq C r^{n-m} && m \leq n \leq 0\\
\end{align*}

where $r$ is independent of the $c_i$.\\

Finally, let $P_0^{s/u}$ be the eigenprojections for $E^{s/u}$. Then we have

\begin{align}\label{projbound}
| P_\pm^{s/u}(n) - P_0^{s/u} | \leq C r^{-|n|}
\end{align}

\subsubsection{Fixed Point Formulation and Inversion}

We write the first equation in our system in fixed-point form using the discrete variation of constants formula.

\begin{align*}
W_i^-(n) &= 
\Phi_s^-(n, -N_{i-1}^-; c_i) a_{i-1}^- + \Phi_u^-(n, 0; c_i) b_i^-  \\
&+ \sum_{j = -N_{i-1}^-}^{n-1} \Phi_s^-(n, j+1; c_i) G_i^-(W_i^-(j)) - \sum_{j = n}^{-1} \Phi_u^-(n, j+1; c_i) G_i^-(W_i^-(j)) \\
W_i^+(n) &= \Phi_u^+(n, N_i^+; c_i) a_i^+ + \Phi_s^+(n, 0) b_i^+ \\
&+ \sum_{j = 0}^{n-1} \Phi_s^+(n, j+1; c_i) G_i^+(W_i^+(j)) 
- \sum_{j = n}^{N_i^+-1} \Phi_u^+(n, j+1; c_i) G_i^+(W_i^+(j))
\end{align*}

where 

\begin{enumerate}
\item $a_i^- \in E^s$, $a_i^+ \in E^u$ and $a_0^- = a_m^+ = 0$
\item $b_i^+ \in Y^+ = T_{Q(0)} W^s(0)$ and $b_i^- \in Y^- = T_{Q(0)} W^u(0)$
\end{enumerate}

We will take

\begin{align*}
W_i^-(n) &\in l^\infty([-N_{i-1}^-, 0]) \\
W_i^+(n) &\in l^\infty([0, N_i^+])
\end{align*}

As in San98, we will solve this in stages.\\

First, we solve for $W_i^\pm$.

% solve for W

\begin{lemma}\label{inv1}

There exist unique bounded functions $W_i^\pm(n)$ such that the first equation in our system is satisfied. They depend smoothly on the initial conditions $a_i^\pm$ and $b_i\pm$, and we have the estimates.

\begin{align*}
||W_i^-|| &\leq C (|a_{i-1}^-| + |b_i^-|) \\
||W_i^+|| &\leq C (|a_i^+| + |b_i^+| )
\end{align*}

\begin{proof}
First, we show that the RHS of the fixed point equations defines a smooth map from $l^\infty$ (on the appropriate interval) to itself. We will verify the ``minus'' pieces; the ``plus'' pieces will be similar.

For the terms not involving sums, we have

\begin{align*}
|\Phi_s^-(n, &-N_{i-1}^-; c_i) a_{i-1}^-| + |\Phi_u^-(n, 0; c_i) b_i^-| \\
&\leq C ( r^{-(n + N_{i-1}^-)} |a_{i-1}^-| +  r^{-n}|b_i^-|) \\
&\leq C ( |a_{i-1}^-| + |b_i^-|) 
\end{align*}

which is independent of $n$. For the terms involving sums, we have

\begin{align*}
&\left| \sum_{j = -N_{i-1}^-}^{n-1} \Phi_s^-(n, j+1; c_i) G_i^-(W_i^-(j))\right| + \left|\sum_{j = n}^{-1} \Phi_u^-(n, j+1; c_i) G_i^-(W_i^-(j))\right| \\
&\leq C ||W_i^-||_{l^\infty([-N_{i-1}, 0])} \left( \sum_{j = -N_{i-1}^-}^{n-1} r^{-(n - j - 1)} + \sum_{j = n}^{-1} r^{-(n - j - 1)} \right) \\
&= C ||W_i^-||_{l^\infty([-N_{i-1}, 0])}^2 \sum_{j = -N_{i-1}^-}^{-1} r^{-(n - j - 1)} \\
&= C ||W_i^-||_{l^\infty([-N_{i-1}, 0])}^2 r^{-n} \sum_{j = 0}^{N_{i-1} - 1} r^{-j} \\
&\leq C ||W_i^-||_{l^\infty([-N_{i-1}, 0])}^2 
\end{align*}

which is finite and also independent of $n$. Define the map $H_i^-: l^\infty([-N_{i-1}, 0]) \times E^s \times Y^- \rightarrow l^\infty([-N_{i-1}, 0])$ by

\begin{align*}
H_i^-(W_i^-(n), &a_{i-1}^-, b_i^-) = W_i^-(n) - \Phi_s^-(n, -N_{i-1}^-; c_i) a_{i-1}^- - \Phi_u^-(n, 0; c_i) b_i^-  \\
&- \sum_{j = -N_{i-1}^-}^{n-1} \Phi_s^-(n, j+1; c_i) G_i^-(W_i^-(j)) + \sum_{j = n}^{-1} \Phi_u^-(n, j+1; c_i) G_i^-(W_i^-(j)) 
\end{align*}

Since 0 is an equilibrium, $H(0, 0, 0) = 0$. We can check that the Frechet derivative of $H_i^-$ with respect to $W_i^-$ at $(W_i^-(n), a_{i-1}^-, b_i^-) = (0, 0, 0)$ is a Banach space isomorphism on $l^\infty([-N_{i-1}, 0])$. Using the implicit function theorem, we can solve for $W_i^-(x)$ in terms of $(a_{i-1}^-, b_i^-)$. This dependence is smooth, since the map $H_i^-$ is smooth. \\

The estimate on $W_i^-$ comes from the terms in the fixed point equations which do not involve sums, since the terms involving sums are quadratic in $W_i^-$. 
\end{proof}
\end{lemma}

Next, we use the center matching conditions at $N_i^\pm$ to solve for the initial conditions $a_i^\pm$.

% match in middles

\begin{lemma}\label{inv2}
For $i = 1, \dots m-1$ there is a unique pair of initial conditions $(a_i^+, a_i^-) \in E^u \times E^s$ such that the matching conditions $W_i^+(N_i^+) - W_{i+1}^-(-N_i^-) = d_i$ are satisfied. $(a_i^+, a_i^-)$ depends smoothly on $(b_i^+, b_{i+1}^-, d_i)$, and we have the following expressions for $a_i^-$ and $a_i^+$. 

\begin{align}
a_i^+ &= P_0^u d_i + \tilde{a}_i^+ \\
a_i^- &= -P_0^s d_i + \tilde{a}_i^-
\end{align}

where 

\[
\tilde{a}_i^\pm = \mathcal{O}(r^{-N}(|b_i^+|+|b_{i+1}^-|) + |b_i^+|^2+|b_{i+1}^-|^2) 
\]

In terms of $Q(\pm N_i^\pm)$, we can write these as

\begin{align*}
a_i^- &= c_i Q(N_i^+) + \tilde{a}_i^- + \mathcal{O}(r^{-2N}) \\
a_i^+ &= c_{i+1} Q(-N_i^-) + \tilde{a}_i^+ + \mathcal{O}(r^{-2N}) \\
\end{align*}

where $\tilde{a}_i^\pm$ is the same and has the same bound.

\begin{proof}
Evaluating the fixed point equations at $\pm N_i^\pm$ and subtracting, we need to solve

\begin{align*}
d_i &= W_i^+(N_i^+) - W_{i+1}^-(-N_i^-) \\
&= P_u^+(N_i^+; c_i) a_i^+ + \Phi_s^+(N_i^+, 0) b_i^+ + \sum_{j = 0}^{N_i^+-1} \Phi_s^+(N_i^+, j+1; c_i) G_i^+(W_i^+(j)) \\
&-P_s^-(-N_i^-; c_i) a_i^- - \Phi_u^-(-N_i^-, 0; c_i) b_{i+1}^-
- \sum_{j = -N_i^-}^{-1} \Phi_u^-(-N_i^-, j+1; c_i) G_i^-(W_i^-(j))
\end{align*}

Define $H_i: E^s \times E^u \times Y^+ \times Y^- \times \R^d \rightarrow \R^d$ by

\begin{align*}
H_i(a_i^+, &a_i^-, b_i^+, b_{i+1}^-, d_i) \\
&= a_i^+ - a_i^- - d_i + (P_u^+(N_i^+; c_i) - P_0^u) a_i^+ - (P_s^-(-N_i^-; c_i) - P_0^s) a_i^- + \Phi_s^+(N_i^+, 0; c_i) b_i^+ - \Phi_u^-(-N_i^-, 0; c_{i+1}) b_{i+1}^- \\
&+ \sum_{j = 0}^{N_i^+-1} \Phi_s^+(N_i^+, j+1; c_i) G_i^+(W_i^+(j; a_i^+, b_i^+)) 
+ \sum_{j = -N_i^-}^{-1} \Phi_u^-(-N_i^-, j+1; c_i) G_i^-(W_{i+1}^-(j; a_i^-, b_{i+1}^-))
\end{align*}

where we plugged in $W_{i+1}^-(n; a_i^-, b_{i+1}^-)$ and $W_i^+(n; a_i^+, b_i^+)$ from the previous lemma. Note that $(a_i^+, a_i^-) \in E^s \oplus E^u = \R^d$, so we are set up to use the IFT. We again have $H_i(0,0,0,0,0) = 0$. In order to use the IFT, we need to differentiate $H_i$ with respect to $a_i^\pm$. When we do this, the derivatives of the terms involving sums with respect to $a_i^\pm$ will be 0 since $G_i^\pm$ is quadratic order in $W_i^\pm$, thus quadratic order in $a_i^\pm$ by the previous lemma.\\

Thus the partial derivative with respect to $a_i^\pm$ at $(a_i^+, a_i^-, b_i^+, b_{i+1}^-, d_i) =(0, 0, 0, 0, 0)$ is 

\begin{align*}
\frac{\partial}{\partial a_i^-} H_i(0, 0, 0, 0) &= \pm 1 + \mathcal{O}(r^{-N_i^-}) \\
\frac{\partial}{\partial a_i^+} H_i(0, 0, 0, 0) &= \pm 1 + \mathcal{O}(r^{-N_i^+})
\end{align*}

For sufficiently large $N$, $D_{a_i^\pm} H(0, 0, 0, 0, 0)$ is invertible in a neighborhood of $(0, 0, 0, 0, 0)$, so we can use the IFT to solve for $a_i^\pm$ in terms of $b_i^+$, $b_{i+1}^-$, and $d_i$ for $(b_i^+, b_{i+1}^-, d_i)$ sufficiently small. \\

To get estimates on and expressions for $a_i^\pm$, we note that we have obtained a solution to $H_i(a_i^+, a_i^-, b_i^+, b_{i+1}^-, d_i) = 0$, which looks like 

\begin{align*}
0 &= a_i^+ - a_i^- - d_i + \mathcal{O}(r^{-N}(|a_i^+|+|a_i^-|+|b_i^+|+|b_{i+1}^-|)\\
&+ \mathcal{O}((|a_i^+|^2+|a_i^-|^2+|b_i^+|^2+|b_{i+1}^-|^2)
\end{align*}

Taking projections on $E^s$ and $E^u$, we have

\begin{align*}
a_i^+ &= P_0^u d_i + \mathcal{O}(r^{-N}(|b_i^+|+|b_{i+1}^-|) + |b_i^+|^2+|b_{i+1}^-|^2) \\
a_i^- &= -P_0^s d_i + \mathcal{O}(r^{-N}(|b_i^+|+|b_{i+1}^-|) + |b_i^+|^2+|b_{i+1}^-|^2)
\end{align*}

which we can write as

\begin{align*}
a_i^+ &= P_0^u d_i + \tilde{a}_i^+ \\
a_i^- &= -P_0^s d_i + \tilde{a}_i^-
\end{align*}

\[
\tilde{a}_i^\pm = \mathcal{O}(r^{-N}(|b_i^+|+|b_{i+1}^-|) + |b_i^+|^2+|b_{i+1}^-|^2) 
\]

To write these expressions in terms of $Q(\pm N_i^\pm)$, we note that

\begin{align*}
P_0^s Q(N_i^+) &= (P_0^s - P_s^+(N_i^+)) Q(N_i^+) + P_s^+(N_i^+) Q(N_i^+) \\
&= P_s^+(N_i^+) Q(N_i^+) + \mathcal{O}(r^{-2N}) \\
&= Q(N_i^+) + \mathcal{O}(r^{-2N}) 
\end{align*}

and 

\begin{align*}
P_0^s Q(-N_i^-) &= P_0^s  Q(-N_i^-) \\
&= P_0^s \Big( ( P_u^-(-N_i^-) - P_0^u) Q(-N_i^-) + P_0^u Q(-N_i^-) \Big) \\
&= P_0^s ( P_u^-(-N_i^-) - P_0^u) Q(-N_i^-) + P_0^s P_0^u Q(-N_i^-) \\
&= \mathcal{O}(r^{-2N})
\end{align*}

Thus we have

\begin{align*}
a_i^- &= c_i Q(N_i^+) + \tilde{a}_i^- + \mathcal{O}(r^{-2N})
\end{align*}

where $\tilde{a}_i^-$ is unchanged. Similarly, we have

\begin{align*}
a_i^+ &= c_{i+1} Q(-N_i^-) + \tilde{a}_i^+ + \mathcal{O}(r^{-2N}) \\
\end{align*}

\end{proof}
\end{lemma}

Finally, we look at the match conditions at 0. These will be straightforward because of the transverse intersection of the stable and unstable manifolds.

% solve for b_i

\begin{lemma}\label{inv3}

For $i = 1, \dots m$ there is a unique pair of initial conditions $(b_i^-, b_i^+) \in Y^- \times Y^+$ such that the matching conditions $W_i^+(0) - W_i^-(0) = 0$ are satisfied. $b = (b_1^+, b_1^-, \dots, b_m^+, b_m^-)$ depends smoothly on $d = (d_1, \dots, d_{m-1})$, and we have the uniform bound

\[
b = \mathcal{O}(r^{-2N})
\]

\begin{proof}
Evaluating the fixed point equations at 0 and subtracting, we have

\begin{align*}
W_i^+(0) &- W_i^-(0) = b_i^+ - b_i^- 
+ \Phi_u^+(0, N_i^+; c_i) a_i^+ - \Phi_s^-(0, -N_{i-1}^-; c_i) a_{i-1}^- \\
&- \sum_{j = 0}^{N_i^+-1} \Phi_u^+(0, j+1; c_i) G_i^+(W_i^+(j)) 
- \sum_{j = -N_{i-1}^-}^{-1} \Phi_s^-(0, j+1; c_i) G_i^-(W_i^-(j)) \\
\end{align*}

Next, plug in $W_i^\pm$ and $a_i^\pm$ from the previous two lemmas, noting the dependencies on the $b_i^\pm$. Let

\begin{align*}
Y &= \bigoplus_{i=1}^m (Y^+ \oplus Y^-) = \bigoplus_{i=1}^m \R^d \\
Z &= \bigoplus_{i=1}^{m-1} \R^d
\end{align*}

$b = (b_1^+, b_1^-, \dots, b_m^+, b_m^-)$, and $d = (d_1, \dots, d_{m-1})$. Define the function $H: Y \times Z \rightarrow Y$ component-wise by

\begin{align*}
H_i(b_i^+, b_i^-) &= 
 b_i^+ - b_i^- + \Phi_u^+(0, N_i^+; c_i) P_0^u d_i + \Phi_s^-(0, -N_{i-1}^-; c_i) P_0^s d_{i-1} \\
&+ \Phi_u^+(0, N_i^+; c_i) \tilde{a}_i^+(b_i^+, b_{i+1}^-) 
- \Phi_s^-(0, -N_{i-1}^-; c_i) \tilde{a}_{i-1}^-(b_{i-1}^+, b_i^-) \\
&- \sum_{j = 0}^{N_i^+-1} \Phi_u^+(0, j+1; c_i) G_i^+(W_i^+(j; b_i^+, b_{i+1}^-)) 
- \sum_{j = -N_{i-1}^-}^{-1} \Phi_s^-(0, j+1; c_i) G_i^-(W_i^-(j; b_{i-1}^+, b_i^-))
\end{align*}

where $d_0 = d_m = 0$. We are again set up to use the IFT. $H(0, 0) = 0$, and for the partial derivatives with respect to the $b_i^\pm$, we have

\begin{align*}
\frac{\partial}{\partial b_i^+} &= 1 + \mathcal{O}(r^{-N})  \\
\frac{\partial}{\partial b_i^-} &= -1 + \mathcal{O}(r^{-N}) \\
\frac{\partial}{\partial b_{i-1}^+},
\frac{\partial}{\partial b_{i+1}^-} &= \mathcal{O}(r^{-N}) \\
\end{align*}

and 

\[
\frac{\partial}{\partial b_j^\pm} = 0
\]

for all other indices. Thus, for sufficiently large $N$, the matrix $D_b H(0,0)$ is invertible. By the IFT, we can solve for $b$ in terms of $d$ for near $(b,d) = (0, 0)$, i.e. there is a function $b: Z \rightarrow Y$ with $b(0) = 0$ such that $H(b(d),d) = 0$ for $d$ sufficiently small. Since $d = \mathcal{O}(r^{-N}$, this will be the case as long as $N$ is sufficiently large. With this, we have solve the jump expressions and have constructed our $m-$pulse.\\

The only thing that remains is to find a bound for $b$. To do this, recall that we have solved $H(b(d),d) = 0$, which, componentwise, looks like

\begin{align*}
0 &= b_i^+ - b_i^- + \mathcal{O}(r^{-N_i^+} d_i + r^{-N_{i-1}^-} d_{i-1}) \\
&+ \mathcal{O}( r^{-N}(|b_i^+| + |b_i^-| + |b_{i+1}^-| + |b_{i-1}^+|)
+ (|b_i^+|^2 + |b_i^-|^2 + |b_{i+1}^-|^2 + |b_{i-1}^+|^2)
\end{align*}

which, since $b$ is small and $d = \mathcal{O}(r^{-N})$, gives us

\[
b = \mathcal{O}(r^{-2N})
\]

\end{proof}
\end{lemma}

We combine these results into the following theorem.

% theorem : m-pulse existence and bounds

\begin{theorem}
For sufficiently large $N$, there exists a unique $m-$pulse solution $Q_m$ to \eqref{diffeq} which resembles $m$ well-separated copies of the primary pulse solution and can be written piecewise as 

\begin{align}
U_i^-(n) &= c_i Q(n) + W_i^-(n) && n \in [-N_{i-1}^-, 0] \\
U_i^+(n) &= c_i Q(n) + W_i^+(n) && n \in [0, N_i^+]
\end{align}

where $N_0 = N_m = \infty$. For the remainder terms $W_i^\pm(n)$, we have the following estimates

\begin{align}
||W_i^\pm|| &\leq C r^{-N} \\
W_i^+(N_i^+) &= c_{i+1} Q(-N_i^-) + \mathcal{O}(r^{-2N}) \\
W_{i+1}^-(-N_i^-) &= c_i Q(N_i^+) + \mathcal{O}(r^{-2N})
\end{align}

\begin{proof}
The result comes from coming the previous three lemmas. For the uniform estimate, we plug in the estimates for $a_i^\pm$ and $b_i^\pm$ from Lemmas \ref{inv2} and \ref{inv3} to get $||W_i^\pm|| \leq C r^{-N}$. For the estimates at $\pm N_i^\pm$, we evaluate the fixed point equations at $\pm N_i^\pm$ and plug in the expressions for $a_i^\pm$ from Lemma \ref{inv2}. For $W_i^+(N_i^+)$, this is 

\begin{align*}
W_i^+(N_i^+) &= P_u^+(N_i^+; c_i) a_i^+ + \Phi_s^+(N_i^+, 0) b_i^+ + \sum_{j = 0}^{N_i^+-1} \Phi_s^+(N_i^+, j+1; c_i) G_i^+(W_i^+(j)) \\
&= P_u^+(N_i^+; c_i) c_{i+1} Q(-N_i^-) + \Phi_s^+(N_i^+, 0) b_i^+ + \sum_{j = 0}^{N_i^+-1} \Phi_s^+(N_i^+, j+1; c_i) G_i^+(W_i^+(j)) + \mathcal{O}(r^{-2N}) \\
&= P_0^u c_{i+1} Q(-N_i^-) + \mathcal{O}(r^{-2N}) \\
&= P_u^-(-N_i^-; c_{i+1}) c_{i+1} Q(-N_i^-) + \mathcal{O}(r^{-2N}) \\
&= c_{i+1} Q(-N_i^-) + \mathcal{O}(r^{-2N}) \
\end{align*}

Similarly,

\begin{align*}
W_{i+1}^-(-N_i^-) &= c_i Q(N_i^+) + \mathcal{O}(r^{-2N}) \
\end{align*}

\end{proof}
\end{theorem}

\subsection{Existence, non-transverse intersection}

\subsubsection{Setup and Main Theorem}

Again, we consider the lattice PDE

\begin{equation}\label{latticePDE3}
\dot{u}(n) = f(u(n))
\end{equation}

together with the equilibrium equation, which we write in system form as

\begin{equation}\label{diffeq3}
U(n+1) = F(u(n))
\end{equation}

where $U \in R^d$ for some $d > 1$ which depends on how many indices near $n$ are used by $f$. For this case, we take the following assumptions.

\begin{hypothesis}\label{initialhyp2}
We assume the following regarding \eqref{diffeq3}.
\begin{enumerate}[(i)]
\item $F$ is a diffeomorphism.
\item 0 is a hyperbolic equilibrium for $F$, i.e. $F(0) = 0$ and $DF(0)$ has no eigenvalues on the unit circle. Thus we can find a radius $r > 1$ such that for all eigenvalues $\nu$ of $DF(0)$ we have $|\nu| \leq 1/r$ or $|\nu| > r$.
\item Let $E^s, E^u$ be the stable and unstable eigenspaces of $DF(0)$. We will assume $\dim E^s, \dim E^u \geq 1$.
\item There exists a symmetric homoclinic orbit (primary pulse) solution $Q(n)$ to \eqref{diffeq3} which connects the equilibrium at 0 to itself.
\item Let $T(\theta)$ be a unitary group of symmetries on $\R^d$ indexed by a parameter $\theta \in \R$. In other words, $T(\theta)$ is a family of $d \times d$ matrices with $T(0) = I$, $T(\theta_1 + \theta_2) = T(\theta_1)T(\theta_2)$, and $T(\theta)^{-1} = T(-\theta)$. Since $T$ is unitary, we also have $T^*(\theta) = T(\theta)^{-1} = T(-\theta)$, and $T(\theta)$ is an isometry. We assume that for all $\theta \in \R$, $T(\theta)Q(n)$ is also a symmetric homoclinic orbit solution to $\eqref{diffeq3}$. 
\item The tangent spaces of the stable and unstable manifolds $W^s(0)$ and $W^u(0)$ have a one-dimensional intersection. At $n = 0$, let $\R S_1 = T_{Q(0)} W^s(0) \cap T_{Q(0)} W^u(0)$.
\end{enumerate}
\end{hypothesis}

We decompose the tangent spaces of $W^s(0)$ and $W^u(0)$ as

\begin{align*}
T_{Q(0)} W^s(0) &= \R S_1 \oplus Y^+ \\
T_{Q(0)} W^u(0) &= \R S_1 \oplus Y^-
\end{align*}

We wish to prove the existence of multi-pulse solutions to \eqref{diffeq3}, which are solutions that resemble multiple, well-separated copies of the primary pulse $Q(n)$ (and its symmetry-transformed versions).\\

Choose an integer $m > 1$ (the number of pulses), and let $N_i$ ($i = 1, \dots, m-1$) be the desired distances (in lattice points) between the $m$ copies of $Q(n)$. Let 

\begin{align*}
N_i^+ &= \lfloor \frac{N_i}{2} \rfloor \\
N_i^- &= N_i - N_i^+
\end{align*}

and note that either $N_i^+ = N_i^-$ (if $N_i$ is even) or $N_i^+ = N_i^- - 1$ (if $N_i$ is odd). Let $N = \min\{ N_i^\pm \}$. In addition, let $N_0^- = -\infty$ and $N_m^+ = \infty$. We will look for a solution which is a piecewise perturbation of $n$ copies of $T(\theta) Q(n)$, i.e. a solution of the form

\begin{align*}
U_i^-(n) &= T(\theta_i) Q(n) + W_i^-(n) && n \in [-N_{i-1}^-, 0] \\
U_i^+(n) &= T(\theta_i) Q(n) + W_i^+(n) && n \in [0, N_i^+]
\end{align*}

where $\theta_i \in \R$.\\

To have a solution, $U_i^\pm$ must satisfy the equations

\begin{align}
(U_i^\pm)(n+1) - F(U_i^\pm(n)) &= 0 \\
U_i^+(N_i^+) - U_{i+1}^-(-N_i^-) &= 0 \\
U_i^+(0) - U_i^-(0) &= 0
\end{align}

for $i = 1, \dots, m$. As in the tranverse intersection case, we expand $F$ in a Taylor series about $T(\theta) Q(n)$ and simplify to obtain the system for $W_i^\pm$

\begin{align}
(W_i^\pm)(n+1) &= A(n; \theta_i) W_i^\pm + G(W_i^\pm(n)) \\
W_i^+(N_i^+) - W_{i+1}^-(-N_i^-) &= d_i \\
W_i^+(N_i^+) - W_{i+1}^-(-N_i^-) &= 0 \\
\end{align}

where

\begin{align*}
A(n; \theta_i) &= DF_{U}( T(\theta_i) Q(n)) \\
G(W_i^\pm(n)) &= \mathcal{O}(|W_i^\pm|^2) \\
d_i &= T(\theta_{i+1}) Q(-N_i^-) - T(\theta_i) Q(N_i^+)
\end{align*}

with $G(0) = DG(0) = 0$.\\

\subsubsection{Variational Equation and Exponential Dichotomy}

The variational and adjoint variational equation associated with $A(n; \theta_i)$ is

\begin{align}
V(n+1; \theta_i) &= A(n; \theta_i) V(n; \theta_i) \label{vareq} \\
Z(n+1; \theta_i) &= [A(n; \theta_i)^*]^{-1} Z(n; \theta_i) \label{adjvareq} 
\end{align}

By Hypothesis \ref{initialhyp2}, it follows that the variational equation \eqref{vareq} for $\theta = 0$ has a unique bounded solution $S_1(n)$ and the adjoint variational equation \eqref{adjvareq} has a unique bounded solution $Z_1(n)$. We should have 

\[
A(n; \theta_i) = T(\theta_i)A(n; 0)T(\theta_i)^{-1} = T(\theta_i)A(n; 0)T(-\theta_i)
\]

(or at least it works for dNLS). Thus it follows that $T(\theta_i) S_1(n)$ is the unique bounded solution to the variational equation \eqref{vareq} for $\theta = \theta_i$ and $T(\theta_i) Z_1(n)$ is the unique bounded solution to the adjoint variational equation \eqref{adjvareq} for $\theta = \theta_i$.\\

Let $\Phi(m, n)$ be the discrete evolution operator for the variational equation with $\theta = 0$. Then since $A(n; 0)$ is exponentially asymptotic to the hyperbolic matrix $DF(0)$, we can decompose $\Phi(m, n)$ in an exponential dichotomy on $\Z^-$ and $\Z^+$. In other words, we can find projections $P_\pm^s(n)$ and $P_\pm^u(n)$ defined on $\Z^\pm$ such that

\begin{equation}\label{projcommute}
P_\pm^{s/u}(m) \Phi(m, n) =  \Phi(m, n) P_\pm^{s/u}(n)
\end{equation}

and

\begin{align*}
\dim \text{range }P_\pm^s(n) &= \dim E^s \\
\dim \text{range }P_\pm^u(n) &= \dim E^u
\end{align*}

Let $\Phi_\pm^{s/u}(m, n) = \Phi(m, n) P_\pm^{s/u}(n)$ for $m, n \geq 0$ and $m, n \leq 0$ (respectively). Then we have the estimates

\begin{align*}
|\Phi_+^s(m, n)| &\leq C r^{-(m - n)} && 0 \leq n \leq m \\
|\Phi_+^u(m, n)| &\leq C r^{n-m} && 0 \leq m \leq n \\
|\Phi_-^s(m, n)| &\leq C r^{-(m - n)} && n \leq m \leq 0 \\
|\Phi_-^u(m, n)| &\leq C r^{n-m} && m \leq n \leq 0\\
\end{align*}

where $r$ is defined in Hypothesis \ref{initialhyp2}. Finally, let $P_0^{s/u}$ be the eigenprojections for $E^{s/u}$. Then we have

\begin{align}\label{projbound}
| P_\pm^{s/u}(n) - P_0^{s/u} | \leq C r^{-|n|}
\end{align}

Using the expression for $A(n; \theta_i)$ above, the evolution operator for the variational equation with $\theta = \theta_i$ is given by 

\[
\Phi(m, n; \theta_i) = T(\theta_i) \Phi(m,n) T(-\theta_i)
\]

For the projections in the exponential dichotomy, we have

\[
P_\pm^{s/u}(m; \theta_i) = T(\theta_i) P_\pm^{s/u}(m) T(-\theta_i)
\]

\subsubsection{Fixed Point Formulation and Inversion}

We write the first equation in our system in fixed-point form using the discrete variation of constants formula.

\begin{align*}
W_i^-(n) &= 
\Phi_s^-(n, -N_{i-1}^-; \theta_i) a_{i-1}^- + \Phi_u^-(n, 0; \theta_i) b_i^-  \\
&+ \sum_{j = -N_{i-1}^-}^{n-1} \Phi_s^-(n, j+1; \theta_i) G_i^-(W_i^-(j)) - \sum_{j = n}^{-1} \Phi_u^-(n, j+1; \theta_i) G_i^-(W_i^-(j)) \\
W_i^+(n) &= \Phi_u^+(n, N_i^+; \theta_i) a_i^+ + \Phi_s^+(n, 0; \theta_i) b_i^+ \\
&+ \sum_{j = 0}^{n-1} \Phi_s^+(n, j+1; \theta_i) G_i^+(W_i^+(j)) 
- \sum_{j = n}^{N_i^+-1} \Phi_u^+(n, j+1; \theta_i) G_i^+(W_i^+(j))
\end{align*}

where 

\begin{enumerate}
\item $a_i^- \in E^s$, $a_i^+ \in E^u$ and $a_0^- = a_m^+ = 0$
\item $b_i^+ \in \R S_1 \oplus Y^+$ and $b_i^- \in \R S_1 \oplus Y^-$
\end{enumerate}

We will take

\begin{align*}
W_i^-(n) &\in l^\infty([-N_{i-1}^-, 0]) \\
W_i^+(n) &\in l^\infty([0, N_i^+])
\end{align*}

As above and in San98, we will solve this in stages. The first two stages are exactly the same as in the transverse intersection case.
\\

First, we solve for $W_i^\pm$.\\

% solve for W

\begin{lemma}\label{inv1}

There exist unique bounded functions $W_i^\pm(n)$ such that the first equation in our system is satisfied. They depend smoothly on the initial conditions $a_i^\pm$ and $b_i\pm$, and we have the estimates.

\begin{align*}
||W_i^-|| &\leq C (|a_{i-1}^-| + |b_i^-|) \\
||W_i^+|| &\leq C (|a_i^+| + |b_i^+| )
\end{align*}

\begin{proof}
First, we show that the RHS of the fixed point equations defines a smooth map from $l^\infty$ (on the appropriate interval) to itself. We will verify the ``minus'' pieces; the ``plus'' pieces will be similar.

For the terms not involving sums, we have

\begin{align*}
|\Phi_s^-(n, &-N_{i-1}^-; \theta_i) a_{i-1}^-| + |\Phi_u^-(n, 0; \theta_i) b_i^-| \\
&\leq C ( r^{-(n + N_{i-1}^-)} |a_{i-1}^-| +  r^{-n}|b_i^-|) \\
&\leq C ( |a_{i-1}^-| + |b_i^-|) 
\end{align*}

which is independent of $n$. For the terms involving sums, we have

\begin{align*}
&\left| \sum_{j = -N_{i-1}^-}^{n-1} \Phi_s^-(n, j+1; \theta_i) G_i^-(W_i^-(j))\right| + \left|\sum_{j = n}^{-1} \Phi_u^-(n, j+1; \theta_i) G_i^-(W_i^-(j))\right| \\
&\leq C ||W_i^-||_{l^\infty([-N_{i-1}, 0])} \left( \sum_{j = -N_{i-1}^-}^{n-1} r^{-(n - j - 1)} + \sum_{j = n}^{-1} r^{-(n - j - 1)} \right) \\
&= C ||W_i^-||_{l^\infty([-N_{i-1}, 0])}^2 \sum_{j = -N_{i-1}^-}^{-1} r^{-(n - j - 1)} \\
&= C ||W_i^-||_{l^\infty([-N_{i-1}, 0])}^2 r^{-n} \sum_{j = 0}^{N_{i-1} - 1} r^{-j} \\
&\leq C ||W_i^-||_{l^\infty([-N_{i-1}, 0])}^2 
\end{align*}

which is finite and also independent of $n$. Define the map $H_i^-: l^\infty([-N_{i-1}, 0]) \times E^s \times Y^- \rightarrow l^\infty([-N_{i-1}, 0])$ by

\begin{align*}
H_i^-(W_i^-(n), &a_{i-1}^-, b_i^-) = W_i^-(n) - \Phi_s^-(n, -N_{i-1}^-; \theta_i) a_{i-1}^- - \Phi_u^-(n, 0; \theta_i) b_i^-  \\
&- \sum_{j = -N_{i-1}^-}^{n-1} \Phi_s^-(n, j+1; \theta_i) G_i^-(W_i^-(j)) + \sum_{j = n}^{-1} \Phi_u^-(n, j+1; \theta_i) G_i^-(W_i^-(j)) 
\end{align*}

Since 0 is an equilibrium, $H(0, 0, 0) = 0$. We can check that the Frechet derivative of $H_i^-$ with respect to $W_i^-$ at $(W_i^-(n), a_{i-1}^-, b_i^-) = (0, 0, 0)$ is a Banach space isomorphism on $l^\infty([-N_{i-1}, 0])$. Using the implicit function theorem, we can solve for $W_i^-(x)$ in terms of $(a_{i-1}^-, b_i^-)$. This dependence is smooth, since the map $H_i^-$ is smooth. \\

The estimate on $W_i^-$ comes from the terms in the fixed point equations which do not involve sums, since the terms involving sums are quadratic in $W_i^-$. 
\end{proof}
\end{lemma}

Next, we use the center matching conditions at $N_i^\pm$ to solve for the initial conditions $a_i^\pm$.

% match in middles

\begin{lemma}\label{inv2}
For $i = 1, \dots m-1$ there is a unique pair of initial conditions $(a_i^+, a_i^-) \in E^u \times E^s$ such that the matching conditions $W_i^+(N_i^+) - W_{i+1}^-(-N_i^-) = d_i$ are satisfied. $(a_i^+, a_i^-)$ depends smoothly on $(b_i^+, b_{i+1}^-, d_i)$, and we have the following expressions for $a_i^-$ and $a_i^+$. 

\begin{align}
a_i^+ &= P_0^u d_i + \tilde{a}_i^+ \\
a_i^- &= -P_0^s d_i + \tilde{a}_i^-
\end{align}

where 

\[
\tilde{a}_i^\pm = \mathcal{O}(r^{-N}(|b_i^+|+|b_{i+1}^-|) + |b_i^+|^2+|b_{i+1}^-|^2) 
\]

In terms of $Q(\pm N_i^\pm)$, we can write these as

\begin{align*}
a_i^- &= T(\theta_i) Q(N_i^+) + \tilde{a}_i^- + \mathcal{O}(r^{-2N}) \\
a_i^+ &= T(\theta_{i+1}) Q(-N_i^-) + \tilde{a}_i^+ + \mathcal{O}(r^{-2N}) \\
\end{align*}

where $\tilde{a}_i^\pm$ is the same and has the same bound.

\begin{proof}
Evaluating the fixed point equations at $\pm N_i^\pm$ and subtracting, we need to solve

\begin{align*}
d_i &= W_i^+(N_i^+) - W_{i+1}^-(-N_i^-) \\
&= P_u^+(N_i^+; \theta_i) a_i^+ + \Phi_s^+(N_i^+, 0; \theta_i) b_i^+ + \sum_{j = 0}^{N_i^+-1} \Phi_s^+(N_i^+, j+1; \theta_i) G_i^+(W_i^+(j)) \\
&-P_s^-(-N_i^-; \theta_{i+1}) a_i^- - \Phi_u^-(-N_i^-, 0; \theta_{i+1}) b_{i+1}^-
- \sum_{j = -N_i^-}^{-1} \Phi_u^-(-N_i^-, j+1; \theta_{i+1}) G_i^-(W_i^-(j))
\end{align*}

Define $H_i: E^s \times E^u \times Y^+ \times Y^- \times \R^d \rightarrow \R^d$ by

\begin{align*}
H_i(a_i^+, &a_i^-, b_i^+, b_{i+1}^-, d_i) \\
&= a_i^+ - a_i^- - d_i + (P_u^+(N_i^+; \theta_i) - P_0^u) a_i^+ - (P_s^-(-N_i^-; \theta_{i+1}) - P_0^s) a_i^- + \Phi_s^+(N_i^+, 0; \theta_i) b_i^+ - \Phi_u^-(-N_i^-, 0; \theta_{i+1}) b_{i+1}^- \\
&+ \sum_{j = 0}^{N_i^+-1} \Phi_s^+(N_i^+, j+1; \theta_i) G_i^+(W_i^+(j; a_i^+, b_i^+)) 
+ \sum_{j = -N_i^-}^{-1} \Phi_u^-(-N_i^-, j+1; \theta_{i+1}) G_i^-(W_{i+1}^-(j; a_i^-, b_{i+1}^-))
\end{align*}

where we plugged in $W_{i+1}^-(n; a_i^-, b_{i+1}^-)$ and $W_i^+(n; a_i^+, b_i^+)$ from the previous lemma. Note that $(a_i^+, a_i^-) \in E^s \oplus E^u = \R^d$, so we are set up to use the IFT. We again have $H_i(0,0,0,0,0) = 0$. In order to use the IFT, we need to differentiate $H_i$ with respect to $a_i^\pm$. When we do this, the derivatives of the terms involving sums with respect to $a_i^\pm$ will be 0 since $G_i^\pm$ is quadratic order in $W_i^\pm$, thus quadratic order in $a_i^\pm$ by the previous lemma.\\

Thus the partial derivative with respect to $a_i^\pm$ at $(a_i^+, a_i^-, b_i^+, b_{i+1}^-, d_i) =(0, 0, 0, 0, 0)$ is 

\begin{align*}
\frac{\partial}{\partial a_i^-} H_i(0, 0, 0, 0) &= \pm 1 + \mathcal{O}(r^{-N_i^-}) \\
\frac{\partial}{\partial a_i^+} H_i(0, 0, 0, 0) &= \pm 1 + \mathcal{O}(r^{-N_i^+})
\end{align*}

For sufficiently large $N$, $D_{a_i^\pm} H(0, 0, 0, 0, 0)$ is invertible in a neighborhood of $(0, 0, 0, 0, 0)$, so we can use the IFT to solve for $a_i^\pm$ in terms of $b_i^+$, $b_{i+1}^-$, and $d_i$ for $(b_i^+, b_{i+1}^-, d_i)$ sufficiently small. \\

To get estimates on and expressions for $a_i^\pm$, we note that we have obtained a solution to $H_i(a_i^+, a_i^-, b_i^+, b_{i+1}^-, d_i) = 0$, which looks like 

\begin{align*}
0 &= a_i^+ - a_i^- - d_i + \mathcal{O}(r^{-N}(|a_i^+|+|a_i^-|+|b_i^+|+|b_{i+1}^-|)\\
&+ \mathcal{O}((|a_i^+|^2+|a_i^-|^2+|b_i^+|^2+|b_{i+1}^-|^2)
\end{align*}

Taking projections on $E^s$ and $E^u$, we have

\begin{align*}
a_i^+ &= P_0^u d_i + \mathcal{O}(r^{-N}(|b_i^+|+|b_{i+1}^-|) + |b_i^+|^2+|b_{i+1}^-|^2) \\
a_i^- &= -P_0^s d_i + \mathcal{O}(r^{-N}(|b_i^+|+|b_{i+1}^-|) + |b_i^+|^2+|b_{i+1}^-|^2)
\end{align*}

which we can write as

\begin{align*}
a_i^+ &= P_0^u d_i + \tilde{a}_i^+ \\
a_i^- &= -P_0^s d_i + \tilde{a}_i^-
\end{align*}

\[
\tilde{a}_i^\pm = \mathcal{O}(r^{-N}(|b_i^+|+|b_{i+1}^-|) + |b_i^+|^2+|b_{i+1}^-|^2) 
\]

To write these expressions in terms of $Q(\pm N_i^\pm)$, we note that

\begin{align*}
P_0^s T(\theta_i) Q(N_i^+) &= (P_0^s - P_s^+(N_i^+; \theta_i)) T(\theta_i) Q(N_i^+) + P_s^+(N_i^+; \theta_i) T(\theta_i) Q(N_i^+) \\
&= P_s^+(N_i^+; \theta_i) T(\theta_i) Q(N_i^+) + \mathcal{O}(r^{-2N}) \\
&= T(\theta_i) Q(N_i^+) + \mathcal{O}(r^{-2N}) 
\end{align*}

and 

\begin{align*}
P_0^s T(\theta_{i+1}) Q(-N_i^-) 
&= P_0^s \Big( ( P_u^-(-N_i^-; \theta_{i+1}) - P_0^u) T(\theta_{i+1}) Q(-N_i^-) + P_0^u T(\theta_{i+1}) Q(-N_i^-) \Big) \\
&= P_0^s ( P_u^-(-N_i^-; \theta_{i+1}) - P_0^u) T(\theta_{i+1}) Q(-N_i^-) + P_0^s P_0^u T(\theta_{i+1}) Q(-N_i^-) \\
&= \mathcal{O}(r^{-2N})
\end{align*}

Thus we have

\begin{align*}
a_i^- &= T(\theta_i) Q(N_i^+) + \tilde{a}_i^- + \mathcal{O}(r^{-2N})
\end{align*}

where $\tilde{a}_i^-$ is unchanged. Similarly, we have

\begin{align*}
a_i^+ &= T(\theta_{i+1}) Q(-N_i^-) + \tilde{a}_i^+ + \mathcal{O}(r^{-2N}) \\
\end{align*}

\end{proof}
\end{lemma}

% jump in center

Evaluating the fixed point equations at 0 and subtracting, we have

\begin{align*}
W_i^+(0) &- W_i^-(0) = P_s^+(0; \theta_i) b_i^+ - P_u^-(0; \theta_i) b_i^- 
+ \Phi_u^+(0, N_i^+; \theta_i) a_i^+ - \Phi_s^-(0, -N_{i-1}^-; \theta_i) a_{i-1}^- \\
&- \sum_{j = 0}^{N_i^+-1} \Phi_u^+(0, j+1; \theta_i) G_i^+(W_i^+(j)) 
- \sum_{j = -N_{i-1}^-}^{-1} \Phi_s^-(0, j+1; \theta_i) G_i^-(W_i^-(j)) \\
\end{align*}

From this, we should be able to come up with the adjoint jump conditions, which is the last thing we need to solve. For the adjoint, we take the inner product with $T(\theta_i Z_1(0))$. Recalling that $a_0^- = a_m^+  = 0$, this gives us the $m$ jump conditions

\begin{align*}
\langle Z_1(N_1^+), T(\theta_{2}-\theta_1) Q(-N_1^-)\rangle
+ h.o.t. &= 0 \\
\langle Z_1(N_i^+), T(\theta_{i+1}-\theta_i) Q(-N_i^-)\rangle
+ \langle Z_1(-N_{i-1}^-), T(\theta_{i-1}-\theta_i) Q(N_{i-1}^+)\rangle
+ h.o.t. &= 0 && i = 2, \dots, m-1 \\
\langle Z_1(-N_{m-1}^-), T(\theta_{m-1}-\theta_m) Q(N_{m-1}^+)\rangle
+ h.o.t. &= 0
\end{align*}

For $m = 2$, this reduces to

\begin{align*}
\langle Z_1(N_1^+), T(\theta_{2}-\theta_1) Q(-N_1^-)\rangle
+ h.o.t. &= 0 \\
\langle Z_1(-N_1^-), T(\theta_{1}-\theta_2) Q(-N_1^+)\rangle
+ h.o.t. &= 0
\end{align*}

In any case, we can make a Hamiltonian hypothesis to eliminate the last equation. 




\end{document}