\documentclass[12pt]{article}
\usepackage[pdfborder={0 0 0.5 [3 2]}]{hyperref}%
\usepackage[left=1in,right=1in,top=1in,bottom=1in]{geometry}%
\usepackage[shortalphabetic]{amsrefs}%
\usepackage{amsmath}
\usepackage{enumerate}
% \usepackage{enumitem}
\usepackage{amssymb}                
\usepackage{amsmath}                
\usepackage{amsfonts}
\usepackage{amsthm}
\usepackage{bbm}
\usepackage[table,xcdraw]{xcolor}
\usepackage{tikz}
\usepackage{float}
\usepackage{booktabs}
\usepackage{svg}
\usepackage{mathtools}
\usepackage{cool}
\usepackage{url}
\usepackage{graphicx,epsfig}
\usepackage{makecell}
\usepackage{array}

\def\noi{\noindent}
\def\T{{\mathbb T}}
\def\R{{\mathbb R}}
\def\N{{\mathbb N}}
\def\C{{\mathbb C}}
\def\Z{{\mathbb Z}}
\def\P{{\mathbb P}}
\def\E{{\mathbb E}}
\def\Q{\mathbb{Q}}
\def\ind{{\mathbb I}}

\DeclareMathOperator{\spn}{span}
\DeclareMathOperator{\ran}{ran}

\newtheorem{lemma}{Lemma}
\newtheorem{theorem}{Theorem}
\newtheorem{corollary}{Corollary}
\newtheorem{definition}{Definition}
\newtheorem{assumption}{Assumption}
\newtheorem{hypothesis}{Hypothesis}

\newtheorem{notation}{Notation}

\graphicspath{ {discrete/} }

\begin{document}

\section{Multipulses in Discrete Systems}

\subsection{Existence}

Consider the lattice PDE

\begin{equation}\label{latticePDE}
\dot{u}_n = f(u_n; \mu)
\end{equation}

where $f$ involves a finite number (greater than 1) of indices near $n$ and $\mu \in \R^P$ is some parameter. An equilibrium solution satisfies $f(u_n; \mu) = 0$. We can rewrite the stationary equation in system form as

\begin{equation}\label{diffeq}
U(n+1) = F(u(n))
\end{equation}

Take the following assumptions.

\begin{hypothesis}\label{initialhyp}
We assume the following regarding \eqref{diffeq}.
\begin{enumerate}[(0)]
\item 0 is a hyperbolic equilibrium for $F$, i.e. $F(0) = 0$ and $DF(0)$ has no eigenvalues on the unit circle. Thus we can find a radius $r > 1$ such that for all eigenvalues $\nu$ of $DF(0)$ we have $|\nu| \leq 1/r$ or $|\nu| > r$.
\item Let $E^s, E^u$ be the stable and unstable eigenspaces of $DF(0)$. We will assume $\dim E^s, \dim E^u \geq 1$.
\item The system is reversible. (We can consider the Hamiltonian case later.)
\item For a parameter value $\mu_0$, there exists a symmetric homoclinic orbit $Q(n; \mu_0)$ which connects the equilibrium at 0 to itself.
\item We may need some sort of nondegeneracy condition.
\end{enumerate}
\end{hypothesis}

We wish to prove the existence of multi-pulse solutions.\\

Let $N_i$ ($i = 1, \dots, m-1$) be the distances (in lattice points) between the $m$ copies of $Q(n)$, and let

\begin{align*}
N_i^+ &= \lfloor \frac{N_i}{2} \rfloor \\
N_i^- &= N_i - N_i^+
\end{align*}

In addition, let $N_0^- = -\infty$ and $N_m^+ = \infty$. We will look for a solution which is a piecewise perturbation of $n$ copies of $Q(n)$, with flips allowed.

\begin{align*}
U_i^-(n) &= c_i Q^-(n; \mu) + W_i^-(n) && n \in [-N_{i-1}^-, 0] \\
U_i^+(n) &= c_i Q^+(n; \mu) + W_i^+(n) && n \in [0, N_i^+]
\end{align*}

where $c_i = \pm 1$ and $Q^-(n; \mu)$ and $Q^+(n;\mu)$ are the $\mu-$dependent perturbations of $Q(n; \mu_0)$ which lie in the $\mu-$dependent unstable/stable manifolds and which do not necessary intersect except at $\mu = \mu_0$. They are, however, still solutions of the equilibrium ODE.\\

We do not allow for additional paramaterizations along these manifolds, since we want this to work in $\R^2$ where they are both one-dimensional. \\

The problem we want to solve is

\begin{align}
(U_i^\pm)(n+1) - F(U_i^\pm(n)) &= 0 \\
U_i^+(N_i) - U_{i+1}^-(-N_i) &= 0 \\
U_i^+(0) - U_i^-(0) &= 0
\end{align}

for $i = 1, \dots, n$. Next, we expand $F$ in a Taylor series about $Q(n; \mu)$ to get

\begin{align*}
F(U_i^\pm(n)) &= F(c_i Q^\pm(n; \mu) + W_i^\pm(n)) \\
&= F(c_i Q^\pm(n; \mu)) + DF_{U}(c_i Q^\pm(n; \mu)) W_i^\pm + G(W_i^\pm(n)) \\
&= A^\pm(n; \mu, c_i) W_i^\pm + G(W_i^\pm(n))
\end{align*}

where

\begin{align*}
A^\pm(n; \mu, c_i) &= DF_{U}(c_i Q^\pm(n; \mu))\\
G(W_i^\pm(n)) &= \mathcal{O}(|W_i^\pm|^2)
\end{align*}

and $G(0) = DG(0) = 0$.\\

Substituting this in, we get the system

\begin{align}
(U_i^\pm)(n+1) - A^\pm(n; \mu, c_i) W_i^\pm - G(W_i^\pm(n)) &= 0 \\
U_i^+(N_i) - U_{i+1}^-(-N_i) &= 0 \\
U_i^+(0) - U_i^-(0) &= 0
\end{align}

Define the variational and adjoint variational equations associated with $A^\pm(n; \mu, c_i)$ by

\begin{align}
V^\pm(n+1; \mu, c_i) &= A^\pm(n; \mu, c_i) V^\pm(n; \mu, c_i) \label{vareq} \\
Z^\pm(n+1; \mu, c_i) &= [A^\pm(n; \mu, c_i)^*]^{-1} Z^\pm(n; \mu, c_i) \label{adjvareq}
\end{align}

Now take the specific variational and adjoint variational equations associated with $Q(n)$ (and $\mu_0$).

\begin{align}
V(n+1) &= A^\pm(n) V(n) \label{vareqQ} \\
Z(n+1) &= [A^\pm(n)^*]^{-1} Z(n) \label{adjvareqQ}
\end{align}

where $A(n) = DF_{U}(Q(n))$. From whatever nondegeneracy condition we decide to use, it will (hopefully) follow that the variational equation has a unique bounded solution $S(n)$ and the adjoint variational equation has a unique bounded solution $Z(n)$. By the properties of such things, we can decompose the tangent space at $Q(0)$ as

\[
\R^m = \R Z(0) \oplus \R S(0) \oplus Y^+ \oplus Y^-
\]

where $Y^\pm$ are the other components of the unstable/stable manifolds at $Q(0)$ (either or both could not exist, as would happen in $\R^2$).\\

Let $\Phi_\pm(m, n; \mu, c_i)$ be the evolution operators for the variational equation. Then for $\mu$ close to $\mu_0$, we have an exponential dichotomy (independent of $\mu$). Briefly, can decompose the evolution on $\N^-$ and $\N^+$ into unstable and stable components, and we have bounds

\begin{align*}
|\Phi_+^s(m, n; \mu, c_i)| \leq C r^{m - n} && 0 \leq n \leq m \\
|\Phi_+^u(m, n; \mu, c_i)| \leq C \left( \dfrac{1}{r} \right)^{n-m} && 0 \leq m \leq n \\
|\Phi_-^s(m, n; \mu, c_i)| \leq C r^{m - n} && n \leq m \leq 0 \\
|\Phi_-^u(m, n; \mu, c_i)| \leq C \left( \dfrac{1}{r} \right)^{n-m} && m \leq n \leq 0\\
\end{align*}

where $r$ is independent of $\mu$ and $c_i$. \\

Finally, we recall that since we have some freedom in choosing complementary subspaces at $n = 0$ for exponential dichotomies, we can require

\begin{align*}
W_i^+(0) &\in \R Z(0) \oplus Y^- \\
W_i^-(0) &\in \R Z(0) \oplus Y^+
\end{align*}

We write the equation for $(U_i^\pm)(n+1)$ in fixed-point form using the discrete variation of constants formula.

\begin{align*}
W_i^-(n) &= 
\Phi_s^-(n, -N_{i-1}^-; \mu, c_i) a_{i-1}^- \\
&+ \sum_{j = -N_{i-1}^-}^{n-1} \Phi_s^-(n, j+1; \mu, c_i) G_i^-(W_i^-(j)) - \sum_{j = n}^{-1} \Phi_u^-(n, j+1; \mu, c_i) G_i^-(W_i^-(j)) \\
W_i^+(n) &= \Phi_u^+(n, N_i^+; \mu, c_i) a_i^+ \\
&+ \sum_{j = 0}^{n-1} \Phi_s^+(n, j+1; \mu, c_i) G_i^+(W_i^+(j)) 
- \sum_{j = n}^{N_i^+-1} \Phi_u^+(n, j+1; \mu, c_i) G_i^+(W_i^+(j))
\end{align*}

where we do not need a separate initial condition at 0 because of our choice of where $W_i^\pm(0)$ lives. We also take $a_0^- = a_m^+ = 0$. The $a_i^\pm$ live in $E^{s/u}(0)$. We solve in stages.\\

First, we solve for $W_i^\pm$. For now, we just use $l^\infty$ as our function space.\\

Define the map $H$ piecewise by

\begin{align*}
H(W_i^-(n); \mu, a_{i-1}^-) &= 
\Phi_s^-(n, -N_{i-1}^-; \mu, c_i) a_{i-1}^- - W_i^-(n) \\
&+ \sum_{j = -N_{i-1}^-}^{n-1} \Phi_s^-(n, j+1; \mu, c_i) G_i^-(W_i^-(j)) - \sum_{j = n}^{-1} \Phi_u^-(n, j+1; \mu, c_i) G_i^-(W_i^-(j)) \\
H(W_i^-(n); \mu, a_i^+) &= \Phi_u^+(n, N_i^+; \mu, c_i) a_i^+ - W_i^+(n)  \\
&+ \sum_{j = 0}^{n-1} \Phi_s^+(n, j+1; \mu, c_i) G_i^+(W_i^+(j)) 
- \sum_{j = n}^{N_i^+-1} \Phi_u^+(n, j+1; \mu, c_i) G_i^+(W_i^+(j))
\end{align*}

It it straighforward to show that if $W_i^\pm$ is bounded, then $H$ is bounded as well. $H(0, \mu_0, 0) = 0$, so (after verifying the hypotheses) we can use the IFT to solve for $W_i^\pm$ in terms of the other stuff to get

\begin{align*}
||W_i^-||_\infty &\leq C |a_{i-1}^-| \\
||W_i^+||_\infty &\leq C |a_i^+|
\end{align*}

which does not depend on $\mu$.\\

The next step is to solve for the $a_i^\pm$ using the condition $U_i^+(N_i^+) - U_{i+1}^-(-N_i^-) = 0$, $i = 1, \dots, m-1$. This is

\begin{align*}
0 &= U_i^+(N_i^+) - U_{i+1}^-(-N_i^-) \\
&= c_i Q^+(N_i^+; \mu) + W_i^+(N_i^+) - c_{i+1} Q^-(-N_i^-; \mu) - W_{i+1}^-(-N_i^-) \\
W_i^+(N_i^+) - W_{i+1}^-(-N_i^+) &= c_{i+1} Q^-(-N_i^-; \mu) - c_i Q^+(N_i^+; \mu) 
\end{align*}

Evaluating the fixed point equations at the appropriate places, this becomes

\begin{align*}
W_i^+(N_i^+) &- W_i^-(-N_i^-) = P_u^+(\mu, c_i) a_i^+ - P_s^-(\mu, c_i) a_i^-\\
&+ \sum_{j = 0}^{N_i^+-1} \Phi_s^+(N_i^+, j+1; \mu, c_i) G_i^+(W_i^+(j)) 
+ \sum_{j = -N_i^-}^{-1} \Phi_u^-(-N_i^-, j+1; \mu, c_i) G_i^-(W_i^-(j)) \\
c_{i+1} Q^-(-N_i^-; \mu) &- c_i Q^+(N_i^+; \mu)  = P_u^+(\mu, c_i) a_i^+ - P_s^-(\mu, c_i) a_i^-\\
&+ \sum_{j = 0}^{N_i^+-1} \Phi_s^+(N_i^+, j+1; \mu, c_i) G_i^+(W_i^+(j)) 
+ \sum_{j = -N_i^-}^{-1} \Phi_u^-(-N_i^-, j+1; \mu, c_i) G_i^-(W_i^-(j)) \\
\end{align*}

Adding and subtracting appropriate things to get the $a_i^\pm$ by themselves, this becomes

\begin{align*}
c_{i+1} Q^-(-N_i^-; \mu) &- c_i Q^+(N_i^+; \mu) = (a_i^+ - a_i^-) + (P_u^+(\mu, c_i) - P_0^u(0)) a_i^+ - (P_s^-(\mu, c_i) - P_0^s(0)) a_i^-\\
&+ \sum_{j = 0}^{N_i^+-1} \Phi_s^+(N_i^+, j+1; \mu, c_i) G_i^+(W_i^+(j)) 
+ \sum_{j = -N_i^-}^{-1} \Phi_u^-(-N_i^-, j+1; \mu, c_i) G_i^-(W_i^-(j)) \\
\end{align*}

As long as the $N_i^\pm$ are large enough, we should easily be able to solve this for the $a_i^\pm$ to get

\begin{align*}
a_i^+ &= P^s_0(0) \left( Q^+(N_i^+; c_i, \mu) - Q^-(-N_i^-; c_{i+1}, \mu) \right) + \mathcal{O}( r^{-N_i} )\\
a_i^- &= P^u_0(0) \left( Q^+(N_i^+; c_i, \mu) - Q^-(-N_i^-; c_{i+1}, \mu) \right) + \mathcal{O}( r^{-N_i} )
\end{align*}

where we recall that $N_i = N_i^+ + N_i^-$.\\

Up to this point, we have solved uniquely for the $W_i^\pm(n)$ and $a_i^\pm$. The resulting solution will have $m$ jumps at $n = 0$. \\

WE SIMPLIFY THINGS NOW and assume we are in $\R^2$, so that the only parameter we have left to play with is $\mu$. In higher dimensions, we will need to do the $\beta_i^\pm$ thing where we also allow perturbations along the stable and unstable manifolds. In other words, we assume $Y^\pm$ are both nonexistent. Then we only have to look at the projection on $Z(0)$ for the jump conditions.

Evaluating the fixed point equations at $n = 0$, we have

\begin{align*}
W_i^-(0) &= \Phi_s^-(0, -N_{i-1}^-; \mu, c_i) a_{i-1}^- + \sum_{j = -N_{i-1}^-}^{-1} \Phi_s^-(0, j+1; \mu, c_i) G_i^-(W_i^-(j)) \\
W_i^+(0) &= \Phi_u^+(0, N_i^+; \mu, c_i) a_i^+ - \sum_{j = 0}^{N_i^+-1} \Phi_u^+(0, j+1; \mu, c_i) G_i^+(W_i^+(j))
\end{align*}

Projecting on $Z(0)$ and evaluating these terms individually, we get

\begin{align*}
\langle Z(0), &\Phi_s^-(0, -N_{i-1}^-; \mu, c_i) a_{i-1}^- \rangle \\
&= \langle Z(0), \Phi_s^-(0, -N_{i-1}^-; \mu, c_i) P^u_0(0) \left( Q^+(N_{i-1}^+; c_i, \mu) - Q^-(-N_{i-1}^-; c_{i+1}, \mu) \right) \rangle + \mathcal{O}( r^{-3 N_{i-1}/2} ) \\
&= \langle Z(0), \Phi_s^-(0, -N_{i-1}^-; 0, c_i) P^u_0(0) \left( Q^+(N_{i-1}^+; c_i, \mu) - Q^-(-N_{i-1}^-; c_{i+1}, \mu) \right) \rangle + \mathcal{O}( r^{-3 N_{i-1}/2} + |\mu - \mu_0| r^{-N_{i-1}}) \\
&= c_i \langle Z(-N_{i-1}^-), Q(N_{i-1}^+) \rangle + \mathcal{O}( r^{-3 N_{i-1}/2} + |\mu - \mu_0| r^{-N_{i-1}})
\end{align*}

Similarly, 

\begin{align*}
\langle Z(0), \Phi_u^+(0, N_i^+; \mu, c_i) a_i^+ \rangle
&= c_{i+1} \langle Z(N_i^+), Q(-N_i^-) \rangle + \mathcal{O}( r^{-3 N_i/2} + |\mu - \mu_0| r^{-N_i})
\end{align*}

Putting these together, we have for the jump expressions

\[
\xi_i = c_{i+1} \langle Z(N_i^+), Q(-N_i^-) \rangle - c_i \langle Z(-N_{i-1}^-), Q(N_{i-1}^+) \rangle  + \mathcal{O}( r^{-3 N_m/2} + |\mu - \mu_0| r^{-N_m})
\]

For now, suppose we have a 2-pulse. Then we only have to solve one of these. The question is can we do it with only $\mu$ to play with?




\end{document}