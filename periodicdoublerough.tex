\documentclass[12pt]{article}
\usepackage[pdfborder={0 0 0.5 [3 2]}]{hyperref}%
\usepackage[left=1in,right=1in,top=1in,bottom=1in]{geometry}%
\usepackage[shortalphabetic]{amsrefs}%
\usepackage{amsmath}
\usepackage{enumerate}
% \usepackage{enumitem}
\usepackage{amssymb}                
\usepackage{amsmath}                
\usepackage{amsfonts}
\usepackage{amsthm}
\usepackage{bbm}
\usepackage[table,xcdraw]{xcolor}
\usepackage{tikz}
\usepackage{float}
\usepackage{booktabs}
\usepackage{svg}
\usepackage{mathtools}
\usepackage{cool}
\usepackage{url}
\usepackage{graphicx,epsfig}
\usepackage{makecell}
\usepackage{array}

\def\noi{\noindent}
\def\T{{\mathbb T}}
\def\R{{\mathbb R}}
\def\N{{\mathbb N}}
\def\C{{\mathbb C}}
\def\Z{{\mathbb Z}}
\def\P{{\mathbb P}}
\def\E{{\mathbb E}}
\def\Q{\mathbb{Q}}
\def\ind{{\mathbb I}}

\DeclareMathOperator{\spn}{span}
\DeclareMathOperator{\ran}{range}

\graphicspath{ {periodic/} }

\newtheorem{lemma}{Lemma}
\newtheorem{theorem}{Theorem}
\newtheorem{corollary}{Corollary}
\newtheorem{definition}{Definition}
\newtheorem{assumption}{Assumption}
\newtheorem{hypothesis}{Hypothesis}

\newtheorem{notation}{Notation}

\begin{document}

\section{Rough idea for 2-pulse}

Standard periodic 2-pulse, $2X_0$ is ``pulse distance'', $2X_1$ is ``period distance''.

\begin{enumerate}

\item Since Hamiltonian, we only have one equation we need to solve (SanStrut (3.9)).

\begin{align}\label{jumpeq}
\langle \Psi(-X_0), Q(X_0) \rangle - \langle \Psi(-X_1), Q(X_1) \rangle + R_0 &= 0
\end{align}

where $R_0 = \mathcal{O}(e^{-\alpha X_m})$, $X_m = \min\{X_i\}$.

\item This is a single equation with two unknowns, so we expect to have a family of solutions paramaterized by one of the unknowns. We will let $X_1$ be the parameter, i.e. we get to choose $X_1$ (possibly with some restrictions that it be sufficiently large or somesuch). Having done that, the term $\langle \Psi(-X_1), Q(X_1) \rangle$ will be constant.

\item From Lemma 6.1 in San98, we have an expression for $\langle \Psi(-x), Q(x) \rangle$ for sufficiently large $x$.

\begin{equation}\label{alphabeta}
\langle \Psi(-x), Q(x) \rangle
= s_0 e^{-2 \alpha x} \sin(2 \beta x + \phi) + \mathcal{O}(e^{-(2 \alpha + \gamma) x})
\end{equation}

where $0 < \gamma \leq 1$.

\item Plug \eqref{alphabeta} into \eqref{jumpeq} to get

\begin{align}\label{jumpeqnew}
s_0 e^{-2 \alpha X_0} \sin(2 \beta X_0 + \phi) - s_0 e^{-2 \alpha X_1} \sin(2 \beta X_1 + \phi) + \mathcal{O}(e^{-(2 \alpha + \gamma) X_m}) &= 0
\end{align}

Remainder term $R_0$ is incorporated into $\mathcal{O}(e^{-(2 \alpha + \gamma) X_m})$ term, since $R_0$ is equal or higher order.

\item For $X \leq \min\{ X_i \}$ let

\begin{align}
a_i &= e^{-2 \alpha (X_i - X)} \\
r &= e^{-\alpha( 2 X + \phi / \beta ) }
\end{align}

Then rewrite only the first term on the LHS of \eqref{jumpeqnew} using this to get

\begin{align}\label{jumpeqnew}
s_0 e^{\alpha \phi / \beta } a_0 r \sin \left( - \frac{\beta}{\alpha} \log (a_0 r) \right) - s_0 e^{-2 \alpha X_1} \sin(2 \beta X_1 + \phi) + \mathcal{O}(r^{1 + \gamma / 2 \alpha}) &= 0
\end{align}

We don't do this for the second term on the LHS since we want it to have nothing to do with $r$. For the same reason, we do not divide this whole thing by $r$. We can make things easier by dividing by all the constants to get

\begin{align}\label{jumpeqnew}
a_0 r \sin \left( - \frac{\beta}{\alpha} \log (a_0 r) \right) - C e^{-2 \alpha X_1} \sin(2 \beta X_1 + \phi) + \mathcal{O}(r^{1 + \gamma / 2 \alpha}) &= 0
\end{align}

\item We want to use the IFT to solve for $a_0$ in terms of $r$, with $X_1$ as a parameter. Thus we want to solve

\begin{align*}
G(a_0, r; X_1) = 
a_0 r \sin \left( - \frac{\beta}{\alpha} \log (a_0 r) \right) - C e^{-2 \alpha X_1} \sin(2 \beta X_1 + \phi) + \mathcal{O}(r^{1 + \gamma / 2 \alpha}) &= 0
\end{align*}

\item If we take $r = r_m$, where $r_m =  e^{-(\pi \alpha / \beta) m}$ for some natural number $m$, then this equation becomes

\begin{align*}
G(a_0, r; X_1) = 
a_0 r_m \sin \left( - \frac{\beta}{\alpha} \log a_0 \right) - C e^{-2 \alpha X_1} \sin(2 \beta X_1 + \phi) + \mathcal{O}(r_m^{1 + \gamma / 2 \alpha}) &= 0
\end{align*}

all we have done in remove the $r$ from the $\log$ term.

\item Let 

\begin{align*}
\tilde{G}(a_0, r; X_1) = 
a_0 r \sin \left( - \frac{\beta}{\alpha} \log a_0 \right) - C e^{-2 \alpha X_1} \sin(2 \beta X_1 + \phi) + \mathcal{O}(r^{1 + \gamma / 2 \alpha}) &= 0
\end{align*}

The idea here should be that we can use the IFT to solve this for $a_0$ near $r = 0$. Then, as long as we take $m$ sufficiently large, 


\end{enumerate}

\end{document}