\documentclass[12pt]{article}
\usepackage[pdfborder={0 0 0.5 [3 2]}, plainpages=false]{hyperref}%
\usepackage[left=1in,right=1in,top=1in,bottom=1in]{geometry}%
\usepackage[shortalphabetic]{amsrefs}%
\usepackage{amsmath}
\usepackage{enumerate}
% \usepackage{enumitem}
\usepackage{amssymb}                
\usepackage{amsmath}                
\usepackage{amsfonts}
\usepackage{amsthm}
\usepackage{bbm}
\usepackage[table,xcdraw]{xcolor}
\usepackage{tikz}
\usepackage{float}
\usepackage{booktabs}
\usepackage{svg}
\usepackage{mathtools}
\usepackage{cool}
\usepackage{url}
\usepackage{graphicx,epsfig}
\usepackage{makecell}
\usepackage{array}

\def\noi{\noindent}
\def\T{{\mathbb T}}
\def\R{{\mathbb R}}
\def\N{{\mathbb N}}
\def\C{{\mathbb C}}
\def\Z{{\mathbb Z}}
\def\P{{\mathbb P}}
\def\E{{\mathbb E}}
\def\Q{\mathbb{Q}}
\def\ind{{\mathbb I}}

\DeclareMathOperator{\spn}{span}
\DeclareMathOperator{\ran}{range}

\graphicspath{ {suspension/} }

\newtheorem{lemma}{Lemma}
\newtheorem{theorem}{Theorem}
\newtheorem{corollary}{Corollary}
\newtheorem{definition}{Definition}
\newtheorem{proposition}{Proposition}
\newtheorem{assumption}{Assumption}
\newtheorem{hypothesis}{Hypothesis}

\newtheorem{notation}{Notation}

\begin{document}

\section{Chen-McKenna Suspension Bridge Equation}

\subsection{Background}

In \cite{McKenna1990}, McKenna and Walter propose the following equation to model traveling waves on an infinitely long suspended beam.

\begin{equation}\label{susp}
u_{tt} + u_{xxxx} + u^+ - 1 = 0
\end{equation}

They explicitly compute localized traveling wave solutions to \eqref{susp}. This equation, however, has a cusp at $u = 0$ which is difficult for numerical analysis. In \cite{Chen1997}, Chen and McKenna use a mountain pass technique to prove existence of localized traveling wave solutions to the more general equation

\begin{equation}\label{suspgen}
u_{tt} + u_{xxxx} + f(u) = 0
\end{equation}

where the nonlinearity $f(u)$ is ``similar'' to $u^+$. They also propose the following model

\begin{equation}\label{susp2}
u_{tt} + u_{xxxx} + e^{u - 1} - 1 = 0
\end{equation}

which is a ``smooth approximation'' to \eqref{susp}. Chen and McKenna were unable to prove existence of solutions to \eqref{susp2}, but they note that there is strong numerical evidence that similar solutions exist for \eqref{susp2} as for \eqref{susp}. In \cite{Smets2002} (Theorem 11), Smets and van den Berg prove existence of a localized traveling wave solution to \eqref{susp2} for almost all speeds $c \in (0, \sqrt{2})$. In \cite{Berg2018}, van den Berg et al use a computer-assisted proof technique (Theorem 1) to prove existence of a localized traveling wave solution to \eqref{susp2} for all speeds $c$ with $c^2 \in [0.5, 1.9]$.\\

We are interested in multi-pulse solutions to \eqref{susp2}, the existence of which is suggested in \cite{Chen1997} and \cite{Sandstede1997}. We make the change of variables $u - 1 \mapsto u$ so that pulse solutions will decay to a baseline of 0 instead of 1. This gives us

\begin{equation}\label{susp3}
u_{tt} + u_{xxxx} + e^{u} - 1 = 0
\end{equation}

Since we are interested in traveling wave solutions, we make the usual traveling wave ansatz. Letting $\xi = x - ct$, substituting this into \eqref{susp3}, and changing $\xi$ back to $x$, we have

\begin{equation}\label{susp3}
u_{tt} - 2 c u_{x t} + u_{xxxx} + c^2 u_{xx} + e^{u} - 1 = 0
\end{equation}

For an equilibrium solution (such as a homoclinic orbit), all time derivatives are zero, so any equilibrium solution must satisfy the ODE

\begin{equation}\label{eqODE}
u_{xxxx} + c^2 u_{xx} + e^{u} - 1 = 0
\end{equation}

which is (46) on p. 342 of \cite{Chen1997}, with $\tilde{f}(u) = e^u - 1$. The equilibrium ODE \eqref{eqODE} is Hamiltonian with energy

\begin{equation}\label{eqH}
H(u) = u_x u_{xxx} - \frac{1}{2}u_{xx}^2 + \frac{c^2}{2}u_x^2 + e^u - u
\end{equation}

Since $u = 0$ is a solution to \eqref{eqODE}, we linearize about this trivial solution to obtain the ODE

\begin{equation}\label{lineartrivial}
v_{xxxx} + c^2 v_{xx} + 1 = 0
\end{equation}

which has four eigenvalues

\begin{equation}
\nu = \pm \sqrt{\frac{-c^2 \pm \sqrt{c^4 - 4}}{2} }
\end{equation}

Note that $\sqrt{c^4 - 4}$ is always less that $c^2$ is magnitude. If $|c| < \sqrt{2}$, we have a complex conjugate quartet $\nu = \pm \alpha \pm \beta i$, so the equilibrium at 0 is hyperbolic. When $c^2 = 2$, these eigenvalues collide on the imaginary axis at $\pm i$, then for $c^2 > 2$ we have a quartet of purely imaginary eigenvalues. Thus a bifurcation occurs at $c^2 = 2$. We expect that multipulse solutions will only be possible for $c^2 < 2$.\\

For the reminder of this discussion, we will take $c \in [0.5, 1.9] \subset (0, \sqrt{2})$, so that a homoclinic orbit solution to \eqref{eqODE} is known to exist and the linearization about 0 is hyperbolic. Let 

\begin{equation}
\nu = \pm \alpha \pm \beta i
\end{equation}

be the eigenvalues of \eqref{lineartrivial}, where $\alpha, \beta > 0$.

\subsection{Eigenvalue Problem}

For linear stability analysis, we look at the PDE eigenvalue problem. To do this, assume we have found an equilibrium solution $u_*(x)$ of \eqref{eqODE}. We linearize around $u^*(x)$ by taking the standard linearization ansatz

\begin{equation}
u(x,t) = u_*(x) + \epsilon e^{\lambda t} v(x)
\end{equation}

Plugging this into \eqref{susp3} and keeping only terms of order $\epsilon$, we obtain the quadratic eigenvalue problem

\begin{equation}\label{evp}
[\lambda^2 - 2 c \partial_x \lambda + (\partial_x^4 + c^2 \partial_x^2 + e^{u_*})]v = 0
\end{equation}

This is in the general form of a quadratic eigenvalue problem 

\begin{equation}\label{quadeig}
P_2(\lambda; u_*)v =  [A_2 \lambda^2 + A_1 \lambda + A_0(u_*)]v = 0
\end{equation}

where

\begin{align}
A_0(u^*) &= \partial_x^4 + c^2 \partial_x^2 + e^{u_*} \\
A_1 &= -2 c \partial_x \\
A_2 &= I
\end{align}

Note that the operator $A_0(u_*)$ is the only of the operators which depends on the equilibrium solution we are linearizing about.\\

We can also write this as

\begin{equation}\label{evp2}
[(\lambda - c \partial_x)^2 + \partial_x^4 + e^{u_*}]v = 0
\end{equation}

\subsection{Multi-symplectic structure}

Before we do Lin's method on this, we will write our equation in a multi-symplectic structure, following Bridges97. The idea is that if we were to naively use Lin's method on the ``standard'' 4th order system, we would wind up with terms in both $\lambda$ and $\lambda^2$, which would annoying, if not impossible, to deal with.\\

Following the nonlinear Klein-Gordon example from Bridges97 (and extending it to two more dimensions), we let $Z = (u, v, w_1, w_2, w_3)^T$, where 

\begin{align*}
v &= u_t \\
w_1 &= u_x \\
w_2 &= u_{xx} \\
w_3 &= u_{xxx}
\end{align*}

Then we can write \eqref{susp3} as the multi-symplectic structure

\[
M Z_t + K Z_x = \nabla S(Z)
\]

where $M$ and $K$ are the constant, skew-symmetric (but noninvertible) matrices

\begin{align*}
M &= \begin{pmatrix}
0 & 1 & 0 & 0 & 0 \\
-1 & 0 & 0 & 0 & 0 \\
0 & 0 & 0 & 0 & 0 \\
0 & 0 & 0 & 0 & 0 \\
0 & 0 & 0 & 0 & 0 \\
\end{pmatrix}, 
K = \begin{pmatrix}
0 & 0 & 0 & 0 & 1 \\
0 & 0 & 0 & 0 & 0 \\
0 & 0 & 0 & 1 & 0 \\
0 & 0 & -1 & 0 & 0 \\
-1 & 0 & 0 & 0 & 0 \\
\end{pmatrix}
\end{align*}

and

\[
S(Z) = -V''(u) + \frac{1}{2}(w_3^2 - v^2 - w_1^2 - w_2^2)
\]

where $V'(u) = e^u - 1$ in our case. Just so we have it, 

\[
\nabla S(Z) = \begin{pmatrix}
-V'(u) \\ -v \\ w_3 \\ -w_2 \\ -w_1
\end{pmatrix}
\]

Next, we substitute our traveling wave ansatz, i.e. we let $\xi = x - ct$. Renaming the spatial variable back to $x$, we have the system

\[
M Z_t + (K - cM)Z_x = \nabla S(Z)
\]

Let

\[
J = K - cM = 
\begin{pmatrix}
0 & -c & 0 & 0 & 1 \\
c & 0 & 0 & 0 & 0 \\
0 & 0 & 0 & 1 & 0 \\
0 & 0 & -1 & 0 & 0 \\
-1 & 0 & 0 & 0 & 0 \\
\end{pmatrix}
\]

Then in the co-moving frame with speed $c$, our equation becomes

\[
M Z_t + J Z_x = \nabla S(Z)
\]

Finally, we linearize about an equlibrium solution $Z^*$ by taking the standard linearization ansatz $\tilde{Z} = Z^* + \epsilon e^{\lambda t} Z$ and keeping only terms up to order $\epsilon$. Thus we get the linearized system

\[
J Z_x = (A(u^*(x)) - \lambda M) Z
\]

where

\[
A(u^*(x)) = 
\begin{pmatrix}
-V''(u^*(x)) & 0 & 0 & 0 & 0 \\
0 & -1 & 0 & 0 & 0 \\
0 & 0 & 0 & 0 & 1 \\
0 & 0 & 0 & -1 & 0 \\
0 & 0 & -1 & 0 & 0 \\
\end{pmatrix}
\]

We note that $J$ is singular with a 1-dimensional kernel, so we cannot just move it to the other size.


% \bibliography{suspension.bib}


\end{document}