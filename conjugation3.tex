\documentclass[12pt]{article}
\usepackage[pdfborder={0 0 0.5 [3 2]}]{hyperref}%
\usepackage[left=1in,right=1in,top=1in,bottom=1in]{geometry}%
\usepackage[shortalphabetic]{amsrefs}%
\usepackage{amsmath}
\usepackage{enumerate}
% \usepackage{enumitem}
\usepackage{amssymb}                
\usepackage{amsmath}                
\usepackage{amsfonts}
\usepackage{amsthm}
\usepackage{bbm}
\usepackage[table,xcdraw]{xcolor}
\usepackage{tikz}
\usepackage{float}
\usepackage{booktabs}
\usepackage{svg}
\usepackage{mathtools}
\usepackage{cool}
\usepackage{url}
\usepackage{graphicx,epsfig}
\usepackage{makecell}
\usepackage{array}

\def\noi{\noindent}
\def\T{{\mathbb T}}
\def\R{{\mathbb R}}
\def\N{{\mathbb N}}
\def\C{{\mathbb C}}
\def\Z{{\mathbb Z}}
\def\P{{\mathbb P}}
\def\E{{\mathbb E}}
\def\Q{\mathbb{Q}}
\def\ind{{\mathbb I}}

\DeclareMathOperator{\spn}{span}
\DeclareMathOperator{\ran}{ran}

\graphicspath{ {periodic/} }

\newtheorem{lemma}{Lemma}
\newtheorem{theorem}{Theorem}
\newtheorem{corollary}{Corollary}
\newtheorem{definition}{Definition}
\newtheorem{assumption}{Assumption}
\newtheorem{proposition}{Proposition}
\newtheorem{hypothesis}{Hypothesis}

\newtheorem{notation}{Notation}

\begin{document}

Here we summarize what we have shown so far for the periodic multi-pulse eigenvalue problem. We assume we have the desired symmetries (as in KdV5) so that the matrix $A$ (below) takes a simpler form.

\section{Block Matrix Theorem}

Let $q_{np}(x)$ be a periodic $n-$pulse solution constructed with lengths $X_0, \dots, X_{n-1}$, where $X_{n-1}$ is the ``periodic length''. Then the jump conditions can be written as the block matrix equation 

\begin{equation}\label{blockeq}
\begin{pmatrix}
K(\lambda) & D_2 \\
C_3 K(\lambda) + K(\lambda) \tilde{C}_3 & A - \lambda^2 MI + D_3
\end{pmatrix}
\begin{pmatrix}c \\ d \end{pmatrix} 
= 0
\end{equation}

$M$ is the Melnikov integral

\begin{align*}
M &= \int_{-\infty}^\infty \langle \Psi(y), H(y) \rangle dy \\
\end{align*}

The matrix $K(\lambda)$ is

\begin{equation}
K(\lambda) = 
\begin{pmatrix}
e^{-\nu(\lambda)X_1} & & & & & -e^{\nu(\lambda)X_0} \\
-e^{\nu(\lambda)X_1} & e^{-\nu(\lambda)X_2} \\
& -e^{\nu(\lambda)X_2} & e^{-\nu(\lambda)X_3} \\
\vdots & & \vdots & &&  \vdots \\
& & & & -e^{\nu(\lambda)X_{n-1}} & e^{-\nu(\lambda)X_0} 
\end{pmatrix}
\end{equation}

where $\nu(\lambda)$ is the small eigenvalue of the asymptotic matrix $A(\lambda)$. \\

The matrix $A$ is 

\begin{align*}
A &= \begin{pmatrix}
-a_0 -a_1 & a_0 + a_1 \\
a_0 + a_1 & -a_0 - a_1
\end{pmatrix} && n = 2 \\
A &= \begin{pmatrix}
-a_{n-1} - a_0 & a_0 & & & \dots & a_{n-1}\\
a_0 & -a_0 - a_1 &  a_1 \\
& a_1 & -a_1 - a_2 &  a_2 \\
& & \vdots & & \vdots \\
a_{n-1} & & & & a_{n-2} & -a_{n-2} - a_{n-1} \\
\end{pmatrix} && n > 2
\end{align*}

where

\begin{align*}
a_i &= \langle \Psi(X_i), Q'(-X_i) \rangle \\
\end{align*}

The remainder terms have bounds

\begin{align*}
C_3, \tilde{C}_3 &= \text{diag}(\mathcal{O}(|\lambda| + e^{-\alpha X_m})) 
+ \mathcal{O}((|\lambda| + e^{-\tilde{\alpha} X_m})( |\lambda| + e^{-\alpha X_m})) \\
D_2 &= \mathcal{O}((|\lambda| + e^{-\tilde{\alpha} X_m})(|\lambda| + e^{-\alpha X_m})) \\
D_3 &= \mathcal{O}((|\lambda| + e^{-\tilde{\alpha} X_m})(|\lambda| + e^{-\alpha X_m})^2)
\end{align*}

where $X_m = \min \{X_0, \dots, X_{n-1}\}$

\section{Eigenvalues of A}

The interaction eigenvalues should satisfy $\lambda^2 \approx \mu / M$, where $\mu$ is an eigenvalue of $A$, and $M \neq 0$ is the Melnikov integral. Thus it useful to look at the eigenvalues of $A$. We note that $A$ is a real, symmetric matrix, so its eigenvalues will be real. We start with simplest cases.

\begin{enumerate}

\item For $n = 2$, it is easy to compute the eigenvalues of $A$.

\[
\mu = \{ 0, -2(a_0 + a_1) \}
\]

If $a_0 = a_1 = 1$, then

\[
\mu = \{ 0, -4 a \}
\]

\item For $n = 3$, Mathematica can do it.

\[
\mu = \left\{0,  -(a_0 + a_1 + a_2) \pm 
\sqrt{ a_0^2 + a_1^2 + a_2^2 - a_0 a_1 - a_1 a_2 - a_0 a_2 }\right\}
\]

If $a_0 = a_1 = a_2 = a$, this becomes

\[
\mu = \{0,  -3a \}
\]

where the eigenvalue $\mu = -3a$ has algebraic multiplicity 2. 

\item For $n > 4$, Mathematica can no longer do it (thanks, Galois!) 

\item Consider the case where all the distances $X_i$ are equal, i.e. $a_k = a$ for all $k$. In this case, $A$ is a circulant matrix. Using the Wikipedia article on circulant matrices and simplifying the result, we have eigenvalues

\begin{align*}
\mu_k &= 2 a\left( \cos \frac{2 \pi k}{n}  - 1 \right) & k = 1, \dots, n
\end{align*}

Things to notice,

\begin{enumerate}
	\item $\mu_n = 0$ always
	\item The remaining $\mu_k$ are either all positive (if $a < 0$) or all negative (if $a > 0$)
	\item If $n$ is odd, the collection $\mu_1, \dots, \mu_{n-1}$ will contain $(n-1)/2$ distinct eigenvalues, all of which have algebraic multiplicity 2.
	\item If $n$ is even, we will have $\mu_{n/2} = -4a$. The collection of $\mu_k$ which are not yet accounted for contains $(n-2)/2$ distinct eigenvalues, all of which have algebraic multiplicity 2.
\end{enumerate}

\end{enumerate}

The next case to consider is when the ``periodic length'' is large. If $X_{n-1}$ is large, then $a_{n-1}$ is very close to 0, so it is useful to look at the eigenvalues of $A$ when this is the case. With this assumption, $A$ is now a tridiagonal matrix. 

\begin{align*}
A &= \begin{pmatrix}
-a_0 & a_0 \\
a_0 & -a_0 
\end{pmatrix} && n = 2 \\
A &= \begin{pmatrix}
- a_0 & a_0 & & &  \\
a_0 & -a_0 - a_1 &  a_1 \\
& a_1 & -a_1 - a_2 &  a_2 \\
& & & \ddots \\
& & & a_{n-2} & -a_{n-2} \\
\end{pmatrix} && n > 2
\end{align*}

\begin{enumerate}

\item For $n = 2$, we have eigenvalues

\[
\mu = \{ 0, -2 a_0 \}
\]

\item For $n = 3$, we have eigenvalues

\[
\mu = \left\{ 0,  -(a_0 + a_1) \pm 
\sqrt{a_0^2 + a_1^2 - a_0 a_1} \right\}
\]

If $a_0 = a_1 = a$, this becomes

\[
\mu = \{0,  -a, -3a \}
\]

and this time these are distinct.

\item Consider the case where all the distances $X_i$ are equal, i.e. $a_k = a$ for all $k$. Unfortunately, $A$ is not a circulant matrix. But I have proved (and verified by Mathematica!) that the eigenvalues are 

\begin{align*}
\mu_k &= 2 a\left( \cos \frac{\pi k}{n}  - 1 \right) & k = 1, \dots, n
\end{align*}

(The only difference between this and the ``periodic'' version above, is that there is not a 2 in front of the $\pi$). We note the following.

\begin{enumerate}
	\item $\mu_n = 0$ always
	\item The remaining $\mu_k$ are either all positive (if $a < 0$) or all negative (if $a > 0$)
	\item All the remaining $\mu_k$ are distinct. 
\end{enumerate}

\item Now consider the general case. Although we cannot find the eigenvalues themselves, we do know their signs. Note that none of the $a_i$ are 0, and that since each row sums to 0, 0 is an eigenvalue with eigenvector $(1, 1, \dots, 1)^T$. Using Lemma 5.4 in San98 (and noting that that matrix there is $-A$), we have

\begin{enumerate}
	\item $A$ has $k_+$ negative real eigenvalues (counting multiplicity), where $k_+$ is the number of positive $a_i$.
	\item $A$ has $k_-$ positive real eigenvalues (counting multiplicity), where $k_-$ is the number of negative $a_i$.
\end{enumerate}

\end{enumerate}

\section{2-periodic pulses}

We consider 2-periodic pulses first, since they are the simplest. Recall that we have proved that for sufficiently large $X_0$ and any $X_1 \geq X_0$, these 2-periodic pulses exist. Let $X = X_0 + X_1$.

\subsection{2-periodic pulse with equal distances}

For the 2-periodic pulse with lengths $X_0 = X_1$,

\begin{itemize}
	\item For $X_0$ sufficiently large, there is a pair of interaction eigenvalues at $\lambda = \pm \lambda(X_0)$, where $\lambda(X_0) = \mathcal{O}(e^{-\alpha X_0})$ is either real or purely imaginary and is close to $\pm \sqrt{-2a/M}$, where
	\[
	a = 2 a_0 = 2 \langle \Psi(X_0), Q'(-X_0) \rangle
	\]
	\item For $n/X_0$ sufficiently small, where $n$ is a nonzero integer, there is a pair of purely imaginary ``essential spectrum'' eigenvalues located at $\lambda = \pm \lambda^c(X_0, n)$, where $\lambda^c(X_0, n)$ is close to $-c_0 \frac{n \pi i }{2 X_0}$.
	\item There is an eigenvalue at 0 (from translation invariance) with algebraic multiplicity 2, for which the eigenfunction and generalized eigenfunction are known. There is an additional eigenvalue at 0 which comes from the fact that $K(0)$ is singular.
	\item Within a ball of radius $\delta = C_0 e^{-\alpha X_0}$, with $C_0$ chosen so that  the ball is sufficiently large to enclose the interaction eigenvalues, there are no other eigenvalues.
\end{itemize}

\subsection{2-periodic pulse with unequal distances (and one restriction)}

For the 2-periodic solution with lengths $X_0 < X_1$, first assume that

\[
e^{-\alpha X_0}X = 
e^{-\alpha X_0}X_0\left( 1 + \frac{X_1}{X_0} \right) = \mathcal{O}(1)
\]

Recall that 

\begin{enumerate}
\item $K(\lambda)$ is singular at

\[
\mu(X, n) = -c_0 \frac{n \pi i}{X} + \mathcal{O}(n/X)^2
\]

\item $A - \lambda^2 M I$ is singular at $\lambda = 0$ and $\lambda = \pm \mu_0 = \pm \sqrt{-2a/M}$. 

\end{enumerate}

Let 

\[
\epsilon = \frac{e^{-\alpha X_0/4}}{X}
\]

and assume that $| \pm \mu_0 - \mu(X, n)| \geq \epsilon$ for all $n$. In other words, the nonzero points where $A - \lambda^2 M I$ is singular are located at least a distance $\epsilon$ from the points where $K(\lambda)$ is singular. Then we have the following.

\begin{itemize}
	\item For $X_0$ sufficiently large, there is a pair of interaction eigenvalues at $\lambda = \pm \lambda(X_0)$, where $\lambda(X_0) = \mathcal{O}(e^{-\alpha X_0})$ is either real or purely imaginary and is close to $\pm \sqrt{-2a/M}$, where 
	\[
	a = a_0 + a_1 = \langle \Psi(X_0), Q'(-X_0) \rangle + \langle \Psi(X_1), Q'(-X_1) \rangle
	\]
	For $X_1$ much greater than $X_0$, $a \approx a_0$.
	\item For $n/X_0$ sufficiently small, where $n$ is a nonzero integer, there is a pair of purely imaginary ``essential spectrum'' eigenvalues located at $\lambda = \pm \lambda^c(X, n)$, where $\lambda^c(X_0, n)$ is close to $\mu(X, n) = -c_0 \frac{n \pi i }{X}$.
	\item There is an eigenvalue at 0 (from translation invariance) with algebraic multiplicity 2, for which the eigenfunction and generalized eigenfunction are known. There is an additional eigenvalue at 0 which comes from the fact that $K(0)$ is singular.
	\item Within a ball of radius $\delta = C_0 e^{-\alpha X_0}$, with $C_0$ chosen so that  the ball is sufficiently large to enclose the interaction eigenvalues, there are no other eigenvalues.
\end{itemize}

\subsection{2-periodic pulse with unequal distances, no restrictions}

In general, $K(\lambda) = \mathcal{O}(e^{|\text{Re } \nu(\lambda)|X})$. If $\nu(\lambda)$ has a nonzero real part, this grows exponentially as $X = X_0 + X_1$ increases, and as of now we have no way to control this other than making an additional assumption that $X$ cannot get too large. The key thing we need to control is $K(\lambda) \tilde{C}_3 K(\lambda)^{-1}$. It is possible that if we could better characterize $\tilde{C}_3$, we could get a better bound that would allow us to get this to work for all $X$. Thus it makes sense to revisit the Lin's method estimates to see if we can do this.    

\section{Higher order multi-pulses}

The interaction eigenvalues should all be $\mathcal{O}(e^{-\alpha X_m})$, where $X_m = \min \{X_i\}$. The singular points of $K(\lambda)$ depend only on $X$. Thus this should work exactly the same as the 2-periodic cases, except we have to make more $\epsilon-$balls where needed to isolate the singular points of $K(\lambda)$ and $A - \lambda^2 M I$.\\

The additional complications are the following.

\begin{enumerate}

\item For now, we will need a similar assumption to what we did in the 2-periodic pulse case with unequal distances, i.e. 

\[
e^{-\alpha X_m}X = \mathcal{O}(1)
\]

to get this to work.

\item Suppose we take all the $X_i$ to be the same. As noted above, for $n \geq 3$ we will always have eigenvalues of $A$ with algebraic multiplicity 2, which allows for the possibility of interaction eigenvalue quartets. I don't see any way of showing this cannot happen, so we probably need to exclude this case. Interestingly, we also only have proven the existence result for $X_{n-1} > \max\{ X_0, \dots, X_{n-2} \}$.

\end{enumerate}

\end{document}