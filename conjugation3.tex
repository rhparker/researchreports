\documentclass[12pt]{article}
\usepackage[pdfborder={0 0 0.5 [3 2]}]{hyperref}%
\usepackage[left=1in,right=1in,top=1in,bottom=1in]{geometry}%
\usepackage[shortalphabetic]{amsrefs}%
\usepackage{amsmath}
\usepackage{enumerate}
% \usepackage{enumitem}
\usepackage{amssymb}                
\usepackage{amsmath}                
\usepackage{amsfonts}
\usepackage{amsthm}
\usepackage{bbm}
\usepackage[table,xcdraw]{xcolor}
\usepackage{tikz}
\usepackage{float}
\usepackage{booktabs}
\usepackage{svg}
\usepackage{mathtools}
\usepackage{cool}
\usepackage{url}
\usepackage{graphicx,epsfig}
\usepackage{makecell}
\usepackage{array}

\def\noi{\noindent}
\def\T{{\mathbb T}}
\def\R{{\mathbb R}}
\def\N{{\mathbb N}}
\def\C{{\mathbb C}}
\def\Z{{\mathbb Z}}
\def\P{{\mathbb P}}
\def\E{{\mathbb E}}
\def\Q{\mathbb{Q}}
\def\ind{{\mathbb I}}

\DeclareMathOperator{\spn}{span}
\DeclareMathOperator{\ran}{ran}

\graphicspath{ {periodic/} }

\newtheorem{lemma}{Lemma}
\newtheorem{theorem}{Theorem}
\newtheorem{corollary}{Corollary}
\newtheorem{definition}{Definition}
\newtheorem{assumption}{Assumption}
\newtheorem{proposition}{Proposition}
\newtheorem{hypothesis}{Hypothesis}

\newtheorem{notation}{Notation}

\begin{document}

\section{Periodic Multipulses}

Here we summarize what we have shown so far for the stability problem for periodic multi-pulses to equations such as KdV5. Unless otherwise stated, everything here has been proved.

\subsection{Necessary Background}

The background information and hypotheses here are only what is absolutely necessary to make the stability results make sense, i.e. to define the stuff that shows up there. This is far from complete.\\

Consider the PDE

\begin{equation}\label{PDE}
u_t = F_2(u) = \partial_x E'(u)
\end{equation}

where $E'(u)$ is a self-adjoint operator of the form

\[
E'(u) = \partial_x^{2m}u + \tilde{E}(u)
\]

and any derivatives in $\tilde{E}(u)$ are of degree than $2m$. This is motivated by KdV5, which, when written in a moving frame with speed $c$, is

\begin{equation}\label{KdV5}
u_t = \partial_x(u_{xxxx} - u_{xx} - u^2 + c)
\end{equation}

For the linear part of KdV5, all derivatives are odd degree, and the coefficient of the first derivative operator is the speed $c$.\\

A solution $u^*$ satisfying $L(u^*) = 0$ is an equilibrium solution. In particular, we have the following equilibrium solutions.

\begin{enumerate}
	\item The trivial solution 0
	\item A primary pulse solution $q$, which is an even function (for KdV5, uses mountain pass argument; otherwise we assume it exists)
	\item A periodic $n-$pulse solution $q_{np}$ (uses Lin's method)
\end{enumerate}

Let $D F_2(0)$ be the linearization about the equilibrium solution. $D F_2(0)$ has a single eigenvalue at 0. We make the following hypothesis, which is necessary for multi-pulse equilibrium solutions to exist. For KdV5, it is satisfied for speeds $c > 1/4$.

\begin{hypothesis}\label{quartet}
$D F_2(0)$ has a quartet of simple eigenvalues at $\pm \alpha \pm \beta i$, $\alpha, \beta > 0$. The real part of any other nonzero eigenvalue of $D F_2(0)$ is greater than $\alpha$ in magnitude.
\end{hypothesis}

For a given equilibrium solution $u^*$, we can write the PDE eigenvalue problem as the first-order ODE in $\R^{2m+1}$

\begin{equation}
V' = A(u^*(x); \lambda)
\end{equation}

The primary pulse solution $q(x)$ is exponentially localized, thus $A(q(x); \lambda)$ is exponentially asymptotic to $A(0; \lambda)$. We note that $A(0; 0) = D F_2(0)$. The asymptotic matrix $A(0; \lambda)$ is a $(2m + 1) \times (2m + 1)$ matrix of the form

\begin{equation}
A(0; \lambda) = \begin{pmatrix}
0 & 1 & 0 & \dots & 0 & 0 \\
0 & 0 & 1 & \dots & 0 & 0 \\
& & & \ddots & & \\
0 & 0 & 0 & \dots & 0 & 1 \\
\lambda & c_0 & c_1 & \dots & c_{2m-2} & c_{2m-1}
\end{pmatrix}
\end{equation}

where the $c_i$ are constants. Motivated by KdV5, we make the following hypothesis regarding the coefficients of $A(0; \lambda)$.

\begin{hypothesis}\label{A0coeffs}
For the coefficients $c_i$ in $A(0; \lambda)$,
\begin{enumerate}[(i)]
\item $c_0 \neq 0$. For KdV5, $c_0 = -c$.
\item $c_k = 0$ for $k$ odd. This is true for KdV5, and comes from the fact that the linear part of $\partial_x E'(u)$ only involves odd derivatives. 
\end{enumerate}
\end{hypothesis}

$A(0; 0)$ has a eigenvalue at 0. By Hypothesis \ref{quartet}, this eigenvalue at 0 is simple and located a distance $\sqrt{\alpha^2 + \beta^2}$ from the nearest other eigenvalues. Using the IFT and hypothesis \ref{A0coeffs}(i), for sufficiently small $\lambda$, $A(0; \lambda)$ has a simple eigenvalue $\nu(\lambda)$ near 0, given by

\begin{equation}\label{nulambda}
\nu(\lambda) = -\frac{1}{c_0} \lambda + \mathcal{O}(|\lambda|^2)
\end{equation}

Using Hypothesis \ref{A0coeffs}(ii), we also have

\begin{enumerate}
\item $\nu(-\lambda) = -\nu(\lambda)$, i.e. $\nu(\lambda)$ is an odd function 
\item If $\lambda$ is pure imaginary, $\nu(\lambda)$ is also pure imaginary
\end{enumerate}

We have now defined everything that shows up in the stability theorem.

\subsection{Block Matrix Theorem}
\begin{theorem}\label{blockmatrixform}

Let $q_{np}(x)$ be a periodic $n-$pulse solution constructed with lengths $X_0, \dots, X_{n-1}$ from a symmetric primary pulse $q(x)$. The $\lambda$ is an eigenvalue of the PDE eigenvalue problem if and only if the following block matrix equation has a nontrivial solution.

\begin{equation}\label{blockeq}
\begin{pmatrix}
K(\lambda) + C_1 K(\lambda) + K_1(\lambda) & D_1 \\
C_2 K(\lambda) + K_2(\lambda) & A - \lambda^2 MI + D_2
\end{pmatrix}
\begin{pmatrix}c \\ d \end{pmatrix} 
= 0
\end{equation}

where 

\begin{enumerate}

\item $A$ is the $n \times n$ matrix

\begin{align}\label{defA}
A &= \begin{pmatrix}
-a_0 -a_1 & a_0 + a_1 \\
a_0 + a_1 & -a_0 - a_1
\end{pmatrix} && n = 2 \\
A &= \begin{pmatrix}
-a_{n-1} - a_0 & a_0 & & & \dots & a_{n-1}\\
a_0 & -a_0 - a_1 &  a_1 \\
& a_1 & -a_1 - a_2 &  a_2 \\
& & \vdots & & \vdots \\
a_{n-1} & & & & a_{n-2} & -a_{n-2} - a_{n-1} \\
\end{pmatrix} && n > 2 \nonumber
\end{align}

where

\begin{align*}
a_i &= \langle \Psi(X_i), Q'(-X_i) \rangle \\
\end{align*}

\item $K(\lambda)$ is the $n \times n$ matrix

\begin{equation}\label{defKlambda}
K(\lambda) = 
\begin{pmatrix}
e^{-\nu(\lambda)X_1} & & & & & -e^{\nu(\lambda)X_0} \\
-e^{\nu(\lambda)X_1} & e^{-\nu(\lambda)X_2} \\
& -e^{\nu(\lambda)X_2} & e^{-\nu(\lambda)X_3} \\
\vdots & & \vdots & &&  \vdots \\
& & & & -e^{\nu(\lambda)X_{n-1}} & e^{-\nu(\lambda)X_0} 
\end{pmatrix}
\end{equation}

where $\nu(\lambda)$ is the small eigenvalue of the asymptotic matrix $A(0; \lambda)$ as defined above.

\item $M$ is the higher order Melnikov integral

\begin{align*}
M &= \int_{-\infty}^\infty q(y) \partial_c q(y) dy \\
\end{align*}

\item The remainder matrices $C_1, C_2, D_1, D_2$ have uniform bounds

\begin{align*}
C_1, C_2 &= \mathcal{O}(e^{-\tilde{\alpha}X_m}](|\lambda| + e^{-\alpha X_m})) \\
D_1 &= \mathcal{O}((|\lambda| + e^{-\tilde{\alpha} X_m})(|\lambda| + e^{-\alpha X_m})) \\
D_2 &= \mathcal{O}((|\lambda| + e^{-\tilde{\alpha} X_m})(|\lambda| + e^{-\alpha X_m})^2)
\end{align*}

where $X_m = \min \{X_0, \dots, X_{n-1}\}$

\item $K_1(\lambda)$ and $K_2(\lambda)$ are small perturbations of $K(\lambda)$ in which each of the nonzero terms of $K(\lambda)$ are perturbed by $\mathcal{O}(|\lambda| + e^{-\alpha X_m})$.

\end{enumerate}
\end{theorem}

\subsection{Eigenvalues of A}

Before we look at nontrivial solutions to the block matrix equation, we need to look at the eigenvalues of $A$. The interaction eigenvalues should satisfy $\lambda^2 \approx \mu / M$, where $\mu$ is an eigenvalue of $A$, and $M \neq 0$ is the Melnikov integral.\\

We note that $A$ is a real, symmetric matrix, so its eigenvalues will be real. We also note that since all the rows of $A$ sum to 0, $(1, 1, \dots, 1)^T$ is an eigenvector of $A$ with eigenvalue 0.

\begin{enumerate}

\item For $n = 2$, it is easy to compute the eigenvalues of $A$.

\[
\mu = \{ 0, -2(a_0 + a_1) \}
\]

If $a_0 = a_1 = 1$, then

\[
\mu = \{ 0, -4 a \}
\]

\item For $n = 3$, Mathematica can do it.

\[
\mu = \left\{0,  -(a_0 + a_1 + a_2) \pm 
\sqrt{ a_0^2 + a_1^2 + a_2^2 - a_0 a_1 - a_1 a_2 - a_0 a_2 }\right\}
\]

If $a_0 = a_1 = a_2 = a$, this becomes

\[
\mu = \{0,  -3a \}
\]

where the eigenvalue $\mu = -3a$ has algebraic multiplicity 2. 

\item For $n > 4$, Mathematica can no longer do it (thanks, Galois!) 

\item Consider the case where all the distances $X_i$ are equal, i.e. $a_k = a$ for all $k$. In this case, $A$ is a circulant matrix, thus we have eigenvalues

\begin{align*}
\mu_k &= 2 a\left( \cos \frac{2 \pi k}{n}  - 1 \right) & k = 1, \dots, n
\end{align*}

In this case,

\begin{enumerate}
	\item $\mu_n = 0$
	\item The remaining $\mu_k$ are either all positive (if $a < 0$) or all negative (if $a > 0$)
	\item If $n$ is odd, the collection $\mu_1, \dots, \mu_{n-1}$ will contain $(n-1)/2$ distinct eigenvalues, all of which have algebraic multiplicity 2.
	\item If $n$ is even, we will have $\mu_{n/2} = -4a$. The collection of $\mu_k$ which are not yet accounted for contains $(n-2)/2$ distinct eigenvalues, all of which have algebraic multiplicity 2.
\end{enumerate}

\end{enumerate}

We are also interested in the case when ``periodic length'' $X_{n-1}$ is large, so $a_{n-1}$ is very close to 0. It is useful to look at the eigenvalues of $A$ when this is the case, i.e. $A$ is the tridiagonal matrix. 

\begin{align}\label{Atridiag}
A &= \begin{pmatrix}
-a_0 & a_0 \\
a_0 & -a_0 
\end{pmatrix} && n = 2 \\
A &= \begin{pmatrix}
- a_0 & a_0 & & &  \\
a_0 & -a_0 - a_1 &  a_1 \\
& a_1 & -a_1 - a_2 &  a_2 \\
& & & \ddots \\
& & & a_{n-2} & -a_{n-2} \\
\end{pmatrix} && n > 2 \nonumber
\end{align}

As above, $(1, 1, \dots, 1)^T$ is an eigenvector of $A$ with eigenvalue 0.

\begin{enumerate}

\item For $n = 2$, we have eigenvalues

\[
\mu = \{ 0, -2 a_0 \}
\]

\item For $n = 3$, we have eigenvalues

\[
\mu = \left\{ 0,  -(a_0 + a_1) \pm 
\sqrt{a_0^2 + a_1^2 - a_0 a_1} \right\}
\]

If $a_0 = a_1 = a$, this becomes

\[
\mu = \{0,  -a, -3a \}
\]

and this time these are distinct.

\item Consider the case where all the distances $X_i$ are equal, i.e. $a_k = a$ for all $k$. Unfortunately, $A$ is not a circulant matrix. I have proved (and verified by Mathematica!) that the eigenvalues are 

\begin{align*}
\mu_k &= 2 a\left( \cos \frac{\pi k}{n}  - 1 \right) & k = 1, \dots, n
\end{align*}

(The only difference between this and the ``periodic'' version above, is that there is not a 2 in front of the $\pi$). We have the following

\begin{enumerate}
	\item $\mu_n = 0$ always
	\item The remaining $\mu_k$ are either all positive (if $a < 0$) or all negative (if $a > 0$)
	\item All the eigenvalues $\mu_k$ are distinct. 
\end{enumerate}

\item Consider the general case. Although we cannot find the eigenvalues themselves, we can determine their signs using Lemma 5.4 in San98 (and noting that that matrix there is $-A$)

\begin{enumerate}
	\item $A$ has $k_+$ negative real eigenvalues (counting multiplicity), where $k_+$ is the number of positive $a_i$.
	\item $A$ has $k_-$ positive real eigenvalues (counting multiplicity), where $k_-$ is the number of negative $a_i$.
\end{enumerate}

\end{enumerate}

Since our stability result will not hold if the eigenvalues of $A$ are not simple, we make the following hypothesis.

\begin{hypothesis}\label{Asimpleeigs}
The eigenvalues of $A$, defined in \eqref{defA}, are simple.
\end{hypothesis}

Using this hypothesis, let $0, \mu_1, \dots, \mu_{n-1}$ be the distinct eigenvalues of $A$.

\subsection{Singularities of K}

Next, we look at when the matrix $K(\lambda)$, defined in \eqref{defKlambda}, is singular. Evaluating $\det K(\lambda)$, we have

\begin{equation}
\det K(\lambda) = -2 \sinh( \nu(\lambda) X )
\end{equation}

where $\nu(\lambda)$ is the small eigenvalue of $A(0; \lambda)$ given in \eqref{nulambda} and $X = X_0 + X_1 + \dots + X_{n-1}$ is half the length of the domain. Thus $K(\lambda)$ is singular if and only if $\nu(\lambda) = k \pi i/X$ for an integer $k$. \\

In terms of $\lambda$, for sufficiently small $k/X$, $K(\lambda)$ is singular if and only if $\lambda = \lambda(X, k)$, where

\begin{align}\label{lambdaXk}
\lambda(X,k)
&= -c_0 \frac{k \pi i }{X} + \mathcal{O}\left(\frac{k}{X}\right)^2 \\
\end{align} 

$\lambda(X,0) = 0$, and by Hypothesis \ref{A0coeffs}, $\lambda(X, -k) = -\lambda(X, k)$ and is pure imaginary for $k \neq 0$.

\subsection{Stability Result}

We know that there is an eigenvalue with algebraic multiplicity 2 at the origin. This comes from translation invariance, and we know what the corresponding eigenfunctions are. We might be able to show there is a third eigenvalue at 0 from some energy argument that I don't yet understand.\\

The strategy is to solve the block matrix equation for $\lambda$ in three steps.

\begin{enumerate}
	\item Find the nonzero interaction eigenvalues, which come from the second line of the block matrix equation, and are close to the points where $(A - \lambda^2 M I)$ is singular.
	\item Find the nonzero ``essential spectrum'' eigenvalues, which come from the first line of the block matrix equation, and are close to the points where $K(\lambda)$ is singular.
	\item Determine the number of eigenvalues in a ball around the origin which is sufficiently large to enclose all the interaction eigenvalues. This will show that there is an additional eigenvalue at 0 and that we have found all the eigenvalues in the ball.
\end{enumerate}

Recall that the eigenvalues of $A$ are given by $0, \mu_1, \dots, \mu_{n-1}$. Then $A - \lambda^2 M$ is singular at the $2n$ points $\lambda = 0, \pm \sqrt{\mu_1/M}, \dots, \pm \sqrt{\mu_{n-1}/M}$. Recall that $K(\lambda)$ is singular at $\lambda = \lambda(X, k)$ for integer $k$.\\

We will need to assume that the points where $A - \lambda^2 M I$ and $K(\lambda)$ are singular (other than at $\lambda = 0$) are not too close to each other. For KdV5, numerical simulations have shown Krein bubbles when this occurs, so it is reasonable to take this hypothesis.

\begin{hypothesis}\label{epsilonballs}
The points $\lambda(X, k)$ and $\sqrt{\mu_k/M}$ (for $k = 1, \dots, n-1$) are separated by at least $\epsilon$, where $\epsilon$ will be specified later. 
\end{hypothesis}

\subsubsection{Interaction Eigenvalues}

To solve for the interaction eigenvalues, we need to invert $K(\lambda)$, which we can do since $\lambda$ is more than $\epsilon$ away from the $\lambda(X,k)$. In Hypothesis \ref{epsilonballs}, choose

\[
\epsilon = e^{-\alpha X_m/2} \frac{1}{X}
\]

Note that for small $k/X$, the points $\lambda(X, k)$ are approximately evenly spaced by $c_0 \pi/X$. For sufficiently large $X_m$, $e^{-\alpha X_m/2}$ is much less than 1, so there is plenty of room for interaction eigenvalues to fall between the $\lambda(X, k)$.\\

Then there are $2n-2$ nonzero interaction eigenvalues which come in pairs and are either real or purely imaginary. These interaction eigenvalues are given by $\lambda = \pm \lambda^i(X_m; k)$, $k = 1, \dots, n-1$, where

\begin{equation}\label{inteigs}
\lambda^i(X_m; k) = \sqrt{\mu_k / M} + \mathcal{O}(e^{-3 \alpha X_m/2})
\end{equation}

The remainder term cannot move this off of the real or imaginary axis. We should be able to show that these interaction eigenvalues are all $\mathcal{O}(e^{-\alpha X_m})$.

\subsubsection{``Essential Spectrum'' Eigenvalues}

To solve for the ``essential spectrum'', we need to invert $A - \lambda^2 M I$, which we can again do since $\lambda$ is more than $\epsilon$ away from $\sqrt{\mu_k/M}$ ($k = 1, \dots, n-1$). However, to get a sufficient bound to make this work, we need to choose a larger $\epsilon$. In Hypothesis \ref{epsilonballs}, choose

\[
\epsilon = e^{-\alpha X_m/(n+1)} \frac{1}{X}
\]

where $n$ is the number of pulses we are dealing with. We can get this sufficiently small by choosing $X_m$ sufficiently large.\\ 

We also require an additional hypothesis which places a restriction on how large $X$ can be.

\begin{hypothesis}\label{Xrestriction}
We take the following restriction on $X = X_0 + \dots + X_{n-1}$.
\[
X \leq C e^{\alpha X_m}
\]
where the constant $C$ is at most $\mathcal{O}(1)$.
\end{hypothesis}

With these additional restrictions, we can find the ``essential spectrum'' eigenvalues. For any positive integer $k$ such that $k/X$ is sufficiently small, we have a pair of purely imaginary ``essential spectrum'' eigenvalues, which are given by $\lambda = \pm \lambda^c(X_m; X, k)$, where

\[
\lambda^c(X_m; X, k) = -c_0 \frac{k \pi i}{X} + \mathcal{O}\left( \frac{1}{X_m} \frac{k}{X} \right)
\]

As with the interaction eigenvalues, the remainder term cannot move these off of the imaginary axis.\\

Although it would be nice to not require Hypothesis \ref{Xrestriction}, it is required to get a good enough bound when inverting $A - \lambda^2 M$. Any method of finding the ``essential spectrum'' eigenvalues which does not require Hypothesis \ref{Xrestriction} must somehow avoid inverting this matrix. 

\subsubsection{Rest of the Picture}

All that remains to do is show that there are three eigenvalues at 0 (including multiplicity) and that there are no other eigenvalues close to 0 that we have not yet accounted for. We already know about an eigenvalue at 0 with algebraic multiplicity 2 which comes from translation invariance.\\

Let $r = C_0 e^{-\alpha X_m}$, where $C_0 = \mathcal{O}(1)$, and choose $C_0$ such that the open ball $B(0, r)$ contains all the interaction eigenvalues and that its boundary is as far from the ``essential spectrum'' eigenvalues as possible. (For this second condition, just put it half way between two ``essential spectrum'' eigenvalues).\\

Then, using Rouche's theorem, we can show that

\begin{enumerate}
\item There is an additional eigenvalue at 0, which comes from the fact that $K(0)$ is singular.
\item There are no other eigenvalues inside $B(0, r)$ other than those which have already been discussed.
\end{enumerate}


\end{document}