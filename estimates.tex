% \documentclass{book}

\documentclass[12pt]{article}
\usepackage[pdfborder={0 0 0.5 [3 2]}]{hyperref}%
\usepackage[left=1in,right=1in,top=1in,bottom=1in]{geometry}%
\usepackage[shortalphabetic]{amsrefs}%
\usepackage{amsmath}
\usepackage{enumerate}
\usepackage{enumitem}
\usepackage{amssymb}                
\usepackage{amsmath}                
\usepackage{amsfonts}
\usepackage{amsthm}
\usepackage{bbm}
\usepackage[table,xcdraw]{xcolor}
\usepackage{tikz}
\usepackage{float}
\usepackage{booktabs}
\usepackage{svg}
\usepackage{mathtools}
\usepackage{cool}
\usepackage{url}
\usepackage{graphicx,epsfig}
\usepackage{makecell}
\usepackage{array}

\def\noi{\noindent}
\def\T{{\mathbb T}}
\def\R{{\mathbb R}}
\def\N{{\mathbb N}}
\def\C{{\mathbb C}}
\def\Z{{\mathbb Z}}
\def\P{{\mathbb P}}
\def\E{{\mathbb E}}
\def\Q{\mathbb{Q}}
\def\ind{{\mathbb I}}

\newtheorem{lemma}{Lemma}
\newtheorem{definition}{Definition}

\begin{document}

\subsection*{Estimates}
Here are some estimates like in Lemma 3.1 in Sandstede (1998)

\begin{lemma}We have the estimates
\begin{align*}
|G_i(x)| &\leq C|R_i(x)| \leq C \sup_{|x| \geq L} |Q(x)| \\
| B Q_2(x) - B Q(x) | & \leq C |R_i(x)| \leq C \sup_{|x| \geq L} |Q(x)| \\
D_1 d &= (Q'(L) + Q'(-L))(d_2 - d_1) +\mathcal{O}\left( e^{-\alpha L} |d| \sup_{|x| \geq L} |Q(x)| \right)\\
|A(Q_2(x))| &\leq C \textrm{ for all }x
\end{align*}
where $\alpha > 0$ is defined as on pages 432 and 434 of Sandstede (1998).
\begin{proof}
The first estimate is the same as in Sandstede (1998) with $Q$ replacing $Q^+$ and $Q^-$, and follows from the smoothness of $A$ together with (2.6)(i) in Sandstede (1998). The second estimate follows from (2.6)(i) in Sandstede (1998) and the expansion of $Q^2$ as $Q + R_i$. The third estimate is as in Lemma 3.1 of Sandstede (1998). The fourth estimate follows since the double pulse $q_2(x)$ is bounded and the rest of the matrix $A$ is constant.
\end{proof}
\end{lemma}

\subsection*{Reduction}
This corresponds to section 3.2 in Sandstede (1998). The integrated eigenvalue problem we are dealing with is

\begin{equation}\label{inteigQ}
(W_i^\pm)' = A(Q)W_i^\pm - \lambda^2 d_i KBQ_{c} + \lambda K_i^\pm BW_i^\pm + G_i^\pm W_i^\pm - \lambda^2 d_i K_i^\pm B (R_i^\pm)_c
\end{equation}

For the purposes of doing estimates, we can simplify this a bit by recombining $Q_c$ and $(R_i^\pm)_c$ to get the double pulse version (and integrating from $-\infty$ since it is no longer piecewise).

\[
Q_c + (R_i^\pm)_c = Q_{2c}
\]

Thus have the piecewise system

\begin{equation}\label{inteigQ2}
(W_i^\pm)' = A(Q)W_i^\pm + G_i^\pm W_i^\pm - \lambda^2 d_i K B Q_{2c} + \lambda K_i^\pm BW_i^\pm 
\end{equation}

subject to conditions

\begin{align*}
W_i^-(0) &= W_i^+(0) \\
W_1^+(L) - W_2^-(-L) &= D_1 d \\
W_i^\pm(x) &\in \C \psi(0) \oplus Y^+ \oplus Y^- \\
W_i^+(0) - W_i^-(0) &\in \C \psi(0) 
\end{align*}

where

\begin{align*}
G_i^\pm(x) &= A(Q_2) - A(Q)) = A(Q + R_i^\pm) - A(Q) \\
D_1 d &= d_2 [ Q'(-L) + R_2'(-L)] - d_1 [ Q'(L) + R_1'(L) ]
\end{align*}

Suppose we have the bounds
\begin{align*}
|G_i^\pm(x)| \leq \delta \\
% |Q_2(x) - Q(x) | \leq \delta \\
|\lambda| \leq \delta \\
\end{align*}

To deal with the integration operator, we will need to work in exponentially weighted spaces, since we have an extra integral involved. Given an exponential weight $\eta > 0$, we define the following exponentially weighted spaces.

\begin{align*}
C^0_\eta[-a, 0] &= \{ f \in C^0[-a, 0] : \sup_{x \in [-a, 0]} |e^{-\eta x} f(x) | < \infty \} && a \geq 0 \\
C^0_\eta[0, a] &= \{ f \in C^0[0, a] : \sup_{x \in [0, a]} |e^{\eta x} f(x) | < \infty \} && a \geq 0 
\end{align*}

Since we want our functions in this space to decay exponentially at $\pm \infty$, we use the negative weight for $x \leq 0$ and the positive weight for $x \geq 0$. The corresponding norms are 

\begin{align*}
|| f ||_\eta &= \sup_{x \in [-a, 0]} |e^{-\eta x} f(x) | && f \in C^0_\eta[-a, 0] \\
|| f ||_\eta &= \sup_{x \in [0, a]} |e^{\eta x} f(x) | && f \in C^0_\eta[0, a] \\
\end{align*}

We define the same spaces as in (3.12) in Sandstede (1998), except we use the weighted spaces for the space $V_w$. To indicate that we have used the exponential weight $\eta$, we will call that space $V_w^\eta$. For the norms on these direct sums, we use the max of the norms on the individual spaces, since each is a finite sum.\\

Following (3.14) in Sandstede (1998) we write our ODE system as the fixed point equation

\begin{align*}
W_i^-(x) &= \Phi^s_-(x, -X_{i-1})a^-_{i-1} + \Phi^u_-(x, 0)b_i^- \\
&+ \int_0^x \Phi^u_-(x, y)[G_i^-(y) W_i^-(y) + \lambda (K_i^- B W_i^-)(y) - \lambda^2 d_i B Q_{2c}(y) ] dy \\
&+ \int_{-X_{i-1}}^x \Phi^s_-(x, y)[G_i^-(y) W_i^-(y) + \lambda (K_i^-B W_i^-)(y) - \lambda^2 d_i B Q_{2c}(y) ] dy \\
W_i^+(x) &= \Phi^u_+(x, X_i)a^+_{i} + \Phi^s_+(x, 0)b_i^+ \\
&+ \int_0^x \Phi^s_+(x, y)[G_i^+(y) W_i^+(y) + \lambda (K_i^+ B W_i^+)(y) - \lambda^2 d_i B Q_{2c}(y) ] dy \\
&+ \int_{X_{i}}^x \Phi^u_+(x, y)[G_i^+(y) W_i^+(y) + \lambda (K_i^+ B W_i^+)(y) - \lambda^2 d_i B Q_{2c}(y) ] dy
\end{align*}

The idea is that we integrate L to R along the stable projection and from R to L along the unstable projection. The $X_i^\pm$ are the appropriate interval endpoints, which for this case are $(X_0, X_1, X_2) = (-\infty, L, \infty)$.\\

From here, we will (attempt to) proceed as in section 3 of Sandstede (1998) via a series of lemmas. In the first lemma, we consider a linear operator consisting of the terms in the fixed point equation which involve $W_i^\pm$. 

\begin{lemma}

Let $L_1(\lambda): V_w^\eta \rightarrow V_w^\eta$ be the linear operator defined piecewise by

\begin{align*}
(L_1(\lambda)W)_i^-(x) = \int_0^x &\Phi^u_-(x, y)[G_i^-(y) W_i^-(y) + \lambda (K_i^- B W_i^-)(y) ] dy \\
&+ \int_{-X_{i-1}}^x \Phi^s_-(x, y)[G_i^-(y) W_i^-(y) + \lambda (K_i^-B W_i^-)(y) ] dy \\
(L_1(\lambda)W)_i^+(x) = \int_0^x &\Phi^s_+(x, y)[G_i^+(y) W_i^+(y) + \lambda (K_i^+ B W_i^+)(y)] dy \\
&+ \int_{X_{i}}^x \Phi^u_+(x, y)[G_i^+(y) W_i^+(y) + \lambda (K_i^+ B W_i^+)(y) ] dy
\end{align*}

where $G_i^\pm$ is uniformly bounded with supremum norm $|G|$. Then $L_1(\lambda)$ is bounded in the weighted supremum norm. This bound does not depend on the interval endpoints $X_i^\pm$.

\begin{proof}
First, we look at the negative piece. Note that when taking the absolute value we have to integrate in the positive direction.

\begin{align*}
|e^{-\eta x} & (L_1(\lambda)W_i^-)(x) | \leq  e^{-\eta x} \int_x^0 |\Phi^u_-(x, y)|[|G_i^-(y)||W_i^-(y)| + |\lambda||(K_i^- B W_i^-)(y)| ] dy \\
&+ e^{-\eta x} \int_{-X_{i-1}}^x |\Phi^s_-(x, y)|[|G_i^-(y)||W_i^-(y)| + |\lambda||(K_i^- B W_i^-)(y)| ] dy \\
&\leq |G| \int_x^0 e^{\alpha^u (x-y)}e^{-\eta(x-y)}|e^{-\eta y} W_i^-(y)| dy 
+ |\lambda|\int_x^0 e^{\alpha^u (x-y)}e^{-\eta(x-y)}|e^{-\eta y} (K_i^- B W_i^-)(y)| dy \\
&+ |G| \int_{-X_{i-1}}^x e^{-\alpha^s (x-y)}e^{-\eta(x-y)}|e^{-\eta y} W_i^-(y)| dy 
+ |\lambda|\int_{-X_{i-1}}^x e^{-\alpha^s (x-y)}e^{-\eta(x-y)}|e^{-\eta y} (K_i^- B W_i^-)(y)| dy  \\ 
\end{align*}

The two terms not involving the integration operator are similar to those in Sandstede (1998), so let's take care of those first.

\begin{align*}
\int_x^0 e^{\alpha^u (x-y)}e^{-\eta(x-y)}|e^{-\eta y} W_i^-(y)| dy &= \int_x^0 e^{(\alpha^u - \eta) (x-y)}|e^{-\eta y} W_i^-(y)| dy \\
&\leq ||W_i^-||_\eta \frac{1 - e^{(\alpha^u - \eta)x}}{\alpha^u - \eta}
\end{align*}

We need the RHS of this to be positive, which is true when $\eta < \alpha^u$. Provided this is the case, we have the bound

\begin{align*}
\int_x^0 e^{\alpha^u (x-y)}e^{-\eta(x-y)}|e^{-\eta y} W_i^-(y)| dy &\leq ||W_i^-||_\eta \frac{1}{\alpha^u - \eta}
\end{align*}

For the other one:

\begin{align*}
\int_{-X_{i-1}}^x e^{-\alpha^s (x-y)}e^{-\eta(x-y)}|e^{-\eta y} W_i^-(y)| dy &= \int_{-X_{i-1}}^x e^{(-\alpha^s - \eta) (x-y)}|e^{-\eta y} W_i^-(y)| dy \\
&\leq ||W_i^-||_\eta \int_{-\infty}^x e^{(-\alpha^s - \eta) (x-y)} dy = ||W_i^-||_\eta \frac{1}{\alpha^s + \eta}
\end{align*}

So this has a nice bound independent of the $X_i$. The RHS is always positive, so this doesn't give us any extra conditions on $\eta$. \\

So far, all of this would have worked without the exponential weight. Where we need it is in the terms involving the integration operator. Let's do those terms now. Recall that because of the operator $B$ we are only integrating the scalar function $w_i^\pm$

\begin{align*}
\int_x^0 e^{\alpha^u (x-y)}e^{-\eta(x-y)}|e^{-\eta y} (K_i^- B W_i^-)(y)| dy &\leq \int_x^0 e^{(\alpha^u - \eta)(x-y)}e^{-\eta y} \int_{a_i^-}^y |w_i^-(u)| du dy \\
&= \int_x^0 e^{(\alpha^u - \eta)(x-y)}e^{-\eta y} \int_{a_i^-}^y e^{\eta u} |e^{-\eta u} w_i^-(u)| du dy \\
&\leq ||W_i^-||_\eta \int_x^0 e^{(\alpha^u - \eta)(x-y)}e^{-\eta y} \int_{-\infty}^y e^{\eta u} du dy \\
&= \frac{||W_i^-||_\eta}{\eta} \int_x^0 e^{(\alpha^u - \eta)(x-y)}e^{-\eta y} e^{\eta y} dy \\
&= \frac{||W_i^-||_\eta}{\eta} \frac{1 - e^{(\alpha^u - \eta)x}}{\alpha^u - \eta} 
\end{align*}

We need the RHS of this to be positive, which is true when $\eta < \alpha^u$, in which case we have the bound

\[
\int_x^0 e^{\alpha^u (x-y)}e^{-\eta(x-y)}|e^{-\eta y} (K_i^- B W_i^-)(y)| dy \leq 
\frac{||W_i^-||_\eta}{\eta} \frac{1}{\alpha^u - \eta} 
\]

We had that condition from before, so that is good.\\

Here is the other integral term. 

\begin{align*}
\int_{-X_{i-1}}^x e^{-\alpha^s (x-y)}e^{-\eta(x-y)}|e^{-\eta y} (K_i^- B W_i^-)(y)| dy &\leq \int_{-X_{i-1}}^x e^{(-\alpha^s - \eta)(x-y)}e^{-\eta y} \int_{a_i^-}^y |w_i^-(u)| du dy \\
&= \int_{-X_{i-1}}^x e^{(-\alpha^s - \eta)(x-y)}e^{-\eta y} \int_{a_i^-}^y e^{\eta u} |e^{-\eta u} w_i^-(u)| du dy \\
&\leq ||W_i^-||_\eta \int_{-\infty}^x e^{(-\alpha^s - \eta)(x-y)}e^{-\eta y} \int_{-\infty}^y e^{\eta u} du dy \\
&= \frac{||W_i^-||_\eta}{\eta} \int_{-\infty}^x e^{(-\alpha^s - \eta)(x-y)}e^{-\eta y} e^{\eta y} dy \\
&= \frac{||W_i^-||_\eta}{\eta} \frac{1}{\alpha^s + \eta}
\end{align*}

This also has a nice bound independent of the $X_i$. The RHS is always positive, so this doesn't give us any extra conditions on $\eta$.\\

So far we have the condition $\eta < \alpha^u$. We expect to also have the condition $\eta < \alpha^s$. We should get that from using the ``+'' equations.\\

The ``+'' equations are similar to the ``-'' equations. We could probably just state the result, but we will show it here for completeness. Recall here that since $x \geq 0$, we have to multiply by $e^{\eta x}$ to get the weighted norm. We also must integrate from L to R when taking the absolute value.\\

\begin{align*}
|e^{\eta x} & (L_1(\lambda)W)_i^+)(x) | \leq e^{\eta x} \int_0^x |\Phi^s_+(x, y)|[|G_i^+(y)||W_i^+(y)| + |\lambda|(K_i^+ B W_i^+)(y)| ] dy \\
&+ e^{\eta x} \int_x^{X_i} |\Phi^u_+(x, y)|[|G_i^+(y)||W_i^+(y)| + |\lambda||(K_i^+ B W_i^+)(y)| ] dy \\
&\leq |G| \int_0^x e^{-\alpha^s (x-y)}e^{\eta(x-y)}|e^{\eta y} W_i^+(y)| dy 
+ |\lambda|\int_0^x e^{-\alpha^s (x-y)}e^{\eta(x-y)}|e^{\eta y} (K_i^+ B W_i^+)(y)| dy \\
&+ |G| \int_x^{X_i} e^{\alpha^u (x-y)}e^{\eta(x-y)}|e^{\eta y} W_i^+(y)| dy 
+ |\lambda|\int_x^{X_i} e^{\alpha^u (x-y)}e^{\eta(x-y)}|e^{\eta y} (K_i^+ B W_i^+)(y)| dy \\ 
\end{align*}

We do the same thing we did above.

\begin{align*}
\int_0^x e^{-\alpha^s (x-y)}e^{\eta(x-y)}|e^{\eta y} W_i^+(y)| dy &= \int_0^x e^{(-\alpha^s + \eta) (x-y)}|e^{\eta y} W_i^-(y)| dy \\
&\leq ||W_i^+||_\eta \frac{1 - e^{-(\alpha^s - \eta)x} }{\alpha^s - \eta}
\end{align*}

We need the RHS of this to be positive, which is true when $\eta < \alpha^s$. (This is the condition we were expecting). In that case, we have the bound

\begin{align*}
\int_0^x e^{-\alpha^s (x-y)}e^{\eta(x-y)}|e^{\eta y} W_i^+(y)| dy &\leq ||W_i^+||_\eta \frac{1}{\alpha^s - \eta}
\end{align*}

For the other one:

\begin{align*}
\int_x^{X_i} e^{\alpha^u (x-y)}e^{\eta(x-y)}|e^{\eta y} W_i^+(y)| dy &= \int_x^{X_i} e^{(\alpha^u + \eta) (x-y)}|e^{\eta y} W_i^+(y)| dy \\
&\leq ||W_i^+||_\eta \int_x^{\infty} e^{(\alpha^u + \eta) (x-y)} dy = ||W_i^+||_\eta \frac{1}{\alpha^u + \eta}
\end{align*}

Again, this bound is independent of the $X_i$. The RHS is always positive, so this doesn't give us any extra conditions on $\eta$. \\

Now we do the integral terms. Since our integration operators $K_i^+$ integrate from R to L, when we take the absolute value, we need to change our limits so we are integrating from L to R.

\begin{align*}
\int_0^x e^{-\alpha^s (x-y)}e^{\eta(x-y)}|e^{\eta y} (K_i^+ B W_i^+)(y)| dy &\leq \int_0^x e^{(-\alpha^s + \eta)(x-y)}e^{\eta y} \int_y^{a_i^+} |w_i^+(u)| du dy \\
&= \int_0^x e^{(-\alpha^s + \eta)(x-y)}e^{\eta y} \int_y^{a_i^+} e^{-\eta u} |e^{\eta u} w_i^+(u)| du dy \\
&\leq ||W_i^+||_\eta \int_0^x e^{(-\alpha^s + \eta)(x-y)}e^{\eta y} \int_y^\infty e^{-\eta u} du dy \\
&= \frac{||W_i^+||_\eta}{\eta} \int_0^x e^{(-\alpha^s + \eta)(x-y)}e^{\eta y} e^{-\eta y} dy \\
&= \frac{||W_i^+||_\eta}{\eta} \frac{1 - e^{-(\alpha^s - \eta)x}}{\alpha^s - \eta} 
\end{align*}

We need the RHS of this to be positive, which is true when $\eta < \alpha^s$, in which case we have the bound

\[  
\int_0^x e^{-\alpha^s (x-y)}e^{\eta(x-y)}|e^{\eta y} (K_i^+ B W_i^+)(y)| dy \leq
\frac{||W_i^+||_\eta}{\eta} \frac{1}{\alpha^s - \eta} 
\]

We had that condition from before, so that is good.\\

For the final integral,

\begin{align*}
\int_x^{X_i} e^{\alpha^u (x-y)}e^{\eta(x-y)}|e^{\eta y} (K_i^+ B W_i^+)(y)| dy &\leq \int_x^{X_i} e^{(\alpha^u + \eta)(x-y)}e^{\eta y} \int_y^{a_i^+} |w_i^+(u)| du dy \\
&= \int_x^{X_i} e^{(\alpha^u + \eta)(x-y)}e^{\eta y} \int_y^{a_i^+} e^{-\eta u} |e^{\eta u} w_i^-(u)| du dy \\
&\leq ||W_i^+||_\eta \int_x^\infty e^{(\alpha^u + \eta)(x-y)}e^{\eta y} \int_y^\infty e^{-\eta u} du dy \\
&= \frac{||W_i^+||_\eta}{\eta} \int_x^\infty e^{(\alpha^u + \eta)(x-y)}e^{\eta y} e^{-\eta y} dy \\
&= \frac{||W_i^+||_\eta}{\eta} \frac{1}{\alpha^u + \eta}
\end{align*}

Putting this all together, we conclude the following. The operator $L_1$ defined above is a bounded linear operator from $V_w^\eta$, the exponentially weighted version of $V_w$, to itself. Since the norm on the product space $V_w^\eta$ is the maximum of the individual (weighted) norms making up the product, we have the bound:

\begin{equation}
	||L_1(\lambda)W||_\eta \leq \left(|G| + \frac{|\lambda|}{\eta}\right)\left(\frac{1}{\alpha^s + \eta} + \frac{1}{\alpha^u + \eta} + \frac{1}{\alpha^s - \eta} + \frac{1}{\alpha^u - \eta}\right)||W||_\eta
\end{equation}

provided that $\eta \leq \alpha^2, \alpha^u$, where $W$ is the element of $V_w^\eta$ formed by the pieces $W_i^\pm$. If we have $|\lambda| \leq \delta$ and $|G| \leq \delta$, this bound becomes

\begin{equation}
	||L_1(\lambda)W||_\eta \leq C \delta ||W||_\eta
\end{equation}

\end{proof}
\end{lemma}

In the next lemma, we consider a linear operator consisting of the remaining terms in the fixed point equation, i.e. those which do not involve $W_i^\pm$. This bound does not actually require the exponentially weighted space, but we do need to use it since the previous operator $L_1$ acts on that space.

\begin{lemma}

Let $L_2(\lambda): V_a \times V_b \times V_d \rightarrow V_w^\eta$ be the linear operator defined piecewise by

\begin{align*}
L_2(\lambda)(a, b, d)_i^-(x) &= \Phi^s_-(x, -X_{i-1})a^-_{i-1} + \Phi^u_-(x, 0)b_i^- \\
&- \lambda^2 d_i \left( \int_0^x \Phi^u_-(x, y)(KBQ_{2c})(y) dy  + \int_{-X_{i-1}}^x \Phi^s_-(x, y)(KBQ_{2c})(y) dy \right)\\
L_2(\lambda)(a, b, d)_i^+(x) &= \Phi^u_+(x, X_i)a^+_{i} + \Phi^s_+(x, 0)b_i^+ \\
&- \lambda^2 d_i \left( \int_0^x \Phi^s_+(x, y)(KBQ_{2c})(y) dy + \int_{X_{i}}^x \Phi^u_+(x, y)(KBQ_{2c})(y) dy \right)
\end{align*}

Then $L_2$ is a bounded linear operator whose bound does not depend on the endpoints $X_i$.

\begin{proof}

As in the previous lemma, we will take $0 < \eta < \alpha^s, \alpha^u$. For the ``minus'' piece, recall that $a_0 = 0$ (so the term involving $a^-_{i-1}$ is zero when $i = 1$ and that $-X_{i-1} \leq x \leq 0$. When we take the absolute value, we need to integrate in the positive direction.

\begin{align*}
| e^{-\eta x} L_2(\lambda)(a, b, d)_i^-(x)| &\leq e^{-\eta x} |\Phi^s_-(x, -X_{i-1})a^-_{i-1}| + e^{-\eta x}|\Phi^u_-(x, 0)b_i^-| \\
&+ |\lambda|^2 |d_i| e^{-\eta x} \left( \int_x^0 |\Phi^u_-(x, y)||(KBQ_{2c})|(y) dy  + \int_{-X_{i-1}}^x |\Phi^s_-(x, y)||(KBQ_{2c})(y)| dy \right) \\
&\leq e^{-\eta x} e^{-\alpha^s [x - (-X_{i-1})]}|a^-_{i-1}| + e^{-\eta x}e^{\alpha^u x}|b_i^-| \\
&+ |\lambda|^2 |d_i| e^{-\eta x} \left( \int_x^0 e^{\alpha^u (x - y)}\int_{-\infty}^y |q_{2c}(z)| dz dy  + \int_{-X_{i-1}}^x e^{-\alpha^s (x - y)}\int_{-\infty}^y |q_{2c}(z)| dz dy\right) \\
&= e^{-(\alpha^s + \eta) [x - (-X_{i-1})]} e^{\eta X_{i-1}}|a^-_{i-1}| + e^{(\alpha^u - \eta) x}|b_i^-| \\
&+ |\lambda|^2 |d_i| e^{-\eta x} \left( \int_x^0 e^{\alpha^u (x - y)}\int_{-\infty}^y |q_{2c}(z)| dz dy  + \int_{-X_{i-1}}^x e^{-\alpha^s (x - y)}\int_{-\infty}^y |q_{2c}(z)| dz dy\right) \\
&= e^{\eta X_{i-1}}|a^-_{i-1}| + |b_i^-| \\
&+ |\lambda|^2 |d_i| e^{-\eta x} \left( \int_x^0 e^{\alpha^u (x - y)}\int_{-\infty}^y |q_{2c}(z)| dz dy  + \int_{-X_{i-1}}^x e^{-\alpha^s (x - y)}\int_{-\infty}^y |q_{2c}(z)| dz dy\right) \\
\end{align*}

For the first integral, we have

\begin{align*}
e^{-\eta x} \int_x^0 e^{\alpha^u (x - y)}\int_{-\infty}^y |q_{2c}(z)| dz dy &=
\int_x^0 e^{-\eta (x-y)} e^{\alpha^u (x - y)}e^{-\eta y}\int_{-\infty}^y |q_{2c}(z)| dz dy \\
&= \int_x^0 e^{(\alpha^u - \eta) (x - y)}\int_{-\infty}^y e^{-\eta(y-z)} |e^{-\eta z} q_{2c}(z)| dz dy \\
&\leq ||q_{2c}||_\eta \int_x^0 e^{(\alpha^u - \eta) (x - y)}\int_{-\infty}^y e^{-\eta(y-z)}  dz dy \\
&= \frac{ ||q_{2c}||_\eta }{\eta} \int_x^0 e^{(\alpha^u - \eta) (x - y)} dy \\
&= \frac{ ||q_{2c}||_\eta }{\eta} \frac{1- e^{(\alpha^u - \eta)x}}{\alpha^u - \eta} \\
&\leq \frac{ ||q_{2c}||_\eta }{\eta} \frac{1}{\alpha^u - \eta}
\end{align*}

where in the last line we used the fact that $x \leq 0$. We also used the fact that $q_c$ is bounded in the exponentially weighted space with weight $\eta$, thus for the double pulse $q_{2c}$ is as well.\\

For the second integral,

\begin{align*}
e^{-\eta x} \int_{-X_{i-1}}^x e^{-\alpha^s (x - y)}\int_{-\infty}^y |q_{2c}(z)| dz dy &=
\int_{-X_{i-1}}^x e^{-\eta (x-y)} e^{-\alpha^s (x - y)}e^{-\eta y}\int_{-\infty}^y |q_{2c}(z)| dz dy \\
&= \int_{-X_{i-1}}^x e^{-(\alpha^s + \eta) (x - y)}\int_{-\infty}^y e^{-\eta(y-z)} |e^{-\eta z} q_{2c}(z)| dz dy \\
&\leq ||q_{2c}||_\eta \int_{-X_{i-1}}^x e^{-(\alpha^s + \eta) (x - y)} \int_{-\infty}^y e^{-\eta(y-z)}  dz dy \\
&= \frac{ ||q_{2c}||_\eta }{\eta} \int_{-X_{i-1}}^x e^{-(\alpha^s + \eta) (x - y)} dy \\
&= \frac{ ||q_{2c}||_\eta }{\eta} \frac{1- e^{-(\alpha^s + \eta)(x + X_{i-1})}}{\alpha^s + \eta} \\
&\leq \frac{ ||q_{2c}||_\eta }{\eta} \frac{1}{\alpha^s + \eta}
\end{align*}

Putting all this together, we have for our bound (for the negative piece),

\begin{align*}
| e^{-\eta x} L_2(\lambda)(a, b, d)_i^-(x)| \leq e^{\eta X_{i-1}}|a^-_{i-1}| + |b_i^-| + |\lambda|^2 |d_i|  \frac{ ||q_{2c}||_\eta }{\eta} \left( \frac{1}{\alpha^s + \eta} + \frac{1}{\alpha^u - \eta} \right)
\end{align*}

The positive piece is similar. The bound should look like (we can verify this if all this ends up working)

\begin{align*}
| e^{\eta x} L_2(\lambda)(a, b, d)_i^+(x)| \leq e^{\eta X_{i}}|a_i^+| + |b_i^+| + |\lambda|^2 |d_i|  \frac{ ||q_{2c}||_\eta }{\eta} \left( \frac{1}{\alpha^u + \eta} + \frac{1}{\alpha^s - \eta} \right)
\end{align*}

Thus, putting all this together, we have the bound

\[
||L_2(\lambda)(a, b, d)(x)||_\eta \leq e^{\eta X}|a| + |b| + C |\lambda|^2 |d|
\]

where

\[
C = \frac{ ||q_{2c}||_\eta }{\eta} \left( \frac{1}{\alpha^s + \eta} + \frac{1}{\alpha^u - \eta} + \frac{1}{\alpha^u + \eta} + \frac{1}{\alpha^s - \eta} \right)
\]

and $X$ is the maximum of the join points $X_i$ which are finite.

\end{proof}

\end{lemma}

We can combine these two lemma to solve for $W_^\pm$.

\begin{lemma}
There exists a bounded linear operator $W_1: V_\lambda \times V_a \times V_b \times V_d \rightarrow V_w^\eta$ such that 
\[
w = W_1(\lambda)(a,b,d)
\]
This operator is analytic in $\lambda$ and linear in $(a, b, d)$. The operator $W_1$ satisfies the bound

\[
\]

\begin{proof}
Define the linear operators $L_1$ and $L_2$ as in the previous two lemmas. Then we can rewrite the fixed point equation as
\[
(I - L_1(\lambda))W = L_2(\lambda)(a,b,d)
\]
For $L_1$ we have the estimate
\[
||L_1(\lambda)W||_\eta \leq C \delta ||W||_\eta
\]
Choose $\delta$ such that $\delta < 1/C$. Then we have
\[
||L_1(\lambda)W|| < 1
\]
in which case the operator $(I - L_1(\lambda))$ is invertible. The inverse $(I - L_1(\lambda))^{-1}$ is analytic in $\lambda$ and has operator norm 
\[
||(I - L_1(\lambda))^{-1}|| \leq \frac{1}{1 - ||L_1||}
\]
We can then write $W$ as
\[
W = W_1(\lambda)(a,b,d) = (I - L_1(\lambda))^{-1} L_2(\lambda)(a,b,d)
\]
which depends linearly on $(a,b,d)$ and analytically on $\lambda$. Since the operator norm of $L_1$ is bounded by a constant (independent of $X$), we have the bound

\[
||W_1(\lambda)(a,b,d)||_\eta \leq C ( e^{\eta X}|a| + |b| + |\lambda|^2 |d|)
\]

\end{proof}

\end{lemma}


The rest of the inversion procedure should proceed exactly as in Sandstede (1998). \\

The analogue of Lemma 3.4 solves the jump equation at the center point, i.e.

\[
W_1^+(L) - W_2^-(-L) = D_1 d = d_2 [ Q'(-L) + R_2'(-L)] - d_1 [ Q'(L) + R_1'(L) ]
\]

Since the only substantial change we made in Lemma 3.3 was the introduction of an exponential weight, Lemma 3.3 should hold with the norm of $W_2(\lambda)(b,d)$ replaced by the exponentially weighted norm. The bound might be slightly difference since there is no $\lambda$ dependence in $L_2$, but we can figure that out later if needed.\\

The analogue of Lemma 3.5 gets $a, b, W$ in terms of $d$. This should again hold with the appropriate norm replaced by the exponentially weighted norm.\\

Thus we should be (mostly) all set to use these estimates.


\end{document}