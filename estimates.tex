% \documentclass{book}

\documentclass[12pt]{article}
\usepackage[pdfborder={0 0 0.5 [3 2]}]{hyperref}%
\usepackage[left=1in,right=1in,top=1in,bottom=1in]{geometry}%
\usepackage[shortalphabetic]{amsrefs}%
\usepackage{amsmath}
\usepackage{enumerate}
\usepackage{enumitem}
\usepackage{amssymb}                
\usepackage{amsmath}                
\usepackage{amsfonts}
\usepackage{amsthm}
\usepackage{bbm}
\usepackage[table,xcdraw]{xcolor}
\usepackage{tikz}
\usepackage{float}
\usepackage{booktabs}
\usepackage{svg}
\usepackage{mathtools}
\usepackage{cool}
\usepackage{url}
\usepackage{graphicx,epsfig}
\usepackage{makecell}
\usepackage{array}

\def\noi{\noindent}
\def\T{{\mathbb T}}
\def\R{{\mathbb R}}
\def\N{{\mathbb N}}
\def\C{{\mathbb C}}
\def\Z{{\mathbb Z}}
\def\P{{\mathbb P}}
\def\E{{\mathbb E}}
\def\Q{\mathbb{Q}}
\def\ind{{\mathbb I}}

\newtheorem{lemma}{Lemma}
\newtheorem{definition}{Definition}

\begin{document}

\subsection*{Estimates}
Here are some estimates like in Lemma 3.1 in Sandstede (1998)

\begin{lemma}We have the estimates
\begin{align*}
|G_i(x)| &\leq C|R_i(x)| \leq C \sup_{|x| \geq L} |Q(x)| \\
| B Q_2(x) - B Q(x) | & \leq C |R_i(x)| \leq C \sup_{|x| \geq L} |Q(x)| \\
D_1 d &= (Q'(L) + Q'(-L))(d_2 - d_1) +\mathcal{O}\left( e^{-\alpha L} |d| \sup_{|x| \geq L} |Q(x)| \right)\\
|A(Q_2(x))| &\leq C \textrm{ for all }x
\end{align*}
where $\alpha > 0$ is defined as on pages 432 and 434 of Sandstede (1998).
\begin{proof}
The first estimate is the same as in Sandstede (1998) with $Q$ replacing $Q^+$ and $Q^-$, and follows from the smoothness of $A$ together with (2.6)(i) in Sandstede (1998). The second estimate follows from (2.6)(i) in Sandstede (1998) and the expansion of $Q^2$ as $Q + R_i$. The third estimate is as in Lemma 3.1 of Sandstede (1998). The fourth estimate follows since the double pulse $q_2(x)$ is bounded and the rest of the matrix $A$ is constant.
\end{proof}
\end{lemma}

\subsection*{Reduction}
This corresponds to section 3.2 in Sandstede (1998). The integrated eigenvalue problem we are dealing with is

\begin{equation}\label{inteigQ}
(W_i^\pm)' = A(Q)W_i^\pm - \lambda^2 d_i KBQ_{c} + \lambda K_i^\pm BW_i^\pm + G_i^\pm W_i^\pm - \lambda^2 d_i K_i^\pm B (R_i^\pm)_c
\end{equation}

For the purposes of doing estimates, we can simplify this a bit by recombining $Q_c$ and $(R_i^\pm)_c$ to get the double pulse version (and integrating from $-\infty$ since it is no longer piecewise).

\[
Q_c + (R_i^\pm)_c = Q_{2c}
\]

Thus have the piecewise system

\begin{equation}\label{inteigQ2}
(W_i^\pm)' = A(Q)W_i^\pm + G_i^\pm W_i^\pm - \lambda^2 d_i K B Q_{2c} + \lambda K_i^\pm BW_i^\pm 
\end{equation}

subject to conditions

\begin{align*}
W_i^-(0) &= W_i^+(0) \\
W_1^+(L) - W_2^-(-L) &= D_1 d \\
W_i^\pm(x) &\in \C \psi(0) \oplus Y^+ \oplus Y^- \\
W_i^+(0) - W_i^-(0) &\in \C \psi(0) 
\end{align*}

where

\begin{align*}
G_i^\pm(x) &= A(Q_2) - A(Q)) = A(Q + R_i^\pm) - A(Q) \\
D_1 d &= d_2 [ Q_2'(-L) + Q_{2c}(-L)] - d_1 [ Q_2'(L) + Q_{2c}(L)]
\end{align*}

where $D_1$ is given by the row matrix 

\[
D_1 = \begin{pmatrix} -(Q_2'(L) + Q_{2c}(L)) & Q_2'(-L) + Q_{2c}(-L) \end{pmatrix}
\]

We can get an improved estimate on $D_1$ in the following way. First we write

\begin{align*}
D_1 d &= d_2 [ Q_2'(-L) + Q_{2c}(-L)] - d_1 [ Q_2'(L) + Q_{2c}(L)] \\
&= d_2 [ Q'(-L) + Q_c(-L) + (R_2^-)'(-L) + (R_2^-)_c(-L)] - d_1 [ Q'(L) + Q_c(L) + (R_1^+)'(-L) + (R_1^+)_c(-L)] \\
\end{align*}

which get everything in terms of the single pulse. Using estimates similar to Sandstede (1998) for the remainder term and its derivaties, we have the bound (WHICH WE SHOULD VERIFY)

\begin{align*}
D_1 d &= d_2 [ Q'(-L) + Q_c(-L) ] - d_1 [ Q'(L) + Q_c(L) ] + \mathcal{O}(e^{-\alpha L}|d| \sup_{|x| \geq L} |Q(x)|)\\
\end{align*}

Next, suppose we have the bounds
\begin{align*}
|G_i^\pm(x)| \leq \delta \\
% |Q_2(x) - Q(x) | \leq \delta \\
|\lambda| \leq \delta \\
\end{align*}

To deal with the integration operator which will be involved, we will need to work in exponentially weighted spaces, since we have an extra integral involved. Given an exponential weight $\eta > 0$, we define the following exponentially weighted spaces.

\begin{align*}
C^0_\eta[-a, 0] &= \{ f \in C^0[-a, 0] : \sup_{x \in [-a, 0]} |e^{-\eta x} f(x) | < \infty \} && a \geq 0 \\
C^0_\eta[0, a] &= \{ f \in C^0[0, a] : \sup_{x \in [0, a]} |e^{\eta x} f(x) | < \infty \} && a \geq 0 
\end{align*}

Since we want our functions in this space to decay exponentially at $\pm \infty$, we use the negative weight for $x \leq 0$ and the positive weight for $x \geq 0$. The corresponding norms are 

\begin{align*}
|| f ||_\eta &= \sup_{x \in [-a, 0]} |e^{-\eta x} f(x) | && f \in C^0_\eta[-a, 0] \\
|| f ||_\eta &= \sup_{x \in [0, a]} |e^{\eta x} f(x) | && f \in C^0_\eta[0, a] \\
\end{align*}

It is clear by our definition of the exponentially weighted spaces that $C_0^\eta \subset C_0$. Using the supremum norm on $C_0$, we have the following relationship between the norm on $C_0$ and the norm on $C_0^\eta$. For $f \in C_0[0, a]$,

\begin{align*}
|| f || &= \sup_{x \in [0,a]} | f(x) | \\
&\leq \sup_{x \in [0,a]} | e^{\eta x} f(x) | = ||f||_\eta
\end{align*}

since we have chosen $\eta > 0$ for funcions defined on $[0, a]$, thus $e^{\eta x} \geq 1$, so we always have $|f(x)| \leq  |e^{\eta x} f(x)|$. The same hold true for functions in $C_0[-a, 0]$. Thus, as long as $f \in C_0^\eta$, we always have

\begin{align*}
||f|| \leq ||f||_\eta && f \in C_0^\eta
\end{align*}

If $S: C_0^\eta \rightarrow C_0^\eta$ is a bounded linear operator with operator norm $||S||_\eta$, then $S: C_0^\eta \rightarrow C_0$ is a also a bounded linear operator with operator norm $||S||$, and $||S|| \leq ||S||_\eta$. For $f \in C_0^\eta[0,a]$,

\begin{align*}
||S f|| &= \sup_{x \in [0,a]} | (S f)(x) | \\
& \leq \sup_{x \in [0,a]} | e^{\eta x} (S f)(x) | \\
&= || S f ||_\eta \leq ||S||_\eta ||f||_\eta
\end{align*}

thus $||S|| \leq ||S||_\eta$.\\
 
We define the same spaces as in (3.12) in Sandstede (1998), except we use the weighted spaces for the space $V_w$. To indicate that we have used the exponential weight $\eta$, we will call that space $V_w^\eta$. For the norms on these direct sums, we use the max of the norms on the individual spaces, since each is a finite sum.\\

Following (3.14) in Sandstede (1998) we write our ODE system as the fixed point equation

\begin{align*}
W_i^-(x) &= \Phi^s_-(x, -X_{i-1})a^-_{i-1} + \Phi^u_-(x, 0)b_i^- \\
&+ \int_0^x \Phi^u_-(x, y)[G_i^-(y) W_i^-(y) + \lambda (K_i^- B W_i^-)(y) - \lambda^2 d_i B Q_{2c}(y) ] dy \\
&+ \int_{-X_{i-1}}^x \Phi^s_-(x, y)[G_i^-(y) W_i^-(y) + \lambda (K_i^-B W_i^-)(y) - \lambda^2 d_i B Q_{2c}(y) ] dy \\
W_i^+(x) &= \Phi^u_+(x, X_i)a^+_{i} + \Phi^s_+(x, 0)b_i^+ \\
&+ \int_0^x \Phi^s_+(x, y)[G_i^+(y) W_i^+(y) + \lambda (K_i^+ B W_i^+)(y) - \lambda^2 d_i B Q_{2c}(y) ] dy \\
&+ \int_{X_{i}}^x \Phi^u_+(x, y)[G_i^+(y) W_i^+(y) + \lambda (K_i^+ B W_i^+)(y) - \lambda^2 d_i B Q_{2c}(y) ] dy
\end{align*}

The idea is that we integrate L to R along the stable projection and from R to L along the unstable projection. The $X_i^\pm$ are the appropriate interval endpoints, which for this case are $(X_0, X_1, X_2) = (-\infty, L, \infty)$.\\

From here, we will (attempt to) proceed as in section 3 of Sandstede (1998) via a series of lemmas. In the first lemma, we consider a linear operator consisting of the terms in the fixed point equation which involve $W_i^\pm$. 

\begin{lemma}

Let $L_1(\lambda): V_w^\eta \rightarrow V_w^\eta$ be the linear operator defined piecewise by

\begin{align*}
(L_1(\lambda)W)_i^-(x) = \int_0^x &\Phi^u_-(x, y)[G_i^-(y) W_i^-(y) + \lambda (K_i^- B W_i^-)(y) ] dy \\
&+ \int_{-X_{i-1}}^x \Phi^s_-(x, y)[G_i^-(y) W_i^-(y) + \lambda (K_i^-B W_i^-)(y) ] dy \\
(L_1(\lambda)W)_i^+(x) = \int_0^x &\Phi^s_+(x, y)[G_i^+(y) W_i^+(y) + \lambda (K_i^+ B W_i^+)(y)] dy \\
&+ \int_{X_{i}}^x \Phi^u_+(x, y)[G_i^+(y) W_i^+(y) + \lambda (K_i^+ B W_i^+)(y) ] dy
\end{align*}

where $G_i^\pm$ is uniformly bounded with norm $|G|$. Choose $\eta$ such that $0 < \eta < \alpha$, where $\alpha = \min(\alpha^s, \alpha^u)$. Then $L_1(\lambda): V_w^\eta \rightarrow V_w^\eta$ is a bounded linear operator, and we have bound

\begin{equation}
||L_1(\lambda)W||_\eta \leq C\left(|G| +|\lambda|\right)||W||_\eta
\end{equation}

Choosing the codomain to be the nonweighted space, $L_1(\lambda): V_w^\eta \rightarrow V_w$ is a also a bounded linear operator, and we have the same bound

\begin{equation}
||L_1(\lambda)W|| \leq C\left(|G| +|\lambda|\right)||W||_\eta
\end{equation}

These bounds do not depend on the interval endpoints $X_i^\pm$.\\

\begin{proof}
First, we look at the negative piece. Note that when taking the absolute value we have to integrate in the positive direction. For all of this, we use the estimates in Lemma 3.2 of Sandstede (1998) to bound $\Phi^u_\pm$ and $\Phi^s_\pm$.

\begin{align*}
|e^{-\eta x} & (L_1(\lambda)W_i^-)(x) | \leq  e^{-\eta x} \int_x^0 |\Phi^u_-(x, y)|[|G_i^-(y)||W_i^-(y)| + |\lambda||(K_i^- B W_i^-)(y)| ] dy \\
&+ e^{-\eta x} \int_{-X_{i-1}}^x |\Phi^s_-(x, y)|[|G_i^-(y)||W_i^-(y)| + |\lambda||(K_i^- B W_i^-)(y)| ] dy \\
&\leq C|G| \int_x^0 e^{\alpha^u (x-y)}e^{-\eta(x-y)}|e^{-\eta y} W_i^-(y)| dy 
+ C|\lambda|\int_x^0 e^{\alpha^u (x-y)}e^{-\eta(x-y)}|e^{-\eta y} (K_i^- B W_i^-)(y)| dy \\
&+ C|G| \int_{-X_{i-1}}^x e^{-\alpha^s (x-y)}e^{-\eta(x-y)}|e^{-\eta y} W_i^-(y)| dy 
+ C|\lambda|\int_{-X_{i-1}}^x e^{-\alpha^s (x-y)}e^{-\eta(x-y)}|e^{-\eta y} (K_i^- B W_i^-)(y)| dy  \\ 
\end{align*}

The two terms not involving the integration operator are similar to those in Sandstede (1998), so let's take care of those first.

\begin{align*}
\int_x^0 e^{\alpha^u (x-y)}e^{-\eta(x-y)}|e^{-\eta y} W_i^-(y)| dy &= \int_x^0 e^{(\alpha^u - \eta) (x-y)}|e^{-\eta y} W_i^-(y)| dy \\
&\leq ||W_i^-||_\eta \frac{1 - e^{(\alpha^u - \eta)x}}{\alpha^u - \eta}
\end{align*}

We need the RHS of this to be positive, which is true when $\eta < \alpha^u$. Provided this is the case, we have the bound

\begin{align*}
\int_x^0 e^{\alpha^u (x-y)}e^{-\eta(x-y)}|e^{-\eta y} W_i^-(y)| dy &\leq ||W_i^-||_\eta \frac{1}{\alpha^u - \eta}
\end{align*}

For the other one:

\begin{align*}
\int_{-X_{i-1}}^x e^{-\alpha^s (x-y)}e^{-\eta(x-y)}|e^{-\eta y} W_i^-(y)| dy &= \int_{-X_{i-1}}^x e^{(-\alpha^s - \eta) (x-y)}|e^{-\eta y} W_i^-(y)| dy \\
&\leq ||W_i^-||_\eta \int_{-\infty}^x e^{(-\alpha^s - \eta) (x-y)} dy = ||W_i^-||_\eta \frac{1}{\alpha^s + \eta}
\end{align*}

So this has a nice bound independent of the $X_i$. The RHS is always positive, so this doesn't give us any extra conditions on $\eta$. \\

So far, all of this would have worked without the exponential weight. Where we need it is in the terms involving the integration operator. Let's do those terms now. Recall that because of the operator $B$ we are only integrating the scalar function $w_i^\pm$

\begin{align*}
\int_x^0 e^{\alpha^u (x-y)}e^{-\eta(x-y)}|e^{-\eta y} (K_i^- B W_i^-)(y)| dy &\leq \int_x^0 e^{(\alpha^u - \eta)(x-y)}e^{-\eta y} \int_{a_i^-}^y |w_i^-(u)| du dy \\
&= \int_x^0 e^{(\alpha^u - \eta)(x-y)}e^{-\eta y} \int_{a_i^-}^y e^{\eta u} |e^{-\eta u} w_i^-(u)| du dy \\
&\leq ||W_i^-||_\eta \int_x^0 e^{(\alpha^u - \eta)(x-y)}e^{-\eta y} \int_{-\infty}^y e^{\eta u} du dy \\
&= \frac{||W_i^-||_\eta}{\eta} \int_x^0 e^{(\alpha^u - \eta)(x-y)}e^{-\eta y} e^{\eta y} dy \\
&= \frac{||W_i^-||_\eta}{\eta} \frac{1 - e^{(\alpha^u - \eta)x}}{\alpha^u - \eta} 
\end{align*}

We need the RHS of this to be positive, which is true when $\eta < \alpha^u$, in which case we have the bound

\[
\int_x^0 e^{\alpha^u (x-y)}e^{-\eta(x-y)}|e^{-\eta y} (K_i^- B W_i^-)(y)| dy \leq 
\frac{||W_i^-||_\eta}{\eta} \frac{1}{\alpha^u - \eta} 
\]

We had that condition from before, so that is good.\\

Here is the other integral term. 

\begin{align*}
\int_{-X_{i-1}}^x e^{-\alpha^s (x-y)}e^{-\eta(x-y)}|e^{-\eta y} (K_i^- B W_i^-)(y)| dy &\leq \int_{-X_{i-1}}^x e^{(-\alpha^s - \eta)(x-y)}e^{-\eta y} \int_{a_i^-}^y |w_i^-(u)| du dy \\
&= \int_{-X_{i-1}}^x e^{(-\alpha^s - \eta)(x-y)}e^{-\eta y} \int_{a_i^-}^y e^{\eta u} |e^{-\eta u} w_i^-(u)| du dy \\
&\leq ||W_i^-||_\eta \int_{-\infty}^x e^{(-\alpha^s - \eta)(x-y)}e^{-\eta y} \int_{-\infty}^y e^{\eta u} du dy \\
&= \frac{||W_i^-||_\eta}{\eta} \int_{-\infty}^x e^{(-\alpha^s - \eta)(x-y)}e^{-\eta y} e^{\eta y} dy \\
&= \frac{||W_i^-||_\eta}{\eta} \frac{1}{\alpha^s + \eta}
\end{align*}

This also has a nice bound independent of the $X_i$. The RHS is always positive, so this doesn't give us any extra conditions on $\eta$.\\

So far we have the condition $\eta < \alpha^u$. We expect to also have the condition $\eta < \alpha^s$. We should get that from using the ``+'' equations.\\

The ``+'' equations are similar to the ``-'' equations. We could probably just state the result, but we will show it here for completeness. Recall here that since $x \geq 0$, we have to multiply by $e^{\eta x}$ to get the weighted norm. We also must integrate from L to R when taking the absolute value.\\

\begin{align*}
|e^{\eta x} & (L_1(\lambda)W)_i^+)(x) | \leq e^{\eta x} \int_0^x |\Phi^s_+(x, y)|[|G_i^+(y)||W_i^+(y)| + |\lambda|(K_i^+ B W_i^+)(y)| ] dy \\
&+ e^{\eta x} \int_x^{X_i} |\Phi^u_+(x, y)|[|G_i^+(y)||W_i^+(y)| + |\lambda||(K_i^+ B W_i^+)(y)| ] dy \\
&\leq C|G| \int_0^x e^{-\alpha^s (x-y)}e^{\eta(x-y)}|e^{\eta y} W_i^+(y)| dy 
+ C|\lambda|\int_0^x e^{-\alpha^s (x-y)}e^{\eta(x-y)}|e^{\eta y} (K_i^+ B W_i^+)(y)| dy \\
&+ C|G| \int_x^{X_i} e^{\alpha^u (x-y)}e^{\eta(x-y)}|e^{\eta y} W_i^+(y)| dy 
+ C|\lambda|\int_x^{X_i} e^{\alpha^u (x-y)}e^{\eta(x-y)}|e^{\eta y} (K_i^+ B W_i^+)(y)| dy \\ 
\end{align*}

We do the same thing we did above.

\begin{align*}
\int_0^x e^{-\alpha^s (x-y)}e^{\eta(x-y)}|e^{\eta y} W_i^+(y)| dy &= \int_0^x e^{(-\alpha^s + \eta) (x-y)}|e^{\eta y} W_i^-(y)| dy \\
&\leq ||W_i^+||_\eta \frac{1 - e^{-(\alpha^s - \eta)x} }{\alpha^s - \eta}
\end{align*}

We need the RHS of this to be positive, which is true when $\eta < \alpha^s$. (This is the condition we were expecting). In that case, we have the bound

\begin{align*}
\int_0^x e^{-\alpha^s (x-y)}e^{\eta(x-y)}|e^{\eta y} W_i^+(y)| dy &\leq ||W_i^+||_\eta \frac{1}{\alpha^s - \eta}
\end{align*}

For the other one:

\begin{align*}
\int_x^{X_i} e^{\alpha^u (x-y)}e^{\eta(x-y)}|e^{\eta y} W_i^+(y)| dy &= \int_x^{X_i} e^{(\alpha^u + \eta) (x-y)}|e^{\eta y} W_i^+(y)| dy \\
&\leq ||W_i^+||_\eta \int_x^{\infty} e^{(\alpha^u + \eta) (x-y)} dy = ||W_i^+||_\eta \frac{1}{\alpha^u + \eta}
\end{align*}

Again, this bound is independent of the $X_i$. The RHS is always positive, so this doesn't give us any extra conditions on $\eta$. \\

Now we do the integral terms. Since our integration operators $K_i^+$ integrate from R to L, when we take the absolute value, we need to change our limits so we are integrating from L to R.

\begin{align*}
\int_0^x e^{-\alpha^s (x-y)}e^{\eta(x-y)}|e^{\eta y} (K_i^+ B W_i^+)(y)| dy &\leq \int_0^x e^{(-\alpha^s + \eta)(x-y)}e^{\eta y} \int_y^{a_i^+} |w_i^+(u)| du dy \\
&= \int_0^x e^{(-\alpha^s + \eta)(x-y)}e^{\eta y} \int_y^{a_i^+} e^{-\eta u} |e^{\eta u} w_i^+(u)| du dy \\
&\leq ||W_i^+||_\eta \int_0^x e^{(-\alpha^s + \eta)(x-y)}e^{\eta y} \int_y^\infty e^{-\eta u} du dy \\
&= \frac{||W_i^+||_\eta}{\eta} \int_0^x e^{(-\alpha^s + \eta)(x-y)}e^{\eta y} e^{-\eta y} dy \\
&= \frac{||W_i^+||_\eta}{\eta} \frac{1 - e^{-(\alpha^s - \eta)x}}{\alpha^s - \eta} 
\end{align*}

We need the RHS of this to be positive, which is true when $\eta < \alpha^s$, in which case we have the bound

\[  
\int_0^x e^{-\alpha^s (x-y)}e^{\eta(x-y)}|e^{\eta y} (K_i^+ B W_i^+)(y)| dy \leq
\frac{||W_i^+||_\eta}{\eta} \frac{1}{\alpha^s - \eta} 
\]

We had that condition from before, so that is good.\\

For the final integral,

\begin{align*}
\int_x^{X_i} e^{\alpha^u (x-y)}e^{\eta(x-y)}|e^{\eta y} (K_i^+ B W_i^+)(y)| dy &\leq \int_x^{X_i} e^{(\alpha^u + \eta)(x-y)}e^{\eta y} \int_y^{a_i^+} |w_i^+(u)| du dy \\
&= \int_x^{X_i} e^{(\alpha^u + \eta)(x-y)}e^{\eta y} \int_y^{a_i^+} e^{-\eta u} |e^{\eta u} w_i^-(u)| du dy \\
&\leq ||W_i^+||_\eta \int_x^\infty e^{(\alpha^u + \eta)(x-y)}e^{\eta y} \int_y^\infty e^{-\eta u} du dy \\
&= \frac{||W_i^+||_\eta}{\eta} \int_x^\infty e^{(\alpha^u + \eta)(x-y)}e^{\eta y} e^{-\eta y} dy \\
&= \frac{||W_i^+||_\eta}{\eta} \frac{1}{\alpha^u + \eta}
\end{align*}

Putting this all together, we conclude the following. The operator $L_1$ defined above is a bounded linear operator from $V_w^\eta$, the exponentially weighted version of $V_w$, to itself. Since the norm on the product space $V_w^\eta$ is the maximum of the individual (weighted) norms making up the product, we have the bound:

\begin{equation}
	||L_1(\lambda)W||_\eta \leq C\left(|G| + \frac{|\lambda|}{\eta}\right)\left(\frac{1}{\alpha^s + \eta} + \frac{1}{\alpha^u + \eta} + \frac{1}{\alpha^s - \eta} + \frac{1}{\alpha^u - \eta}\right)||W||_\eta
\end{equation}

provided that $\eta \leq \alpha^2, \alpha^u$, where $W$ is the element of $V_w^\eta$ formed by the pieces $W_i^\pm$. Incorporating many of the constants above into the constant $C$, this becomes

\begin{equation}
||L_1(\lambda)W||_\eta \leq C\left(|G| +|\lambda|\right)||W||_\eta
\end{equation}

Thus $L_1(\lambda): V_w^\eta \rightarrow V_w^\eta$ is a bounded linear operator. 

If we have $|\lambda| \leq \delta$ and $|G| \leq \delta$, this bound becomes

\begin{equation}
||L_1(\lambda)W||_\eta \leq C \delta ||W||_\eta
\end{equation}

By the remarks above, $L_1(\lambda): V_w^\eta \rightarrow V_w$ is a also a bounded linear operator, and

\begin{equation}
||L_1(\lambda)W|| \leq C\left(|G| +|\lambda|\right)||W||_\eta
\end{equation}

and if $|\lambda| \leq \delta$ and $|G| \leq \delta$, this bound becomes

\begin{equation}
||L_1(\lambda)W|| \leq C \delta ||W||_\eta
\end{equation}

\end{proof}
\end{lemma}

In the next lemma, we consider a linear operator consisting of the remaining terms in the fixed point equation, i.e. those which do not involve $W_i^\pm$.

\begin{lemma}\label{L2}

Let $L_2(\lambda): V_a \times V_b \times V_d \rightarrow V_w^\eta$ be the linear operator defined piecewise by

\begin{align*}
L_2(\lambda)(a, b, d)_i^-(x) &= \Phi^s_-(x, -X_{i-1})a^-_{i-1} + \Phi^u_-(x, 0)b_i^- \\
&- \lambda^2 d_i \left( \int_0^x \Phi^u_-(x, y)(KBQ_{2c})(y) dy  + \int_{-X_{i-1}}^x \Phi^s_-(x, y)(KBQ_{2c})(y) dy \right)\\
L_2(\lambda)(a, b, d)_i^+(x) &= \Phi^u_+(x, X_i)a^+_{i} + \Phi^s_+(x, 0)b_i^+ \\
&- \lambda^2 d_i \left( \int_0^x \Phi^s_+(x, y)(KBQ_{2c})(y) dy + \int_{X_{i}}^x \Phi^u_+(x, y)(KBQ_{2c})(y) dy \right)
\end{align*}

Then $L_2$ is a bounded linear operator with the following bounds

\begin{align*}
||L_2(\lambda)(a, b, d)||_\eta &\leq C(e^{\eta L}|a| + |b| + |\lambda|^2 |d|) \\
\end{align*}

If we consider $L_2(\lambda): V_a \times V_b \times V_d \rightarrow V_w$, i.e. take the codomain to be the unweighted space, we have the bound 

\begin{align*}
||L_2(\lambda)(a, b, d)|| &\leq C(|a| + |b| + |\lambda|^2 |d|)
\end{align*}

The bound in the exponentially weighted space depends on the domain parameter $L$, whereas the bound in the unweighted space does not.

\begin{proof}

As in the previous lemma, we will take $0 < \eta < \alpha^s, \alpha^u$. For the ``minus'' piece, recall that $a_0 = 0$ (so the term involving $a^-_{i-1}$ is zero when $i = 1$ and that $-X_{i-1} \leq x \leq 0$. When we take the absolute value, we need to integrate in the positive direction.

\begin{align*}
| e^{-\eta x} L_2(\lambda)(a, b, d)_i^-(x)| &\leq e^{-\eta x} |\Phi^s_-(x, -X_{i-1})a^-_{i-1}| + e^{-\eta x}|\Phi^u_-(x, 0)b_i^-| \\
&+ |\lambda|^2 |d_i| e^{-\eta x} \left( \int_x^0 |\Phi^u_-(x, y)||(KBQ_{2c})|(y) dy  + \int_{-X_{i-1}}^x |\Phi^s_-(x, y)||(KBQ_{2c})(y)| dy \right) \\
&\leq C e^{-\eta x} e^{-\alpha^s [x - (-X_{i-1})]}|a^-_{i-1}| + C e^{-\eta x}e^{\alpha^u x}|b_i^-| \\
&+ C|\lambda|^2 |d_i| e^{-\eta x} \left( \int_x^0 e^{\alpha^u (x - y)}\int_{-\infty}^y |q_{2c}(z)| dz dy  + \int_{-X_{i-1}}^x e^{-\alpha^s (x - y)}\int_{-\infty}^y |q_{2c}(z)| dz dy\right) \\
&= C e^{-(\alpha^s + \eta) [x - (-X_{i-1})]} e^{\eta X_{i-1}}|a^-_{i-1}| + C e^{(\alpha^u - \eta) x}|b_i^-| \\
&+ C|\lambda|^2 |d_i| e^{-\eta x} \left( \int_x^0 e^{\alpha^u (x - y)}\int_{-\infty}^y |q_{2c}(z)| dz dy  + \int_{-X_{i-1}}^x e^{-\alpha^s (x - y)}\int_{-\infty}^y |q_{2c}(z)| dz dy\right) \\
&= C e^{\eta X_{i-1}}|a^-_{i-1}| + C |b_i^-| \\
&+ C|\lambda|^2 |d_i| e^{-\eta x} \left( \int_x^0 e^{\alpha^u (x - y)}\int_{-\infty}^y |q_{2c}(z)| dz dy  + \int_{-X_{i-1}}^x e^{-\alpha^s (x - y)}\int_{-\infty}^y |q_{2c}(z)| dz dy\right) \\
\end{align*}

For the first integral, we have

\begin{align*}
e^{-\eta x} \int_x^0 e^{\alpha^u (x - y)}\int_{-\infty}^y |q_{2c}(z)| dz dy &=
\int_x^0 e^{-\eta (x-y)} e^{\alpha^u (x - y)}e^{-\eta y}\int_{-\infty}^y |q_{2c}(z)| dz dy \\
&= \int_x^0 e^{(\alpha^u - \eta) (x - y)}\int_{-\infty}^y e^{-\eta(y-z)} |e^{-\eta z} q_{2c}(z)| dz dy \\
&\leq ||q_{2c}||_\eta \int_x^0 e^{(\alpha^u - \eta) (x - y)}\int_{-\infty}^y e^{-\eta(y-z)}  dz dy \\
&= \frac{ ||q_{2c}||_\eta }{\eta} \int_x^0 e^{(\alpha^u - \eta) (x - y)} dy \\
&= \frac{ ||q_{2c}||_\eta }{\eta} \frac{1- e^{(\alpha^u - \eta)x}}{\alpha^u - \eta} \\
&\leq \frac{ ||q_{2c}||_\eta }{\eta} \frac{1}{\alpha^u - \eta}
\end{align*}

where in the last line we used the fact that $x \leq 0$. We also used the fact that $q_c$ is bounded in the exponentially weighted space with weight $\eta$, thus for the double pulse $q_{2c}$ is as well.\\

For the second integral,

\begin{align*}
e^{-\eta x} \int_{-X_{i-1}}^x e^{-\alpha^s (x - y)}\int_{-\infty}^y |q_{2c}(z)| dz dy &=
\int_{-X_{i-1}}^x e^{-\eta (x-y)} e^{-\alpha^s (x - y)}e^{-\eta y}\int_{-\infty}^y |q_{2c}(z)| dz dy \\
&= \int_{-X_{i-1}}^x e^{-(\alpha^s + \eta) (x - y)}\int_{-\infty}^y e^{-\eta(y-z)} |e^{-\eta z} q_{2c}(z)| dz dy \\
&\leq ||q_{2c}||_\eta \int_{-X_{i-1}}^x e^{-(\alpha^s + \eta) (x - y)} \int_{-\infty}^y e^{-\eta(y-z)}  dz dy \\
&= \frac{ ||q_{2c}||_\eta }{\eta} \int_{-X_{i-1}}^x e^{-(\alpha^s + \eta) (x - y)} dy \\
&= \frac{ ||q_{2c}||_\eta }{\eta} \frac{1- e^{-(\alpha^s + \eta)(x + X_{i-1})}}{\alpha^s + \eta} \\
&\leq \frac{ ||q_{2c}||_\eta }{\eta} \frac{1}{\alpha^s + \eta}
\end{align*}

Putting all this together, we have for our bound (for the negative piece),

\begin{align*}
| e^{-\eta x} L_2(\lambda)(a, b, d)_i^-(x)| \leq C \left[ e^{\eta X_{i-1}}|a^-_{i-1}| + |b_i^-| + |\lambda|^2 |d_i|  \frac{ ||q_{2c}||_\eta }{\eta} \left( \frac{1}{\alpha^s + \eta} + \frac{1}{\alpha^u - \eta} \right)\right]
\end{align*}

The positive piece is similar. The bound should look like (we can verify this if all this ends up working)

\begin{align*}
| e^{\eta x} L_2(\lambda)(a, b, d)_i^+(x)| \leq C \left[ e^{\eta X_{i}}|a_i^+| + |b_i^+| + |\lambda|^2 |d_i|  \frac{ ||q_{2c}||_\eta }{\eta} \left( \frac{1}{\alpha^u + \eta} + \frac{1}{\alpha^s - \eta} \right)\right]
\end{align*}

Thus, putting all this together, we have the bound in the expontially weighted norm

\[
||L_2(\lambda)(a, b, d)||_\eta \leq C(e^{\eta L}|a| + |b| + |\lambda|^2 |d|)
\]

since for the double pulse, the two nonzero join points are $\pm L$.\\

For the bound in the unweighted norm, we do the same thing except we do not multiply by $e^{-\eta x}$. Recall that $x < 0$ for the ``minus'' piece.

\begin{align*}
| L_2(\lambda)(a, b, d)_i^-(x)| &\leq |\Phi^s_-(x, -X_{i-1})a^-_{i-1}| + |\Phi^u_-(x, 0)b_i^-| \\
&+ |\lambda|^2 |d_i| \left( \int_x^0 |\Phi^u_-(x, y)||(KBQ_{2c})|(y) dy  + \int_{-X_{i-1}}^x |\Phi^s_-(x, y)||(KBQ_{2c})(y)| dy \right) \\
&\leq C e^{-\alpha^s [x - (-X_{i-1})]}|a^-_{i-1}| + C e^{\alpha^u x}|b_i^-| \\
&+ C|\lambda|^2 |d_i| e^{-\eta x} \left( \int_x^0 e^{\alpha^u (x - y)}\int_{-\infty}^y |q_{2c}(z)| dz dy  + \int_{-X_{i-1}}^x e^{-\alpha^s (x - y)}\int_{-\infty}^y |q_{2c}(z)| dz dy\right) \\
\end{align*}

where in the second inequality we multiplied the integral by $e^{-\eta x} \geq 1$ for $x \leq 0$. The integral terms then can be bounded exactly as above, so we get the bound

\begin{align*}
| L_2(\lambda)(a, b, d)_i^-(x)| \leq C \left[ |a^-_{i-1}| + |b_i^-| + |\lambda|^2 |d_i|  \frac{ ||q_{2c}||_\eta }{\eta} \left( \frac{1}{\alpha^s + \eta} + \frac{1}{\alpha^u - \eta} \right)\right]
\end{align*}

The ``positive'' piece is similar. Thus we have the bound 

\[
||L_2(\lambda)(a, b, d)|| \leq C(|a| + |b| + |\lambda|^2 |d|)
\]

\end{proof}

\end{lemma}

This is essentially identical to what we have in (3.20) in Sandstede (1998). We can combine these two lemmas to solve for $W_i^\pm$.

\begin{lemma}\label{W1}
There exists a bounded linear operator $W_1: V_\lambda \times V_a \times V_b \times V_d \rightarrow V_w^\eta$ such that 
\[
w = W_1(\lambda)(a,b,d)
\]
This operator is analytic in $\lambda$ and linear in $(a, b, d)$. The operator $W_1$ satisfies the bound

\[
||W_1(\lambda)(a,b,d)||_\eta \leq C ( e^{\eta L}|a| + |b| + |\lambda|^2 |d|)
\]

\begin{proof}
Define the linear operators $L_1$ and $L_2$ as in the previous two lemmas. Then we can rewrite the fixed point equation as
\[
(I - L_1(\lambda))W = L_2(\lambda)(a,b,d)
\]
For $L_1$ we have the estimate
\[
||L_1(\lambda)W||_\eta \leq C \delta ||W||_\eta
\]
Choose $\delta$ such that $\delta < 1/C$. Then we have
\[
||L_1(\lambda)W||_\eta < ||W||_\eta
\]
in which case the operator $(I - L_1(\lambda))$ is invertible. The inverse $(I - L_1(\lambda))^{-1}$ is analytic in $\lambda$ and has operator norm 
\[
||(I - L_1(\lambda))^{-1}||_\eta \leq \frac{1}{1 - ||L_1||_\eta}
\]
We can then write $W$ as
\[
W = W_1(\lambda)(a,b,d) = (I - L_1(\lambda))^{-1} L_2(\lambda)(a,b,d)
\]
which depends linearly on $(a,b,d)$ and analytically on $\lambda$. Since the operator norm of $L_1$ is bounded by a constant (independent of $X$), we have the bound

\[
||W_1(\lambda)(a,b,d)||_\eta \leq C ( e^{\eta L}|a| + |b| + |\lambda|^2 |d|)
\]

\end{proof}
\end{lemma}

However, we would really like to have a bound on $W_1(\lambda)$ in the unweighted norm. We can do that.


\begin{lemma}\label{W1unweighted}
The operator $W_1$ defined in Lemma \ref{W1} also satisfies the bound in the unweighted space

\[
||W_1(\lambda)(a,b,d)|| \leq C ( |a| + |b| + |\lambda|^2 |d|)
\]

\begin{proof}
Define $W_1$ as in Lemma \ref{W1}. Since $V_w^\eta \subset V_w$, $W_1$ is well-defined with codomain $V_w$. Choosing the same $\delta$ as in Lemma \ref{W1}, we have
\[
||L_1(\lambda)W|| \leq ||L_1(\lambda)W||_\eta < ||W||_\eta
\]
Thus the operator $(I - L_1(\lambda)): V_w^\eta \rightarrow V_w$ is invertible on this domain-codomain pair. The inverse $(I - L_1(\lambda))^{-1}: V_w \rightarrow V_w^\eta$ is analytic in $\lambda$ and has operator norm 
\[
||(I - L_1(\lambda))^{-1}|| \leq \frac{1}{1 - ||L_1||}
\]
Writing $W$ as in Lemma \ref{W1}, we have an operator $W: V_a \times V_b \times V_d \rightarrow V_w^\eta$
\[
W = W_1(\lambda)(a,b,d) = (I - L_1(\lambda))^{-1} L_2(\lambda)(a,b,d)
\]

which has bound

\[
||W_1(\lambda)(a,b,d)||_\eta \leq C ( |a| + |b| + |\lambda|^2 |d| )
\]

since the unweighted norm is bounded by the weighted norm, we conclude 

\[
||W_1(\lambda)(a,b,d)|| \leq C ( |a| + |b| + |\lambda|^2 |d| )
\]

\end{proof}
\end{lemma}

In the next lemma we solve for the center join $W_1^+(L) - W_2^-(-L) = D_1 d$.

\begin{lemma}
There exist operators

\begin{align*}
A_1: V_\lambda \times V_b \times V_d \rightarrow V_a \\
W_2: V_\lambda \times V_b \times V_d \rightarrow V_w^\eta \\
\end{align*}

such that $(a,w) = ( A_1(\lambda)(b,d), W_2(\lambda)(b,d) )$ solves our system. These operators are analytic in $\lambda$, linear in $(b,d)$, and bounds for them are given below.

\begin{proof}
\begin{align*}
D_1 d &= W_1^+(L) - W_2^-(-L) \\
&= \Phi^u_+(L, L)a^+_{1} + \Phi^s_+(L, 0)b_1^+ 
+ \int_0^L \Phi^s_+(L, y)[G_1^+(y) W_1^+(y) + \lambda (K_1^+ B W_1^+)(y) - \lambda^2 d_1 B Q_{2c}(y) ] dy \\  
&- \Phi^s_-(-L, -L)a^-_{1} - \Phi^u_-(-L, 0)b_2^- - \int_0^{-L} \Phi^u_-(-L, y)[G_2^-(y) W_2^-(y) + \lambda (K_2^- B W_2^-)(y) - \lambda^2 d_2 B Q_{2c}(y) ] dy \\
&= P_u^+(L) a^+_{1} - P_s^-(-L) a^-_{1} + \Phi^s_+(L, 0)b_1^+ - \Phi^u_-(-L, 0)b_2^- \\
&+ \int_0^L \Phi^s_+(L, y)[G_1^+(y) W_1^+(y) + \lambda (K_1^+ B W_1^+)(y) - \lambda^2 d_1 B Q_{2c}(y) ] dy \\
&+ \int_{-L}^0 \Phi^u_-(-L, y)[G_2^-(y) W_2^-(y) + \lambda (K_2^- B W_2^-)(y) - \lambda^2 d_2 B Q_{2c}(y) ] dy 
\end{align*}

Where we have the projections from Lemma 3.2 in Sandstede (1998)
\begin{align*}
P^s_\pm(x) &= \Phi^s_\pm(x,x) \\
P^u_\pm(x) &= \Phi^u_\pm(x,x) \\
\end{align*}

Recall that $a_i^- \in E^s$ and $a_i^+ \in E^u$. Writing the projections on $E^s$ and $E^u$ as $P_0^s$ and $P_0^u$, we have $P_0^s a_i^- = a_i^-$ and $P_0^u a_i^+ = a_i^+$. Using these above with $i = 1$, we get 

\begin{align*}
D_1 d &=  a^+_1 - a^-_1 + (P_u^+(L) - P_0^u) a^+_1 - (P_0^s - P_s^-(-L)) a^-_1 + \Phi^s_+(L, 0)b_1^+ - \Phi^u_-(-L, 0)b_2^- \\
&+ \int_0^L \Phi^s_+(L, y)[G_1^+(y) W_1^+(y) + \lambda (K_1^+ B W_1^+)(y) - \lambda^2 d_1 B Q_{2c}(y) ] dy \\
&+ \int_{-L}^0 \Phi^u_-(-L, y)[G_2^-(y) W_2^-(y) + \lambda (K_2^- B W_2^-)(y) - \lambda^2 d_2 B Q_{2c}(y) ] dy 
\end{align*}

which we will write as

\begin{equation}
D_1 d = (a^+_1 - a^-_1) + L_3(\lambda)(a,b,d)
\end{equation}

where 

\begin{align*}
L_3(\lambda)(a,b,d) &= (P_u^+(L) - P_0^u) a^+_1 - (P_0^s - P_s^-(-L)) a^-_1 + \Phi^s_+(L, 0)b_1^+ - \Phi^u_-(-L, 0)b_2^- \\
&+ \int_0^L \Phi^s_+(L, y)[G_1^+(y) W_1^+(y) + \lambda (K_1^+ B W_1^+)(y) - \lambda^2 d_1 B Q_{2c}(y) ] dy \\
&+ \int_{-L}^0 \Phi^u_-(-L, y)[G_2^-(y) W_2^-(y) + \lambda (K_2^- B W_2^-)(y) - \lambda^2 d_2 B Q_{2c}(y) ] dy 
\end{align*}

If we substitute $W = W_1(\lambda)(a,b,d)$ from Lemma \ref{W1}, we see that this is linear in $(a,b,d)$ and analytic in $\lambda$. Now we will show that $L_3(\lambda)$ is bounded. First, let

\[
p(L) = \sup_{x \geq L} (|P_u^+(x) - P_0^u| + |P_0^s - P_s^-(-x)|)
\]

Then for our bound we have

\begin{align*}
|L_3(\lambda)(a,b,d)| &\leq |P_u^+(L) - P_0^u|| a^+_1| + |P_0^s - P_s^-(-L)||a^-_1| + |\Phi^s_+(L, 0)||b_1^+| + |\Phi^u_-(-L, 0)||b_2^-| \\
&+ \int_0^L |\Phi^s_+(L, y)|[|G_1^+(y)||W_1^+(y)| + |\lambda| |(K_1^+ B W_1^+)(y)| + |\lambda|^2 |d_1| B |Q_{2c}(y)| ] dy \\
&+ \int_{-L}^0 \Phi^u_-(-L, y)[|G_2^-(y)||W_2^-(y)| + |\lambda| |(K_2^- B W_2^-)(y)| + |\lambda|^2 |d_2| B |Q_{2c}(y)| ] dy 
\end{align*}

Doing the bounds one at a time, we have

\begin{enumerate}

\item
\begin{align*}
|P_u^+(L) - P_0^u|| a^+_1| + |P_0^s - P_s^-(-L)||a^-_1| \leq p(L)|a|
\end{align*}

\item
\begin{align*}
|\Phi^s_+(L, 0)||b_1^+| + |\Phi^u_-(-L, 0)||b_2^-| &\leq C e^{-\alpha^s L} |b_1^+| + C e^{-\alpha^u L}|b_2^-| \\
&\leq C e^{-\alpha L}|b|
\end{align*}

\item
\begin{align*}
\int_0^L |\Phi^s_+(L, y)|[|G_1^+(y)||W_1^+(y)| dy &\leq C|G| \int_0^L e^{-\alpha^s(L-y)}e^{-\eta y}|e^{\eta y}W_1^+(y)| dy \\
&\leq C|G|||W_1^+||_\eta e^{-\eta L} \int_0^L e^{-\alpha^s(L-y)}e^{\eta L} e^{-\eta y} dy \\
&= C|G| e^{-\eta L}||W_1^+||_\eta \int_0^L e^{-(\alpha^s - \eta)(L-y)} dy \\
&= C|G| e^{-\eta L} ||W_1^+||_\eta \frac{1 - e^{-(\alpha^s - \eta)L}}{\alpha^s - \eta}\\
&\leq C|G| \frac{ e^{-\eta L} }{\alpha^s - \eta} ||W_1^+||_\eta
\end{align*}

Similarly,

\begin{align*}
\int_{-L}^0 |\Phi^u_-(-L, y)|[|G_2^-(y)||W_2^-(y)| dy &\leq C|G| \int_{-L}^0  e^{-\alpha^u(L+y)}e^{\eta y}|e^{-\eta y}W_2^-(y)| dy \\
&\leq C|G| e^{-\eta L}||W_2^-||_\eta \int_0^L e^{-(\alpha^u - \eta)(L+y)} dy \\
&= C|G| e^{-\eta L} ||W_2^-||_\eta \frac{1 - e^{-(\alpha^u - \eta)L}}{\alpha^u - \eta}\\
&\leq C|G| \frac{ e^{-\eta L} }{\alpha^u - \eta} ||W_2^-||_\eta
\end{align*}

Combining these, we have

\begin{align*}
\int_0^L |\Phi^s_+(L, y)|&|G_1^+(y)||W_1^+(y)| dy + \int_{-L}^0 |\Phi^u_i(-L, y)||G_2^-(y)||W_2^-(y)| dy \\
&\leq C|G| \left( \frac{1}{\alpha^s - \eta} + \frac{1}{\alpha^u - \eta} \right) e^{-\eta L}||W||_\eta
\end{align*}

\item

\begin{align*}
\int_0^L |\Phi^s_+(L, y)||(K_1^+ B W_1^+)(y)| dy &\leq C \int_0^L e^{-\alpha^s(L-y)}\int_y^L|w_1^+(z)|dz dy \\
&= C \int_0^L e^{-\alpha^s(L-y)}\int_y^L|e^{\eta z} w_1^+(z)|e^{-\eta z} dz dy \\
&\leq C e^{-\eta L} ||W_1^+||_\eta \int_0^L e^{-\alpha^s(L-y)}\int_y^L e^{\eta(L-z)} dz dy \\
&= C e^{-\eta L} ||W_1^+||_\eta \int_0^L e^{-\alpha^s(L-y)} \frac{e^{\eta(L - y)} - 1}{\eta} dy \\
&\leq C \frac{e^{-\eta L}}{\eta} ||W_1^+||_\eta \int_0^L e^{-(\alpha^s-\eta)(L-y)} dy \\
&= C \frac{e^{-\eta L}}{\eta} ||W_1^+||_\eta \frac{1 - e^{-(\alpha^s - \eta)L}}{\alpha^s - \eta} \\
&= C \frac{e^{-\eta L}}{\eta(\alpha^s - \eta)} ||W_1^+||_\eta
\end{align*}

Similarly,

\begin{align*}
\int_{-L}^0 |\Phi^u_-(-L, y)||(K_2^+ B W_1^+)(y)| dy &\leq C \int_{-L}^0 e^{-\alpha^u(L+y)}\int_{-L}^y|w_2^-(z)|dz dy \\
&= C \int_{-L}^0 e^{-\alpha^u(L+y)}\int_{-L}^y|e^{-\eta z} w_2^-(z)|e^{\eta z} dz dy \\
&\leq C e^{-\eta L} ||W_2^-||_\eta \int_{-L}^0 e^{-\alpha^u(L+y)}\int_{-L}^y e^{\eta(L+z)} dz dy \\
&= C e^{-\eta L} ||W_2^-||_\eta \int_{-L}^0 e^{-\alpha^u(L+y)} \frac{e^{\eta(L + y)} - 1}{\eta} dy \\
&\leq C \frac{e^{-\eta L}}{\eta} ||W_2^-||_\eta \int_{-L}^0 e^{-(\alpha^u - \eta)(L+y)} dy \\
&= C \frac{e^{-\eta L}}{\eta} ||W_2^-||_\eta \frac{1 - e^{-(\alpha^u - \eta)L}}{\alpha^u - \eta} \\
&= C \frac{e^{-\eta L}}{\eta(\alpha^u - \eta)} ||W_2^-||_\eta
\end{align*}

Combining these, we have

\begin{align*}
\int_0^L &|\Phi^s_+(L, y)||(K_1^+ B W_1^+)(y)| dy + \int_{-L}^0 |\Phi^u_-(-L, y)||(K_2^+ B W_1^+)(y)| dy \\
& \leq C \left( \frac{1}{\alpha^s - \eta} + \frac{1}{\alpha^u - \eta} \right) \frac{e^{-\eta L}}{\eta} ||W||_\eta
\end{align*}

\item Recall that for our pulse $q$ we showed that $q_{c}$ is bounded in the exponentially weighted space with weight $\eta$. The same holds for the double pulse $q_2$. Thus we have

\begin{align*}
\int_0^L |\Phi^s_+(L, y)| B |Q_{2c}(y)| dy &\leq C \int_0^L e^{-\alpha^s(L-y)}|e^{\eta y} q_{2c}(y)|e^{-\eta y} dy\\
& \leq C ||Q_{2c}||_\eta e^{-\eta L} \int_0^L e^{-(\alpha^s - \eta)(L-y)}dy \\
&= C ||Q_{2c}||_\eta e^{-\eta L}\frac{1 - e^{-(\alpha^s - \eta)L}}{\alpha^s - \eta} \\
&\leq C ||Q_{2c}||_\eta\frac{ e^{-\eta L}}{\alpha^s - \eta} 
\end{align*}

Similarly,

\begin{align*}
\int_0^L |\Phi^s_+(L, y)| B |Q_{2c}(y)| dy&\leq C ||Q_{2c}||_\eta\frac{ e^{-\eta L}}{\alpha^u - \eta} 
\end{align*}

Combining these, we have

\begin{align*}
\int_0^L |\Phi^s_+(L, y)| B |Q_{2c}(y)| dy + \int_0^L |\Phi^s_+(L, y)| B |Q_{2c}(y)| dy \leq
C ||Q_{2c}||_\eta \left( \frac{1}{\alpha^s - \eta} + \frac{1}{\alpha^u - \eta} \right) e^{-\eta L}
\end{align*}

\end{enumerate}

From all of this, we have a bound for $L_3(\lambda)$

\begin{align*}
|L_3(\lambda)(a,b,d)| &\leq p(L)|a| + C e^{-\alpha L}|b| \\
&+  C \left( \frac{1}{\alpha^s - \eta} + \frac{1}{\alpha^u - \eta} \right) e^{-\eta L} 
\left( |G| ||W||_\eta + \frac{1}{\eta} |\lambda| ||W||_\eta + |\lambda|^2 |d| ||Q_{2c}||_\eta \right)\\
&\leq C\left( p(L)|a| + e^{-\alpha L}|b| + \left(|G| + \frac{|\lambda|}{\eta}\right)e^{-\eta L}||W||_\eta + |\lambda|^2 |d| e^{-\eta L} ||Q_{2c}||_\eta \right)
\end{align*}

Since we are substituting $W = W_1(\lambda)(a,b,d)$ from Lemma \ref{W1}, we can use our bound for $|W||_\eta$ from there to get

\begin{align*}
|L_3(\lambda)(a,b,d)| 
&\leq C\left( p(L)|a| + e^{-\alpha L}|b| + \left(|G| + \frac{|\lambda|}{\eta}\right)e^{-\eta L}C(e^{\eta L}|a| + |b| + |\lambda|^2 |d|) + |\lambda|^2 |d| e^{-\eta L} ||Q_{2c}||_\eta \right) \\
&= C\left( p(L)|a| + e^{-\alpha L}|b| + C \left(|G| + \frac{|\lambda|}{\eta}\right)(|a| + e^{-\eta L}|b| + e^{-\eta L}|\lambda|^2 |d|) + |\lambda|^2 |d| e^{-\eta L} ||Q_{2c}||_\eta \right) \\
&= C \left( p(L) + |G| + \frac{|\lambda|}{\eta}\right)|a| + C \left[ e^{-\alpha L} + e^{-\eta L} \left(|G| + \frac{|\lambda|}{\eta}\right) \right] |b|  \\
&+ C e^{-\eta L} \left( |G| + \frac{|\lambda|}{\eta} + ||Q_{2c}||_\eta \right)|\lambda|^2 |d|
\end{align*}

We can simplify this a bit by lumping a few things into the constant $C$. $\eta$ is a constant, so we can incorportate that into $C$. The expression $|G| + \frac{|\lambda|}{\eta} + ||Q_{2c}||_\eta$ is the same order as $||Q_{2c}||_\eta$, since both $|G|$ and $\lambda$ are small, so we can lump that into the constant $C$. Thus this all becomes

\begin{align*}
|L_3&(\lambda)(a,b,d)| \leq C \left( \left( p(L) + |G| + |\lambda|\right) |a| + \left( e^{-\alpha L} + e^{-\eta L} \left(|G| + |\lambda|\right) \right) |b| + e^{-\eta L} |\lambda|^2 |d| \right)
\end{align*}


Recalling that $|G|\leq \delta$ and $|\lambda| \leq \delta$, we this becomes 

\[
|L_3(\lambda)(a,b,d)| \leq C\delta(|a| + |b| + |d|)
\]

Let $J_1: V_a \rightarrow \C^n$ be defined by $J_1(a) = a_1^+ - a_1^-$. (These are the only two nonzero components of $a$). Since $\C^n = E^u \oplus E^s$, the map $J_1$ is a linear isomorphism. Now consider the operator $S_1$ defined by

\[
S_1(a) = J_1 a + L_3(\lambda)(a, 0, 0) = J_1( I + J_1^{-1} L_3(\lambda)(a, 0, 0) )
\]

For suffiently small $\delta$, we will have the operator norm $||J_1^{-1} L_3(\lambda)(\cdot, 0, 0)|| < 1$, thus the map $a \rightarrow I + J_1^{-1} L_3(\lambda)(a, 0, 0)$ is invertible (for same reason as in Lemma \ref{W1}) and so the operator $S_1$ is invertible.\\

Recall that we have the following expression for $D_1 d$.

\[
D_1 d = (a^+_1 - a^-_1) + L_3(\lambda)(a,b,d)
\] 

Since $L_3$ is linear, we can write this as

\begin{align*}
D_1 d &= J_1 a + L_3(\lambda)(a, 0, 0) + L_3(\lambda)(0, b, d) \\
&= S_1(a) + L_3(\lambda)(0, b, d) 
\end{align*}

Writing this as 

\[
S_1(a) = D_1 d - L_3(\lambda)(0, b, d) 
\]

We can solve this for $a$ to get

\begin{equation}
a = A_1(\lambda)(b,d) = S_1^{-1}[D_1 d - L_3(\lambda)(0, b, d)] 
\end{equation}

We can find a bound for $A_1$. Note that $S_1^{-1}$ is a bounded linear operator.

\begin{align*}
|A_1(\lambda)(b,d)| &\leq ||S_1^{-1}|||D_1 d - L_3(\lambda)(0, b, d)| \\
&\leq C (|D_1||d| + |L_3(\lambda)(0, b, d)|) \\
&\leq C (|D_1||d| +  \left[e^{-\alpha L} + e^{-\eta L} \left(|G| + |\lambda| \right) \right]|b| + e^{-\eta L} |\lambda|^2 |d|)\\
&= C\left( \left[e^{-\alpha L} + e^{-\eta L} \left(|G| + |\lambda|\right) \right]|b| + \left[ e^{-\eta L} |\lambda|^2 + |D_1| \right] |d| \right)\\
\end{align*}

We also define

\begin{equation}\label{W2def}
W_2(\lambda)(b,d) = W_1(\lambda)(A_1(\lambda)(b,d),b,d)
\end{equation}

We can get a bound on this by plugging in the bound for $A_1(\lambda)$ into that for $W_1(\lambda)$.

\begin{align*}
||W_2&(\lambda)(a,b,d)||_\eta \\
&\leq C\left( e^{\eta L} \left( \left[e^{-\alpha L} + e^{-\eta L} \left(|G| + |\lambda|\right) \right]|b| + \left[ e^{-\eta L} |\lambda|^2 + |D_1| \right] |d| \right) + C( |b| + |\lambda|^2 |d| ) \right)\\
&\leq C \left( \left[e^{-(\alpha-\eta) L} + |G| + |\lambda| + 1 \right] |b| 
+ \left[ |\lambda|^2 + e^{\eta L}|D_1| \right] |d|\right)
\end{align*} 

Since $e^{-(\alpha-\eta) L} + |G| + |\lambda| + 1$ is of order 1 (the first three terms in the sum are small), we can again lump the first three terms into the contant to get the bound

\begin{align*}
||W_2(\lambda)(a,b,d)||_\eta \leq C \left( |b| + \left( |\lambda|^2 + e^{\eta L}|D_1| \right) |d|\right)
\end{align*} 

In the unweighted norm, the only difference is that we are missing the $e^{\eta L}$ term, so the unweighted bound is

\begin{align*}
||W_2(\lambda)(a,b,d)|| \leq C \left( |b| + \left( |\lambda|^2 + |D_1| \right) |d|\right)
\end{align*} 

Next we take the equation

\[
D_1 d = (a^+_1 - a^-_1) + L_3(\lambda)(a,b,d)
\]

and project both sides by $P_0^s$ and $P_0^u$, the projections on $E^s$ and $E_u$. Recalling that $a^-_1 \in E^u$ and $a^+_1 \in E_s$, we have

\begin{align*}
a^-_1 &= -P_0^s D_1 d + P_0^s L_3(\lambda)(a,b,d) \\
a^+_1 &= P_0^u D_1 d - P_0^u L_3(\lambda)(a,b,d)
\end{align*}

Substituting $a = A_1(\lambda)(b,d)$ into this, we get

\begin{align*}
(A_1(\lambda)(b,d))^- &= -P_0^s D_1 d + P_0^s L_3(\lambda)(A_1(\lambda)(b,d),b,d) \\
(A_1(\lambda)(b,d))^+ &= P_0^u D_1 d - P_0^u L_3(\lambda)(A_1(\lambda)(b,d),b,d)
\end{align*}

Define $A_2(\lambda): V_\lambda \times V_b \times V_d \rightarrow V_a$ piecewise by
\begin{align*}
(A_2(\lambda)(b,d))^- &= P_0^s L_3(\lambda)(A_1(\lambda)(b,d),b,d) \\
(A_2(\lambda)(b,d))^+ &= - P_0^u L_3(\lambda)(A_1(\lambda)(b,d),b,d)
\end{align*}

Then we have
\begin{align*}
(A_1(\lambda)(b,d))^- &= -P_0^s D_1 d + (A_2(\lambda)(b,d))^-\\
(A_1(\lambda)(b,d))^+ &= P_0^u D_1 d + (A_2(\lambda)(b,d))^+
\end{align*}

And, finally, we have the following estimate for $A_2(\lambda)$

\begin{align*}
|(A_2&(\lambda)(b,d))| \leq |P_0^{s/u}|| L_3(\lambda)(A_1(\lambda)(b,d),b,d) |\\
&\leq C \left( \left( p(L) + |G| + |\lambda|\right) |(A_1(\lambda)(b,d)| + \left( e^{-\alpha L} + e^{-\eta L} \left(|G| + |\lambda|\right) \right) |b| + e^{-\eta L} |\lambda|^2 |d| \right)\\
&\leq C \left( p(L) + |G| + |\lambda|\right) \left( \left[e^{-\alpha L} + e^{-\eta L} \left(|G| + |\lambda|\right) \right]|b| + \left[ e^{-\eta L} |\lambda|^2 + |D_1| \right] |d| \right) \\
&+ C\left( e^{-\alpha L} + e^{-\eta L} \left(|G| + |\lambda|\right) \right) |b| + C e^{-\eta L} |\lambda|^2 |d| \\
&= C \left[ \left( p(L) + |G| + |\lambda| \right) \left( e^{-\alpha L} + e^{-\eta L} \left(|G| + |\lambda|\right)\right) +  e^{-\alpha L} + e^{-\eta L} \left(|G| + |\lambda|\right) \right] |b| \\
&+ C \left[  \left( p(L) + |G| + |\lambda| \right)(e^{-\eta L} |\lambda|^2 + |D_1|) + e^{-\eta L} |\lambda|^2) \right]|d| \\
&= C \left[ \left( p(L) + |G| + |\lambda| + 1 \right) e^{-\alpha L} + \left( p(L) + |G| + |\lambda| + 1 \right) e^{-\eta L} \left(|G| + |\lambda|\right)\right] |b| \\
&+ C \left[ \left( p(L) + |G| + |\lambda| + 1 \right)e^{-\eta L} |\lambda|^2 + \left( p(L) + |G| + |\lambda| \right)|D_1| \right]|d| 
\end{align*}

Since $p(L) + |G| + |\lambda|$ is small compared to 1, we can simplify this bound by lumping more things into the constant $C$ to get

\begin{align*}
|(A_2&(\lambda)(b,d))| \leq \\
&= C \left[ \left( e^{-\alpha L} + e^{-\eta L} \left(|G| + |\lambda|\right)\right) |b|
+ \left( e^{-\eta L} |\lambda|^2 + \left( p(L) + |G| + |\lambda| \right)|D_1| \right)|d| \right]
\end{align*}

\end{proof}
\end{lemma}

At this point, we are ready to look at the remaining equations

\begin{align*}
W_i^\pm(x) &\in \C \Psi(0) \oplus Y^+ \oplus Y^- \\
W_i^+(0) - W_i^-(0) &\in \C \Psi(0) 
\end{align*}

First, we take $x = 0$ in the fixed point equations.

\begin{align*}
W_i^-(0) &= \Phi^s_-(0, -X_{i-1})a^-_{i-1} + \Phi^u_-(0, 0)b_i^- \\
&+ \int_{-X_{i-1}}^x \Phi^s_-(0, y)[G_i^-(y) W_i^-(y) + \lambda (K_i^-B W_i^-)(y) - \lambda^2 d_i B Q_{2c}(y) ] dy \\
W_i^+(0) &= \Phi^u_+(0, X_i)a^+_{i} + \Phi^s_+(0, 0)b_i^+ \\
&+ \int_{X_{i}}^x \Phi^u_+(0, y)[G_i^+(y) W_i^+(y) + \lambda (K_i^+ B W_i^+)(y) - \lambda^2 d_i B Q_{2c}(y) ] dy
\end{align*}

Recall that $\Phi^u_-(0, 0) = P^u_i(0)$, $\Phi^s_+(0, 0) = P^s_+(0)$. Since $b_i^-$ and $b_i^+$ are in the range of these projections, $\Phi^u_-(0, 0)b_i^- = b_i^-$ and $\Phi^s_+(0, 0)b_i^+ = b_i^+ $. Thus we have

\begin{align*}
W_i^-(0) &= \Phi^s_-(0, -X_{i-1})a^-_{i-1} + b_i^- \\
&+ \int_{-X_{i-1}}^0 \Phi^s_-(0, y)[G_i^-(y) W_i^-(y) + \lambda (K_i^-B W_i^-)(y) - \lambda^2 d_i B Q_{2c}(y) ] dy \\
W_i^+(0) &= \Phi^u_+(0, X_i)a^+_{i} + b_i^+ \\
&+ \int_{X_{i}}^0 \Phi^u_+(0, y)[G_i^+(y) W_i^+(y) + \lambda (K_i^+ B W_i^+)(y) - \lambda^2 d_i B Q_{2c}(y) ] dy
\end{align*}

We use $W = W_2(\lambda)(b,d)$ and $a = A_1(\lambda)(b,d)$ in this. By the definition of $V_b$, we can decompose the $b_i^\pm$ uniquely as
\[
b_i^\pm = x_i^\pm + y_i^\pm
\]
where $x_i^\pm \in \C Q'(0)$ and $y_i^\pm \in Y^\pm$. 

Since

\[
\C^n = \C\Psi(0) \oplus \C Q'(0) \oplus Y^- \oplus Y^+
\]

the conditions above are equivalent to

\begin{align*}\label{projeq}
P(\C Q'(0), \C\Psi(0) \oplus Y^- \oplus Y^+)W_i^-(0) &= 0 \\
P(\C Q'(0), \C\Psi(0) \oplus Y^- \oplus Y^+)W_i^+(0) &= 0 \\
P(Y^+ \oplus Y^-, \C Q'(0) \oplus \C\Psi(0) )(W_i^+(0) - W_i^-(0)) &= 0 \\
\end{align*}

where $P(X,Y)$ is the projection onto $X$ with kernel $Y$.

\begin{lemma}
There exist operators 
\begin{align*}
B_1: V_\lambda \times V_d \rightarrow V_b \\
A_3: V_\lambda \times V_d \rightarrow V_a \\
W_3: V_\lambda \times V_d \rightarrow V_w^\eta
\end{align*}
such that 
\[
(a,b,w) = (A_3(\lambda)d, B_1(\lambda)d, W_3(\lambda)d)
\]
solves the thing we want for any $d$ and $\lambda$. The operators are analytic in $\lambda$ and linear in $d$. Bounds for them are given in the proof.

\begin{proof}
Substituting the fixed point equations evaluated at $x = 0$ into the projection equations above, we have 
\begin{align*}
0 &= x_i^- + P(\C Q'(0), \C\Psi(0) \oplus Y^- \oplus Y^+) \Big( \Phi^s_-(0, -X_{i-1})a^-_{i-1}  \\
&+ \int_{-X_{i-1}}^0 \Phi^s_-(0, y)[G_i^-(y) W_i^-(y) + \lambda (K_i^-B W_i^-)(y) - \lambda^2 d_i B Q_{2c}(y) ] dy \Big) \\
0 &= x_i^+ + P(\C Q'(0), \C\Psi(0) \oplus Y^- \oplus Y^+) \Big( \Phi^u_+(0, X_i)a^+_{i} \\
&+ \int_{X_{i}}^0 \Phi^u_+(0, y)[G_i^+(y) W_i^+(y) + \lambda (K_i^+ B W_i^+)(y) - \lambda^2 d_i B Q_{2c}(y) ] dy \Big)\\
0 &= y_i^+ - y_i^- + P(Y^+ \oplus Y^-, \C Q'(0) \oplus \C\Psi(0) )\Big( \Phi^u_+(0, X_i)a^+_{i} - \Phi^s_-(0, -X_{i-1})a^-_{i-1} \\
&+ \int_{X_{i}}^0 \Phi^u_+(0, y)[G_i^+(y) W_i^+(y) + \lambda (K_i^+ B W_i^+)(y) - \lambda^2 d_i B Q_{2c}(y) ] dy \Big)\\
&- \int_{-X_{i-1}}^0 \Phi^s_-(0, y)[G_i^-(y) W_i^-(y) + \lambda (K_i^-B W_i^-)(y) - \lambda^2 d_i B Q_{2c}(y) ] dy \Big)
\end{align*}

We can write this in the form

\[ 
\begin{pmatrix}x_i^- \\ x_i^+ \\ y_i^+ - y_i^- \end{pmatrix} = (L_4(\lambda)(b,d))_i = 0
\]
where $L_4(\lambda)(b,d)$ is the rest of the RHS above.\\

We can get an estimate for $L_4(\lambda)$ as follows. Projections have norm 1, so those are not an issue. Since we need the maximum over the three components and all the terms in the first two are in the third, we only have to look at the third component. Since the two integrals are the same as in the definition of $L_3(\lambda)$, we have already done the initial bound.

\begin{align*}
|L_4&(\lambda)(b,d)| \\
&\leq C\left( e^{-\alpha^u L}|a_i^+| +  e^{-\alpha^s L}|a_i^-|
\right) + C \left( \frac{1}{\alpha^s - \eta} + \frac{1}{\alpha^u - \eta} \right) e^{-\eta L} 
\left( |G| ||W||_\eta + \frac{1}{\eta} |\lambda| ||W||_\eta + |\lambda|^2 |d| ||Q_{2c}||_\eta \right)
\end{align*}

Adjusting the constant as before, this becomes

\begin{align*}
|L_4&(\lambda)(b,d)| \leq C\left( e^{-\alpha L}|a|
+ e^{-\eta L} 
\left( |G| + |\lambda| \right) ||W||_\eta + |\lambda|^2 |d| \right)
\end{align*}

Substituting in $A_1(\lambda(b,d)$ and $W_2(\lambda)(b,d)$ and using their bounds, we get

\begin{align*}
|L_4&(\lambda)(b,d)| \leq C\left( e^{-\alpha L}|a|
+ e^{-\eta L} \left( |G| + |\lambda| \right) ||W||_\eta + |\lambda|^2 |d| \right)\\
&\leq C( e^{-\alpha L}\left( \left[e^{-\alpha L} + e^{-\eta L} \left(|G| + |\lambda|\right) \right]|b| + \left[ e^{-\eta L} |\lambda|^2 + |D_1| \right] |d| \right) \\
&+ e^{-\eta L} \left( |G| + |\lambda| \right) \left( |b| + \left( |\lambda|^2 + e^{\eta L}|D_1| \right) |d|\right) + |\lambda|^2 |d| ) \\
&= C(e^{-2 \alpha L} + (e^{-\eta L} + e^{-(\eta + \alpha)L})(|G| + |\lambda|))|b| \\
&+ C((e^{-\eta L}(|G| + |\lambda|) + e^{-(\eta + \alpha)L} + 1)|\lambda|^2 + (e^{-\alpha L} + |G| + |\lambda|)|D_1| )|d|
\end{align*}

Readjusting the constant $C$, we have


\begin{align*}
|L_4&(\lambda)(b,d)| \leq
C\left( (e^{-2 \alpha L} + e^{-\eta L}(|G| + |\lambda|))|b| 
+ (|\lambda|^2 + (e^{-\alpha L} + |G| + |\lambda|)|D_1| )|d| \right)
\end{align*}

Recalling that $b = x + y$, $|G| \leq \delta$, $|\lambda| \leq \delta$, and choosing $L$ sufficiently large so that $e^{-2 \alpha L} \leq \delta$, this becomes

\begin{align*}
|L_4&(\lambda)(b,d)| \leq C\delta(|x| + |y|) + C(|\lambda|^2 + (e^{-\alpha L} + |G| + |\lambda|)|D_1| )|d|
\end{align*}

Since the map $J_2$ defined by
\[
J_2( (x_i^+, x_i^-),(y_i^+, y_i^-)) \rightarrow ( x_i^+, x_i^-, y_i^+ -  y_i^- )
\]

is an isomorphism, the operator

\[
S_2(x,y) = J_2(x+y) + L^4(\lambda)(x+y,0)
\]

is invertible (as in the previous lemma). Thus

\[
b = B_1(\lambda)(d) = -S_2^{-1}L_4(\lambda)(0,d)
\]

solves our problem. For the operator $B_1(\lambda)$, since $S_2^{-1}$ is a bounded linear operator, we have estimate

\begin{align*}
|B_1(\lambda)(d)| &\leq C |L_4(\lambda)(0,d)| \\
&\leq C(|\lambda|^2 + (e^{-\alpha L} + |G| + |\lambda|)|D_1| )|d|
\end{align*}

Substituting this into the operators $A_1(\lambda)$ and $W_3(\lambda)$ from the previous lemma gives us operators $A_3$ and $W_3$, which have bounds

\begin{align*}
|A_3(\lambda)(d)| &\leq C\left( \left[e^{-\alpha L} + e^{-\eta L} \left(|G| + |\lambda|\right) \right]|B_1(\lambda)(d)| + \left[ e^{-\eta L} |\lambda|^2 + |D_1| \right] |d| \right)\\
&\leq C\left( \left[e^{-\alpha L} + e^{-\eta L} \left(|G| + |\lambda|\right) \right](|\lambda|^2 + (e^{-\alpha L} + |G| + |\lambda|)|D_1| )|d|)+ \left[ e^{-\eta L} |\lambda|^2 + |D_1| \right] |d| \right)
\end{align*}

Adjusting constants, this bound becomes

\begin{align*}
|A_3(\lambda)(d)| &\leq C\left( (e^{-\alpha L} + e^{-\eta L})|\lambda^2| + |D_1| \right)|d|
\end{align*}

Since $\eta \leq \alpha$, we can readjust the constant $C$ to get the simpler bound

\begin{align*}
|A_3(\lambda)(d)| &\leq C\left( e^{-\eta L}|\lambda^2| + |D_1| \right)|d|
\end{align*}


For $W_3(\lambda)$ we have the bound in the weighted space
\begin{align*}
||W_3&(\lambda)(a,b,d)||_\eta \leq C \left( |B_1(\lambda)(d)| + \left( |\lambda|^2 + e^{\eta L}|D_1| \right) |d|\right) \\
&\leq C \left( |\lambda|^2 + e^{\eta L}|D_1|\right)|d|
\end{align*} 

The unweighted bound is
\begin{align*}
||W_3&(\lambda)(a,b,d)|| \leq C \left( |\lambda|^2 + |D_1|\right)|d|
\end{align*} 


Using $A_2(\lambda)$ from the previous lemma, we can define

\[
A_4(\lambda)d = A_2(\lambda)(B_1(\lambda)d,d)
\]

which lets us write $A_3(\lambda)$ as 

\begin{align*}
(A_3(\lambda)(d))^- &= -P_0^s D_1 d + (A_4(\lambda)(d))^-\\
(A_3(\lambda)(d))^+ &= P_0^u D_1 d + (A_4(\lambda)(d))^+
\end{align*}

Plugging in the estimate for $B_1(\lambda)$ into that of $A_2(\lambda)$ from the previous lemma, we get an estimate for $A_4(\lambda)$.

\begin{align*}
|(A_4&(\lambda)d)| \leq \\
&= C \left[ \left( e^{-\alpha L} + e^{-\eta L} \left(|G| + |\lambda|\right)\right) |B_1(\lambda)d|
+ \left( e^{-\eta L} |\lambda|^2 + \left( p(L) + |G| + |\lambda| \right)|D_1| \right)|d| \right]\\
&\leq C [ \left( e^{-\alpha L} + e^{-\eta L} \left(|G| + |\lambda|\right)\right) (|\lambda|^2 + (e^{-\alpha L} + |G| + |\lambda|)|D_1| )|d|\\
&+ \left( e^{-\eta L} |\lambda|^2 + \left( p(L) + |G| + |\lambda| \right)|D_1| \right)|d| ]
\end{align*}

Again, adjusting the constant $C$, this becomes

\begin{align*}
|(A_4&(\lambda)d| \leq C ( e^{-\eta L}|\lambda|^2 + (e^{-\alpha L} + |G| + |\lambda|)|D_1|)|d|
\end{align*}

Note that compared to (3.39) in Sandstede (1998), we have an extra $|\lambda|$ in the $D_1$ term. In that case, we had another term in $|\lambda|$ to absorb it. In this case, we have a term in $|\lambda^2|$ instead of $|\lambda|$, so we cannot absorb it. Fortunately, it will not end up mattering.

\end{proof}
\end{lemma}

We can use what we have done above to get a sharper estimate for $W_i^\pm$ in the unweighted norm, which we will need.

\begin{lemma}
We have the following estimates for $|W_i^\pm(x)|$ in the unweighted norm

\begin{align*}
| W_i^-(x)| &\leq C\left(\left( e^{-\alpha(x + X_{i-1})} + e^{-\alpha L} + e^{\eta L}(|G| + |\lambda|) \right)|D_1| + |\lambda|^2 \right) |d| \\
| W_i^+(x)| &\leq C\left(\left( e^{\alpha(x - X_{i})} + e^{-\alpha L} + e^{\eta L}(|G| + |\lambda|) \right)|D_1| + |\lambda|^2 \right) |d|
\end{align*}
where $\alpha = \min(\alpha^s, \alpha^u)$.

\begin{proof}
Since we have 
\[
W_i^\pm = L_1(\lambda)W_i^\pm + L_2(\lambda)W_i^\pm 
\]

we can use the estimate for $L_1(\lambda)$ together with an improved estimate for $L_2(\lambda)$. We use the same estimate for $L_1(\lambda)$ as we have above.

\[
| (L_1(\lambda) W_i^-)(x) | \leq C\left(|G| +|\lambda|\right)||W||_\eta
\]

For $L_2(\lambda)$ we will use the following estimate for the negative piece.

\begin{align*}
| (L_2(\lambda)(a, b, d)_i^-)(x)| &\leq C \left( e^{-\alpha^s(x + X_{i-1})} |a^-_{i-1}| + |b_i^-| + |\lambda|^2 |d_i| \right)
\end{align*}

Combining these, we have

\begin{align*}
| W_i^-(x)| &\leq C \left( e^{-\alpha^s(x + X_{i-1})} |a^-_{i-1}| + |b_i^-| + \left(|G| +|\lambda|\right)||W||_\eta + |\lambda|^2 |d_i| \right)
\end{align*}

Substituting in $A_3(\lambda)$, $B_1(\lambda)d$, and $W_3(\lambda)d$, this estimate becomes

\begin{align*}
| W_i^-(x)| &\leq C ( e^{-\alpha^s(x + X_{i-1})} \left( e^{-\eta L}|\lambda^2| + |D_1| \right)|d| \\
&+ (|\lambda|^2 + (e^{-\alpha L} + |G| + |\lambda|)|D_1| )|d| \\
&+ \left(|G| +|\lambda|\right)\left( |\lambda|^2 + e^{\eta L}|D_1|\right)|d| + |\lambda|^2 |d| )\\
&\leq C\left(\left( e^{-\alpha(x + X_{i-1})} + e^{-\alpha L} + |G| + |\lambda| + e^{\eta L}(|G| + |\lambda|) \right)|D_1| + |\lambda|^2 \right) |d|
\end{align*}

Since $e^{\eta L}$ is at least order 1, we have

\begin{align*}
| W_i^-(x)| &\leq C\left(\left( e^{-\alpha(x + X_{i-1})} + e^{-\alpha L} + e^{\eta L}(|G| + |\lambda|) \right)|D_1| + |\lambda|^2 \right) |d|
\end{align*}

Similarly

\begin{align*}
| W_i^+(x)| \leq C\left(\left( e^{\alpha(x - X_{i})} + e^{-\alpha L} + e^{\eta L}(|G| + |\lambda|) \right)|D_1| + |\lambda|^2 \right) |d|
\end{align*}

\end{proof}

\end{lemma}

We can do something similar in the weighted norm. Here we split $W_i^\pm$ up into two terms and provide an estimate for the higher order term in the weighted norm.

\begin{lemma}

We can write $W_i^\pm(x)$ as 
\begin{align*}
W_i^-(x) &= -\Phi^s_-(x, -X_{i-1}) P_0^s D_1 d + [T_2(\lambda)(b,d)]_i^- W_i^-(x)\\
W_i^+(x) &= \Phi^u_+(x, X_i) P_0^u D_1 d + [T_2(\lambda)(b,d)]_i^+ W_i^+(x)\\
\end{align*}

We have the following estimates for $|e^{\pm \eta x} T_2(\lambda)(b,d)_i^\pm(x) W_i^\pm(x)|$.

\begin{align*}
| e^{-\eta x} T_2(\lambda)(b,d)_i^\pm W_i^\pm(x) | &\leq C\left(\left( e^{\eta L}(e^{-\alpha L} + |G| + |\lambda|) \right)|D_1| + |\lambda|^2 \right) |d| \\
\end{align*}

where $\alpha = \min(\alpha^s, \alpha^u)$.

\begin{proof}
We have from a prior lemma that
\[
W_i^\pm = L_1(\lambda)W_i^\pm + L_2(\lambda)W_i^\pm 
\]

We have the following estimate for $L_1(\lambda)$.

\[
|e^{-\eta x} (L_1(\lambda) W_i^\pm)(x) | \leq C\left(|G| +|\lambda|\right)||W||_\eta
\]

Recall that $L_2(\lambda)$ is defined by

\begin{align*}
L_2(\lambda)(a, b, d)_i^-(x) &= \Phi^s_-(x, -X_{i-1})a^-_{i-1} + \Phi^u_-(x, 0)b_i^- \\
&- \lambda^2 d_i \left( \int_0^x \Phi^u_-(x, y)(KBQ_{2c})(y) dy  + \int_{-X_{i-1}}^x \Phi^s_-(x, y)(KBQ_{2c})(y) dy \right)\\
L_2(\lambda)(a, b, d)_i^+(x) &= \Phi^u_+(x, X_i)a^+_{i} + \Phi^s_+(x, 0)b_i^+ \\
&- \lambda^2 d_i \left( \int_0^x \Phi^s_+(x, y)(KBQ_{2c})(y) dy + \int_{X_{i}}^x \Phi^u_+(x, y)(KBQ_{2c})(y) dy \right)
\end{align*}

We will look a the ``negative'' piece only; the ``positive'' piece is similar. For the ``negative'', we write $L_2(\lambda)_i^-$ as 

\begin{align*}
L_2(\lambda)(a, b, d)_i^-(x) &= \Phi^s_-(x, -X_{i-1})a^-_{i-1} + T_1(\lambda)(b,d)_i^-
\end{align*}

where

\begin{align*}
T_1(\lambda)(b,d)_i^- &= \Phi^u_-(x, 0)b_i^- - \lambda^2 d_i \left( \int_0^x \Phi^u_-(x, y)(KBQ_{2c})(y) dy  + \int_{-X_{i-1}}^x \Phi^s_-(x, y)(KBQ_{2c})(y) dy \right)
\end{align*}


From the estimate for $L_2(\lambda)$ in Lemma \ref{L2} we have the following estimate for $ e^{-\eta x} T_1(\lambda)(b,d)_i^-$

\begin{align*}
| e^{-\eta x} T_1(\lambda)(b,d)_i^-| &\leq C \left( |b_i^-| + |\lambda|^2 |d_i| \right)
\end{align*}

For $a^-_{i-1}$ we use
\[
(A_3(\lambda)(d))^- = -P_0^s D_1 d + (A_4(\lambda)(d))^-
\]

This gives us

\begin{align*}
L_2(\lambda)(a, b, d)_i^-(x) &= -\Phi^s_-(x, -X_{i-1}) P_0^s D_1 d + \Phi^s_-(x, -X_{i-1}) (A_4(\lambda)(d))^- + T_1(\lambda)_i^-(b,d)
\end{align*}

Combining everything, we have

\begin{align*}
W_i^- &= \Phi^s_-(x, -X_{i-1}) P_0^s D_1 d + L_1(\lambda)W_i^- + \Phi^s_-(x, -X_{i-1}) (A_4(\lambda)(d))^- + T_1(\lambda)(b,d)_i^-
\end{align*}

Define $T_2(\lambda)(b,d)_i^- W_i^-$ by 
\[
T_2(\lambda)(b,d)_i^- W_i^- = L_1(\lambda) W_i^- + \Phi^s_-(x, -X_{i-1}) (A_4(\lambda)(d))^- + T_1(\lambda)_i^-(b,d)
\]

Then

\begin{align*}
W_i^- &= -\Phi^s_-(x, -X_{i-1}) P_0^s D_1 d + T_2(\lambda)(b,d)_i^-\\
\end{align*}

For $T_2(\lambda)(b,d)_i^- W_i^-$, we have the bound

\begin{align*}
|e^{-\eta x} T_2^\pm(\lambda)(b,d)_i^- W_i^-(x)| &\leq C \left( e^{\eta X_{i-1}} e^{-\alpha^s(x + X_{i-1})} |A_4(\lambda)d| + |b_i^-| + \left(|G| +|\lambda|\right)||W||_\eta + |\lambda|^2 |d_i| \right)
\end{align*}

Next we substitute in $B_1(\lambda)d$ and $W_3(\lambda)d$ and use the estimate for $A_4(\lambda)d$. This estimate then becomes

\begin{align*}
| e^{-\eta x} T_2^-(\lambda)(b,d)_i^- W_i^-(x) | &\leq C ( e^{\eta X_{i-1}} e^{-\alpha^s(x + X_{i-1})} \left( e^{-\eta L}|\lambda^2| + (e^{-\alpha L} + |G| + |\lambda|)|D_1| \right)|d| \\
&+ (|\lambda|^2 + (e^{-\alpha L} + |G| + |\lambda|)|D_1| )|d| \\
&+ \left(|G| +|\lambda|\right)\left( |\lambda|^2 + e^{\eta L}|D_1|\right)|d| + |\lambda|^2 |d| )\\
&\leq C\left( \left( e^{-\alpha L} + |G| + |\lambda| + e^{\eta L}(e^{-\alpha L} + |G| + |\lambda|) \right)|D_1| + |\lambda|^2 \right) |d| \\
\end{align*}

where we took $X_{i-1} = L$ since we have a double pulse.\\

Since $e^{\eta L}$ is at least order 1, we have

\begin{align*}
| e^{-\eta x} T_2^-(\lambda)(b,d)_i^- W_i^-(x) | &\leq C\left(\left( e^{\eta L}(e^{-\alpha L} + |G| + |\lambda|) \right)|D_1| + |\lambda|^2 \right) |d| \\
\end{align*}

Similarly,

\begin{align*}
W_i^+ &= \Phi^u_+(x, X_i) P_0^u D_1 d + T_2(\lambda)(b,d)_i^+\\
\end{align*}

where 

\begin{align*}
| e^{-\eta x} T_2^+(\lambda)(b,d)_i^+ W_i^+(x) | &\leq C\left(\left( e^{\eta L}(e^{-\alpha L} + |G| + |\lambda|) \right)|D_1| + |\lambda|^2 \right) |d| \\
\end{align*}

\end{proof}
\end{lemma}

Next, we use this bound to estimate some integrals.

\begin{lemma}
We have the following bound for these integrals involving $W_i^\pm$

\begin{align*}
&\left| \int_{X_i}^0 \langle \Psi(x), (K_i^+ B W_i^+)(x)\rangle dx \right| \\
&\leq \left| \int_{X_i}^0 \langle \Psi(x), \int_{X_i}^x \Phi^u_+(y, X_i) P_0^u D_1 d \:dy \rangle dx \right| + C \left(\left( e^{\eta L}(e^{-\alpha L} + |G| + |\lambda|) \right)|D_1| + |\lambda|^2 \right) |d| \\
&\left| \int_{-X_{i-1}}^0 \langle \Psi(x), (K_i^- B W_i^-)(x)\rangle dx \right| \\
&\leq \left| \int_{-X_{i-1}}^0 \langle \Psi(x), \int_{-X_{i-1}}^x \Phi^s_-(y, -X_{i-1}) P_0^s D_1 d  \:dy \rangle dx \right| + C \left(\left( e^{\eta L}(e^{-\alpha L} + |G| + |\lambda|) \right)|D_1| + |\lambda|^2 \right) |d|
\end{align*}

\begin{proof}

For the adjoint solution $\Psi$, we use the estimate

\[
|\Psi(x)| \leq C e^{\tilde{\alpha}|x|}
\]

Using the estimate from the previous lemma,
\begin{align*}
\left| \int_0^{X_i} \langle \Psi(x), G_i^+(x) W_i^+(x) \rangle dx \right| &\leq \int_0^{X_i} |\Psi(x)||G| |W_i^+(x)| dx \\
&\leq C|G| \int_0^{X_i} e^{-\tilde{\alpha} x} \left(\left( e^{\alpha(x - X_{i})} + e^{-\alpha L} + e^{\eta L}(|G| + |\lambda|) \right)|D_1| + |\lambda|^2 \right) |d| dx 
\end{align*}

For the first term inside the integral,

\begin{align*}
e^{-\tilde{\alpha} x} e^{\alpha(x - X_{i})} &= e^{-\alpha X_i} e^{-(\tilde{\alpha} - \alpha)x}
\end{align*}

By our hyperbolicity assumption (CHECK THIS), we can make the constant $\tilde{\alpha}$ a little larger than $\alpha$ so that the above expression decays since $x \geq 0$. Thus we have

\begin{align*}
\left| \int_0^{X_i} \langle \Psi(x), G_i^+(x) W_i^+(x) \rangle dx \right| &\leq C|G| \int_0^{\infty} \left(  e^{-(\tilde{\alpha} - \alpha)x} e^{-\alpha X_i} |D_1| +  e^{-\tilde{\alpha} x} \left( \left(e^{-\alpha L} + e^{\eta L}(|G| + |\lambda|)  \right) |D_1| + |\lambda|^2 \right) \right) |d|dx \\
&\leq C|G| \left(  \frac{1}{\tilde{\alpha} - \alpha}e^{-\alpha X_i}|D_1| + \frac{1}{\tilde{\alpha}}\left( \left(e^{-\alpha L} + e^{\eta L}( |G| + |\lambda|)  \right) |D_1| + |\lambda|^2 \right) \right)|d| \\
&\leq C|G| \left( \left( e^{-\alpha L} + e^{\eta L}(|G| + |\lambda|)  \right) |D_1| + |\lambda|^2 \right)|d|
\end{align*}

where in the last line we took $X_i = L$, as is the case for the double pulse. Similarly,

\begin{align*}
\left| \int_{-X_{i-1}}^0 \langle \Psi(x), G_i^-(x) W_i^-(x) \rangle dx \right| &\leq C|G| \left( \left( e^{-\alpha L} + e^{\eta L}(|G| + |\lambda|)  \right) |D_1| + |\lambda|^2 \right)|d|
\end{align*}

\end{proof}
\end{lemma}

In the next lemma, we use what we have done to get an estimate on integrals involving our integration operator.

\begin{lemma}
We have the following bound on these integrals involving the integration operator.

\begin{align*}
\left| \int_0^{X_i} \langle \Psi(x), (K_i^+ W_i^+)(x)\rangle dx \right| &\leq C\left((1 + e^{\eta L}(|G| + |\lambda|)) |D_1| + |\lambda|^2 \right) |d| \\
\left| \int_{-X_{i-1}}^0 \langle \Psi(x), (K_i^- W_i^-)(x)\rangle dx \right| &\leq C\left((1 + e^{\eta L}(|G| + |\lambda|)) |D_1| + |\lambda|^2 \right) |d|
\end{align*}

\begin{proof}

We do this for the ``positive'' piece. The ``negative'' piece is similar. For the ``positive'' piece, we use the above lemma to write $W_i^+(x)$ as 

\begin{align*}
W_i^+(x) &= \Phi^u_+(x, X_i) P_0^u D_1 d + [T_2(\lambda)(b,d)]_i^+ W_i^+(x)
\end{align*}

where we have an estimate for $e^{-\eta x} T_2(\lambda)(b,d)_i^\pm W_i^\pm(x)$. Using these expressions, this becomes

\begin{align*}
&\left| \int_{X_i}^0 \langle \Psi(x), (K_i^+ B W_i^+)(x)\rangle dx \right| \\
&\leq \left| \int_{X_i}^0 \langle \Psi(x), K_i^+ B\Phi^u_+(x, X_i) P_0^u D_1 d \rangle dx \right| + C \int_0^{X_i} e^{-\alpha x} \int_x^{X_i} |T_2(\lambda)(b,d)_i^+ W_i^+(y)| dy dx \\
&\leq \left| \int_{X_i}^0 \langle \Psi(x), \int_{X_i}^x B\Phi^u_+(y, X_i) P_0^u D_1 d \:dy \rangle dx \right| + C \int_0^{X_i} e^{-\alpha x} \int_x^{X_i} e^{-\eta y} |e^{\eta y} T_2(\lambda)(b,d)_i^+ W_i^+(y)| dy dx \\
&\leq \left| \int_{X_i}^0 \langle \Psi(x), \int_{X_i}^x B\Phi^u_+(y, X_i) P_0^u D_1 d \: dy \rangle dx \right| \\
&+ C \int_0^{X_i} e^{-\alpha x} \int_x^{X_i} e^{-\eta y} \left(\left( e^{\eta L}(e^{-\alpha L} + |G| + |\lambda|) \right)|D_1| + |\lambda|^2 \right) |d| dy dx \\
&= \left| \int_{X_i}^0 \langle \Psi(x), \int_{X_i}^x B\Phi^u_+(y, X_i) P_0^u D_1 d \: dy \rangle dx \right| \\
&+ C \left(\left( e^{\eta L}(e^{-\alpha L} + |G| + |\lambda|) \right)|D_1| + |\lambda|^2 \right) |d| \int_0^{X_i} e^{-\alpha x} \int_x^{X_i} e^{-\eta y} dy dx \\
\end{align*}

For the second integral on the RHS,

\begin{align*}
\int_0^{X_i} &e^{-\alpha x} \int_x^{X_i} e^{-\eta y} dy dx
\leq \int_0^{X_i} e^{-\alpha x} \int_x^{\infty} e^{-\eta y} dy dx \\
&\leq \frac{1}{\eta} \int_0^{\infty} e^{-(\alpha + \eta) x} dx \\
&= \frac{1}{\eta(\alpha + \eta)}
\end{align*}

Combining these, we have

\begin{align*}
&\left| \int_{X_i}^0 \langle \Psi(x), (K_i^+ B W_i^+)(x)\rangle dx \right| \\
&\leq \left| \int_{X_i}^0 \langle \Psi(x), \int_{X_i}^x \Phi^u_+(y, X_i) P_0^u D_1 d \:dy \rangle dx \right| + C \left(\left( e^{\eta L}(e^{-\alpha L} + |G| + |\lambda|) \right)|D_1| + |\lambda|^2 \right) |d|
\end{align*}

Similarly, we have

\begin{align*}
&\left| \int_{-X_{i-1}}^0 \langle \Psi(x), (K_i^- B W_i^-)(x)\rangle dx \right| \\
&\leq \left| \int_{-X_{i-1}}^0 \langle \Psi(x), \int_{-X_{i-1}}^x \Phi^s_-(y, -X_{i-1}) P_0^s D_1 d  \:dy \rangle dx \right| + C \left(\left( e^{\eta L}(e^{-\alpha L} + |G| + |\lambda|) \right)|D_1| + |\lambda|^2 \right) |d|
\end{align*}

As of now, we do not have estimates for the leading order terms.

\end{proof}
\end{lemma}


\end{document}