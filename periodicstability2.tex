\documentclass[12pt]{article}
\usepackage[pdfborder={0 0 0.5 [3 2]}]{hyperref}%
\usepackage[left=1in,right=1in,top=1in,bottom=1in]{geometry}%
\usepackage[shortalphabetic]{amsrefs}%
\usepackage{amsmath}
\usepackage{enumerate}
% \usepackage{enumitem}
\usepackage{amssymb}                
\usepackage{amsmath}                
\usepackage{amsfonts}
\usepackage{amsthm}
\usepackage{bbm}
\usepackage[table,xcdraw]{xcolor}
\usepackage{tikz}
\usepackage{float}
\usepackage{booktabs}
\usepackage{svg}
\usepackage{mathtools}
\usepackage{cool}
\usepackage{url}
\usepackage{graphicx,epsfig}
\usepackage{makecell}
\usepackage{array}

\def\noi{\noindent}
\def\T{{\mathbb T}}
\def\R{{\mathbb R}}
\def\N{{\mathbb N}}
\def\C{{\mathbb C}}
\def\Z{{\mathbb Z}}
\def\P{{\mathbb P}}
\def\E{{\mathbb E}}
\def\Q{\mathbb{Q}}
\def\ind{{\mathbb I}}

\DeclareMathOperator{\spn}{span}
\DeclareMathOperator{\ran}{ran}

\graphicspath{ {periodic/} }

\newtheorem{lemma}{Lemma}
\newtheorem{theorem}{Theorem}
\newtheorem{corollary}{Corollary}
\newtheorem{definition}{Definition}
\newtheorem{assumption}{Assumption}
\newtheorem{hypothesis}{Hypothesis}

\newtheorem{notation}{Notation}

\begin{document}

Here we look at the variational problem / adjoint variational equation for the linearization of KdV5 about an equilbrium solution $u^*$. This linearization is

\begin{equation}
L(u^*) = \partial_x H(u^*) = \partial_x ( \partial_x^4 - \partial_x^2 + c - 2 u^*)
\end{equation}

The variational / adjoint variational equations for the linearization about the primary pulse solution $q(x)$ are given by

\begin{align}
V' = A(q(x))V \label{vareq} \\
W' = -A(q(x))^*W \label{adjvareq}
\end{align}

where

\begin{align}
A(u^*(x)) = \begin{pmatrix}0 & 1 & 0 & 0 & 0 \\0 & 0 & 1 & 0 & 0 \\0 & 0 & 0 & 1 & 0 \\0 & 0 & 0 & 0 & 1 \\
2u^*_x(x) & 2u^*(x) - c & 0 & 1 & 0 \end{pmatrix}
\end{align}

If we consider $A(0)$ for $c > 1/4$ we have eigenvalues $\nu = \{ 0, \pm \alpha \pm \beta i\}$, where $\alpha, \beta > 0$. The equilibrium at 0 is not hyperbolic, so we cannot directly apply the results of San98. The equilibrium at 0 has 2-dimensional stable/unstable manifolds, and a 1-dimensional center manifold. Let $W^{s/u/c}(0)$ be these manifolds.\\

Let $q(x)$ be the primary pulse (homoclinic orbit) solution, whose existence is known. Since its existence comes from the 4th order (integrated) problem, for which the equilibrim at 0 is hyperbolic, we will still make the same nondegeneracy assumption as before, i.e. we will assume the stable and unstable manifolds have a 1-dimensional intersection.

\begin{hypothesis}\label{nondegen}
\[
T_{Q(0)} W^u(0) \cap T_{Q(0)} W_s(0) = \R Q'(0)
\]
\end{hypothesis}

where $Q(x)$ is the single pulse solution on $\R$, written as a vector-valued function in $R^5$ in the usual way. We cannot conclude from Hypothesis \ref{nondegen} that $Q'(x)$ is the unique bounded solution to the variational equation \eqref{vareq}.\\

Now, we look at the adjoint variational equation. Since $L(q)^* = (\partial_x H(q))^* = -H(q) \partial_x$, we can easily see that there are two bounded solutions to this: $q(x)$ and the constant function $1$. Since we are on a periodic domain, the constant function is periodic and in $L^2$, so we have to consider it here.\\

Thus there are two bounded solutions $\Psi(x) = \nabla H(q(x))$ and $\Psi_c(x)$ of \eqref{adjvareq}, corresponding to $q(x)$ and 1.\\

Next, as in San98, we will decompose the tangent space at $Q(0)$. First, we define $Y^-$ and $Y^+$ as the ``rest of'' the tangent space of the unstable/stable manifolds.

\begin{align*}
T_{Q(0)} W^u(0) &= \R Q'(0) \oplus Y^- \\
T_{Q(0)} W^s(0) &= \R Q'(0) \oplus Y^+
\end{align*}

Counting dimensions and using Hypothesis \ref{nondegen}, $\R Q'(0) \oplus Y^- \oplus Y^+$ is ``missing'' two dimensions, so we want to fill that in.\\

Before we do that, we need to look at how the variational equation works. The idea, as far as I know, is that that the variational equation evolves the tangent spaces of the stable/unstable/center manifolds $W^{s/u/c}(0)$. I have not found a good reference on this but it seems to make sense. In any case, let $\Phi(x, y)$ be the evolution operator for the variational equation \eqref{vareq} on $\R$. This, of course, means that $\Phi(x, 0)Y$ is the unique solution to \eqref{vareq} for any IC $Y$given at $x = 0$. In this case, there are no perturbations or parameters. Then we should have the following.

\begin{enumerate}
	\item $Q'(x)$ is a solution to \eqref{vareq}, so we know what $U(x) = \Phi(x)Q'(0)$ is.
	\item For an IC $Y \in Y^-$ and $x \leq 0$, $|\Phi(x, 0) Y| \leq C e^{\alpha x}$. The decay rate $\alpha$ is the known decay rate of the unstable/stable manifold of the original equation.
	\item For an IC $Y \in Y^+$ and $x \geq 0$, $|\Phi(x, 0) Y| \leq C e^{-\alpha x}$. 
\end{enumerate}

Recall that we showed (somewhere) that

\begin{enumerate}[(i)]
\item $\langle V(x), W(x) \rangle$ is constant in $x$ for any solutions $V(x)$ to \eqref{vareq} and $W(x)$ to \eqref{adjvareq}.
\item $\Phi(x, y)^*$ is the evolution operator for \eqref{adjvareq}.
\end{enumerate}

Since the inner product is continuous in $x$, it follows that if as $x \rightarrow \infty$ (or $-\infty$), if either $V(x)$ or $W(x)$ decays exponentially and the other one is bounded (it could decay too of course), then $\langle V(0), W(0) \rangle = 0$, i.e. $V(0) \perp W(0)$. Choosing the appropriate things for $V(x)$ and either $\Psi(x)$ or $\Psi_c(x)$ for $W(x)$ we conclude that

\begin{align*}
\Psi(0) &\perp Q'(0) \oplus Y^- \oplus Y^+ \\
\Psi_c(0) &\perp Q'(0) \oplus Y^- \oplus Y^+
\end{align*}

Since $\Psi(0)$ and $\Psi_c(0)$ are linearly independent, we conclude that 

\[
\R^{2m + 1} = Q'(0) \oplus Y^- \oplus Y^+ \oplus \Psi(0) \oplus \Psi_c(0) 
\]

(this is $\R^5$ for KdV5).\\

Now we use the Gap Lemma from Zum2018. Taking $\lambda = 0$ (and the parameters $\beta_i^\pm = 0$), we can find solutions $V^\pm(x)$ and $W^\pm(x)$ on $\R^\pm$ which are given by

\begin{align}
V^\pm(x) = V(0) + \mathcal{O}(e^{-(\alpha - \epsilon)|x|}|V(0)|) \\
W^\pm(x) = W(0) + \mathcal{O}(e^{-(\alpha - \epsilon)|x|}|W(0)|)
\end{align}

$V(0)$ and $W(0)$ are the eigenvectors of $A(0)$ and $A(0)^*$ corresponding to the eigenvalue 0. In this case, we have $V(0) = (1, 0, 0, 0, 0)$ and $W(0) = (c, 0, -1, 0, 1)$. $\epsilon > 0$ is arbitrary.


\end{document}