\documentclass[12pt]{article}
\usepackage[pdfborder={0 0 0.5 [3 2]}]{hyperref}%
\usepackage[left=1in,right=1in,top=1in,bottom=1in]{geometry}%
\usepackage[shortalphabetic]{amsrefs}%
\usepackage{amsmath}
\usepackage{enumerate}
% \usepackage{enumitem}
\usepackage{amssymb}                
\usepackage{amsmath}                
\usepackage{amsfonts}
\usepackage{amsthm}
\usepackage{bbm}
\usepackage[table,xcdraw]{xcolor}
\usepackage{tikz}
\usepackage{float}
\usepackage{booktabs}
\usepackage{svg}
\usepackage{mathtools}
\usepackage{cool}
\usepackage{url}
\usepackage{graphicx,epsfig}
\usepackage{makecell}
\usepackage{array}

\def\noi{\noindent}
\def\T{{\mathbb T}}
\def\R{{\mathbb R}}
\def\N{{\mathbb N}}
\def\C{{\mathbb C}}
\def\Z{{\mathbb Z}}
\def\P{{\mathbb P}}
\def\E{{\mathbb E}}
\def\Q{\mathbb{Q}}
\def\ind{{\mathbb I}}

\DeclareMathOperator{\spn}{span}
\DeclareMathOperator{\ran}{ran}

\graphicspath{ {periodic/} }

\newtheorem{lemma}{Lemma}
\newtheorem{theorem}{Theorem}
\newtheorem{corollary}{Corollary}
\newtheorem{definition}{Definition}
\newtheorem{assumption}{Assumption}
\newtheorem{hypothesis}{Hypothesis}

\newtheorem{notation}{Notation}

\begin{document}

\subsection*{Variational Equation}

We look at the linearization of KdV5 about an equilbrium solution $u^*$. This linearization is

\begin{equation}
L(u^*) = \partial_x H(u^*) = \partial_x ( \partial_x^4 - \partial_x^2 + c - 2 u^*)
\end{equation}

The variational / adjoint variational equations for the linearization about the primary pulse solution $q(x)$ are given by

\begin{align}
V' = A(q(x))V \label{vareq} \\
W' = -A(q(x))^*W \label{adjvareq}
\end{align}

where

\begin{align}
A(u^*(x)) = \begin{pmatrix}0 & 1 & 0 & 0 & 0 \\ 0 & 0 & 1 & 0 & 0 \\ 0 & 0 & 0 & 1 & 0 \\ 0 & 0 & 0 & 0 & 1 \\
2u^*_x(x) & 2u^*(x) - c & 0 & 1 & 0 \end{pmatrix}
\end{align}

If we consider $A(0)$ for $c > 1/4$ we have eigenvalues $\nu = \{ 0, \pm \alpha \pm \beta i\}$, where $\alpha, \beta > 0$. The equilibrium at 0 is not hyperbolic, so we cannot directly apply the results of San98. The equilibrium at 0 has 2-dimensional stable/unstable manifolds, and a 1-dimensional center manifold. Let $W^{s/u/c}(0)$ be these manifolds.\\

Let $q(x)$ be the primary pulse (homoclinic orbit) solution, whose existence is known. Since its existence comes from the 4th order (integrated) problem, for which the equilibrim at 0 is hyperbolic, we will still make the same nondegeneracy assumption as before, i.e. we will assume the stable and unstable manifolds have a 1-dimensional intersection.

\begin{hypothesis}\label{nondegen}
\[
T_{Q(0)} W^u(0) \cap T_{Q(0)} W_s(0) = \R Q'(0)
\]
\end{hypothesis}

where $Q(x)$ is the single pulse solution on $\R$, written as a vector-valued function in $R^5$ in the usual way. We cannot, however, conclude from Hypothesis \ref{nondegen} that $Q'(x)$ is the unique bounded solution to the variational equation \eqref{vareq}.\\

Next, look at the adjoint operator $L(q)^*$ Since $L(q)^* = (\partial_x H(q))^* = -H(q) \partial_x$, we can easily see that there are two bounded solutions to this: $q(x)$ and the constant function $1$. Since we are on a periodic domain, the constant function is periodic and in $L^2$, so we have to consider it here.\\

Thus there are two bounded solutions $\Psi(x) = \nabla H(q(x))$ and $\Psi^c(x)$ of \eqref{adjvareq}, corresponding to $q(x)$ and 1. We computed $\Psi(x)$ previously.

\[
\Psi(x) = \begin{pmatrix}
q^{(4)}(x) - q''(x) + (-2q(x) + c)q(x)\\
-q^{(3)}(x) + q'(x) \\
q''(x) - q(x) \\
-q'(x) \\
q(x)
\end{pmatrix}
\]

All that really matters is that the last component is $q(x)$. We know this decays exponentially to 0, which is also clear upon inspection. Similarly, we can obtain an expression for $\Psi_c(x)$, which will have 1 as the last component.

\[
\Psi_c(x) = (c - 2 q(x), 0, -1, 0, 1)^T
\]

As $x \rightarrow \pm \infty$, this decays to $(c, 0, -1, 0, 1)$, which is the eigenvector of $-A(0)^*$ corresponding to the eigenvalue $0$ (i.e. the center eigenvalue of $-A(0)^*$).\\

\subsection{Decomposition of Tangent Space at 0}

Next, as in San98, we will decompose the tangent space at $Q(0)$. 

Counting dimensions and using Hypothesis \ref{nondegen}, $\R Q'(0) \oplus Y^- \oplus Y^+$ is ``missing'' two dimensions needed to fill out $\R^5$.\\

We can obtain two more useful directions from solutions to the adjoint variational equation: $\Psi(0)$ and $\Psi^c(0)$. We need to figure out how these fit into the puzzle.\\

Recall that if we have a solution $V(x)$ to \eqref{vareq} and a solution $W(x)$ to \eqref{adjvareq}, the inner product $\langle V(x), W(x) \rangle$ is constant in $x$. By the continuity of the inner product, this implies that if one of the two is bounded and the other decays exponentially at (at least) one end, their inner product must be 0.\\

$\Psi(0)$ and $\Psi_c(0)$ are linearly independent, but not (in general) perpendicular. In this specific case, we can compute their inner product and verify that it is nonzero. Since $\Psi(0)$ and $\Psi_c(0)$ are linearly independent and perpendicular to $\R Q'(0) \oplus Y^+ \oplus Y^-$, we have

\[
\R^5 = \R Q'(0) \oplus Y^+ \oplus Y^- \oplus \R \Psi(0) \oplus \Psi^c(0).
\]

% Using the Gap Lemma, we can find solutions $w^\pm(x)$ on $\R\pm$ to the adjoint variational equation \eqref{adjvareq} such that

% \[
% w^\pm(x) = w_0 + \mathcal{O}(e^{-\tilde{\alpha}|x|})
% \]

% where $w_0 = (c, 0, -1, 0, 1)$ is the eigenvector of the asymptotic matrix $-A(0)^*$ corrresponding to eigenvalue 0.\\

% By linearity, for any constant $c$,

% \begin{align*}
% [w^\pm(x) + c \Psi(x)]' &= (w^\pm)'(x) + c \Psi'(x) \\
% &= -A(0)^* w^\pm(x) - A(0)^* c \Psi(x) \\
% &= -A(0)^* ( w^\pm(x) + c \Psi(x) )
% \end{align*}

% Thus $w^\pm(x) + c \Psi(x)$ also solves \eqref{adjvareq}, and 

% Since $w^\pm(x)$ are bounded solutions to \eqref{adjvareq}, we have

% \begin{align*}
% w^+(0) &\perp Q'(0), Y^+ \\
% w^-(0) &\perp Q'(0), Y^-
% \end{align*}

% Furthermore, if we take $V^-(x)$ to be a solution to \eqref{vareq} with initial condition $V^-(0) \in Y^-$, then $|V^-(x)| \rightarrow \infty$ as $x \rightarrow \infty$ by the definition of $Y^-$. Since $\langle w^+(x), V^-(x)\rangle$ is constant in $x$, this is only possible if $\langle w^+(x), V^-(x)\rangle = 0$ for all $x$, since as $x \rightarrow \infty$, $w^+(x)$ remains bounded but $V^-(x)$ blows up. Thus $w^+(0) \perp Y^-$. Similarly, we can show that $w^-(0) \perp Y^+$. We have thus shown that

% \begin{align*}
% w^\pm(0) &\perp Q'(0), Y^+, Y^- \\
% \end{align*}

Use the Gap Lemma to find solutions $v^\pm(x)$ on $\R \pm$ to the variational equation \eqref{vareq} such that for any $0 < \tilde{\alpha} < \alpha$,

\[
v^\pm(x) = v_0 + \mathcal{O}(e^{-\tilde{\alpha}|x|})
\]

where $v_0 = (1/c, 0, 0, 0, 0)$ is the eigenvector of the asymptotic matrix $A(0)$ corrresponding to eigenvalue 0. We scaled $v_0$ so that $\langle v_0, w_0 \rangle = 1$, where $w_0 = (c, 0, -1, 0, 1)$ is the eigenvector of the asymptotic matrix $-A(0)^*$ corrresponding to eigenvalue 0.\\

If the initial condition $v^+(0)$ contains any component in $\R Q'(0) \oplus Y^+$, the contribution from that component to $v^+(x)$ decays exponentially to 0 faster than $e^{-\tilde{\alpha}x}$. Thus we can subtract out any such component in $v^+(0)$ without changing anything, and so we may assume that $v^+(0)$ has no component in $\R Q'(0) \oplus Y^+$. Furthermore, if $v^+(0)$ contains a component in $Y^-$, the inner product $\langle \Psi^c(x), v^+(x) \rangle$ will blow up as $x \rightarrow \infty$, since the component in $Y^-$ causes $v^+(x)$ to blow up in magnitude, but $\Psi^c(x)$ remains bounded. We conclude that $v^+(0)$ contains no components in $\R Q'(0) \oplus Y^+ \oplus Y^-$. Similarly, $v^-(0)$ contains no components in $\R Q'(0) \oplus Y^+ \oplus Y^-$.\\

Since $v^\pm(x)$ is a bounded solution to \eqref{vareq} and $\Psi(x$ is an exponentially decaying solution to \eqref{adjvareq}, we must have $v^\pm(0) \perp \Psi(0)$. Thus we have the expressions

\[
v^\pm 
\]

Taking the inner product with 

Note that from what we showed above, $v^\pm(0) \perp \Psi(0)$, since $\Psi(x)$ decays exponentially at both ends and $v^\pm(x)$ remains bounded. 



GARBAGE FOLLOWS.

Let $Y^0 = (\Psi(0) \oplus \perp Q'(0) \oplus Y^- \oplus Y^+)^\perp$, which is 1-dimensional. Then 

\[
\R^5 = \R Q'(0) \oplus Y^- \oplus Y^+ \oplus \R \Psi(0) \oplus Y^0
\]

This is not an orthogonal system, since $Q'(0)$, $Y^+$, and $Y^-$ are not necessarily mutually perpendicular, but it should be good enough.\\

It is worth noting that

\[
(\R \Psi(0))^\perp = \R Q'(0) \oplus Y^- \oplus Y^+ \oplus Y^0
\]

\subsection*{Center Subspace}

Now we use the Gap Lemma from Zum2018. Taking $\lambda = 0$ (and the parameters $\beta_i^\pm = 0$), we can find solutions $V^\pm(x)$ and $W^\pm(x)$ to \eqref{vareq} and \eqref{adjvareq} on $\R^\pm$ which are given by

\begin{align*}
V^\pm(x) = V_0 + \mathcal{O}(e^{-(\alpha - \epsilon)|x|}|V_0|) \\
W^\pm(x) = W_0 + \mathcal{O}(e^{-(\alpha - \epsilon)|x|}|W_0|)
\end{align*}

where the decay is in the appropriate direction. $V_0$ and $W_0$ are the eigenvectors of $A(0)$ and $A(0)^*$ corresponding to the eigenvalue 0. In this case, we have $V_0 = (1, 0, 0, 0, 0)$ and $W_0 = (c, 0, -1, 0, 1)$. The constant $\epsilon > 0$ is arbitrary. Since the inner product $\langle V^\pm(x), W^\pm(x) \rangle$ is constant and the inner product is continuous in both arguments, from the above expressions we have for all (appropriate) $x$

\[
\langle V^\pm(x), W^\pm(x) \rangle = \langle V_0, W_0 \rangle = c \neq 0
\]

In particular, $\langle V^\pm(0), W^\pm(0) \rangle \neq 0$. We would like to know where $V^\pm(0)$ and $W^\pm(0)$ fit into the overall scheme above. \\

\subsubsection*{Adjoint Center Solution}

Looking at the expression for $\Psi_c(x)$, it has the same decay properties as $W^\pm(x)$, and both it and $W^\pm(x)$ decay (in the appropriate direction) to the constant vector $W_0$. Thus we can take $W^\pm(x) = \Psi_c(x)$, so from now on, we will do that. Thus we have

\[
\langle V^\pm(0), \Psi_c(0) \rangle = \langle V_0, W_0 \rangle = c \neq 0
\]

\subsubsection*{Variational Center Solution}

We now turn our attention to $V^\pm(x)$. We look at $V^+(x)$ first. Since $V^+(x)$ is bounded and $\Psi(x)$ decays exponentially as $x \rightarrow \infty$, we have $V^+(0) \perp \Psi(0)$, i.e. 

\[
V^+(0) \in (\R \Psi(0))^\perp = \R Q'(0) \oplus Y^- \oplus Y^+ \oplus Y^0
\]

Thus we can write 

\[
V^+(0) = y^0 + y^- + y^+ + x^+
\]

where $y^0 \in Y^0, y^- \in Y^+, y^- \in Y^-, x^+ \in \R Q'(0)$. Since we can pick whatever we want as an IC, let's use the IC $V^+(0) - y^+ - x^+ = y^0 + y^-$. Since the variational equation is linear,

\begin{align*}
\Phi(x, 0)( V^+(0) - y^+ - x^+ ) &= \Phi(x, 0)V^+(0) - \Phi(x, 0) y^+ - \Phi(x, 0) x^+ \\
&= V_0 + \mathcal{O}(e^{-(\alpha - \epsilon)x}|V_0|) + \mathcal{O}(e^{-\alpha x}|y^+| + 
\mathcal{O}(e^{-\alpha x}|x^+|) \\
&= V_0 + \mathcal{O}(e^{-(\alpha - \epsilon)x})
\end{align*}

Using this IC, we still have a solution with the property we want. Let $\tilde{V}^+(x)$ be be solution to \eqref{vareq} with IC $V^+(0) - y^+ - x^+ = y^- + y^0$, i.e. with $\tilde{V}^+(0) \in Y^- \oplus Y^0$. We can do the same thing with $V^-(x)$ to get a solution $\tilde{V}^-(x)$ to \eqref{vareq} with $\tilde{V}^-(0) \in Y^+ \oplus Y^0$. Thus we have solutions to \eqref{vareq}

\begin{align*}
\tilde{V}^+(x) &= V_0 + \mathcal{O}(e^{-(\alpha - \epsilon)x}) && \tilde{V}^+(0) \in Y^0 \oplus Y^-, x \geq 0 \\
\tilde{V}^-(x) &= V_0 + \mathcal{O}(e^{(\alpha - \epsilon)x}) && \tilde{V}^-(0) \in Y^0 \oplus Y^+, x \geq 0
\end{align*}

To match these things up at $x = 0$, all we need to do is show that their initial conditions can have no component in $Y^\pm$. But this must be the case. To see that (in a handwavey fashion), suppose $\tilde{V}^+(0) = y^0 + y^-$, where $y^0 \in Y^0$ and $y^- \in Y^-$. Since $Y^-$ is the tangent space of the unstable manifold $W^u(0)$, ICs there will blow up (exponentially) as $x \rightarrow \infty$; or, at least, as $x \rightarrow \infty$, $\Phi(x,0)y^-$ will increase in magnitude and give us in the limit a component in $E^u$, which we know from above cannot happen. Thus $\tilde{V}^+(0)$ cannot have a component in $Y^-$. Similarly, $\tilde{V}^-(0)$ cannot have a component in $Y^+$. Thus we have solutions 

\begin{align*}
\tilde{V}^+(x) &= V_0 + \mathcal{O}(e^{-(\alpha - \epsilon)x}) && \tilde{V}^+(0) \in Y_0, x \geq 0 \\
\tilde{V}^-(x) &= V_0 + \mathcal{O}(e^{(\alpha - \epsilon)x}) && \tilde{V}^-(0) \in Y_0, x \geq 0
\end{align*}

We are now done, since these two solutions behave the way we want, and at $x = 0$, both $\tilde{V}^+(0)$ and $\tilde{V}^-(0)$ are contained in $Y^0$. This does not mean that $\tilde{V}^+(0) = \tilde{V}^-(0)$; we just know they are both contained in $Y^0$. Since this space is 1-dimensional, these must be scalar multiples of each other.\\

We would like to show that they are in fact equal. To do this, choose any initial condition $y^0 \in Y_0$ (say, a unit vector in that space). Then by what we have shown above, $V(x) = \Phi(x, 0)y^0$ is a bounded solution to \eqref{vareq} on $\R$, and its limits at $\pm \infty$ are scalar multiples of $V_0$, which may (in general) be different.\\

We will use a reversibility argument here. To see this, it is easiest to take this out of the 5th order system and write the variational equation as

\[
L(q)v = \partial_x H(q)v = \partial_x ( \partial_x^4 - \partial_x^2 + c - 2 q)v = 
(\partial_x^5 - \partial_x^3 + c \partial_x - 2 q \partial_x - 2 q_x)v
\]

Suppose $v(x)$ is a solution to this. Then it is not hard to show that so is $v(-x)$. Since the primary pulse $q(x)$ is even,

\begin{align*}
L(q(x))v(-x) &= -v^{(5)}(-x) + v'''(-x) - c v'(-x) + 2 q(x)v'(-x) - 2 q_x(x)v(-x) \\
&= -( v^{(5)}(-x) - v'''(-x) + c v'(-x) - 2 q(x)v'(-x) + 2 q_x(x)v(-x) ) \\
&= -( v^{(5)}(-x) - v'''(-x) + c v'(-x) - 2 q(-x)v'(-x) - 2 q_x(-x)v(-x) ) \\
&= -L(q(-x))v(-x) \\
&= -L(q(y))v(y) = 0
\end{align*}

where in the last line we let $y = -x$. 




\subsection{Coordinate Changes, etc}

We have (hopefully) dealt with the variational problem for the unperturbed case. However, we need to deal with the perturbed case, where we have ICs in $Y^\pm$ (the $\beta^\pm$) and $\lambda \neq 0$ (but small). We have to deal with this separately on $R^\pm$. Thus we have

\begin{align}
V_+' &= A(q^+(0, \beta^+)(x); \lambda) V_+ && x \geq 0 \label{eig:V+} \\
W_+' &= -A(q^+(0, \beta^+)(x); \lambda)^* W_+ && x \geq 0\label{eig:W+} \\
V_-' &= A(q^-(0, \beta^-)(x); \lambda) V_- && x \leq 0 \label{eig:V-} \\
W_-' &= -A(q^-(0, \beta^-)(x); \lambda)^* W_- && x \leq 0 \label{eig:W-}
\end{align}

We can still use Zum2018, i.e. we can find solutions on $\R^+$. Recalling that $\nu(\lambda)$ is the small ``center'' eigenvalue of $A(0; \lambda)$, we have for $\R^+$ solutions

\begin{align}
\tilde{v}_+(x; \beta^+, \lambda) &= e^{\nu(\lambda) x } v_+(x; \beta^+, \lambda) \label{tildev+} \\
\tilde{w}_+(x; \beta^+, \lambda) &= e^{-\overline{\nu(\lambda)} x } w_+(x; \beta^+, \lambda) \label{tildew+} 
\end{align}

where

\begin{align*}
v_+(x; \beta^+, \lambda) &= v_0(\lambda) + \mathcal{O}(e^{-(\tilde{\alpha} + |\nu(\lambda)|)|x|}) \\
w_+(x; \beta^+, \lambda) &= w_0(\lambda) + \mathcal{O}(e^{-(\tilde{\alpha} + |\nu(\lambda)|)|x|})\\
\end{align*}

and solutions on $\R^-$

\begin{align}
\tilde{v}_-(x; \beta^-, \lambda) &= e^{\nu(\lambda) x } v_-(x; \beta^-, \lambda) \label{tildev-} \\
\tilde{w}_-(x; \beta^-, \lambda) &= e^{-\overline{\nu(\lambda)} x } w_-(x; \beta^-, \lambda) \label{tildew-} 
\end{align}

where

\begin{align*}
v_-(x; \beta^-, \lambda) &= v_0(\lambda) + \mathcal{O}(e^{-(\tilde{\alpha} + |\nu(\lambda)|)|x|}) \\
w_-(x; \beta^-, \lambda) &= w_0(\lambda) + \mathcal{O}(e^{-(\tilde{\alpha} + |\nu(\lambda)|)|x|})\\
\end{align*}

(These should really all be capital letters by my notation convention, but I'll deal with that once we get all the stuff to, like, actually work.)\\

Ideal change of coordinates: For sufficiently small $\lambda$ and $\beta^\pm$, change coordinates so that $v_-(0; \beta^-, \lambda)$ is in the same direction as $v_-(0; 0, 0)$ and $v_+(0; \beta^+, \lambda)$ is in the same direction as $v_+(0; 0, 0)$. By what we did above, both $v_-(0; 0, 0)$ and $v_+(0; 0, 0)$ are in the same direction as $\Psi_c(0)$, so the change of coordinates puts all of this stuff into the same 1-dimensional subspace. The idea is that we can completely separate out what happens on this ``center'' subspace from the other stable/unstable subspaces.\\

Less ideal change of coordinates. Same as above, except only for $\lambda$. Since the $\beta^\pm$ perturbations are small (order $e^{-2\alpha X_m}$), we should be able to get away with only doing this, but we will have more small (annoying) remainder terms.

\end{document}