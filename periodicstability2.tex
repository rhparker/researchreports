\documentclass[12pt]{article}
\usepackage[pdfborder={0 0 0.5 [3 2]}]{hyperref}%
\usepackage[left=1in,right=1in,top=1in,bottom=1in]{geometry}%
\usepackage[shortalphabetic]{amsrefs}%
\usepackage{amsmath}
\usepackage{enumerate}
% \usepackage{enumitem}
\usepackage{amssymb}                
\usepackage{amsmath}                
\usepackage{amsfonts}
\usepackage{amsthm}
\usepackage{bbm}
\usepackage[table,xcdraw]{xcolor}
\usepackage{tikz}
\usepackage{float}
\usepackage{booktabs}
\usepackage{svg}
\usepackage{mathtools}
\usepackage{cool}
\usepackage{url}
\usepackage{graphicx,epsfig}
\usepackage{makecell}
\usepackage{array}

\def\noi{\noindent}
\def\T{{\mathbb T}}
\def\R{{\mathbb R}}
\def\N{{\mathbb N}}
\def\C{{\mathbb C}}
\def\Z{{\mathbb Z}}
\def\P{{\mathbb P}}
\def\E{{\mathbb E}}
\def\Q{\mathbb{Q}}
\def\ind{{\mathbb I}}

\DeclareMathOperator{\spn}{span}
\DeclareMathOperator{\ran}{ran}

\graphicspath{ {periodic/} }

\newtheorem{lemma}{Lemma}
\newtheorem{theorem}{Theorem}
\newtheorem{corollary}{Corollary}
\newtheorem{definition}{Definition}
\newtheorem{assumption}{Assumption}
\newtheorem{hypothesis}{Hypothesis}

\newtheorem{notation}{Notation}

\begin{document}

\subsection*{Variational Equation}

Here we look at the variational problem / adjoint variational equation for the linearization of KdV5 about an equilbrium solution $u^*$. This linearization is

\begin{equation}
L(u^*) = \partial_x H(u^*) = \partial_x ( \partial_x^4 - \partial_x^2 + c - 2 u^*)
\end{equation}

The variational / adjoint variational equations for the linearization about the primary pulse solution $q(x)$ are given by

\begin{align}
V' = A(q(x))V \label{vareq} \\
W' = -A(q(x))^*W \label{adjvareq}
\end{align}

where

\begin{align}
A(u^*(x)) = \begin{pmatrix}0 & 1 & 0 & 0 & 0 \\ 0 & 0 & 1 & 0 & 0 \\ 0 & 0 & 0 & 1 & 0 \\ 0 & 0 & 0 & 0 & 1 \\
2u^*_x(x) & 2u^*(x) - c & 0 & 1 & 0 \end{pmatrix}
\end{align}

If we consider $A(0)$ for $c > 1/4$ we have eigenvalues $\nu = \{ 0, \pm \alpha \pm \beta i\}$, where $\alpha, \beta > 0$. The equilibrium at 0 is not hyperbolic, so we cannot directly apply the results of San98. The equilibrium at 0 has 2-dimensional stable/unstable manifolds, and a 1-dimensional center manifold. Let $W^{s/u/c}(0)$ be these manifolds.\\

Let $q(x)$ be the primary pulse (homoclinic orbit) solution, whose existence is known. Since its existence comes from the 4th order (integrated) problem, for which the equilibrim at 0 is hyperbolic, we will still make the same nondegeneracy assumption as before, i.e. we will assume the stable and unstable manifolds have a 1-dimensional intersection.

\begin{hypothesis}\label{nondegen}
\[
T_{Q(0)} W^u(0) \cap T_{Q(0)} W_s(0) = \R Q'(0)
\]
\end{hypothesis}

where $Q(x)$ is the single pulse solution on $\R$, written as a vector-valued function in $R^5$ in the usual way. We cannot conclude from Hypothesis \ref{nondegen} that $Q'(x)$ is the unique bounded solution to the variational equation \eqref{vareq}.\\

Now, we look at the adjoint variational equation. Since $L(q)^* = (\partial_x H(q))^* = -H(q) \partial_x$, we can easily see that there are two bounded solutions to this: $q(x)$ and the constant function $1$. Since we are on a periodic domain, the constant function is periodic and in $L^2$, so we have to consider it here.\\

Thus there are two bounded solutions $\Psi(x) = \nabla H(q(x))$ and $\Psi_c(x)$ of \eqref{adjvareq}, corresponding to $q(x)$ and 1. We computed $\Psi(x)$ previously.

\[
\Psi(x) = \begin{pmatrix}
q^{(4)}(x) - q''(x) + (-2q(x) + c)q(x)\\
-q^{(3)}(x) + q'(x) \\
q''(x) - q(x) \\
-q'(x) \\
q(x)
\end{pmatrix}
\]

All that really matters is that the last component is $q(x)$. We know this decays exponentially to 0, which is also clear upon inspection. Similarly, we can obtain an expression for $\Psi_c(x)$, which will have 1 as the last component.

\[
\Psi_c(x) = (c - 2 q(x), 0, -1, 0, 1)
\]

\subsection{Decomposition of Tangent Space}

Next, as in San98, we will decompose the tangent space at $Q(0)$. First, we define $Y^-$ and $Y^+$ as the ``rest of'' the tangent space of the unstable/stable manifolds.

\begin{align*}
T_{Q(0)} W^u(0) &= \R Q'(0) \oplus Y^- \\
T_{Q(0)} W^s(0) &= \R Q'(0) \oplus Y^+
\end{align*}

Counting dimensions and using Hypothesis \ref{nondegen}, $\R Q'(0) \oplus Y^- \oplus Y^+$ is ``missing'' two dimensions needed to fill out $\R^5$, so we want to find those.\\

Before we do that, we need to look at how the variational equation works. The idea, as far as I know, is that that the variational equation evolves the tangent spaces of the stable/unstable/center manifolds $W^{s/u/c}(0)$. I have not found a good reference on this but it seems to make sense. In any case, let $\Phi(x, y)$ be the evolution operator for the variational equation \eqref{vareq} on $\R$. This, of course, means that $\Phi(x, 0)Y$ is the unique solution to \eqref{vareq} for any IC $Y$given at $x = 0$. In this case, there are no perturbations or parameters. Then we should have the following.

\begin{enumerate}
	\item $Q'(x)$ is a solution to \eqref{vareq}, so we know that $Q'(x) = \Phi(x, 0)Q'(0)$.
	\item For an IC $Y \in Y^-$ and $x \leq 0$, $|\Phi(x, 0) Y| \leq C e^{\alpha x}$. The decay rate $\alpha$ is the known decay rate of the unstable/stable manifold of the original (nonlinear) equation.
	\item For an IC $Y \in Y^+$ and $x \geq 0$, $|\Phi(x, 0) Y| \leq C e^{-\alpha x}$. 
\end{enumerate}

We showed before that

\begin{enumerate}[(i)]
\item $\langle V(x), W(x) \rangle$ is constant in $x$ for any solutions $V(x)$ to \eqref{vareq} and $W(x)$ to \eqref{adjvareq}.
\item $\Phi(x, y)^*$ is the evolution operator for \eqref{adjvareq}.
\end{enumerate}

Since the inner product is continuous in both components, it follows that if as $x \rightarrow \infty$ (or $-\infty$), if either $V(x)$ or $W(x)$ decays exponentially and the other one is bounded (it could decay too of course), then $\langle V(0), W(0) \rangle = 0$, i.e. $V(0) \perp W(0)$. Choosing the appropriate things for $V(x)$ and either $\Psi(x)$ or $\Psi_c(x)$ for $W(x)$ we conclude that

\begin{align*}
\Psi(0) &\perp Q'(0) \oplus Y^- \oplus Y^+ \\
\Psi_c(0) &\perp Q'(0) \oplus Y^- \oplus Y^+
\end{align*}

$\Psi(0)$ and $\Psi_c(0)$ are linearly independent, but we do not expect them to be perpendicular. We can, if we like, compute the inner product $\langle \Psi(0), \Psi_c(0) \rangle$, and it is nonzero.\\

Let $Y^0 = (\Psi(0) \oplus \perp Q'(0) \oplus Y^- \oplus Y^+)^\perp$, which is 1-dimensional. Then 

\[
\R^5 = \R Q'(0) \oplus Y^- \oplus Y^+ \oplus \R \Psi(0) \oplus Y^0
\]

This is not an orthogonal system, since $Q'(0)$, $Y^+$, and $Y^-$ are not necessarily mutually perpendicular, but it should be good enough.\\

It is worth noting that

\[
(\R \Psi(0))^\perp = \R Q'(0) \oplus Y^- \oplus Y^+ \oplus Y^0
\]

\subsection*{Center Subspace}

Now we use the Gap Lemma from Zum2018. Taking $\lambda = 0$ (and the parameters $\beta_i^\pm = 0$), we can find solutions $V^\pm(x)$ and $W^\pm(x)$ to \eqref{vareq} and \eqref{adjvareq} on $\R^\pm$ which are given by

\begin{align*}
V^\pm(x) = V_0 + \mathcal{O}(e^{-(\alpha - \epsilon)|x|}|V_0|) \\
W^\pm(x) = W_0 + \mathcal{O}(e^{-(\alpha - \epsilon)|x|}|W_0|)
\end{align*}

where the decay is in the appropriate direction. $V_0$ and $W_0$ are the eigenvectors of $A(0)$ and $A(0)^*$ corresponding to the eigenvalue 0. In this case, we have $V_0 = (1, 0, 0, 0, 0)$ and $W_0 = (c, 0, -1, 0, 1)$. The constant $\epsilon > 0$ is arbitrary. Since the inner product $\langle V^\pm(x), W^\pm(x) \rangle$ is constant and the inner product is continuous in both arguments, from the above expressions we have for all (appropriate) $x$

\[
\langle V^\pm(x), W^\pm(x) \rangle = \langle V_0, W_0 \rangle = c \neq 0
\]

In particular, $\langle V^\pm(0), W^\pm(0) \rangle \neq 0$. We would like to know where $V^\pm(0)$ and $W^\pm(0)$ fit into the overall scheme above. \\

\subsubsection*{Adjoint Center Solution}

Looking at the expression for $\Psi_c(x)$, it has the same decay properties as $W^\pm(x)$, and noth it and $W^\pm(x)$ decay (in the appropriate direction) to the constant vector $W_0$. Thus we can take $W^\pm(x) = \Psi_c(x)$, so from now on, we will do that. Thus we have

\[
\langle V^\pm(0), \Psi_c(0) \rangle = \langle V_0, W_0 \rangle = c \neq 0
\]

\subsubsection*{Variational Center Solution}

We now turn our attention to $V^\pm(x)$. We look at $V^+(x)$ first. Since $V^+(x)$ is bounded and $\Psi(x)$ decays exponentially as $x \rightarrow \infty$, we have $V^+(0) \perp \Psi(0)$, i.e. 

\[
V^+(0) \in (\R \Psi(0))^\perp = \R Q'(0) \oplus Y^- \oplus Y^+ \oplus Y^0
\]

Thus we can write 

\[
V^+(0) = y^0 + y^- + y^+ + x^+
\]

where $y^0 \in Y^0, y^- \in Y^+, y^- \in Y^-, x^+ \in \R Q'(0)$. Since we can pick whatever we want as an IC, let's use the IC $V^+(0) - y^+ - x^+ = y^0 + y^-$. Since the variational equation is linear,

\begin{align*}
\Phi(x, 0)( V^+(0) - y^+ - x^+ ) &= \Phi(x, 0)V^+(0) - \Phi(x, 0) y^+ - \Phi(x, 0) x^+ \\
&= V_0 + \mathcal{O}(e^{-(\alpha - \epsilon)x}|V_0|) + \mathcal{O}(e^{-\alpha x}|y^+| + 
\mathcal{O}(e^{-\alpha x}|x^+|) \\
&= V_0 + \mathcal{O}(e^{-(\alpha - \epsilon)x})
\end{align*}

Using this IC, we still have a solution with the property we want. Let $\tilde{V}^+(x)$ be be solution to \eqref{vareq} with IC $V^+(0) - y^+ - x^+ = y^- + y^0$, i.e. with $\tilde{V}^+(0) \in Y^- \oplus Y^0$. We can do the same thing with $V^-(x)$ to get a solution $\tilde{V}^-(x)$ to \eqref{vareq} with $\tilde{V}^-(0) \in Y^+ \oplus Y^0$. Thus we have solutions to \eqref{vareq}

\begin{align*}
\tilde{V}^+(x) &= V_0 + \mathcal{O}(e^{-(\alpha - \epsilon)x}) && \tilde{V}^+(0) \in Y^0 \oplus Y^-, x \geq 0 \\
\tilde{V}^-(x) &= V_0 + \mathcal{O}(e^{(\alpha - \epsilon)x}) && \tilde{V}^-(0) \in Y^0 \oplus Y^+, x \geq 0
\end{align*}

To match these things up at $x = 0$, all we need to do is show that their initial conditions can have no component in $Y^\pm$. But this must be the case. To see that (in a handwavey fashion), suppose $\tilde{V}^+(0) = y^0 + y^-$, where $y^0 \in Y^0$ and $y^- \in Y^-$. Since $Y^-$ is the tangent space of the unstable manifold $W^u(0)$, ICs there will blow up (exponentially) as $x \rightarrow \infty$; or, at least, as $x \rightarrow \infty$, $\Phi(x,0)y^-$ will increase in magnitude and give us in the limit a component in $E^u$, which we know from above cannot happen. Thus $\tilde{V}^+(0)$ cannot have a component in $Y^-$. Similarly, $\tilde{V}^-(0)$ cannot have a component in $Y^+$. Thus we have solutions 

\begin{align*}
\tilde{V}^+(x) &= V_0 + \mathcal{O}(e^{-(\alpha - \epsilon)x}) && \tilde{V}^+(0) \in Y_0, x \geq 0 \\
\tilde{V}^-(x) &= V_0 + \mathcal{O}(e^{(\alpha - \epsilon)x}) && \tilde{V}^-(0) \in Y_0, x \geq 0
\end{align*}

We are now done, since these two solutions behave the way we want, and at $x = 0$, both $\tilde{V}^+(0)$ and $\tilde{V}^-(0)$ are contained in $Y^0$. This does not mean that $\tilde{V}^+(0) = \tilde{V}^-(0)$; we just know they are both contained in $Y^0$. Since this space is 1-dimensional, these must be scalar multiples of each other.\\

We would like to show that they are in fact equal. To do this, choose any initial condition $y^0 \in Y_0$ (say, a unit vector in that space). Then by what we have shown above, $V(x) = \Phi(x, 0)y^0$ is a bounded solution to \eqref{vareq} on $\R$, and its limits at $\pm \infty$ are scalar multiples of $V_0$, which may (in general) be different.\\

We will use a reversibility argument here. To see this, it is easiest to take this out of the 5th order system and write the variational equation as

\[
L(q)v = \partial_x H(q)v = \partial_x ( \partial_x^4 - \partial_x^2 + c - 2 q)v = 
(\partial_x^5 - \partial_x^3 + c \partial_x - 2 q \partial_x - 2 q_x)v
\]

Suppose $v(x)$ is a solution to this. Then it is not hard to show that so is $v(-x)$. Since the primary pulse $q(x)$ is even,

\begin{align*}
L(q(x))v(-x) &= -v^{(5)}(-x) + v'''(-x) - c v'(-x) + 2 q(x)v'(-x) - 2 q_x(x)v(-x) \\
&= -( v^{(5)}(-x) - v'''(-x) + c v'(-x) - 2 q(x)v'(-x) + 2 q_x(x)v(-x) ) \\
&= -( v^{(5)}(-x) - v'''(-x) + c v'(-x) - 2 q(-x)v'(-x) - 2 q_x(-x)v(-x) ) \\
&= -L(q(-x))v(-x) \\
&= -L(q(y))v(y) = 0
\end{align*}

where in the last line we let $y = -x$. 




\subsection{Coordinate Changes, etc}

We have (hopefully) dealt with the variational problem for the unperturbed case. However, we need to deal with the perturbed case, where we have ICs in $Y^\pm$ (the $\beta^\pm$) and $\lambda \neq 0$ (but small). We have to deal with this separately on $R^\pm$. Thus we have

\begin{align}
V_+' &= A(q^+(0, \beta^+)(x); \lambda) V_+ && x \geq 0 \label{eig:V+} \\
W_+' &= -A(q^+(0, \beta^+)(x); \lambda)^* W_+ && x \geq 0\label{eig:W+} \\
V_-' &= A(q^-(0, \beta^-)(x); \lambda) V_- && x \leq 0 \label{eig:V-} \\
W_-' &= -A(q^-(0, \beta^-)(x); \lambda)^* W_- && x \leq 0 \label{eig:W-}
\end{align}

We can still use Zum2018, i.e. we can find solutions on $\R^+$. Recalling that $\nu(\lambda)$ is the small ``center'' eigenvalue of $A(0; \lambda)$, we have for $\R^+$ solutions

\begin{align}
\tilde{v}_+(x; \beta^+, \lambda) &= e^{\nu(\lambda) x } v_+(x; \beta^+, \lambda) \label{tildev+} \\
\tilde{w}_+(x; \beta^+, \lambda) &= e^{-\overline{\nu(\lambda)} x } w_+(x; \beta^+, \lambda) \label{tildew+} 
\end{align}

where

\begin{align*}
v_+(x; \beta^+, \lambda) &= v_0(\lambda) + \mathcal{O}(e^{-(\tilde{\alpha} + |\nu(\lambda)|)|x|}) \\
w_+(x; \beta^+, \lambda) &= w_0(\lambda) + \mathcal{O}(e^{-(\tilde{\alpha} + |\nu(\lambda)|)|x|})\\
\end{align*}

and solutions on $\R^-$

\begin{align}
\tilde{v}_-(x; \beta^-, \lambda) &= e^{\nu(\lambda) x } v_-(x; \beta^-, \lambda) \label{tildev-} \\
\tilde{w}_-(x; \beta^-, \lambda) &= e^{-\overline{\nu(\lambda)} x } w_-(x; \beta^-, \lambda) \label{tildew-} 
\end{align}

where

\begin{align*}
v_-(x; \beta^-, \lambda) &= v_0(\lambda) + \mathcal{O}(e^{-(\tilde{\alpha} + |\nu(\lambda)|)|x|}) \\
w_-(x; \beta^-, \lambda) &= w_0(\lambda) + \mathcal{O}(e^{-(\tilde{\alpha} + |\nu(\lambda)|)|x|})\\
\end{align*}

(These should really all be capital letters by my notation convention, but I'll deal with that once we get all the stuff to, like, actually work.)\\

Ideal change of coordinates: For sufficiently small $\lambda$ and $\beta^\pm$, change coordinates so that $v_-(0; \beta^-, \lambda)$ is in the same direction as $v_-(0; 0, 0)$ and $v_+(0; \beta^+, \lambda)$ is in the same direction as $v_+(0; 0, 0)$. By what we did above, both $v_-(0; 0, 0)$ and $v_+(0; 0, 0)$ are in the same direction as $\Psi_c(0)$, so the change of coordinates puts all of this stuff into the same 1-dimensional subspace. The idea is that we can completely separate out what happens on this ``center'' subspace from the other stable/unstable subspaces.\\

Less ideal change of coordinates. Same as above, except only for $\lambda$. Since the $\beta^\pm$ perturbations are small (order $e^{-2\alpha X_m}$), we should be able to get away with only doing this, but we will have more small (annoying) remainder terms.

\end{document}