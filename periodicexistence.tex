% \documentclass{book}

\documentclass[12pt]{article}
\usepackage[pdfborder={0 0 0.5 [3 2]}]{hyperref}%
\usepackage[left=1in,right=1in,top=1in,bottom=1in]{geometry}%
% \usepackage[shortalphabetic]{amsrefs}%
\usepackage{amsmath}
\usepackage{enumerate}
\usepackage{enumitem}
\usepackage{amssymb}               
\usepackage{amsfonts}
\usepackage{amsthm}
\usepackage{bbm}
\usepackage[table,xcdraw]{xcolor}
\usepackage{tikz}
\usepackage{float}
\usepackage{booktabs}
\usepackage{svg}
\usepackage{mathtools}
\usepackage{cool}
\usepackage{url}
\usepackage{graphicx,epsfig}
\usepackage{makecell}
\usepackage{array}

\def\noi{\noindent}
\def\T{{\mathbb T}}
\def\R{{\mathbb R}}
\def\N{{\mathbb N}}
\def\C{{\mathbb C}}
\def\Z{{\mathbb Z}}
\def\P{{\mathbb P}}
\def\E{{\mathbb E}}
\def\Q{\mathbb{Q}}
\def\ind{{\mathbb I}}

\newtheorem{lemma}{Lemma}
\newtheorem{theorem}{Theorem}
\newtheorem{proposition}{Proposition}
\newtheorem{corollary}{Corollary}
\newtheorem{definition}{Definition}
\newtheorem{assumption}{Assumption}
\newtheorem{hypothesis}{Hypothesis}

\DeclareMathOperator{\spn}{span}
\renewcommand{\vec}[1]{\mathbf{#1}}

\graphicspath{ {periodic/} }

\begin{document}

\section{Periodic Pulses and Multipulses}

We are interested in proving existence of periodic, multipulse solutions to KdV5 together with estimates sufficient to use for the stability problem. This method (or attempt) is based on San97.\\

Since we are only intererested in the existence problem here, we can use the 4th order (integrated) equation

\begin{equation}\label{4thorder}
f(u) = u_{xxxx} - u_{xx} + c u - u^2 = 0
\end{equation}

We know that single pulse solutions exist for $c > 0$ and that multipulse solutions exist for $c > 1/4$. Choose $c > 1/4$, and let $q(x)$ be the corresponding single pulse solution. (We could write this as $q(x; c)$, but we suppress the dependence on $c$ for convenience. Write this as a first order system as

\[
U'(x) = F(U) = LU + N(U)
\]

where $N$ is strictly nonlinear. Let $A$ be the linearization of \eqref{4thorder} about the equilibrium at 0 (which persists for all values of $c$). Then we know $A$ is hyperbolic, has two stable and two unstable directions, and the magnitude of the real part of all of them is given by $\alpha$. 
\\

What we want to do now is paramaterize the stable and unstable manifolds near $q(0)$. To do that we first write

\begin{align*}
T_{q(0)}W^u(0) &= \R Q'(0) \oplus Y^- \\
T_{q(0)}W^s(0) &= \R Q'(0) \oplus Y^+ \\
Z &= \R \Psi(0)
\end{align*}

These four things are one-dimensional and span $\R^4$, and we have shown before that $Z$ is perpendicular to the other three. \\

In San97, the ODE under consideration is parameterized by $\mu$. The only parameter we have available here is the speed $c$, so we will go ahead and use that as our parameter. This is a little weird, since we would like this to work for all $c > 0$ (or $c > 1/4$ for multipulses), rather than assuming there is a homoclinic orbit for a speed $c_0$ and then proving that a periodic pulse (wavetrain) exists for a nearby $c$. That being said, we will do it this way for now, and see what happens.\\

We now write $W^u(0)$ and $W^s(0)$ as graphs over their own tangent spaces near $Q(0)$. Following San97, we can paramaterize the unstable and stable manifolds near $Q(0)$ by

\begin{align*}
Q^-(\alpha, \beta^-, c) \\
Q^+(\alpha, \beta^+, c)
\end{align*}

where $\alpha \in \R$, $\beta^\pm \in Y^\pm$ and we have set things up so that $Q^+(\alpha, \beta^+, c) - Q^-(\alpha, \beta^-, c) \in Z$. \\

Since, in our case, the homoclinic orbit persists for all $c > 0$ (this has been shown), we should have for all $c > 0$

\[
Q^-(0, 0, c) = Q^+(0, 0, c)
\]

though it is hard to say at this point if this will be useful or not.\\

To construct a periodic single pulse, we want to solve the following periodic BVP for some $X > 0$

\begin{align*}
U' - LU - N(U) &= 0 \\
U(X) - U(-X) &= 0 
\end{align*}

Since for $c > 1/4$ we have ``twisting manifolds'', $X$ may be restricted to be approximately an integer multiple of some phase parameter.\\

Using the parameterization of the stable and unstable manifolds, we want to then solve the problem for piecewise $U$ given by

\begin{align*}
U^-(x) &= Q^-(\alpha, \beta^-, c)(x) + V-^(x) \\
U^+(x) &= Q^+(\alpha, \beta^+, c)(x) + V-+(x)
\end{align*}

where $Q^-(\alpha, \beta^-, c)(x)$ is the unique solution to our ODE with IC $Q^-(\alpha, \beta^-, c)$ for $x \leq 0$, and similar for the other one. We will choose $V^\pm(x)$ so that

\begin{align*}
V^-(0) &\in Z \oplus Y^- \\
V^+(0) &\in Z \oplus Y^+
\end{align*}

since the other directions are covered by the $\alpha$ and $\beta^\pm$.\\

Much of Lemma 4.1 in San97 should hold as written. However, the Melnikov integral in (H2) will be 0. To see this, take any $c > 0$. Then the (standard) Melnikov integral is

\begin{align*}
M &= \int_{-\infty}^\infty \langle \psi(x), D_c f(q(x), c) \rangle dx \\
&= \int_{-\infty}^\infty \langle \psi(x), q(x) \rangle dx \\
&= \int_{-\infty}^\infty \langle q'(x), q(x) \rangle dx \\
&= 0
\end{align*}

In the penultimate line, we used the following facts

\begin{enumerate}

\item The linearization of $f$ is self-adjoint, thus the solutions to the variational equation and the adjoint variational equation are identical.
\item Thus $\psi(x) = q'(x)$, since $q'(x)$ solves the variational equation.
\item $q(x)$ is even and $q'(x)$ is odd
\item For $D_c f(q, c)$ we have
\begin{align*}
D_c f(q, c) &= q_{xxxxc} - q_{xxc} + c q_c + q - 2qq_c \\
&= \frac{\partial}{\partial c}(q_{xxxx} - q_{xx} + cq - q^2) + q \\
&= q
\end{align*}

\end{enumerate}

Thus we will need a version of Lemma 4.1 that uses our higher order Melnikov integral, which is nonzero by hypothesis. \\

We will return to the problem, and see where this comes up, which should hopefully give us an idea of what to do.\\

For convenience, we take 

\[
\gamma^\pm = (\alpha, \beta^\pm, c)
\]

and we let $\Phi_\pm(c^\pm, t, s)$ be the evolution operator for the variational equation

\begin{equation}
V' = A(Q^\pm(\alpha, \beta^\pm, c)(x), c) V
\end{equation}

For convenience, let

\begin{align*}
\Phi^s_\pm(\gamma, t, s) &= \Phi_\pm(\gamma, t, s) P^s_\pm(\gamma, s) \\
\Phi^u_\pm(\gamma, t, s) &= \Phi_\pm(\gamma, t, s) P^u_\pm(\gamma, s) 
\end{align*}

The estimates in Lemma 5.1 in San97 are similar to those we have been using this entire time, and should hold as written.\\

Then, plugging in our piecewise expression for $U$, $U^\pm$ solves the ODE $U' = F(U, c)$ if and only if $V\pm$ solves

\begin{align*}
(V\pm)' = A(Q^\pm(\gamma^\pm)(x), c)V^\pm + G^\pm(x, V^\pm, \gamma^\pm) 
\end{align*}

where $G = \mathcal{O}(|V|^2)$ (Taylor).\\

We then write the ODE as a piecewise integral equation on the intervals $[-X, 0]$ and $[0, X]$. This is pretty much the same thing we always do. The IC is taken 

\begin{align*}
V^+(x) &= \Phi^u_+(\gamma^+, x, X) a^+ + \Phi^s_+(\gamma^+, x, 0) b^+ \\
&+ \int_{-X}^x \Phi_+^u(\gamma^+, x, y) G^+(y, V^+(y),\gamma^+)dy \\
&+ \int_0^x \Phi_+^s(\gamma^+, x, y) G^+(y, V^+(y),\gamma^+)dy \\ 
V^-(x) &= \Phi^s_-(\gamma^-, x, -X) a^- + \Phi^u_-(\gamma^-, x, 0) b^- \\
&+ \int_X^x \Phi_-^s(\gamma^-, x, y) G^-(y, V^-(y),\gamma^-)dy \\
&+ \int_0^x \Phi_-^u(\gamma^-, x, y) G^-(y, V^-(y),\gamma^-)dy \\
\end{align*}

where for the ICs on the two parts of the exponential dichotomy we have

\begin{align*}
a^+ &\in E_0^u \\
a^- &\in E_0^s \\
\end{align*}

We also need to specify where the ICs $b^\pm$ live. Intuitively I would take $b^\pm \in Y^\pm \oplus Z$, which would mean the projection at $x = 0$ is the identity. However, since these disappear in (5.8) in San97, it looks like we might want to take them to be a space which is wiped out by the appropriate projection. To accomplish this, we could take $b^\pm \in Y^\mp \oplus Z$. For now, we will do that since it gets rid of something, but maybe we want to keep the $b^\pm$ around to do the matching at $x = 0$.\\

Thus we have

\begin{align*}
V^+(x) &= \Phi^u_+(\gamma^+, x, X) a^+  \\
&+ \int_{-X}^x \Phi_+^u(\gamma^+, x, y) G^+(y, V^+(y),\gamma^+)dy \\
&+ \int_0^x \Phi_+^s(\gamma^+, x, y) G^+(y, V^+(y),\gamma^+)dy \\ 
V^-(x) &= \Phi^s_-(\gamma^-, x, -X) a^-  \\
&+ \int_X^x \Phi_-^s(\gamma^-, x, y) G^-(y, V^-(y),\gamma^-)dy \\
&+ \int_0^x \Phi_-^u(\gamma^-, x, y) G^-(y, V^-(y),\gamma^-)dy \\
\end{align*}

Define the exponentially weighted norms

\begin{align*}
||V||_+ = sup_{x \in [0, X]} e^{\alpha(X - x)}|V(x)| \\
||V||_- = sup_{x \in [-X, 0]} e^{\alpha(x + X)}|V(x)| \\
\end{align*}

and let $C^\pm$ be the spaces equipped with these norms. These are known to be Banach spaces. Let $B_\rho^\pm$ be the ball of radius $\rho$ in these norms about the zero function.\\

In Lemma 5.2 in San97, we use the IFT to solve for $V^\pm$ in terms of the ICs $a^\pm$. This should work as it does there, so for now assume we can do this, and we can fill in details as needed. Thus we have the estimate

\[
||V^\pm||_\pm \leq C |a^\pm|
\]

Now we look at the matching BC at $\pm X$. Recall that the matching BC is for $U$ not $V$. Substituting in the fixed point equations we get

\begin{align*}
U^+(X) &= Q^+(\gamma^+)(X) + P^u_+(\gamma^+, X) a^+ + \int_0^X \Phi_+^s(\gamma^+, x, y) G^+(y, V^+(y),\gamma^+)dy \\ 
U^-(-X) &= Q^-(\gamma^-)(-X) + P^s_-(\gamma^-, -X) a^- + \int_0^{-X} \Phi_-^u(\gamma^-, x, y) G^-(y, V^-(y),\gamma^-)dy \\
\end{align*}

Adding and subtracting $P_0^{s/u}$ and recalling where the $a_i$ live, this becomes

\begin{align*}
U^+(X) &= Q^+(\gamma^+)(X) + a^+ + (P^u_+(\gamma^+, X) - P^u_0) a^+ + \int_0^X \Phi_+^s(\gamma^+, x, y) G^+(y, V^+(y),\gamma^+)dy \\ 
U^-(-X) &= Q^-(\gamma^-)(-X) + a^- + (P^s_-(\gamma^-, -X) - P^s_0) a^- + \int_0^{-X} \Phi_-^u(\gamma^-, x, y) G^-(y, V^-(y),\gamma^-)dy \\
\end{align*}

Now set these equal and solve for $a^+ + a^-$, which then gives us $a^\pm$ since $E^s$ and $E^u$ are linearly independendent. Our estimates should be good enough to do this, so we be able to solve for the $a^\pm$ in terms of the $c^\pm$ with estimates

\begin{align*}
|a^\pm| \leq C |Q^\pm(\gamma^\pm)(\pm X)| \leq C e^{-\alpha X}
\end{align*}

To get a solution to the full problem, we need to match our two pieces of $U$ at $X = 0$. At $x = 0$, we have

\begin{align*}
U^+(0) &= Q^+(\gamma^+) + V^+(0) \\
U^-(0) &= Q^-(\gamma^-) + V^-(0)
\end{align*}

which we can write out as

\begin{align*}
U^+(0) &= Q^+(\gamma^+) + \Phi^u_+(\gamma^+, 0, X) a^+ + \int_{-X}^0 \Phi_+^u(\gamma^+, 0, y) G^+(y, V^+(y),\gamma^+)dy \\
U^-(0) &= Q^-(\gamma^-) + \Phi^s_-(\gamma^-, 0, -X) a^- - \int_0^X \Phi_-^s(\gamma^-, 0, y) G^-(y, V^-(y),\gamma^-)dy \\
\end{align*}

We can solve $U^+(0) - U^-(0)$ by taking projections on the pieces of $\R^4 = \R Q'(0) \oplus Z \oplus Y^- \oplus Y^+$. Since we have chosen $V^\pm(0) \in Z \oplus Y^\pm$, we should be able to show that the projection on $\R Q'^(0)$ is automatically 0. Thus it suffices to consider the other ones.\\

The projection on $Z = \R \Psi(0)$ should give us the Melnikov integral, so let's make sure that will actually happen, especially since the lower order Melnikov integral is 0.\\

From what I understand, the $P_Z( Q^+(\gamma^+) - Q^-(\gamma^-) )$ term should be the one that gives us the Melnikov integral, although I am not sure how at this time.



\end{document}