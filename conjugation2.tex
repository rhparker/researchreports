\documentclass[12pt]{article}
\usepackage[pdfborder={0 0 0.5 [3 2]}]{hyperref}%
\usepackage[left=1in,right=1in,top=1in,bottom=1in]{geometry}%
\usepackage[shortalphabetic]{amsrefs}%
\usepackage{amsmath}
\usepackage{enumerate}
% \usepackage{enumitem}
\usepackage{amssymb}                
\usepackage{amsmath}                
\usepackage{amsfonts}
\usepackage{amsthm}
\usepackage{bbm}
\usepackage[table,xcdraw]{xcolor}
\usepackage{tikz}
\usepackage{float}
\usepackage{booktabs}
\usepackage{svg}
\usepackage{mathtools}
\usepackage{cool}
\usepackage{url}
\usepackage{graphicx,epsfig}
\usepackage{makecell}
\usepackage{array}

\def\noi{\noindent}
\def\T{{\mathbb T}}
\def\R{{\mathbb R}}
\def\N{{\mathbb N}}
\def\C{{\mathbb C}}
\def\Z{{\mathbb Z}}
\def\P{{\mathbb P}}
\def\E{{\mathbb E}}
\def\Q{\mathbb{Q}}
\def\ind{{\mathbb I}}

\DeclareMathOperator{\spn}{span}
\DeclareMathOperator{\ran}{ran}

\graphicspath{ {periodic/} }

\newtheorem{lemma}{Lemma}
\newtheorem{theorem}{Theorem}
\newtheorem{corollary}{Corollary}
\newtheorem{definition}{Definition}
\newtheorem{assumption}{Assumption}
\newtheorem{hypothesis}{Hypothesis}

\newtheorem{notation}{Notation}

\begin{document}

\section{Stability of Periodic Multi-Pulse Solutions}

\subsection{Background and Eigenvalue Problem}

In a previous section, we showed that a periodic $n-$pulse solution $q_{np}(x)$ to KdV5 exists, where the distances between the peaks are given by the $n$ lengths $X_0, \dots, X_{n-1}$. We can write $q_{np}(x)$ piecewise as

\[
q_i^\pm(x) = q^\pm(x; \beta_i^\pm) + u_i^\pm(x)
\]

on the $2n$ intervals 

\begin{align*}
\{ [-X_{i-1}, 0], [0, X_i] \} && i = 0, \dots, n-1
\end{align*}

where the subscript $i$ is $\mod n$, since we are in a periodic domain. The distances between consecutive peaks in $q_{np}(x)$ are $2 X_0, \dots, 2 X_{n-1}$.\\

The functions $q^\pm(x; \beta_i^\pm)$ evolve in the $W^s(0)$ and $W^u(0)$, with initial conditions $\beta_i^\pm$. The functions $u_i^\pm(x)$ are small remainder terms. From the existence problem, we have bounds on all of these terms.

\begin{align*}
|q^\pm(x; \beta_i^\pm)| &\leq C |\beta_i^\pm| e^{-\alpha |x|} \\
|\beta_i^\pm| &\leq C (e^{-2 \alpha X_i} + e^{-2 \alpha X_{i-1}}) \\
|u_i^-(x)| &\leq C e^{-\alpha X_{i-1}} e^{-\alpha(X_{i-1} + x) } \\
|u_i^+(x)| &\leq C e^{-\alpha X_i} e^{-\alpha(X_i - x) } 
\end{align*}

Our goal is to determine the linear stability of periodic multi-pulse solutions to KdV5. To do this, we look at the PDE eigenvalue problem resulting from the linearization of KdV5 about an the periodic multi-pulse $q_{np}(x)$.\\

Recall that the linearization of KdV5 about an equilibrium solution $q(x)$ is given by the linear operator $L(q)$.

\begin{equation}
L(q) = \partial_x^5 - \partial_x^3 + c \partial_x
 - 2 q \partial_x - 2 q_x
\end{equation}

We can write this operator as $L = \partial_x H(q)$, where $H(q)$ is the self-adjoint operator

\begin{equation}
H(q) = \partial_x^4 - \partial_x^2 + c - 2 q
\end{equation}

Thus for the adjoint operator $L(q)^*$, we have

\begin{equation}
L(q)^* = -H(q) \partial_x
\end{equation}

It is not hard to show that

\begin{align*}
L(q)q_x &= 0 \\
L(q)(-\partial_c q) &= q_x \\
L(q)^* q &= 0
\end{align*}

In addition, for a finite or periodic domain, the constant functions are in the kernel of $L(q)^*$.

\[
L(q)^* 1 = 0
\]

For the primary pulse solution $q(x)$ to KdV5, we will assume that that $L(q)$ has a one-dimensional kernel which is spanned by $q_x$. If there were another generalized eigenfunction in the kernel of $L(q)$, we would be able to solve the equation $L(q) v = \partial_c q$, which is only possible if $\partial_c q \perp \ker L(q)^*$. In particular, this would require $\partial_c q \perp q$. We will assume this is not the case, i.e. we take the following hypothesis.

\begin{hypothesis}\label{Melnikovnonzero}
The following higher order Melnikov integral is nonzero.
\[
M = \langle q, \partial_c q \rangle_{L^2(\R)} 
= \int_{-\infty}^\infty q(x) \partial_c q(x) dx \neq 0
\]
\end{hypothesis}

Returning to the linearization about the periodic multipulse $q_{np}(x)$, let $V = (v, v_x, v_{xx}, v_{xxx}, v_{xxxx})^T$, and write the eigenvalue problem $L(q_{np}(x))v(x) = \lambda v(x)$ as the first-order system

\begin{align*}
V'(x) = A(q_{np}(x)) V(x) + \lambda B V(x)
\end{align*}

where

\begin{align*}
A(q(x)) &= \begin{pmatrix}0 & 1 & 0 & 0 & 0 \\0 & 0 & 1 & 0 & 0 \\0 & 0 & 0 & 1 & 0 \\0 & 0 & 0 & 0 & 1 \\
2 \partial_x q(x) + \lambda & 2 q(x) - c & 0 & 1 & 0 \end{pmatrix}, &&
B = \begin{pmatrix}0 & 0 & 0 & 0 & 0 \\0 & 0 & 0 & 0 & 0 \\0  & 0 & 0 & 0 & 0 \\0 & 0 & 0 & 0 & 0 \\1 & 0 & 0 & 0 & 0 \end{pmatrix} 
\end{align*}

\subsection{Variational Equation}

Consider the variational and adjoint variational equations for the linearization about the primary pulse solution $q(x)$ to KdV5. These are given by

\begin{align}
V' = A(q(x))V \label{vareq} \\
W' = -A(q(x))^*W \label{adjvareq}
\end{align}

Note that $A(q(x))$ is exponentially asymptotic to the constant-coefficient matrix $A(0)$, which is given by

\begin{align}\label{A0}
A(0) = \begin{pmatrix}0 & 1 & 0 & 0 & 0 \\ 0 & 0 & 1 & 0 & 0 \\ 0 & 0 & 0 & 1 & 0 \\ 0 & 0 & 0 & 0 & 1 \\
0 & -c & 0 & 1 & 0 
\end{pmatrix}
\end{align}

For $c > 1/4$, $A(0)$ has eigenvalues $\nu = \{ 0, \pm \alpha_0 \pm \beta_0 i\}$, where $\alpha_0, \beta_0 > 0$. The eigenvectors of $A(0)$ and $-A(0)^*$ corresponding to the eigenvalue 0 are $V_0$ and $W_0$ (respectively), which are given by

\begin{align*}
V_0 &= (1/c, 0, 0, 0, 0)^T \\
W_0 &= (c, 0, -1, 0, 1)^T 
\end{align*}

where we have scaled $V_0$ so that $\langle W_0, V_0 \rangle = 1$.\\

Since $A(0)$ is not hyperbolic, we cannot directly apply the results of San98. Thus the equilibrium at 0 has 2-dimensional stable/unstable manifolds, and a 1-dimensional center manifold. Let $W^{s/u/c}(0)$ be these manifolds.\\

Since $L(q)q_x = 0$ and $L(q)(-\partial_c q) q_x$, we have the expressions

\begin{align*}
(Q')' &= A(q) Q' \\
(\partial_c Q)' &= A(q) (\partial_c Q) + B Q'
\end{align*}

For the adjoint variational problem, we have an exponentially decaying solution $\Psi(x)$, which is given by

\begin{equation}\label{Psi}
\Psi(x) = \begin{pmatrix}
q^{(4)}(x) - q''(x) + (-2q(x) + c)q(x)\\
-q^{(3)}(x) + q'(x) \\
q''(x) - q(x) \\
-q'(x) \\
q(x)
\end{pmatrix}
\end{equation}

For a bounded or periodic domain, we also have a solution $\Psi^c(x)$ to the adjoint variational problem which is bounded but does not decay exponentially.

\begin{align*}
\Psi^c(x) = (c - 2 q(x), 0, -1, 0, 1)^T
\end{align*}

In fact, $\Psi^c(x) \rightarrow W_0$ as $x \rightarrow \pm \infty$.

We will now decompose the tangent space at $Q(0)$. First, we make the following non-degeneracy assumption.

\begin{hypothesis}\label{nondegen}
\[
T_{Q(0)} W^u(0) \cap T_{Q(0)} W_s(0) = \R Q'(0)
\]
\end{hypothesis}

where $Q(x)$ is the single pulse solution on $\R$, written as a vector-valued function in $R^5$. Next, we define $Y^-$ and $Y^+$ to be the remaining dimensions of the tangent space of the unstable/stable manifolds.

\begin{align*}
T_{Q(0)} W^u(0) &= \R Q'(0) \oplus Y^- \\
T_{Q(0)} W^s(0) &= \R Q'(0) \oplus Y^+
\end{align*}

So far $\text{dim }\R Q'(0) \oplus Y^- \oplus Y^+ = 3$. To fill out the remaining 2 dimensions, we look at solutions to the adjoint variational equation.\\

First, we summarize some useful facts about the variational equation in the following lemma. We define the inner product on $\C^n$ by $\langle x, y \rangle = \sum_i x_i \bar{y_i}$, i.e. the complex conjugation is on the second component.

% lemma : facts about our eigenvalue problem

\begin{lemma}\label{eigadjoint}
Consider the linear ODE $V' = A(x)V$ and the corresponding adjoint problem $W' = -A(x)^* W$, where $A$ is an $n \times n$ matrix depending on $x$. Then the following are true.
\begin{enumerate}[(i)]
\item $\dfrac{d}{dx}\langle V(x), W(x) \rangle = 0$, thus the inner product is constant in $x$.
\item If $\Phi(y, x)$ is the evolution operator for $V' = A(x)V$, then $\Phi(x, y)^*$ is the evolution operator for the adjoint problem $W' = -W(x)^* W$.
\end{enumerate}
\begin{proof}
For (i), take the derivative of the inner product and use the expressions for $V'$ and $W'$. For (ii), take the derivative of the expression $\Phi(y, x)\Phi(x, y) = I$.
\end{proof}
\end{lemma}

Since $\langle \Psi(x), Q'(x) \rangle$ is constant in $x$ and $Q'(x) \rightarrow 0$ as $x \rightarrow \infty$, by the continuity of the inner product, we must have $\langle \Psi(0), Q'(0) \rangle = 0$. Similarly, if taking a solution $V(x)$ to the variational equation \eqref{vareq} with initial condition in $Y^+$ or $Y^-$, we can show that $\Psi(0) \perp Y^+$ and $\Psi(0) \perp Y^-$. The same holds for $\Psi^c(0)$. Thus we have shown that

\[
\Psi(0), \Psi^c(0) \perp \R Q'(0) \oplus Y^- \oplus Y^-
\]

Let 

\begin{equation}
S = \text{span }\{ \Psi(0), \Psi^c(0) \}
\end{equation}

Since $\Psi(0), \Psi^c(0)$ are linearly independent (but not orthogonal), $S$ has dimension 2. Since $S \perp \R Q'(0) \oplus Y^- \oplus Y^-$, we can write the tangent space at $Q(0)$ as the direct sum

\begin{equation}
\R^5 = \R Q'(0) \oplus Y^- \oplus Y^+ \oplus S
\end{equation}

\subsubsection{Piecewise Formulation}

To exploit these relations, we take the following piecewise ansatz for the eigenfunction $V(x)$

\begin{equation}
V_i^\pm(x) = d_i (Q_{np}'(x) + \lambda (Q_{np})_c(x)) + W_i^\pm 
\end{equation}

where the $V_i^-$ equation is defined on $[-X_{i-1}, 0]$, the $V_i^+$ equation is defined on $[0, X_i]$, and the $d_i \in \C$ are arbitrary constants. Substituting this into the eigenvalue problem and simplifying, we obtain the system for $W_i^\pm$

\begin{align*}
&(W_i^\pm)' = A( q_i^\pm(x) ) W_i^\pm + \lambda B W_i^\pm + \lambda^2 d_i \tilde{H}_i^\pm \\
&W_i^-(0) = W_i^+(0) \\
&W_i^\pm(0) \in S \oplus Y^+ \oplus Y^- \\
&W_i^+(X_i) - W_{i+1}^-(-X_i) = D_i d
\end{align*}

where

\begin{align*}
D_i d &= d_{i+1}[Q_{i+1}'(-X_i) + \lambda \partial_c Q_{i+1}(-X_i)]
- d_i [ Q_i'(X_i) + \lambda \partial_c Q_i(-X_i) ] \\
\tilde{H}_i^\pm &= -B \partial_c Q_i^\pm
\end{align*}

The conditions at $x = \pm X_i$ and $x = 0$ are the requisite matching conditions that guarantee continuity of the eigenfunction $V(x)$.\\

For the final form of the eigenvalue problem, we will combine the matrices $A( q_i^\pm(x) )$ and $\lambda B$ to obtain the piecewise eigenvalue problem

\begin{align*}
&(W_i^\pm)' = A_i^\pm(x; \lambda) W_i^\pm + \lambda^2 d_i \tilde{H}_i^\pm \\
&W_i^-(0) = W_i^+(0) \\
&W_i^\pm(0) \in S \oplus Y^+ \oplus Y^- \\
&W_i^+(X_i) - W_{i+1}^-(-X_i) = D_i d
\end{align*}

where

\begin{align*}
A_i^\pm(x; \lambda) &= A( q_i^\pm(x) ) + \lambda B 
\end{align*}

The system we will investigate is 

\begin{align*}
&(W_i^\pm)' = A_i^\pm(x; \lambda) W_i^\pm + \lambda^2 d_i \tilde{H}_i^\pm \\
&W_i^\pm(0) \in S \oplus Y^+ \oplus Y^- \\
&W_i^+(0) - W_i^-(0) \in S \\
&W_i^+(X_i) - W_{i+1}^-(-X_i) = D_i d
\end{align*}

A solution to this system solves the eigenvalue problem if and only if the $n$ jumps at $x = 0$, which can only be in the subspace $S$, are 0. Since $S$ is spanned by $\Psi(0)$ and $\Psi^c(0)$, this is true if and only if 

\begin{align*}
\langle \Psi(0), W_i^+(0) - W_i^-(0) \rangle &= 0 \\
\langle \Psi^c(0), W_i^+(0) - W_i^-(0) \rangle &= 0
\end{align*}

From the existence problem and from San98 we have the following estimates.

\begin{align*}
|H(x)|, |\tilde{H}_i^\pm(x)| &\leq C e^{-\alpha |x|} \\
|\Delta H_i^\pm| &= |\tilde{H}_i^\pm - H| \leq C(e^{-\alpha X_i} + e^{-\alpha X_{i-1}} ) \\
|\Delta H_i^-(x)| &\leq C e^{-\alpha X_{i-1}} e^{-\alpha(X_{i-1} + x) } \\
|\Delta H_i^+(x)| &\leq C e^{-\alpha X_i} e^{-\alpha(X_i - x) } \\
D_i d &= ( Q'(X_i) + Q'(-X_i))(d_{i+1} - d_i ) + \mathcal{O} \left( e^{-\alpha X_i} \left( |\lambda| +  e^{-\alpha X_i}  \right) |d| \right) \\
\end{align*}

\subsection{Conjugation}

To simplify the system, we would like to apply a change of coordinates so that the linear operator $A_i^\pm(x; \lambda)$ is transformed into a constant coefficient matrix. To do that, we will use the Conjugation Lemma, which follows from the Gap Lemma. Both are stated below.\\

First, we state and prove the Gap Lemma, which is modified from Zum2018. 

\begin{lemma}[Gap Lemma]\label{gaplemma}
Let $W \in \C^N$, and consider the family of ODEs on $\R$

\begin{equation}\label{LambdaEVP}
W(x)' = A(x; \Lambda) W
\end{equation}

where $\Lambda \in \Omega$ is a parameter vector and $\Omega$ is a Banach space. Assume that

\begin{enumerate}
	\item The map $\Lambda \mapsto A(\cdot; \Lambda)$ is analytic in $\Lambda$.
	\item $A(x; \Lambda) \rightarrow A_\pm(\lambda)$ (independent of $\Lambda$) as $x \rightarrow \pm \infty$, and for $|\Lambda| < \delta$ we have the uniform exponential decay estimates 
	\begin{align}
	\left| \frac{\partial^k}{\partial x^k} A(x; \Lambda) - A_\pm(\Lambda) \right| 
	&\leq C e^{-\theta |x|} && 0 \leq k \leq K
	\end{align}
	where $\alpha > 0$, $C > 0$, and $K$ is a nonnegative integer.
\end{enumerate}

Suppose $V^-(\Lambda)$ is an eigenvector of $A_-(\Lambda)$ with corresponding eigenvalue $\mu(\Lambda)$, both analytic in $\Lambda$. Then there exists a solution of \ref{LambdaEVP} of the form 

\begin{equation}
W(x; \Lambda) = V(x; \Lambda) e^{\mu(\Lambda)x}
\end{equation}

where $V$ is $C^1$ in $x$ and analytic in $\Lambda$ for $|\Lambda| < \delta$, and for any fixed $\tilde{\theta} < \theta$

\begin{align}
V(x; \Lambda) = V^-(\Lambda) + \mathcal{O}(e^{-\tilde{\theta}|x|}|V^-(\Lambda)|) && x < 0
\end{align}

\begin{proof}
This is almost identical to Zum2018. The only difference here is that the parameter vector $\Lambda$ is in a general Banach space instead of a subset of $C^p$.\\

Let $W(x; \Lambda) = V(x; \Lambda) e^{\mu(\Lambda) x}$. Substituting this into \eqref{LambdaEVP} and simplifying, we obtain the equivalent ODE

\begin{equation}\label{VEVP}
V(x; \Lambda)' = (A_- - \mu(\Lambda)I)V(x; \Lambda) + \Theta(x; \Lambda) V(x; \Lambda)
\end{equation}

where $\Theta(x; \Lambda) = (A(x; \Lambda) - A_-(\Lambda)) = \mathcal{O}(e^{-\theta|x|})$. Choose any $\tilde{\theta} < \theta_1 < \theta$ such that the real part of the spectrum of $A_-$ lies either to the left or to the right of the vertical line $\text{Re}(\nu) = \text{Re}(\mu(\Lambda) + \theta_1$ in the complex plane. We should be able to make sure this is case for all $|\Lambda| < \delta$ since all the eigenvalues of $A_(\Lambda)$ are analytic in $\Lambda$.\\

Then for $|\Lambda| < \delta$, we can define the spectral projections $P(\Lambda)$ and $Q(\Lambda)$, where $P(\Lambda)$ projects onto the direct sum of all eigenspaces of $A_-(\Lambda)$ corresponding to eigenvalues $\nu$ with $\text{Re}(\nu) < \text{Re}(\mu(\Lambda) + \theta_1$, and $Q(\Lambda)$ projects onto the direct sum of all eigenspaces of $A_-(\Lambda)$ corresponding to eigenvalues $\nu$ with $\text{Re}(\nu) > \text{Re}(\mu(\Lambda) + \theta_1$. $P(\Lambda)$ and $Q(\Lambda)$ are analytic in $\Lambda$ for $|\Lambda| < \delta$, and from our definition of $\theta_1$ we have the estimates

\begin{align*}
\left|e^{(A_-(\Lambda) - \mu(\Lambda)I)x}P \right| &\leq C e^{\theta_1 x} && x \geq 0 \\
\left|e^{(A_-(\Lambda) - \mu(\Lambda)I)x}Q \right| &\leq C e^{\theta_1 x} && x \leq 0
\end{align*}

Note that $P(\Lambda) + Q(\Lambda) = I$. Define the map $T$ on $L^\infty(-\infty, -M]$ by

\begin{align*}
TV(x; \Lambda) &= V^-(\Lambda) 
+ \int_{-\infty}^x e^{(A_-(\Lambda) - \mu(\Lambda)I)(x-y)}P\Theta(y; \Lambda) V(y; \Lambda) dy \\
&- \int_x^{-M} e^{(A_-(\Lambda) - \mu(\Lambda)I)(x-y)}Q\Theta(y; \Lambda) V(y; \Lambda) dy
\end{align*}

Taking the absolute value of both sides, for $x \leq 0$

\begin{align*}
|TV(x; \Lambda)| &\leq |V^-(\Lambda)| + C ||V(x; \Lambda)||_{L^\infty(-\infty, -M]}
\left( \int_{-\infty}^x e^{\theta_1 (x - y)} e^{\theta y} dy + \int_x^{-M} e^{\theta_1 (x - y)} e^{\theta y} dy \right) \\
&\leq |V^-(\Lambda)| + C ||V(x; \Lambda)||_{L^\infty(-\infty, -M]} e^{\theta_1 x} \int_{-\infty}^M e^{(\theta - \theta_1) y} dy \\
&= \leq |V^-(\Lambda)| + C ||V(x; \Lambda)||_{L^\infty(-\infty, -M]} e^{\theta_1 x} \frac{e^{-(\theta - \theta_1)M}}{\theta - \theta_1}\\
&\leq |V^-(\Lambda)| + C ||V(x; \Lambda)||_{L^\infty(-\infty, -M]} e^{\theta_1 x} e^{-(\theta - \theta_1)M} \\
&\leq |V^-(\Lambda)| + C ||V(x; \Lambda)||_{L^\infty(-\infty, -M]} e^{-(\theta - \theta_1)M} \\
& < \infty
\end{align*}

Since the RHS is independent of $x$, we have $T: L^\infty(-\infty, -M] \rightarrow L^\infty(-\infty, -M]$. Next, we look at

\begin{align*}
|TV_1(x; \Lambda) - TV_2(x; \Lambda)| &\leq C ||V_1(x; \Lambda) - V_2(x; \Lambda)||_{L^\infty(-\infty, -M]} e^{\theta_1 x} \frac{e^{-(\theta - \theta_1)M}}{\theta - \theta_1}\\
\end{align*}

Since $e^{-(\theta - \theta_1)M} \rightarrow 0$ as $m \rightarrow \infty$, for sufficiently large $M$ we have 

\begin{align*}
|TV_1(x; \Lambda) - TV_2(x; \Lambda)|_{L^\infty(-\infty, -M]} &\leq \frac{1}{2} ||V_1(x; \Lambda) - V_2(x; \Lambda)||_{L^\infty(-\infty, -M]} 
\end{align*}

Thus the map $T$ is a contraction. Since $L^\infty(-\infty, -M]$ is a Banach space, by the Banach fixed point theorem, the map $T$ has a unique fixed point $V = TV$, i.e. we have a function $V \in L^\infty(-\infty, -M]$ such that 

\begin{align*}
V(x; \lambda) &= V^-(\Lambda) 
+ \int_{-\infty}^x e^{(A_-(\Lambda) - \mu(\Lambda)I)x}P\Theta(y; \Lambda) V(y; \Lambda) dy 
- \int_x^{-M} e^{(A_-(\Lambda) - \mu(\Lambda)I)x}Q\Theta(y; \Lambda) V(y; \Lambda) dy
\end{align*}

Differentiating this with respect to $x$, we obtain

\begin{align*}
V'(x; \Lambda) &= P\Theta(x; \Lambda) V(x; \Lambda) +
(A_-(\Lambda) - \mu(\Lambda)I) \int_{-\infty}^x e^{(A_-(\Lambda) - \mu(\Lambda)I)(x-y)}P\Theta(y; \Lambda) V(y; \Lambda) dy \\
&-(-Q\Theta(x; \Lambda) V(x; \Lambda))
-(A_-(\Lambda) - \mu(\Lambda)I) \int_x^{-M} e^{(A_-(\Lambda) - \mu(\Lambda)I)(x-y)}Q\Theta(y; \Lambda) V(y; \Lambda) dy \\
&= P\Theta(x; \Lambda) V(x; \Lambda) + Q\Theta(y; \Lambda) V(x; \Lambda) + (A_-(\Lambda) - \mu(\Lambda)I)(T V(x; \lambda) - V^-(\Lambda) ) \\
&= (P + Q)\Theta(x; \Lambda) V(x; \Lambda) + (A_-(\Lambda) - \mu(\Lambda)I)(V(x; \lambda) - V^-(\Lambda) ) \\
&= (A_-(\Lambda) - \mu(\Lambda)I)V(x; \lambda) + \Theta(x; \Lambda) V(x; \Lambda) - (A_-(\Lambda) - \mu(\Lambda)I)V^-(\Lambda) \\
&= (A_-(\Lambda) - \mu(\Lambda)I)V(x; \lambda) + \Theta(x; \Lambda) V(x; \Lambda)
\end{align*}

where we used the fact that $TV = V$ and $(A_-(\Lambda) - \mu(\Lambda)I)V^-(\Lambda) = 0$. Thus $V(x; \Lambda$ solves \eqref{VEVP}. Since $TV = V$, we let $V_1 = V$ and $V_2 = 0$ in the above to get the estimate

\begin{align*}
|V(x; \Lambda) - V^-(\Lambda)| &= |T(V(x; \Lambda)) - T(0)| \\
&\leq C ||V(x; \Lambda) - 0||_{L^\infty(-\infty, -M]} e^{\theta_1 x} \\
\end{align*}

Similarly, for sufficiently large $M$, we have

\begin{align*}
|V(x; \Lambda)| - |V^-(\Lambda)| &\leq | |V(x; \Lambda)| - |V^-(\Lambda)| | \\
&\leq |V(x; \Lambda) - V^-(\Lambda)| \\
&= |T(V(x; \Lambda)) - T(0)| \\
&\leq \frac{1}{2} ||V(x; \Lambda)||_{L^\infty(-\infty, -M]}
\end{align*}

Thus

\begin{align*}
||V(x; \Lambda)||_{L^\infty(-\infty, -M]} \leq 2 |V^-(\Lambda)|
\end{align*}

Combining these, we have

\begin{align*}
|V(x; \Lambda) - V^-(\Lambda)| &\leq C e^{\tilde{\theta} x}|V^-(\Lambda)| \\
\end{align*}

from which we get

\begin{align*}
|V(x; \Lambda) = V^-(\Lambda) + \mathcal{O}( e^{\tilde{\theta} x}|V^-(\Lambda)| )\\
\end{align*}

At the moment, $V(x; \Lambda)$ is only defined for $x < -M$. We extend $V(x; \Lambda)$ to all of $R^-$ using the evolution operator for the system.

\end{proof}
\end{lemma}

As a corollary to this, we state and prove the Conjugation Lemma, which allows us to make a smooth change of coordinates to convert the linear ODE $Z'(x) = A^\pm(x) Z(x)$ into a constant coefficient system.

\begin{lemma}[Conjugation Lemma]
Let $W \in \C^N$, and consider the family of ODEs on $\R$

\begin{equation}\label{EVPconj}
W(x)' = A(x; \Lambda) W(x) + F(x) 
\end{equation}

where $\Lambda \in \Omega$ is a parameter vector and $\Omega$ is a Banach space. Take the same assumptions as in the Gap Lemma, i.e. 

\begin{enumerate}
	\item The map $\Lambda \mapsto A(\cdot; \Lambda)$ is analytic in $\Lambda$.
	\item $A(x; \Lambda) \rightarrow A_\pm(\lambda)$ (independent of $\Lambda$) as $x \rightarrow \pm \infty$, and for $|\Lambda| < \delta$ we have the uniform exponential decay estimates 
	\begin{align}
	\left| \frac{\partial^k}{\partial x^k} A(x; \Lambda) - A_\pm(\Lambda) \right| 
	&\leq C e^{-\theta |x|} && 0 \leq k \leq K
	\end{align}
	where $\alpha > 0$, $C > 0$, and $K$ is a nonnegative integer.
\end{enumerate}

Then in a neighborhood of any $\Lambda_0 \in \Omega$ there exist invertible linear transformations

\begin{align*}
P_+(x, \Lambda) &= I + \Theta_+(x, \Lambda) \\
P_-(x, \Lambda) &= I + \Theta_-(x, \Lambda) 
\end{align*}

defined on $\R^+$ and $\R^-$, respectively, such that

\begin{enumerate}[(i)]
\item The change of coordinates $W = P_\pm Z$ reduces \eqref{EVPconj} to the equations on $\R^\pm$

\begin{align}\label{conjZ}
Z'(x) = A^\pm(\Lambda) Z(x) + P_\pm(x, \Lambda)^{-1} F(x)
\end{align}

\item For any fixed $0 < \tilde{\theta} < \theta$, $0 \leq k \leq K+1$, and $j \geq 0$ we have the decay rates
\begin{align*}
\left| \partial_\Lambda^j \partial_x^k \Theta_\pm \right| \leq C(j, k)e^{-\tilde{\theta}|x|}
\end{align*}
\end{enumerate}
\begin{proof}
We prove this for the case where $B(x) = 0$ and $F(x) = 0$. The form of the conjugated system easily follow for general $F$.\\

We will do the case on $\R^-$. The other case is similar. 
Let $W = P_-(x, \Lambda) Z$, where we will figure out what $P_-(x, \Lambda)$ is later. Suppose that \eqref{conjZ} holds, and substitute these into \eqref{EVPconj}.

\begin{align*}
[P_-(x, \Lambda) Z(x)]' &= A(x; \Lambda)(P_-(x, \Lambda) Z(x)) \\
P_-'(x, \Lambda) Z(x) + P_-(x, \Lambda) Z'(x)
&= A(x; \Lambda)P_-(x, \Lambda) Z(x) \\
P_-'(x, \Lambda) Z(x) + P_-(x, \Lambda) A_- Z(x)
&= A(x; \Lambda)P_-(x, \Lambda) Z(x)
\end{align*}

Rearranging this, we obtain

\begin{equation}
P_-'(x, \Lambda) Z(x)
= [A(x; \Lambda)P_-(x, \Lambda) - P_-(x, \Lambda) A_-]Z(x)
\end{equation}

Suppose now that
\[
P_-'(x, \Lambda) = A(x; \Lambda)P_-(x, \Lambda) - P_-(x, \Lambda) A_-
\]

Then, upon making the substitution $W = P_-(x, \Lambda) Z$, \eqref{EVPconj} reduces to

\begin{align*}
[P_-(x, \Lambda) Z(x)]' &= A(x; \Lambda)(P_-(x, \Lambda) Z(x)) \\
P_-'(x, \Lambda) Z(x) + P_-(x, \Lambda) Z'(x)
&= A(x; \Lambda)P_-(x, \Lambda) Z(x) \\
(A(x; \Lambda)P_-(x, \Lambda) - P_-(x, \Lambda) A_-)Z(x) + P_-(x, \Lambda) Z'(x)
&= A(x; \Lambda)P_-(x, \Lambda) Z(x) \\
A(x; \Lambda)P_-(x, \Lambda)Z(x) - P_-(x, \Lambda) A_- Z(x) + P_-(x, \Lambda) Z'(x)
&= A(x; \Lambda)P_-(x, \Lambda) Z(x) \\
P_-(x, \Lambda) Z'(x) &= P_-(x, \Lambda) A_- Z(x) \\
Z'(x) &= A_- Z(x)
\end{align*}

which is what we want. In the last line, we used the fact that $P_-(x, \Lambda)$ is invertible, so we should make sure that is the case. Thus, we wish to find $P_-(x, \Lambda)$ such that

\[
P_-'(x, \Lambda) = A(x; \Lambda)P_-(x, \Lambda) - P_-(x, \Lambda) A_-
\]

We note that the this equation has the form 

\begin{equation}\label{solvePminus}
P_-'(x, \Lambda) = \mathcal{A}(x; \Lambda) P_-(x, \Lambda)
\end{equation}

where $\mathcal{A}(x; \Lambda)$ is the linear operator

\[
\mathcal{A}(x; \Lambda) P = A(x; \Lambda) P - P A_-
\]

By our assumptions on $A(x; \Lambda)$, $\mathcal{A} \rightarrow \mathcal{A}_-$ as $x \rightarrow -\infty$, where the limiting linear operator $\mathcal{A}_-$ is defined by

\[
\mathcal{A}_- P = A_- P - P A_-
\]

The limiting operator has analytic eigenvalue/eigenvector pair $0, I$ for all $\Lambda$, thus by the Gap Lemma, there exists a solution of \eqref{solvePminus} of the form 

\begin{equation}
P_-(x, \Lambda) = I + \mathcal{O}(e^{-\tilde{\theta}|x|})
\end{equation}

In other words, 

\begin{equation}
P_-(x, \Lambda) = I + \Theta_-(x, \Lambda)
\end{equation}

where 

\begin{equation}\label{Thetabound}
|\Theta_-(x, \Lambda)| \leq C e^{-\tilde{\theta}|x|}
\end{equation}

The $x$-derivative bound follow from the derivative bounds in the Gap Lemma, and the $\Lambda$-derivative bounds follow from standard analytic function theory.\\

Finally, we need to show that $P_-(x, \Lambda)$ is invertible for all $x \in \R^-$. Using \eqref{Thetabound}, we can find $M$ sufficiently large and negative such that for all $x \leq M$,

\[
|\Theta_-(x, \Lambda)| < 1/2
\]

It follows that $P_-(x, \Lambda)$ is invertible for $X \leq M$. To extend invertibility to all $x \in \R^-$, suppose that $P_-(x, \Lambda)^{-1}$ exists for all $x \in R^-$. Then, differentiating $P_-(x, \Lambda)^{-1} P_-(x, \Lambda) = I$ and solving for $[P_-(x, \Lambda)^{-1}]'$ (as in the proof of the inverse function theorem), we have (suppressing the dependence on $\Lambda$ for convenience)

\begin{align*}
(P_-^{-1})'(x) &= -P_-^{-1}(x)P_-'(x)P_-^{-1}(x) \\
&= -P_-^{-1}(x)( A(x)P_-(x) - P_-(x) A_-)P_-^{-1}(x) \\
&= A_- P_-^{-1}(x) - A(x) P_-^{-1}(x)
\end{align*}

We have a solution to this ODE for $x \leq M$, and by variation of constants, this ODE has a unique solution for all $x \in \R^-$. Thus $P_-(x, \Lambda)^{-1}$ is obtained for all $x \in \R^-$ by evolving this ODE forward from an initial condition at some $x \leq M$. In this manner, we have shown that $P_-(x, \Lambda)^{-1}$ exists for all $x \in \R^-$.

\end{proof}
\end{lemma}

We will use the Conjugation Lemma to transform the linear operator $A_i^\pm(x; \lambda) = A( q_i^\pm(x) ) + \lambda B$ into a constant coefficient matrix. For all $\lambda$, $A^\pm(x; \lambda)$ decays exponentially to the constant-coefficient matrix $A(\lambda)$. 

\begin{align}\label{Alambda}
A(\lambda) &= \begin{pmatrix}0 & 1 & 0 & 0 & 0 \\0 & 0 & 1 & 0 & 0 \\0 & 0 & 0 & 1 & 0 \\0 & 0 & 0 & 0 & 1 \\
\lambda & -c & 0 & 1 & 0 \end{pmatrix}
\end{align}

Let $\Lambda = (\lambda, q(x))$ be the parameter vector we will use in the Conjugation Lemma, where $q(x)$ is in the Banach space of continuous functions on $[X_{i-1}, 0]$ or $[0, X_i]$. (This requires a version of the Conjugation Lemma which allows parameters to be in an arbitrary Banach space.) Since 
$A( q(x) ) + \lambda B$ is linear, thus analytic, in $\lambda$ and in $q(x)$, let $P_i^\pm(x; \lambda, q(x)$ be the conjugation operator for $A( q(x) ) + \lambda B$. For convenience, let  

\begin{equation}
P_i^\pm(x; \lambda) = P^\pm(x; \lambda, q_i^\pm(x) )
\end{equation}

and let

\begin{equation}
P^\pm(x; \lambda) = P^\pm(x; \lambda, q(x) )
\end{equation}

Using the Conjugation Lemma, we make the substitution $W_i^\pm = P_i^\pm(x; \lambda) Z_i^\pm$. Then our system becomes

\begin{align*}
&(Z_i^\pm(x))' = A(\lambda) Z_i^\pm(x) + \lambda^2 d_i P_i^\pm(x; \lambda)^{-1} \tilde{H}_i^\pm(x) \\
&P_i^\pm(0; \lambda) Z_i^\pm(0) \in \C \Psi(0) \oplus Y^+ \oplus Y^- \oplus Y^0 \\
&P_i^+(0; \lambda) Z_i^+(0) - P_i^-(0; \lambda) Z_i^-(0) \in S  \\
&P_i^+(X_i; \lambda) Z_i^+(X_i)\ - P_{i+1}^-(-X_i; \lambda) Z_{i+1}^-(-X_i; \lambda) = D_i d
\end{align*}

and the jump conditions become

\begin{align*}
\langle \Psi(0), P_i^+(0; \lambda) Z_i^+(0) - P_i^-(0; \lambda) Z_i^-(0) \rangle &= 0 \\
\langle \Psi^c(0), P_i^+(0; \lambda) Z_i^+(0) - P_i^-(0; \lambda) Z_i^-(0) \rangle &= 0
\end{align*}

\subsection{Evolution}

Since $A(\lambda)$ is constant coefficient, we know exactly how solutions of $V' = A(\lambda)V$ will behave.\\

Since $A(\lambda)$ depends linearly on $\lambda$ and $A(0)$ has a simple eigenvalue at 0, for sufficiently small $\lambda$, $A(\lambda)$ will have a simple, small eigenvalue $\nu(\lambda)$, and $\nu(\lambda) = \mathcal{O}(\lambda)$.\\ 

We define the following.

\begin{enumerate}
	\item Let

	\begin{align*}
	X_m &= \min(X_0, \dots, X_{n-1}) \\
	X_M &= \max(X_0, \dots, X_{n-1}) \\
	\end{align*}

	\item Choose $\rho > 0$ sufficiently small so that $\alpha_0 - 4 \rho > 0$. Let

	\begin{align*}
	\alpha &= \alpha_0 - \rho \\
	\tilde{\alpha} &= \alpha - 3 \rho
	\end{align*}

	\item Choose $\delta$ sufficiently small so that for all $|\lambda| < \delta$
	\begin{enumerate}
		\item For the small eigenvalue of $A(\lambda)$ we have $|\nu(\lambda)| < \rho$
		\item The real part of any other eigenvalue of $A(\lambda)$ lies outside the interval $[-\alpha, \alpha]$.
	\end{enumerate}

	\item Choose $X_m$ sufficiently large so that
	\begin{equation}
	e^{-\tilde{\alpha} X_m} < \delta
	\end{equation}

\end{enumerate}

Let $E^{u/s/c}(\lambda)$ be the stable/unstable/center eigenspaces of $A(\lambda)$, where $E^c$ is the one-dimensional subspace spanned by the eigenvector corresponding to the small eigenvalue $\nu(\lambda)$. This is a ``true'' center subspace only when $\nu(\lambda)$ has no real part, but we will always call it a center subspace for convenience. Let $P^{u/s/c}_0(\lambda)$ be the corresponding eigenprojections for the eigenspaces $E^{u/s/c}(\lambda)$.\\

Let $\Phi(x, y; \lambda) = e^{A(\lambda)(x-y)}$ be the evolution of the constant-coefficient ODE

\[
Z' = A(\lambda) Z
\]

Let $\Phi^{u/s/c}(x, y; \lambda) = \Phi(x, y; \lambda)P^{u/s/c}_0(\lambda)$ be the evolutions on the respective eigenspaces. For these evolutions, we have bounds

\begin{align*}
|\Phi^s(x, y; \lambda)| &\leq C e^{-\alpha(x - y)} \\
|\Phi^u(x, y; \lambda)| &\leq C e^{-\alpha(y - x)} \\
|\Phi^c(x, y; \lambda)| &\leq C e^{\rho|x - y|} 
\end{align*}

Since $E^c(\lambda)$ is one-dimensional, we have a formula for $\Phi^c(x, y; \lambda)$.

\begin{align*}
\Phi^c(x, y; \lambda) v &= e^{\nu(\lambda)(x - y)} v && v \in E^c(\lambda)
\end{align*}

Finally, we will look at the variational and adjoint variational e
equations for the linearization about the primary pulse. Recall that these are given by 

\begin{align*}
V_i' &= A(q(x)) V_i \\
W_i' &= -A(q(x))^* W_i
\end{align*}

Let $\Theta(y, x)$ be the evolution operator for the variational equation. Then $\Theta(x, y)^*$ is the evolution operator for the adjoint variational equation. Then (as defined above) $P^\pm(x)$ conjugate $A(q(x))$. We have the following relationship between $\Theta(y, x)$ and $\Phi(y, x;, 0)$.

\begin{align*}
\Theta(y, x) &= P^+(y) \Phi(y, x; 0) P^+(x)^{-1} && x, y \geq 0 \\
\Theta(y, x) &= P^-(y) \Phi(y, x; 0) P^-(x)^{-1} && x, y \leq 0
\end{align*}

Finally, recalling that

\begin{align*}
P_i^\pm(x; \lambda) = P^\pm(x; \lambda; q_i^\pm(x)) \\
P^\pm(x; \lambda) = P^\pm(x; \lambda; q(x))
\end{align*}

we can expand $P_i^\pm(x; \lambda)$ in a Taylor series about $(0, q(x))$ to get

\begin{align*}
P_i^\pm(x; \lambda) = P^\pm(0) + C_1 |\lambda| 
+ C_2| q_i^\pm(x) - q(x) | + \text{h.o.t.}
\end{align*}

Since 

\begin{align*}
| q_i^\pm(x) - q(x) | &= | q^\pm(x; \beta_i^\pm) + u_i^\pm(x) - q(x) | \\
&\leq |q^\pm(x; \beta_i^\pm) - q(x)| + |u_i^\pm(x)|
\end{align*}

we have at $x = 0$

\begin{equation}\label{PTaylor}
P_i^\pm(0; \lambda) = P^\pm(0) + \mathcal{O}(|\lambda| + e^{-2 \alpha X_m})
\end{equation}

We will only need the Taylor expansion at $x = 0$, since for large $x$, the conjugation operators are approximately equal to the identity.

\subsection{Inversion}

Define the spaces

\begin{align*}
V_a &= \bigoplus_{i=0}^{n-1} E^u(\lambda) \oplus E^s(\lambda) \\
V_b &= \bigoplus_{i=0}^{n-1} E^u(0) \oplus E^s(0) \\
V_c^- &= \bigoplus_{i=0}^{n-1} E^c(\lambda) \\
V_c^+ &= \bigoplus_{i=0}^{n-1} E^c(\lambda) \\
V_c &= V_c^- \oplus V_c^+ \\
V_\lambda &= B_\delta(0) \subset \C
\end{align*}

where the subscripts are all $\mod n$, as in the existence problem. We use the $\lambda-$dependent eigenspaces for $a_i^\pm$ and $c_i^\pm$, since we will be evolving them under the $\lambda-$dependent evolution $\Phi(y, x; \lambda)$.\\

All the product spaces are endowed with the maximum norm, e.g. for $V_c$, $|c| = \max(|c_0^-|, \dots, |c_{n-1}^-|, |c_0^+|, \dots, |c_{n-1}^+|)$. In addition, we take the following convention. If we eliminate either a subscript or a superscript (or both) in the norm, we are taking the maximum over the eliminated thing. For example,

\begin{enumerate}
	\item $|c_i| = \max(|c_i^+|, |c_i^-|)$ 
	\item $|c^+| = \max(|c_0^+|, \dots, |c_{n-1}^+|)$
\end{enumerate}

In order to solve our system, we first look at the ODE

\[
(Z_i^\pm(x))' = A(\lambda) Z_i^\pm(x) + \lambda^2 d_i P_i^\pm(x; \lambda)^{-1} \tilde{H}_i^\pm(x)
\]

The solution this solves the following fixed point equations. For $i = 0, \dots, n-1$, the fixed point equations for $Z_i^\pm(x)$ are

\begin{align*}
Z_i^-(x) &= \Phi^s(x, -X_{i-1}; \lambda) a_{i-1}^- + \Phi^u(x, 0; \lambda) b_i^- + \Phi^c(x, -X_{i-1}; \lambda) c_{i-1}^- \\
&+ \lambda^2 d_i \int_0^x \Phi^u(x, y; \lambda) P_i^-(y; \lambda)^{-1} \tilde{H}_i^-(y)] dy \\
&+ \lambda^2 d_i \int_{-X_{i-1}}^x \Phi^s(x, y; \lambda) P_i^-(y; \lambda)^{-1} \tilde{H}_i^-(y) dy \\
&+ \lambda^2 d_i \int_{-X_{i-1}}^x \Phi^c(x, y; \lambda) P_i^-(y; \lambda)^{-1} \tilde{H}_i^-(y) dy  \\ 
Z_i^+(x) &= \Phi^u(x, X_i; \lambda) a_i^+ + \Phi^s(x, 0; \lambda) b_i^+ + \Phi^c(x, X_i; \lambda) c_i^+ \\
&+ \lambda^2 d_i \int_0^x \Phi^s(x, y; \lambda) P_i^+(y; \lambda)^{-1} \tilde{H}_i^+(y) dy \\
&+ \lambda^2 d_i \int_{X_i}^x \Phi^u(x, y; \lambda) P_i^+(y; \lambda)^{-1} \tilde{H}_i^+(y) dy \\
&+ \lambda^2 d_i \int_{X_i}^x \Phi^c(x, y; \lambda) P_i^+(y; \lambda)^{-1} \tilde{H}_i^+(y) dy \\
\end{align*}

% match at ends

\subsubsection{Matching at ends}

In the first inversion lemma, we solve the matching conditions at $\pm X_i$, $i = 0, \dots, n-1$.

\[
P_i^+(X_i; \lambda) Z_i^+(X_i) - P_{i+1}^-(-X_i; \lambda) Z_{i+1}^-(-X_i) = D_i d
\]

% first inversion lemma : match at \pm X_i

\begin{lemma}\label{inv1}

There exists an operator

\begin{align*}
A_1: V_\lambda \times V_b \times V_c^- \times V_d \rightarrow V_a \times V_c^+\\
\end{align*}

such that $(a, c^+) = A_1(\lambda)(b, c^-,d)$ solves our system. This operator is analytic in $\lambda$ and linear in $(b,vc^-,d)$. Piecewise bounds for $A_1$ are given by

\begin{align}\label{A1bound}
|A_1&(\lambda)_i(b, c^-, d)|
\leq C \Big( e^{-\alpha X_i} (|b_i^+| + |b_{i+1}^-|) + |c_i^-| + e^{-(\alpha - \rho) X_i} |\lambda^2||d| + |D_i||d| \Big)
\end{align} 

In addition, we can write $a_i^\pm$ and $c_i^+$ as 

\begin{align*}
a_i^+ &= P_i^+(X_i; \lambda) P_0^u(\lambda) D_i d + A_2(\lambda)_i^+(b, c^-, d) \\
a_i^- &= -P_i^-(-X_i; \lambda) P_0^s(\lambda) D_i d + A_2(\lambda)_i^-(b, c^-, d) \\
c_i^+ &= c_i^- + P_0^c(\lambda) D_i d + A_2(\lambda)_i^c(b, c^-, d) )
\end{align*}

where $A_2$ is analytic in $\lambda$, linear in $(b, c^-, d)$, and has piecewise bounds

\begin{align*}
|A_2&(\lambda)_i(b, c^-, d)|
\leq C \Big( e^{-\alpha X_i} (|b_i^+| + |b_{i+1}^-| + |c_i^-|) + e^{-(\alpha - \rho) X_i} |\lambda|^2|d| + e^{-\alpha X_i} |D_i||d| \Big)
\end{align*}

Finally, we have the estimate

\begin{equation}\label{P0cDid}
|P_0^c(\lambda) D_i d| \leq C e^{-\alpha X_i}(|\lambda| + e^{-\alpha X_i})|d|
\end{equation}

\begin{proof}

At $\pm X_i$, the fixed point equations become

\begin{align*}
Z_{i+1}^-(-X_i) &= a_i^- + \Phi^u(-X_i, 0; \lambda) b_{i+1}^- + c_i^- 
+ \lambda^2 d_{i+1} \int_0^{-X_i} \Phi^u(-X_i, y; \lambda) P_{i+1}^-(y; \lambda)^{-1} \tilde{H}_i^-(y) dy \\
Z_i^+(X_i) &= a_i^+ + \Phi^s(X_i, 0; \lambda) b_i^+ + c_i^+ 
+ \lambda^2 d_i \int_0^{X_i} \Phi^s(X_i, y; \lambda) P_i^+(y; \lambda)^{-1} \tilde{H}_i^+(y) dy
\end{align*}

To obtain these, we used the fact that, for example, $a_i^- \in E^s(\lambda)$ and $\Phi^s(-X_{i-1}, -X_{i-1}; \lambda)$ is the identity on $E^s(\lambda)$. From the Conjugation Lemma, we have the estimate

\begin{equation}\label{conjest}
P_i^\pm(\pm X_i; \lambda) = I + \mathcal{O}(e^{-\alpha X_i})
\end{equation}

which we will use on the $a_i^\pm$ and $c_i^\pm$ terms. Thus we obtain the equation

\begin{align}\label{Dideq1}
D_i d &= a_i^+ - a_i^- + c_i^+ - c_i^- + L_3(\lambda)_i(a, b, c^+, c^-, d)
\end{align}

For a bound on $L_3$, we look at the individual terms. As usual, we will in general only look at one of the two pieces.

\begin{enumerate}

\item For the $a_i^\pm$ and $c_i^\pm$ terms, we have a term of order $\mathcal{O}(e^{-\alpha X_i}(|a_i| + |c_i^+| + |c_i^-|)$, which comes from the estimate \eqref{conjest} for the conjugation operators $P_i^\pm(\pm X_i; \lambda)$.

\item For the terms involving $b$, we have

\[
| P_i^-(-X_i; \lambda) \Phi^u(-X_i, 0; \lambda) b_{i+1}^-| \leq C e^{-\alpha X_i} |b_{i+1}
^-|
\]

\item For the integral terms, we have

\begin{align*}
&\left|
P^+(X_i; \beta_i^+, \lambda) \int_0^{X_i} \Phi^s(X_i, y; \lambda) P^+(X_i; \beta_i^+, \lambda)^{-1} \tilde{H}_i^+(y) dy \right| \\
&\leq C \int_0^{X_i} e^{-\alpha(X_i - y)}e^{-\alpha y} dy \\
&\leq C \int_0^{X_i} e^{-(\alpha - \rho)(X_i - y)}e^{-\alpha y} dy \\
&= C e^{-(\alpha - \rho) X_i} \int_0^{X_i} e^{-\rho y} dy \\ 
&\leq C e^{-(\alpha - \rho) X_i} 
\end{align*}

\end{enumerate}

Putting these all together, we have the following bound for $L_3$.

\begin{equation}\label{L3bound}
|L_3(\lambda)_i(a, b, c^+, c^-, d)| \leq C \Big( e^{-\alpha X_i} ( |a_i| + |b_i^+| + |b_{i+1}^-| + |c_i^+| + |c_i^-|) + e^{-(\alpha - \rho) X_i} |\lambda^2| |d| \Big)
\end{equation}

Since $e^{-\alpha X_m} < \delta$, this becomes

\begin{align*}
|L_3(\lambda)_i(a, b, c^+, c^-, d)| \leq C \delta ( |a_i| + |c_i^+| ) + C\Big( e^{-\alpha X_i} ( |b_i^+| + |b_{i+1}^-| + |c_i^-|) + e^{-(\alpha - \rho) X_i} |\lambda^2| |d| \Big)
\end{align*}

Let 

\[
J_1: \bigoplus_{j=1}^n (E^s(\lambda) \times E^u(
\lambda) \times E^c(\lambda) ) \rightarrow \bigoplus_{j=1}^n \rightarrow \C^{2m+1}
\]

be defined by $(J_1)_i(a_i^+, a_i^-, c_i^+) = (a_i^+ - a_i^-, c_i^+)$. The map $J_i$ is a linear isomorphism since $E^s(\lambda) \oplus E^u(\lambda) \oplus E^c(\lambda) = \C^{2m + 1}$. Consider the map

\[
S_1(a, c^+) = J_1 (a, c^+) + L_3(\lambda)(a, 0, c^+, 0, 0) = J_1( I + J_1^{-1} L_3(\lambda)(a, 0, c^+, 0, 0))
\]

For sufficiently small $\delta$, we will have the operator norm $||J_1^{-1} L_3(\lambda)(a, 0, c^+, 0, 0)|| < 1$, thus the operator $S_1(a, c^+)$ is invertible. We can solve for $(a, c^+)$ to get

\[
(a, c^+) = A_1(\lambda)(b, c^-, d) = S_i^{-1}(-D d + L_3(\lambda)(0, b, 0, c^-, d)
\]

Using the bound on $L_3$ and noting which pieces are involved, $A_1$ will have piecewise bounds

\begin{align*}
|A_1&(\lambda)_i(b, c^-, d)|
\leq C \Big( e^{-\alpha X_i} (|b_i^+| + |b_{i+1}^-|) + |c_i^-| + e^{-(\alpha - \rho) X_i} |\lambda^2||d| + |D_i||d| \Big)
\end{align*} 

Next, we hit \eqref{Dideq1} with the projections $P_0^{s/u/c}(\lambda)$ on the eigenspaces $E^{s/u/c}(\lambda)$ to obtain the expressions

\begin{align*}
a_i^+ &= P_0^u(\lambda) D_i d + A_2(\lambda)_i^+(b, c^-, d) \\
a_i^- &= -P_0^s(\lambda) D_i d + A_2(\lambda)_i^-(b, c^-, d) \\
c_i^+ &= c_i^- + P_0^c(\lambda) D_i d + A_2(\lambda)_i^c(b, c^-, d) )
\end{align*}

The bound on the remainder term $A_2(\lambda)_i(b, c^-, d)$ is found by substituting the bound for $A_1$ into the bound for $L_3$ and simplifying. 

\begin{align*}
|A_2&(\lambda)_i(b, c^-, d)|
\leq C \Big( e^{-\alpha X_i} (|b_i^+| + |b_{i+1}^-| + |c_i^-|) + e^{-(\alpha - \rho) X_i} |\lambda|^2|d| + e^{-\alpha X_i} |D_i||d| \Big)
\end{align*} 

Anticipating what we will need at the end, we will derive slightly different expressions for $a_i^+$ and $a_i^-$. Using the conjugation operator $P_i^+(X_i; \lambda)$, we write $a_i^+$ as

\begin{align*}
a_i^+ = P_i^+(X_i; \lambda)a_i^+ + (I - P_i^+(X_i; \lambda))a_i^+ &= P_0^u(\lambda) D_i d + A_2(\lambda)_i^+(b, c^-, d)
\end{align*}

Rearranging this, we obtain

\begin{align*}
P_i^+(X_i; \lambda) a_i^+ &= P_0^u(\lambda) D_i d + A_2(\lambda)_i^+(b, c^-, d) - (I - P_i^+(X_i; \lambda))a_i^+ \\
&= P_0^u(\lambda) D_i d + A_2(\lambda)_i^+(b, c^-, d) + \mathcal{O}\Big( e^{-\alpha X_i} ( e^{-\alpha X_i} (|b_i^+| + |b_{i+1}^-|) + |c_i^-| + e^{-(\alpha - \rho) X_i} |\lambda^2||d| + |D_i||d| )\Big)
\end{align*}

where we used the bound $A_1$ and the estimate \eqref{conjest}. The last term on the RHS is the same (or higher) order as $A_2$, so we incorporate that into the bound on $A_2(\lambda)_i^+(b, c^-, d)$ to get

\begin{align*}
P_i^+(X_i; \lambda)a_i^+ &= P_0^u(\lambda) D_i d + A_2(\lambda)_i^+(b, c^-, d)
\end{align*}

Finally, we operate on both sides on the left by $P_i^+(X_i; \lambda)^{-1}$ to solve for $a_i^+$. Since $P_i^+(X_i; \lambda)^{-1}$ a bounded operator, we will also incorporate this into $A_2(\lambda)_i^+(b, c^-, d)$. In doing so, the bound will be unchanged. Thus we have

\begin{align*}
a_i^+ &= P^+(X_i; \beta_i^+, \lambda) P_0^u(\lambda) D_i d + A_2(\lambda)_i^+(b, c^-, d)
\end{align*}

We do the same thing for $a_i^-$, which gives us

\begin{align*}
a_i^- &= -P_i^-(-X_i; \lambda) P_0^s(\lambda) D_i d + A_2(\lambda)_i^-(b, c^-, d)
\end{align*}

Finally, we would like to obtain an estimate for $P_0^c(\lambda) D_i d$. Recall that

\[
D_i d = ( Q'(X_i) + Q'(-X_i))(d_{i+1} - d_i ) + \mathcal{O} \left( e^{-\alpha X_i} \left( |\lambda| +  e^{-\alpha X_i}  \right) |d| \right) 
\]

Looking at the lower order terms,

\begin{align*}
P_0^c(\lambda)&( Q'(X_i) + Q'(-X_i)) 
= P_0^c(0)( Q'(X_i) + Q'(-X_i)) + \mathcal{O}(|\lambda|e^{-\alpha X_i}) \\
&= \mathcal{O}(e^{-\alpha X_i}(|\lambda| + e^{-\alpha X_i}))
\end{align*}

Thus we have

\[
|P_0^c(\lambda) D_i d| \leq C e^{-\alpha X_i}(|\lambda| + e^{-\alpha X_i})|d|
\]

\end{proof}
\end{lemma}

\subsubsection{Conditions at 0}

In the second inversion lemma, we solve the conditions at $x = 0$

\begin{align*}
P_i^\pm(0; \lambda) Z_i^\pm(0) &\in \C \Psi(0) \oplus Y^0 \oplus Y^+ \oplus Y^- \\
P_i^+(0; \lambda) Z_i^+(0) - P_i^-(0; \lambda) Z_i^-(0) &\in \C \Psi(0) \oplus Y^0
\end{align*}

Recall that we have

\[
\C^m = \C Q'(0) \oplus Y^+ \oplus Y^- \oplus S
\]

This condition is equivalent to the three projections

\begin{align*}
P(\C Q'(0) ) P_i^-(0; \lambda) Z_i^-(0) &= 0 \\
P(\C Q'(0) ) P_i^+(0; \lambda) Z_i^+(0) &= 0 \\
P(Y_i^+ \oplus Y_i^-) ( P_i^+(0; \lambda) Z_i^+(0) - P_i^-(0; \lambda) Z_i^-(0) ) &= 0
\end{align*}

where the kernel of each projection is the remaining spaces in the direct sum. We don't need to include $\C Q'(0)$ in the third equation since we eliminated any component in it in the first two equations.

% second inversion lemma

\begin{lemma}\label{inv2}
There exist operators

\begin{align*}
B_1: &V_\lambda \times V_c^- \times V_d \rightarrow V_b \\
A_3: &V_\lambda \times V_c^- \times V_d \rightarrow V_a 
\end{align*}

such that $( (a, c^+) , b ) = ( A_3(\lambda)(c^-,d), B_1(\lambda)(c^-, d) )$ solves our system. These operators are analytic in $\lambda$ and linear in $(c^-,d)$. Piecewise bounds for $B_1$ and $A_3$ are given by

\begin{align}
|B_1&(\lambda)_i(\tilde{c}, d)| \leq C\Big( 
(|\lambda| + e^{-\tilde{\alpha}X_m})( |\tilde{c}_{i-1}^+| + |\tilde{c}_i^-|) + ( e^{-\tilde{\alpha}X_m} |D| + e^{-\tilde{\alpha}X_m}|\lambda| + |\lambda|^2)|d| \Big) \\
|A_3&(\lambda)_i(c^-, d)|
\leq C \Big(  
e^{-\alpha X_m} (|\lambda| + e^{-\tilde{\alpha}X_m})(|\tilde{c}_{i-1}^+| + |\tilde{c}_{i+1}^-|) +|c_i^-| + e^{-(\alpha - \rho) X_i} |\lambda^2||d| + |D_i||d| \Big)
\end{align} 

where

\begin{equation}\label{tildec}
\tilde{c}_i^\pm = e^{\pm \nu(\lambda) X_i} c_i^-
\end{equation}

In addition, we can write

\begin{align*}
a_i^+ &= P_i^+(X_i; \lambda) P_0^u(\lambda) D_i d + A_4(\lambda)_i^+(b, c^-, d) \\
a_i^- &= -P_i^-(-X_i; \lambda) P_0^s(\lambda) D_i d + A_4(\lambda)_i^-(b, c^-, d) \\
c_i^+ &= c_i^- + P_0^c(\lambda) D_i d + A_4(\lambda)_i^c(b, c^-, d) )
\end{align*}

where $A_4(\lambda)(c^-, d)$ is analytic in $\lambda$, linear in $(c^-, d)$, and has piecewise bounds

\begin{align}
|A_4&(\lambda)_i(c^-, d)|
\leq C \Big( 
e^{-\alpha X_i} (|\lambda| + e^{-\tilde{\alpha}X_m})(|\tilde{c}_{i-1}^+| + |\tilde{c}_{i+1}^-|) + e^{-(\alpha - \rho) } |c_i^-| + e^{-\tilde{\alpha} X_m} |\lambda|^2|d| + e^{-\alpha X_m}|D||d| \Big)
\end{align}

Finally, we have the following expression for $e^{-\nu(\lambda)X_i} c_i^+$.

\begin{align}\label{tildecminus2}
e^{-\nu(\lambda)X_i} &c_i^+ = e^{-\nu(\lambda)X_i} c_i^- 
+ \mathcal{O}\Big( e^{-\tilde{\alpha}X_m} (|\lambda| + e^{-\tilde{\alpha}X_m})( |\tilde{c}_{i-1}^+| + |\tilde{c}_{i+1}^-|) 
+ e^{-(\alpha - 2 \rho)X_i}|c_i^-| +  e^{-\tilde{\alpha}X_m}(|\lambda| + |D_i|)|d| \Big)
\end{align}

\begin{proof}

Recall that at $Q(0)$, the tangent spaces to the stable and unstable manifold are given by

\begin{align*}
T_{Q(0)} W^u(0) &= \R Q'(0) \oplus Y^- \\
T_{Q(0)} W^s(0) &= \R Q'(0) \oplus Y^+
\end{align*}

Thus we have

\begin{align*}
P^-(0)^{-1} Q'(0) &= V^- \in E^u(0) \\
P^+(0)^{-1} Q'(0) &= V^+ \in E^s(0)
\end{align*}

Let

\begin{align*}
E^u(0) &= \C V^- \oplus E^- \\
E^s(0) &= \C V^+ \oplus E^+ \\
\end{align*}

Then we have

\begin{align*}
P^-(0)^{-1} Y^- = E^- \\
P^+(0)^{-1} Y^+ = E^+ \\
\end{align*}

We decompose $b_i^\pm$ uniquely as $b_i^\pm = x_i^\pm + y_i^\pm$, where $x_i^\pm \in \C V^\pm$ and $y_i^\pm \in E^\pm$.\\

At $x = 0$, the fixed point equations become

\begin{align*}
Z_i^-(0) &= \Phi^s(0, -X_{i-1}; \lambda) a_{i-1}^- + \Phi^u(0, 0; \lambda) b_i^- + \Phi^c(0, -X_{i-1}; \lambda) c_{i-1}^- \\
&+ \lambda^2 d_i \int_{-X_{i-1}}^0 \Phi^s(0, y; \lambda) P_i^-(y; \lambda)^{-1} \tilde{H}_i^-(y) dy \\
&+ \lambda^2 d_i \int_{-X_{i-1}}^0 \Phi^c(0, y; \lambda) P_i^-(y; \lambda)^{-1} \tilde{H}_i^-(y) dy  \\ 
Z_i^+(0) &= \Phi^u(0, X_i; \lambda) a_i^+ + \Phi^s(0, 0; \lambda) b_i^+ + \Phi^c(0, X_i; \lambda) c_i^+ \\
&+ \lambda^2 d_i \int_{X_i}^0 \Phi^u(0, y; \lambda) P_i^+(y; \lambda)^{-1} \tilde{H}_i^+(y) dy \\
&+ \lambda^2 d_i \int_{X_i}^0 \Phi^c(0, y; \lambda) P_i^+(y; \lambda)^{-1} \tilde{H}_i^+(y) dy \\
\end{align*}

Noting that $\Phi^u(0, 0; \lambda) = P_0^u(\lambda)$, doing a little manipulation on the $b_i$ terms, and using the known form of the evolution $\Phi^c$ on $E^c(\lambda)$, this becomes

\begin{align*}
Z_i^-(0) &= \Phi^s(0, -X_{i-1}; \lambda) a_{i-1}^- + x_i^- + y_i^- + (P_0^u(\lambda) - P_0^u(0))b_i^- + e^{\nu(\lambda) X_{i-1}} c_{i-1}^- \\
&+ \lambda^2 d_i \int_{-X_{i-1}}^0 \Phi^s(0, y; \lambda) P_i^-(y; \lambda)^{-1} \tilde{H}_i^-(y) dy \\
&+ \lambda^2 d_i \int_{-X_{i-1}}^0 \Phi^c(0, y; \lambda) P_i^-(y; \lambda)^{-1} \tilde{H}_i^-(y) dy  \\ 
Z_i^+(0) &= \Phi^u(0, X_i; \lambda) a_i^+ + x_i^+ + y_i^+ + (P_0^s(\lambda) - P_0^s(0)) b_i^+ + e^{-\nu(\lambda)X_i} c_i^+ \\
&+ \lambda^2 d_i \int_{X_i}^0 \Phi^u(0, y; \lambda) P_i^+(y; \lambda)^{-1} \tilde{H}_i^+(y) dy \\
&+ \lambda^2 d_i \int_{X_i}^0 \Phi^c(0, y; \lambda) P_i^+(y; \lambda)^{-1} \tilde{H}_i^+(y) dy \\
\end{align*}

Since $c_i^\pm$ are in the eigenspaces $E^c(\lambda)$, we do some further manipulation to separate out a component in $E^c(0)$.

\begin{align*}
Z_i^-(0) &= \Phi^s(0, -X_{i-1}; \lambda) a_{i-1}^- + x_i^- + y_i^- + (P_0^u(\lambda) - P_0^u(0))b_i^- \\
&+ P_0^c(0) e^{\nu(\lambda) X_{i-1}} c_{i-1}^- + (P_0^c(\lambda) - P_0^c(0)) e^{\nu(\lambda) X_{i-1}} c_{i-1}^- \\
&+ \lambda^2 d_i \int_{-X_{i-1}}^0 \Phi^s(0, y; \lambda) P_i^-(y; \lambda)^{-1} \tilde{H}_i^-(y) dy \\
&+ \lambda^2 d_i \int_{-X_{i-1}}^0 \Phi^c(0, y; \lambda) P_i^-(y; \lambda)^{-1} \tilde{H}_i^-(y) dy  \\ 
Z_i^+(0) &= \Phi^u(0, X_i; \lambda) a_i^+ + x_i^+ + y_i^+ + (P_0^s(\lambda) - P_0^s(0)) b_i^+ \\
&+ P_0^c(0) e^{-\nu(\lambda)X_i} c_i^+ + (P_0^c(\lambda) - P_0^c(0)) e^{-\nu(\lambda)X_i} \\
&+ \lambda^2 d_i \int_{X_i}^0 \Phi^u(0, y; \lambda) P_i^+(y; \lambda)^{-1} \tilde{H}_i^+(y) dy \\
&+ \lambda^2 d_i \int_{X_i}^0 \Phi^c(0, y; \lambda) P_i^+(y; \lambda)^{-1} \tilde{H}_i^+(y) dy \\
\end{align*}

Finally, we operate on these by $P_i^\pm(0; \lambda)$. For the $c_i^-$ and $b$ terms, we write these as

\[
P_i^\pm(0; \lambda) = P^\pm(0) + (P_i^\pm(0; \lambda) - P^\pm(0))
\]

We finally wind up with the equations

\begin{align*}
P_i^-(0; \lambda) Z_i^-(0) &= P^-(0)( x_i^- + y_i^- + P_0^c(0) e^{\nu(\lambda) X_{i-1}} c_{i-1}^- ) \\
&+ P_i^-(0; \lambda) \Phi^s(0, -X_{i-1}; \lambda) a_{i-1}^- + (P_i^-(0; \lambda) - P^-(0))b_i^- + P_i^-(0; \lambda)(P_0^u(\lambda) - P_0^u(0))b_i^- \\
&+ (P_i^-(0; \lambda) - P^-(0)) P_0^c(0) e^{\nu(\lambda) X_{i-1}} c_{i-1}^- + P_i^-(0; \lambda) (P_0^c(\lambda) - P_0^c(0)) e^{\nu(\lambda) X_{i-1}} c_{i-1}^- \\
&+ \lambda^2 d_i P_i^-(0; \lambda) \int_{-X_{i-1}}^0 \Phi^s(0, y; \lambda) P_i^-(y; \lambda)^{-1} \tilde{H}_i^-(y) dy \\
&+ \lambda^2 d_i P_i^-(0; \lambda) \int_{-X_{i-1}}^0 \Phi^c(0, y; \lambda) P_i^-(y; \lambda)^{-1} \tilde{H}_i^-(y) dy  \\ 
P_i^+(0; \lambda) Z_i^+(0) &=  P^+(0)( x_i^+ + y_i^+ + P_0^c(0) e^{-\nu(\lambda)X_i} c_i^+ )\\
&+ P_i^+(0; \lambda) \Phi^u(0, X_i; \lambda) a_i^+ + (P_i^+(0; \lambda) - P^+(0)) b_i^+ + P_i^+(0; \lambda) (P_0^s(\lambda) - P_0^s(0)) b_i^+ \\
&+ (P_i^+(0; \lambda) - P^+(0))P_0^c(0) e^{-\nu(\lambda)X_i} c_i^+ + P_i^+(0; \lambda) (P_0^c(\lambda) - P_0^c(0)) e^{-\nu(\lambda)X_i} c_i^+\\
&+ \lambda^2 d_i P_i^+(0; \lambda) \int_{X_i}^0 \Phi^u(0, y; \lambda) P_i^+(y; \lambda)^{-1} \tilde{H}_i^+(y) dy \\
&+ \lambda^2 d_i P_i^+(0; \lambda) \int_{X_i}^0 \Phi^c(0, y; \lambda) P_i^+(y; \lambda)^{-1} \tilde{H}_i^+(y) dy \\
\end{align*}

With this setup, the projections on $Q'(0)$ and $Y^+ \oplus Y^-$ will either eliminate or act as the identity on the terms in the first lines of $P_i^-(0; \lambda) Z_i^-(0)$ and $P_i^+(0; \lambda) Z_i^+(0)$. Thus, applying the appropriate projections, we obtain an expression of the form

\begin{equation}\label{projxy}
\begin{pmatrix}x_i^- \\ x_i^+ \\ 
y_i^+ - y_i^- \end{pmatrix} + L_4(\lambda)_i(b, c^-, d) = 0
\end{equation}

To get a bound on $L_4$, we need to bound the individual terms from the fixed point equations above. Where appropriate, we only look at one of each term, i.e. only look at the ``positive'' piece or the ``negative'' piece. For convenience, we define

\[
\tilde{c}_i^\pm = e^{\pm \nu(\lambda) X_i} c_i^-
\]

\begin{enumerate}

\item For the $a_i$ terms, we substitute $a_i^+ = P_0^u(\lambda) D_i d + A_2(\lambda)_i^+(b, c^-, d)$ and $a_{i-1}^- = -P_0^s(\lambda) D_{i-1} d + A_2(\lambda)_{i-1}^-(b, c^-, d)$ and use the bounds for $A_2$.

\begin{align*}
|P_i^-(0; \lambda) &\Phi^s(0, -X_{i-1}; \lambda) a_{i-1}^-| \\
&\leq C \Big( e^{-2 \alpha X_{i-1}} (|b_{i-1}^+| + |b_i^-| + |c_{i-1}^-|) + e^{-(2 \alpha - \rho) X_{i-1}} |\lambda^2| + e^{-\alpha X_{i-1}}|D_{i-1}|)|d| \Big) \\
|P_i^+(0; \lambda) &\Phi^u(0, X_i; \lambda) a_i^+| \\
&\leq C \Big( e^{-2 \alpha X_i} (|b_i^+| + |b_{i+1}^-| + |c_i^-|) + e^{-(2 \alpha - \rho) X_i} |\lambda|^2|d| + e^{-\alpha X_i} |D_i||d| \Big)
\end{align*}

\item For the $b_i$ terms, we have

\begin{align*}
|(P_i^-(0; \lambda) &- P^-(0))b_i^- + P_i^-(0; \lambda)(P_0^u(\lambda) - P_0^u(0))b_i^-| \\
&\leq C ( e^{-\alpha X_m} + |\lambda|)|b_i^-|
\end{align*}

\item For the $c_i^-$ terms, we have

\begin{align*}
|(P_i^-(0; \lambda) &- P^-(0)) P_0^c(0) e^{\nu(\lambda) X_{i-1}} c_{i-1}^- + P_i^-(0; \lambda) (P_0^c(\lambda) - P_0^c(0)) e^{\nu(\lambda) X_{i-1}} c_{i-1}^- | \\
&\leq C (e^{-\alpha X_m} + |\lambda|)|\tilde{c}_{i-1}^+|)
\end{align*}

\item For the $c_i^+$ terms, we have

\begin{align*}
|(P_i^+(0; \lambda) &- P^+(0))P_0^c(0) e^{-\nu(\lambda)X_i} c_i^+ + P_i^+(0; \lambda) (P_0^c(\lambda) - P_0^c(0)) e^{-\nu(\lambda)X_i} c_i^+| \\
&\leq C (e^{-\alpha X_m} + |\lambda|)|e^{-\nu(\lambda)X_i} c_i^+|
\end{align*}

To put this in terms of $c_i^-$, we use the expression

\[
c_i^+ = c_i^- + P_0^c(\lambda) D_i d + A_2(\lambda)_i^c(b, c^-, d) )
\]

from Lemma \ref{inv1}, together with the bound for $A_2$.

\begin{align*}
e^{-\nu(\lambda)X_i} c_i^+ &= e^{-\nu(\lambda)X_i} c_i^- 
+ e^{-\nu(\lambda)X_i} P_0^c(\lambda) D_i d + e^{-\nu(\lambda)X_i} A_2(\lambda)_i^c(b, d)\\
&= e^{-\nu(\lambda)X_i} c_i^- + \mathcal{O}\Big( e^{-(\alpha - \rho) X_i} ( |\lambda| + e^{-\alpha X_i} ) |d|) + e^{-(\alpha - \rho) X_i} (|b_i^+| + |b_{i+1}^-| + |c_i^-|)\\
&+ e^{-(\alpha - 2 \rho) X_i} |\lambda|^2|d| + e^{-(\alpha - \rho) X_i} |D_i||d| )
\end{align*}

Thus we have the expression for $e^{-\nu(\lambda)X_i} c_i^+$

\begin{align}\label{tildecminus}
e^{-\nu(\lambda)X_i} c_i^+
&= e^{-\nu(\lambda)X_i} c_i^- + \mathcal{O}\Big( e^{-(\alpha - \rho) X_i} ( |b_i^+| + |b_{i+1}^-| + |c_i^-| + |\lambda||d| + |D_i||d|) \Big)
\end{align}

which gives us the overall estimate

\begin{align*}
&|(P_i^+(0; \lambda) - P^+(0))P_0^c(0) e^{-\nu(\lambda)X_i} c_i^+ + P_i^+(0; \lambda) (P_0^c(\lambda) - P_0^c(0)) e^{-\nu(\lambda)X_i} c_i^+| \\
&\leq C \Big( (e^{-\alpha X_m} + |\lambda|)( e^{-\nu(\lambda)X_i} c_i^- + e^{-(\alpha - \rho) X_i} ( |b_i^+| + |b_{i+1}^-| + |c_i^-| + |\lambda||d| + |D_i||d|) \Big) \\
&\leq C \Big( (e^{-\alpha X_m} + |\lambda|)( |\tilde{c}_i^-| + e^{-(\alpha - \rho) X_i} ( |b_i^+| + |b_{i+1}^-| + |c_i^-| + |\lambda||d| + |D_i||d|) \Big)
\end{align*}

\item The bound on the integral terms is determined by the integrals involving the center subspace, since there is potential growth in that subspace.

\begin{align*}
\left| \lambda^2 d_i P_i^-(0; \lambda) \int_{-X_{i-1}}^0 \Phi^c(0, y; \lambda) P_i^-(y; \lambda)^{-1} \tilde{H}_i^-(y) dy \right| &\leq C |\lambda|^2 |d| \int_{-X_{i-1}}^0 e^{-\rho y} e^{\alpha y} dy \\
&\leq C |\lambda|^2 |d|
\end{align*}

\end{enumerate}

Putting all these together, we obtain the bound for $L_4(\lambda)_i(b, c, d)$.

\begin{align}\label{L4bound}
L_4(\lambda)_i(b, c, d) &\leq 
C\Big( (|\lambda| + e^{-\tilde{\alpha}X_m})|b| 
+ (|\lambda| + e^{-\tilde{\alpha}X_m}) |\tilde{c}_{i-1}^+| + |\tilde{c}_i^-|) + (|\lambda| + e^{-\tilde{\alpha}X_m})( e^{-\alpha X_{i-1}} |c_{i-1}^-| + e^{-\alpha X_i} |c_i^-| ) \nonumber \\
&+ ( e^{-\tilde{\alpha}X_m} |D| + e^{-\tilde{\alpha}X_m}|\lambda| + |\lambda|^2)|d| \Big) 
\end{align}

Since $|\lambda|, e^{-\alpha X_m} < \delta$, this becomes

\begin{align}\label{L4bound}
L_4(\lambda)_i(b, c, d) &\leq C \delta |b| + 
C\Big( (|\lambda| + e^{-\tilde{\alpha}X_m}) |\tilde{c}_{i-1}^+| + |\tilde{c}_i^-|) + (|\lambda| + e^{-\tilde{\alpha}X_m})( e^{-\alpha X_{i-1}} |c_{i-1}^-| + e^{-\alpha X_i} |c_i^-| ) \nonumber \\
&+ ( e^{-\tilde{\alpha}X_m} |D| + e^{-\tilde{\alpha}X_m}|\lambda| + |\lambda|^2)|d| \Big) 
\end{align}

which uniform in $|b|$. Define the map

\[
J_2: \left( \bigoplus_{j=1}^n \C V^+ \oplus \C V^- \right) \oplus
\left( \bigoplus_{j=1}^n E^+ \oplus E^- \right) 
\rightarrow \bigoplus_{j=1}^n \C V^+ \oplus \C V^- \oplus (E^+ \oplus E^-)
\]

by 

\[
J_2( (x_i^+, x_i^-),(y_i^+, y_i^-))_i = ( x_i^+, x_i^-, y_i^+ - y_i^- )
\]

Since $\C^{2m} = E^s(0) \oplus E^u(0) = \C V^+ \oplus \C V^- \oplus (E^+ \oplus E^-)$, $J_2$ is an isomorphism. Using this as the fact that $b_i = (x_i^- + y_i^-, x_i^+ + y_i^+)$, we can write \eqref{projxy} as

\begin{equation}\label{projxy2}
J_2( (x_i^+, x_i^-),(y_i^+, y_i^-))_i 
+ L_4(\lambda)_i(b_i, 0, 0) + L_4(\lambda)_i(0, c^-, d) = 0
\end{equation}

Consider the map

\begin{align*}
S_2(b)_i &= J_2( (x_i^+, x_i^-),(y_i^+, y_i^-))_i 
+ L_4(\lambda)_i(b_i, 0, 0) 
\end{align*}

Substituting this in \eqref{projxy2}, we have

\begin{align*}
S_2(b) &= -L_4(\lambda)(0, c^-, d)
\end{align*}

For sufficiently small $\delta$, the operator $S_2(b)$ is invertible. Thus we can solve for $b$ to get

\begin{align}
b = B_1(\lambda)(c^-,d) 
= -S_2^{-1} L_4(\lambda)(0, c^-, d)
\end{align}

The bound on $B_1$ is given by the bound on $L_4$, where we note which piece is involved.

\begin{align*}
|B_1(\lambda)_i(\tilde{c}, d)| \leq C\Big( 
(|\lambda| + e^{-\tilde{\alpha}X_m})( |\tilde{c}_{i-1}^+| + |\tilde{c}_i^-|
+ e^{-\alpha X_{i-1}} |c_{i-1}^-| + e^{-\alpha X_i} |c_i^-|) + ( e^{-\tilde{\alpha}X_m} |D| + e^{-\tilde{\alpha}X_m}|\lambda| + |\lambda|^2)|d| \Big)
\end{align*}

Since $|e^{-\alpha X_{i-1}} |c_{i-1}^-| \leq |e^{-\tilde{\alpha} X_{i-1}} |\tilde{c}_{i-1}^+|$ and $|e^{-\alpha X_i} |c_i^-| \leq |e^{-\tilde{\alpha} X_i} |\tilde{c}_i^-|$, this simplifies to

\begin{align*}
|B_1(\lambda)_i(\tilde{c}, d)| \leq C\Big( 
(|\lambda| + e^{-\tilde{\alpha}X_m})( |\tilde{c}_{i-1}^+| + |\tilde{c}_i^-|) + ( e^{-\tilde{\alpha}X_m} |D| + e^{-\tilde{\alpha}X_m}|\lambda| + |\lambda|^2)|d| \Big)
\end{align*}

We can plug this into the bound for $A_1$ to get $A_3$ with bound

\begin{align*}
|A_3&(\lambda)_i(c^-, d)|
\leq C \Big(  
e^{-\alpha X_m} (|\lambda| + e^{-\tilde{\alpha}X_m})(|\tilde{c}_{i-1}^+| + |\tilde{c}_{i+1}^-|)  \\
&+|c_i^-| + e^{-(\alpha - \rho) X_i} |\lambda^2||d| + |D_i||d| \Big)
\end{align*} 

We can also plug this into the bound for $A_2$ to get $A_4$ with bound

\begin{align*}
|A_4&(\lambda)_i(c^-, d)|
\leq C \Big( 
e^{-\alpha X_i} (|\lambda| + e^{-\tilde{\alpha}X_m})(|\tilde{c}_{i-1}^+| + |\tilde{c}_{i+1}^-|) + e^{-(\alpha - \rho) } |c_i^-| + e^{-\tilde{\alpha} X_m} |\lambda|^2|d| + e^{-\alpha X_m}|D||d| \Big)
\end{align*} 

Finally, we plug $B_1$ into \eqref{tildecminus2} to obtain an expression for $e^{-\nu(\lambda)X_i} c_i^+$.

\begin{align*}
e^{-\nu(\lambda)X_i} &c_i^+ = e^{-\nu(\lambda)X_i} c_i^- 
+ \mathcal{O}\Big( e^{-\tilde{\alpha}X_m} (|\lambda| + e^{-\tilde{\alpha}X_m})( |\tilde{c}_{i-1}^+| + |\tilde{c}_{i+1}^-|) 
+ e^{-(\alpha - 2 \rho) X_i}|c_i^-| +  e^{-\tilde{\alpha}X_m}(|\lambda| + |D_i|)|d| \Big)
\end{align*}

\end{proof}
\end{lemma}

Up to this point, we have solved uniquely for everything except for the $c_i^-$ and $d$. To do that, we will compute the jump conditions in the direction of $\Psi(0)$ and $\Psi^c(0)$.

\subsection{Jump Conditions}

\subsubsection{Center Adjoint Jump}

First, we compute the jump in the direction of $\Psi^c(0)$. Before we do that, we prove the following lemma regarding inner products with $\Psi^c(0)$ and $\Psi(0)$.

% lemma : inner products with Psi and Psi^c

\begin{lemma}\label{PsiIP}
We have the following expressions involving the inner product with $\Psi^c(0)$.
\begin{enumerate}[(i)]
	\item $\langle \Phi^c(0), P^\pm(0) V \rangle = V$ for all $V \in E^c(0)$.
	\item $\langle \Phi(0), P^\pm(0) V \rangle = 0$ for all $V \in E^c(0)$.
	\item $\langle \Phi^c(0), P^-(0) V \rangle = 0$ and $\langle \Phi(0), P^-(0) V \rangle = 0$ for all $V \in E^u(0)$.
	\item $\langle \Phi^c(0), P^+(0) V \rangle = 0$ and $\langle \Phi(0), P^+(0) V \rangle = 0$ for all $V \in E^s(0)$.
\end{enumerate}
\begin{proof}
For (i), recall that $E^c(0) = \text{span }\{ V_0 \}$, where $V_0$ is the eigenvector of $A(0)$ corresponding to the eigenvalue 0. Furthermore, the constant function $Z(x) = V_0$ solves $Z' = A(0) Z$ with initial condition $V_0$. Let $W^-(x) = P^-(x) Z(x) = P^-(x) V_0$. By the Conjugation Lemma, we can write
\[
W^-(x) = V_0 + \mathcal{O}({e^{-\tilde{\alpha}|x|}})
\]
Since the inner product $\langle \Phi^c(x), W^-(x) \rangle$ is constant in $x$, sending $x \rightarrow -\infty$, we conclude by the continuity of the inner product that
\[
\langle \Phi^c(0), W^-(0) \rangle = \langle W_0, V_0 \rangle = 1 
\]
Thus we conclude that $\langle \Phi^c(0), P^-(0) V \rangle = V$ for all $V \in E^c(0)$. Similarly, the same holds for $\langle \Phi^c(0), P^+(0) V \rangle$.\\

For (ii), we use the same argument as in (i), except we look at the inner product $\langle \Phi(x), W^-(x) \rangle$. Since this is constant in $x$, we send $x \rightarrow \infty$. This time, $W^-(x)$ remains bounded, but $\Phi(x)$ decays to 0, thus by the continuity of the inner product, we conclude that $\langle \Phi(0), W^-(0) \rangle = 0$, from which (ii) follows.\\

For (iii) and (iv), we note that $P^-(0)E^u = \C Q'(0) \oplus Y^-$ and $P^+(0)E^u = \C Q'(0) \oplus Y^+$. The result follows since $\Psi^c(0), \Psi(0) \perp \C Q'(0) \oplus Y^+ \oplus Y^-$.
\end{proof}
\end{lemma}

% jump lemma : center adjoint

\begin{lemma}\label{jumpcenteradj}

The jumps in the direction of $\Psi^c(0)$ are given by

\begin{align}\label{xic}
\xi^c_i = e^{-\nu(\lambda) X_i} c_i^- - e^{\nu(\lambda) X_{i-1}} c_{i-1}^- - \lambda^2 d_i M^c + R^c(\lambda)_i(c^-, d)
\end{align}

where $M^c$ is the center Melnikov integral

\begin{equation}\label{Mc}
M^c =  \int_{-\infty}^\infty \langle \Psi^c(y), H(y) \rangle dy 
\end{equation}

and the remainder term $R^c_i(c^-, d)$ has bound

\begin{align}\label{Rc}
R^c&(c^-, d)_i \leq C \Big(
(|\lambda| + e^{-\alpha X_m})(|\tilde{c}_{i-1}^+| + |\tilde{c}_{i}^-|) + e^{-\alpha X_m}(|\lambda| + e^{-\alpha X_m})( |\tilde{c}_{i-2}^+| + |\tilde{c}_{i+1}^-|)  \nonumber \\
&+ e^{-\alpha X_i} |c_i^-| + (|\lambda| + e^{-\tilde{\alpha} X_m})( e^{-\alpha X_{i-1}} |c_{i-1}^-| + e^{-\alpha X_{i-2}} |c_{i-2}^-| + e^{-\alpha X_{i+1}} |c_{i+1}^-|) \\
&+ (|\lambda| + e^{-\tilde{\alpha} X_m})(|\lambda| + e^{-\alpha X_m})|d|
\Big) \nonumber
\end{align}

The jump conditions can be written as the matrix equation

\begin{equation}\label{matrixjumpc}
K(\lambda) c + D_2 d = 0
\end{equation}

where

\begin{align*}
K(\lambda) =  
\begin{pmatrix}
e^{-\nu(\lambda)X_1} & & & & & -e^{\nu(\lambda)X_0} \\
-e^{\nu(\lambda)X_1} & e^{-\nu(\lambda)X_2} \\
& -e^{\nu(\lambda)X_2} & e^{-\nu(\lambda)X_3} \\
\vdots & & \vdots & &&  \vdots \\
& & & & -e^{\nu(\lambda)X_{n-1}} & e^{-\nu(\lambda)X_0} 
\end{pmatrix}
\end{align*}

and

\begin{align*}
D_2 &= \mathcal{O}((|\lambda| + e^{-\tilde{\alpha} X_m})(|\lambda| + e^{-\alpha X_m}))
\end{align*}

\begin{proof}

Recall from the previous section that $P_i^\pm(0; \lambda) Z_i^\pm(0)$ are given by

\begin{align*}
P_i^-(0; \lambda) Z_i^-(0) &= P^-(0)( b_i^- + P_0^c(0) e^{\nu(\lambda) X_{i-1}} c_{i-1}^- ) \\
&+ P_i^-(0; \lambda) \Phi^s(0, -X_{i-1}; \lambda) a_{i-1}^- + (P_i^-(0; \lambda) - P^-(0))b_i^- + P_i^-(0; \lambda)(P_0^u(\lambda) - P_0^u(0))b_i^- \\
&+ (P_i^-(0; \lambda) - P^-(0)) P_0^c(0) e^{\nu(\lambda) X_{i-1}} c_{i-1}^- + P_i^-(0; \lambda) (P_0^c(\lambda) - P_0^c(0)) e^{\nu(\lambda) X_{i-1}} c_{i-1}^- \\
&+ \lambda^2 d_i P_i^-(0; \lambda) \int_{-X_{i-1}}^0 \Phi^s(0, y; \lambda) P_i^-(y; \lambda)^{-1} \tilde{H}_i^-(y) dy \\
&+ \lambda^2 d_i P_i^-(0; \lambda) \int_{-X_{i-1}}^0 \Phi^c(0, y; \lambda) P_i^-(y; \lambda)^{-1} \tilde{H}_i^-(y) dy  \\ 
P_i^+(0; \lambda) Z_i^+(0) &=  P^+(0)( b_i^+ + P_0^c(0) e^{-\nu(\lambda)X_i} c_i^+ )\\
&+ P_i^+(0; \lambda) \Phi^u(0, X_i; \lambda) a_i^+ + (P_i^+(0; \lambda) - P^+(0)) b_i^+ + P_i^+(0; \lambda) (P_0^s(\lambda) - P_0^s(0)) b_i^+ \\
&+ (P_i^+(0; \lambda) - P^+(0))P_0^c(0) e^{-\nu(\lambda)X_i} c_i^+ + P_i^+(0; \lambda) (P_0^c(\lambda) - P_0^c(0)) e^{-\nu(\lambda)X_i} c_i^+\\
&+ \lambda^2 d_i P_i^+(0; \lambda) \int_{X_i}^0 \Phi^u(0, y; \lambda) P_i^+(y; \lambda)^{-1} \tilde{H}_i^+(y) dy \\
&+ \lambda^2 d_i P_i^+(0; \lambda) \int_{X_i}^0 \Phi^c(0, y; \lambda) P_i^+(y; \lambda)^{-1} \tilde{H}_i^+(y) dy \\
\end{align*}

We will compute the leading order terms first.

\begin{enumerate}
\item For the leading order terms involving $c$, using Lemma \ref{PsiIP}, we have

\begin{align*}
\langle \Psi(0), &P^-(0) P_0^c(0) e^{\nu(\lambda) X_{i-1}} c_{i-1}^- - P^+(0) P_0^c(0) e^{-\nu(\lambda)X_i} c_i^+) = e^{\nu(\lambda) X_{i-1}} c_{i-1}^- - e^{-\nu(\lambda)X_i} c_i^+ 
\end{align*}

For the higher order terms involving $c$, from the previous section we have
\begin{align*}
|(P_i^-(0; \lambda) &- P^-(0)) P_0^c(0) e^{\nu(\lambda) X_{i-1}} c_{i-1}^- + P_i^-(0; \lambda) (P_0^c(\lambda) - P_0^c(0)) e^{\nu(\lambda) X_{i-1}} c_{i-1}^-| \\
&\leq C (|\lambda| + e^{-\alpha X_m}) |e^{\nu(\lambda) X_{i-1}} c_{i-1}^-|\\
&= C (|\lambda| + e^{-\alpha X_m}) |\tilde{c}_{i-1}^+|
\end{align*}

and

\begin{align*}
|(P_i^+(0; \lambda) &- P^+(0))P_0^c(0) e^{-\nu(\lambda)X_i} c_i^+ + P_i^+(0; \lambda) (P_0^c(\lambda) - P_0^c(0)) e^{-\nu(\lambda)X_i} c_i^+| \\
&\leq C (|\lambda| + e^{-\alpha X_m}) |e^{-\nu(\lambda)X_i} c_i^+|
\end{align*}

All that remains is to write $e^{-\nu(\lambda)X_i} c_i^+$ in terms of $e^{-\nu(\lambda)X_i} c_i^-$ using Lemma \ref{inv2}. For the lower order term involving $c_i^+$, we have

\begin{align*}
e^{-\nu(\lambda)X_i} c_i^+ &= e^{-\nu(\lambda)X_i} c_i^- 
+ \mathcal{O}\Big( e^{-\tilde{\alpha}X_m} (|\lambda| + e^{-\tilde{\alpha}X_m})( |\tilde{c}_{i-1}^+| + |\tilde{c}_{i+1}^-|) 
+ e^{-(\alpha - 2 \rho) X_i}|c_i^-| +  e^{-\tilde{\alpha}X_m}(|\lambda| + |D_i|)|d| \Big) \\
&= e^{-\nu(\lambda)X_i} c_i^- 
+ \mathcal{O}\Big( e^{-\tilde{\alpha}X_m} (|\lambda| + e^{-\tilde{\alpha}X_m})( |\tilde{c}_{i-1}^+| + |\tilde{c}_{i+1}^-|) 
+ e^{-\tilde{\alpha}X_i}|\tilde{c}_i^-| + e^{-\tilde{\alpha}X_m}(|\lambda| + |D_i|)|d| \Big)
\end{align*}

For the higher order term involving $c_i^+$, we have

\begin{align*}
|(P_i^+(0; \lambda) &- P^+(0))P_0^c(0) e^{-\nu(\lambda)X_i} c_i^+ + P_i^+(0; \lambda) (P_0^c(\lambda) - P_0^c(0)) e^{-\nu(\lambda)X_i} c_i^+| \\
&\leq C (|\lambda| + e^{-\alpha X_m}) |e^{-\nu(\lambda)X_i} c_i^+| \\
&\leq C (|\lambda| + e^{-\alpha X_m}) \Big( |e^{-\nu(\lambda)X_i} c_i^-| + e^{-\tilde{\alpha}X_m} (|\lambda| + e^{-\tilde{\alpha}X_m})( |\tilde{c}_{i-1}^+| + |\tilde{c}_{i+1}^-|) \\
&+ e^{-(\alpha - 2 \rho) X_i}|c_i^-| +  e^{-\tilde{\alpha}X_m}(|\lambda| + |D_i|)|d| \Big) \\
&\leq C (|\lambda| + e^{-\alpha X_m}) \Big( |\tilde{c}_i^-| + e^{-\tilde{\alpha}X_m} (|\lambda| + e^{-\tilde{\alpha}X_m})( |\tilde{c}_{i-1}^+| + |\tilde{c}_{i+1}^-|) +  e^{-\tilde{\alpha}X_m}(|\lambda| + |D_i|)|d| \Big)
\end{align*}

Combining all of these, the terms involving $c_i$ are given by

\begin{align*}
e^{\nu(\lambda) X_{i-1} } &c_{i-1}^- - e^{-\nu(\lambda)X_i} c_i^- + \mathcal{O}\Big( (|\lambda| + e^{-\alpha X_m})(|\tilde{c}_{i-1}^+| + |\tilde{c}_i^-|) + e^{-\tilde{\alpha}X_m} (|\lambda| + e^{-\tilde{\alpha}X_m})|\tilde{c}_{i+1}^-| \\
&+ e^{-\tilde{\alpha}X_m}(|\lambda| + |D_i|)|d| \Big)  
\end{align*}

\item The integral term involving the center subspace will give us the center Melnikov integral

\begin{align*}
&\langle \Psi^c(0), P_i^-(0; \lambda) \int_{-X_{i-1}}^0 \Phi^c(0, y; \lambda) P_i^-(y; \lambda) \tilde{H}_i^-(y) dy \rangle \\
&= \langle \Psi^c(0), \int_{-X_{i-1}}^0 P_i^-(0; \lambda) \Phi^c(0, y; \lambda) P_i^-(y; \lambda)^{-1} \tilde{H}_i^-(y) dy \rangle \\
&= \langle \Psi^c(0), \int_{-X_{i-1}}^0 P^-(0) \Phi^c(0, y; 0) P^-(y)^{-1} \tilde{H}_i^-(y) dy \rangle + \mathcal{O}(|\lambda| + e^{-\alpha X_m}) \\
&= \int_{-X_{i-1}}^0 \langle \Psi^c(0), \Theta^c(0, y) \tilde{H}_i^-(y) \rangle dy + \mathcal{O}(|\lambda| + e^{-\alpha X_m}) \\
&= \int_{-X_{i-1}}^0 \langle \Theta^c(y, 0)^* \Psi^c(0), H(y) \rangle dy + \int_{-X_{i-1}}^0 \langle \Psi^c(0), \Theta^c(0, y) \Delta H_i^-(y) \rangle dy + \mathcal{O}(|\lambda| + e^{-\alpha X_m}) \\
&= \int_{-\infty}^0 \langle \Psi^c(y), H(y) \rangle dy + \int_{-X_{i-1}}^0 \langle \Psi^c(0), \Theta^c(0, y) \Delta H_i^-(y) \rangle dy + \mathcal{O}(e^{-\alpha X_m} + |\lambda|) \\
\end{align*}

For the integral involving $\Delta H_i^-(y)$,

\begin{align*}
\left| \int_{-X_{i-1}}^0 \langle \Psi^c(0), \Theta^c(0, y) \Delta H_i^-(y) \rangle dy \right| &\leq C \int_{-X_{i-1}}^0 e^{-\rho y} e^{-\alpha X_{i-1}} e^{-\alpha(X_{i-1} + y)} dy \\
&\leq C e^{-(\alpha - \rho)X_{i-1}} \int_{-X_{i-1}}^0 e^{-\alpha(X_{i-1} + y)} dy \\
&\leq C e^{-\tilde{\alpha}X_{i-1}}
\end{align*}

Thus we have

\begin{align*}
&\langle \Psi^c(0), P^-(0; \beta_i^\pm, \lambda) \int_{-X_{i-1}}^0 \Phi^c(0, y; \lambda) P^-(y; \beta_i^\pm, \lambda) \tilde{H}_i^-(y) dy \rangle \\
&= \int_{-\infty}^0 \langle \Psi^c(y), H(y) \rangle dy + \mathcal{O}(e^{-\tilde{\alpha} X_m} + |\lambda|) \\
\end{align*}

The ``positive'' integral is similar, and gives us the other half of the center Melnikov integral.

\end{enumerate}

The remaining terms are higher order. We will evaluate them in turn.

\begin{enumerate}

\item For the term involving $a$, we plug in $A_4$.

\begin{align*}
P_i^-(0; \lambda) \Phi^s(0, -X_{i-1}; \lambda) a_{i-1}^- &= 
P_i^-(0; \lambda) \Phi^s(0, -X_{i-1}; \lambda) P_0^s(\lambda) D_i d +
P_i^-(0; \lambda) \Phi^s(0, -X_{i-1}; \lambda) A_4(\lambda)_{i-1}^-(c^-, d) \\
&= P_i^-(0; \lambda) \Phi^s(0, -X_{i-1}; \lambda) A_4(\lambda)_{i-1}^-(c^-, d) + \mathcal{O}( e^{-\alpha X_m} |D|)
\end{align*}

Combining this with the bound for $A_4$, we have

\begin{align*}
|P_i^-(0; \lambda) &\Phi^s(0, -X_{i-1}; \lambda) a_{i-1}^-| \\
&\leq C\Big( 
e^{-2 \alpha X_m} (|\lambda| + e^{-\tilde{\alpha}X_m})(|\tilde{c}_{i-2}^+| + |\tilde{c}_i^-|) + e^{-\alpha X_m} e^{-(\alpha - \rho)X_{i-1}}|c_{i-1}^-|\\
&+ (e^{-(\alpha + \tilde{\alpha}) X_m} |\lambda|^2 + e^{-\alpha X_m}|D| + e^{-2 \alpha X_m}|\lambda|) |d| \Big)
\end{align*}

Similarly, we have

\begin{align*}
|P_i^+(0; \lambda) &\Phi^u(0, X_i; \lambda) a_i^+| \\
&\leq C\Big( 
e^{-2 \alpha X_m} (|\lambda| + e^{-\tilde{\alpha}X_m})(|\tilde{c}_{i-1}^+| + |\tilde{c}_{i+1}^-|) + e^{-\alpha X_m} e^{-(\alpha - \rho)X_i}|c_i^-| \\
&+ (e^{-(\alpha + \tilde{\alpha}) X_m} |\lambda|^2 + e^{-2 \alpha X_m}|D| + e^{-\alpha X_m}|\lambda|) |d| \Big)
\end{align*}

\item For the terms involving $b$, note that by Lemma \ref{PsiIP}, the terms $P^-(0) b_i^-$ and $P^+(0)b_i^+$ are eliminated outright when we take the inner product with $\Psi^c(0)$. For the other terms, we use the estimate for $B_1$ from Lemma \ref{inv2}.

\begin{align*}
&|(P_i^-(0; \lambda) - P^-(0))b_i^- + P_i^-(0; \lambda)(P_0^u(\lambda) - P_0^u(0))b_i^-| \\
&\leq C(|\lambda| + e^{-\alpha X_m}) |B_1(c, \tilde{c}, d)| \\
&\leq C(|\lambda| + e^{-\alpha X_m}) \Big( 
(|\lambda| + e^{-\tilde{\alpha}X_m})( |\tilde{c}_{i-1}^+| + |\tilde{c}_i^-|)
+  ( e^{-\tilde{\alpha}X_m} |D| + e^{-\tilde{\alpha}X_m}|\lambda| + |\lambda|^2)|d| \Big)
\end{align*}

\item For the non-center integral terms, we have the bound

\begin{align*}
&\left| P_i^-(0; \lambda) 
\int_{-X_{i-1}}^0 \Phi^s(0, y; \lambda) \lambda^2 d_i P^-(y; \beta_i^-, \lambda)^{-1} \tilde{H}_i^-(y) dy \right| \\
&\leq C |\lambda|^2 |d_i| \int_{-X_{i-1}}^0 e^{\alpha y} e^{\alpha y} dy \\
&\leq C |\lambda|^2 |d_i|
\end{align*}

\end{enumerate}

Putting all of this together, we obtain the center jump expressions

\begin{align*}
\xi^c_i = e^{-\nu(\lambda) X_i} c_i^- - e^{\nu(\lambda) X_{i-1}} c_{i-1}^- - \lambda^2 d_i M^c + R^c(\lambda)_i(c^-, d)
\end{align*}

where $M^c$ is the center Melnikov integral

\[
M^c = \int_{-\infty}^\infty \langle \Psi^c(y), H(y) \rangle dy 
\]

The remainder term $R^c_i(c^-, d)$ has bound

\begin{align*}
R^c&(c^-, d)_i \leq C \Big(
(|\lambda| + e^{-\alpha X_m})(|\tilde{c}_{i-1}^+| + |\tilde{c}_{i}^-|) + e^{-\alpha X_m}(|\lambda| + e^{-\alpha X _m})( |\tilde{c}_{i-2}^+| + |\tilde{c}_{i+1}^-|)  \\
&+ e^{-\alpha X_m}( e^{-(\alpha - \rho) X_i}|c_i^-| + e^{-(\alpha - \rho) X_{i-1}}|c_{i-1}^-| )
+ (|\lambda| + e^{-\tilde{\alpha} X_m})(|\lambda| + e^{-\alpha X_m})|d|
\Big)
\end{align*}

where we used the estimate $|D| = \mathcal{O}(e^{-\alpha X_m})$. We can absorb the remainders terms involving $|c_i^-|$ and $|c_{i-1}^-|$ into those involving $|\tilde{c}_{i-1}^+|$ and $|\tilde{c}_{i}^-|$ to get

\begin{align*}
R^c&(c^-, d)_i \leq C \Big(
(|\lambda| + e^{-\alpha X_m})(|\tilde{c}_{i-1}^+| + |\tilde{c}_{i}^-|) + e^{-\alpha X_m}(|\lambda| + e^{-\alpha X _m})( |\tilde{c}_{i-2}^+| + |\tilde{c}_{i+1}^-|)  \\
&+ (|\lambda| + e^{-\tilde{\alpha} X_m})(|\lambda| + e^{-\alpha X_m})|d|
\Big)
\end{align*}

Writing this in matrix form, we have

\[
C_1 K(\lambda) \tilde{C}_1 c + (-\lambda^2 M^c I + D_1)d = 0
\]

where

\begin{align*}
K(\lambda) =  
\begin{pmatrix}
e^{-\nu(\lambda)X_1} & & & & & -e^{\nu(\lambda)X_0} \\
-e^{\nu(\lambda)X_1} & e^{-\nu(\lambda)X_2} \\
& -e^{\nu(\lambda)X_2} & e^{-\nu(\lambda)X_3} \\
\vdots & & \vdots & &&  \vdots \\
& & & & -e^{\nu(\lambda)X_{n-1}} & e^{-\nu(\lambda)X_0} 
\end{pmatrix}
\end{align*}

and

\begin{align*}
C_1 &= I + \text{diag}(\mathcal{O}(|\lambda| + e^{-\alpha X_m})) 
+ \mathcal{O}(e^{-\alpha X_m}( |\lambda| + e^{-\alpha X_m}))\\
\tilde{C}_1 &= I + \text{diag}(\mathcal{O}(|\lambda| + e^{-\alpha X_m})) 
+ \mathcal{O}(e^{-\alpha X_m}( |\lambda| + e^{-\alpha X_m}))\\
D_1 &= \mathcal{O}((|\lambda| + e^{-\tilde{\alpha} X_m})(|\lambda| + e^{-\alpha X_m}))
\end{align*}

where $\text{diag}(\mathcal{O}(|\lambda| + e^{-\alpha X_m}))$ indicates a diagonal matrix whose diagonal entries are all of order $\mathcal{O}(|\lambda| + e^{-\alpha X_m})$.\\

Since $C_1, \tilde{C}_1$ are small perturbations of the identity matrix, they are invertible for sufficiently large $X_m$ (and the inverse is bounded with operator norm approximately 1). Thus we can write the matrix equation as

\[
K(\lambda) c + C_1^{-1} (-\lambda^2 M^c I + D_1) \tilde{C}_1^{-1}d = 0
\]

Let $D_2 = C_1^{-1} (-\lambda^2 M^c I + D_1) \tilde{C}_1^{-1}$. Substituting this and combining bounds, we have

\[
K(\lambda) c + D_2 d = 0
\]
with

\[
D_2 = \mathcal{O}((|\lambda| + e^{-\tilde{\alpha} X_m})(|\lambda| + e^{-\alpha X_m}))
\]

\end{proof}
\end{lemma}

\subsubsection{Decaying adjoint jump}

Finally, we compute the jump in the direction of $\Psi(0)$.

\begin{lemma}\label{jumpadj}

The jumps in the direction of $\Psi(0)$ are given by

\begin{align}\label{xi}
\xi_i = \langle \Psi(X_i), Q'(-X_i) \rangle (d_{i+1} - d_i)
+ \langle \Psi(-X_i), Q'(X_i) \rangle (d_i - d_{i-1})
- \lambda_2 d_i M + R_i(\lambda)(c^-, d)
\end{align}

where $M$ is the higher order Melnikov integral

\begin{equation}\label{M}
M = \int_{-\infty}^\infty \langle \Psi(y), H(y) \rangle dy 
\end{equation}

and the remainder term has bound

\begin{align}\label{R}
|R(\lambda)&(\tilde{c}, d)| \leq C \Big(
(|\lambda| + e^{-\alpha X_m})(|\tilde{c}_{i-1}^+| + |\tilde{c}_{i}^-|) + (|\lambda| + e^{-\tilde{\alpha} X_m})(e^{-\alpha X_m} + |\lambda|) ( |\tilde{c}_{i-2}^+| + |\tilde{c}_{i+1}^-|) \\
&+ (|\lambda| + e^{-\tilde{\alpha} X_m})( e^{-\alpha X_{i-2}} |c_{i-2}^-| + e^{-\alpha X_{i-1}} |c_{i-1}^-| + e^{-\alpha X_i} |c_i^-| + e^{-\alpha X_{i+1}} |c_{i+1}^-|) \nonumber \\
&+ (|\lambda| + e^{-\tilde{\alpha} X_m})(|\lambda| + e^{-\alpha X_m})^2 |d| \nonumber \Big)
\end{align}

We can write these conditions in matrix form as

\begin{equation}
(C_3 K(\lambda) + K(\lambda) \tilde{C}_3) c + (A -\lambda^2 M I + D_3)d = 0
\end{equation}

where the matrix $K(\lambda)$ is defined in Lemma \ref{jumpcenteradj} and the matrix $A$ is given by

\begin{align*}
A &= \begin{pmatrix}
-a_0 + \tilde{a}_1 & a_0 - \tilde{a}_1 \\
-\tilde{a}_0 + a_1 & \tilde{a}_0 - a_1
\end{pmatrix} && n = 2 \\
A &= \begin{pmatrix}
\tilde{a}_{n-1} - a_0 & a_0 & & & \dots & -\tilde{a}_{n-1}\\
-\tilde{a}_0 & \tilde{a}_0 - a_1 &  a_1 \\
& -\tilde{a}_1 & \tilde{a}_1 - a_2 &  a_2 \\
& & \vdots & & \vdots \\
a_{n-1} & & & & -\tilde{a}_{n-2} & \tilde{a}_{n-2} - a_{n-1} \\
\end{pmatrix} && n > 2
\end{align*}

with

\begin{align*}
a_i &= \langle \Psi(X_i), Q'(-X_i) \rangle \\
\tilde{a}_i &= \langle \Psi(-X_i), Q'(X_i) \rangle
\end{align*}

and we have bounds

\begin{align*}
C_3 &= \text{diag}(\mathcal{O}(|\lambda| + e^{-\alpha X_m})) 
+ \mathcal{O}((|\lambda| + e^{-\tilde{\alpha} X_m})( |\lambda| + e^{-\alpha X_m})) \\
\tilde{C}_3 &= \text{diag}(\mathcal{O}(|\lambda| + e^{-\alpha X_m})) 
+ \mathcal{O}((|\lambda| + e^{-\tilde{\alpha} X_m})( |\lambda| + e^{-\alpha X_m})) \\
D_3 &= \mathcal{O}((|\lambda| + e^{-\tilde{\alpha} X_m})(|\lambda| + e^{-\alpha X_m})^2)
\end{align*}

\begin{proof}

Recall that the terms $P_i^\pm(0; \lambda) Z_i^\pm(0)$ are given by

\begin{align*}
P_i^-(0; \lambda) Z_i^-(0) &= P^-(0)( b_i^- + P_0^c(0) e^{\nu(\lambda) X_{i-1}} c_{i-1}^- ) \\
&+ P_i^-(0; \lambda) \Phi^s(0, -X_{i-1}; \lambda) a_{i-1}^- + (P_i^-(0; \lambda) - P^-(0))b_i^- + P_i^-(0; \lambda)(P_0^u(\lambda) - P_0^u(0))b_i^- \\
&+ (P_i^-(0; \lambda) - P^-(0)) P_0^c(0) e^{\nu(\lambda) X_{i-1}} c_{i-1}^- + P_i^-(0; \lambda) (P_0^c(\lambda) - P_0^c(0)) e^{\nu(\lambda) X_{i-1}} c_{i-1}^- \\
&+ \lambda^2 d_i P_i^-(0; \lambda) \int_{-X_{i-1}}^0 \Phi^s(0, y; \lambda) P_i^-(y; \lambda)^{-1} \tilde{H}_i^-(y) dy \\
&+ \lambda^2 d_i P_i^-(0; \lambda) \int_{-X_{i-1}}^0 \Phi^c(0, y; \lambda) P_i^-(y; \lambda)^{-1} \tilde{H}_i^-(y) dy  \\ 
P_i^+(0; \lambda) Z_i^+(0) &=  P^+(0)( b_i^+ + P_0^c(0) e^{-\nu(\lambda)X_i} c_i^+ )\\
&+ P_i^+(0; \lambda) \Phi^u(0, X_i; \lambda) a_i^+ + (P_i^+(0; \lambda) - P^+(0)) b_i^+ + P_i^+(0; \lambda) (P_0^s(\lambda) - P_0^s(0)) b_i^+ \\
&+ (P_i^+(0; \lambda) - P^+(0))P_0^c(0) e^{-\nu(\lambda)X_i} c_i^+ + P_i^+(0; \lambda) (P_0^c(\lambda) - P_0^c(0)) e^{-\nu(\lambda)X_i} c_i^+\\
&+ \lambda^2 d_i P_i^+(0; \lambda) \int_{X_i}^0 \Phi^u(0, y; \lambda) P_i^+(y; \lambda)^{-1} \tilde{H}_i^+(y) dy \\
&+ \lambda^2 d_i P_i^+(0; \lambda) \int_{X_i}^0 \Phi^c(0, y; \lambda) P_i^+(y; \lambda)^{-1} \tilde{H}_i^+(y) dy \\
\end{align*}

As with the first jump, we will start out by computing the significant terms.

\begin{enumerate}
\item The non-center integral will give us the higher order Melnikov integral. For the ``minus'' piece, we have

\begin{align*}
&\langle \Psi(0), P_i^-(0; \lambda) \int_{-X_{i-1}}^0 \Phi^s(0, y; \lambda) P_i^-(y; \lambda)^{-1} \tilde{H}_i^-(y) dy \rangle \\
&= \int_{-X_{i-1}}^0 \langle \Psi(0), P_i^-(0; \lambda), \Phi^s(0, y; \lambda) P_i^-(y; \lambda)^{-1} \tilde{H}(y) \rangle dy \\
&= \int_{-X_{i-1}}^0 \langle \Psi(0), P_i^-(0; \lambda), \Phi^s(0, y; \lambda) P_i^-(y; \lambda)^{-1} H(y) \rangle dy + \mathcal{O}({e^{-\alpha X_m}})\\
&= \int_{-X_{i-1}}^0 \langle \Psi(0), \Theta(0, y) H(y) \rangle dy + \mathcal{O}(|\lambda| + {e^{-\alpha X_m}})\\
&= \int_{-X_{i-1}}^0 \langle \Theta(y, 0)^* \Psi_i(0), H(y) \rangle dy + \mathcal{O}(|\lambda| + {e^{-\alpha X_m}})\\
&= \int_{-X_{i-1}}^0 \langle \Psi(y), H(y) \rangle dy + \mathcal{O}(|\lambda| + {e^{-\alpha X_m}})\\
&= \int_{-\infty}^0 \langle \Psi(y), H(y) \rangle dy + \mathcal{O}(|\lambda| + {e^{-\alpha X_m}})\\
\end{align*}

The ``positive'' piece is similar, and gives us the other half of the Melnikov integral.

\item For the terms involving $a_i$, we plug in $A_4$.

\begin{align*}
\langle &\Psi(0), P_i^-(0; \lambda) \Phi^s(0, -X_{i-1}; \lambda) a_{i-1}^- \rangle \\
&= \langle \Psi_i(0), P_i^-(0; \lambda) \Phi^s(0, -X_{i-1}; \lambda) (- P_i^-(-X_{i-1}; \lambda)^{-1} P_0^s(\lambda) D_{i-1} d + A_4(\lambda)_{i-1}^-(c^-, d)) \rangle \\
&= -\langle \Psi(0), \Theta^s(0, -X_{i-1}) P_0^s(0) D_{i-1} d \rangle + \mathcal{O}( |\lambda|e^{-2 \alpha X_m} + e^{-\alpha X_{i-1}} |A_4(\lambda)_{i-1}^-(c^-, d)|)\\
&= -\langle \Theta^s(-X_{i-1}, 0)^* \Psi_i(0), P_0^s(0) D_{i-1} d \rangle + \mathcal{O}( |\lambda|e^{-2 \alpha X_m} + e^{-\alpha X_{i-1}} |A_4(\lambda)_{i-1}^-(c^-, d)|)\\
&= -\langle \Psi(-X_{i-1}), P_0^s(0) D_{i-1} d \rangle + \mathcal{O}\Big( |\lambda|e^{-2 \alpha X_m} + e^{-\alpha X_{i-1}} ( 
e^{-\alpha X_{i-1}}(|\lambda| + e^{-\tilde{\alpha}X_m})(|\tilde{c}_{i-2}^+| + |\tilde{c}_i^-|) \\
&+ e^{-(\alpha - \rho) X_{i-1}} |c_{i-1}^-| + e^{-\tilde{\alpha} X_m} |\lambda|^2|d| + e^{-\alpha X_m}|D||d|) \Big) \\
&= -\langle \Psi(-X_{i-1}), P_0^s(0) D_{i-1} d \rangle 
+ \mathcal{O}\Big(  
e^{-2 \alpha X_m}(|\lambda| + e^{-\tilde{\alpha}X_m})(|\tilde{c}_{i-2}^+| + |\tilde{c}_i^-|) \\
&+ e^{-\alpha X_m}e^{-(\alpha - \rho) X_{i-1}} |c_{i-1}^-| + e^{-(\alpha + \tilde{\alpha}) X_m} |\lambda|^2|d| + e^{-2 \alpha X_m}(|D| + |\lambda|)|d|) \Big) \\ 
\end{align*}

Similarly, for the $a_i^+$ term, we have

\begin{align*}
\langle &\Psi(0), P_i^+(0; \lambda) \Phi^u(0, X_i; \lambda) a_i^+ \rangle \\
&= \langle \Psi(X_i), P_0^u(0) D_i d \rangle + \mathcal{O}\Big( e^{-2 \alpha X_m} (|\lambda| + e^{-\tilde{\alpha}X_m})(|\tilde{c}_{i-1}^+| + |\tilde{c}_{i+1}^-|) \\
&+ e^{-\alpha X_m} e^{-(\alpha - \rho) X_i} |c_i^-| e^{-(\alpha + \tilde{\alpha}) X_m} |\lambda|^2|d| + e^{-2 \alpha X_m}(|D| + |\lambda|)|d|) \Big)
\end{align*}

\end{enumerate}

The remaining terms will be higher order. Doing these in turn, we have

\begin{enumerate}
\item For the terms involving $b$, we first note that by Lemma \ref{PsiIP}, the terms $P^-(0) b_i^-$ and $P^+(0)b_i^+$ will vanish when we take the inner product with $\Psi(0)$. For the remaining terms, we substitute the estimate for $B_1$ from Lemma \ref{inv2}.

\begin{align*}
&|\langle \Psi(0), (P_i^-(0; \lambda) - P^-(0))b_i^- + P_i^-(0; \lambda)(P_0^u(\lambda) - P_0^u(0))b_i^-| \\
&\leq C (|\lambda| + e^{-\alpha X_m})\Big( 
(|\lambda| + e^{-\tilde{\alpha}X_m})( |\tilde{c}_{i-1}^+| + |\tilde{c}_i^-|)+ ( e^{-\tilde{\alpha}X_m} |D| + e^{-\tilde{\alpha}X_m}|\lambda| + |\lambda|^2)|d| \Big)
\end{align*}

\item For the terms involving $c$, we first note that by Lemma \ref{PsiIP}, the terms $P_0^c(0) e^{\nu(\lambda) X_{i-1}} c_{i-1}^-$ and $P_0^c(0) e^{-\nu(\lambda)X_i} c_i^+$ will be eliminated by taking the inner product with $\Psi(0)$. For the remaining term involving $c_{i-1}^-$, we have

\begin{align*}
|(P_i^-(0; \lambda) - P^-(0)) P_0^c(0) e^{\nu(\lambda) X_{i-1}} c_{i-1}^- + P_i^-(0; \lambda) (P_0^c(\lambda) - P_0^c(0)) e^{\nu(\lambda) X_{i-1}} c_{i-1}^-| \leq C (|\lambda| + e^{-\alpha X_m})|\tilde{c}_{i-1}^+|
\end{align*}

For the term involving $c_i^+$, we also have to use our expression from Lemma \ref{inv2} to convert $e^{-\nu(\lambda)X_i} c_i^+$ to $e^{-\nu(\lambda)X_i} c_i^-$.

\begin{align*}
&|(P_i^+(0; \lambda) - P^+(0))P_0^c(0) e^{-\nu(\lambda)X_i} c_i^+ + P_i^+(0; \lambda) (P_0^c(\lambda) - P_0^c(0)) e^{-\nu(\lambda)X_i} c_i^+| \\
&\leq C(|\lambda| + e^{-\alpha X_m})\Big( |\tilde{c}_i^-| + e^{-\tilde{\alpha}X_m} (|\lambda| + e^{-\tilde{\alpha}X_m})( |\tilde{c}_{i-1}^+| + |\tilde{c}_{i+1}^-|) 
+ e^{-(\alpha - 2 \rho) X_i}|c_i^-| +  e^{-\tilde{\alpha}X_m}(|\lambda| + |D_i|)|d| \Big) \\
&\leq C(|\lambda| + e^{-\alpha X_m})\Big( |\tilde{c}_i^-| + e^{-\tilde{\alpha}X_m} (|\lambda| + e^{-\tilde{\alpha}X_m})( |\tilde{c}_{i-1}^+| + |\tilde{c}_{i+1}^-|) +  e^{-\tilde{\alpha}X_m}(|\lambda| + |D_i|)|d| \Big) 
\end{align*}

where we absorbed the $e^{-(\alpha - 2 \rho) X_i}|c_i^-|$ term into the lower order term $|\tilde{c}_i^-|$.

\item For the center integral term, we have

\begin{align*}
&\langle \Psi(0), P_i^-(0; \lambda)
\int_{-X_{i-1}}^0 \Phi^c(0, y; \lambda) P_i^-(y; \lambda)^{-1} \tilde{H}_i^-(y) dy \rangle \\
&= \int_{-X_{i-1}}^0 \langle \Psi(0), P_i^-(0; \lambda) \Phi^c(0, y; \lambda) P_i^-(y; \lambda)^{-1} \tilde{H}_i^-(y) \rangle dy \\
&= \int_{-X_{i-1}}^0 \langle \Psi(0), P^-(0) \Phi^c(0, y; 0) P^-(y)^{-1} \tilde{H}_i^-(y) \rangle dy + \mathcal{O}(|\lambda| + e^{-\alpha X_m}) \\
&= \mathcal{O}(|\lambda| + e^{-\alpha X_m})
\end{align*}

where the integral vanishes by Lemma \ref{PsiIP} since $\Phi^c(0, y; 0) P^-(y)^{-1} \tilde{H}_i^-(y) \in E^c(0)$.

\end{enumerate}

Putting this all together, we have the jump expressions

\begin{align*}
\xi_i = \langle \Psi(X_i), P_0^u(0) D_i d \rangle
+ \langle \Psi(-X_{i-1}), P_0^s(0) D_{i-1} d \rangle 
- \lambda_2 d_i M + R_i(\lambda)(c, \tilde{c}, d)
\end{align*}

where $M$ is the higher order Melnikov integral

\[
M = \int_{-\infty}^\infty \langle \Psi(y), H(y) \rangle dy 
\]

and the remainder term has bound

\begin{align*}
|R(\lambda)&(\tilde{c}, d)| \leq C \Big( \\
&(|\lambda| + e^{-\alpha X_m})(|\tilde{c}_{i-1}^+| + |\tilde{c}_{i}^-|) + (|\lambda| + e^{-\tilde{\alpha} X_m})(e^{-\alpha X_m} + |\lambda|) ( |\tilde{c}_{i-2}^+| + |\tilde{c}_{i+1}^-|)  \\
&+ e^{-\alpha X_m}( e^{-(\alpha - \rho) X_{i-1}} |c_{i-1}^-| + e^{-(\alpha - \rho) X_i} |c_i^-|) \\
&+ (|\lambda| + e^{-\tilde{\alpha} X_m})|\lambda|^2 + e^{-2 \alpha X_m}(|\lambda| + |D|)|d| \Big)
\end{align*}

Using the fact that $|D| = \mathcal{O}(e^{-\alpha X_m})$ and absorbing the higher order $|c_{i-1}^-|$ and $|c_i^|$ terms into the lower order $|\tilde{c}_{i-1}^+|$ and $|\tilde{c}_{i}^-|$ terms, this becomes

\begin{align*}
|R(\lambda)&(\tilde{c}, d)| \leq C \Big( \\
&(|\lambda| + e^{-\alpha X_m})(|\tilde{c}_{i-1}^+| + |\tilde{c}_{i}^-|) + (|\lambda| + e^{-\tilde{\alpha} X_m})(e^{-\alpha X_m} + |\lambda|) ( |\tilde{c}_{i-2}^+| + |\tilde{c}_{i+1}^-|)  \\
&+ (|\lambda| + e^{-\tilde{\alpha} X_m})(|\lambda| + e^{-\alpha X_m})^2 |d| \Big)
\end{align*}

Before we write this in matrix form, we substitute for $D_i d$. Recalling the expression for $D_i$, we have

\begin{align*}
\langle \Psi(X_i), P_0^u(0) D_i d \rangle
&= \langle \Psi(X_i), P_0^u(0) (Q'(X_i) + Q'(-X_i)) \rangle (d_{i+1} - d_i)
+\mathcal{O}(e^{-2 \alpha X_i}(|\lambda| + e^{-\alpha X_i})) \\
&= \langle \Psi(X_i), Q'(X_i) + Q'(-X_i) \rangle (d_{i+1} - d_i)
+\mathcal{O}(e^{-2 \alpha X_i}(|\lambda| + e^{-\alpha X_i})) \\
&= \langle \Psi(X_i), Q'(-X_i) \rangle (d_{i+1} - d_i)
+\mathcal{O}(e^{-2 \alpha X_i}(|\lambda| + e^{-\alpha X_i})) 
\end{align*}

since $\langle \Psi(X_i), Q'(X_i) \rangle = 0$. Similarly, 

\begin{align*}
\langle \Psi(-X_i), P_0^s(0) D_i d \rangle
&= \langle \Psi(-X_i), Q'(X_i) \rangle (d_i - d_{i-1})
+\mathcal{O}(e^{-2 \alpha X_i}(|\lambda| + e^{-\alpha X_i})) 
\end{align*}

Since these remainder terms are already included in our remainder term, we obtain the final jump expressions

\begin{align*}
\xi_i = \langle \Psi(X_i), Q'(-X_i) \rangle (d_{i+1} - d_i)
+ \langle \Psi(-X_i), Q'(X_i) \rangle (d_i - d_{i-1})
- \lambda_2 d_i M + R_i(\lambda)(c^-, d)
\end{align*}

where $R_i(\lambda)(c^-, d)$ has the same remainder bound as above. We can write this in matrix form as

\[
(C_3 K(\lambda) + K(\lambda) \tilde{C}_3)  + (A - \lambda^2 M I + D_3)d = 0
\]

where the matrix $A$ is given by

\begin{align*}
A &= \begin{pmatrix}
-a_0 + \tilde{a}_1 & a_0 - \tilde{a}_1 \\
-\tilde{a}_0 + a_1 & \tilde{a}_0 - a_1
\end{pmatrix} && n = 2 \\
A &= \begin{pmatrix}
\tilde{a}_{n-1} - a_0 & a_0 & & & \dots & -\tilde{a}_{n-1}\\
-\tilde{a}_0 & \tilde{a}_0 - a_1 &  a_1 \\
& -\tilde{a}_1 & \tilde{a}_1 - a_2 &  a_2 \\
& & \vdots & & \vdots \\
a_{n-1} & & & & -\tilde{a}_{n-2} & \tilde{a}_{n-2} - a_{n-1} \\
\end{pmatrix} && n > 2
\end{align*}

where

\begin{align*}
a_i &= \langle \Psi(X_i), Q'(-X_i) \rangle \\
\tilde{a}_i &= \langle \Psi(-X_i), Q'(X_i) \rangle
\end{align*}

$M$ is the higher order Melnikov integral

\[
M = \int_{-\infty}^\infty \langle \Psi(y), H(y) \rangle dy
\]

and we have bounds

\begin{align*}
C_3 &= \text{diag}(\mathcal{O}(|\lambda| + e^{-\alpha X_m})) 
+ \mathcal{O}((|\lambda| + e^{-\tilde{\alpha} X_m})( |\lambda| + e^{-\alpha X_m})) \\
\tilde{C}_3 &= \text{diag}(\mathcal{O}(|\lambda| + e^{-\alpha X_m})) 
+ \mathcal{O}((|\lambda| + e^{-\tilde{\alpha} X_m})( |\lambda| + e^{-\alpha X_m})) \\
D_3 &= \mathcal{O}((|\lambda| + e^{-\tilde{\alpha} X_m})(|\lambda| + e^{-\alpha X_m})^2)
\end{align*}

\end{proof}
\end{lemma}

\subsubsection{Main Theorem}

We will now combine two jump expressions from Lemma \ref{jumpcenteradj} and Lemma \ref{jumpadj} into a single theorem. Before we do that, we will need one more result about the determinant of a particular matrix.

% bidiagonal determinant

\begin{lemma}\label{bidiag}
Let $A$ be the ``periodic'' bi-diagonal matrix
\begin{equation}
A = \begin{pmatrix}
a_1 & & & & & & b_n \\
b_1 & a_2 \\
& b_2 & a_3 \\
\vdots & & & \vdots & &&  \vdots \\
& & & & b_{n-2} & a_{n-1} \\
& & & & & b_{n-1} & a_n
\end{pmatrix}
\end{equation}

Then 

\begin{equation}
\det{A} = \prod_{k = 1}^n a_k + (-1)^n \prod_{k = 1}^{n-1} b_k
\end{equation}

\begin{proof}
Expanding by minors using the last column, we have
\begin{align*}
\det A &= a_n \det
\begin{pmatrix}
a_1 \\
b_1 & a_2 \\
& b_2 & a_3 \\
\vdots & & & & \vdots \\
& & & & b_{n-2} & a_{n-1}
\end{pmatrix}
+ (-1)^{n-1} \det
\begin{pmatrix}
b_1 & a_2 \\
& b_2 & a_3 \\
\vdots & & & & \vdots \\
& & & & & b_{n-2} & a_{n-1} \\
& & & & & & b_{n-1}
\end{pmatrix} \\
&= \prod_{k = 1}^n a_k + (-1)^{n-1} \prod_{k = 1}^n b_k
\end{align*}
since both of the matrices on the RHS are triangular.
\end{proof}
\end{lemma}

% theorem : block diagonal matrix expression

We can finally state the main theorem of this section.

\begin{theorem}\label{blockmatrixform}

Let $q_{np}(x)$ be a periodic $n-$pulse solution constructed with lengths $X_0, \dots, X_{n-1}$. Then the jump conditions can be written as the block matrix equation 

\begin{equation}\label{blockeq}
\begin{pmatrix}
K(\lambda) & D_2 \\
C_3 K(\lambda) + K(\lambda) \tilde{C}_3 & A - \lambda^2 MI + D_3
\end{pmatrix}
\begin{pmatrix}c \\ d \end{pmatrix} 
= 0
\end{equation}

where 

\begin{enumerate}

\item The remainder terms have bounds

\begin{align*}
C_3, \tilde{C}_3 &= \text{diag}(\mathcal{O}(|\lambda| + e^{-\alpha X_m})) 
+ \mathcal{O}((|\lambda| + e^{-\tilde{\alpha} X_m})( |\lambda| + e^{-\alpha X_m})) \\
D_2 &= \mathcal{O}((|\lambda| + e^{-\tilde{\alpha} X_m})(|\lambda| + e^{-\alpha X_m})) \\
D_3 &= \mathcal{O}((|\lambda| + e^{-\tilde{\alpha} X_m})(|\lambda| + e^{-\alpha X_m})^2)
\end{align*}

where $X_m = \min \{X_0, \dots, X_{n-1}\}$

\item $M$ is the Melnikov integrals

\begin{align*}
M &= \int_{-\infty}^\infty \langle \Psi(y), H(y) \rangle dy \\
\end{align*}

For KdV5, this is

\begin{align*}
M &= \int_{-\infty}^\infty q(y) q_c(y) dy \\
\end{align*}

\item The matrix $K(\lambda)$ is given by

\begin{equation}
K(\lambda) = 
\begin{pmatrix}
e^{-\nu(\lambda)X_1} & & & & & -e^{\nu(\lambda)X_0} \\
-e^{\nu(\lambda)X_1} & e^{-\nu(\lambda)X_2} \\
& -e^{\nu(\lambda)X_2} & e^{-\nu(\lambda)X_3} \\
\vdots & & \vdots & &&  \vdots \\
& & & & -e^{\nu(\lambda)X_{n-1}} & e^{-\nu(\lambda)X_0} 
\end{pmatrix}
\end{equation}

where $\nu(\lambda)$ is the small eigenvalue of the asympotic matrix $A(\lambda)$.

\item The matrix $A$ is given by

\begin{align*}
A &= \begin{pmatrix}
-a_0 + \tilde{a}_1 & a_0 - \tilde{a}_1 \\
-\tilde{a}_0 + a_1 & \tilde{a}_0 - a_1
\end{pmatrix} && n = 2 \\
A &= \begin{pmatrix}
\tilde{a}_{n-1} - a_0 & a_0 & & & \dots & -\tilde{a}_{n-1}\\
-\tilde{a}_0 & \tilde{a}_0 - a_1 &  a_1 \\
& -\tilde{a}_1 & \tilde{a}_1 - a_2 &  a_2 \\
& & \vdots & & \vdots \\
a_{n-1} & & & & -\tilde{a}_{n-2} & \tilde{a}_{n-2} - a_{n-1} \\
\end{pmatrix} && n > 2
\end{align*}

where

\begin{align*}
a_i &= \langle \Psi(X_i), Q'(-X_i) \rangle \\
\tilde{a}_i &= \langle \Psi(-X_i), Q'(X_i) \rangle
\end{align*}

For KdV5, we have $\tilde{a}_i = a_i$.

\end{enumerate}

% To leading order, equation \eqref{blockeq} is upper triangular block equation

% \begin{equation}\label{blocktri}
% \begin{pmatrix}
% K(\lambda) & D_2  \\
% 0 & A - \lambda^2 MI 
% \end{pmatrix}
% \begin{pmatrix}c \\ d \end{pmatrix} = 0
% \end{equation}

% The leading order equations have a nontrivial solution if either of the following conditions holds.

% \begin{enumerate}[(i)]
% \item $\nu(\lambda) = i \dfrac{n \pi}{X}, n \in \Z$ 
% \item $\det(A - \lambda^2 MI) = 0$
% \end{enumerate}

% where $X = X_0 + \dots + X_{n-1}$ is half the length of the domain. 

\begin{proof}
The block matrix \eqref{blockeq} combines the jump conditions from Lemma \ref{jumpcenteradj} and Lemma \ref{jumpadj}. \\

% As it is a square matrix, \eqref{blockeq} has a nontrivial solution if and only if its determinant is 0.\\

% To leading order, the block matrix \eqref{blockeq} is the upper triangular block matrix \eqref{blocktri}, which has determinant 

% \begin{align*}
% \det \begin{pmatrix}
% K(\lambda) & -\lambda^2 M(\lambda) I \\
% 0 & A - \lambda^2 MI
% \end{pmatrix} &= \det K(\lambda) \det(A - \lambda^2 MI)\\
% \end{align*}

% Using Lemma \ref{bidiag}, 

% \begin{align*}
% \det K(\lambda) &= \prod_{k = 0}^{n-1} e^{-\nu(\lambda)X_k} + (-1)^{n-1} \prod_{k = 0}^{n-1} (-e^{\nu(\lambda)X_k}) \\
% &= e^{-\nu(\lambda) X} - e^{\nu(\lambda) X} \\
% &= -2 \sinh( \nu(\lambda) X )
% \end{align*}

% where $X = X_0 + \dots + X_{n-1}$ is half the length of the domain. We have a nontrivial solution to the leading order equations if and only if 

% \[
% \nu(\lambda) = i \frac{n \pi}{X}, n \in \Z
% \]

% or 

% \[
% \det(A - \lambda^2 MI) = 0
% \]

\end{proof}
\end{theorem}

\subsection{Trying to Make This Work}

In most cases, or at least the cases we care about (such as KdV5), we will have (from symmetry of the primary pulse)

\[
\tilde{a}_i = \langle \Psi(-X_i), Q'(X_i) \rangle
= -\langle \Psi(X_i), Q'(-X_i) \rangle = a_i
\]

Thus the matrix $A$ simplifies to

\begin{align*}
A &= \begin{pmatrix}
-a_0 -a_1 & a_0 + a_1 \\
a_0 + a_1 & -a_0 - a_1
\end{pmatrix} && n = 2 \\
A &= \begin{pmatrix}
-a_{n-1} - a_0 & a_0 & & & \dots & a_{n-1}\\
a_0 & -a_0 - a_1 &  a_1 \\
& a_1 & -a_1 - a_2 &  a_2 \\
& & \vdots & & \vdots \\
a_{n-1} & & & & a_{n-2} & -a_{n-2} - a_{n-1} \\
\end{pmatrix} && n > 2
\end{align*}

Note that this is a real symmetric matrix, so its eigenvalues are all real.\\

The interaction eigenvalues are approximately the values of $\lambda$ for which $\det(A - M \lambda^2 I) = 0$. Mathematica suggests that the characteristic polynomial of $A$, i.e. $\det(A - t I)$ is of the form

\[
t\left( t^{m-1} + 2 t^{m-2} \sum_{j=0}^{m-1} a_j
+ \dots + m \sum_{j=0}^{m=1} \prod_{k \neq j} a_k \right) = 0
\]

where the coeffient of $t^k$ is a linear combination of all possible products of $(m-k-1)$ of the $a_j$. \\

Since this annoying, we will do the simple 2-pulse cases first, and hopefully this will give us intuition for the general case.


\subsubsection{2-pulse with equal distances}

We do the easiest case first. Take the 2-periodic solution, with the special case $X_0 = X_1$. 

\begin{enumerate}

\item First, we write down the matrices $A$ and $K(\lambda)$ for this specific case. For the matrix $A$, we have

\[
A = a \begin{pmatrix}
-1 & 1 \\
1 & -1
\end{pmatrix}
\]

where $a = a_0 + a_1 = 2 \langle \Psi(X_0), Q'(-X_0) \rangle$. For the matrix $K(\lambda)$ we have

\begin{equation}
K(\lambda) = 
\begin{pmatrix}
e^{-\nu(\lambda)X_0} & -e^{\nu(\lambda)X_0} \\
-e^{\nu(\lambda)X_0} & e^{-\nu(\lambda)X_0}
\end{pmatrix}
\end{equation}

\item We will look for the interaction eigenvalues first. We expect these to occur when $\det(A - \lambda^2 M I) = 0$. Since $a = \mathcal{O}(e^{-2 \alpha X_0})$, we use the following scaling for $\lambda$ and $a$.

\begin{align*}
\lambda &= e^{-\alpha X_0} \tilde{\lambda} \\
a &= e^{-2 \alpha X_0} \tilde{a}
\end{align*}

With this scaling, the matrix $A$ becomes 

\[
A = e^{-2\alpha X_0} \tilde{a} \tilde{A}
\]

where
\[
\tilde{A} = \begin{pmatrix}
-1 & 1 \\
1 & -1
\end{pmatrix}
\]

\item Next, we expand $\nu(\lambda)$ in a Taylor series about $\lambda = 0$. We can always do this, since $\nu(\lambda)$ is a root of a polynomial. Since $\nu(0) = 0$, the Taylor expansion of $\nu(\lambda)$ is 

\begin{align*}
\nu(\lambda) &= \nu'(0)\lambda + \mathcal{O}(|\lambda|^2) \\
&= \nu'(0)\tilde{\lambda}e^{-\alpha X_0} + \mathcal{O}(e^{-2 \alpha X_0})
\end{align*}

where we used our scaling for $\lambda$ in the second line. We should be able to compute $\nu'(0)$, so let's do that now. The characteristic polynomial of $A_\infty(\lambda)$ is of the form

\[
P(\nu; \lambda) = 
-\nu^{2m+1} + c_{2m - 1} \nu^{2m} + \dots + c_0 \nu + \lambda = 0
\]

where the $c_i$ are real. $P(0; 0) = 0$ and $P_\nu(0; 0) = c_0$. Thus, as long as $c_0 \neq 0$ (for KdV5, $c_0 = -c$, where $c$ is the moving frame speed; we will take $c_0 \neq 0$ as a hypothesis), we can use the IFT to solve for $\nu$ as a function of $\lambda$ near $\lambda = 0$. This is our function $\nu(\lambda)$. By the IFT again, $\nu'(\lambda) = -P_\nu(\nu(\lambda); \lambda)^{-1} P_\lambda(\nu(\lambda); \lambda)$. At $\lambda = 0$, this is

\begin{align*}
\nu'(0) &= -\frac{1}{c_0}
\end{align*}

Numerics on KdV5 supports this result. Thus we have

\begin{align*}
\nu(\lambda) &= -\frac{1}{c_0}\tilde{\lambda}e^{-\alpha X_0} + \mathcal{O}(e^{-2 \alpha X_0})
\end{align*}

\item In this special case, we can use the Taylor expansion of $\nu(\lambda)$ to show that $K(\lambda) = \mathcal{O}(1)$.

\begin{align*}
||K(\lambda)|| &\leq C e^{|\nu(\lambda)|X_0} \\
 &= \mathcal{O} (1 + |\nu(\lambda)|X_0 + \mathcal{O}(|\nu(\lambda)|^2 X_0^2) \\
&= \mathcal{O}( 1 + |1/c_0| |\tilde{\lambda} | e^{-\alpha X_0} X_0 ) \\
&= \mathcal{O}(1)
\end{align*}

as long as we take $X_0$ sufficiently large so that, say, $e^{-\alpha X_0} X_0 \leq 1$.

\item We want to invert $K(\lambda)$ (when possible). To do this, we look at the determinant of $K(\lambda)$, which is given by

\[
\det K(\lambda) = e^{-2 \nu(\lambda)X_0} - e^{2 \nu(\lambda)X_0} = -2 \sinh(2 \nu(\lambda) X_0)
\]

From this, we see that $\det K(\lambda) = 0$ if and only if $2 \nu(\lambda)X_0 = n \pi i$, i.e. $\nu(\lambda) = n \pi i / 2 X_0$ for $n \in \Z$. This tells us the values of $\nu(\lambda)$ for which $\det K(\lambda) = 0$. What we really want is the values of $\lambda$ for which $\det K(\lambda) = 0$. So let's look at that.\\

Define the function $G(\lambda, r) = \nu(\lambda) - r$. Then $G(0, 0) = 0$ and $D_\lambda G(0, 0) = \nu'(0) = -1/c_0$, which we are assuming is nonzero. Using the IFT, we can solve for $\lambda$ in terms of $r$ for $r$ near 0. In other words, we can find a function $\lambda(r)$ such that $\lambda(0) = 0$ and $G(\lambda(r), r) = 0$ for sufficiently small $r$. Thus, for sufficiently small $r$, $\nu(\lambda(r)) = r$. For suffiently large $X_0$ and sufficiently small $n \in \Z$, $|n \pi i / 2 X_0|$ is small, so we can find $\lambda(X_0, n)$ such that $\nu(\lambda(X_0, n)) = n \pi i / 2 X_0$, and so $\det K(\lambda(X_0, n)) = 0$.\\

To better understand $\lambda(r)$, we expand $\lambda(r)$ in a Taylor series about $r = 0$. For $r = n \pi i / 2 X_0 $, we have

\begin{align*}
\lambda(X_0,n)
&= \lambda(0) + \lambda'(0) \frac{n \pi i }{2 X_0} + \mathcal{O}(|n \pi i / 2 X_0|^2) \\
&= \frac{1}{\nu'(0)} \frac{n \pi i }{2 X_0} + \mathcal{O}(n/X_0)^2 \\
&= -c_0 \frac{n \pi i }{2 X_0} + \mathcal{O}(n/X_0)^2 \\
\end{align*} 

It is clear that $\det K(0) = 0$, so there will be a problem at $\lambda = 0$. Using our scaling for the interaction eigenvalues, we do not have to be concerned with the other values of $\lambda$ for which $K(\lambda)$ is singular. To see this, we want to invert $K(\lambda)$ near $\tilde{\lambda} e^{-\alpha X_0}$, which, for sufficiently large $X_0$, is significantly smaller than $\lambda(X_0,1) = \mathcal{O}(1/X_0)$. If we like, we can, say, choose $X_0$ sufficiently large so that $e^{-\alpha X_0 /2} < 1/X_0$.

\item Next, we obtain an expression and a bound for $K(\lambda)^{-1}$ for $\lambda$ near $\tilde{\lambda} e^{-\alpha X_0}$. To do this, we expand the determinant of $K(\lambda)$ in a Taylor series at $\lambda = 0$. Using the Taylor series for $\sinh x$ and our scaling, we get

\begin{align*}
\det K(\lambda) &= -4 \nu(\lambda) X_0 + \mathcal{O}(|\nu(\lambda)|^3) \\
&= -4 X_0 [ (-1/c_0)\tilde{\lambda}e^{-\alpha X_0} + \mathcal{O}(e^{-2 \alpha X_0}) ] + \mathcal{O}(e^{-3 \alpha X_0}) \\
&= \frac{4}{c_0}\tilde{\lambda}e^{-\alpha X_0}X_0 + \mathcal{O}(e^{-2 \alpha X_0}X_0) 
\end{align*}

Thus for $K(\lambda)^{-1}$ (near $\lambda = \tilde{\lambda} e^{-\alpha X_0}$) we have

\begin{equation}
K(\lambda)^{-1} = 
\frac{c_0}{4 \tilde{\lambda}e^{-\alpha X_0}X_0 + \mathcal{O}(e^{-2 \alpha X_0}X_0)}
\begin{pmatrix}
e^{-\nu(\lambda)X_0} & e^{\nu(\lambda)X_0} \\
e^{\nu(\lambda)X_0} & e^{-\nu(\lambda)X_0}
\end{pmatrix}
\end{equation}

where the matrix on the RHS is $\mathcal{O}(1)$ by the same argument we used above to show that $||K(\lambda)|| = \mathcal{O}(1)$. For a bound, we have

\[
||K(\lambda)^{-1}|| \leq C \frac{ e^{\alpha X_0} }{X_0}
\]

\item To find the interaction eigenvalues, we can now solve the first block matrix equation for $c$. Since we are using our scaling, as long as $\lambda \neq 0$, we can invert $K(\lambda)$. The inverse $K(\lambda)^{-1}$ will blow up in norm as $\lambda$ approaches 0, but we hope our bounds on the other terms will take care of that for us. Solving for $c$, we have

\begin{align*}
c = -K(\lambda)^{-1} D_2 d \\
\end{align*}

Plugging this into the second block matrix equation, we get

\begin{align*}
(C_3 K(\lambda) + K(\lambda) \tilde{C}_3) c + (A - \lambda^2 MI + D_3)d &= 0 \\
-(C_3 K(\lambda) + K(\lambda) \tilde{C}_3)K(\lambda)^{-1} D_2 d + (A - \lambda^2 MI + D_3)d &= 0 \\
(A - \lambda^2 MI )d + (D_3 - C_3 D_2 - K(\lambda) \tilde{C}_3 K(\lambda)^{-1} D_2) d &= 0 \\
\end{align*}

We need to check that remainder term on the LHS is really higher order. Using our scaling, $D_3, C_3 D_2 = \mathcal{O}(e^{-3 \alpha X_0})$, which will be good. That leaves the last term. Again, using our scaling and the bound on $||K(\lambda)||^{-1}$ we found above, this becomes

\begin{align*}
|K(\lambda) \tilde{C}_3 K(\lambda)^{-1} D_2| &\leq
C e^{-\alpha X_0} \frac{e^{\alpha X_0}}{X_0} e^{-2 \alpha X_0} \\
&= \frac{e^{-2 \alpha X_0}}{X_0}
\end{align*}

Putting all of this together, we have 

\[
(A - \lambda^2 MI )d + \mathcal{O}\left( e^{-3 \alpha X_0} + \frac{e^{-2 \alpha X_0}}{X_0} \right) d = 0
\]

Now we use our scaling again to write this using $\tilde{\lambda}$ and $\tilde{a}$.

\[
(e^{-2\alpha X_0} \tilde{a} \tilde{A} - e^{-2 \alpha X_0} \tilde{\lambda}^2 M I )d + \mathcal{O}\left( e^{-3 \alpha X_0} + \frac{e^{-2 \alpha X_0}}{X_0} \right) d = 0
\]

This simplifies to

\begin{equation}\label{tildeaEq1}
(\tilde{a} \tilde{A} - \tilde{\lambda}^2 M I )d + \mathcal{O}\left( e^{-\alpha X_0} + \frac{1}{X_0} \right) d = 0
\end{equation}

where because of our scaling, $\tilde{a} \tilde{A} - \tilde{\lambda}^2 M I  = \mathcal{O}(1)$. The equation $(\tilde{a} \tilde{A} - \tilde{\lambda}^2 M I )d = 0$ has a nontrivial solution if and only if its determinant is 0, which is true if and only if $\tilde{\lambda} = 0$ (which we already know) or $\tilde{\lambda} = \pm \sqrt{-2 \tilde{a}/M}$ (which does not depend on $X_0$). Let

\[
\tilde{\mu}_0 = \sqrt{-2 \tilde{a}/M}
\]

$\tilde{\mu}_0$ is either real or pure imaginary, depending on the signs of $\tilde{a}$ and $M$. We write the matrix equation \label{tildeaEq1} as

\[
G(\tilde{\lambda}, r_0)d = 0
\]

where $r_0 = 1/X_0$. Since $\det G(\pm \tilde{\mu}_0, r_0) = 0$, and $G(\tilde{\lambda}, r_0)$ is smooth in $\tilde{\lambda}$ (it is in fact a polynomial in $\tilde{\lambda}$), for sufficiently small $r$ (i.e. sufficiently large $X_0$) we can use the IFT to find unique $\tilde{\lambda}^\pm(X_0)$ near $\pm \tilde{\mu}_0$ such that $\det( \tilde{\lambda}^\pm(X_0), X_0) = 0$. \\

Finally, we undo our scaling. Let

\[
\lambda^\pm(X_0) = e^{-\alpha X_0} \tilde{\lambda}^\pm(X_0)
\]

These are the interaction eigenvalues we seek. By Hamiltonian symmetry of the underlying system (we will have already mentioned this), eigenvalues must come in quartets, i.e. if $\alpha + \beta i$ is an eigenvalue, so are $\pm \alpha \pm \beta i$. Since there are only two eigenvalues of this magnitude, we conclude that we have a pair of interaction eigenvalues at $\pm \lambda(X_0)$. This pair must be real or purely imaginary.\\

$\lambda(X_0)$ is close to $\pm e^{-\alpha X_0} \tilde{\mu}_0 = \sqrt{-2a/M}$.

\item Now let's look at the eigenvalues which arise for those $\lambda$ near where $K(\lambda)$ is singular. Because we chose the two distances to be equal, the only value of $\lambda$ for which both $A - \lambda^2 M I$ and $K(\lambda)$ are singular is $\lambda = 0$.\\

Choose $\lambda$ for which $A - \lambda^2 M I$ is nonsingular, i.e. $\lambda \neq 0$ and $\lambda \neq \pm \lambda(X_0)$. Then for our chosen $\lambda$, we can invert $A - \lambda^2 M I$. Since $\det(A - \lambda^2 M I) = \lambda^2 M(2 a + \lambda^2 M)$, the inverse of $A - \lambda^2 M I$ is

\begin{align*}
(A - \lambda^2 M I)^{-1} &=
\frac{1}{\lambda^2 M(2a + \lambda^2 M)}
\begin{pmatrix}
-a - \lambda^2 M & -a \\
-a & -a - \lambda^2 M
\end{pmatrix} \\
\end{align*}

To go further, we need to bound this. Since we expect to find eigenvalues near $\lambda(X_0, n) = -c_0 \frac{n \pi i }{2 X_0} + \mathcal{O}(n/X_0)^2$, we take the scaling $\lambda = \tilde{\lambda}(n/X_0)$. This gives us the bound

\[
||(A - \lambda^2 M I)^{-1}||\leq C \left( \frac{n}{X_0} \right)^{-2}
\]

\item We then use this to solve for $d$ in the second block matrix equation.

\begin{align*}
(C_3 K(\lambda) + K(\lambda) \tilde{C}_3) c + (A - \lambda^2 MI + D_3)d &= 0 \\
(C_3 K(\lambda) + K(\lambda) \tilde{C}_3) c + (A - \lambda^2 MI)(I + (A - \lambda^2 MI)^{-1} D_3)d &= 0
\end{align*}

With our scaling, $D_3 = \mathcal{O}(n/X_0)^3$, thus $(A - \lambda^2 MI)^{-1} D_3 = \mathcal{O}(n/X_0)$. Thus for sufficiently large $X_0$ and sufficiently small $n$, $I + (A - \lambda^2 MI)^{-1} D_3$ is invertible. We can now solve for $d$.

\begin{align*}
 d &= 
 -(I + (A - \lambda^2 MI)^{-1} D_3)^{-1}(A - \lambda^2 MI)^{-1}(C_3 K(\lambda) + K(\lambda) \tilde{C}_3) c \\
 &= -C_4 (C_3 K(\lambda) + K(\lambda) \tilde{C}_3) c
\end{align*}

where $C_4 = (I + (A - \lambda^2 MI)^{-1} D_3)^{-1}(A - \lambda^2 MI)^{-1} = \mathcal{O}(n/X_0)^{-2}$.\\

Finally, we substitute this for $d$ into the first line of the block matrix equation to get

\begin{align*}
K(\lambda)c + D_2 d &= 0 \\
K(\lambda)c - D_2 C_4 (C_3 K(\lambda) + K(\lambda) \tilde{C}_3) c &= 0 \\
(I - D_2 C_4 C_3 ) K(\lambda)c - K(\lambda) \tilde{C}_3) c &= 0
\end{align*}

Since $D_2 C_4 C_3 = \mathcal{O}(n/X_0)$, for sufficiently large $X_0$ and sufficiently small $n$, $I + D_2 C_4 C_3$ is invertible, and this reduces to

\[
K(\lambda)c + C_5 K(\lambda) \tilde{C}_3 c = 0
\]

where $C_5 = (I + D_2 C_4 C_3)^{-1} = \mathcal{O}(1)$ and $\tilde{C}_3 = \mathcal{O}(n/X_0)$.

\item Finally, we can solve $K(\lambda)c + C_5 K(\lambda) \tilde{C}_3 c = 0$. We expect this will be possible for $\lambda$ close to $\lambda(X_0, n) = -c_0 \frac{n \pi i }{2 X_0} + \mathcal{O}(n/X_0)^2$. To show this works, we need to rewrite the equation in a form where we have a matrix which does not depend on $X_0$ and $n$ and a remainder term which does. To do that, we use our scaling on $K(\lambda)$, i.e. we write this in terms of $\tilde{\lambda}$. \\

From the Taylor series expansion for $\nu(\lambda)$, we have 

\begin{align*}
\nu( \lambda ) &= \nu( \tilde{\lambda}(n/X_0) ) \\
&= -(1/c_0)\tilde{\lambda}(n/X_0) + \mathcal{O}(n/X_0)^2
\end{align*}

Thus we have

\begin{align*}
e^{\nu(\lambda) X_0} &= \exp( { -(1/c_0)\tilde{\lambda}n + \mathcal{O}(n/X_0^2) } ) \\
&= e^{ -(n/c_0)\tilde{\lambda} }(1 + \mathcal{O}(n/X_0^2)  )
\end{align*}

and 

\begin{align*}
e^{-\nu(\lambda) X_0}  
&= e^{ (n/c_0)\tilde{\lambda} }(1 + \mathcal{O}(n/X_0^2)  )
\end{align*}

Combining these, we have

\begin{equation}
K(\tilde{\lambda}) = 
\begin{pmatrix}
e^{ (n/c_0)\tilde{\lambda} } & -e^{-(n/c_0)\tilde{\lambda} } \\
-e^{ (-n/c_0)\tilde{\lambda} } & e^{ (n/c_0)\tilde{\lambda} }
\end{pmatrix}
+ \mathcal{O}(n/X_0^2)
\end{equation}

This gives us an equation that looks like

\[
(K(\tilde{\lambda}) + \mathcal{O}(n^2/X_0))c = 0
\]

The remainder term does depend on $\tilde{\lambda}$, but its magnitude does not. We can write the determinant as 

\[
\det (K(\tilde{\lambda}) + \mathcal{O}(n^2/X_0)) = G(\tilde{\lambda}, r)
\]

If we take $\tilde{\lambda} = (X_0/n) \lambda(X_0, n)$, we have $G((X_0/n) \lambda(X_0, n), 0) = 0)$. An argument using the IFT should conclude that for sufficiently large $X_0$ and sufficiently small $n$ (and after undoing the scaling), we can find $\lambda^c(X_0, n)$ close to $\lambda(X_0, n)$ such that $\det ( K(\lambda^c(X_0, n))c + C_5 K(\lambda^c(X_0, n)) \tilde{C}_3 ) = 0$. By Hamiltonian symmetry, we conclude that $\lambda^c(X_0, n)$ is pure imaginary, and $\lambda^c(X_0, -n) = \lambda^c(X_0, n)$.

\item Let us summarize the results that we have obtained.

\begin{itemize}
	\item For $X_0$ sufficiently large, there is a pair of interaction eigenvalues at $\lambda = \pm \lambda(X_0)$, where $\lambda(X_0)$ is close to $\pm \sqrt{-2a/M}$ and is either real or purely imaginary.
	\item For $X_0$ sufficiently large and $n$ sufficiently small (but nonzero), there is a pair of purely imaginary ``essential spectrum'' eigenvalues located at $\lambda = \pm \lambda^c(X_0, n)$, where $\lambda^c(X_0, n)$ is close to $-c_0 \frac{n \pi i }{2 X_0}$.
	\item There is an eigenvalue at 0 (from translation invariance), where the eigenfunction is the derivative of the solution we are linearizing about. At this point, we cannot use Lin's method to conclude anything else about what happens at $\lambda = 0$.
\end{itemize}
I

\end{enumerate}

\subsubsection{2-pulse with Unequal Distances}

The next easiest case is the 2-pulse $X_0 < X_1$. (It doesn't matter which way we do it, but by the convention from the existence problem, $X_1$ is the ``periodic distance''). Much of this is similar to the case with equal distances.

\begin{enumerate}

\item First, we write down the matrices $A$ and $K(\lambda)$. For the matrix $A$ we have

\[
A = a \begin{pmatrix}
-1 & 1 \\
1 & -1
\end{pmatrix}
\]

where $a = a_0 + a_1 = \langle \Psi(X_0), Q'(-X_0) \rangle + \langle \Psi(X_1), Q'(-X_1) \rangle$. For the matrix $K(\lambda)$ we have

\begin{equation}
K(\lambda) = 
\begin{pmatrix}
e^{-\nu(\lambda)X_1} & -e^{\nu(\lambda)X_0} \\
-e^{\nu(\lambda)X_1} & e^{-\nu(\lambda)X_0}
\end{pmatrix}
\end{equation}

\item We will look for the interaction eigenvalues first. We expect these to occur when $\det(A - \lambda^2 M I) = 0$. Since $e^{-\alpha X_1}$ is small compared to $e^{-\alpha X_0}$, we take the scaling

\begin{align*}
\lambda &= e^{-\alpha X_0} \tilde{\lambda} \\
a &= e^{-2 \alpha X_0} \tilde{a}
\end{align*}

Then we can write the matrix $A$ as 

\[
A = e^{-2 \alpha X_0} \tilde{a} \tilde{A}
\]

where
\[
\tilde{A} = \begin{pmatrix}
-1 & 1 \\
1 & -1
\end{pmatrix}
\]

So far, this is the same as the previous case.

\item Next, we expand $\nu(\lambda)$ in a Taylor series about $\lambda = 0$. This is exactly the same as in the previous section.

\begin{align*}
\nu(\lambda) &= -\frac{1}{c_0}\tilde{\lambda}e^{-\alpha X_0} + \mathcal{O}(e^{-2 \alpha X_0})
\end{align*}

\item We can use this to get a bound on $K(\lambda)$ itself.

\begin{align*}
||K(\lambda)|| &= \mathcal{O}( e^{|\nu(\lambda)|X_0} + e^{|\nu(\lambda)|X_1} ) \\
 &= \mathcal{O} (1 + |\nu(\lambda)X_0| + |\nu(\lambda)X_1 |+ (|\nu(\lambda)X_0|^2 + (|\nu(\lambda)X_1|^2)) \\
&= \mathcal{O}( 1 + |1/c_0| |\tilde{\lambda} | e^{-\alpha X_0} X_0 + |1/c_0| |\tilde{\lambda} | e^{-\alpha X_0} X_1) \\ 
&= \mathcal{O}( 1 + e^{-\alpha X_0}(X_0 + X_1) ) 
\end{align*}

Writing this in a different way, we have

\begin{align*}
||K(\lambda)|| 
= \mathcal{O}\left( 1 + e^{-\alpha X_0}X_0\left( 1 + \frac{X_1}{X_0} \right) \right) 
\end{align*}

I THINK WE NEED TO DO SOMETHING LIKE THIS. Choose $X_0$ sufficiently large so that

\[
e^{-\alpha X_0}X_0\left( 1 + \frac{X_1}{X_0} \right) = \mathcal{O}(1)
\]
Then we have $||K(\lambda)|| = \mathcal{O}(1)$.

\item Next, we would like to invert $K(\lambda)$, when possible. The determinant of $K(\lambda)$ is given by

\[
\det K(\lambda) = e^{-\nu(\lambda)(X_0+X_1)} - e^{\nu(\lambda)(X_0+X_1)} = -2 \sinh(\nu(\lambda)(X_0+X_1))
\]

This is 0 if and only if $\nu(\lambda)(X_0+X_1) = n \pi i$, i.e. $\nu(\lambda) = n \pi i / X$ for $n \in \Z$, where $X = X_0 + X_1$ is half the length of the domain. This includes $\lambda = 0$.\\

Inverting this is more complicated than in the previous case since. There will potentially be many places on the imaginary axis where $K(\lambda)$ is singular, and these will ``pile up'' as $X_1$ gets larger. We need to avoid all of them. Essentially, we need to put small open balls around each point where $K(\lambda)$ is singular, and the result will only hold outside these open balls. The hope is that we can do this, i.e. there is actually space outside these balls that we can work in!\\

To do this, first we need to know the values of $\lambda$ (not $\nu(\lambda))$ where $K(\lambda)$ is singular. Adapting what we did above, for sufficiently large $X_1$ and sufficiently small $n$, $K(\lambda)$ is singular for $\lambda = \lambda(X_1, n)$, which is given by

\begin{align*}
\lambda(X_1, n)
&= -c_0 \frac{n \pi i }{X_0 + X_1} + \mathcal{O}\left( \frac{n}{X_0+X_1}\right)^2 \\
\end{align*}

We note that $\nu(\lambda(X_1, n)) = n \pi i/(X_0 + X_1)$. We also note that $X_0$ is fixed here and that, near the origin, these are roughly equally spaced on the imaginary axis.

\item Now we Taylor expand $K(\lambda)$ about each $\lambda(X_1, n)$ which is within $\mathcal{O}(e^{-\alpha X_0})$ of the origin. (We only have to worry about stuff near the interaction eigenvalues we are looking for.)

\begin{align*}
\det K( \lambda(X_1, n) + \epsilon )
&= -2 \sinh \Big( \nu(\lambda(X_1, n) + \epsilon )(X_0 + X_1) \Big) \\
\end{align*}

To keep going, we need to substitute the Taylor series for $\nu(\lambda)$. This time, we use the Taylor expansion about $\lambda(X_1, n)$, since we know that $\nu( \lambda(X_1, n) ) = n \pi i / (X_0 + X_1)$.

\begin{align*}
\nu(\lambda(X_1, n) + \epsilon ) &= 
\frac{n \pi i }{X_0 + X_1} + \mathcal{O}(\epsilon)
\end{align*}

WE MIGHT WANT A BETTER TAYLOR EXPANSION, i.e. USE $\mathcal{O}(\epsilon) = \nu'(\lambda(X_1, n))\epsilon + \mathcal{O}(\epsilon^2)$. Plugging this in above, we get

\begin{align*}
\det K( \lambda(X_1, n) + \epsilon )
&= -2 \sinh \Big( \nu(\lambda(X_0, n) + \epsilon )(X_0 + X_1) \Big) \\
&= -2 \sinh \left( n \pi i + \mathcal{O}(\epsilon(X_0 + X_1) \right) \\
&= \mathcal{O}(\epsilon(X_0 + X_1) )
\end{align*}

Note that, to leading order, this does not depend on $n$. As long we we are further than $\epsilon$ in distance from any of the $\lambda(X_0, n)$, we have the following bound.

\begin{align*}
|| K( \lambda(X_1, n))^{-1}|| 
&\leq C \left( \frac{1}{\epsilon (X_0 + X_1)} \right)
\end{align*}

All that remains now is to choose an appropriate $\epsilon$. Recall that the spacing between the $\lambda(X_0, n)$ is order $\mathcal{O}(1/(X_0 + X_1))$, so we need $\epsilon$ to be some fraction of that. Since $X_0$ is large, $e^{-\alpha X_0}$ is small, so choose

\[
\epsilon = \frac{e^{-\alpha X_0/2}}{X_0 + X_1}
\]

Substituting this in, we attain the bound we want.

\begin{align*}
|| K( \lambda(X_1, n))^{-1}|| 
&\leq C e^{\alpha X_0/2}
\end{align*}

which holds outside the $\epsilon-$balls.

\item As long as $\lambda$ is outside of the $\epsilon-$balls, we can solve for $c$ in terms of $d$ using the first line of the block matrix equation. Solving for $c$, we have

\begin{align*}
c = -K(\lambda)^{-1} D_2 d \\
\end{align*}

Plugging this into the second block matrix equation, we get

\begin{align*}
(C_3 K(\lambda) + K(\lambda) \tilde{C}_3) c + (A - \lambda^2 MI + D_3)d &= 0 \\
-(C_3 K(\lambda) + K(\lambda) \tilde{C}_3)K(\lambda)^{-1} D_2 d + (A - \lambda^2 MI + D_3)d &= 0 \\
(A - \lambda^2 MI )d + (D_3 - C_3 D_2 - K(\lambda) \tilde{C}_3 K(\lambda)^{-1} D_2) d &= 0 \\
\end{align*}

We need to check that remainder term on the LHS is really higher order. Using our scaling, $D_3, C_3 D_2 = \mathcal{O}(e^{-3 \alpha X_0})$, which will be good. That leaves the last term. For that, we have

\begin{align*}
|K(\lambda) \tilde{C}_3 K(\lambda)^{-1} D_2| &\leq
C e^{-\alpha X_0} e^{\alpha X_0/2} e^{-2 \alpha X_0} \\
&= e^{-(5/2) \alpha X_0}
\end{align*}

Putting all of this together, we have 

\[
(A - \lambda^2 MI )d + \mathcal{O}\left( e^{-(5/2) \alpha X_0} \right) d = 0
\]

Recall that in the equal length case, we had a term which looked like $1/X_0$ and then had to take $X_0$ sufficiently large. We already took $X_0$ large enough above, so that should be taken care of.

\item The rest of this works out as in the previous case where both lengths were equal. We use our scaling again to write this using $\tilde{\lambda}$ and $\tilde{a}$. After simplifying, this becomes

\[
(\tilde{a}\tilde{A} - \tilde{\lambda}^2 MI )d + \mathcal{O}\left( e^{-(1/2) \alpha X_0} \right) d = 0
\]

Because of our scaling, $\tilde{a} \tilde{A} - \tilde{\lambda}^2 M I  = \mathcal{O}(1)$. The equation $(\tilde{a} \tilde{A} - \tilde{\lambda}^2 M I )d = 0$ has a nontrivial solution if and only if its determinant is 0, which is true if and only if $\tilde{\lambda} = 0$ (which we already know) or $\tilde{\lambda} = \pm \sqrt{-2 \tilde{a}/M}$ (which does not depend on $X_0$). Let

\[
\tilde{\mu}_0 = \sqrt{-2 \tilde{a}/M}
\]

$\tilde{\mu}_0$ is either real or pure imaginary, depending on the signs of $\tilde{a}$ and $M$. We write the matrix equation \label{tildeaEq1} as

\[
G(\tilde{\lambda}, r_0)d = 0
\]

where $r_0 = \mathcal{O}(e^{-(1/2) \alpha X_0})$. Since $\det G(\pm \tilde{\mu}_0, r_0) = 0$, and $G(\tilde{\lambda}, r)$ is smooth in $\tilde{\lambda}$ (it is in fact a polynomial in $\tilde{\lambda}$), for sufficiently small $r$ (i.e. sufficiently large $X_0$) we can use the IFT to find unique $\tilde{\lambda}^\pm(X_0)$ near $\pm \tilde{\mu}_0$ such that $\det( \tilde{\lambda}^\pm(X_0), X_0) = 0$. \\

Finally, we undo our scaling. Let

\[
\lambda^\pm(X_0) = e^{-\alpha X_0} \tilde{\lambda}^\pm(X_0)
\]

These are the interaction eigenvalues we seek. By Hamiltonian symmetry of the underlying system (we will have already mentioned this), eigenvalues must come in quartets, i.e. if $\alpha + \beta i$ is an eigenvalue, so are $\pm \alpha \pm \beta i$. Since there are only two eigenvalues of this magnitude, we conclude that we have a pair of interaction eigenvalues at $\pm \lambda(X_0)$. This pair must be real or purely imaginary.\\

$\lambda(X_0)$ is close to $\pm e^{-\alpha X_0} \tilde{\mu}_0 = \sqrt{-2a/M}$.

\end{enumerate}

\subsubsection{General Case}

Now we return to the general case. Evaluating $\det K(\lambda)$ using a formula we have proved, we have

\[
\det K(\lambda) = e^{-\nu(\lambda)X} - e^{\nu(\lambda)X} = -2 \sinh(\nu(\lambda) X)
\]

where $X = X_0 + \dots + X_{m-1}$ is half the length of the domain. This is 0 if and only if $\nu(\lambda)X = n \pi i$, i.e. $\nu(\lambda) = n \pi i / X$ for $n \in \Z$. It is clear that $\det K(0) = 0$, so we have to avoid $\lambda = 0$. We also, for now, have to avoid the other points where this is singular.
\\

Next, we recall that $\nu(\lambda)$ is the eigenvalue (near 0) of the asymptotic matrix for the linearization about the homoclinic orbit, which takes the general form 

\begin{equation}
A_\infty(\lambda) = \begin{pmatrix}
0 & 1 & \dots & 0 & 0 \\
0 & 0 & \dots & 0 & 0 \\
& \ddots & \ddots & \ddots & & \\
0 & 0 & \dots & 0 & 1 \\
-\lambda & c_0 & \dots & c_{m-3} & c_{m-2}
\end{pmatrix}
\end{equation}

where $m$ is the dimension of the space we are in. The spatial eigenvalues of this are the roots of its characteristic polynomial, which is

\[
-t^{m} + c_{m-2} t^{m-1} + \dots + c_0 t + \lambda = 0
\]

For $\lambda = 0$, $\nu(\lambda) = 0$. Since eigenvalues are roots of polynomials, they are smooth in the parameter $\lambda$, so $\nu(\lambda)$ is smooth in $\lambda$. \\

For $\lambda$ large and purely imaginary, i.e. $\lambda = \beta i$, we have approximately $t^m \approx i \beta$, so for $m$ odd (which is what we have), $\nu(\lambda) \approx \beta^{1/m} i$. What this means is that if we take $X < (e^{\alpha X_m})^{1/m}$, we should be able to avoid everywhere where $K(\lambda)$ is singular except for $\lambda = 0$. We should easily be able to do this.\\ 

For $\lambda$ small, we expand $\nu(\lambda)$ in a Taylor series about $\lambda = 0$. Recalling that $\nu(0) = 0$, we have

\begin{align*}
\nu(\lambda) &= \nu'(0)\lambda + \mathcal{O}(|\lambda|^2) \\
&= \nu'(0)\tilde{\lambda}e^{-\alpha X_m} + \mathcal{O}(e^{-2 \alpha X_m})
\end{align*}

Expanding $\det K(\lambda)$ in a Taylor series about $\nu(\lambda) = 0$ gives us

\begin{align*}
\det K(\lambda) &= -2 \nu(\lambda) X + \mathcal{O}(\nu(\lambda)^3 X^3)
\end{align*}

Plugging the Taylor expansion for $\nu(\lambda)$ into the Taylor expansion of the determinant of $K(\lambda)$, we get

\begin{align*}
\det K(\lambda) &= -2 \nu'(0)\tilde{\lambda}e^{-\alpha X_m} X + \mathcal{O}(e^{-2 \alpha X_m} X )
\end{align*}

With all of this, we can bound $K(\lambda)$ and $K(\lambda)^{-1}$ for $\lambda = e^{-\alpha X_m}\tilde{\alpha}$, $\lambda \neq 0$. Recall that we took $X < (e^{\alpha X_m})^{1/m}$. For $K(\lambda)$ we have

\begin{align*}
||K(\lambda)|| &\leq C e^{|\nu(\lambda)|X} \\
 &= \mathcal{O} (1 + |\nu(\lambda)X| ) \\
&= \mathcal{O}( 1 + |\nu'(0)\tilde{\lambda}|e^{-\alpha X_m}X ) \\
&= \mathcal{O}( 1 + |\nu'(0)\tilde{\lambda}|e^{-\alpha X_m}(e^{\alpha X_m})^{1/m} )\\
&= \mathcal{O}(1)
\end{align*}

For $K(\lambda)^{-1}$, we have for $\lambda \neq 0$,

\begin{align*}
||K(\lambda)^{-1}|| &\leq C \frac{1}{\det K(\lambda)} e^{|\nu(\lambda)|X} \\
 &\leq C \frac{1}{-2 \nu'(0)\tilde{\lambda}e^{-\alpha X_m} X + \mathcal{O}(e^{-2 \alpha X_m} X )} \\
 &= \mathcal{O} \left( \frac{1}{e^{-\alpha X_m} X} \right)
\end{align*}

Now that we have all of this, we can solve for $c$ using the first block matrix equation and plug this into the second block matrix equation. From the first block matrix equation, we have for $\lambda \neq 0$ and $\lambda = e^{-\alpha X_m} \tilde{\lambda}$

\[
c = -K(\lambda)^{-1} D_2 d
\]

Plug this into the second block matrix equation to get

\begin{align*}
-(C_3 K(\lambda) + K(\lambda) \tilde{C}_3)K(\lambda)^{-1} D_2 d + (A - \lambda^2 MI + D_3)d &= 0 \\
-(C_3 + K(\lambda) \tilde{C}_3 K(\lambda)^{-1}) D_2 d + (A - \lambda^2 MI + D_3)d &= 0 \\ 
(A - \lambda^2 MI)d + (D_3 - C_3 D_2 + K(\lambda) \tilde{C}_3 K(\lambda)^{-1} D_2) d + &= 0 \\
\end{align*}

We want to solve this for $d$ and have the solution occur for $\lambda$ close to when $\det(A - \lambda^2 MI) = 0$. By our scaling, $A, \lambda^2 = \mathcal{O}(e^{-2 \alpha X_m})$, for our wish to come true, we need the remainder term on the LHS to be of higher order. $D_3$ and $C_3 D_2$ are $\mathcal{O}(e^{-3 \alpha X_m})$, so those present no problem. We just have to deal with the last one.

\begin{align*}
||K(\lambda) \tilde{C}_3 K(\lambda)^{-1} D_2|| &\leq
C (|\lambda| + e^{-\tilde{\alpha} X_m}) \left( \frac{1}{e^{-\alpha X_m} X} \right) \mathcal{O}((|\lambda| + e^{-\tilde{\alpha} X_m})(|\lambda| + e^{-\alpha X_m})) \\
&\leq C X^{-1} e^{-2 \alpha X_m}
\end{align*}

which is okay, but there is a limit to how large we can take $X$ in the general case.

\end{document}