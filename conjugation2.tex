\documentclass[12pt]{article}
\usepackage[pdfborder={0 0 0.5 [3 2]}]{hyperref}%
\usepackage[left=1in,right=1in,top=1in,bottom=1in]{geometry}%
\usepackage[shortalphabetic]{amsrefs}%
\usepackage{amsmath}
\usepackage{enumerate}
% \usepackage{enumitem}
\usepackage{amssymb}                
\usepackage{amsmath}                
\usepackage{amsfonts}
\usepackage{amsthm}
\usepackage{bbm}
\usepackage[table,xcdraw]{xcolor}
\usepackage{tikz}
\usepackage{float}
\usepackage{booktabs}
\usepackage{svg}
\usepackage{mathtools}
\usepackage{cool}
\usepackage{url}
\usepackage{graphicx,epsfig}
\usepackage{makecell}
\usepackage{array}

\def\noi{\noindent}
\def\T{{\mathbb T}}
\def\R{{\mathbb R}}
\def\N{{\mathbb N}}
\def\C{{\mathbb C}}
\def\Z{{\mathbb Z}}
\def\P{{\mathbb P}}
\def\E{{\mathbb E}}
\def\Q{\mathbb{Q}}
\def\ind{{\mathbb I}}

\DeclareMathOperator{\spn}{span}
\DeclareMathOperator{\ran}{ran}

\graphicspath{ {periodic/} }

\newtheorem{lemma}{Lemma}
\newtheorem{theorem}{Theorem}
\newtheorem{corollary}{Corollary}
\newtheorem{definition}{Definition}
\newtheorem{assumption}{Assumption}
\newtheorem{hypothesis}{Hypothesis}

\newtheorem{notation}{Notation}

\begin{document}

\subsection*{Conjugation}

We will use the Conjugation Lemma to simplify the eigenvalue problem. Here is a version of the Conjugation Lemma, which is adapted and cleaned up from Zum2018. 

\begin{lemma}[Conjugation Lemma]
Let $W \in \C^N$, and consider the family of ODEs on $\R$

\begin{equation}\label{LambdaEVPconj}
W(x)' = A(x; \Lambda) W(x) + B(x) W(x) + F(x) 
\end{equation}

where $\Lambda \in \Omega$ is a parameter vector and $\Omega$ is a Banach space. Take the same assumptions as in the Gap Lemma, i.e. 

\begin{enumerate}
	\item The map $\Lambda \mapsto A(\cdot; \Lambda)$ is analytic in $\Lambda$.
	\item $A(x; \Lambda) \rightarrow A_\pm(\lambda)$ (independent of $\Lambda$) as $x \rightarrow \pm \infty$, and for $|\Lambda| < \delta$ we have the uniform exponential decay estimates 
	\begin{align}
	\left| \frac{\partial^k}{\partial x^k} A(x; \Lambda) - A_\pm(\Lambda) \right| 
	&\leq C e^{-\theta |x|} && 0 \leq k \leq K
	\end{align}
	where $\alpha > 0$, $C > 0$, and $K$ is a nonnegative integer.
\end{enumerate}

Then in a neighborhood of any $\Lambda_0 \in \Omega$ there exist invertible linear transformations

\begin{align*}
P_+(x, \Lambda) &= I + \Theta_+(x, \Lambda) \\
P_-(x, \Lambda) &= I + \Theta_-(x, \Lambda) 
\end{align*}

defined on $\R^+$ and $\R^-$, respectively, such that

\begin{enumerate}[(i)]
\item The change of coordinates $W = P_\pm Z$ reduces \eqref{LambdaEVPconj} to the equations on $\R^\pm$

\begin{align}
Z'(x) = A^\pm(\Lambda) Z(x) + P_\pm(x, \Lambda)^{-1} B(x) P_\pm(x, \Lambda) Z(x) + P_\pm(x, \Lambda)^{-1} F(x)
\end{align}

where

\[
G(x; \Lambda) = P_\pm(x, \Lambda)^{-1} F(x)
\]

\item For any fixed $0 < \tilde{\theta} < \theta$, $0 \leq k \leq K+1$, and $j \geq 0$ we have the decay rates
\begin{align*}
\left| \partial_\Lambda^j \partial_x^k \Theta_\pm \right| \leq C(j, k)e^{-\tilde{\theta}|x|}
\end{align*}
\end{enumerate}
\begin{proof}
I have written out the proof somewhere else for the case where $B(x) = 0$ and $F(x) = 0$, which essentially follows Zum2018 but fills in more details. Only a small modification is necessary for the general case. It should also not be hard to modify the proof to work for a general Banach space $\Omega$ instead of a subset of $\C^n$.
\end{proof}
\end{lemma}

\subsection*{The Problem}

We want to do Lin's method on the eigenvalue problem for KdV5 with periodic BCs. Recall that from the existence problem, we write the $n-$pulse piecewise on the appropriate intervals as

\[
q_i^\pm(x) = q^\pm(x; \beta_i^\pm) + u_i^\pm(x)
\]

The functions $q^\pm(x; \beta_i^\pm)$ evolve in the stable/unstable manifolds with initial conditions $\beta_i^\pm$. The functions $u_i^\pm(x)$ are small remainder terms. We have bounds on all of these from the existence problem.

\begin{align*}
|q^\pm(x; \beta_i^\pm)| &\leq C |\beta_i^\pm| e^{-\alpha |x|} \\
|\beta_i^\pm| &\leq C (e^{-2 \alpha X_i} + e^{-2 \alpha X_{i-1}}) \\
|u_i^-(x)| &\leq C e^{-\alpha X_{i-1}} e^{-\alpha(X_{i-1} + x) } \\
|u_i^+(x)| &\leq C e^{-\alpha X_i} e^{-\alpha(X_i - x) } 
\end{align*}

The piecewise eigenvalue problem we need to solve is then

\begin{align*}
&(W_i^\pm)' = A_i^\pm(x; \lambda) W_i^\pm + \lambda^2 d_i \tilde{H}_i^\pm \\
&W_i^\pm(0) \in \C \Psi(0) \oplus Y^+ \oplus Y^- \oplus Y^0 \\
&W_i^-(0) = W_i^+(0) \\
&W_i^+(X_i) - W_{i+1}^-(-X_i) = D_i d
\end{align*}

where

\begin{align*}
A_i^\pm(x; \lambda) &= A(x; q^\pm(x; \beta_i^\pm + u_i^\pm(x)) \\
A(x; f(x), \lambda) &= \begin{pmatrix}0 & 1 & 0 & 0 & 0 \\0 & 0 & 1 & 0 & 0 \\0 & 0 & 0 & 1 & 0 \\0 & 0 & 0 & 0 & 1 \\
2 \partial_x f(x) + \lambda & 2 f(x) - c & 0 & 1 & 0 \end{pmatrix} \\
D_i d &= d_{i+1}[Q_{i+1}'(-X_i) + \lambda \partial_c Q_{i+1}(-X_i)]
- d_i [ Q_i'(X_i) + \lambda \partial_c Q_i(-X_i) ] \\
\end{align*}

and we have estimates/bounds from the existence problem and from San98

\begin{align*}
|H(x)|, |\tilde{H}_i^\pm(x)| &\leq C e^{-\alpha |x|} \\
|\Delta H_i^\pm| &= |\tilde{H}_i^\pm - H| \leq C(e^{-\alpha X_i} + e^{-\alpha X_{i-1}} ) \\
|\Delta H_i^-(x)| &\leq C e^{-\alpha X_{i-1}} e^{-\alpha(X_{i-1} + x) } \\
|\Delta H_i^+(x)| &\leq C e^{-\alpha X_i} e^{-\alpha(X_i - x) } \\
D_i d &= ( Q'(X_i) + Q'(-X_i))(d_{i+1} - d_i ) + \mathcal{O} \left( e^{-\alpha X_i} \left( |\lambda| +  e^{-\alpha X_i}  \right) |d| \right) \\
\end{align*}

We want to use the Conjugation Lemma so that the evolution of the variational equation does not depend on the parameter vector $\Lambda = (\lambda, \beta_i^\pm, u_i^\pm(x))$. Note that this last component of the parameter is in the Banach space of continuous functions on the appropriate interval, thus we will need to use a version of the Conjugation Lemma which allows for this. $A_i^\pm(x; \lambda)$ is linear, thus analytic, in $\lambda$ and in $u_i^\pm(x)$. It should also depend nicely on the initial conditions $\beta_i^\pm$. For all parameter values, $A^\pm(x; \lambda)$ decays exponentially to $A(\lambda)$, which is the constant-coefficient, $\lambda-$dependent matrix 

\begin{align*}
A(\lambda) &=  \begin{pmatrix}0 & 1 & 0 & 0 & 0 \\0 & 0 & 1 & 0 & 0 \\0 & 0 & 0 & 1 & 0 \\0 & 0 & 0 & 0 & 1 \\
\lambda & -c & 0 & 1 & 0 \end{pmatrix}
\end{align*}

Let

\begin{equation}
P_i^\pm(x; \lambda) = P^\pm(x; \lambda, \beta_i^\pm, u_i^\pm)
\end{equation}

be the conjugation operator for $A_i^\pm(x; \lambda) = A(x; q^\pm(x; \beta_i^\pm + u_i^\pm(x))$.  Using the Conjugation Lemma, make the substitution $W_i^\pm = P_i^\pm(x; \lambda) Z_i^\pm$. Then the eigenvalue problem becomes

\begin{align*}
&(Z_i^\pm(x))' = A(\lambda) Z_i^\pm(x) + \lambda^2 d_i P_i^\pm(x; \lambda)^{-1} \tilde{H}_i^\pm(x) \\
&P_i^-(0; \lambda) Z_i^-(0) = P_i^+(0; \lambda) Z_i^+(0) \\
&P_i^\pm(0; \lambda) Z_i^\pm(0) \in \C \Psi(0) \oplus Y^+ \oplus Y^- \oplus Y^0 \\
&P_i^+(X_i; \lambda) Z_i^+(X_i)\ - P_{i+1}^-(-X_i; \lambda) Z_{i+1}^-(-X_i; \lambda) = D_i d
\end{align*}

Since $A(\lambda)$ is constant coefficient (and we know exactly what it is), we don't have to bother with the exponential trichotomy stuff, since everything evolves in the eigenspaces of $A(\lambda)$, which do not depend on $x$. Before we proceed, make the following hypothesis.

\begin{hypothesis}\label{Aspectrumhyp}
\begin{enumerate}
	\item The spectrum of $A(0)$ has isolated, simple eigenvalues at $\{ 0, \pm \alpha_0 \pm \beta_0 \}$, where $\alpha_0, \beta_0 > 0$. The real part of any other eigenvalue of $A(0)$ lies outside the interval $[-\alpha_0, \alpha_0]$.
\end{enumerate}
\end{hypothesis}

We then define the following.

\begin{enumerate}
	\item Let

	\begin{align*}
	X_m &= \min(X_0, \dots, X_{n-1}) \\
	X_M &= \max(X_0, \dots, X_{n-1}) \\
	\end{align*}

	\item Let $n_-$ be the number of eigenvalues of $A(0; 0)$ with negative real part, and $n_+$ be the number of eigenvalues of $A(0; 0)$ with positive real part. Then, by Hypothesis \ref{Aspectrumhyp}, $n_-, n_+ \geq 2$, and $n_- + n_+ + 1 = m$.

	\item Let $\nu(\lambda)$ be the simple, small eigenvalue of $A(0; \lambda)$. By Hypothesis \ref{Aspectrumhyp}, $\nu(0) = 0$, and $\nu(\lambda) = \mathcal{O}(\lambda)$. 

	\item Let $\rho > 0$, $\delta > 0$ be a small. How small will be determined later. We will take $|\lambda| < \delta$.

	\item Let $\alpha = \alpha_0 - \rho$. We choose $\delta$ sufficiently small so that for all $|\lambda| < \delta$,

	\begin{enumerate}
		\item $|\nu(\lambda)| < \rho$
		\item The real part of any other eigenvalue of $A(0; \lambda)$ lies outside the interval $[-\alpha, \alpha]$.
	\end{enumerate}

	\item Let $\tilde{\alpha} = \alpha - 3 \rho > 0$.

	\item Choose $X_m$ sufficiently large so that
	\begin{equation}
	e^{-\tilde{\alpha} X_m}, |\lambda|, ||\Delta H|| < \delta
	\end{equation}

\end{enumerate}

Let $P^{u/s/c}_0(\lambda)$ be the eigenprojections for the unstable/stable/center subspaces $E^{u/s/c}(\lambda)$ of $A(\lambda)$. The center subspace is a ``true'' center subspace only when $\nu(\lambda)$ has no real part, e.g. when $\lambda = 0$, but we will always call it a center subspace for convenience. \\

Let $\Phi(x, y; \lambda) = e^{A(\lambda)(x-y)}$ be the evolution of the constant-coefficient ODE

\[
Z' = A(\lambda) Z
\]

Let $\Phi^{u/s/c}(x, y; \lambda)$ be the evolutions on the respective eigenspaces. Since $E^c(\lambda)$ is one-dimensional, we have in particular that

\begin{align*}
\Phi^c(x, y; \lambda) v &= e^{\nu(\lambda)(x - y)} v && v \in E^c(\lambda)
\end{align*}

We also have the bounds

\begin{align*}
|\Phi^s(x, y; \lambda)| &\leq C e^{-\alpha(x - y)} \\
|\Phi^u(x, y; \lambda)| &\leq C e^{-\alpha(y - x)} \\
|\Phi^c(x, y; \lambda)| &\leq C e^{\rho|x - y|} 
\end{align*}

We will want to relate things to the variational/adjoint variational problem for the nontransformed system, in particular to the variational equation for the linearization about the primary pulse.

\begin{align*}
V_i' &= A(q(x), 0) V_i \\
W_i' &= -A(q(x), 0)^* W_i
\end{align*}

Recall that $Q'(x)$ solves the variational problem, and we have solutions $\Psi(x)$ and $1$ (in the periodic case) to the adjoint variational problem. Let $P^\pm(x)$ conjugate $A(q(x), 0)$.\\

Let $\Theta(y, x)$ be the evolution for the untransformed variational equation. We would like to relate this to the evolution of the transformed equation. For appropriate values of $x, y$, we have

\[
\Theta(y, x) = P^\pm(y) \Phi(y, x; 0) P^\pm(x)^{-1}
\]

Finally, we will have occasion to Taylor expand the conjugation operators. Note that 

\[
P^\pm(y) = P^\pm(y; 0, 0, 0)
\]

For $y$ large, all the conjugation operators are approximately the identity, so that will not matter. The Taylor expansion we will need is that at $x = 0$.

\begin{equation}\label{PTaylor}
P_i^\pm(0; \lambda) = P^\pm(0) + \mathcal{O}(|\lambda| + e^{-\alpha X_m})
\end{equation}

\subsection*{The Inversion}

Define the spaces

\begin{align*}
V_a &= \bigoplus_{i=0}^{n-1} E^u(\lambda) \oplus E^s(\lambda) \\
V_b &= \bigoplus_{i=0}^{n-1} E^u(0) \oplus E^s(0) \\
V_c^+ &= \bigoplus_{i=0}^{n-1} E^c(\lambda) \\
V_c^- &= \bigoplus_{i=0}^{n-1} E^c(\lambda) \\
V_c &= V_c^+ \oplus V_c^- \\
V_\lambda &= B_\delta(0) \subset \C
\end{align*}

where the subscripts are all $\mod n$, as in the existence problem. We use the $\lambda-$dependent eigenspaces for $a_i^\pm$ and $c_i^\pm$, since we will be evolving them under the $\lambda-$dependent evolution. All the product spaces are endowed with the maximum norm, e.g. for $V_c$, $|c| = \max(|c_0^-|, \dots, |c_{n-1}^-|, |c_0^+|, \dots, |c_{n-1}^+|)$. In addition, we take the following convention: if we eliminate either a subscript or a superscript (or both) in the norm, we are taking the maximum over the eliminated thing. For example,
\begin{enumerate}
	\item $|c_i| = \max(|c_i^+|, |c_i^-|)$ 
	\item $|c^+| = \max(|c_0^+|, \dots, |c_{n-1}^+|)$
\end{enumerate}

Next, we write down the fixed point equations for the problem. For $i = 0, \dots, n-1$, the fixed point equations are

\begin{align*}
Z_i^-(x) &= \Phi^s(x, -X_{i-1}; \lambda) a_{i-1}^- + \Phi^u(x, 0; \lambda) b_i^- + \Phi^c(x, -X_{i-1}; \lambda) c_{i-1}^- \\
&+ \lambda^2 d_i \int_0^x \Phi^u(x, y; \lambda) P_i^-(y; \lambda)^{-1} \tilde{H}_i^-(y)] dy \\
&+ \lambda^2 d_i \int_{-X_{i-1}}^x \Phi^s(x, y; \lambda) P_i^-(y; \lambda)^{-1} \tilde{H}_i^-(y) dy \\
&+ \lambda^2 d_i \int_{-X_{i-1}}^x \Phi^c(x, y; \lambda) P_i^-(y; \lambda)^{-1} \tilde{H}_i^-(y) dy  \\ 
Z_i^+(x) &= \Phi^u(x, X_i; \lambda) a_i^+ + \Phi^s(x, 0; \lambda) b_i^+ + \Phi^c(x, X_i; \lambda) c_i^+ \\
&+ \lambda^2 d_i \int_0^x \Phi^s(x, y; \lambda) P_i^+(y; \lambda)^{-1} \tilde{H}_i^+(y) dy \\
&+ \lambda^2 d_i \int_{X_i}^x \Phi^u(x, y; \lambda) P_i^+(y; \lambda)^{-1} \tilde{H}_i^+(y) dy \\
&+ \lambda^2 d_i \int_{X_i}^x \Phi^c(x, y; \lambda) P_i^+(y; \lambda)^{-1} \tilde{H}_i^+(y) dy \\
\end{align*}

% match at ends

\subsubsection*{Matching at ends}

Here, we solve the condition

\[
P_i^+(X_i; \lambda) Z_i^+(X_i) - P_{i+1}^-(-X_i; \lambda) Z_{i+1}^-(-X_i) = D_i d
\]

At $\pm X_i$, the fixed point equations become

\begin{align*}
Z_{i+1}^-(-X_i) &= a_i^- + \Phi^u(-X_i, 0; \lambda) b_{i+1}^- + c_i^- 
+ \lambda^2 d_{i+1} \int_0^{-X_i} \Phi^u(-X_i, y; \lambda) P_{i+1}^-(y; \lambda)^{-1} \tilde{H}_i^-(y) dy \\
Z_i^+(X_i) &= a_i^+ + \Phi^s(X_i, 0; \lambda) b_i^+ + c_i^+ 
+ \lambda^2 d_i \int_0^{X_i} \Phi^s(X_i, y; \lambda) P_i^+(y; \lambda)^{-1} \tilde{H}_i^+(y) dy
\end{align*}

To obtain these, we used the fact that, for example, $a_i^- \in E^s(\lambda)$ and $\Phi^s(-X_{i-1}, -X_{i-1}; \lambda)$ is the identity on $E^s(\lambda)$. From the Conjugation Lemma, we have

\begin{equation}\label{conjest}
P_i^\pm(\pm X_i; \lambda) = I + \mathcal{O}(e^{-\alpha X_i})
\end{equation}

which we will use on the $a_i^\pm$ and $c_i^\pm$ terms. Thus we obtain the equation

\begin{align}\label{Dideq1}
D_i d &= a_i^+ - a_i^- + c_i^+ - c_i^- + L_3(\lambda)_i(a, b, c^+, c^-, d)
\end{align}

For a bound on $L_3$, we look at the individual terms. As usual, we will in general only look at one of the two pieces.

\begin{enumerate}

\item For the $a_i^\pm$ and $c_i^\pm$ terms, we have a term of order $\mathcal{O}(e^{-\alpha X_i}(|a_i| + |c_i^+| + |c_i^-|)$, which comes from the conjugation operators $P_i^\pm(\pm X_i; \lambda)$.

\item For the terms involving $b$, we have

\[
| P_i^-(-X_i; \lambda) \Phi^u(-X_i, 0; \lambda) b_{i+1}^-| \leq C e^{-\alpha X_i} |b_{i+1}
^-|
\]

\item For the integral terms, we have

\begin{align*}
&\left|
P^+(X_i; \beta_i^+, \lambda) \int_0^{X_i} \Phi^s(X_i, y; \lambda) P^+(X_i; \beta_i^+, \lambda)^{-1} \tilde{H}_i^+(y) dy \right| \\
&\leq C \int_0^{X_i} e^{-\alpha(X_i - y)}e^{-\alpha y} dy \\
&\leq C \int_0^{X_i} e^{-(\alpha - \rho)(X_i - y)}e^{-\alpha y} dy \\
&= C e^{-(\alpha - \rho) X_i} \int_0^{X_i} e^{-\rho y} dy \\ 
&\leq C e^{-(\alpha - \rho) X_i} 
\end{align*}

\end{enumerate}

Putting these all together, we have the following bound for $L_3$.
\[
|L_3(\lambda)_i(a, b, c^+, c^-, d)| \leq C \Big( e^{-\alpha X_i} ( |a_i| + |b_i^+| + |b_{i+1}^-| + |c_i^+| + |c_i^-|) + e^{-(\alpha - \rho) X_i} |\lambda^2| |d| \Big)
\]

Following San98 (and leaving out some steps for now), we can solve this for $(a, c^+)$ to get $(a_i, c_i^+) = A_1(\lambda)_i(b, c_i^-, d)$, with bound

\begin{align*}
|A_1&(\lambda)_i(b, c^-, d)|
\leq C \Big( e^{-\alpha X_i} (|b_i^+| + |b_{i+1}^-|) + |c_i^-| + e^{-(\alpha - \rho) X_i} |\lambda^2||d| + |D_i||d| \Big)
\end{align*} 

As in San98, we hit \eqref{Dideq1} with projections on the subspaces eigenspaces $E^{s/u/c}(\lambda)$. The remainder term $A_2(\lambda)_i(b, c^-, d)$ is found by substituting the bound for $A_1$ into $L_3$ and simplifying.

\begin{align*}
a_i^+ &= P_0^u(\lambda) D_i d + A_2(\lambda)_i^+(b, c^-, d) \\
a_i^- &= -P_0^s(\lambda) D_i d + A_2(\lambda)_i^-(b, c^-, d) \\
c_i^+ &= c_i^- + P_0^c(\lambda) D_i d + A_2(\lambda)_i^c(b, c^-, d) )
\end{align*}

where we have bound

\begin{align*}
|A_2&(\lambda)_i(b, d)|
\leq C \Big( e^{-\alpha X_i} (|b_i^+| + |b_{i+1}^-| + |c_i^-|) + e^{-(\alpha - \rho) X_i} |\lambda|^2|d| + e^{-\alpha X_i} |D_i||d| \Big)
\end{align*} 

For the first two, this is not quite what we want. Anticipating what we will need later, we write $a_i^+$ as

\begin{align*}
a_i^+ = P_i^+(X_i; \lambda)a_i^+ + (I - P_i^+(X_i; \lambda))a_i^+ &= P_0^u(\lambda) D_i d + A_2(\lambda)_i^+(b, c^-, d)
\end{align*}

Rearranging this, we obtain

\begin{align*}
P_i^+(X_i; \lambda) a_i^+ &= P_0^u(\lambda) D_i d + A_2(\lambda)_i^+(b, c^-, d) - (I - P_i^+(X_i; \lambda))a_i^+ \\
&= P_0^u(\lambda) D_i d + A_2(\lambda)_i^+(b, c^-, d) + \mathcal{O}\Big( e^{-\alpha X_i} ( e^{-\alpha X_i} (|b_i^+| + |b_{i+1}^-|) + |c_i^-| + e^{-(\alpha - \rho) X_i} |\lambda^2||d| + |D_i||d| )\Big)
\end{align*}

where we used the bound $A_1$ and the estimate \eqref{conjest}. The last term on the RHS is the same (or higher) order as $A_2$, so we incorporate that into $A_2(\lambda)_i^+(b, c^-, d)$ to get

\begin{align*}
P_i^+(X_i; \lambda)a_i^+ &= P_0^u(\lambda) D_i d + A_2(\lambda)_i^+(b, c^-, d)
\end{align*}

Finally, we operate on both sides on the left by $P_i^+(X_i; \lambda)^{-1}$ to solve for $a_i^+$. This is a bounded operator, so we will also incorporate this into $A_2(\lambda)_i^+(b, c^-, d)$ (the bound will be unchanged). Thus we have

\begin{align*}
a_i^+ &= P^+(X_i; \beta_i^+, \lambda) P_0^u(\lambda) D_i d + A_2(\lambda)_i^+(b, c^-, d)
\end{align*}

We do the same thing for $a_i^-$, giving us

\begin{align*}
a_i^+ &= P_i^+(X_i; \lambda) P_0^u(\lambda) D_i d + A_2(\lambda)_i^+(b, c^-, d) \\
a_i^- &= -P_i^-(-X_i; \lambda) P_0^s(\lambda) D_i d + A_2(\lambda)_i^-(b, c^-, d)
\end{align*}

For the third one, we would like to evaluate and get an estimate for $P_0^c(\lambda) D_i d$. Recall that

\[
D_i d &= ( Q'(X_i) + Q'(-X_i))(d_{i+1} - d_i ) + \mathcal{O} \left( e^{-\alpha X_i} \left( |\lambda| +  e^{-\alpha X_i}  \right) |d| \right) 
\]

Looking at the lower order terms,

\begin{align*}
P_0^c(\lambda)&( Q'(X_i) + Q'(-X_i)) 
= P_0^c(0)( Q'(X_i) + Q'(-X_i)) + \mathcal{O}(|\lambda|e^{-\alpha X_i}) \\
&= \mathcal{O}(e^{-\alpha X_i}(|\lambda| + e^{-\alpha X_i}))
\end{align*}

Thus we have

\[
|P_0^c(\lambda) D_i d| \leq C e^{-\alpha X_i}(|\lambda| + e^{-\alpha X_i})|d|
\]

\subsubsection*{Matching at 0}

The next step is to satisfy the conditions

\begin{align*}
P_i^\pm(0; \lambda) Z_i^\pm(0) &\in \C \Psi(0) \oplus Y^0 \oplus Y^+ \oplus Y^- \\
P_i^+(0; \lambda) Z_i^+(0) - P_i^-(0; \lambda) Z_i^-(0) &\in \C \Psi(0) \oplus Y^0
\end{align*}

Recall that we have

\[
\C^m = \C \Psi(0) \oplus \C Q'(0) \oplus Y^0 \oplus Y^+ \oplus Y^- 
\]

This condition is equivalent to the three projections

\begin{align*}
P(\C Q'(0) ) P_i^-(0; \lambda) Z_i^-(0) &= 0 \\
P(\C Q'(0) ) P_i^+(0; \lambda) Z_i^+(0) &= 0 \\
P(Y_i^+ \oplus Y_i^-) ( P_i^+(0; \lambda) Z_i^+(0) - P_i^-(0; \lambda) Z_i^-(0) ) &= 0
\end{align*}

where the kernel of each projection is the remaining spaces in the direct sum. We don't need $\C Q'(0)$ in the third equation since we eliminated any component in it in the first two equations.\\

Recall that for $\lambda = 0$, the tangent space to the stable manifold at $x = 0$ is spanned by $Y^+$ and $Q'(0)$, and the tangent space to the unstable manifold at $x = 0$ is spanned by $Y^-$ and $Q'(0)$. Thus we have

\begin{align*}
P^-(0)^{-1} Q'(0) &= v^- \in E^u(0) \\
P^+(0)^{-1} Q'(0) &= v^+ \in E^s(0)
\end{align*}

Let

\begin{align*}
E^u(0) &= \C v^- \oplus E^- \\
E^s(0) &= \C v^+ \oplus E^+ \\
\end{align*}

Then we have

\begin{align*}
P^-(0)^{-1} Y^- = E^- \\
P^+(0)^{-1} Y^+ = E^+ \\
\end{align*}

Following San98, we decompose $b_i^\pm$ uniquely as $b_i^\pm = x_i^\pm + y_i^\pm$, where $x_i^\pm \in \C v^\pm$ and $y_i^\pm \in E^\pm$.\\

At $x = 0$, the fixed point equations become

\begin{align*}
Z_i^-(0) &= \Phi^s(0, -X_{i-1}; \lambda) a_{i-1}^- + \Phi^u(0, 0; \lambda) b_i^- + \Phi^c(0, -X_{i-1}; \lambda) c_{i-1}^- \\
&+ \lambda^2 d_i \int_{-X_{i-1}}^0 \Phi^s(0, y; \lambda) P_i^-(y; \lambda)^{-1} \tilde{H}_i^-(y) dy \\
&+ \lambda^2 d_i \int_{-X_{i-1}}^0 \Phi^c(0, y; \lambda) P_i^-(y; \lambda)^{-1} \tilde{H}_i^-(y) dy  \\ 
Z_i^+(0) &= \Phi^u(0, X_i; \lambda) a_i^+ + \Phi^s(0, 0; \lambda) b_i^+ + \Phi^c(0, X_i; \lambda) c_i^+ \\
&+ \lambda^2 d_i \int_{X_i}^0 \Phi^u(0, y; \lambda) P_i^+(y; \lambda)^{-1} \tilde{H}_i^+(y) dy \\
&+ \lambda^2 d_i \int_{X_i}^0 \Phi^c(0, y; \lambda) P_i^+(y; \lambda)^{-1} \tilde{H}_i^+(y) dy \\
\end{align*}

Noting that $\Phi^u(0, 0; \lambda) = P_0^u(0)$, doing a little manipulation on the $b_i$ terms, and using the known form of the evolution $\Phi^c$ on $E^c(\lambda)$, this becomes

\begin{align*}
Z_i^-(0) &= \Phi^s(0, -X_{i-1}; \lambda) a_{i-1}^- + x_i^- + y_i^- + (P_0^u(\lambda) - P_0^u(0))b_i^- + e^{\nu(\lambda) X_{i-1}} c_{i-1}^- \\
&+ \lambda^2 d_i \int_{-X_{i-1}}^0 \Phi^s(0, y; \lambda) P_i^-(y; \lambda)^{-1} \tilde{H}_i^-(y) dy \\
&+ \lambda^2 d_i \int_{-X_{i-1}}^0 \Phi^c(0, y; \lambda) P_i^-(y; \lambda)^{-1} \tilde{H}_i^-(y) dy  \\ 
Z_i^+(0) &= \Phi^u(0, X_i; \lambda) a_i^+ + x_i^+ + y_i^+ + (P_0^s(\lambda) - P_0^s(0)) b_i^+ + e^{-\nu(\lambda)X_i} c_i^+ \\
&+ \lambda^2 d_i \int_{X_i}^0 \Phi^u(0, y; \lambda) P_i^+(y; \lambda)^{-1} \tilde{H}_i^+(y) dy \\
&+ \lambda^2 d_i \int_{X_i}^0 \Phi^c(0, y; \lambda) P_i^+(y; \lambda)^{-1} \tilde{H}_i^+(y) dy \\
\end{align*}

Since $c_i^\pm$ are in the eigenspaces $E^c(\lambda)$, we do some further manipulation to separate out a component in $E^c(0)$.

\begin{align*}
Z_i^-(0) &= \Phi^s(0, -X_{i-1}; \lambda) a_{i-1}^- + x_i^- + y_i^- + (P_0^u(\lambda) - P_0^u(0))b_i^- \\
&+ P_0^c(0) e^{\nu(\lambda) X_{i-1}} c_{i-1}^- + (P_0^c(\lambda) - P_0^c(0)) e^{\nu(\lambda) X_{i-1}} c_{i-1}^- \\
&+ \lambda^2 d_i \int_{-X_{i-1}}^0 \Phi^s(0, y; \lambda) P_i^-(y; \lambda)^{-1} \tilde{H}_i^-(y) dy \\
&+ \lambda^2 d_i \int_{-X_{i-1}}^0 \Phi^c(0, y; \lambda) P_i^-(y; \lambda)^{-1} \tilde{H}_i^-(y) dy  \\ 
Z_i^+(0) &= \Phi^u(0, X_i; \lambda) a_i^+ + x_i^+ + y_i^+ + (P_0^s(\lambda) - P_0^s(0)) b_i^+ \\
&+ P_0^c(0) e^{-\nu(\lambda)X_i} c_i^+ + (P_0^c(\lambda) - P_0^c(0)) e^{-\nu(\lambda)X_i} \\
&+ \lambda^2 d_i \int_{X_i}^0 \Phi^u(0, y; \lambda) P_i^+(y; \lambda)^{-1} \tilde{H}_i^+(y) dy \\
&+ \lambda^2 d_i \int_{X_i}^0 \Phi^c(0, y; \lambda) P_i^+(y; \lambda)^{-1} \tilde{H}_i^+(y) dy \\
\end{align*}

Finally, we operate on these by $P_i^\pm(0; \lambda)$. For the $c_i^-$ and $b$ terms, we write these as

\[
P_i^\pm(0; \lambda) = P^\pm(0) + (P_i^\pm(0; \lambda) - P^\pm(0))
\]

We finally wind up with

\begin{align*}
P_i^-(0; \lambda) Z_i^-(0) &= P^-(0)( x_i^- + y_i^- + P_0^c(0) e^{\nu(\lambda) X_{i-1}} c_{i-1}^- ) \\
&+ P_i^-(0; \lambda) \Phi^s(0, -X_{i-1}; \lambda) a_{i-1}^- + (P_i^-(0; \lambda) - P^-(0))b_i^- + P_i^-(0; \lambda)(P_0^u(\lambda) - P_0^u(0))b_i^- \\
&+ (P_i^-(0; \lambda) - P^-(0)) P_0^c(0) e^{\nu(\lambda) X_{i-1}} c_{i-1}^- + P_i^-(0; \lambda) (P_0^c(\lambda) - P_0^c(0)) e^{\nu(\lambda) X_{i-1}} c_{i-1}^- \\
&+ \lambda^2 d_i P_i^-(0; \lambda) \int_{-X_{i-1}}^0 \Phi^s(0, y; \lambda) P_i^-(y; \lambda)^{-1} \tilde{H}_i^-(y) dy \\
&+ \lambda^2 d_i P_i^-(0; \lambda) \int_{-X_{i-1}}^0 \Phi^c(0, y; \lambda) P_i^-(y; \lambda)^{-1} \tilde{H}_i^-(y) dy  \\ 
P_i^+(0; \lambda) Z_i^+(0) &=  P^+(0)( x_i^+ + y_i^+ + P_0^c(0) e^{-\nu(\lambda)X_i} c_i^+ )\\
&+ P_i^+(0; \lambda) \Phi^u(0, X_i; \lambda) a_i^+ + (P_i^+(0; \lambda) - P^+(0)) b_i^+ + P_i^+(0; \lambda) (P_0^s(\lambda) - P_0^s(0)) b_i^+ \\
&+ (P_i^+(0; \lambda) - P^+(0))P_0^c(0) e^{-\nu(\lambda)X_i} c_i^+ + P_i^+(0; \lambda) (P_0^c(\lambda) - P_0^c(0)) e^{-\nu(\lambda)X_i} c_i^+\\
&+ \lambda^2 d_i P_i^+(0; \lambda) \int_{X_i}^0 \Phi^u(0, y; \lambda) P_i^+(y; \lambda)^{-1} \tilde{H}_i^+(y) dy \\
&+ \lambda^2 d_i P_i^+(0; \lambda) \int_{X_i}^0 \Phi^c(0, y; \lambda) P_i^+(y; \lambda)^{-1} \tilde{H}_i^+(y) dy \\
\end{align*}

Note that with this setup, the projections we will take either eliminate or act as the identity on the terms in the first lines of $P_i^-(0; \lambda) Z_i^-(0)$ and $P_i^+(0; \lambda) Z_i^+(0)$. Thus we obtain an expression of the form

\[
\begin{pmatrix}x_i^- \\ x_i^+ \\ 
y_i^+ - y_i^- \end{pmatrix} + L_4(\lambda)_i(b, c, \tilde{c}, d) = 0
\]

where, for convenience, we define

\begin{equation}\label{tildec}
\tilde{c}_i^\pm = e^{\pm \nu(\lambda) X_i} c_i^-
\end{equation}

To get a bound on $L_4$, we need to bound the individual terms from the fixed point equations above.

\begin{enumerate}

\item For the $a_i$ terms, we substitute the bound for $A_1(\lambda)$ to get

\begin{align*}
|P_i^-(0; \lambda) \Phi^s(0, -X_{i-1}; \lambda) a_{i-1}^-|
&\leq C \Big( e^{-2 \alpha X_{i-1}} (|b_{i-1}^+| + |b_i^-|) + e^{-\alpha X_{i-1}}|c_{i-1}^-| + e^{-\alpha X_{i-1}}(e^{-(\alpha - \rho) X_{i-1}} |\lambda^2| + |D_{i-1}|)|d| \Big) \\
|P_i^+(0; \lambda) \Phi^u(0, X_i; \lambda) a_i^+|
&\leq C \Big( e^{-2 \alpha X_i} (|b_i^+| + |b_{i+1}^-|) + e^{-\alpha X_i} |c_i^-| + e^{-\alpha X_i} (e^{-(\alpha - \rho) X_i} |\lambda^2| + |D_i|)|d| \Big)
\end{align*}

\item For the $b_i$ terms, we have

\[
|(P_i^-(0; \lambda) - P^-(0))b_i^- + P_i^-(0; \lambda)(P_0^u(\lambda) - P_0^u(0))b_i^-| \leq C ( e^{-\alpha X_m} + |\lambda|)|b_i^-|
\]

\item For the $c_i^-$ terms, we have

\begin{align*}
|P_i^-(0; \lambda) - P^-(0)) P_0^c(0) e^{\nu(\lambda) X_{i-1}} c_{i-1}^- + P_i^-(0; \lambda) (P_0^c(\lambda) - P_0^c(0)) e^{\nu(\lambda) X_{i-1}} c_{i-1}^- |
\leq C (e^{-\alpha X_m} + |\lambda|)|\tilde{c}_{i-1}^+|)
\end{align*}

\item For the $c_i^+$ terms, we have

\begin{align*}
|P_i^+(0; \lambda) - P^+(0))P_0^c(0) e^{-\nu(\lambda)X_i} c_i^+ + P_i^+(0; \lambda) (P_0^c(\lambda) - P_0^c(0)) e^{-\nu(\lambda)X_i} c_i^+| \leq C (e^{-\alpha X_m} + |\lambda|)|e^{-\nu(\lambda)X_i} c_i^+|
\end{align*}

The only extra complication for $c_i^+$ is that we have to use the expression from the previous section involving $A_2$ to write $c_i^+$ in terms of $c_i^-$. Doing this, we obtain

\begin{align*}
e^{-\nu(\lambda)X_i} c_i^+ &= e^{-\nu(\lambda)X_i} c_i^- 
+ e^{-\nu(\lambda)X_i} P_0^c(\lambda) D_i d + e^{-\nu(\lambda)X_i} A_2(\lambda)_i^c(b, d)\\
&= e^{-\nu(\lambda)X_i} c_i^- + \mathcal{O}\Big( e^{-(\alpha - \rho) X_i} ( |\lambda| + e^{-\alpha X_i} ) |d|) + e^{-(\alpha - \rho) X_i} (|b_i^+| + |b_{i+1}^-| + |c_i^-|)\\
&+ e^{-(\alpha - 2 \rho) X_i} |\lambda|^2|d| + e^{-(\alpha - \rho) X_i} |D_i||d| ) \\
&= e^{-\nu(\lambda)X_i} c_i^- + \mathcal{O}\Big( e^{-(\alpha - 2 \rho) X_i} ( |b_i^+| + |b_{i+1}^-| + |c_i^-| + |\lambda||d| + |D_i||d|) \Big) \\
\end{align*}

Thus we have

\begin{align*}
&|P_i^+(0; \lambda) - P^+(0))P_0^c(0) e^{-\nu(\lambda)X_i} c_i^+ + P_i^+(0; \lambda) (P_0^c(\lambda) - P_0^c(0)) e^{-\nu(\lambda)X_i} c_i^+| \\
&\leq C \Big( (e^{-\alpha X_m} + |\lambda|)|\tilde{c}_i^-| + e^{-(\alpha - 2 \rho) X_i} ( |b_i^+| + |b_{i+1}^-| + |c_i^-| + |\lambda||d| + |D_i||d|) \Big)
\end{align*}

\item The bound on the integral terms is determined by the bound on the center subspace, since there is potential growth in that subspace. The integral terms involving $\tilde{H}$ are bounded by

\begin{align*}
\left| \lambda^2 d_i P_i^-(0; \lambda) \int_{-X_{i-1}}^0 \Phi^c(0, y; \lambda) P_i^-(y; \lambda)^{-1} \tilde{H}_i^-(y) dy \right| &\leq C |\lambda|^2 |d| \int_{-X_{i-1}}^0 e^{-\rho y} e^{\alpha y} dy \\
&\leq C |\lambda|^2 |d|
\end{align*}

\end{enumerate}

Putting all these together, we obtain the bound for $L_4(\lambda)_i(b, \tilde{c}, d)$. Note that the $\tilde{c}$ depend on the $c$, but we separate them out for convenience.

\begin{align*}
L_4(\lambda)_i(b, \tilde{c}, d) &\leq 
C\Big( (|\lambda| + e^{-\tilde{\alpha}X_m})|b| 
+ (|\lambda| + e^{-\tilde{\alpha}X_m}) |\tilde{c}_{i-1}^+| + |\tilde{c}_i^-|) + e^{-\tilde{\alpha} X_{i-1}} |c_{i-1}^-| + e^{-\tilde{\alpha} X_i} |c_i^-| \\
&+ ( e^{-\tilde{\alpha}X_m} |D| + e^{-\tilde{\alpha}X_m}|\lambda| + |\lambda|^2)|d| \Big)
\end{align*}

Peforming the inversion, we solve for $b$ to get $B_1(\lambda)(\tilde{c}, d)$, which has bound

\begin{align*}
|B_1(\lambda)_i(\tilde{c}, d)| \leq C\Big( 
(|\lambda| + e^{-\tilde{\alpha}X_m})( |\tilde{c}_{i-1}^+| + |\tilde{c}_i^-|)
+ e^{-\tilde{\alpha} X_{i-1}} |c_{i-1}^-| + e^{-\tilde{\alpha} X_i} |c_i^-| + ( e^{-\tilde{\alpha}X_m} |D| + e^{-\tilde{\alpha}X_m}|\lambda| + |\lambda|^2)|d| \Big)
\end{align*}

We can plug this into the bound for $A_2$ to get $A_4$ with bound

\begin{align*}
|A_4&(\lambda)_i(\tilde{c}, d)|
\leq C \Big( 
e^{-\alpha X_i} (|\lambda| + e^{-\tilde{\alpha}X_m})(|\tilde{c}_{i-1}^+| + |\tilde{c}_{i+1}^-|) + e^{-\tilde{\alpha}X_{i-1}}|c_{i-1}^-| + e^{-\tilde{\alpha}X_i}|c_i^-| + e^{-\tilde{\alpha}X_{i+1}}|c_{i+1}^-| \\
&+ e^{-\tilde{\alpha} X_m} |\lambda|^2|d| + e^{-\alpha X_m}|D||d| \Big)
\end{align*} 

We will also plug $B_1$ into the expression for $e^{-\nu(\lambda)X_i} c_i^+$.

\begin{align*}
e^{-\nu(\lambda)X_i} &c_i^+ = e^{-\nu(\lambda)X_i} c_i^- 
+ \mathcal{O}\Big( e^{-\tilde{\alpha}X_m} (|\lambda| + e^{-\tilde{\alpha}X_m})( |\tilde{c}_{i-1}^+| + |\tilde{c}_{i+1}^-|) 
+ e^{-\tilde{\alpha}X_i}|c_i^-| \\
&+ e^{-\tilde{\alpha}X_m}( e^{-\alpha X_{i-1}}|c_{i-1}^-| + e^{-\alpha X_{i+1}}|c_{i+1}^-| ) +|\lambda||d| + |D_i||d|) \Big) \\
\end{align*}

Now that we have solved (uniquely) for everything except for the $c_i^-$ and $d$, we are ready to compute the jump conditions in the two directions.

\subsubsection*{Jump in Y0 direction}

For this jump, we project on $Y_0$.

\[
\xi^c_i = P(Y^0) ( P_i^+(0; \lambda) Z_i^+(0) - P_i^-(0; \lambda) Z_i^-(0) )
\]

Recall that, from the previous section, the terms $P_i^\pm(0; \lambda) Z_i^\pm(0)$ are given by

\begin{align*}
P_i^-(0; \lambda) Z_i^-(0) &= P^-(0)( b_i^- + P_0^c(0) e^{\nu(\lambda) X_{i-1}} c_{i-1}^- ) \\
&+ P_i^-(0; \lambda) \Phi^s(0, -X_{i-1}; \lambda) a_{i-1}^- + (P_i^-(0; \lambda) - P^-(0))b_i^- + P_i^-(0; \lambda)(P_0^u(\lambda) - P_0^u(0))b_i^- \\
&+ (P_i^-(0; \lambda) - P^-(0)) P_0^c(0) e^{\nu(\lambda) X_{i-1}} c_{i-1}^- + P_i^-(0; \lambda) (P_0^c(\lambda) - P_0^c(0)) e^{\nu(\lambda) X_{i-1}} c_{i-1}^- \\
&+ \lambda^2 d_i P_i^-(0; \lambda) \int_{-X_{i-1}}^0 \Phi^s(0, y; \lambda) P_i^-(y; \lambda)^{-1} \tilde{H}_i^-(y) dy \\
&+ \lambda^2 d_i P_i^-(0; \lambda) \int_{-X_{i-1}}^0 \Phi^c(0, y; \lambda) P_i^-(y; \lambda)^{-1} \tilde{H}_i^-(y) dy  \\ 
P_i^+(0; \lambda) Z_i^+(0) &=  P^+(0)( b_i^+ + P_0^c(0) e^{-\nu(\lambda)X_i} c_i^+ )\\
&+ P_i^+(0; \lambda) \Phi^u(0, X_i; \lambda) a_i^+ + (P_i^+(0; \lambda) - P^+(0)) b_i^+ + P_i^+(0; \lambda) (P_0^s(\lambda) - P_0^s(0)) b_i^+ \\
&+ (P_i^+(0; \lambda) - P^+(0))P_0^c(0) e^{-\nu(\lambda)X_i} c_i^+ + P_i^+(0; \lambda) (P_0^c(\lambda) - P_0^c(0)) e^{-\nu(\lambda)X_i} c_i^+\\
&+ \lambda^2 d_i P_i^+(0; \lambda) \int_{X_i}^0 \Phi^u(0, y; \lambda) P_i^+(y; \lambda)^{-1} \tilde{H}_i^+(y) dy \\
&+ \lambda^2 d_i P_i^+(0; \lambda) \int_{X_i}^0 \Phi^c(0, y; \lambda) P_i^+(y; \lambda)^{-1} \tilde{H}_i^+(y) dy \\
\end{align*}

We will look at the significant terms first. There should be two of them.

\begin{enumerate}
\item For the terms involving $c$, we have

\begin{align*}
P(Y^0) &P^-(0)( P_0^c(0) e^{\nu(\lambda) X_{i-1}} c_{i-1}^- - P_0^c(0) e^{-\nu(\lambda)X_i} c_i^+) = e^{\nu(\lambda) X_{i-1}} c_{i-1}^- - e^{-\nu(\lambda)X_i} c_i^+ \\
&= e^{\nu(\lambda) X_{i-1}} c_{i-1}^- - e^{-\nu(\lambda)X_i} c_i^- + e^{-\nu(\lambda)X_i} P_0^c(\lambda) D_i d + e^{-\nu(\lambda)X_i} A_4(\lambda)_i^c(c, \tilde{c}, d)\\
&= e^{\nu(\lambda) X_{i-1}} c_{i-1}^- - e^{-\nu(\lambda)X_i} c_i^- + \mathcal{O} e^{-\nu(\lambda)X_i} A_4(\lambda)_i^c(c, \tilde{c}, d)\\
\end{align*}

\begin{align*}
P^-(0; \beta_i^-, \lambda)&[ P_0^c(0) e^{\nu(\lambda) X_{i-1}} c_{i-1}^- + (P_0^c(\lambda) - P_0^c(0)) e^{\nu(\lambda) X_{i-1}} c_{i-1}^-] \\
&= P^-(0; 0, 0) P_0^c(0) \tilde{c}_{i-1}^+ + (P^-(0; \beta_i^-, \lambda) - P^-(0; 0, 0)) P_0^c(0) \tilde{c}_{i-1}^+ + P^-(0; \beta_i^-, \lambda) (P_0^c(\lambda) - P_0^c(0)) \tilde{c}_{i-1}^+ \\
&= P^-(0; 0, 0) P_0^c(0) e^{\nu(\lambda) X_{i-1}} c_{i-1}^- + \mathcal{O}((e^{-\alpha X_m} + |\lambda|)|\tilde{c}_{i-1}^+|)
\end{align*}

and

\begin{align*}
P^+(0; \beta_i^+, \lambda)&[ P_0^c(0) e^{-\nu(\lambda) X_i} c_i^+ + (P_0^c(\lambda) - P_0^c(0)) e^{-\nu(\lambda) X_i} c_i^+] \\
&= P^+(0; \beta_i^+, 0) P_0^c(0) e^{-\nu(\lambda) X_i} c_i^- + \mathcal{O}((e^{-\alpha X_m} + |\lambda|)|\tilde{c}_i^-|) \\
&+ \mathcal{O}\Big( e^{-\tilde{\alpha} X_i} ( |b_i^+| + |b_{i+1}^-| + |c_i^-| + (|\lambda| + |D_i| ) |d|) \Big)
\end{align*}

We do need to plug in $B_1$ for the $b$ terms to get

\begin{align*}
P^+(0; \beta_i^+, \lambda)&[ P_0^c(0) e^{-\nu(\lambda) X_i} c_i^+ + (P_0^c(\lambda) - P_0^c(0)) e^{-\nu(\lambda) X_i} c_i^+] \\
&= P^+(0; \beta_i^+, 0) P_0^c(0) e^{-\nu(\lambda) X_i} c_i^- + \mathcal{O}((e^{-\alpha X_m} + |\lambda|)|\tilde{c}_i^-|) \\
&+ \mathcal{O}\Big( e^{-\tilde{\alpha} X_i} ( |c_i^-| + (|\lambda| + e^{-\tilde{\alpha}X_m})( |\tilde{c}_{i-1}^+| + |\tilde{c}_{i+1}^-|) + e^{-\tilde{\alpha} X_{i-1}} |c_{i-1}^-| + e^{-\tilde{\alpha} X_{i+1}} |c_{i+1}^-|  \\
&+ (|\lambda| + |D_i| ) |d|) \Big)
\end{align*}

In this case, the leading order terms are preserved, which gives us the term

\[
e^{-\nu(\lambda) X_i} c_i^- - e^{\nu(\lambda) X_{i-1}} c_{i-1}^-
\]

\item The center integral term will give us the center Melnikov integral

\begin{align*}
&\langle \Psi^c(0), P^-(0; \beta_i^-, \lambda) \int_{-X_{i-1}}^0 \Phi^c(0, y; \lambda) P^-(y; \beta_i^-, \lambda) \tilde{H}_i^-(y) dy \rangle \\
&= \langle \Psi^c(0), \int_{-X_{i-1}}^0 P^-(0; \beta_i^-, \lambda) \Phi^c(0, y; \lambda) P^-(y; \beta_i^-, \lambda)^{-1} \tilde{H}_i^-(y) dy \rangle \\
&= \langle \Psi^c(0), \int_{-X_{i-1}}^0 P^-(0; 0, 0) \Phi^c(0, y; 0) P^-(y; 0, \lambda)^{-1} \tilde{H}_i^-(y) dy \rangle + \mathcal{O}(|\lambda|) \\
&= \int_{-X_{i-1}}^0 \langle \Psi^c(0), \Theta^c(0, y) \tilde{H}_i^-(y) \rangle dy + \mathcal{O}(|\lambda|) \\
&= \int_{-X_{i-1}}^0 \langle \Theta^c(y, 0)^* \Psi^c(0), H(y) \rangle dy + \int_{-X_{i-1}}^0 \langle \Psi^c(0), \Theta^c(0, y) \Delta H_i^-(y) \rangle dy + \mathcal{O}(|\lambda|) \\
&= \int_{-\infty}^0 \langle \Psi^c(y), H(y) \rangle dy + \int_{-X_{i-1}}^0 \langle \Psi^c(0), \Theta^c(0, y) \Delta H_i^-(y) \rangle dy + \mathcal{O}(e^{-\alpha X_m} + |\lambda|) \\
\end{align*}

For the integral involving $\Delta H_i^-(y)$, we have

\begin{align*}
\left| \int_{-X_{i-1}}^0 \langle \Psi^c(0), \Theta^c(0, y) \Delta H_i^-(y) \rangle dy \right| &\leq C \int_{-X_{i-1}}^0 e^{-\rho y} e^{-\alpha X_{i-1}} e^{-\alpha(X_{i-1} + y)} dy \\
&\leq C e^{-(\alpha - \rho)X_{i-1}} \int_{-X_{i-1}}^0 e^{-\alpha(X_{i-1} + y)} dy \\
&\leq C e^{-\tilde{\alpha}X_{i-1}}
\end{align*}

Thus we have

\begin{align*}
&\langle \Psi^c(0), P^-(0; \beta_i^\pm, \lambda) \int_{-X_{i-1}}^0 \Phi^c(0, y; \lambda) P^-(y; \beta_i^\pm, \lambda) \tilde{H}_i^-(y) dy \rangle \\
&= \int_{-\infty}^0 \langle \Psi^c(y), H(y) \rangle dy + \mathcal{O}(e^{-\tilde{\alpha} X_m} + |\lambda|) \\
\end{align*}

The ``positive'' integral is similar.

\end{enumerate}

The remaining terms are higher order.

\begin{enumerate}

\item For the term involving $a$,

\begin{align*}
P^-(0; \beta_i^-, \lambda) &\Phi^s(0, -X_{i-1}; \lambda) a_{i-1}^- \\
&= -P^-(0; \beta_i^-, \lambda) \Phi^s(0, -X_{i-1}; \lambda) P_0^s(\lambda) D_i d + P^-(0; \beta_i^-, \lambda) \Phi^s(0, -X_{i-1}; \lambda) A_4(\lambda)_i^-(b, c^-, d) 
\end{align*}

Thus we have

\begin{align*}
P^-(0; \beta_i^-, \lambda) &\Phi^s(0, -X_{i-1}; \lambda) a_{i-1}^- 
= P^-(0; 0, 0) \Phi^s(0, -X_{i-1}; 0)a_{i-1}^- \\
&+ \mathcal{O}(e^{-\alpha X_{i-1}}( e^{-\alpha X_m} + |\lambda|)( |D||d| +|A_2(\lambda)_{i-1}^-(b, c^-, d)|
\end{align*}

The first term on the RHS is eliminated outright by the projection. The other is order

\begin{align*}
&\mathcal{O}(e^{-\alpha X_{i-1}}( e^{-\alpha X_m} + |\lambda|)( |D||d| +  
e^{-\alpha X_{i-1}} (|\lambda| + e^{-\tilde{\alpha}X_m})(|\tilde{c}_{i-2}^+| + |\tilde{c}_{i}^-|) + e^{-\tilde{\alpha}X_{i-2}}|c_{i-2}^-| + e^{-\tilde{\alpha}X_{i-1}}|c_{i-1}^-| + e^{-\tilde{\alpha}X_{i}}|c_{i}^-| \\
&+ e^{-\tilde{\alpha} X_m} |\lambda|^2|d| + e^{-\alpha X_m}|D||d| \Big) \\
&\mathcal{O}(e^{-\alpha X_m}( e^{-\alpha X_m} + |\lambda|)(  
e^{-\alpha X_m} (|\lambda| + e^{-\tilde{\alpha}X_m})(|\tilde{c}_{i-2}^+| + |\tilde{c}_{i}^-|) + e^{-\tilde{\alpha}X_{i-2}}|c_{i-2}^-| + e^{-\tilde{\alpha}X_{i-1}}|c_{i-1}^-| + e^{-\tilde{\alpha}X_{i}}|c_{i}^-| \\
&+ e^{-\tilde{\alpha} X_m} |\lambda|^2|d| + |D||d| \Big)
\end{align*}

For the ``plus'' term, we have a remainder term of

\begin{align*}
&\mathcal{O}(e^{-\alpha X_m}( e^{-\alpha X_m} + |\lambda|)(  
e^{-\alpha X_m} (|\lambda| + e^{-\tilde{\alpha}X_m})(|\tilde{c}_{i-1}^+| + |\tilde{c}_{i+1}^-|) + e^{-\tilde{\alpha}X_{i-1}}|c_{i-1}^-| + e^{-\tilde{\alpha}X_{i}}|c_{i}^-| + e^{-\tilde{\alpha}X_{i+1}}|c_{i+1}^-| \\
&+ e^{-\tilde{\alpha} X_m} |\lambda|^2|d| + |D||d| \Big) 
\end{align*}

\item For the terms involving $b$, we have

\begin{align*}
P^-(0; \beta_i^-, \lambda)&( b_i^- + (P_0^u(\lambda) - P_0^u(0))b_i^-) \\
&= P^-(0; 0, 0) b_i^- + \mathcal{O}(|\lambda| + e^{-\alpha X_m})\Big( 
(|\lambda| + e^{-\tilde{\alpha}X_m})( |\tilde{c}_{i-1}^+| + |\tilde{c}_i^-|)
+ e^{-\tilde{\alpha} X_{i-1}} c_{i-1}^- + e^{-\tilde{\alpha} X_i} c_i^- + \\
&( e^{-\tilde{\alpha}X_m} |D| + e^{-\tilde{\alpha}X_m}|\lambda| + |\lambda|^2)|d| \Big)
\end{align*}

The first term on the RHS is eliminated outright by the projection.

\item For the noncenter integral terms involving $\tilde{H}$, since this mostly evolves in the stable subspace, we should have

\begin{align*}
&P^-(0; \beta_i^-, \lambda) 
\int_{-X_{i-1}}^0 \Phi^s(0, y; \lambda) \lambda^2 d_i P^-(y; \beta_i^-, \lambda)^{-1} \tilde{H}_i^-(y) dy \\
&= P^-(0; 0, 0) 
\int_{-X_{i-1}}^0 \Phi^s(0, y; 0) \lambda^2 d_i P^-(0; 0, 0)^{-1} H_i^-(y) dy + \mathcal{O}((|\lambda| + e^{-\alpha X_m})|\lambda|^2|d|) 
\end{align*}
 
Of course, we should check this. The first term on the RHS is eliminated by the projection.

\item For the integral terms involving $Z$, these will be bounded by the center one, so we only have to do that one.

\begin{align*}
&\left| \int_{-X_{i-1}}^0 \Phi^c(0, y; \lambda) G_i^-(y)Z_i^-(y) \right| dy \\ 
&\leq C \int_{-X_{i-1}}^0 e^{-\rho y} e^{-\alpha X_{i-1}}e^{-\alpha({X_{i-1} + y)}}||Z_3(\lambda)(b,c^-,d)_i^-|| dy \\
&\leq C e^{-(\alpha - \rho)X_{i-1}} ( (|\lambda| + e^{-\tilde{\alpha}X_m})(|\tilde{c}_i^-| + |\tilde{c}_{i-2}^+|) + e^{-\tilde{\alpha} X_{i-2}} |c_{i-2}^-| + e^{-\tilde{\alpha} X_i} |c_i^-| + e^{\rho X_{i-1}}|c_{i-1}^-| \\ 
&+ |\lambda|^2 |d| + e^{-(\alpha - \rho)X_{i-1}}|\lambda||d| + |D_{i-1}||d| ) \\
&\leq C e^{-\tilde{\alpha} X_{i-1}} ( (|\lambda| + e^{-\tilde{\alpha}X_m})(|\tilde{c}_i^-| + |\tilde{c}_{i-2}^+|) + e^{-\tilde{\alpha} X_{i-2}} |c_{i-2}^-| + e^{-\tilde{\alpha} X_i} |c_i^-| + |c_{i-1}^-| \\ 
&+ |\lambda|^2 |d| + e^{-\tilde{\alpha}X_m}|\lambda||d| + |D_{i-1}||d| )
\end{align*}

Similarly,

\begin{align*}
&\left| \int_{-X_i}^0 \Phi^c(0, y; \lambda) G_i^+(y)Z_i^+(y) \right| dy \\ 
&\leq C e^{-\tilde{\alpha} X_i} ( (|\lambda| + e^{-\tilde{\alpha}X_m})(|\tilde{c}_{i+1}^-| + |\tilde{c}_{i-1}^+|) + e^{-\tilde{\alpha} X_{i-1}} |c_{i-1}^-| + e^{-\tilde{\alpha} X_{i+1}} |c_{i+1}^-| + |c_i^-| \\
&+ |\lambda|^2 |d| + e^{-(\alpha - \rho)X_i}|\lambda||d| + |D_i||d| )
\end{align*}

\end{enumerate}

Believe it or not, this is all of them. Putting all of this together, we have

\begin{align*}
\xi^c_i = e^{-\nu(\lambda) X_i} c_i^- - e^{\nu(\lambda) X_{i-1}} c_{i-1}^- - \lambda_2 d_i M^c + R^c(\lambda)(c, \tilde{c}, d)
\end{align*}

where $M^c$ is the center Melnikov integral

\[
\int_{-\infty}^\infty \langle \Psi^c(y), H(y) \rangle dy 
\]

and the remainder term $R^c(c, \tilde{c}, d)$ has bound

\begin{align*}
R^c&(c, \tilde{c}, d)_i \leq C \Big( \\
&(|\lambda + e^{-\alpha X_m})(|\tilde{c}_{i-1}^+| + |\tilde{c}_{i}^-| + e^{-\alpha X_m}( |\tilde{c}_{i-2}^+| + |\tilde{c}_{i+1}^-|) ) \\
&+ e^{-\alpha X_i} |c_i^-| + e^{-\alpha X_m}( e^{-\alpha X_{i-1}} |c_{i-1}^-| + e^{-\alpha X_{i-2}} |c_{i-2}^-| + e^{-\alpha X_{i+1}} |c_{i+1}^-| \\
&+ e^{-\tilde{\alpha} X_m} (|\lambda| + |D|)|d|
\Big)
\end{align*}

This is horrible. But at least we know exactly which of the $c_i$ are involved where. We will write out only the matrices to solve for the $c_i^-$, since those are the ones we are worried about.\\

We (more or less) have the matrix equation

\[
(C_1 K(\lambda)+ C_2) c_i^- = 0
\]

where

\begin{align*}
K(\lambda) =  
\begin{pmatrix}
e^{-\nu(\lambda)X_1} & & & & & -e^{\nu(\lambda)X_0} \\
-e^{\nu(\lambda)X_1} & e^{-\nu(\lambda)X_2} \\
& -e^{\nu(\lambda)X_2} & e^{-\nu(\lambda)X_3} \\
\vdots & & \vdots & &&  \vdots \\
& & & & -e^{\nu(\lambda)X_{n-1}} & e^{-\nu(\lambda)X_0} 
\end{pmatrix}
\end{align*}

and

\begin{align*}
C_1 &= I + \mathcal{O}(|\lambda| + e^{-\alpha X_m}) I 
+ \mathcal{O}(e^{-\alpha X_m}( |\lambda| + e^{-\alpha X_m}))\\
C_2 &= \mathcal{O}(e^{-\alpha X_m}) I + \mathcal{O}(e^{-2 \alpha X_m})
\end{align*}

IGNORE REST OF THIS

\subsubsection*{Jump in Psi direction}

For this jump, we project on $\Psi_i(0)$.

\[
\xi_i = \langle \Psi_i(0), P_i^+(0; \lambda) Z_i^+(0) - P_i^-(0; \lambda) Z_i^-(0) \rangle
\]

Recall that $Z_i^\pm(0)$ are given by

\begin{align*}
Z_i^-(0) &= \Phi^s(0, -X_{i-1}; \lambda) a_{i-1}^- + b_i^- + (P_0^u(\lambda) - P_0^u(0))b_i^- \\
&+ e^{\nu(\lambda) X_{i-1}} P_0^c(0) c_{i-1}^- + e^{\nu(\lambda) X_{i-1}} (P_0^c(\lambda) - P_0^c(0))c_{i-1}^- \\
&+ \lambda^2 d_i \int_{-X_{i-1}}^0 \Phi^s(0, y; \lambda) P_i^-(y; \lambda)^{-1} \tilde{H}_i^-(y) dy 
+ \lambda^2 d_i \int_{-X_{i-1}}^0 \Phi^c(0, y; \lambda) P_i^-(y; \lambda)^{-1} \tilde{H}_i^-(y) dy  \\ 
Z_i^+(0) &= \Phi^u(0, X_i; \lambda) a_i^+ + b_i^+ + (P_0^s(\lambda) - P_0^s(0)) b_i^+ \\
&+ e^{-\nu(\lambda)X_i} P_0^c(0) c_i^+ + e^{-\nu(\lambda)X_i} (P_0^c(\lambda) - P_0^c(0))c_i^+ \\
&+ \lambda^2 d_i \int_{X_i}^0 \Phi^u(0, y; \lambda) P_i^+(y; \lambda)^{-1} \tilde{H}_i^+(y) dy 
+ \lambda^2 d_i \int_{X_i}^0 \Phi^c(0, y; \lambda) P_i^+(y; \lambda)^{-1} \tilde{H}_i^+(y) dy \\
\end{align*}

We will compute the significant terms first. The noncenter integral will give us the Melnikov integral. For the ``minus'' piece, we have

\begin{align*}
&\langle \Psi_i(0), P_i^-(0; \lambda) \int_{-X_{i-1}}^0 \Phi^s(0, y; \lambda) P_i^-(y; \lambda)^{-1} \tilde{H}_i^-(y) dy \rangle \\
&= \int_{-X_{i-1}}^0 \langle \Psi_i(0), P_i^-(0; \lambda), \Phi^s(0, y; \lambda) P_i^-(y; \lambda)^{-1} \tilde{H}(y) \rangle dy \\
&= \int_{-X_{i-1}}^0 \langle \Psi_i(0), P_i^-(0; \lambda), \Phi^s(0, y; \lambda) P_i^-(y; \lambda)^{-1} H(y) \rangle dy + \mathcal{O}({e^{-\alpha X_m}})\\
&= \int_{-X_{i-1}}^0 \langle \Psi_i(0), \Theta_i^s(0, y; \lambda) H(y) \rangle dy + \mathcal{O}({e^{-\alpha X_m}})\\
&= \int_{-X_{i-1}}^0 \langle \Psi_i(0), \Theta_i^s(0, y; 0) H(y) \rangle dy + \mathcal{O}(|\lambda| + {e^{-\alpha X_m}})\\
&= \int_{-X_{i-1}}^0 \langle \Theta_i^s(y, 0; 0)^* \Psi_i(0), H(y) \rangle dy + \mathcal{O}(|\lambda| + {e^{-\alpha X_m}})\\
&= \int_{-X_{i-1}}^0 \langle \Psi_i(y), H(y) \rangle dy + \mathcal{O}(|\lambda| + {e^{-\alpha X_m}})\\
&= \int_{-X_{i-1}}^0 \langle \Psi(y), H(y) \rangle dy + \mathcal{O}(|\lambda| + {e^{-\alpha X_m}})\\
\end{align*}

Note that in the last step, we put this in terms of $\Psi(x)$, the solution to the adjoint variational problem for the primary pulse.\\

Next, we look at the terms involving $a$. For these, we plug in $A_4$.

\begin{align*}
\langle &\Psi_i(0), P_i^-(0; \lambda) \Phi^s(0, -X_{i-1}; \lambda) a_{i-1}^- \rangle \\
&= \langle \Psi_i(0), P_i^-(0; \lambda) \Phi^s(0, -X_{i-1}; \lambda) (- P_i^-(-X_{i-1}; \lambda)^{-1} P_0^s(\lambda) D_{i-1} d + A_4(\lambda)_i^-(\tilde{c}, d)) \rangle \\
&= -\langle \Psi_i(0), \Theta_i^s(0, -X_{i-1}; \lambda) P_0^s(\lambda) D_{i-1} d \rangle + \mathcal{O}( e^{-(\alpha + \tilde{\alpha})X_m}(|\tilde{c}| + |\lambda|^2 |d| + |D||d|) \\
&= -\langle \Psi_i(0), \Theta_i^s(0, -X_{i-1}; 0) P_0^s(0) D_{i-1} d \rangle + \mathcal{O}( e^{-(\alpha + \tilde{\alpha})X_m}(|\tilde{c}| + |\lambda||d| + |D||d|) \\
&= -\langle \Theta_i^s(-X_{i-1}, 0; 0)^* \Psi_i(0), P_0^s(0) D_{i-1} d \rangle + \mathcal{O}( e^{-(\alpha + \tilde{\alpha})X_m}(|\tilde{c}| + |\lambda||d| + |D||d|) \\
&= -\langle \Psi_i(-X_{i-1}), P_0^s(0) D_{i-1} d \rangle + \mathcal{O}( e^{-(\alpha + \tilde{\alpha})X_m}(|\tilde{c}| + (|\lambda| + |D|)|d|) \\
\end{align*}

The remaining terms will be higher order. Doing these in turn, we have

\begin{enumerate}
\item For the terms involving $b$,

\begin{align*}
\langle \Psi(0), P_i^-(0; \lambda) \Phi^u(0, 0; \lambda) b_i^- \rangle
&= \langle \Psi(0), (P_i^-(0; 0) + \mathcal{O}(|\lambda|))(I + \mathcal{O}(|\lambda|)) b_i^- \rangle \\
&= \langle \Psi(0), P_i^-(0; 0) b_i^- \rangle + \mathcal{O}(|\lambda||b_i^-|) \\
&= |\lambda| \Big(
(|\lambda| + e^{-\tilde{\alpha}X_m})|\tilde{c}|
+ (e^{-\tilde{\alpha}X_m}|D| 
+ |\lambda|^2)|d|
\Big)
\end{align*}

where we substituted in $B_1$.

\item For the terms involving $\tilde{c}$, we have

\begin{align*}
\langle \Psi(0), P_i^-(0; \lambda)e^{\nu(\lambda) X_{i-1}} P^c(\lambda) c_{i-1}^-\rangle &= 
\langle \Psi(0), (P_i^-(0; 0) + \mathcal{O}(\lambda))(I + \mathcal{O}(\lambda)) \tilde{c}_{i-1}^+ \rangle \\
&= \mathcal{O}(|\lambda|) \tilde{c}_{i-1}^+
\end{align*}

We also have to do the same substitution as above to convert $c_i^+$ to $c_i^+$, as we did above. As above, we have

\begin{align*}
e^{-\nu(\lambda)X_i} c_i^+ &= e^{-\nu(\lambda)X_i} c_i^- + \mathcal{O}\Big( e^{-\tilde{\alpha} X_m} |\tilde{c}| + e^{-\tilde{\alpha} X_m}(|\lambda| + |D| ) |d| \Big) \\
\end{align*}

which gives us

\begin{align*}
\langle \Psi(0), P_i^+(0; \lambda)e^{-\nu(\lambda) X_i} P^c(\lambda) c_i^+ \rangle 
&= \mathcal{O}(|\lambda|) \tilde{c}_i^- + \mathcal{O}\Big( |\lambda|( e^{-\tilde{\alpha} X_m} |\tilde{c}| + e^{-\tilde{\alpha} X_m}(|\lambda| + |D| ) |d| ) \Big)
\end{align*}

\item For the center integral term, we have

\begin{align*}
&\langle \Psi(0), P_i^-(0; \lambda)
\int_{-X_{i-1}}^0 \Phi^c(0, y; \lambda) P_i^-(y; \lambda)^{-1} \tilde{H}_i^-(y) dy \rangle \\
&= \int_{-X_{i-1}}^0 \langle \Psi(0), P_i^-(0; \lambda) \Phi^c(0, y; \lambda) P_i^-(y; \lambda)^{-1} \tilde{H}_i^-(y) \rangle dy \\
&= \int_{-X_{i-1}}^0 \langle \Psi(0), \Theta^c(0, y; \lambda) \tilde{H}_i^-(y) \rangle dy \\
&= \int_{-X_{i-1}}^0 \langle \Psi(0), \Theta^c(0, y; 0) \tilde{H}_i^-(y) \rangle dy + \mathcal{O}(|\lambda|) \\
&= \mathcal{O}(|\lambda|)
\end{align*}

\end{enumerate}

Putting this all together, we have

\begin{align*}
\xi_i = \langle \Psi(X_i), P_0^u(0) D_i d \rangle
+ \langle \Psi(-X_{i-1}), P_0^s(0) D_{i-1} d \rangle 
+ \lambda_2 d_i M + R(\lambda)(\tilde{c}, d)
\end{align*}

where $M$ is the higher order Melnikov integral

\[
\int_{-\infty}^\infty \langle \Psi(y), H(y) \rangle dy 
\]

and the remainder term has bound

\begin{align*}
|R(\lambda)(\tilde{c}, d)| \leq C \Big(
|\lambda||\tilde{c}| + |\lambda|(|\lambda|^2 + e^{-\alpha X_m}|\lambda| + e^{-(\alpha + \tilde{\alpha}) X_m})|d| + (e^{-\alpha X_m} |\lambda| + e^{-(\alpha + \tilde{\alpha}) X_m})|D||d|
 \Big)
\end{align*}

Using the fact that $|D| = \mathcal{O}(e^{-\alpha X_m})$, this becomes

\begin{align*}
|R(\lambda)(\tilde{c}, d)| \leq C \Big(
|\lambda||\tilde{c}| + (|\lambda|^3 + e^{-\alpha X_m}|\lambda|^2 + e^{-(\alpha + \tilde{\alpha}) X_m}|\lambda| + e^{-(2 \alpha + \tilde{\alpha}) X_m})|d|
 \Big)
\end{align*}



\end{document}