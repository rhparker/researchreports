\documentclass[12pt]{article}
\usepackage[pdfborder={0 0 0.5 [3 2]}]{hyperref}%
\usepackage[left=1in,right=1in,top=1in,bottom=1in]{geometry}%
\usepackage[shortalphabetic]{amsrefs}%
\usepackage{amsmath}
\usepackage{enumerate}
% \usepackage{enumitem}
\usepackage{amssymb}                
\usepackage{amsmath}                
\usepackage{amsfonts}
\usepackage{amsthm}
\usepackage{bbm}
\usepackage[table,xcdraw]{xcolor}
\usepackage{tikz}
\usepackage{float}
\usepackage{booktabs}
\usepackage{svg}
\usepackage{mathtools}
\usepackage{cool}
\usepackage{url}
\usepackage{graphicx,epsfig}
\usepackage{makecell}
\usepackage{array}

\def\noi{\noindent}
\def\T{{\mathbb T}}
\def\R{{\mathbb R}}
\def\N{{\mathbb N}}
\def\C{{\mathbb C}}
\def\Z{{\mathbb Z}}
\def\P{{\mathbb P}}
\def\E{{\mathbb E}}
\def\Q{\mathbb{Q}}
\def\ind{{\mathbb I}}

\DeclareMathOperator{\spn}{span}
\DeclareMathOperator{\ran}{ran}

\graphicspath{ {periodic/} }

\newtheorem{lemma}{Lemma}
\newtheorem{theorem}{Theorem}
\newtheorem{corollary}{Corollary}
\newtheorem{definition}{Definition}
\newtheorem{assumption}{Assumption}
\newtheorem{hypothesis}{Hypothesis}

\newtheorem{notation}{Notation}

\begin{document}

\section{Stability of Periodic Multi-Pulse Solutions}

\subsection{Background and Eigenvalue Problem}

In a previous section, we showed that a periodic $n-$pulse solution $q_{np}(x)$ to KdV5 exists, where the distances between the peaks are given by the $n$ lengths $X_0, \dots, X_{n-1}$. We can write $q_{np}(x)$ piecewise as

\[
q_i^\pm(x) = q^\pm(x; \beta_i^\pm) + u_i^\pm(x)
\]

on the $2n$ intervals 

\begin{align*}
\{ [-X_{i-1}, 0], [0, X_i] \} && i = 0, \dots, n-1
\end{align*}

where the subscript $i$ is $\mod n$, since we are in a periodic domain. The distances between consecutive peaks in $q_{np}(x)$ are $2 X_0, \dots, 2 X_{n-1}$.\\

The functions $q^\pm(x; \beta_i^\pm)$ evolve in the $W^s(0)$ and $W^u(0)$, with initial conditions $\beta_i^\pm$. The functions $u_i^\pm(x)$ are small remainder terms. From the existence problem, we have bounds on all of these terms.

\begin{align*}
|q^\pm(x; \beta_i^\pm)| &\leq C |\beta_i^\pm| e^{-\alpha |x|} \\
|\beta_i^\pm| &\leq C (e^{-2 \alpha X_i} + e^{-2 \alpha X_{i-1}}) \\
|u_i^-(x)| &\leq C e^{-\alpha X_{i-1}} e^{-\alpha(X_{i-1} + x) } \\
|u_i^+(x)| &\leq C e^{-\alpha X_i} e^{-\alpha(X_i - x) } 
\end{align*}

Our goal is to determine the linear stability of periodic multi-pulse solutions to KdV5. To do this, we look at the PDE eigenvalue problem resulting from the linearization of KdV5 about an the periodic multi-pulse $q_{np}(x)$.\\

Recall that the linearization of KdV5 about an equilibrium solution $q(x)$ is given by the linear operator $L(q)$.

\begin{equation}
L(q) = \partial_x^5 - \partial_x^3 + c \partial_x
 - 2 q \partial_x - 2 q_x
\end{equation}

We can write this operator as $L = \partial_x H(q)$, where $H(q)$ is the self-adjoint operator

\begin{equation}
H(q) = \partial_x^4 - \partial_x^2 + c - 2 q
\end{equation}

Thus for the adjoint operator $L(q)^*$, we have

\begin{equation}
L(q)^* = -H(q) \partial_x
\end{equation}

It is not hard to show that

\begin{align*}
L(q)q_x &= 0 \\
L(q)(-\partial_c q) &= q_x \\
L(q)^* q &= 0
\end{align*}

In addition, for a finite or periodic domain, the constant functions are in the kernel of $L(q)^*$.

\[
L(q) 1 = 0
\]

For the primary pulse solution $q(x)$ to KdV5, we will assume that that $L(q)$ has a one-dimensional kernel which is spanned by $q_x$. If there were another generalized eigenfunction in the kernel of $L(q)$, we would be able to solve the equation $L(q) v = \partial_c q$, which is only possible if $\partial_c q \perp \ker L(q)^*$. In particular, this would require $\partial_c q \perp q$. We will assume this is not the case, i.e. we take the following hypothesis.

\begin{hypothesis}\label{Melnikovnonzero}
The following higher order Melnikov integral is nonzero.
\[
M = \langle q, \partial_c q \rangle_{L^2(\R)} 
= \int_{-\infty}^\infty q(x) \partial_c q(x) dx \neq 0
\]
\end{hypothesis}

Returning to the linearization about the periodic multipulse $q_{np}(x)$, let $V = (v, v_x, v_{xx}, v_{xxx}, v_{xxxx})^T$, and write the eigenvalue problem $L(q_{np}(x))v(x) = \lambda v(x)$ as the first-order system

\begin{align*}
V'(x) = A(q_{np}(x)) V(x) + \lambda B V(x)
\end{align*}

where

\begin{align*}
A(q(x)) &= \begin{pmatrix}0 & 1 & 0 & 0 & 0 \\0 & 0 & 1 & 0 & 0 \\0 & 0 & 0 & 1 & 0 \\0 & 0 & 0 & 0 & 1 \\
2 \partial_x q(x) + \lambda & 2 q(x) - c & 0 & 1 & 0 \end{pmatrix}, &&
B = \begin{pmatrix}0 & 0 & 0 & 0 & 0 \\0 & 0 & 0 & 0 & 0 \\0  & 0 & 0 & 0 & 0 \\0 & 0 & 0 & 0 & 0 \\1 & 0 & 0 & 0 & 0 \end{pmatrix} 
\end{align*}

\subsection{Variational Equation}

Consider the variational and adjoint variational equations for the linearization about the primary pulse solution $q(x)$ to KdV5. These are given by

\begin{align}
V' = A(q(x))V \label{vareq} \\
W' = -A(q(x))^*W \label{adjvareq}
\end{align}

Note that $A(q(x))$ is exponentially asymptotic to the constant-coefficient matrix $A(0)$, which is given by

\begin{align}\label{A0}
A(0) = \begin{pmatrix}0 & 1 & 0 & 0 & 0 \\ 0 & 0 & 1 & 0 & 0 \\ 0 & 0 & 0 & 1 & 0 \\ 0 & 0 & 0 & 0 & 1 \\
0 & -c & 0 & 1 & 0 
\end{pmatrix}
\end{align}

For $c > 1/4$, $A(0)$ has eigenvalues $\nu = \{ 0, \pm \alpha_0 \pm \beta_0 i\}$, where $\alpha_0, \beta_0 > 0$. The eigenvectors of $A(0)$ and $-A(0)^*$ corresponding to the eigenvalue 0 are $V_0$ and $W_0$ (respectively), which are given by

\begin{align*}
V_0 &= (1, 0, 0, 0, 0)^T \\
W_0 &= (c, 0, -1, 0, 1)^T 
\end{align*}

Since $A(0)$ is not hyperbolic, we cannot directly apply the results of San98. Thus the equilibrium at 0 has 2-dimensional stable/unstable manifolds, and a 1-dimensional center manifold. Let $W^{s/u/c}(0)$ be these manifolds.\\

Since $L(q)q_x = 0$ and $L(q)(-\partial_c q) q_x$, we have the expressions

\begin{align*}
(Q')' &= A(q) Q' \\
(\partial_c Q)' &= A(q) (\partial_c Q) + B Q'
\end{align*}

For the adjoint variational problem, we have an exponentially decaying solution $\Psi(x)$, which is given by

\begin{equation}\label{Psi}
\Psi(x) = \begin{pmatrix}
q^{(4)}(x) - q''(x) + (-2q(x) + c)q(x)\\
-q^{(3)}(x) + q'(x) \\
q''(x) - q(x) \\
-q'(x) \\
q(x)
\end{pmatrix}
\end{equation}

For a bounded or periodic domain, we also have a solution $\Psi^c(x)$ to the adjoint variational problem which is bounded but does not decay exponentially.

\begin{align*}
\Psi^c(x) = (c - 2 q(x), 0, -1, 0, 1)^T
\end{align*}

In fact, $\Psi^c(x) \rightarrow W_0$ as $x \rightarrow \pm \infty$.

We will now decompose the tangent space at $Q(0)$. First, we make the following nondegeneracy assumption.

\begin{hypothesis}\label{nondegen}
\[
T_{Q(0)} W^u(0) \cap T_{Q(0)} W_s(0) = \R Q'(0)
\]
\end{hypothesis}

where $Q(x)$ is the single pulse solution on $\R$, written as a vector-valued function in $R^5$. Next, we define $Y^-$ and $Y^+$ to be the remaining dimensions of the tangent space of the unstable/stable manifolds.

\begin{align*}
T_{Q(0)} W^u(0) &= \R Q'(0) \oplus Y^- \\
T_{Q(0)} W^s(0) &= \R Q'(0) \oplus Y^+
\end{align*}

So far $\text{dim }\R Q'(0) \oplus Y^- \oplus Y^+ = 3$. To fill out the remaining 2 dimensions, we look at solutions to the adjoint variational equation.\\

First, we summarize some useful facts about the variational equation in the following lemma. We define the inner product on $\C^n$ by $\langle x, y \rangle = \sum_i x_i \bar{y_i}$, i.e. the complex conjugation is on the second component.

% lemma : facts about our eigenvalue problem

\begin{lemma}\label{eigadjoint}
Consider the linear ODE $V' = A(x)V$ and the corresponding adjoint problem $W' = -A(x)^* W$, where $A$ is an $n \times n$ matrix depending on $x$. Then the following are true.
\begin{enumerate}[(i)]
\item $\dfrac{d}{dx}\langle V(x), W(x) \rangle = 0$, thus the inner product is constant in $x$.
\item If $\Phi(y, x)$ is the evolution operator for $V' = A(x)V$, then $\Phi(x, y)^*$ is the evolution operator for the adjoint problem $W' = -W(x)^* W$.
\end{enumerate}
\begin{proof}
For (i), take the derivative of the inner product and use the expressions for $V'$ and $W'$. For (ii), take the derivative of the expression $\Phi(y, x)\Phi(x, y) = I$.
\end{proof}
\end{lemma}

Since $\langle \Psi(x), Q'(x) \rangle$ is constant in $x$ and $Q'(x) \rightarrow 0$ as $x \rightarrow \infty$, by the continuity of the inner product, we must have $\langle \Psi(0), Q'(0) \rangle = 0$. Similarly, if taking a solution $V(x)$ to the variational equation \eqref{vareq} with initial condition in $Y^+$ or $Y^-$, we can show that $\Psi(0) \perp Y^+$ and $\Psi(0) \perp Y^-$. The same holds for $\Psi^c(0)$. Thus we have shown that

\[
\Psi(0), \Psi^c(0) \perp \R Q'(0) \oplus Y^- \oplus Y^-
\]

Let 

\begin{equation}
S = \text{span }\{ \Psi(0), \Psi^c(0) \}
\end{equation}

Since $\Psi(0), \Psi^c(0)$ are linearly independent (but not orthogonal), $S$ has dimension 2. Since $S \perp \R Q'(0) \oplus Y^- \oplus Y^-$, we can write the tangent space at $Q(0)$ as the direct sum

\begin{equation}
\R^5 = \R Q'(0) \oplus Y^- \oplus Y^+ \oplus S
\end{equation}

\subsubsection{Piecewise Formulation}

To exploit these relations, we take the following piecewise ansatz for the eigenfunction $V(x)$

\begin{equation}
V_i^\pm(x) = d_i (Q_{np}'(x) + \lambda (Q_{np})_c(x)) + W_i^\pm 
\end{equation}

where the $V_i^-$ equation is defined on $[-X_{i-1}, 0]$, the $V_i^+$ equation is defined on $[0, X_i]$, and the $d_i \in \C$ are arbitrary constants. Substituting this into the eigenvalue problem and simplifying, we obtain the system for $W_i^\pm$

\begin{align*}
&(W_i^\pm)' = A( q_i^\pm(x) ) W_i^\pm + \lambda B W_i^\pm + \lambda^2 d_i \tilde{H}_i^\pm \\
&W_i^-(0) = W_i^+(0) \\
&W_i^\pm(0) \in S \oplus Y^+ \oplus Y^- \\
&W_i^+(X_i) - W_{i+1}^-(-X_i) = D_i d
\end{align*}

where

\begin{align*}
D_i d &= d_{i+1}[Q_{i+1}'(-X_i) + \lambda \partial_c Q_{i+1}(-X_i)]
- d_i [ Q_i'(X_i) + \lambda \partial_c Q_i(-X_i) ] \\
\tilde{H}_i^\pm &= -B \partial_c Q_i^\pm
\end{align*}

The conditions at $x = \pm X_i$ and $x = 0$ are the requisite matching conditions that guarantee continuity of the eigenfunction $V(x)$.\\

For the final form of the eigenvalue problem, we will combine the matrices $A( q_i^\pm(x) )$ and $\lambda B$ to obtain the piecewise eigenvalue problem

\begin{align*}
&(W_i^\pm)' = A_i^\pm(x; \lambda) W_i^\pm + \lambda^2 d_i \tilde{H}_i^\pm \\
&W_i^-(0) = W_i^+(0) \\
&W_i^\pm(0) \in S \oplus Y^+ \oplus Y^- \\
&W_i^+(X_i) - W_{i+1}^-(-X_i) = D_i d
\end{align*}

where

\begin{align*}
A_i^\pm(x; \lambda) &= A( q_i^\pm(x) ) + \lambda B 
\end{align*}

The system we will investigate is 

\begin{align*}
&(W_i^\pm)' = A_i^\pm(x; \lambda) W_i^\pm + \lambda^2 d_i \tilde{H}_i^\pm \\
&W_i^\pm(0) \in S \oplus Y^+ \oplus Y^- \\
&W_i^+(0) - W_i^-(0) \in S \\
&W_i^+(X_i) - W_{i+1}^-(-X_i) = D_i d
\end{align*}

A solution to this system solves the eigenvalue problem if and only if the $n$ jumps at $x = 0$, which can only be in the subspace $S$, are 0. Since $S$ is spanned by $\Psi(0)$ and $\Psi^c(0)$, this is true if and only if 

\begin{align*}
\langle \Psi(0), W_i^+(0) - W_i^-(0) \rangle &= 0 \\
\langle \Psi^c(0), W_i^+(0) - W_i^-(0) \rangle &= 0
\end{align*}

From the existence problem and from San98 we have the following estimates.

\begin{align*}
|H(x)|, |\tilde{H}_i^\pm(x)| &\leq C e^{-\alpha |x|} \\
|\Delta H_i^\pm| &= |\tilde{H}_i^\pm - H| \leq C(e^{-\alpha X_i} + e^{-\alpha X_{i-1}} ) \\
|\Delta H_i^-(x)| &\leq C e^{-\alpha X_{i-1}} e^{-\alpha(X_{i-1} + x) } \\
|\Delta H_i^+(x)| &\leq C e^{-\alpha X_i} e^{-\alpha(X_i - x) } \\
D_i d &= ( Q'(X_i) + Q'(-X_i))(d_{i+1} - d_i ) + \mathcal{O} \left( e^{-\alpha X_i} \left( |\lambda| +  e^{-\alpha X_i}  \right) |d| \right) \\
\end{align*}

\subsection{Conjugation}

To simplify the system, we would like to apply a change of coordinates so that the linear operator $A_i^\pm(x; \lambda)$ is transformed into a constant coefficient matrix. To do that, we will use the Conjugation Lemma, which follows from the Gap Lemma. Both are stated below.\\

First, we state and prove the Gap Lemma, which is modified from Zum2018. 

\begin{lemma}[Gap Lemma]\label{gaplemma}
Let $W \in \C^N$, and consider the family of ODEs on $\R$

\begin{equation}\label{LambdaEVP}
W(x)' = A(x; \Lambda) W
\end{equation}

where $\Lambda \in \Omega$ is a parameter vector and $\Omega$ is a Banach space. Assume that

\begin{enumerate}
	\item The map $\Lambda \mapsto A(\cdot; \Lambda)$ is analytic in $\Lambda$.
	\item $A(x; \Lambda) \rightarrow A_\pm(\lambda)$ (independent of $\Lambda$) as $x \rightarrow \pm \infty$, and for $|\Lambda| < \delta$ we have the uniform exponential decay estimates 
	\begin{align}
	\left| \frac{\partial^k}{\partial x^k} A(x; \Lambda) - A_\pm(\Lambda) \right| 
	&\leq C e^{-\theta |x|} && 0 \leq k \leq K
	\end{align}
	where $\alpha > 0$, $C > 0$, and $K$ is a nonnegative integer.
\end{enumerate}

Suppose $V^-(\Lambda)$ is an eigenvector of $A_-(\Lambda)$ with corresponding eigenvalue $\mu(\Lambda)$, both analytic in $\Lambda$. Then there exists a solution of \ref{paramEVP} of the form 

\begin{equation}
W(x; \Lambda) = V(x; \Lambda) e^{\mu(\Lambda)x}
\end{equation}

where $V$ is $C^1$ in $x$ and analytic in $\Lambda$ for $|\Lambda| < \delta$, and for any fixed $\tilde{\theta} < \theta$

\begin{align}
V(x; \Lambda) = V^-(\Lambda) + \mathcal{O}(e^{-\tilde{\theta}|x|}|V^-(\Lambda)|) && x < 0
\end{align}

\begin{proof}
This is almost identical to Zum2018. The only difference here is that the parameter vector $\Lambda$ is in a general Banach space instead of a subset of $C^p$.\\

Let $W(x; \Lambda) = V(x; \Lambda) e^{\mu(\Lambda) x}$. Substituting this into \eqref{LambdaEVP} and simplifying, we obtain the equivalent ODE

\begin{equation}\label{VEVP}
V(x; \Lambda)' = (A_- - \mu(\Lambda)I)V(x; \Lambda) + \Theta(x; \Lambda) V(x; \Lambda)
\end{equation}

where $\Theta(x; \Lambda) = (A(x; \Lambda) - A_-(\Lambda)) = \mathcal{O}(e^{-\theta|x|})$. Choose any $\tilde{\theta} < \theta_1 < \theta$ such that the real part of the spectrum of $A_-$ lies either to the left or to the right of the vertical line $\text{Re}(\nu) = \text{Re}(\mu(\Lambda) + \theta_1$ in the complex plane. We should be able to make sure this is case for all $|\Lambda| < \delta$ since all the eigenvalues of $A_(\Lambda)$ are analytic in $\Lambda$.\\

Then for $|\Lambda| < \delta$, we can define the spectral projections $P(\Lambda)$ and $Q(\Lambda)$, where $P(\Lambda)$ projects onto the direct sum of all eigenspaces of $A_-(\Lambda)$ corresponding to eigenvalues $\nu$ with $\text{Re}(\nu) < \text{Re}(\mu(\Lambda) + \theta_1$, and $Q(\Lambda)$ projects onto the direct sum of all eigenspaces of $A_-(\Lambda)$ corresponding to eigenvalues $\nu$ with $\text{Re}(\nu) > \text{Re}(\mu(\Lambda) + \theta_1$. $P(\Lambda)$ and $Q(\Lambda)$ are analytic in $\Lambda$ for $|\Lambda| < \delta$, and from our definition of $\theta_1$ we have the estimates

\begin{align*}
\left|e^{(A_-(\Lambda) - \mu(\Lambda)I)x}P \right| &\leq C e^{\theta_1 x} && x \geq 0 \\
\left|e^{(A_-(\Lambda) - \mu(\Lambda)I)x}Q \right| &\leq C e^{\theta_1 x} && x \leq 0
\end{align*}

Note that $P(\Lambda) + Q(\Lambda) = I$. Define the map $T$ on $L^\infty(-\infty, -M]$ by

\begin{align*}
TV(x; \Lambda) &= V^-(\Lambda) 
+ \int_{-\infty}^x e^{(A_-(\Lambda) - \mu(\Lambda)I)(x-y)}P\Theta(y; \Lambda) V(y; \Lambda) dy \\
&- \int_x^{-M} e^{(A_-(\Lambda) - \mu(\Lambda)I)(x-y)}Q\Theta(y; \Lambda) V(y; \Lambda) dy
\end{align*}

Taking the absolute value of both sides, for $x \leq 0$

\begin{align*}
|TV(x; \Lambda)| &\leq |V^-(\Lambda)| + C ||V(x; \Lambda)||_{L^\infty(-\infty, -M]}
\left( \int_{-\infty}^x e^{\theta_1 (x - y)} e^{\theta y} dy + \int_x^{-M} e^{\theta_1 (x - y)} e^{\theta y} dy \right) \\
&\leq |V^-(\Lambda)| + C ||V(x; \Lambda)||_{L^\infty(-\infty, -M]} e^{\theta_1 x} \int_{-\infty}^M e^{(\theta - \theta_1) y} dy \\
&= \leq |V^-(\Lambda)| + C ||V(x; \Lambda)||_{L^\infty(-\infty, -M]} e^{\theta_1 x} \frac{e^{-(\theta - \theta_1)M}}{\theta - \theta_1}\\
&\leq |V^-(\Lambda)| + C ||V(x; \Lambda)||_{L^\infty(-\infty, -M]} e^{\theta_1 x} e^{-(\theta - \theta_1)M} \\
&\leq |V^-(\Lambda)| + C ||V(x; \Lambda)||_{L^\infty(-\infty, -M]} e^{-(\theta - \theta_1)M} \\
& < \infty
\end{align*}

Since the RHS is independent of $x$, we have $T: L^\infty(-\infty, -M] \rightarrow L^\infty(-\infty, -M]$. Next, we look at

\begin{align*}
|TV_1(x; \Lambda) - TV_2(x; \Lambda)| &\leq C ||V_1(x; \Lambda) - V_2(x; \Lambda)||_{L^\infty(-\infty, -M]} e^{\theta_1 x} \frac{e^{-(\theta - \theta_1)M}}{\theta - \theta_1}\\
\end{align*}

Since $e^{-(\theta - \theta_1)M} \rightarrow 0$ as $m \rightarrow \infty$, for sufficiently large $M$ we have 

\begin{align*}
|TV_1(x; \Lambda) - TV_2(x; \Lambda)|_{L^\infty(-\infty, -M]} &\leq \frac{1}{2} ||V_1(x; \Lambda) - V_2(x; \Lambda)||_{L^\infty(-\infty, -M]} 
\end{align*}

Thus the map $T$ is a contraction. Since $L^\infty(-\infty, -M]$ is a Banach space, by the Banach fixed point theorem, the map $T$ has a unique fixed point $V = TV$, i.e. we have a function $V \in L^\infty(-\infty, -M]$ such that 

\begin{align*}
V(x; \lambda) &= V^-(\Lambda) 
+ \int_{-\infty}^x e^{(A_-(\Lambda) - \mu(\Lambda)I)x}P\Theta(y; \Lambda) V(y; \Lambda) dy 
- \int_x^{-M} e^{(A_-(\Lambda) - \mu(\Lambda)I)x}Q\Theta(y; \Lambda) V(y; \Lambda) dy
\end{align*}

Differentiating this with respect to $x$, we obtain

\begin{align*}
V'(x; \Lambda) &= P\Theta(x; \Lambda) V(x; \Lambda) +
(A_-(\Lambda) - \mu(\Lambda)I) \int_{-\infty}^x e^{(A_-(\Lambda) - \mu(\Lambda)I)(x-y)}P\Theta(y; \Lambda) V(y; \Lambda) dy \\
&-(-Q\Theta(x; \Lambda) V(x; \Lambda))
-(A_-(\Lambda) - \mu(\Lambda)I) \int_x^{-M} e^{(A_-(\Lambda) - \mu(\Lambda)I)(x-y)}Q\Theta(y; \Lambda) V(y; \Lambda) dy \\
&= P\Theta(x; \Lambda) V(x; \Lambda) + Q\Theta(y; \Lambda) V(x; \Lambda) + (A_-(\Lambda) - \mu(\Lambda)I)(T V(x; \lambda) - V^-(\Lambda) ) \\
&= (P + Q)\Theta(x; \Lambda) V(x; \Lambda) + (A_-(\Lambda) - \mu(\Lambda)I)(V(x; \lambda) - V^-(\Lambda) ) \\
&= (A_-(\Lambda) - \mu(\Lambda)I)V(x; \lambda) + \Theta(x; \Lambda) V(x; \Lambda) - (A_-(\Lambda) - \mu(\Lambda)I)V^-(\Lambda) \\
&= (A_-(\Lambda) - \mu(\Lambda)I)V(x; \lambda) + \Theta(x; \Lambda) V(x; \Lambda)
\end{align*}

where we used the fact that $TV = V$ and $(A_-(\Lambda) - \mu(\Lambda)I)V^-(\Lambda) = 0$. Thus $V(x; \Lambda$ solves \eqref{VEVP}. Since $TV = V$, we let $V_1 = V$ and $V_2 = 0$ in the above to get the estimate

\begin{align*}
|V(x; \Lambda) - V^-(\Lambda)| &= |T(V(x; \Lambda)) - T(0)| \\
&\leq C ||V(x; \Lambda) - 0||_{L^\infty(-\infty, -M]} e^{\theta_1 x} \\
\end{align*}

Similarly, for suffiently large $M$, we have

\begin{align*}
|V(x; \Lambda)| - |V^-(\Lambda)| &\leq | |V(x; \Lambda)| - |V^-(\Lambda)| | \\
&\leq |V(x; \Lambda) - V^-(\Lambda)| \\
&= |T(V(x; \Lambda)) - T(0)| \\
&\leq \frac{1}{2} ||V(x; \Lambda)||_{L^\infty(-\infty, -M]}
\end{align*}

Thus

\begin{align*}
||V(x; \Lambda)||_{L^\infty(-\infty, -M]} \leq 2 |V^-(\Lambda)|
\end{align*}

Combining these, we have

\begin{align*}
|V(x; \Lambda) - V^-(\Lambda)| &\leq C e^{\tilde{\theta} x}|V^-(\Lambda)| \\
\end{align*}

from which we get

\begin{align*}
|V(x; \Lambda) = V^-(\Lambda) + \mathcal{O}( e^{\tilde{\theta} x}|V^-(\Lambda)| )\\
\end{align*}

At the moment, $V(x; \Lambda)$ is only defined for $x < -M$. We extend $V(x; \Lambda)$ to all of $R^-$ using the evolution operator for the system.

\end{proof}
\end{lemma}

As a corollary to this, we state and prove the Conjugation Lemma, which allows us to make a smooth change of coordinates to convert the linear ODE $Z'(x) = A^\pm(x) Z(x)$ into a constant coefficient system.

\begin{lemma}[Conjugation Lemma]
Let $W \in \C^N$, and consider the family of ODEs on $\R$

\begin{equation}\label{LambdaEVPconj}
W(x)' = A(x; \Lambda) W(x) + B(x) W(x) + F(x) 
\end{equation}

where $\Lambda \in \Omega$ is a parameter vector and $\Omega$ is a Banach space. Take the same assumptions as in the Gap Lemma, i.e. 

\begin{enumerate}
	\item The map $\Lambda \mapsto A(\cdot; \Lambda)$ is analytic in $\Lambda$.
	\item $A(x; \Lambda) \rightarrow A_\pm(\lambda)$ (independent of $\Lambda$) as $x \rightarrow \pm \infty$, and for $|\Lambda| < \delta$ we have the uniform exponential decay estimates 
	\begin{align}
	\left| \frac{\partial^k}{\partial x^k} A(x; \Lambda) - A_\pm(\Lambda) \right| 
	&\leq C e^{-\theta |x|} && 0 \leq k \leq K
	\end{align}
	where $\alpha > 0$, $C > 0$, and $K$ is a nonnegative integer.
\end{enumerate}

Then in a neighborhood of any $\Lambda_0 \in \Omega$ there exist invertible linear transformations

\begin{align*}
P_+(x, \Lambda) &= I + \Theta_+(x, \Lambda) \\
P_-(x, \Lambda) &= I + \Theta_-(x, \Lambda) 
\end{align*}

defined on $\R^+$ and $\R^-$, respectively, such that

\begin{enumerate}[(i)]
\item The change of coordinates $W = P_\pm Z$ reduces \eqref{LambdaEVPconj} to the equations on $\R^\pm$

\begin{align}
Z'(x) = A^\pm(\Lambda) Z(x) + P_\pm(x, \Lambda)^{-1} B(x) P_\pm(x, \Lambda) Z(x) + P_\pm(x, \Lambda)^{-1} F(x)
\end{align}

where

\[
G(x; \Lambda) = P_\pm(x, \Lambda)^{-1} F(x)
\]

\item For any fixed $0 < \tilde{\theta} < \theta$, $0 \leq k \leq K+1$, and $j \geq 0$ we have the decay rates
\begin{align*}
\left| \partial_\Lambda^j \partial_x^k \Theta_\pm \right| \leq C(j, k)e^{-\tilde{\theta}|x|}
\end{align*}
\end{enumerate}
\begin{proof}
We prove this for the case where $B(x) = 0$ and $F(x) = 0$. The form of the conjugated system easily follow for general $B$ and $F$.\\

We will do the case on $\R^-$. The other case is similar. 
Let $W = P_-(x, \Lambda) Z$, where we will figure out what $P_-(x, \Lambda)$ is later. Suppose that \eqref{conjZ} holds, and substitute these into \eqref{EVPconj}.

\begin{align*}
[P_-(x, \Lambda) Z(x)]' &= A(x; \Lambda)(P_-(x, \Lambda) Z(x)) \\
P_-'(x, \Lambda) Z(x) + P_-(x, \Lambda) Z'(x)
&= A(x; \Lambda)P_-(x, \Lambda) Z(x) \\
P_-'(x, \Lambda) Z(x) + P_-(x, \Lambda) A_- Z(x)
&= A(x; \Lambda)P_-(x, \Lambda) Z(x)
\end{align*}

Rearranging this, we obtain

\begin{equation}
P_-'(x, \Lambda) Z(x)
= [A(x; \Lambda)P_-(x, \Lambda) - P_-(x, \Lambda) A_-]Z(x)
\end{equation}

Suppose now that
\[
P_-'(x, \Lambda) = A(x; \Lambda)P_-(x, \Lambda) - P_-(x, \Lambda) A_-
\]

Then, upon making the substitution $W = P_-(x, \Lambda) Z$, \eqref{EVPconj} reduces to

\begin{align*}
[P_-(x, \Lambda) Z(x)]' &= A(x; \Lambda)(P_-(x, \Lambda) Z(x)) \\
P_-'(x, \Lambda) Z(x) + P_-(x, \Lambda) Z'(x)
&= A(x; \Lambda)P_-(x, \Lambda) Z(x) \\
(A(x; \Lambda)P_-(x, \Lambda) - P_-(x, \Lambda) A_-)Z(x) + P_-(x, \Lambda) Z'(x)
&= A(x; \Lambda)P_-(x, \Lambda) Z(x) \\
A(x; \Lambda)P_-(x, \Lambda)Z(x) - P_-(x, \Lambda) A_- Z(x) + P_-(x, \Lambda) Z'(x)
&= A(x; \Lambda)P_-(x, \Lambda) Z(x) \\
P_-(x, \Lambda) Z'(x) &= P_-(x, \Lambda) A_- Z(x) \\
Z'(x) &= A_- Z(x)
\end{align*}

which is what we want. In the last line, we used the fact that $P_-(x, \Lambda)$ is inverible, so we should make sure that is the case. Thus, we wish to find $P_-(x, \Lambda)$ such that

\[
P_-'(x, \Lambda) = A(x; \Lambda)P_-(x, \Lambda) - P_-(x, \Lambda) A_-
\]

We note that the this equation has the form 

\begin{equation}\label{solvePminus}
P_-'(x, \Lambda) = \mathcal{A}(x; \Lambda) P_-(x, \Lambda)
\end{equation}

where $\mathcal{A}(x; \Lambda)$ is the linear operator

\[
\mathcal{A}(x; \Lambda) P = A(x; \Lambda) P - P A_-
\]

By our assumptions on $A(x; \Lambda)$, $\mathcal{A} \rightarrow \mathcal{A}_-$ as $x \rightarrow -\infty$, where the limiting linear operator $\mathcal{A}_-$ is defined by

\[
\mathcal{A}_- P = A_- P - P A_-
\]

The limiting operator has analytic eigenvalue/eigenvector pair $0, I$ for all $\Lambda$, thus by the Gap Lemma, there exists a solution of \eqref{solvePminus} of the form 

\begin{equation}
P_-(x, \Lambda) = I + \mathcal{O}(e^{-\tilde{\theta}|x|})
\end{equation}

In other words, 

\begin{equation}
P_-(x, \Lambda) = I + \Theta_-(x, \Lambda)
\end{equation}

where 

\begin{equation}\label{Thetabound}
|\Theta_-(x, \Lambda)| \leq C e^{-\tilde{\theta}|x|}
\end{equation}

The $x$-derivative bound follow from the derivative bounds in the Gap Lemma, and the $\Lambda$-derivative bounds follow from standard analytic function theory.\\

Finally, we need to show invertibility of $P_-(x, \Lambda)$ for all $x \in \R^-$. Using \eqref{Thetabound}, we can find $M$ sufficiently large and negative such that for all $x \leq M$,

\[
|\Theta_-(x, \Lambda)| < 1/2
\]

It follows that $P_-(x, \Lambda)$ is invertible for $X \leq M$. To extend invertibility to all $x \in \R^-$, suppose that $P_-(x, \Lambda)^{-1}$ exists for all $x \in R^-$. Then, differentiating $P_-(x, \Lambda)^{-1} P_-(x, \Lambda) = I$ and solving for $[P_-(x, \Lambda)^{-1}]'$ (as in the proof of the inverse function theorem), we have (suppressing the dependence on $\Lambda$ for convenience)

\begin{align*}
(P_-^{-1})'(x) &= -P_-^{-1}(x)P_-'(x)P_-^{-1}(x) \\
&= -P_-^{-1}(x)( A(x)P_-(x) - P_-(x) A_-)P_-^{-1}(x) \\
&= A_- P_-^{-1}(x) - A(x) P_-^{-1}(x)
\end{align*}

We have a solution to this ODE for $x \leq M$, and by variation of constants, this ODE has a unique solution for all $x \in \R^-$. Thus $P_-(x, \Lambda)^{-1}$ is obtained for all $x \in \R^-$ by evolving this ODE forward from an initial condition at some $x \leq M$. In this manner, we have shown that $P_-(x, \Lambda)^{-1}$ exists for all $x \in \R^-$.

\end{proof}
\end{lemma}

We will use the Conjugation Lemma so that the evolution of the variational equation does not depend on the parameter vector $\Lambda = (\lambda, q_i^\pm)$. Note that the parameter $q_i^\pm$  is in the Banach space of continuous functions on $[X_{i-1}, 0]$ or $[0, X_i]$, so we do need to use a version of the Conjugation Lemma which allows parameters to be in an arbitraty Banach space.\\

$A_i^\pm(x; \lambda)$ is linear, thus analytic, in $\lambda$ and in $q_i^\pm(x)$. It should also depend nicely on the initial conditions $\beta_i^\pm$. For all $\lambda$, $A^\pm(x; \lambda)$ decays exponentially to the constant-coefficient matrix $A(\lambda)$. 

\begin{align}\label{Alambda}
A(\lambda) &=  \begin{pmatrix}0 & 1 & 0 & 0 & 0 \\0 & 0 & 1 & 0 & 0 \\0 & 0 & 0 & 1 & 0 \\0 & 0 & 0 & 0 & 1 \\
\lambda & -c & 0 & 1 & 0 \end{pmatrix}
\end{align}

Let

\begin{equation}
P_i^\pm(x; \lambda) = P^\pm(x; \lambda, \beta_i^\pm, u_i^\pm)
\end{equation}

be the conjugation operator for $A_i^\pm(x; \lambda) = A(x; q_i^\pm(x), \lambda)$.  Using the Conjugation Lemma, we make the substitution $W_i^\pm = P_i^\pm(x; \lambda) Z_i^\pm$. Then our system becomes

\begin{align*}
&(Z_i^\pm(x))' = A(\lambda) Z_i^\pm(x) + \lambda^2 d_i P_i^\pm(x; \lambda)^{-1} \tilde{H}_i^\pm(x) \\
&P_i^\pm(0; \lambda) Z_i^\pm(0) \in \C \Psi(0) \oplus Y^+ \oplus Y^- \oplus Y^0 \\
&P_i^+(0; \lambda) Z_i^+(0) - P_i^-(0; \lambda) Z_i^-(0) \in S  \\
&P_i^+(X_i; \lambda) Z_i^+(X_i)\ - P_{i+1}^-(-X_i; \lambda) Z_{i+1}^-(-X_i; \lambda) = D_i d
\end{align*}

and the jump conditions become

\begin{align*}
\langle \Psi(0), P_i^+(0; \lambda) Z_i^+(0) - P_i^-(0; \lambda) Z_i^-(0) \rangle &= 0 \\
\langle \Psi^c(0), P_i^+(0; \lambda) Z_i^+(0) - P_i^-(0; \lambda) Z_i^-(0) \rangle &= 0
\end{align*}

\subsection{Evolution}

Since $A(\lambda)$ is constant coefficient, we know exactly how solutions of $V' = A(\lambda)V$ will behave.\\

Since $A(\lambda)$ depends linearly on $\lambda$ and $A(0)$ has a simple eigenvalue at 0, for sufficiently small $\lambda$, $A(\lambda)$ will have a simple, small eigenvalue $\nu(\lambda)$, and $\nu(\lambda) = \mathcal{O}(\lambda)$.\\ 

We define the following.

\begin{enumerate}
	\item Let

	\begin{align*}
	X_m &= \min(X_0, \dots, X_{n-1}) \\
	X_M &= \max(X_0, \dots, X_{n-1}) \\
	\end{align*}

	\item Choose $\rho > 0$ sufficiently small so that $\alpha_0 - 4 \rho > 0$. Let

	\begin{align*}
	\alpha &= \alpha_0 - \rho \\
	\tilde{\alpha} &= \alpha - 3 \rho
	\end{align*}

	\item Choose $\delta$ sufficiently small so that for all $|\lambda| < \delta$
	\begin{enumerate}
		\item For the small eigenvalue of $A(\lambda)$ we have $|\nu(\lambda)| < \rho$
		\item The real part of any other eigenvalue of $A(\lambda)$ lies outside the interval $[-\alpha, \alpha]$.
	\end{enumerate}

	\item Choose $X_m$ sufficiently large so that
	\begin{equation}
	e^{-\tilde{\alpha} X_m} < \delta
	\end{equation}

\end{enumerate}

I STOPPED HERE.

Let $P^{u/s/c}_0(\lambda)$ be the eigenprojections for the unstable/stable/center subspaces $E^{u/s/c}(\lambda)$ of $A(\lambda)$. The center subspace is a ``true'' center subspace only when $\nu(\lambda)$ has no real part, e.g. when $\lambda = 0$, but we will always call it a center subspace for convenience. \\

Let $\Phi(x, y; \lambda) = e^{A(\lambda)(x-y)}$ be the evolution of the constant-coefficient ODE

\[
Z' = A(\lambda) Z
\]

Let $\Phi^{u/s/c}(x, y; \lambda)$ be the evolutions on the respective eigenspaces. Since $E^c(\lambda)$ is one-dimensional, we have in particular that

\begin{align*}
\Phi^c(x, y; \lambda) v &= e^{\nu(\lambda)(x - y)} v && v \in E^c(\lambda)
\end{align*}

We also have the bounds

\begin{align*}
|\Phi^s(x, y; \lambda)| &\leq C e^{-\alpha(x - y)} \\
|\Phi^u(x, y; \lambda)| &\leq C e^{-\alpha(y - x)} \\
|\Phi^c(x, y; \lambda)| &\leq C e^{\rho|x - y|} 
\end{align*}

We will want to relate things to the variational/adjoint variational problem for the nontransformed system, in particular to the variational equation for the linearization about the primary pulse.

\begin{align*}
V_i' &= A(q(x), 0) V_i \\
W_i' &= -A(q(x), 0)^* W_i
\end{align*}

Recall that $Q'(x)$ solves the variational problem, and we have solutions $\Psi(x)$ and $1$ (in the periodic case) to the adjoint variational problem. Let $P^\pm(x)$ conjugate $A(q(x), 0)$.\\

Let $\Theta(y, x)$ be the evolution for the untransformed variational equation. We would like to relate this to the evolution of the transformed equation. For appropriate values of $x, y$, we have

\[
\Theta(y, x) = P^\pm(y) \Phi(y, x; 0) P^\pm(x)^{-1}
\]

Finally, we will have occasion to Taylor expand the conjugation operators. Note that 

\[
P^\pm(y) = P^\pm(y; 0, 0, 0)
\]

For $y$ large, all the conjugation operators are approximately the identity, so that will not matter. The Taylor expansion we will need is that at $x = 0$.

\begin{equation}\label{PTaylor}
P_i^\pm(0; \lambda) = P^\pm(0) + \mathcal{O}(|\lambda| + e^{-\alpha X_m})
\end{equation}

\subsection*{Conjugation}

We will use the Conjugation Lemma to simplify the eigenvalue problem. Here is a version of the Conjugation Lemma, which is adapted and cleaned up from Zum2018. 

\begin{lemma}[Conjugation Lemma]
Let $W \in \C^N$, and consider the family of ODEs on $\R$

\begin{equation}\label{LambdaEVPconj}
W(x)' = A(x; \Lambda) W(x) + B(x) W(x) + F(x) 
\end{equation}

where $\Lambda \in \Omega$ is a parameter vector and $\Omega$ is a Banach space. Take the same assumptions as in the Gap Lemma, i.e. 

\begin{enumerate}
	\item The map $\Lambda \mapsto A(\cdot; \Lambda)$ is analytic in $\Lambda$.
	\item $A(x; \Lambda) \rightarrow A_\pm(\lambda)$ (independent of $\Lambda$) as $x \rightarrow \pm \infty$, and for $|\Lambda| < \delta$ we have the uniform exponential decay estimates 
	\begin{align}
	\left| \frac{\partial^k}{\partial x^k} A(x; \Lambda) - A_\pm(\Lambda) \right| 
	&\leq C e^{-\theta |x|} && 0 \leq k \leq K
	\end{align}
	where $\alpha > 0$, $C > 0$, and $K$ is a nonnegative integer.
\end{enumerate}

Then in a neighborhood of any $\Lambda_0 \in \Omega$ there exist invertible linear transformations

\begin{align*}
P_+(x, \Lambda) &= I + \Theta_+(x, \Lambda) \\
P_-(x, \Lambda) &= I + \Theta_-(x, \Lambda) 
\end{align*}

defined on $\R^+$ and $\R^-$, respectively, such that

\begin{enumerate}[(i)]
\item The change of coordinates $W = P_\pm Z$ reduces \eqref{LambdaEVPconj} to the equations on $\R^\pm$

\begin{align}
Z'(x) = A^\pm(\Lambda) Z(x) + P_\pm(x, \Lambda)^{-1} B(x) P_\pm(x, \Lambda) Z(x) + P_\pm(x, \Lambda)^{-1} F(x)
\end{align}

where

\[
G(x; \Lambda) = P_\pm(x, \Lambda)^{-1} F(x)
\]

\item For any fixed $0 < \tilde{\theta} < \theta$, $0 \leq k \leq K+1$, and $j \geq 0$ we have the decay rates
\begin{align*}
\left| \partial_\Lambda^j \partial_x^k \Theta_\pm \right| \leq C(j, k)e^{-\tilde{\theta}|x|}
\end{align*}
\end{enumerate}
\begin{proof}
I have written out the proof somewhere else for the case where $B(x) = 0$ and $F(x) = 0$, which essentially follows Zum2018 but fills in more details. Only a small modification is necessary for the general case. It should also not be hard to modify the proof to work for a general Banach space $\Omega$ instead of a subset of $\C^n$.
\end{proof}
\end{lemma}



\subsection*{The Inversion}

Define the spaces

\begin{align*}
V_a &= \bigoplus_{i=0}^{n-1} E^u(\lambda) \oplus E^s(\lambda) \\
V_b &= \bigoplus_{i=0}^{n-1} E^u(0) \oplus E^s(0) \\
V_c^+ &= \bigoplus_{i=0}^{n-1} E^c(\lambda) \\
V_c^- &= \bigoplus_{i=0}^{n-1} E^c(\lambda) \\
V_c &= V_c^+ \oplus V_c^- \\
V_\lambda &= B_\delta(0) \subset \C
\end{align*}

where the subscripts are all $\mod n$, as in the existence problem. We use the $\lambda-$dependent eigenspaces for $a_i^\pm$ and $c_i^\pm$, since we will be evolving them under the $\lambda-$dependent evolution. All the product spaces are endowed with the maximum norm, e.g. for $V_c$, $|c| = \max(|c_0^-|, \dots, |c_{n-1}^-|, |c_0^+|, \dots, |c_{n-1}^+|)$. In addition, we take the following convention: if we eliminate either a subscript or a superscript (or both) in the norm, we are taking the maximum over the eliminated thing. For example,
\begin{enumerate}
	\item $|c_i| = \max(|c_i^+|, |c_i^-|)$ 
	\item $|c^+| = \max(|c_0^+|, \dots, |c_{n-1}^+|)$
\end{enumerate}

Next, we write down the fixed point equations for the problem. For $i = 0, \dots, n-1$, the fixed point equations are

\begin{align*}
Z_i^-(x) &= \Phi^s(x, -X_{i-1}; \lambda) a_{i-1}^- + \Phi^u(x, 0; \lambda) b_i^- + \Phi^c(x, -X_{i-1}; \lambda) c_{i-1}^- \\
&+ \lambda^2 d_i \int_0^x \Phi^u(x, y; \lambda) P_i^-(y; \lambda)^{-1} \tilde{H}_i^-(y)] dy \\
&+ \lambda^2 d_i \int_{-X_{i-1}}^x \Phi^s(x, y; \lambda) P_i^-(y; \lambda)^{-1} \tilde{H}_i^-(y) dy \\
&+ \lambda^2 d_i \int_{-X_{i-1}}^x \Phi^c(x, y; \lambda) P_i^-(y; \lambda)^{-1} \tilde{H}_i^-(y) dy  \\ 
Z_i^+(x) &= \Phi^u(x, X_i; \lambda) a_i^+ + \Phi^s(x, 0; \lambda) b_i^+ + \Phi^c(x, X_i; \lambda) c_i^+ \\
&+ \lambda^2 d_i \int_0^x \Phi^s(x, y; \lambda) P_i^+(y; \lambda)^{-1} \tilde{H}_i^+(y) dy \\
&+ \lambda^2 d_i \int_{X_i}^x \Phi^u(x, y; \lambda) P_i^+(y; \lambda)^{-1} \tilde{H}_i^+(y) dy \\
&+ \lambda^2 d_i \int_{X_i}^x \Phi^c(x, y; \lambda) P_i^+(y; \lambda)^{-1} \tilde{H}_i^+(y) dy \\
\end{align*}

% match at ends

\subsubsection*{Matching at ends}

Here, we solve the condition

\[
P_i^+(X_i; \lambda) Z_i^+(X_i) - P_{i+1}^-(-X_i; \lambda) Z_{i+1}^-(-X_i) = D_i d
\]

At $\pm X_i$, the fixed point equations become

\begin{align*}
Z_{i+1}^-(-X_i) &= a_i^- + \Phi^u(-X_i, 0; \lambda) b_{i+1}^- + c_i^- 
+ \lambda^2 d_{i+1} \int_0^{-X_i} \Phi^u(-X_i, y; \lambda) P_{i+1}^-(y; \lambda)^{-1} \tilde{H}_i^-(y) dy \\
Z_i^+(X_i) &= a_i^+ + \Phi^s(X_i, 0; \lambda) b_i^+ + c_i^+ 
+ \lambda^2 d_i \int_0^{X_i} \Phi^s(X_i, y; \lambda) P_i^+(y; \lambda)^{-1} \tilde{H}_i^+(y) dy
\end{align*}

To obtain these, we used the fact that, for example, $a_i^- \in E^s(\lambda)$ and $\Phi^s(-X_{i-1}, -X_{i-1}; \lambda)$ is the identity on $E^s(\lambda)$. From the Conjugation Lemma, we have

\begin{equation}\label{conjest}
P_i^\pm(\pm X_i; \lambda) = I + \mathcal{O}(e^{-\alpha X_i})
\end{equation}

which we will use on the $a_i^\pm$ and $c_i^\pm$ terms. Thus we obtain the equation

\begin{align}\label{Dideq1}
D_i d &= a_i^+ - a_i^- + c_i^+ - c_i^- + L_3(\lambda)_i(a, b, c^+, c^-, d)
\end{align}

For a bound on $L_3$, we look at the individual terms. As usual, we will in general only look at one of the two pieces.

\begin{enumerate}

\item For the $a_i^\pm$ and $c_i^\pm$ terms, we have a term of order $\mathcal{O}(e^{-\alpha X_i}(|a_i| + |c_i^+| + |c_i^-|)$, which comes from the conjugation operators $P_i^\pm(\pm X_i; \lambda)$.

\item For the terms involving $b$, we have

\[
| P_i^-(-X_i; \lambda) \Phi^u(-X_i, 0; \lambda) b_{i+1}^-| \leq C e^{-\alpha X_i} |b_{i+1}
^-|
\]

\item For the integral terms, we have

\begin{align*}
&\left|
P^+(X_i; \beta_i^+, \lambda) \int_0^{X_i} \Phi^s(X_i, y; \lambda) P^+(X_i; \beta_i^+, \lambda)^{-1} \tilde{H}_i^+(y) dy \right| \\
&\leq C \int_0^{X_i} e^{-\alpha(X_i - y)}e^{-\alpha y} dy \\
&\leq C \int_0^{X_i} e^{-(\alpha - \rho)(X_i - y)}e^{-\alpha y} dy \\
&= C e^{-(\alpha - \rho) X_i} \int_0^{X_i} e^{-\rho y} dy \\ 
&\leq C e^{-(\alpha - \rho) X_i} 
\end{align*}

\end{enumerate}

Putting these all together, we have the following bound for $L_3$.
\[
|L_3(\lambda)_i(a, b, c^+, c^-, d)| \leq C \Big( e^{-\alpha X_i} ( |a_i| + |b_i^+| + |b_{i+1}^-| + |c_i^+| + |c_i^-|) + e^{-(\alpha - \rho) X_i} |\lambda^2| |d| \Big)
\]

Following San98 (and leaving out some steps for now), we can solve this for $(a, c^+)$ to get $(a_i, c_i^+) = A_1(\lambda)_i(b, c_i^-, d)$, with bound

\begin{align*}
|A_1&(\lambda)_i(b, c^-, d)|
\leq C \Big( e^{-\alpha X_i} (|b_i^+| + |b_{i+1}^-|) + |c_i^-| + e^{-(\alpha - \rho) X_i} |\lambda^2||d| + |D_i||d| \Big)
\end{align*} 

As in San98, we hit \eqref{Dideq1} with projections on the subspaces eigenspaces $E^{s/u/c}(\lambda)$. The remainder term $A_2(\lambda)_i(b, c^-, d)$ is found by substituting the bound for $A_1$ into $L_3$ and simplifying.

\begin{align*}
a_i^+ &= P_0^u(\lambda) D_i d + A_2(\lambda)_i^+(b, c^-, d) \\
a_i^- &= -P_0^s(\lambda) D_i d + A_2(\lambda)_i^-(b, c^-, d) \\
c_i^+ &= c_i^- + P_0^c(\lambda) D_i d + A_2(\lambda)_i^c(b, c^-, d) )
\end{align*}

where we have bound

\begin{align*}
|A_2&(\lambda)_i(b, d)|
\leq C \Big( e^{-\alpha X_i} (|b_i^+| + |b_{i+1}^-| + |c_i^-|) + e^{-(\alpha - \rho) X_i} |\lambda|^2|d| + e^{-\alpha X_i} |D_i||d| \Big)
\end{align*} 

For the first two, this is not quite what we want. Anticipating what we will need later, we write $a_i^+$ as

\begin{align*}
a_i^+ = P_i^+(X_i; \lambda)a_i^+ + (I - P_i^+(X_i; \lambda))a_i^+ &= P_0^u(\lambda) D_i d + A_2(\lambda)_i^+(b, c^-, d)
\end{align*}

Rearranging this, we obtain

\begin{align*}
P_i^+(X_i; \lambda) a_i^+ &= P_0^u(\lambda) D_i d + A_2(\lambda)_i^+(b, c^-, d) - (I - P_i^+(X_i; \lambda))a_i^+ \\
&= P_0^u(\lambda) D_i d + A_2(\lambda)_i^+(b, c^-, d) + \mathcal{O}\Big( e^{-\alpha X_i} ( e^{-\alpha X_i} (|b_i^+| + |b_{i+1}^-|) + |c_i^-| + e^{-(\alpha - \rho) X_i} |\lambda^2||d| + |D_i||d| )\Big)
\end{align*}

where we used the bound $A_1$ and the estimate \eqref{conjest}. The last term on the RHS is the same (or higher) order as $A_2$, so we incorporate that into $A_2(\lambda)_i^+(b, c^-, d)$ to get

\begin{align*}
P_i^+(X_i; \lambda)a_i^+ &= P_0^u(\lambda) D_i d + A_2(\lambda)_i^+(b, c^-, d)
\end{align*}

Finally, we operate on both sides on the left by $P_i^+(X_i; \lambda)^{-1}$ to solve for $a_i^+$. This is a bounded operator, so we will also incorporate this into $A_2(\lambda)_i^+(b, c^-, d)$ (the bound will be unchanged). Thus we have

\begin{align*}
a_i^+ &= P^+(X_i; \beta_i^+, \lambda) P_0^u(\lambda) D_i d + A_2(\lambda)_i^+(b, c^-, d)
\end{align*}

We do the same thing for $a_i^-$, giving us

\begin{align*}
a_i^+ &= P_i^+(X_i; \lambda) P_0^u(\lambda) D_i d + A_2(\lambda)_i^+(b, c^-, d) \\
a_i^- &= -P_i^-(-X_i; \lambda) P_0^s(\lambda) D_i d + A_2(\lambda)_i^-(b, c^-, d)
\end{align*}

For the third one, we would like to evaluate and get an estimate for $P_0^c(\lambda) D_i d$. Recall that

\[
D_i d = ( Q'(X_i) + Q'(-X_i))(d_{i+1} - d_i ) + \mathcal{O} \left( e^{-\alpha X_i} \left( |\lambda| +  e^{-\alpha X_i}  \right) |d| \right) 
\]

Looking at the lower order terms,

\begin{align*}
P_0^c(\lambda)&( Q'(X_i) + Q'(-X_i)) 
= P_0^c(0)( Q'(X_i) + Q'(-X_i)) + \mathcal{O}(|\lambda|e^{-\alpha X_i}) \\
&= \mathcal{O}(e^{-\alpha X_i}(|\lambda| + e^{-\alpha X_i}))
\end{align*}

Thus we have

\[
|P_0^c(\lambda) D_i d| \leq C e^{-\alpha X_i}(|\lambda| + e^{-\alpha X_i})|d|
\]

\subsubsection*{Matching at 0}

The next step is to satisfy the conditions

\begin{align*}
P_i^\pm(0; \lambda) Z_i^\pm(0) &\in \C \Psi(0) \oplus Y^0 \oplus Y^+ \oplus Y^- \\
P_i^+(0; \lambda) Z_i^+(0) - P_i^-(0; \lambda) Z_i^-(0) &\in \C \Psi(0) \oplus Y^0
\end{align*}

Recall that we have

\[
\C^m = \C \Psi(0) \oplus \C Q'(0) \oplus Y^0 \oplus Y^+ \oplus Y^- 
\]

This condition is equivalent to the three projections

\begin{align*}
P(\C Q'(0) ) P_i^-(0; \lambda) Z_i^-(0) &= 0 \\
P(\C Q'(0) ) P_i^+(0; \lambda) Z_i^+(0) &= 0 \\
P(Y_i^+ \oplus Y_i^-) ( P_i^+(0; \lambda) Z_i^+(0) - P_i^-(0; \lambda) Z_i^-(0) ) &= 0
\end{align*}

where the kernel of each projection is the remaining spaces in the direct sum. We don't need $\C Q'(0)$ in the third equation since we eliminated any component in it in the first two equations.\\

Recall that for $\lambda = 0$, the tangent space to the stable manifold at $x = 0$ is spanned by $Y^+$ and $Q'(0)$, and the tangent space to the unstable manifold at $x = 0$ is spanned by $Y^-$ and $Q'(0)$. Thus we have

\begin{align*}
P^-(0)^{-1} Q'(0) &= v^- \in E^u(0) \\
P^+(0)^{-1} Q'(0) &= v^+ \in E^s(0)
\end{align*}

Let

\begin{align*}
E^u(0) &= \C v^- \oplus E^- \\
E^s(0) &= \C v^+ \oplus E^+ \\
\end{align*}

Then we have

\begin{align*}
P^-(0)^{-1} Y^- = E^- \\
P^+(0)^{-1} Y^+ = E^+ \\
\end{align*}

Following San98, we decompose $b_i^\pm$ uniquely as $b_i^\pm = x_i^\pm + y_i^\pm$, where $x_i^\pm \in \C v^\pm$ and $y_i^\pm \in E^\pm$.\\

At $x = 0$, the fixed point equations become

\begin{align*}
Z_i^-(0) &= \Phi^s(0, -X_{i-1}; \lambda) a_{i-1}^- + \Phi^u(0, 0; \lambda) b_i^- + \Phi^c(0, -X_{i-1}; \lambda) c_{i-1}^- \\
&+ \lambda^2 d_i \int_{-X_{i-1}}^0 \Phi^s(0, y; \lambda) P_i^-(y; \lambda)^{-1} \tilde{H}_i^-(y) dy \\
&+ \lambda^2 d_i \int_{-X_{i-1}}^0 \Phi^c(0, y; \lambda) P_i^-(y; \lambda)^{-1} \tilde{H}_i^-(y) dy  \\ 
Z_i^+(0) &= \Phi^u(0, X_i; \lambda) a_i^+ + \Phi^s(0, 0; \lambda) b_i^+ + \Phi^c(0, X_i; \lambda) c_i^+ \\
&+ \lambda^2 d_i \int_{X_i}^0 \Phi^u(0, y; \lambda) P_i^+(y; \lambda)^{-1} \tilde{H}_i^+(y) dy \\
&+ \lambda^2 d_i \int_{X_i}^0 \Phi^c(0, y; \lambda) P_i^+(y; \lambda)^{-1} \tilde{H}_i^+(y) dy \\
\end{align*}

Noting that $\Phi^u(0, 0; \lambda) = P_0^u(0)$, doing a little manipulation on the $b_i$ terms, and using the known form of the evolution $\Phi^c$ on $E^c(\lambda)$, this becomes

\begin{align*}
Z_i^-(0) &= \Phi^s(0, -X_{i-1}; \lambda) a_{i-1}^- + x_i^- + y_i^- + (P_0^u(\lambda) - P_0^u(0))b_i^- + e^{\nu(\lambda) X_{i-1}} c_{i-1}^- \\
&+ \lambda^2 d_i \int_{-X_{i-1}}^0 \Phi^s(0, y; \lambda) P_i^-(y; \lambda)^{-1} \tilde{H}_i^-(y) dy \\
&+ \lambda^2 d_i \int_{-X_{i-1}}^0 \Phi^c(0, y; \lambda) P_i^-(y; \lambda)^{-1} \tilde{H}_i^-(y) dy  \\ 
Z_i^+(0) &= \Phi^u(0, X_i; \lambda) a_i^+ + x_i^+ + y_i^+ + (P_0^s(\lambda) - P_0^s(0)) b_i^+ + e^{-\nu(\lambda)X_i} c_i^+ \\
&+ \lambda^2 d_i \int_{X_i}^0 \Phi^u(0, y; \lambda) P_i^+(y; \lambda)^{-1} \tilde{H}_i^+(y) dy \\
&+ \lambda^2 d_i \int_{X_i}^0 \Phi^c(0, y; \lambda) P_i^+(y; \lambda)^{-1} \tilde{H}_i^+(y) dy \\
\end{align*}

Since $c_i^\pm$ are in the eigenspaces $E^c(\lambda)$, we do some further manipulation to separate out a component in $E^c(0)$.

\begin{align*}
Z_i^-(0) &= \Phi^s(0, -X_{i-1}; \lambda) a_{i-1}^- + x_i^- + y_i^- + (P_0^u(\lambda) - P_0^u(0))b_i^- \\
&+ P_0^c(0) e^{\nu(\lambda) X_{i-1}} c_{i-1}^- + (P_0^c(\lambda) - P_0^c(0)) e^{\nu(\lambda) X_{i-1}} c_{i-1}^- \\
&+ \lambda^2 d_i \int_{-X_{i-1}}^0 \Phi^s(0, y; \lambda) P_i^-(y; \lambda)^{-1} \tilde{H}_i^-(y) dy \\
&+ \lambda^2 d_i \int_{-X_{i-1}}^0 \Phi^c(0, y; \lambda) P_i^-(y; \lambda)^{-1} \tilde{H}_i^-(y) dy  \\ 
Z_i^+(0) &= \Phi^u(0, X_i; \lambda) a_i^+ + x_i^+ + y_i^+ + (P_0^s(\lambda) - P_0^s(0)) b_i^+ \\
&+ P_0^c(0) e^{-\nu(\lambda)X_i} c_i^+ + (P_0^c(\lambda) - P_0^c(0)) e^{-\nu(\lambda)X_i} \\
&+ \lambda^2 d_i \int_{X_i}^0 \Phi^u(0, y; \lambda) P_i^+(y; \lambda)^{-1} \tilde{H}_i^+(y) dy \\
&+ \lambda^2 d_i \int_{X_i}^0 \Phi^c(0, y; \lambda) P_i^+(y; \lambda)^{-1} \tilde{H}_i^+(y) dy \\
\end{align*}

Finally, we operate on these by $P_i^\pm(0; \lambda)$. For the $c_i^-$ and $b$ terms, we write these as

\[
P_i^\pm(0; \lambda) = P^\pm(0) + (P_i^\pm(0; \lambda) - P^\pm(0))
\]

We finally wind up with

\begin{align*}
P_i^-(0; \lambda) Z_i^-(0) &= P^-(0)( x_i^- + y_i^- + P_0^c(0) e^{\nu(\lambda) X_{i-1}} c_{i-1}^- ) \\
&+ P_i^-(0; \lambda) \Phi^s(0, -X_{i-1}; \lambda) a_{i-1}^- + (P_i^-(0; \lambda) - P^-(0))b_i^- + P_i^-(0; \lambda)(P_0^u(\lambda) - P_0^u(0))b_i^- \\
&+ (P_i^-(0; \lambda) - P^-(0)) P_0^c(0) e^{\nu(\lambda) X_{i-1}} c_{i-1}^- + P_i^-(0; \lambda) (P_0^c(\lambda) - P_0^c(0)) e^{\nu(\lambda) X_{i-1}} c_{i-1}^- \\
&+ \lambda^2 d_i P_i^-(0; \lambda) \int_{-X_{i-1}}^0 \Phi^s(0, y; \lambda) P_i^-(y; \lambda)^{-1} \tilde{H}_i^-(y) dy \\
&+ \lambda^2 d_i P_i^-(0; \lambda) \int_{-X_{i-1}}^0 \Phi^c(0, y; \lambda) P_i^-(y; \lambda)^{-1} \tilde{H}_i^-(y) dy  \\ 
P_i^+(0; \lambda) Z_i^+(0) &=  P^+(0)( x_i^+ + y_i^+ + P_0^c(0) e^{-\nu(\lambda)X_i} c_i^+ )\\
&+ P_i^+(0; \lambda) \Phi^u(0, X_i; \lambda) a_i^+ + (P_i^+(0; \lambda) - P^+(0)) b_i^+ + P_i^+(0; \lambda) (P_0^s(\lambda) - P_0^s(0)) b_i^+ \\
&+ (P_i^+(0; \lambda) - P^+(0))P_0^c(0) e^{-\nu(\lambda)X_i} c_i^+ + P_i^+(0; \lambda) (P_0^c(\lambda) - P_0^c(0)) e^{-\nu(\lambda)X_i} c_i^+\\
&+ \lambda^2 d_i P_i^+(0; \lambda) \int_{X_i}^0 \Phi^u(0, y; \lambda) P_i^+(y; \lambda)^{-1} \tilde{H}_i^+(y) dy \\
&+ \lambda^2 d_i P_i^+(0; \lambda) \int_{X_i}^0 \Phi^c(0, y; \lambda) P_i^+(y; \lambda)^{-1} \tilde{H}_i^+(y) dy \\
\end{align*}

Note that with this setup, the projections we will take either eliminate or act as the identity on the terms in the first lines of $P_i^-(0; \lambda) Z_i^-(0)$ and $P_i^+(0; \lambda) Z_i^+(0)$. Thus we obtain an expression of the form

\[
\begin{pmatrix}x_i^- \\ x_i^+ \\ 
y_i^+ - y_i^- \end{pmatrix} + L_4(\lambda)_i(b, c, \tilde{c}, d) = 0
\]

where, for convenience, we define

\begin{equation}\label{tildec}
\tilde{c}_i^\pm = e^{\pm \nu(\lambda) X_i} c_i^-
\end{equation}

To get a bound on $L_4$, we need to bound the individual terms from the fixed point equations above.

\begin{enumerate}

\item For the $a_i$ terms, we substitute the bound for $A_1(\lambda)$ to get

\begin{align*}
|P_i^-(0; \lambda) \Phi^s(0, -X_{i-1}; \lambda) a_{i-1}^-|
&\leq C \Big( e^{-2 \alpha X_{i-1}} (|b_{i-1}^+| + |b_i^-|) + e^{-\alpha X_{i-1}}|c_{i-1}^-| + e^{-\alpha X_{i-1}}(e^{-(\alpha - \rho) X_{i-1}} |\lambda^2| + |D_{i-1}|)|d| \Big) \\
|P_i^+(0; \lambda) \Phi^u(0, X_i; \lambda) a_i^+|
&\leq C \Big( e^{-2 \alpha X_i} (|b_i^+| + |b_{i+1}^-|) + e^{-\alpha X_i} |c_i^-| + e^{-\alpha X_i} (e^{-(\alpha - \rho) X_i} |\lambda^2| + |D_i|)|d| \Big)
\end{align*}

\item For the $b_i$ terms, we have

\[
|(P_i^-(0; \lambda) - P^-(0))b_i^- + P_i^-(0; \lambda)(P_0^u(\lambda) - P_0^u(0))b_i^-| \leq C ( e^{-\alpha X_m} + |\lambda|)|b_i^-|
\]

\item For the $c_i^-$ terms, we have

\begin{align*}
|P_i^-(0; \lambda) - P^-(0)) P_0^c(0) e^{\nu(\lambda) X_{i-1}} c_{i-1}^- + P_i^-(0; \lambda) (P_0^c(\lambda) - P_0^c(0)) e^{\nu(\lambda) X_{i-1}} c_{i-1}^- |
\leq C (e^{-\alpha X_m} + |\lambda|)|\tilde{c}_{i-1}^+|)
\end{align*}

\item For the $c_i^+$ terms, we have

\begin{align*}
|P_i^+(0; \lambda) - P^+(0))P_0^c(0) e^{-\nu(\lambda)X_i} c_i^+ + P_i^+(0; \lambda) (P_0^c(\lambda) - P_0^c(0)) e^{-\nu(\lambda)X_i} c_i^+| \leq C (e^{-\alpha X_m} + |\lambda|)|e^{-\nu(\lambda)X_i} c_i^+|
\end{align*}

The only extra complication for $c_i^+$ is that we have to use the expression from the previous section involving $A_2$ to write $c_i^+$ in terms of $c_i^-$. Doing this, we obtain

\begin{align*}
e^{-\nu(\lambda)X_i} c_i^+ &= e^{-\nu(\lambda)X_i} c_i^- 
+ e^{-\nu(\lambda)X_i} P_0^c(\lambda) D_i d + e^{-\nu(\lambda)X_i} A_2(\lambda)_i^c(b, d)\\
&= e^{-\nu(\lambda)X_i} c_i^- + \mathcal{O}\Big( e^{-(\alpha - \rho) X_i} ( |\lambda| + e^{-\alpha X_i} ) |d|) + e^{-(\alpha - \rho) X_i} (|b_i^+| + |b_{i+1}^-| + |c_i^-|)\\
&+ e^{-(\alpha - 2 \rho) X_i} |\lambda|^2|d| + e^{-(\alpha - \rho) X_i} |D_i||d| ) \\
&= e^{-\nu(\lambda)X_i} c_i^- + \mathcal{O}\Big( e^{-(\alpha - 2 \rho) X_i} ( |b_i^+| + |b_{i+1}^-| + |c_i^-| + |\lambda||d| + |D_i||d|) \Big) \\
\end{align*}

Thus we have

\begin{align*}
&|P_i^+(0; \lambda) - P^+(0))P_0^c(0) e^{-\nu(\lambda)X_i} c_i^+ + P_i^+(0; \lambda) (P_0^c(\lambda) - P_0^c(0)) e^{-\nu(\lambda)X_i} c_i^+| \\
&\leq C \Big( (e^{-\alpha X_m} + |\lambda|)|\tilde{c}_i^-| + e^{-(\alpha - 2 \rho) X_i} ( |b_i^+| + |b_{i+1}^-| + |c_i^-| + |\lambda||d| + |D_i||d|) \Big)
\end{align*}

\item The bound on the integral terms is determined by the bound on the center subspace, since there is potential growth in that subspace. The integral terms involving $\tilde{H}$ are bounded by

\begin{align*}
\left| \lambda^2 d_i P_i^-(0; \lambda) \int_{-X_{i-1}}^0 \Phi^c(0, y; \lambda) P_i^-(y; \lambda)^{-1} \tilde{H}_i^-(y) dy \right| &\leq C |\lambda|^2 |d| \int_{-X_{i-1}}^0 e^{-\rho y} e^{\alpha y} dy \\
&\leq C |\lambda|^2 |d|
\end{align*}

\end{enumerate}

Putting all these together, we obtain the bound for $L_4(\lambda)_i(b, \tilde{c}, d)$. Note that the $\tilde{c}$ depend on the $c$, but we separate them out for convenience.

\begin{align*}
L_4(\lambda)_i(b, \tilde{c}, d) &\leq 
C\Big( (|\lambda| + e^{-\tilde{\alpha}X_m})|b| 
+ (|\lambda| + e^{-\tilde{\alpha}X_m}) |\tilde{c}_{i-1}^+| + |\tilde{c}_i^-|) + e^{-\tilde{\alpha} X_{i-1}} |c_{i-1}^-| + e^{-\tilde{\alpha} X_i} |c_i^-| \\
&+ ( e^{-\tilde{\alpha}X_m} |D| + e^{-\tilde{\alpha}X_m}|\lambda| + |\lambda|^2)|d| \Big)
\end{align*}

Peforming the inversion, we solve for $b$ to get $B_1(\lambda)(\tilde{c}, d)$, which has bound

\begin{align*}
|B_1(\lambda)_i(\tilde{c}, d)| \leq C\Big( 
(|\lambda| + e^{-\tilde{\alpha}X_m})( |\tilde{c}_{i-1}^+| + |\tilde{c}_i^-|)
+ e^{-\tilde{\alpha} X_{i-1}} |c_{i-1}^-| + e^{-\tilde{\alpha} X_i} |c_i^-| + ( e^{-\tilde{\alpha}X_m} |D| + e^{-\tilde{\alpha}X_m}|\lambda| + |\lambda|^2)|d| \Big)
\end{align*}

We can plug this into the bound for $A_2$ to get $A_4$ with bound

\begin{align*}
|A_4&(\lambda)_i(\tilde{c}, d)|
\leq C \Big( 
e^{-\alpha X_i} (|\lambda| + e^{-\tilde{\alpha}X_m})(|\tilde{c}_{i-1}^+| + |\tilde{c}_{i+1}^-|) + e^{-\tilde{\alpha}X_{i-1}}|c_{i-1}^-| + e^{-\tilde{\alpha}X_i}|c_i^-| + e^{-\tilde{\alpha}X_{i+1}}|c_{i+1}^-| \\
&+ e^{-\tilde{\alpha} X_m} |\lambda|^2|d| + e^{-\alpha X_m}|D||d| \Big)
\end{align*} 

We will also plug $B_1$ into the expression for $e^{-\nu(\lambda)X_i} c_i^+$.

\begin{align*}
e^{-\nu(\lambda)X_i} &c_i^+ = e^{-\nu(\lambda)X_i} c_i^- 
+ \mathcal{O}\Big( e^{-\tilde{\alpha}X_m} (|\lambda| + e^{-\tilde{\alpha}X_m})( |\tilde{c}_{i-1}^+| + |\tilde{c}_{i+1}^-|) 
+ e^{-\tilde{\alpha}X_i}|c_i^-| \\
&+ e^{-\tilde{\alpha}X_m}( e^{-\alpha X_{i-1}}|c_{i-1}^-| + e^{-\alpha X_{i+1}}|c_{i+1}^-| ) + e^{-\tilde{\alpha}X_m}(|\lambda| + |D_i|)|d| \Big) \\
\end{align*}

Now that we have solved (uniquely) for everything except for the $c_i^-$ and $d$, we are ready to compute the jump conditions in the two directions.

\subsubsection*{Jump in Y0 direction}

For this jump, we project on $Y_0$.

\[
\xi^c_i = P(Y^0) ( P_i^+(0; \lambda) Z_i^+(0) - P_i^-(0; \lambda) Z_i^-(0) )
\]

Recall that, from the previous section, the terms $P_i^\pm(0; \lambda) Z_i^\pm(0)$ are given by

\begin{align*}
P_i^-(0; \lambda) Z_i^-(0) &= P^-(0)( b_i^- + P_0^c(0) e^{\nu(\lambda) X_{i-1}} c_{i-1}^- ) \\
&+ P_i^-(0; \lambda) \Phi^s(0, -X_{i-1}; \lambda) a_{i-1}^- + (P_i^-(0; \lambda) - P^-(0))b_i^- + P_i^-(0; \lambda)(P_0^u(\lambda) - P_0^u(0))b_i^- \\
&+ (P_i^-(0; \lambda) - P^-(0)) P_0^c(0) e^{\nu(\lambda) X_{i-1}} c_{i-1}^- + P_i^-(0; \lambda) (P_0^c(\lambda) - P_0^c(0)) e^{\nu(\lambda) X_{i-1}} c_{i-1}^- \\
&+ \lambda^2 d_i P_i^-(0; \lambda) \int_{-X_{i-1}}^0 \Phi^s(0, y; \lambda) P_i^-(y; \lambda)^{-1} \tilde{H}_i^-(y) dy \\
&+ \lambda^2 d_i P_i^-(0; \lambda) \int_{-X_{i-1}}^0 \Phi^c(0, y; \lambda) P_i^-(y; \lambda)^{-1} \tilde{H}_i^-(y) dy  \\ 
P_i^+(0; \lambda) Z_i^+(0) &=  P^+(0)( b_i^+ + P_0^c(0) e^{-\nu(\lambda)X_i} c_i^+ )\\
&+ P_i^+(0; \lambda) \Phi^u(0, X_i; \lambda) a_i^+ + (P_i^+(0; \lambda) - P^+(0)) b_i^+ + P_i^+(0; \lambda) (P_0^s(\lambda) - P_0^s(0)) b_i^+ \\
&+ (P_i^+(0; \lambda) - P^+(0))P_0^c(0) e^{-\nu(\lambda)X_i} c_i^+ + P_i^+(0; \lambda) (P_0^c(\lambda) - P_0^c(0)) e^{-\nu(\lambda)X_i} c_i^+\\
&+ \lambda^2 d_i P_i^+(0; \lambda) \int_{X_i}^0 \Phi^u(0, y; \lambda) P_i^+(y; \lambda)^{-1} \tilde{H}_i^+(y) dy \\
&+ \lambda^2 d_i P_i^+(0; \lambda) \int_{X_i}^0 \Phi^c(0, y; \lambda) P_i^+(y; \lambda)^{-1} \tilde{H}_i^+(y) dy \\
\end{align*}

We will look at the significant terms first. There should be two of them.

\begin{enumerate}
\item For the terms involving $c$, we have

\begin{align*}
P(Y^0) &P^-(0)( P_0^c(0) e^{\nu(\lambda) X_{i-1}} c_{i-1}^- - P_0^c(0) e^{-\nu(\lambda)X_i} c_i^+) = e^{\nu(\lambda) X_{i-1}} c_{i-1}^- - e^{-\nu(\lambda)X_i} c_i^+ 
\end{align*}

We also have from the previous section
\begin{align*}
|(P_i^-(0; \lambda) - P^-(0)) P_0^c(0) e^{\nu(\lambda) X_{i-1}} c_{i-1}^- + P_i^-(0; \lambda) (P_0^c(\lambda) - P_0^c(0)) e^{\nu(\lambda) X_{i-1}} c_{i-1}^-| \leq C (|\lambda| + e^{-\alpha X_m}) |\tilde{c}_{i-1}^+|
\end{align*}

and

\begin{align*}
|(P_i^+(0; \lambda) - P^+(0))P_0^c(0) e^{-\nu(\lambda)X_i} c_i^+ + P_i^+(0; \lambda) (P_0^c(\lambda) - P_0^c(0)) e^{-\nu(\lambda)X_i} c_i^+| \leq C (|\lambda| + e^{-\alpha X_m}) |e^{-\nu(\lambda)X_i} c_i^+|
\end{align*}

All that remains is to write $e^{-\nu(\lambda)X_i} c_i^+$ in terms of $e^{-\nu(\lambda)X_i} c_i^-$ using the result from the previous section. Doing that, the terms involving the $c_i$ are given by

\begin{align*}
e^{\nu(\lambda) X_{i-1} } &c_{i-1}^- - e^{-\nu(\lambda)X_i} c_i^- + \mathcal{O}\Big( (|\lambda| + e^{-\alpha X_m})(|\tilde{c}_{i-1}^+| + |\tilde{c}_i^-|) + e^{-\tilde{\alpha}X_m} (|\lambda| + e^{-\tilde{\alpha}X_m})|\tilde{c}_{i+1}^-| \\
&+ e^{-\tilde{\alpha}X_i}|c_i^-|
+ e^{-\tilde{\alpha}X_m}( e^{-\alpha X_{i-1}}|c_{i-1}^-| + e^{-\alpha X_{i+1}}|c_{i+1}^-| ) + e^{-\tilde{\alpha}X_m}(|\lambda| + |D_i|)|d| \Big)  
\end{align*}

\item The center integral term will give us the center Melnikov integral

\begin{align*}
&\langle \Psi^c(0), P_i^-(0; \lambda) \int_{-X_{i-1}}^0 \Phi^c(0, y; \lambda) P_i^-(y; \lambda) \tilde{H}_i^-(y) dy \rangle \\
&= \langle \Psi^c(0), \int_{-X_{i-1}}^0 P_i^-(0; \lambda) \Phi^c(0, y; \lambda) P_i^-(y; \lambda)^{-1} \tilde{H}_i^-(y) dy \rangle \\
&= \langle \Psi^c(0), \int_{-X_{i-1}}^0 P^-(0) \Phi^c(0, y; 0) P^-(y)^{-1} \tilde{H}_i^-(y) dy \rangle + \mathcal{O}(|\lambda| + e^{-\alpha X_m}) \\
&= \int_{-X_{i-1}}^0 \langle \Psi^c(0), \Theta^c(0, y) \tilde{H}_i^-(y) \rangle dy + \mathcal{O}(|\lambda| + e^{-\alpha X_m}) \\
&= \int_{-X_{i-1}}^0 \langle \Theta^c(y, 0)^* \Psi^c(0), H(y) \rangle dy + \int_{-X_{i-1}}^0 \langle \Psi^c(0), \Theta^c(0, y) \Delta H_i^-(y) \rangle dy + \mathcal{O}(|\lambda| + e^{-\alpha X_m}) \\
&= \int_{-\infty}^0 \langle \Psi^c(y), H(y) \rangle dy + \int_{-X_{i-1}}^0 \langle \Psi^c(0), \Theta^c(0, y) \Delta H_i^-(y) \rangle dy + \mathcal{O}(e^{-\alpha X_m} + |\lambda|) \\
\end{align*}

For the integral involving $\Delta H_i^-(y)$,

\begin{align*}
\left| \int_{-X_{i-1}}^0 \langle \Psi^c(0), \Theta^c(0, y) \Delta H_i^-(y) \rangle dy \right| &\leq C \int_{-X_{i-1}}^0 e^{-\rho y} e^{-\alpha X_{i-1}} e^{-\alpha(X_{i-1} + y)} dy \\
&\leq C e^{-(\alpha - \rho)X_{i-1}} \int_{-X_{i-1}}^0 e^{-\alpha(X_{i-1} + y)} dy \\
&\leq C e^{-\tilde{\alpha}X_{i-1}}
\end{align*}

Thus we have

\begin{align*}
&\langle \Psi^c(0), P^-(0; \beta_i^\pm, \lambda) \int_{-X_{i-1}}^0 \Phi^c(0, y; \lambda) P^-(y; \beta_i^\pm, \lambda) \tilde{H}_i^-(y) dy \rangle \\
&= \int_{-\infty}^0 \langle \Psi^c(y), H(y) \rangle dy + \mathcal{O}(e^{-\tilde{\alpha} X_m} + |\lambda|) \\
\end{align*}

The ``positive'' integral is similar, and gives us the other half of the center Melnikov integral.

\end{enumerate}

The remaining terms are higher order. We will evaluate them in turn.

\begin{enumerate}

\item For the term involving $a$, we plug in $A_4$.

\begin{align*}
P_i^-(0; \lambda) \Phi^s(0, -X_{i-1}; \lambda) a_{i-1}^- = 
P_i^-(0; \lambda) \Phi^s(0, -X_{i-1}; \lambda) P_0^s(\lambda) D_i d +
P_i^-(0; \lambda) \Phi^s(0, -X_{i-1}; \lambda) A_4(\lambda)_{i-1}^-(c^-, \tilde{c}, d) 
\end{align*}

Since we are projecting on $Y^0$, we can write the first term on the RHS as

\begin{align*}
P_i^-(0; \lambda) \Phi^s(0, -X_{i-1}; \lambda) P_0^s(\lambda) D_i d
&= P_i^-(0; \lambda) \Phi^s(0, -X_{i-1}; 0) D_i d + \mathcal{O}(|\lambda| e^{-2 \alpha X_m})|d| ) \\
&= P^-(0) \Phi^s(0, -X_{i-1}; 0) D_i d + \mathcal{O}(e^{-2 \alpha X_m}(|\lambda| + e^{-\alpha X_m})|d|)
\end{align*}

Since the first term on the RHS is eliminated by $P(Y^0)$, we have

\[
P(Y^0) P_i^-(0; \lambda) \Phi^s(0, -X_{i-1}; \lambda) P_0^s(\lambda) D_i d = \mathcal{O}(e^{-2 \alpha X_m}(|\lambda| + e^{-\alpha X_m})|d| ) 
\]

Thus, combining this with the bound for $A_4$, we have

\begin{align*}
|P(Y^0) &P_i^-(0; \lambda) \Phi^s(0, -X_{i-1}; \lambda) a_{i-1}^-| \\
&\leq C\Big( 
e^{-2 \alpha X_m} (|\lambda| + e^{-\tilde{\alpha}X_m})(|\tilde{c}_{i-2}^+| + |\tilde{c}_i^-|) + e^{-\alpha X_m}( e^{-\tilde{\alpha}X_{i-2}}|c_{i-2}^-| + e^{-\tilde{\alpha}X_{i-1}}|c_{i-1}^-| + e^{-\tilde{\alpha}X_i}|c_i^-|) \\
&+ (e^{-(\alpha + \tilde{\alpha}) X_m} |\lambda|^2 + e^{-2 \alpha X_m}|D| + e^{-2 \alpha X_m}|\lambda|) |d| \Big)
\end{align*}

Similarly, we have

\begin{align*}
|P(Y^0) &P_i^+(0; \lambda) \Phi^u(0, X_i; \lambda) a_i^+| \\
&\leq C\Big( 
e^{-2 \alpha X_m} (|\lambda| + e^{-\tilde{\alpha}X_m})(|\tilde{c}_{i-1}^+| + |\tilde{c}_{i+1}^-|) + e^{-\alpha X_m}( e^{-\tilde{\alpha}X_{i-1}}|c_{i-1}^-| + e^{-\tilde{\alpha}X_i}|c_i^-| + e^{-\tilde{\alpha}X_{i+1}}|c_{i+1}^-|) \\
&+ (e^{-(\alpha + \tilde{\alpha}) X_m} |\lambda|^2 + e^{-2 \alpha X_m}|D| + e^{-2 \alpha X_m}|\lambda|) |d| \Big)
\end{align*}

\item For the terms involving $b$, note that the terms $P^-(0) b_i^-$ and $P^+(0)b_i^+$ are eliminated outright by the projection. For the other terms, we use the estimate for $B_1$.

\begin{align*}
&|(P_i^-(0; \lambda) - P^-(0))b_i^- + P_i^-(0; \lambda)(P_0^u(\lambda) - P_0^u(0))b_i^-| \\
&\leq C(|\lambda| + e^{-\alpha X_m}) |B_1(c, \tilde{c}, d)| \\
&\leq C(|\lambda| + e^{-\alpha X_m}) \Big( 
(|\lambda| + e^{-\tilde{\alpha}X_m})( |\tilde{c}_{i-1}^+| + |\tilde{c}_i^-|)
+ e^{-\tilde{\alpha} X_{i-1}} |c_{i-1}^-| + e^{-\tilde{\alpha} X_i} |c_i^-| + ( e^{-\tilde{\alpha}X_m} |D| + e^{-\tilde{\alpha}X_m}|\lambda| + |\lambda|^2)|d| \Big)
\end{align*}

\item For the noncenter integral terms involving $\tilde{H}$, we should be able to use the simple estimate

\begin{align*}
&\left| P^-(0; \beta_i^-, \lambda) 
\int_{-X_{i-1}}^0 \Phi^s(0, y; \lambda) \lambda^2 d_i P^-(y; \beta_i^-, \lambda)^{-1} \tilde{H}_i^-(y) dy \right| 
\leq C |\lambda|^2 |d| \int_{-X_{i-1}}^0 e^{\alpha y} e^{\alpha y} dy
&\leq C |\lambda|^2 |d|
\end{align*}
 
If we need, we can probably get a better estimate, since we are projecting on the center subspace $Y^0$ and the integral involves evolution in the stable subspace, but since we have other terms of similar order here, this hopefully is good enough.

\end{enumerate}

Putting all of this together, we obtain the center jump expressions

\begin{align*}
\xi^c_i = e^{-\nu(\lambda) X_i} c_i^- - e^{\nu(\lambda) X_{i-1}} c_{i-1}^- - \lambda^2 d_i M^c + R^c(\lambda)_i(c, \tilde{c}, d)
\end{align*}

where $M^c$ is the center Melnikov integral

\[
\int_{-\infty}^\infty \langle \Psi^c(y), H(y) \rangle dy 
\]

and the remainder term $R^c_i(c, \tilde{c}, d)$ has bound

\begin{align*}
R^c&(c, \tilde{c}, d)_i \leq C \Big( \\
&(|\lambda| + e^{-\alpha X_m})(|\tilde{c}_{i-1}^+| + |\tilde{c}_{i}^-|) + e^{-\alpha X_m}(|\lambda| + e^{-\alpha X_m})( |\tilde{c}_{i-2}^+| + |\tilde{c}_{i+1}^-|)  \\
&+ e^{-\alpha X_i} |c_i^-| + (|\lambda| + e^{-\tilde{\alpha} X_m})( e^{-\alpha X_{i-1}} |c_{i-1}^-| + e^{-\alpha X_{i-2}} |c_{i-2}^-| + e^{-\alpha X_{i+1}} |c_{i+1}^-|) \\
&+ (|\lambda| + e^{-\tilde{\alpha} X_m})(|\lambda| + |D|)|d|
\Big)
\end{align*}

This looks horrible, but we can actually get something nice out of it. Writing this in matrix form, we have

\[
(C_1 K(\lambda) + \tilde{C}_1) c + (-\lambda^2 M^c I + D_1)d = 0
\]

where

\begin{align*}
K(\lambda) =  
\begin{pmatrix}
e^{-\nu(\lambda)X_1} & & & & & -e^{\nu(\lambda)X_0} \\
-e^{\nu(\lambda)X_1} & e^{-\nu(\lambda)X_2} \\
& -e^{\nu(\lambda)X_2} & e^{-\nu(\lambda)X_3} \\
\vdots & & \vdots & &&  \vdots \\
& & & & -e^{\nu(\lambda)X_{n-1}} & e^{-\nu(\lambda)X_0} 
\end{pmatrix}
\end{align*}

and

\begin{align*}
C_1 &= I + \mathcal{O}(|\lambda| + e^{-\alpha X_m}) I 
+ \mathcal{O}(e^{-\alpha X_m}( |\lambda| + e^{-\alpha X_m}))\\
\tilde{C}_1 &= \mathcal{O}(e^{-\alpha X_m}) I + \mathcal{O}(e^{-\alpha X_m}(|\lambda| + e^{-\tilde{\alpha} X_m})) \\
D_1 &= \mathcal{O}(|\lambda| + e^{-\tilde{\alpha} X_m})^2
\end{align*}

where we used the fact that $|D| = \mathcal{O}(e^{-\alpha X^m})$. Since $C_1$ is a small perturbation of the identity matrix, it is invertible for sufficiently large $X_m$ (with operator norm approximately 1), therefore we can write this equation as


\[
(K(\lambda) + C_2 ) c + (-\lambda^2 M^c I + D_1)d = 0
\]

where

\[
C_2 = C_1^{-1} \tilde{C}_1 = \mathcal{O}(e^{-\alpha X_m})
\]

\subsubsection*{Jump in Psi direction}

For the second jump, we project on $\Psi(0)$.

\[
\xi_i = \langle \Psi(0), P_i^+(0; \lambda) Z_i^+(0) - P_i^-(0; \lambda) Z_i^-(0) \rangle
\]

Recall that the terms $P_i^\pm(0; \lambda) Z_i^\pm(0)$ are given by

\begin{align*}
P_i^-(0; \lambda) Z_i^-(0) &= P^-(0)( b_i^- + P_0^c(0) e^{\nu(\lambda) X_{i-1}} c_{i-1}^- ) \\
&+ P_i^-(0; \lambda) \Phi^s(0, -X_{i-1}; \lambda) a_{i-1}^- + (P_i^-(0; \lambda) - P^-(0))b_i^- + P_i^-(0; \lambda)(P_0^u(\lambda) - P_0^u(0))b_i^- \\
&+ (P_i^-(0; \lambda) - P^-(0)) P_0^c(0) e^{\nu(\lambda) X_{i-1}} c_{i-1}^- + P_i^-(0; \lambda) (P_0^c(\lambda) - P_0^c(0)) e^{\nu(\lambda) X_{i-1}} c_{i-1}^- \\
&+ \lambda^2 d_i P_i^-(0; \lambda) \int_{-X_{i-1}}^0 \Phi^s(0, y; \lambda) P_i^-(y; \lambda)^{-1} \tilde{H}_i^-(y) dy \\
&+ \lambda^2 d_i P_i^-(0; \lambda) \int_{-X_{i-1}}^0 \Phi^c(0, y; \lambda) P_i^-(y; \lambda)^{-1} \tilde{H}_i^-(y) dy  \\ 
P_i^+(0; \lambda) Z_i^+(0) &=  P^+(0)( b_i^+ + P_0^c(0) e^{-\nu(\lambda)X_i} c_i^+ )\\
&+ P_i^+(0; \lambda) \Phi^u(0, X_i; \lambda) a_i^+ + (P_i^+(0; \lambda) - P^+(0)) b_i^+ + P_i^+(0; \lambda) (P_0^s(\lambda) - P_0^s(0)) b_i^+ \\
&+ (P_i^+(0; \lambda) - P^+(0))P_0^c(0) e^{-\nu(\lambda)X_i} c_i^+ + P_i^+(0; \lambda) (P_0^c(\lambda) - P_0^c(0)) e^{-\nu(\lambda)X_i} c_i^+\\
&+ \lambda^2 d_i P_i^+(0; \lambda) \int_{X_i}^0 \Phi^u(0, y; \lambda) P_i^+(y; \lambda)^{-1} \tilde{H}_i^+(y) dy \\
&+ \lambda^2 d_i P_i^+(0; \lambda) \int_{X_i}^0 \Phi^c(0, y; \lambda) P_i^+(y; \lambda)^{-1} \tilde{H}_i^+(y) dy \\
\end{align*}

As with the first jump, we will compute the significant terms first.

\begin{enumerate}
\item The noncenter integral will give us the higher order Melnikov integral. For the ``minus'' piece, we have

\begin{align*}
&\langle \Psi(0), P_i^-(0; \lambda) \int_{-X_{i-1}}^0 \Phi^s(0, y; \lambda) P_i^-(y; \lambda)^{-1} \tilde{H}_i^-(y) dy \rangle \\
&= \int_{-X_{i-1}}^0 \langle \Psi(0), P_i^-(0; \lambda), \Phi^s(0, y; \lambda) P_i^-(y; \lambda)^{-1} \tilde{H}(y) \rangle dy \\
&= \int_{-X_{i-1}}^0 \langle \Psi(0), P_i^-(0; \lambda), \Phi^s(0, y; \lambda) P_i^-(y; \lambda)^{-1} H(y) \rangle dy + \mathcal{O}({e^{-\alpha X_m}})\\
&= \int_{-X_{i-1}}^0 \langle \Psi(0), \Theta(0, y) H(y) \rangle dy + \mathcal{O}(|\lambda| + {e^{-\alpha X_m}})\\
&= \int_{-X_{i-1}}^0 \langle \Theta(y, 0)^* \Psi_i(0), H(y) \rangle dy + \mathcal{O}(|\lambda| + {e^{-\alpha X_m}})\\
&= \int_{-X_{i-1}}^0 \langle \Psi(y), H(y) \rangle dy + \mathcal{O}(|\lambda| + {e^{-\alpha X_m}})\\
&= \int_{-\infty}^0 \langle \Psi(y), H(y) \rangle dy + \mathcal{O}(|\lambda| + {e^{-\alpha X_m}})\\
\end{align*}

The ``positive'' piece is similar, and gives us the other half of the Melnikov integral.

\item For the terms involving $a_i$, we plug in $A_4$.

\begin{align*}
\langle &\Psi(0), P_i^-(0; \lambda) \Phi^s(0, -X_{i-1}; \lambda) a_{i-1}^- \rangle \\
&= \langle \Psi_i(0), P_i^-(0; \lambda) \Phi^s(0, -X_{i-1}; \lambda) (- P_i^-(-X_{i-1}; \lambda)^{-1} P_0^s(\lambda) D_{i-1} d + A_4(\lambda)_{i-1}^-(c, \tilde{c}, d)) \rangle \\
&= -\langle \Psi(0), \Theta^s(0, -X_{i-1}) P_0^s(0) D_{i-1} d \rangle + \mathcal{O}( |\lambda|e^{-2 \alpha X_m} + e^{-\alpha X_{i-1}} |A_4(\lambda)_{i-1}^-(c, \tilde{c}, d)|)\\
&= -\langle \Theta^s(-X_{i-1}, 0)^* \Psi_i(0), P_0^s(0) D_{i-1} d \rangle + \mathcal{O}( |\lambda|e^{-2 \alpha X_m} + e^{-\alpha X_{i-1}} |A_4(\lambda)_{i-1}^-(c, \tilde{c}, d)|)\\
&= -\langle \Psi(-X_{i-1}), P_0^s(0) D_{i-1} d \rangle + \mathcal{O}\Big( |\lambda|e^{-2 \alpha X_m} + e^{-\alpha X_{i-1}} ( 
e^{-\alpha X_{i-1}}(|\lambda| + e^{-\tilde{\alpha}X_m})(|\tilde{c}_{i-2}^+| + |\tilde{c}_i^-|) \\
&+ e^{-\tilde{\alpha}X_{i-2}}|c_{i-2}^-| + e^{-\tilde{\alpha}X_{i-1}}|c_{i-1}^-| + e^{-\tilde{\alpha}X_i}|c_i^-| + e^{-\tilde{\alpha} X_m} |\lambda|^2|d| + e^{-\alpha X_m}|D||d|) \Big) \\
&= -\langle \Psi(-X_{i-1}), P_0^s(0) D_{i-1} d \rangle 
+ \mathcal{O}\Big(  
e^{-2 \alpha X_m}(|\lambda| + e^{-\tilde{\alpha}X_m})(|\tilde{c}_{i-2}^+| + |\tilde{c}_i^-|) \\
&+ e^{-\alpha X_m}( e^{-\tilde{\alpha}X_{i-2}}|c_{i-2}^-| + e^{-\tilde{\alpha}X_{i-1}}|c_{i-1}^-| + e^{-\tilde{\alpha}X_i}|c_i^-|) + e^{-(\alpha + \tilde{\alpha}) X_m} |\lambda|^2|d| + e^{-2 \alpha X_m}(|D| + |\lambda|)|d|) \Big) \\ 
\end{align*}

For the $a_i^+$ term, we have

\begin{align*}
\langle &\Psi(0), P_i^+(0; \lambda) \Phi^u(0, X_i; \lambda) a_i^+ \rangle \\
&= \langle \Psi(X_i), P_0^u(0) D_i d \rangle + \mathcal{O}\Big( e^{-2 \alpha X_m} (|\lambda| + e^{-\tilde{\alpha}X_m})(|\tilde{c}_{i-1}^+| + |\tilde{c}_{i+1}^-|) \\
&+ e^{-\alpha X_m}( e^{-\tilde{\alpha}X_{i-1}}|c_{i-1}^-| + e^{-\tilde{\alpha}X_i}|c_i^-| + e^{-\tilde{\alpha}X_{i+1}}|c_{i+1}^-|) e^{-(\alpha + \tilde{\alpha}) X_m} |\lambda|^2|d| + e^{-2 \alpha X_m}(|D| + |\lambda|)|d|) \Big)
\end{align*}

\end{enumerate}

The remaining terms will be higher order. Doing these in turn, we have

\begin{enumerate}
\item For the terms involving $b$, we first note that the terms $P^-(0) b_i^-$ and $P^+(0)b_i^+$ will vanish with the projection. For the remaining terms, we substitute the estimate for $B_1$.

\begin{align*}
&|\langle \Psi(0), (P_i^-(0; \lambda) - P^-(0))b_i^- + P_i^-(0; \lambda)(P_0^u(\lambda) - P_0^u(0))b_i^-| \\
&\leq C (|\lambda| + e^{-\alpha X_m})\Big( 
(|\lambda| + e^{-\tilde{\alpha}X_m})( |\tilde{c}_{i-1}^+| + |\tilde{c}_i^-|)\\
&+ e^{-\tilde{\alpha} X_{i-1}} |c_{i-1}^-| + e^{-\tilde{\alpha} X_i} |c_i^-| + ( e^{-\tilde{\alpha}X_m} |D| + e^{-\tilde{\alpha}X_m}|\lambda| + |\lambda|^2)|d| \Big)
\end{align*}

\item For the terms involving $c$, we first note that the terms $P_0^c(0) e^{\nu(\lambda) X_{i-1}} c_{i-1}^-$ and $P_0^c(0) e^{-\nu(\lambda)X_i} c_i^+$ will be eliminated outright by the projections. For the term involving $c_{i-1}^-$, we have

\begin{align*}
|(P_i^-(0; \lambda) - P^-(0)) P_0^c(0) e^{\nu(\lambda) X_{i-1}} c_{i-1}^- + P_i^-(0; \lambda) (P_0^c(\lambda) - P_0^c(0)) e^{\nu(\lambda) X_{i-1}} c_{i-1}^-| \leq C (|\lambda| + e^{-\alpha X_m})|\tilde{c}_{i-1}^+|
\end{align*}

For the term involving $c_{i-1}^-$, we also have to use our expression to convert $e^{-\nu(\lambda)X_i} c_i^+$ to $e^{-\nu(\lambda)X_i} c_i^-$. Doing this, we have

\begin{align*}
&|(P_i^+(0; \lambda) - P^+(0))P_0^c(0) e^{-\nu(\lambda)X_i} c_i^+ + P_i^+(0; \lambda) (P_0^c(\lambda) - P_0^c(0)) e^{-\nu(\lambda)X_i} c_i^+| \\
&\leq C(|\lambda| + e^{-\alpha X_m})\Big( |\tilde{c}_i^-| + e^{-\tilde{\alpha}X_m} (|\lambda| + e^{-\tilde{\alpha}X_m})( |\tilde{c}_{i-1}^+| + |\tilde{c}_{i+1}^-|) 
+ e^{-\tilde{\alpha}X_i}|c_i^-| \\
&+ e^{-\tilde{\alpha}X_m}( e^{-\alpha X_{i-1}}|c_{i-1}^-| + e^{-\alpha X_{i+1}}|c_{i+1}^-| ) + e^{-\tilde{\alpha}X_m}(|\lambda| + |D_i|)|d| \Big) 
\end{align*}

\item For the center integral term, we have

\begin{align*}
&\langle \Psi(0), P_i^-(0; \lambda)
\int_{-X_{i-1}}^0 \Phi^c(0, y; \lambda) P_i^-(y; \lambda)^{-1} \tilde{H}_i^-(y) dy \rangle \\
&= \int_{-X_{i-1}}^0 \langle \Psi(0), P_i^-(0; \lambda) \Phi^c(0, y; \lambda) P_i^-(y; \lambda)^{-1} \tilde{H}_i^-(y) \rangle dy \\
&= \int_{-X_{i-1}}^0 \langle \Psi(0), P^-(0) \Phi^c(0, y; 0) P^-(y)^{-1} \tilde{H}_i^-(y) \rangle dy + \mathcal{O}(|\lambda| + e^{-\alpha X_m}) \\
&= \mathcal{O}(|\lambda| + e^{-\alpha X_m})
\end{align*}

where the integral vanishes since the RHS of the inner product is an element of $Y^0$, which we have chosen perpendicular to $\Psi(0)$.

\end{enumerate}

Putting this all together, we have

\begin{align*}
\xi_i = \langle \Psi(X_i), P_0^u(0) D_i d \rangle
+ \langle \Psi(-X_{i-1}), P_0^s(0) D_{i-1} d \rangle 
- \lambda_2 d_i M + R_i(\lambda)(c, \tilde{c}, d)
\end{align*}

where $M$ is the higher order Melnikov integral

\[
\int_{-\infty}^\infty \langle \Psi(y), H(y) \rangle dy 
\]

and the remainder term has bound

\begin{align*}
|R(\lambda)&(\tilde{c}, d)| \leq C \Big( \\
&(|\lambda| + e^{-\alpha X_m})(|\tilde{c}_{i-1}^+| + |\tilde{c}_{i}^-|) + (|\lambda| + e^{-\tilde{\alpha} X_m})(e^{-\alpha X_m} + |\lambda|) ( |\tilde{c}_{i-2}^+| + |\tilde{c}_{i+1}^-|)  \\
&+ (|\lambda| + e^{-\tilde{\alpha} X_m})( e^{-\alpha X_{i-2}} |c_{i-2}^-| + e^{-\alpha X_{i-1}} |c_{i-1}^-| + e^{-\alpha X_i} |c_i^-| + e^{-\alpha X_{i+1}} |c_{i+1}^-|) \\
&+ (|\lambda| + e^{-\tilde{\alpha} X_m})|\lambda|^2 + e^{-2 \alpha X_m}(|\lambda| + |D|)|d| \Big)
\end{align*}

Using the fact that $|D| = \mathcal{O}(e^{-\alpha X_m})$, this becomes

\begin{align*}
|R(\lambda)&(\tilde{c}, d)| \leq C \Big( \\
&(|\lambda| + e^{-\alpha X_m})(|\tilde{c}_{i-1}^+| + |\tilde{c}_{i}^-|) + (|\lambda| + e^{-\tilde{\alpha} X_m})(e^{-\alpha X_m} + |\lambda|) ( |\tilde{c}_{i-2}^+| + |\tilde{c}_{i+1}^-|)  \\
&+ (|\lambda| + e^{-\tilde{\alpha} X_m})( e^{-\alpha X_{i-2}} |c_{i-2}^-| + e^{-\alpha X_{i-1}} |c_{i-1}^-| + e^{-\alpha X_i} |c_i^-| + e^{-\alpha X_{i+1}} |c_{i+1}^-|) \\
&+ (|\lambda| + e^{-\tilde{\alpha} X_m})(|\lambda| + e^{-\alpha X_m})^2 |d| \Big)
\end{align*}

Before we write this in matrix form, we substitute for $D_i d$. Recalling the expression for $D_i$, we have

\begin{align*}
\langle \Psi(X_i), P_0^u(0) D_i d \rangle
&= \langle \Psi(X_i), P_0^u(0) (Q'(X_i) + Q'(-X_i)) \rangle
+\mathcal{O}(e^{-2 \alpha X_i}(|\lambda| + e^{-\alpha X_i})) \\
&= \langle \Psi(X_i), Q'(X_i) + Q'(-X_i) \rangle
+\mathcal{O}(e^{-2 \alpha X_i}(|\lambda| + e^{-\alpha X_i})) \\
&= \langle \Psi(X_i), Q'(-X_i) \rangle
+\mathcal{O}(e^{-2 \alpha X_i}(|\lambda| + e^{-\alpha X_i})) 
\end{align*}

since $\langle \Psi(X_i), Q'(X_i) \rangle = 0$. Similarly, 

\begin{align*}
\langle \Psi(-X_i), P_0^s(0) D_i d \rangle
&= \langle \Psi(-X_i), Q'(X_i) \rangle
+\mathcal{O}(e^{-2 \alpha X_i}(|\lambda| + e^{-\alpha X_i})) 
\end{align*}

Since these remainder terms are already included in our remainder term, we obtain the final jump expressions

\begin{align*}
\xi_i = \langle \Psi(X_i), P_0^u(0) D_i d \rangle
+ \langle \Psi(-X_{i-1}), P_0^s(0) D_{i-1} d \rangle 
- \lambda_2 d_i M + R_i(\lambda)(c, \tilde{c}, d)
\end{align*}

where $R_i(\lambda)(c, \tilde{c}, d)$ has the same remainder as above. We can write this in matrix form as

\[
(C_3 K(\lambda) + C_4) c_i^- + (A -\lambda^2 M I + D_2)d = 0
\]

where the matrix $A$ is given by

\begin{align*}
A &= \begin{pmatrix}
-a_0 + \tilde{a}_1 & a_0 - \tilde{a}_1 \\
-\tilde{a}_0 + a_1 & \tilde{a}_0 - a_1
\end{pmatrix} && n = 2 \\
A &= \begin{pmatrix}
\tilde{a}_{n-1} - a_0 & a_0 & & & \dots & -\tilde{a}_{n-1}\\
-\tilde{a}_0 & \tilde{a}_0 - a_1 &  a_1 \\
& -\tilde{a}_1 & \tilde{a}_1 - a_2 &  a_2 \\
& & \vdots & & \vdots \\
a_{n-1} & & & & -\tilde{a}_{n-2} & \tilde{a}_{n-2} - a_{n-1} \\
\end{pmatrix} && n > 2
\end{align*}

where

\begin{align*}
a_i &= \langle \Psi(X_i), Q'(-X_i) \rangle \\
\tilde{a}_i &= \langle \Psi(-X_i), Q'(X_i) \rangle
\end{align*}

$M$ is the higher order Melnikov integral

\[
M = \int_{-\infty}^\infty \langle \Psi(y), H(y) \rangle dy
\]

and we have bounds

\begin{align*}
C_3 &= \mathcal{O}(|\lambda| + e^{-\alpha X_m}) I
+ \mathcal{O}(|\lambda| + e^{-\tilde{\alpha} X_m})( |\lambda| + e^{-\alpha X_m})\\
C_4 &= \mathcal{O}(e^{-\alpha X_m}(|\lambda| + e^{-\tilde{\alpha} X_m})) \\
D_2 &= \mathcal{O}((|\lambda| + e^{-\tilde{\alpha} X_m})(|\lambda| + e^{-\alpha X_m})^2)
\end{align*}

We state the final result combining the two jump expressions in the following theorem.

% theorem : block diagonal matrix expression
\pagebreak

\begin{theorem}\label{blockmatrixform}

Let $q_n(x)$ be a periodic $n-$pulse solution constructed with lengths $X_0, \dots, X_{n-1}$. Then the jump conditions can be written as the block diagonal matrix equation 

\begin{equation}\label{blockdiag}
\begin{pmatrix}
K(\lambda) + C_2 & -\lambda^2 M^c I + D_1 \\
C_3 K(\lambda) + C_4 & A - \lambda^2 MI + D_2
\end{pmatrix}
\begin{pmatrix}c \\ d \end{pmatrix} 
= 0
\end{equation}

where 

\begin{enumerate}

\item The remainder terms have bounds

\begin{align*}
C_2 &= \mathcal{O}(e^{-\alpha X_m}) \\
C_3 &= \mathcal{O}(|\lambda| + e^{-\alpha X_m}) I
+ \mathcal{O}((|\lambda| + e^{-\tilde{\alpha} X_m})( |\lambda| + e^{-\alpha X_m}))\\
C_4 &= \mathcal{O}(e^{-\alpha X_m}(|\lambda| + e^{-\tilde{\alpha} X_m})) \\
D_1 &= \mathcal{O}((|\lambda| + e^{-\tilde{\alpha} X_m})^2) \\
D_2 &= \mathcal{O}((|\lambda| + e^{-\tilde{\alpha} X_m})(|\lambda| + e^{-\alpha X_m})^2) 
\end{align*}

where $X_m = \min \{X_0, \dots, X_{n-1}\}$

\item $M$, $M^c$ are the Melnikov integrals

\begin{align}
M &= \int_{-\infty}^\infty \langle \Psi(y), H(y) \rangle dy \\
M^c &= \int_{-\infty}^\infty \langle \Psi^c(y), H(y) \rangle dy
\end{align}

For KdV5, these are given by

\begin{align}
M &= \int_{-\infty}^\infty q(y) q_c(y) dy \\
M^c &= \int_{-\infty}^\infty q_c(y) dy
\end{align}

\item The matrix $K(\lambda)$ is given by

\begin{equation}
K(\lambda) = 
\begin{pmatrix}
e^{-\nu(\lambda)X_1} & & & & & -e^{\nu(\lambda)X_0} \\
-e^{\nu(\lambda)X_1} & e^{-\nu(\lambda)X_2} \\
& -e^{\nu(\lambda)X_2} & e^{-\nu(\lambda)X_3} \\
\vdots & & \vdots & &&  \vdots \\
& & & & -e^{\nu(\lambda)X_{n-1}} & e^{-\nu(\lambda)X_0} 
\end{pmatrix}
\end{equation}

where $\nu(\lambda)$ is the small eigenvalue of the asympotic matrix $A(\lambda)$.

\item The matrix $A$ is given by

\begin{align*}
A &= \begin{pmatrix}
-a_0 + \tilde{a}_1 & a_0 - \tilde{a}_1 \\
-\tilde{a}_0 + a_1 & \tilde{a}_0 - a_1
\end{pmatrix} && n = 2 \\
A &= \begin{pmatrix}
\tilde{a}_{n-1} - a_0 & a_0 & & & \dots & -\tilde{a}_{n-1}\\
-\tilde{a}_0 & \tilde{a}_0 - a_1 &  a_1 \\
& -\tilde{a}_1 & \tilde{a}_1 - a_2 &  a_2 \\
& & \vdots & & \vdots \\
a_{n-1} & & & & -\tilde{a}_{n-2} & \tilde{a}_{n-2} - a_{n-1} \\
\end{pmatrix} && n > 2
\end{align*}

where

\begin{align*}
a_i &= \langle \Psi(X_i), Q'(-X_i) \rangle \\
\tilde{a}_i &= \langle \Psi(-X_i), Q'(X_i) \rangle
\end{align*}

For KdV5, we have $\tilde{a}_i = a_i$.

\end{enumerate}

To leading order, equation \eqref{blockdiag} is
\begin{equation}
\begin{pmatrix}
K(\lambda) & -\lambda^2 M^c I  \\
0 & A - \lambda^2 MI 
\end{pmatrix}
\begin{pmatrix}c \\ d \end{pmatrix} = 0
\end{equation}

Thus, to leading order, the eigenvalue problem has a nontrivial solution if either of the following conditions holds.

\begin{enumerate}[(i)]
\item $\nu(\lambda) = i \dfrac{n \pi}{X}, n \in \Z$ 
\item $\det(A - \lambda^2 MI) = 0$
\end{enumerate}

where $X = X_0 + \dots + X_{n-1}$ is half the length of the domain. The first condition gives us the essential spectrum, and the second condition gives us the point spectrum. Note that $M^c$ does not appear in these conditions.\\

\end{theorem}

\end{document}