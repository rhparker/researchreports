\documentclass[12pt]{article}
\usepackage[pdfborder={0 0 0.5 [3 2]}]{hyperref}%
\usepackage[left=1in,right=1in,top=1in,bottom=1in]{geometry}%
\usepackage[shortalphabetic]{amsrefs}%
\usepackage{amsmath}
\usepackage{enumerate}
% \usepackage{enumitem}
\usepackage{amssymb}                
\usepackage{amsmath}                
\usepackage{amsfonts}
\usepackage{amsthm}
\usepackage{bbm}
\usepackage[table,xcdraw]{xcolor}
\usepackage{tikz}
\usepackage{float}
\usepackage{booktabs}
\usepackage{svg}
\usepackage{mathtools}
\usepackage{cool}
\usepackage{url}
\usepackage{graphicx,epsfig}
\usepackage{makecell}
\usepackage{array}

\def\noi{\noindent}
\def\T{{\mathbb T}}
\def\R{{\mathbb R}}
\def\N{{\mathbb N}}
\def\C{{\mathbb C}}
\def\Z{{\mathbb Z}}
\def\P{{\mathbb P}}
\def\E{{\mathbb E}}
\def\Q{\mathbb{Q}}
\def\ind{{\mathbb I}}

\DeclareMathOperator{\spn}{span}
\DeclareMathOperator{\ran}{ran}

\graphicspath{ {periodic/} }

\newtheorem{lemma}{Lemma}
\newtheorem{theorem}{Theorem}
\newtheorem{corollary}{Corollary}
\newtheorem{definition}{Definition}
\newtheorem{assumption}{Assumption}
\newtheorem{hypothesis}{Hypothesis}

\newtheorem{notation}{Notation}

\begin{document}

\subsection*{Conjugation}

I am curious if we can use the Conjugation Lemma to simplify things. Here is a version of the Conjugation Lemma, which is adapted and cleaned up from Zum2018. 

\begin{lemma}[Conjugation Lemma]
Let $W \in \C^N$, and consider the family of ODEs on $\R$

\begin{equation}\label{LambdaEVPconj}
W(x)' = A(x; \Lambda) W(x) + B(x) W(x) + F(x) 
\end{equation}

where $\Lambda \in \Omega \subset C^m$ is a parameter vector. Take the same assumptions as in the Gap Lemma, i.e. 

\begin{enumerate}
	\item The map $\Lambda \mapsto A(\cdot; \Lambda)$ is analytic in $\Lambda$.
	\item $A(x; \Lambda) \rightarrow A_\pm(\lambda)$ (independent of $\Lambda$) as $x \rightarrow \pm \infty$, and for $|\Lambda| < \delta$ we have the uniform exponential decay estimates 
	\begin{align}
	\left| \frac{\partial^k}{\partial x^k} A(x; \Lambda) - A_\pm(\Lambda) \right| 
	&\leq C e^{-\theta |x|} && 0 \leq k \leq K
	\end{align}
	where $\alpha > 0$, $C > 0$, and $K$ is a nonnegative integer.
\end{enumerate}

Then in a neighborhood of any $\Lambda_0 \in \Omega$ there exist invertible linear transformations

\begin{align*}
P_+(x, \Lambda) &= I + \Theta_+(x, \Lambda) \\
P_-(x, \Lambda) &= I + \Theta_-(x, \Lambda) 
\end{align*}

defined on $\R^+$ and $\R^-$, respectively, such that

\begin{enumerate}[(i)]
\item The change of coordinates $W = P_\pm Z$ reduces \eqref{LambdaEVPconj} to the equations on $\R^\pm$

\begin{align}
Z'(x) = A^\pm(\Lambda) Z(x) + P_\pm(x, \Lambda)^{-1} B(x) P_\pm(x, \Lambda) Z(x) + P_\pm(x, \Lambda)^{-1} F(x)
\end{align}

where

\[
G(x; \Lambda) = P_\pm(x, \Lambda)^{-1} F(x)
\]

\item For any fixed $0 < \tilde{\theta} < \theta$, $0 \leq k \leq K+1$, and $j \geq 0$ we have the decay rates
\begin{align*}
\left| \partial_\Lambda^j \partial_x^k \Theta_\pm \right| \leq C(j, k)e^{-\tilde{\theta}|x|}
\end{align*}
\end{enumerate}
\begin{proof}
I have written out the proof somewhere else for the case where $B(x) = 0$ and $F(x) = 0$, which essentially follows Zum2018 but fills in more details. Only a small modification is necessary for the general case.
\end{proof}
\end{lemma}

\subsection*{The Problem}

We want to do Lin's method on the eigenvalue problem for KdV5 with periodic BCs. (Eventually, we may try this for the case on the real line). Recall that from the existence problem, we write the $n-$pulse piecewise on the appropriate intervals as

\[
(q_n)_i^\pm(x) = q^\pm(x; \beta_i^\pm) + u_i^\pm(x)
\]

The functions $q^\pm(x; \beta_i^\pm)$ evolve in the stable/unstable manifolds with initial conditions $\beta_i^\pm$. The functions $u_i^\pm(x)$ are small remainder terms. We have bounds on all of these from the existence problem.

\begin{align*}
|q^\pm(x; \beta_i^\pm)| &\leq C |\beta_i^\pm| e^{-\alpha |x|} \\
|\beta_i^\pm| &\leq C (e^{-2 \alpha X_i} + e^{-2 \alpha X_{i-1}}) \\
|u_i^-(x)| &\leq C e^{-\alpha X_{i-1}} e^{-\alpha(X_{i-1} + x) } \\
|u_i^+(x)| &\leq C e^{-\alpha X_i} e^{-\alpha(X_i - x) } 
\end{align*}

The piecewise eigenvalue problem we need to solve is then

\begin{align*}
&(W_i^\pm)' = A(x; \beta_i^\pm, \lambda ) W_i^\pm + \tilde{G}_i^\pm(x) W_i^\pm + \lambda^2 d_i \tilde{H}_i^\pm \\
&W_i^\pm(0) \in \C \Psi(0) \oplus Y^+ \oplus Y^- \oplus Y^0 \\
&W_i^-(0) = W_i^+(0) \\
&W_i^+(X_i) - W_{i+1}^-(-X_i) = D_i d
\end{align*}

where

\begin{align*}
A^\pm(x; \lambda, \beta_i^\pm) &= \begin{pmatrix}0 & 1 & 0 & 0 & 0 \\0 & 0 & 1 & 0 & 0 \\0 & 0 & 0 & 1 & 0 \\0 & 0 & 0 & 0 & 1 \\
2 \partial_x q^\pm(x; \beta_i^\pm) + \lambda & 2 q^\pm(x; \beta_i^\pm) - c & 0 & 1 & 0 \end{pmatrix} \\
D_i d &= d_{i+1}[(Q_n)_{i+1}'(-X_i) + \lambda \partial_c (Q_n)_{i+1}(-X_i)]
- d_i [ (Q_n)_i'(X_i) + \lambda \partial_c (Q_n)_i(-X_i) ] \\
\end{align*}

and we have nice estimates/bounds

\begin{align*}
|H(x)|, |\tilde{H}_i^\pm(x)| &\leq C e^{-\alpha |x|} \\
|\Delta H_i^\pm| &= |\tilde{H}_i^\pm - H| \leq C(e^{-\alpha X_i} + e^{-\alpha X_{i-1}} ) \\
|\tilde{G}_i^-(x)|, |\Delta H_i^-(x)| &\leq C e^{-\alpha X_{i-1}} e^{-\alpha(X_{i-1} + x) } \\
|\tilde{G}_i^+(x)|, |\Delta H_i^+(x)| &\leq C e^{-\alpha X_i} e^{-\alpha(X_i - x) } \\
D_i d &= ( Q'(X_i) + Q'(-X_i))(d_{i+1} - d_i ) + \mathcal{O} \left( e^{-\alpha X_i} \left( |\lambda| +  e^{-\alpha X_i}  \right) |d| \right) \\
\end{align*}

We want to use the Conjugation Lemma so that the evolution of the variational equation does not depend on the parameters $\lambda$ or $\beta_i^\pm$. We take $\Lambda = (\beta_i^\pm, \lambda)$ in the conjugation lemma. $A(x, \beta_i^\pm; \lambda)$ is linear, thus analytic, in $\lambda$. It should also depend nicely on the initial conditions $\beta_i^\pm$. We also note that for all parameter values, $A^\pm(x, \beta_i^\pm; \lambda)$ decays exponentially to $A(\lambda)$, which is the constant-coefficient, $\lambda-$dependent matrix 

\begin{align*}
A(\lambda) &=  \begin{pmatrix}0 & 1 & 0 & 0 & 0 \\0 & 0 & 1 & 0 & 0 \\0 & 0 & 0 & 1 & 0 \\0 & 0 & 0 & 0 & 1 \\
\lambda & -c & 0 & 1 & 0 \end{pmatrix}
\end{align*}

Using the Conjugation Lemma, make the substitution $W_i^\pm = P^\pm(x; \beta_i^\pm, \lambda) Z_i^\pm$, where $P^\pm(x; \beta_i^\pm, \lambda)$ conjugates $A_i^\pm(x; \lambda, \beta_i^\pm)$. Then the eigenvalue problem becomes

\begin{align*}
&(Z_i^\pm(x))' = A(\lambda) Z_i^\pm(x) + P^\pm(x; \beta_i^\pm, \lambda)^{-1} \tilde{G}_i^\pm P^\pm(x; \beta_i^\pm, \lambda) Z_i^\pm(x) + \lambda^2 d_i P^\pm(x; \beta_i^\pm, \lambda)^{-1} \tilde{H}_i^\pm(x) \\
&P^-(0; \beta_i^-, \lambda) Z_i^-(0) = P^+(0; \beta_i^+, \lambda) Z_i^+(0) \\
&P^\pm(0; \beta_i^\pm, \lambda) Z_i^\pm(0) \in \C \Psi(0) \oplus Y^+ \oplus Y^- \oplus Y^0 \\
&P^+(X_i; \beta_i^+, \lambda) Z_i^+(X_i)\ - P^-(-X_i; \beta_{i+1}^-, \lambda) Z_{i+1}^-(-X_i; \lambda) = D_i d
\end{align*}

Since the $P^\pm(x; \lambda)^{-1} G_i^\pm P^\pm(x; \lambda)$ is always a remainder term, let

\[
G_i^\pm(x) = P^\pm(x; \lambda)^{-1} G_i^\pm(x) P^\pm(x; \lambda)
\]

Then the eigenvalue problem becomes

\begin{align*}
&(Z_i^\pm(x))' = A(\lambda) Z_i^\pm(x) + G_i^\pm Z_i^\pm(x) + \lambda^2 d_i P^\pm(x; \lambda)^{-1} \tilde{H}_i^\pm(x) \\
&P^-(0; \lambda) Z_i^-(0) = P^+(0; \lambda) Z_i^+(0) \\
&P^\pm(0; \lambda) Z_i^\pm(0) \in \C \Psi(0) \oplus Y^+ \oplus Y^- \oplus Y^0 \\
&P^+(X_i; \lambda) Z_i^+(X_i)\ - P^-(-X_i; \lambda) Z_{i+1}^-(-X_i; \lambda) = D_i d
\end{align*}

Since $A(\lambda)$ is constant coefficient (and we know exactly what it is), we don't have to bother with the exponential trichotomy stuff, since everything evolves in the eigenspaces of $A(\lambda)$, which do not depend on $x$. Before we proceed, make the following hypothesis.

\begin{hypothesis}\label{Aspectrumhyp}
\begin{enumerate}
	\item The spectrum of $A(0)$ has isolated, simple eigenvalues at $\{ 0, \pm \alpha_0 \pm \beta_0 \}$, where $\alpha_0, \beta_0 > 0$. The real part of any other eigenvalue of $A(0)$ lies outside the interval $[-\alpha_0, \alpha_0]$.
\end{enumerate}
\end{hypothesis}

We then define the following.

\begin{enumerate}
	\item Let

	\begin{align*}
	X_m &= \min(X_0, \dots, X_{n-1}) \\
	X_M &= \max(X_0, \dots, X_{n-1}) \\
	\end{align*}

	\item Let $n_-$ be the number of eigenvalues of $A(0; 0)$ with negative real part, and $n_+$ be the number of eigenvalues of $A(0; 0)$ with positive real part. Then, by Hypothesis \ref{Aspectrumhyp}, $n_-, n_+ \geq 2$, and $n_- + n_+ + 1 = m$.

	\item Let $\nu(\lambda)$ be the simple, small eigenvalue of $A(0; \lambda)$. By Hypothesis \ref{Aspectrumhyp}, $\nu(0) = 0$, and $\nu(\lambda) = \mathcal{O}(\lambda)$. 

	\item Let $\rho > 0$, $\delta > 0$ be a small. How small will be determined later. We will take $|\lambda| < \delta$.

	\item Let $\alpha = \alpha_0 - \rho$. We choose $\delta$ sufficiently small so that for all $|\lambda| < \delta$,

	\begin{enumerate}
		\item $|\nu(\lambda)| < \rho$
		\item The real part of any other eigenvalue of $A(0; \lambda)$ lies outside the interval $[-\alpha, \alpha]$.
	\end{enumerate}

	\item Let $\tilde{\alpha} = \alpha - 2 \rho > 0$.

	\item Choose $X_m$ sufficiently large so that
	\begin{equation}
	e^{-\tilde{\alpha} X_m}, |\lambda|, ||\Delta H|| < \delta
	\end{equation}

\end{enumerate}

Let $P^{u/s/c}_0(\lambda)$ be the eigenprojections for the unstable/stable/center subspaces $E^{u/s/c}(\lambda)$ of $A(\lambda)$. The center subspace is a legit center subspace only when $\nu(\lambda)$ has no real part, e.g. when $\lambda = 0$, but we will always call it a center subspace for convenience. Let $\Phi(x, y; \lambda) = e^{A(\lambda)(x-y)}$ be the evolution of the constant-coefficient ODE

\[
Z' = A(\lambda) Z
\]

Let $\Phi^{u/s/c}(x, y; \lambda)$ be the evolutions on the respective eigenspaces. Since $E^c(\lambda)$ is one-dimensional, we have in particular that

\begin{align*}
\Phi^c(x, y; \lambda) v &= e^{\nu(\lambda)(x - y)} v && v \in E^c(\lambda)
\end{align*}

We also have the bounds

\begin{align*}
|\Phi^s(x, y; \lambda)| &\leq C e^{-\alpha(x - y)} \\
|\Phi^u(x, y; \lambda)| &\leq C e^{-\alpha(y - x)} \\
|\Phi^c(x, y; \lambda)| &\leq C e^{\rho|x - y|} 
\end{align*}

We are going to want to relate things to the variational/adjoint variational problem for the nontransformed system. The only one of these we actually will need is the variational equation for the linearization about the primary pulse.

\begin{align*}
V_i' &= A(q(x)) V_i \\
W_i' &= -A(q(x))^* W_i
\end{align*}

Recall that $Q'(x)$ solves the variational problem, and we have solutions $\Psi(x)$ and $1$ to the adjoint variational problem. To get the primary pulse, we take $\beta_i^\pm = 0$, thus is it conjugated by $P_i^\pm(x; 0, 0)$. Let $\Theta(y, x)$ be the evolution for the untransformed variational equation. We would like to relate this to the evolution of the transformed equation. We should have for appropriate values of $x, y$

\[
\Theta(y, x) = P^\pm(y; 0, 0) \Phi(y, x; 0) P^\pm(x; 0, 0)^{-1}
\]

\subsection*{The Inversion}

Define the spaces

\begin{align*}
V_Z &= \bigoplus_{i=0}^{n-1} C^0([-X_{i-1}, 0], \C^m) \oplus C^0([0, X_i], \C^m) \\
V_a &= \bigoplus_{i=0}^{n-1} E^u(\lambda) \oplus E^s(\lambda) \\
V_b &= \bigoplus_{i=0}^{n-1} E^u(0) \oplus E^s(0) \\
V_c^+ &= \bigoplus_{i=0}^{n-1} E^c(\lambda) \\
V_c^- &= \bigoplus_{i=0}^{n-1} E^c(\lambda) \\
V_c &= V_c^+ \oplus V_c^- \\
V_\lambda &= B_\delta(0) \subset \C
\end{align*}

where the subscripts are all $\mod n$, as in the existence problem. We use the $\lambda-$dependent eigenspaces for $a_i^\pm$ and $c_i^\pm$, since we will be evolving them under the $\lambda-$dependent evolution. All the product spaces are endowed with the maximum norm, e.g. for $V_c$, $|c| = \max(|c_0^-|, \dots, |c_{n-1}^-|, |c_0^+|, \dots, |c_{n-1}^+|)$. In addition, we take the following convention: if we eliminate either a subscript or a superscript (or both) in the norm, we are taking the maximum over the eliminated thing. For example,
\begin{enumerate}
	\item $|c_i| = \max(|c_i^+|, |c_i^-|)$ 
	\item $|c^+| = \max(|c_0^+|, \dots, |c_{n-1}^+|)$
\end{enumerate}

Next, we write down the fixed point equations for the problem. For $i = 0, \dots, n-1$, the fixed point equations are

\begin{align*}
Z_i^-(x) &= \Phi^s(x, -X_{i-1}; \lambda) a_{i-1}^- + \Phi^u(x, 0; \lambda) b_i^- + \Phi^c(x, -X_{i-1}; \lambda) c_{i-1}^- \\
&+ \int_0^x \Phi^u(x, y; \lambda) [G_i^-(y)Z_i^-(y) + \lambda^2 d_i P^-(y; \beta_i^-, \lambda)^{-1} \tilde{H}_i^-(y)] dy \\
&+ \int_{-X_{i-1}}^x \Phi^s(x, y; \lambda) [G_i^-(y)Z_i^-(y) + \lambda^2 d_i P^-(y; \beta_i^-, \lambda)^{-1} \tilde{H}_i^-(y)] dy \\
&+ \int_{-X_{i-1}}^x \Phi^c(x, y; \lambda) [G_i^-(y)Z_i^-(y) + \lambda^2 d_i P^-(y; \beta_i^-, \lambda)^{-1} \tilde{H}_i^-(y)] dy  \\ 
Z_i^+(x) &= \Phi^u(x, X_i; \lambda) a_i^+ + \Phi^s(x, 0; \lambda) b_i^+ + \Phi^c(x, X_i; \lambda) c_i^+ \\
&+ \int_0^x \Phi^s(x, y; \lambda) [G_i^+(y)Z_i^+(y) + \lambda^2 d_i P^+(y; \beta_i^+, \lambda)^{-1} \tilde{H}_i^+(y) ] dy \\
&+ \int_{X_i}^x \Phi^u(x, y; \lambda) [G_i^+(y)Z_i^+(y) + \lambda^2 d_i P^+(y; \beta_i^+, \lambda)^{-1} \tilde{H}_i^+(y) ] dy \\
&+ \int_{X_i}^x \Phi^c(x, y; \lambda) [G_i^+(y)Z_i^+(y) + \lambda^2 d_i P^+(y; \beta_i^+, \lambda)^{-1} \tilde{H}_i^+(y) ] dy \\
\end{align*}

which is the same as in San98, except for the presence of a center evolution.

\subsubsection*{Solving for Z}

The first step is to solve for $Z_i^\pm$ in terms of $(a, b, c, d)$.\\

First, we obtain a bound on the terms from the fixed point equations which involve $Z$. 

% Z inversion, lemma 1 (L1 bound)

\begin{lemma}\label{L1}

Let $L_1(\lambda): V_Z \rightarrow V_Z$ be the linear operator defined piecewise by

\begin{align*}
(L_1(\lambda)Z)_i^-(x) &= \int_0^x \Phi^u(x, y; \lambda) G_i^-(y)Z_i^-(y) dy \\
&+ \int_{-X_{i-1}}^x \Phi^s(x, y; \lambda) G_i^-(y)Z_i^-(y) dy 
+ \int_{-X_{i-1}}^x \Phi^c(x, y; \lambda) G_i^-(y)Z_i^-(y) dy \\ 
(L_1(\lambda)Z)_i^+(x) &= \int_0^x \Phi^s(x, y; \lambda) G_i^+(y)Z_i^+(y) dy \\
&+ \int_{X_i}^x \Phi^u(x, y; \lambda) G_i^+(y)Z_i^+(y) dy
+ \int_{X_i}^x \Phi^c(x, y; \lambda) G_i^+(y)Z_i^+(y) dy 
\end{align*}

Then $L_1(\lambda): V_Z \rightarrow V_Z$ is a bounded linear operator with uniform bound

\begin{equation}\label{L1bound2}
||L_1(\lambda)Z|| \leq C e^{-(\alpha - \rho )X_m} ||Z||
\end{equation}

Piecewise bounds are

\begin{align*}
||L_1(\lambda)_i^- Z_i^-|| &\leq C e^{-(\alpha - \rho )X_{i-1}} ||Z_i^-|| \\
||L_1(\lambda)_i^+ Z_i^+|| &\leq C e^{-(\alpha - \rho )X_i} ||Z_i^+||
\end{align*}

\begin{proof}
The center integrals will have the weakest bound, so it suffices to only look at those. For $x \leq 0$, we have 

\begin{align*}
\left| \int_{-X_{i-1}}^x \Phi^c(x, y; \lambda) G_i^-(y)Z_i^-(y) dy \right| 
&\leq C ||Z_i^-|| \int_{-X_{i-1}}^x e^{\rho (x - y)} e^{-\alpha X_{i-1}} e^{-\alpha(X_{i-1} + y) } dy \\
&\leq C ||Z_i^-|| e^{-(\alpha - \rho) X_{i-1}} \int_{-X_{i-1}}^0  e^{-\alpha(X_{i-1} + y) } dy \\
&\leq C ||Z_i^-|| e^{-(\alpha - \rho) X_{i-1}} 
\end{align*}

This can be made arbitrarily small for sufficiently large $X_{i-1}$. The $x \geq 0$ case is similar. Thus we have our piecewise bounds

\begin{align*}
||L_1(\lambda)_i^- Z_i^-|| &\leq C e^{-(\alpha - \rho)X_{i-1}} ||Z_i^-|| \\
||L_1(\lambda)_i^+ Z_i^+|| &\leq C e^{-(\alpha - \rho)X_i} ||Z_i^+||
\end{align*}

The uniform bound depends only on the smallest of the $X_i$.

\[
||L_1(\lambda)Z|| \leq C e^{-(\alpha - \rho)X_m}||Z||
\]

\end{proof}
\end{lemma}

Next, we obtain a bound on the terms from the fixed point equations which do not involve $Z$. 

% inversion, lemma 2 (L2 bound)

\begin{lemma}\label{L2}

Let $L_2(\lambda): V_a \times V_b \times V_c \times V_d \rightarrow V_Z$ be the linear operator defined piecewise by

\begin{align*}
L_2(\lambda)&(a,b,c,d)_i^-(x) = \Phi^s(x, -X_{i-1}; \lambda) a_{i-1}^- + \Phi^u(x, 0; \lambda) b_i^- + \Phi^c(x, -X_{i-1}; \lambda) c_{i-1}^- \\
&+ \int_0^x \Phi^u(x, y; \lambda) \lambda^2 d_i P^-(y; \beta_i^-, \lambda)^{-1} \tilde{H}_i^-(y) dy
+ \int_{-X_{i-1}}^x \Phi^s(x, y; \lambda) \lambda^2 d_i P^-(y; \beta_i^-, \lambda)^{-1} \tilde{H}_i^-(y) dy \\
&+ \int_{-X_{i-1}}^x \Phi^c(x, y; \lambda) \lambda^2 d_i P^-(y; \beta_i^-, \lambda)^{-1} \tilde{H}_i^-(y) \\
L_2(\lambda)&(a,b,c,d)_i^+(x) = \Phi^u(x, X_i; \lambda) a_i^+ + \Phi^s(x, 0; \lambda) b_i^+ + \Phi^c(x, X_i; \lambda) c_i^+ \\
&+ \int_0^x \Phi^s(x, y; \lambda) \lambda^2 d_i P^+(y; \beta_i^+, \lambda)^{-1} \tilde{H}_i^+(y) dy + \int_{X_i}^x \Phi^u(x, y; \lambda) \lambda^2 d_i P^+(y; \beta_i^+, \lambda)^{-1} \tilde{H}_i^+(y) dy \\
&+ \int_{X_i}^x \Phi^c(x, y; \lambda) \lambda^2 d_i P^+(y; \beta_i^+, \lambda)^{-1} \tilde{H}_i^+(y) dy \\
\end{align*}

Then $L_2$ is a bounded linear operator with bounds

\begin{align*}
|L_2(\lambda)(a,b,c,d)_i^-(x)| &\leq C (|a_{i-1}^-| + |b_i^-| + e^{\rho X_{i-1}} |c_{i-1}^-| + |\lambda|^2 |d| ) \\
|L_2(\lambda)(a,b,c,d)_i^+(x)| &\leq C (|a_i^+| + |b_i^+| + e^{\rho X_i} |c_i^+| + |\lambda|^2 |d| ) 
\end{align*}

We also have the following piecewise, $x$-dependent bounds for $L_2$.

\begin{align*}
|L_2(\lambda)(a,b,c,d)_i^-(x)| &\leq C (e^{-\alpha(X_{i-1} + x)}|a_{i-1}^-| + |b_i^-| + e^{\rho(X_{i-1} + x)} |c_{i-1}^-| + |\lambda|^2 |d| ) \\
|L_2(\lambda)(a,b,c,d)_i^+(x)| &\leq C (e^{-\alpha(X_i - x)}|a_i^+| + |b_i^+| + e^{\rho(X_i - x)} |c_i^+| + |\lambda|^2 |d| ) 
\end{align*}

\begin{proof}
We compute the bounds for each term for $L_2(\lambda)(a,b,c,d)_i^-(x)$. The ``positive'' piece is similar. Note that $x \leq 0$ for all of these.

\begin{enumerate}
\item For the term involving $a$,
\[
|\Phi^s(x, -X_{i-1}; \lambda) a_{i-1}^-| 
\leq C e^{-\alpha(x + X_{i-1})} | a_{i-1}^-| \leq C | a_{i-1}^-|
\]
\item For the term involving $b$,
\[
|\Phi^u(x, 0; \lambda) b_i^-| \leq C e^{-\alpha x}|b_i^-| \leq C |b_i^-|
\]
\item For the term involving $c$,
\[
|\Phi^c(x, -X_{i-1}; \lambda) c_{i-1}^-| 
\leq C e^{\rho(x + X_{i-1})}|c_{i-1}^-| \leq C e^{\rho X_{i-1}}|c_{i-1}^-|
\]
\item For the integral terms, the center integral term will have the weakest bound, so it suffices to look at that one.
\begin{align*}
\left| \int_{-X_{i-1}}^x \Phi^c(x, y; \lambda) \lambda^2 d_i P_i^-(y; \lambda)^{-1} \tilde{H}_i^-(y) \right| 
&\leq C |\lambda|^2 |d| \int_{-X_{i-1}}^x e^{\rho(x-y) }e^{\alpha y} dy \\
&\leq C |\lambda|^2 |d| \int_{-X_{i-1}}^x e^{\rho x} e^{(\alpha - \rho) y} dy \\
&\leq C |\lambda|^2 |d| \int_{-X_{i-1}}^0 e^{(\alpha - \rho) y} dy \\
&\leq C |\lambda|^2 |d|
\end{align*}

\end{enumerate}

Thus we have piecewise, $x-$dependent bounds

\begin{align*}
|L_2(\lambda)(a,b,c,d)_i^-(x)| &\leq C (e^{-\alpha(X_{i-1} + x)}|a_{i-1}^-| + |b_i^-| + e^{\rho(X_{i-1} + x)} |c_{i-1}^-| + |\lambda|^2 |d| ) \\
|L_2(\lambda)(a,b,c,d)_i^+(x)| &\leq C (e^{-\alpha(X_i - x)}|a_i^+| + |b_i^+| + e^{\rho(X_i - x)} |c_i^+| + |\lambda|^2 |d| ) 
\end{align*}

The uniform bounds above give us the bounds independent of $x$.
\end{proof}
\end{lemma}

Now that we have these bounds, we can perform the inversion

% inversion lemma 3 - invert to solve for Z

\begin{lemma}\label{Z1}
There exists a bounded linear operator $Z_1: V_\lambda \times V_a \times V_b \times V_c \times V_d \rightarrow V_Z$ such that 

\[
Z = Z_1(\lambda)(a,b,c,d)
\]

This operator is analytic in $\lambda$ and linear in $(a, b, c, d)$. The operator $Z_1$ satisfies the piecewise bounds

\begin{align*}
||Z_1(\lambda)(a,b,c,d)_i^-|| &\leq C ( |a_{i-1}^-| + |b_i^-| + e^{\rho X_{i-1}}|c_{i-1}^-| + |\lambda|^2 |d| ) \\
||Z_1(\lambda)(a,b,c,d)_i^+|| &\leq C ( |a_i^+| + |b_i^+| + e^{\rho X_i} |c_i^+| + |\lambda|^2 |d| )
\end{align*}

\begin{proof}
Let the linear operators $L_1$ and $L_2$ be defined as in the previous two lemmas. Then we can rewrite the fixed point equations as

\[
(I - L_1(\lambda))Z = L_2(\lambda)(a,b,c,d)
\]

For $L_1$ we have the estimate from Lemma \ref{L1}

\begin{align*}
||L_1(\lambda)Z|| &\leq C e^{-(\alpha - \rho)X_m}||Z|| \\
&\leq C e^{-\tilde{\alpha} X_m}||Z|| \\
&\leq C \delta ||Z||
\end{align*}

Thus if we choose $\delta$ sufficiently small (i.e. smaller than $C$), the operator norm of $L_1$ is less than 1, which implies that the operator $(I - L_1(\lambda))$ is invertible. The inverse $(I - L_1(\lambda))^{-1}$ is analytic in $\lambda$ and has operator norm 

\[
||(I - L_1(\lambda))^{-1}|| \leq \frac{1}{1 - ||L_1||}
\]

We can then write $Z$ as
\[
Z = Z_1(\lambda)(a,b,c,d) = (I - L_1(\lambda))^{-1} L_2(\lambda)(a,b,c,d)
\]

which depends linearly on $(a,b,c,d)$ and analytically on $\lambda$. Since the operator norm of $L_1$ is bounded by a constant (independent of the $X_i$), and we have a piecewise bound on $L_2$ from Lemma \ref{L2}, the piecewise bounds on $||Z||$ are obtained by noting which piece of $L_2$ is involved with which piece of $Z$.
\end{proof}
\end{lemma}

% match at ends

\subsubsection*{Matching at ends}

Here, we solve the condition

\[
P^+(X_i; \beta_i^+, \lambda) Z_i^+(X_i) - P^-(-X_i; \beta_{i+1}^-, \lambda) Z_{i+1}^-(-X_i) = D_i d
\]

At $\pm X_i$, the fixed point equations become

\begin{align*}
Z_{i+1}^-(-X_i) &= a_i^- + \Phi^u(-X_i, 0; \lambda) b_{i+1}^- + c_i^- \\
&+ \int_0^{-X_i} \Phi^u(-X_i, y; \lambda)[ G_{i+1}^-(y) Z_{i+1}^-(y) + \lambda^2 d_{i+1} P^-(y; \beta_{i+1}^-, \lambda)^{-1} \tilde{H}_i^-(y) ] dy \\
Z_i^+(X_i) &= a_i^+ + \Phi^s(X_i, 0; \lambda) b_i^+ + c_i^+ \\
&+ \int_0^{X_i} \Phi^s(X_i, y; \lambda)[ G_i^+(y) Z_i^+(y) + \lambda^2 d_i P^+(y; \beta_i^+, \lambda)^{-1} \tilde{H}_i^+(y) ] dy
\end{align*}

We used here the fact that, for example, $a_i^- \in E^s(\lambda)$ and $\Phi^s(-X_{i-1}, -X_{i-1}; \lambda)$ is the identity on $E^s(\lambda)$. From the Conjugation Lemma, we have

\begin{equation}\label{conjest}
P^\pm(\pm X_i; \beta_i^\pm \lambda) = I + \mathcal{O}(e^{-\alpha X_i})
\end{equation}

which we will use on the $a_i^\pm$ and $c_i^\pm$ terms. Thus we obtain the equation

\begin{align}\label{Dideq1}
D_i d &= a_i^+ - a_i^- + c_i^+ - c_i^- + L_3(\lambda)_i(a, b, c^+, c^-, d)
\end{align}

For a bound on $L_3$, we look at the individual terms. As usual, we will only look at one of the two pieces.

\begin{enumerate}
\item For the $a$ and $c$ terms, we have a term of order $\mathcal{O}(e^{-\alpha X_i}(|a_i| + |c_i^+| + |c_i^-|)$, which comes from the conjugation operators $P^\pm(\pm X_i; \beta_i^\pm \lambda)$.

\item For the terms involving $b$, we have

\[
| P^-(-X_i; \beta_{i+1}^-, \lambda) \Phi^u(-X_i, 0; \lambda) b_{i+1}^-| \leq C e^{-\alpha X_i} |b_{i+1}
^-|
\]

\item For the integral terms involving $Z$, we use the bound $Z_1$ from Lemma \ref{Z1}.

\begin{align*}
&\left|
P^+(X_i; \beta_i^+, \lambda) \int_0^{X_i} \Phi^s(X_i, y; \lambda) G_i^\pm(y) Z_i^\pm(y) dy \right| \\
&\leq C \int_0^{X_i} e^{-\alpha(X_i - y)} e^{-\alpha X_i} e^{-\alpha(X_i - y)} ( |a_i^+| + |b_i^+| + e^{\rho X_i} |c_i^+| + |\lambda|^2 |d| ) \\
&\leq C \left( e^{-\alpha X_i} ( |a_i^+| + |b_i^+| + |\lambda|^2 |d| ) + e^{(\alpha - \rho) X_i} |c_i^+| \right) \int_0^{X_i} e^{-2 \alpha(X_i - y)} dy \\
&\leq C \left( e^{-\alpha X_i} ( |a_i^+| + |b_i^+| + |\lambda|^2 |d| ) + e^{(\alpha - \rho) X_i} |c_i^+| \right)
\end{align*}

\item For the integral terms involving $\tilde{H}$,

\begin{align*}
\left|
P^+(X_i; \beta_i^+, \lambda) \int_0^{X_i} \Phi^s(X_i, y; \lambda) P^+(X_i; \beta_i^+, \lambda)^{-1} \tilde{H}_i^+(y) dy \right|
&\leq C \int_0^{X_i} e^{-\alpha(X_i - y)}e^{-\alpha y} dy \\
&\leq C \int_0^{X_i} e^{-(\alpha - \rho)(X_i - y)}e^{-\alpha y} dy \\
&= C e^{-(\alpha - \rho) X_i} \int_0^{X_i} e^{-\rho y} dy \\ 
&\leq C e^{-(\alpha - \rho) X_i} 
\end{align*}

\end{enumerate}

Putting these all together, we have the following bound for $L_3$.
\[
|L_3(\lambda)_i(a, b, c^+, c^-, d)| \leq C \Big( e^{-\alpha X_i} |a_i| + e^{-\alpha X_i} (|b_i^+| + |b_{i+1}^-|) + e^{-(\alpha - \rho) X_i} (|c_i^+| + |c_i^-|) + e^{-(\alpha - \rho) X_i} |\lambda^2| |d| \Big)
\]

Following San98 (and leaving out some steps), we can solve this for $(a, c^+)$ to get $(a_i, c_i^+) = A_1(\lambda)_i(b, c_i^-, d)$, with bound

\begin{align*}
|A_1&(\lambda)_i(b, c^-, d)|
\leq C \Big( e^{-\alpha X_i} (|b_i^+| + |b_{i+1}^-|) + |c_i^-| + (e^{-(\alpha - \rho) X_i} |\lambda^2| + |D_i|)|d| \Big)
\end{align*} 

As in San98, we hit \eqref{Dideq1} with projections on the various eigenspaces $E^{s/u/c}(\lambda)$. The remainder term $A_2(\lambda)_i^c(b, d)$ is found by substituting the bound for $A_1$ into $L_3$ and simplifying.

\begin{align*}
a_i^+ &= P_0^u(\lambda) D_i d + A_2(\lambda)_i^+(b, c^-, d) \\
a_i^- &= -P_0^s(\lambda) D_i d + A_2(\lambda)_i^-(b, c^-, d) \\
c_i^+ &= c_i^- + P_0^c(\lambda) D_i d + A_2(\lambda)_i^c(b, c^-, d) )
\end{align*}

where we have bound

\begin{align*}
|A_2&(\lambda)_i(b, d)|
\leq C \Big( e^{-\alpha X_i} (|b_i^+| + |b_{i+1}^-|) + e^{-(\alpha - \rho) X_i} |c_i^-| + e^{-(\alpha - \rho) X_i} |\lambda|^2|d| + e^{-\alpha X_i} |D_i||d| \Big)
\end{align*} 

For the first two, this is not quite what we want. Anticipating what we will need later, we write this as

\begin{align*}
a_i^+ = P^+(X_i; \beta_i^+, \lambda)a_i^+ + (I - P^+(X_i; \beta_i^+, \lambda))a_i^+ &= P_0^u(\lambda) D_i d + A_2(\lambda)_i^+(b, c^-, d)
\end{align*}

Rearranging this, we obtain

\begin{align*}
P^+(X_i; \beta_i^+, \lambda)a_i^+ &= P_0^u(\lambda) D_i d + A_2(\lambda)_i^+(b, c^-, d) - (I - P^+(X_i; \beta_i^+, \lambda))a_i^+ \\
&= P_0^u(\lambda) D_i d + A_2(\lambda)_i^+(b, c^-, d) + \mathcal{O}\Big( e^{-\alpha X_i} (|b_i^+| + |b_{i+1}^-| + |c_i^-| + (|\lambda^2| + |D_i|)|d|)\Big)
\end{align*}

where we used the bound $A_1$ and the estimate \eqref{conjest}. The last term on the RHS is the same (or higher) order as $A_2$, so we incorporate that into $A_2(\lambda)_i^+(b, c^-, d)$ to get

\begin{align*}
P^+(X_i; \beta_i^+, \lambda)a_i^+ &= P_0^u(\lambda) D_i d + A_2(\lambda)_i^+(b, c^-, d)
\end{align*}

Finally, we operate on both sides on the left by $P^+(X_i; \beta_i^+, \lambda)^{-1}$ to solve for $a_i^+$. This is a bounded operator, so we will also incorporate this into $A_2(\lambda)_i^+(b, c^-, d)$ (with the same bound). Thus we have

\begin{align*}
a_i^+ &= P^+(X_i; \beta_i^+, \lambda) P_0^u(\lambda) D_i d + A_2(\lambda)_i^+(b, c^-, d)
\end{align*}

The same thing works for $a_i^-$, giving us

\begin{align*}
a_i^+ &= P^+(X_i; \beta_i^+, \lambda) P_0^u(\lambda) D_i d + A_2(\lambda)_i^+(b, c^-, d) \\
a_i^- &= -P^-(-X_i; \beta_i^-, \lambda) P_0^s(\lambda) D_i d + A_2(\lambda)_i^-(b, c^-, d)
\end{align*}

For the third one, we would like to evaluate and get an estimate for $P_0^c(\lambda) D_i d$. We will fill in the details later, but this should be

\[
P_0^c(\lambda) D_i d = \mathcal{O}(e^{-\alpha X_i}(|\lambda| + e^{-\alpha X_i})|d|)
\]

Finally, we substitute $A_1$ and the expression for $c_i^+$ involving $A_2$into $Z_1$ to get $Z_2$ with bound

\begin{align*}
||Z_2(\lambda)(b,c^-,d)_i^-|| &\leq C ( e^{-(\alpha - \rho) X_{i-1}} |b_{i-1}^+| + |b_i^-| + e^{\rho X_{i-1}}|c_{i-1}^-| + |\lambda|^2 |d| + e^{-(\alpha - \rho)X_{i-1}}|\lambda||d| + |D_{i-1}||d| ) \\
||Z_2(\lambda)(b,c^-,d)_i^+|| &\leq C ( e^{-(\alpha - \rho) X_i}|b_{i+1}^-| + |b_i^+| + e^{\rho X_i} |c_i^-| + |\lambda|^2 |d| + e^{-(\alpha - \rho)X_i}|\lambda||d| + |D_i||d| )
\end{align*}

\subsubsection*{Matching at 0}

The next step is to satisfy the conditions

\begin{align*}
P_i^\pm(0; \beta_i^\pm, \lambda) Z_i^\pm(0) &\in \C \Psi(0) \oplus Y^0 \oplus Y^+ \oplus Y^- \\
P^+(0; \beta_i^+, \lambda) Z_i^+(0) - P^-(0; \beta_i^-, \lambda) Z_i^-(0) &\in \C \Psi(0) \oplus Y^0
\end{align*}

Recall that we have

\[
\C^m = \C \Psi(0) \oplus \C Q'(0) \oplus Y^0 \oplus Y^+ \oplus Y^- 
\]

This condition is equivalent to the three projections

\begin{align*}
P(\C Q'(0) ) P^-(0; \beta_i^-, \lambda) Z_i^-(0) &= 0 \\
P(\C Q'(0) ) P^+(0; \beta_i^+, \lambda) Z_i^+(0) &= 0 \\
P(Y_i^+ \oplus Y_i^-) ( P^+(0; \beta_i^+, \lambda) Z_i^+(0) - P^-(0; \beta_i^-, \lambda) Z_i^-(0) ) &= 0
\end{align*}

where the kernel of each projection is the remaining spaces in the direct sum. We don't need $\C Q'(0)$ in the third equation since we eliminated any component in it in the first two equations.\\

Recall that for $\lambda = 0$, the tangent space to the stable manifold at $x = 0$ is spanned by $Y^+$ and $Q'(0)$, and the tangent space to the unstable manifold at $x = 0$ is spanned by $Y^-$ and $Q'(0)$. Thus we have

\begin{align*}
P^-(0; 0, 0)^{-1} Q'(0) &= v^- \in E^u(0) \\
P^+(0; 0, 0)^{-1} Q'(0) &= v^+ \in E^s(0)
\end{align*}

Let

\begin{align*}
E^u(0) &= \C v^- \oplus E^- \\
E^s(0) &= \C v^+ \oplus E^+ \\
\end{align*}

Then we have

\begin{align*}
P^-(0; 0, 0)^{-1} Y^- = E^- \\
P^+(0; 0, 0)^{-1} Y^+ = E^+ \\
\end{align*}

Next, following San98, we decompose $b_i^\pm$ uniquely as $b_i^\pm = x_i^\pm + y_i^\pm$, where $x_i^\pm \in \C v^\pm$ and $y_i^\pm \in E^\pm$.\\

At $x = 0$, the fixed point equations become

\begin{align*}
Z_i^-(0) &= \Phi^s(0, -X_{i-1}; \lambda) a_{i-1}^- + \Phi^u(0, 0; \lambda) b_i^- + \Phi^c(0, -X_{i-1}; \lambda) c_{i-1}^- \\
&+ \int_{-X_{i-1}}^0 \Phi^s(0, y; \lambda) [G_i^-(y)Z_i^-(y) + \lambda^2 d_i P^-(y; \beta_i^-, \lambda)^{-1} \tilde{H}_i^-(y)] dy \\
&+ \int_{-X_{i-1}}^0 \Phi^c(0, y; \lambda) [G_i^-(y)Z_i^-(y) + \lambda^2 d_i P^-(y; \beta_i^-, \lambda)^{-1} \tilde{H}_i^-(y)] dy  \\ 
Z_i^+(0) &= \Phi^u(0, X_i; \lambda) a_i^+ + \Phi^s(0, 0; \lambda) b_i^+ + \Phi^c(0, X_i; \lambda) c_i^+ \\
&+ \int_{X_i}^0 \Phi^u(0, y; \lambda) [G_i^+(y)Z_i^+(y) + \lambda^2 d_i P^+(y; \beta_i^+, \lambda)^{-1} \tilde{H}_i^+(y) ] dy \\
&+ \int_{X_i}^0 \Phi^c(0, y; \lambda) [G_i^+(y)Z_i^+(y) + \lambda^2 d_i P^+(y; \beta_i^+, \lambda)^{-1} \tilde{H}_i^+(y) ] dy \\
\end{align*}

Doing a little manipulation on the $b_i$ terms, and using the known form of the evolution $\Phi^c$ on $E^c(\lambda)$, we have

\begin{align*}
Z_i^-(0) &= \Phi^s(0, -X_{i-1}; \lambda) a_{i-1}^- + x_i^- + y_i^- + (P_0^u(\lambda) - P_0^u(0))b_i^- + e^{\nu(\lambda) X_{i-1}} c_{i-1}^- \\
&+ \int_{-X_{i-1}}^0 \Phi^s(0, y; \lambda) [G_i^-(y)Z_i^-(y) + \lambda^2 d_i P^-(y; \beta_i^-, \lambda)^{-1} \tilde{H}_i^-(y)] dy \\
&+ \int_{-X_{i-1}}^0 \Phi^c(0, y; \lambda) [G_i^-(y)Z_i^-(y) + \lambda^2 d_i P^-(y; \beta_i^-, \lambda)^{-1} \tilde{H}_i^-(y)] dy  \\ 
Z_i^+(0) &= \Phi^u(0, X_i; \lambda) a_i^+ + x_i^+ + y_i^+ + (P_0^s(\lambda) - P_0^s(0)) b_i^+ + e^{-\nu(\lambda)X_i} c_i^+ \\
&+ \int_{X_i}^0 \Phi^u(0, y; \lambda) [G_i^+(y)Z_i^+(y) + \lambda^2 d_i P^+(y; \beta_i^+, \lambda)^{-1} \tilde{H}_i^+(y) ] dy \\
&+ \int_{X_i}^0 \Phi^c(0, y; \lambda) [G_i^+(y)Z_i^+(y) + \lambda^2 d_i P^+(y; \beta_i^+, \lambda)^{-1} \tilde{H}_i^+(y) ] dy \\
\end{align*}

Since $c_i^\pm$ are in the eigenspaces $E^c(\lambda)$, we do some further manipulation to separate out a component in $E^c(0)$ so that can be eliminated by the projections.

\begin{align*}
Z_i^-(0) &= \Phi^s(0, -X_{i-1}; \lambda) a_{i-1}^- + x_i^- + y_i^- + (P_0^u(\lambda) - P_0^u(0))b_i^- \\
&+ P_0^c(0) e^{\nu(\lambda) X_{i-1}} c_{i-1}^- + (P_0^c(\lambda) - P_0^c(0)) e^{\nu(\lambda) X_{i-1}} c_{i-1}^- \\
&+ \int_{-X_{i-1}}^0 \Phi^s(0, y; \lambda) [G_i^-(y)Z_i^-(y) + \lambda^2 d_i P^-(y; \beta_i^-, \lambda)^{-1} \tilde{H}_i^-(y)] dy \\
&+ \int_{-X_{i-1}}^0 \Phi^c(0, y; \lambda) [G_i^-(y)Z_i^-(y) + \lambda^2 d_i P^-(y; \beta_i^-, \lambda)^{-1} \tilde{H}_i^-(y)] dy  \\ 
Z_i^+(0) &= \Phi^u(0, X_i; \lambda) a_i^+ + x_i^+ + y_i^+ + (P_0^s(\lambda) - P_0^s(0)) b_i^+ \\
&+ P_0^c(0) e^{-\nu(\lambda)X_i} c_i^+ + (P_0^c(\lambda) - P_0^c(0)) e^{-\nu(\lambda)X_i} \\
&+ \int_{X_i}^0 \Phi^u(0, y; \lambda) [G_i^+(y)Z_i^+(y) + \lambda^2 d_i P^+(y; \beta_i^+, \lambda)^{-1} \tilde{H}_i^+(y) ] dy \\
&+ \int_{X_i}^0 \Phi^c(0, y; \lambda) [G_i^+(y)Z_i^+(y) + \lambda^2 d_i P^+(y; \beta_i^+, \lambda)^{-1} \tilde{H}_i^+(y) ] dy \\
\end{align*}
 
Before we take projections, we will operate on these with the conjugation operators $P^\pm(0; \beta_i^\pm, \lambda)$. Where needed, we will use the Taylor expansion $P^\pm(0; \beta_i^\pm, \lambda) = P^\pm(0; 0, 0) + \mathcal{O}(e^{-\alpha X_m} + |\lambda|)$ (from the Taylor expansion). If we do that, then take the projections, we eventually wind up with a equation of the form

\[
\begin{pmatrix}x_i^- \\ x_i^+ \\ 
y_i^+ - y_i^- \end{pmatrix} + L_4(\lambda)_i(b, \tilde{c}, d) = 0
\]

where we define

\begin{equation}\label{tildec}
\tilde{c}_i^\pm = e^{\pm \nu(\lambda) X_i} c_i^-
\end{equation}

To get a bound on $L_4$, we need to bound the individual terms from the fixed point equations.

\begin{enumerate}

\item For the $a_i$ terms, we substitute the bound for $A_1(\lambda)$ to get

\begin{align*}
|P^-(0; \beta_i^-, \lambda) \Phi^s(0, -X_{i-1}; \lambda) a_{i-1}^-|
&\leq C \Big( e^{-2 \alpha X_{i-1}} (|b_{i-1}^+| + |b_i^-|) + e^{-\alpha X_{i-1}}|c_{i-1}^-| + e^{-\alpha X_{i-1}}(e^{-(\alpha - \rho) X_{i-1}} |\lambda^2| + |D_{i-1}|)|d| \Big) \\
|P_i^+(0; \beta_i^+, \lambda) \Phi^u(0, X_i; \lambda) a_i^+|
&\leq C \Big( e^{-2 \alpha X_i} (|b_i^+| + |b_{i+1}^-|) + e^{-\alpha X_i} |c_i^-| + e^{-\alpha X_i} (e^{-(\alpha - \rho) X_i} |\lambda^2| + |D_i|)|d| \Big)
\end{align*}

\item For the $b_i$ terms, we have

\[
|P_i^-(0; \lambda)(P_0^u(\lambda) - P_0^u(0))b_i^-| \leq C |\lambda||b_i^-| 
\]

We also have from the projection

\begin{align*}
P_i^-(0; \beta_i^-, \lambda)( x_i^- + y_i^- ) = P_i^-(0; 0, 0)( x_i^- + y_i^- ) + \mathcal{O}((e^{-\alpha X_m} + |\lambda|)|b_i^-|)
\end{align*}

\item For the $c_i^-$ terms, we have

\begin{align*}
P_i^-(0; \beta_i^-, \lambda)&[ P_0^c(0) e^{\nu(\lambda) X_{i-1}} c_{i-1}^- + (P_0^c(\lambda) - P_0^c(0)) e^{\nu(\lambda) X_{i-1}} c_{i-1}^-] \\
&= P_i^-(0; 0, 0) P_0^c(0) \tilde{c}_{i-1}^+ + (P_i^-(0; \beta_i^-, \lambda) - P_i^-(0; 0, 0)) P_0^c(0) \tilde{c}_{i-1}^+ + P_i^-(0; \beta_i^-, \lambda) (P_0^c(\lambda) - P_0^c(0)) \tilde{c}_{i-1}^+ \\
&= P_i^-(0; 0, 0) P_0^c(0) \tilde{c}_{i-1}^+ + \mathcal{O}((e^{-\alpha X_m} + |\lambda|)|\tilde{c}_{i-1}^+|)
\end{align*}

Note that the leading order term here is eliminated by all the projections we will be using.

\item The $c_i^+$ terms are the same, except we have to use $A_2$ to write $c_i^+$ in terms of $c_i^-$. Using the bound for $A_2$ and the estimate for $P_0^c(\lambda) D_i d$, we obtain 

\begin{align*}
e^{-\nu(\lambda)X_i} c_i^+ &= e^{-\nu(\lambda)X_i} c_i^- 
+ e^{-\nu(\lambda)X_i} P_0^c(\lambda) D_i d + e^{-\nu(\lambda)X_i} A_2(\lambda)_i^c(b, d)\\
&= e^{-\nu(\lambda)X_i} c_i^- + \mathcal{O}\Big( e^{-(\alpha - \rho) X_i} ( |\lambda| + e^{-\alpha X_i} ) |d|) + e^{-(\alpha - \rho) X_i} (|b_i^+| + |b_{i+1}^-|) \\
&+ e^{-(\alpha - 2 \rho) X_i} |c_i^-| + e^{-(\alpha - 2 \rho) X_i} |\lambda|^2|d| + e^{-(\alpha - \rho) X_i} |D_i||d| ) \Big) \\
&= e^{-\nu(\lambda)X_i} c_i^- + \mathcal{O}\Big( e^{-\tilde{\alpha} X_i} ( |b_i^+| + |b_{i+1}^-| + |c_i^-| + (|\lambda| + |D_i| ) |d|) \Big) \\
\end{align*}

Thus we have

\begin{align*}
P_i^+(0; \beta_i^+, \lambda)&[ P_0^c(0) e^{-\nu(\lambda) X_i} c_i^+ + (P_0^c(\lambda) - P_0^c(0)) e^{-\nu(\lambda) X_i} c_i^+] \\
&= P_i^-(0; \beta_i^-, 0) P_0^c(0) \tilde{c}_i^- + \mathcal{O}((e^{-\alpha X_m} + |\lambda|)|\tilde{c}_i^-|)
+ \mathcal{O}\Big( e^{-\tilde{\alpha} X_i} ( |b_i^+| + |b_{i+1}^-| + |c_i^-| + (|\lambda| + |D_i| ) |d|) \Big)
\end{align*}

Again, the leading order term is eliminated by all the projections.

\item The bound on the integral terms is determined by the bound on the center subspace, since there is potential growth in that subspace. For the integral terms involving $Z$, we have

\begin{align*}
&\left| P_i^-(0; \beta_i^-, \lambda) \int_{-X_{i-1}}^0 \Phi^c(0, y; \lambda) G_i^-(y) Z_i^-(y) dy \right| \\
&\leq C \int_{-X_{i-1}}^0 e^{-\rho y} e^{-\alpha X_{i-1}} e^{-\alpha(X_{i-1} + y) } ||Z_2(b, c^-, d)_i^-|| dy \\
&\leq C e^{-(\alpha - \rho) X_{i-1}} ||Z_2(b, c^-, d)_i^-|| \int_{-X_{i-1}}^0 e^{-\alpha(X_{i-1} + y) } dy \\
&\leq C e^{-(\alpha - \rho) X_{i-1}} ( e^{-(\alpha - \rho) X_{i-1}} |b_{i-1}^+| + |b_i^-| + e^{\rho X_{i-1}}|c_{i-1}^-| 
+ |\lambda|^2 |d| + e^{-(\alpha - \rho)X_{i-1}}|\lambda||d| + |D_{i-1}||d| ) \\
&\leq C e^{-\tilde{\alpha} X_{i-1}} \left( e^{-\alpha X_{i-1}} |b_{i-1}^+| + |b_i^-| + |c_{i-1}^-| + (|\lambda|^2 + e^{-\tilde{\alpha} X_{i-1}}|\lambda| + |D_{i-1}|) |d| \right)
\end{align*}

For the other one,

\begin{align*}
&\left| P_i^+(0; \beta_i^+, \lambda) \int_{X_i}^0 \Phi^c(0, y; \lambda) G_i^+(y) Z_i^+(y) dy \right| \\
&\leq C e^{-\tilde{\alpha} X_i} \left( e^{-\alpha X_i} |b_{i+1}^-| + |b_i^+| + |c_i^-| + (|\lambda|^2 + e^{-\tilde{\alpha} X_i} \lambda| + |D_i|) |d| \right)
\end{align*}

\item The integral terms involving $\tilde{H}$ are bounded by

\begin{align*}
\left| \lambda^2 d P_i^-(0; \beta_i^-, \lambda) \int_{-X_{i-1}}^0 \Phi^c(0, y; \lambda) P_i^-(y; \lambda)^{-1} \tilde{H}_i^-(y) dy \right| &\leq C |\lambda|^2 |d| \int_{-X_{i-1}}^0 e^{\rho y} e^{-\alpha y} dy \\
&\leq C |\lambda|^2 |d|
\end{align*}

\end{enumerate}

Putting all these together, we obtain the bound for $L_4(\lambda)_i(b, \tilde{c}, d)$.

\begin{align*}
L_4(\lambda)_i(b, \tilde{c}, d) &\leq 
C\Big( (|\lambda| + e^{-\tilde{\alpha}X_m})|b| 
+ (|\lambda| + e^{-\tilde{\alpha}X_m})( |\tilde{c}_{i-1}^+| + |\tilde{c}_i^-|) + e^{-\tilde{\alpha} X_{i-1}} c_{i-1}^- + e^{-\tilde{\alpha} X_i} c_i^-\\
&+ ( e^{-\tilde{\alpha}X_m} |D| + e^{-\tilde{\alpha}X_m}|\lambda| + |\lambda|^2)|d| \Big)
\end{align*}

Peforming the inversion, we solve for $b$ to get $B_1(\lambda)(\tilde{c}, d)$, which has bound

\begin{align*}
|B_1(\lambda)_i(\tilde{c}, d)| \leq C\Big( 
(|\lambda| + e^{-\tilde{\alpha}X_m})( |\tilde{c}_{i-1}^+| + |\tilde{c}_i^-|)
+ e^{-\tilde{\alpha} X_{i-1}} |c_{i-1}^-| + e^{-\tilde{\alpha} X_i} |c_i^-| + ( e^{-\tilde{\alpha}X_m} |D| + e^{-\tilde{\alpha}X_m}|\lambda| + |\lambda|^2)|d| \Big)
\end{align*}

We keep the $\tilde{c}^\pm$ terms separate, since that might end up mattering. We can plug this into the bound for $A_2$ to get $A_4$ with bound

\begin{align*}
|A_4&(\lambda)_i(\tilde{c}, d)|
\leq C \Big( 
e^{-\alpha X_i} (|\lambda| + e^{-\tilde{\alpha}X_m})(|\tilde{c}_{i-1}^+| + |\tilde{c}_{i+1}^-|) + e^{-\tilde{\alpha}X_{i-1}}|c_{i-1}^-| + e^{-\tilde{\alpha}X_i}|c_i^-| + e^{-\tilde{\alpha}X_{i+1}}|c_{i+1}^-| \\
&+ e^{-\tilde{\alpha} X_m} |\lambda|^2|d| + e^{-\alpha X_m}|D||d| \Big)
\end{align*} 

We can also plug this into the bound for $Z_2$ to get $Z_3$.

\begin{align*}
||Z_3(\lambda)(b,c^-,d)_i^-|| &\leq C ( (|\lambda| + e^{-\tilde{\alpha}X_m})(|\tilde{c}_i^-| + |\tilde{c}_{i-2}^+|) + e^{-\tilde{\alpha} X_{i-2}} |c_{i-2}^-| + e^{-\tilde{\alpha} X_i} |c_i^-| + e^{\rho X_{i-1}}|c_{i-1}^-| \\ 
&+ |\lambda|^2 |d| + e^{-(\alpha - \rho)X_{i-1}}|\lambda||d| + |D_{i-1}||d| ) \\
||Z_3(\lambda)(b,c^-,d)_i^+|| &\leq C \Big( (|\lambda| + e^{-\tilde{\alpha}X_m})(|\tilde{c}_{i+1}^-| + |\tilde{c}_{i-1}^+|) + e^{-\tilde{\alpha} X_{i-1}} |c_{i-1}^-| + e^{-\tilde{\alpha} X_{i+1}} |c_{i+1}^-| + e^{\rho X_i} |c_i^-| \\
&+ |\lambda|^2 |d| + e^{-(\alpha - \rho)X_i}|\lambda||d| + |D_i||d| \Big)
\end{align*}

Now that we have solved (uniquely) for everything except for the $c_i^-$ and $d$, we are ready to compute the jump conditions in the two directions.

\subsubsection*{Jump in Y0 direction}

For this jump, we project on $Y_0$.

\[
\xi^c_i = P(Y^0) ( P^+(0; \beta_i^+, \lambda) Z_i^+(0) - P^-(0; \beta_i^-, \lambda) Z_i^-(0) )
\]

We can choose to either (1) project $W_i^\pm(0) = P^\pm(0; \beta_i^\pm, \lambda) Z_i^\pm(0)$ on $Y_0$ (which should be just the inner product with $\Psi^c(0)$); or (2) project $P^\pm(0; \beta_i^\pm, \lambda)Z_i^\pm(0)$ on $E^c(0)$. We will verify this, but it makes sense. We can make this choice for each term, so we will do whatever is easiest.\\

Recall that $Z_i^\pm(0)$ are given by

\begin{align*}
Z_i^-(0) &= \Phi^s(0, -X_{i-1}; \lambda) a_{i-1}^- + b_i^- + (P_0^u(\lambda) - P_0^u(0))b_i^- \\
&+ P_0^c(0) e^{\nu(\lambda) X_{i-1}} c_{i-1}^- + (P_0^c(\lambda) - P_0^c(0)) e^{\nu(\lambda) X_{i-1}} c_{i-1}^- \\
&+ \int_{-X_{i-1}}^0 \Phi^s(0, y; \lambda) [G_i^-(y)Z_i^-(y) + \lambda^2 d_i P^-(y; \beta_i^-, \lambda)^{-1} \tilde{H}_i^-(y)] dy \\
&+ \int_{-X_{i-1}}^0 \Phi^c(0, y; \lambda) [G_i^-(y)Z_i^-(y) + \lambda^2 d_i P^-(y; \beta_i^-, \lambda)^{-1} \tilde{H}_i^-(y)] dy  \\ 
Z_i^+(0) &= \Phi^u(0, X_i; \lambda) a_i^+ + b_i^+ + (P_0^s(\lambda) - P_0^s(0)) b_i^+ \\
&+ P_0^c(0) e^{-\nu(\lambda)X_i} c_i^+ + (P_0^c(\lambda) - P_0^c(0)) e^{-\nu(\lambda)X_i} \\
&+ \int_{X_i}^0 \Phi^u(0, y; \lambda) [G_i^+(y)Z_i^+(y) + \lambda^2 d_i P^+(y; \beta_i^+, \lambda)^{-1} \tilde{H}_i^+(y) ] dy \\
&+ \int_{X_i}^0 \Phi^c(0, y; \lambda) [G_i^+(y)Z_i^+(y) + \lambda^2 d_i P^+(y; \beta_i^+, \lambda)^{-1} \tilde{H}_i^+(y) ] dy \\
\end{align*}

We will look at the lower order terms first. There should be two of them.

\begin{enumerate}
\item We did the terms involving $c$ in the previous section

\begin{align*}
P^-(0; \beta_i^-, \lambda)&[ P_0^c(0) e^{\nu(\lambda) X_{i-1}} c_{i-1}^- + (P_0^c(\lambda) - P_0^c(0)) e^{\nu(\lambda) X_{i-1}} c_{i-1}^-] \\
&= P^-(0; 0, 0) P_0^c(0) \tilde{c}_{i-1}^+ + (P^-(0; \beta_i^-, \lambda) - P^-(0; 0, 0)) P_0^c(0) \tilde{c}_{i-1}^+ + P^-(0; \beta_i^-, \lambda) (P_0^c(\lambda) - P_0^c(0)) \tilde{c}_{i-1}^+ \\
&= P^-(0; 0, 0) P_0^c(0) e^{\nu(\lambda) X_{i-1}} c_{i-1}^- + \mathcal{O}((e^{-\alpha X_m} + |\lambda|)|\tilde{c}_{i-1}^+|)
\end{align*}

and

\begin{align*}
P^+(0; \beta_i^+, \lambda)&[ P_0^c(0) e^{-\nu(\lambda) X_i} c_i^+ + (P_0^c(\lambda) - P_0^c(0)) e^{-\nu(\lambda) X_i} c_i^+] \\
&= P^+(0; \beta_i^+, 0) P_0^c(0) e^{-\nu(\lambda) X_i} c_i^- + \mathcal{O}((e^{-\alpha X_m} + |\lambda|)|\tilde{c}_i^-|) \\
&+ \mathcal{O}\Big( e^{-\tilde{\alpha} X_i} ( |b_i^+| + |b_{i+1}^-| + |c_i^-| + (|\lambda| + |D_i| ) |d|) \Big)
\end{align*}

We do need to plug in $B_1$ for the $b$ terms to get

\begin{align*}
P^+(0; \beta_i^+, \lambda)&[ P_0^c(0) e^{-\nu(\lambda) X_i} c_i^+ + (P_0^c(\lambda) - P_0^c(0)) e^{-\nu(\lambda) X_i} c_i^+] \\
&= P^+(0; \beta_i^+, 0) P_0^c(0) e^{-\nu(\lambda) X_i} c_i^- + \mathcal{O}((e^{-\alpha X_m} + |\lambda|)|\tilde{c}_i^-|) \\
&+ \mathcal{O}\Big( e^{-\tilde{\alpha} X_i} ( |c_i^-| + (|\lambda| + e^{-\tilde{\alpha}X_m})( |\tilde{c}_{i-1}^+| + |\tilde{c}_{i+1}^-|) + e^{-\tilde{\alpha} X_{i-1}} |c_{i-1}^-| + e^{-\tilde{\alpha} X_{i+1}} |c_{i+1}^-|  \\
&+ (|\lambda| + |D_i| ) |d|) \Big)
\end{align*}

In this case, the leading order terms are preserved, which gives us the term

\[
e^{-\nu(\lambda) X_i} c_i^- - e^{\nu(\lambda) X_{i-1}} c_{i-1}^-
\]

\item The center integral term will give us the center Melnikov integral

\begin{align*}
&\langle \Psi^c(0), P^-(0; \beta_i^-, \lambda) \int_{-X_{i-1}}^0 \Phi^c(0, y; \lambda) P^-(y; \beta_i^-, \lambda) \tilde{H}_i^-(y) dy \rangle \\
&= \langle \Psi^c(0), \int_{-X_{i-1}}^0 P^-(0; \beta_i^-, \lambda) \Phi^c(0, y; \lambda) P^-(y; \beta_i^-, \lambda)^{-1} \tilde{H}_i^-(y) dy \rangle \\
&= \langle \Psi^c(0), \int_{-X_{i-1}}^0 P^-(0; 0, 0) \Phi^c(0, y; 0) P^-(y; 0, \lambda)^{-1} \tilde{H}_i^-(y) dy \rangle + \mathcal{O}(|\lambda|) \\
&= \int_{-X_{i-1}}^0 \langle \Psi^c(0), \Theta^c(0, y) \tilde{H}_i^-(y) \rangle dy + \mathcal{O}(|\lambda|) \\
&= \int_{-X_{i-1}}^0 \langle \Theta^c(y, 0)^* \Psi^c(0), H(y) \rangle dy + \int_{-X_{i-1}}^0 \langle \Psi^c(0), \Theta^c(0, y) \Delta H_i^-(y) \rangle dy + \mathcal{O}(|\lambda|) \\
&= \int_{-\infty}^0 \langle \Psi^c(y), H(y) \rangle dy + \int_{-X_{i-1}}^0 \langle \Psi^c(0), \Theta^c(0, y) \Delta H_i^-(y) \rangle dy + \mathcal{O}(e^{-\alpha X_m} + |\lambda|) \\
\end{align*}

For the integral involving $\Delta H_i^-(y)$, we have

\begin{align*}
\left| \int_{-X_{i-1}}^0 \langle \Psi^c(0), \Theta^c(0, y) \Delta H_i^-(y) \rangle dy \right| &\leq C \int_{-X_{i-1}}^0 e^{-\rho y} e^{-\alpha X_{i-1}} e^{-\alpha(X_{i-1} + y)} dy \\
&\leq C e^{-(\alpha - \rho)X_{i-1}} \int_{-X_{i-1}}^0 e^{-\alpha(X_{i-1} + y)} dy \\
&\leq C e^{-\tilde{\alpha}X_{i-1}}
\end{align*}

Thus we have

\begin{align*}
&\langle \Psi^c(0), P^-(0; \beta_i^\pm, \lambda) \int_{-X_{i-1}}^0 \Phi^c(0, y; \lambda) P^-(y; \beta_i^\pm, \lambda) \tilde{H}_i^-(y) dy \rangle \\
&= \int_{-\infty}^0 \langle \Psi^c(y), H(y) \rangle dy + \mathcal{O}(e^{-\tilde{\alpha} X_m} + |\lambda|) \\
\end{align*}

The ``positive'' integral is similar.

\end{enumerate}

The remaining terms are higher order.

\begin{enumerate}

\item For the term involving $a$,

\begin{align*}
P^-(0; \beta_i^-, \lambda) &\Phi^s(0, -X_{i-1}; \lambda) a_{i-1}^- \\
&= -P^-(0; \beta_i^-, \lambda) \Phi^s(0, -X_{i-1}; \lambda) P_0^s(\lambda) D_i d + P^-(0; \beta_i^-, \lambda) \Phi^s(0, -X_{i-1}; \lambda) A_4(\lambda)_i^-(b, c^-, d) 
\end{align*}

Thus we have

\begin{align*}
P^-(0; \beta_i^-, \lambda) &\Phi^s(0, -X_{i-1}; \lambda) a_{i-1}^- 
= P^-(0; 0, 0) \Phi^s(0, -X_{i-1}; 0)a_{i-1}^- \\
&+ \mathcal{O}(e^{-\alpha X_{i-1}}( e^{-\alpha X_m} + |\lambda|)( |D||d| +|A_2(\lambda)_{i-1}^-(b, c^-, d)|
\end{align*}

The first term on the RHS is eliminated outright by the projection. The other is order

\begin{align*}
&\mathcal{O}(e^{-\alpha X_{i-1}}( e^{-\alpha X_m} + |\lambda|)( |D||d| +  
e^{-\alpha X_{i-1}} (|\lambda| + e^{-\tilde{\alpha}X_m})(|\tilde{c}_{i-2}^+| + |\tilde{c}_{i}^-|) + e^{-\tilde{\alpha}X_{i-2}}|c_{i-2}^-| + e^{-\tilde{\alpha}X_{i-1}}|c_{i-1}^-| + e^{-\tilde{\alpha}X_{i}}|c_{i}^-| \\
&+ e^{-\tilde{\alpha} X_m} |\lambda|^2|d| + e^{-\alpha X_m}|D||d| \Big) \\
&\mathcal{O}(e^{-\alpha X_m}( e^{-\alpha X_m} + |\lambda|)(  
e^{-\alpha X_m} (|\lambda| + e^{-\tilde{\alpha}X_m})(|\tilde{c}_{i-2}^+| + |\tilde{c}_{i}^-|) + e^{-\tilde{\alpha}X_{i-2}}|c_{i-2}^-| + e^{-\tilde{\alpha}X_{i-1}}|c_{i-1}^-| + e^{-\tilde{\alpha}X_{i}}|c_{i}^-| \\
&+ e^{-\tilde{\alpha} X_m} |\lambda|^2|d| + |D||d| \Big)
\end{align*}

For the ``plus'' term, we have a remainder term of

\begin{align*}
&\mathcal{O}(e^{-\alpha X_m}( e^{-\alpha X_m} + |\lambda|)(  
e^{-\alpha X_m} (|\lambda| + e^{-\tilde{\alpha}X_m})(|\tilde{c}_{i-1}^+| + |\tilde{c}_{i+1}^-|) + e^{-\tilde{\alpha}X_{i-1}}|c_{i-1}^-| + e^{-\tilde{\alpha}X_{i}}|c_{i}^-| + e^{-\tilde{\alpha}X_{i+1}}|c_{i+1}^-| \\
&+ e^{-\tilde{\alpha} X_m} |\lambda|^2|d| + |D||d| \Big) 
\end{align*}

\item For the terms involving $b$, we have

\begin{align*}
P^-(0; \beta_i^-, \lambda)&( b_i^- + (P_0^u(\lambda) - P_0^u(0))b_i^-) \\
&= P^-(0; 0, 0) b_i^- + \mathcal{O}(|\lambda| + e^{-\alpha X_m})\Big( 
(|\lambda| + e^{-\tilde{\alpha}X_m})( |\tilde{c}_{i-1}^+| + |\tilde{c}_i^-|)
+ e^{-\tilde{\alpha} X_{i-1}} c_{i-1}^- + e^{-\tilde{\alpha} X_i} c_i^- + \\
&( e^{-\tilde{\alpha}X_m} |D| + e^{-\tilde{\alpha}X_m}|\lambda| + |\lambda|^2)|d| \Big)
\end{align*}

The first term on the RHS is eliminated outright by the projection.

\item For the noncenter integral terms involving $\tilde{H}$, since this mostly evolves in the stable subspace, we should have

\begin{align*}
&P^-(0; \beta_i^-, \lambda) 
\int_{-X_{i-1}}^0 \Phi^s(0, y; \lambda) \lambda^2 d_i P^-(y; \beta_i^-, \lambda)^{-1} \tilde{H}_i^-(y) dy \\
&= P^-(0; 0, 0) 
\int_{-X_{i-1}}^0 \Phi^s(0, y; 0) \lambda^2 d_i P^-(0; 0, 0)^{-1} H_i^-(y) dy + \mathcal{O}((|\lambda| + e^{-\alpha X_m})|\lambda|^2|d|) 
\end{align*}
 
Of course, we should check this. The first term on the RHS is eliminated by the projection.

\item For the integral terms involving $Z$, these will be bounded by the center one, so we only have to do that one.

\begin{align*}
&\left| \int_{-X_{i-1}}^0 \Phi^c(0, y; \lambda) G_i^-(y)Z_i^-(y) \right| dy \\ 
&\leq C \int_{-X_{i-1}}^0 e^{-\rho y} e^{-\alpha X_{i-1}}e^{-\alpha({X_{i-1} + y)}}||Z_3(\lambda)(b,c^-,d)_i^-|| dy \\
&\leq C e^{-(\alpha - \rho)X_{i-1}} ( (|\lambda| + e^{-\tilde{\alpha}X_m})(|\tilde{c}_i^-| + |\tilde{c}_{i-2}^+|) + e^{-\tilde{\alpha} X_{i-2}} |c_{i-2}^-| + e^{-\tilde{\alpha} X_i} |c_i^-| + e^{\rho X_{i-1}}|c_{i-1}^-| \\ 
&+ |\lambda|^2 |d| + e^{-(\alpha - \rho)X_{i-1}}|\lambda||d| + |D_{i-1}||d| ) \\
&\leq C e^{-\tilde{\alpha} X_{i-1}} ( (|\lambda| + e^{-\tilde{\alpha}X_m})(|\tilde{c}_i^-| + |\tilde{c}_{i-2}^+|) + e^{-\tilde{\alpha} X_{i-2}} |c_{i-2}^-| + e^{-\tilde{\alpha} X_i} |c_i^-| + |c_{i-1}^-| \\ 
&+ |\lambda|^2 |d| + e^{-\tilde{\alpha}X_m}|\lambda||d| + |D_{i-1}||d| )
\end{align*}

Similarly,

\begin{align*}
&\left| \int_{-X_i}^0 \Phi^c(0, y; \lambda) G_i^+(y)Z_i^+(y) \right| dy \\ 
&\leq C e^{-\tilde{\alpha} X_i} ( (|\lambda| + e^{-\tilde{\alpha}X_m})(|\tilde{c}_{i+1}^-| + |\tilde{c}_{i-1}^+|) + e^{-\tilde{\alpha} X_{i-1}} |c_{i-1}^-| + e^{-\tilde{\alpha} X_{i+1}} |c_{i+1}^-| + |c_i^-| \\
&+ |\lambda|^2 |d| + e^{-(\alpha - \rho)X_i}|\lambda||d| + |D_i||d| )
\end{align*}

\end{enumerate}

Believe it or not, this is all of them. Putting all of this together, we have

\begin{align*}
\xi^c_i = e^{-\nu(\lambda) X_i} c_i^- - e^{\nu(\lambda) X_{i-1}} c_{i-1}^- - \lambda_2 d_i M^c + R^c(\lambda)(c, \tilde{c}, d)
\end{align*}

where $M^c$ is the center Melnikov integral

\[
\int_{-\infty}^\infty \langle \Psi^c(y), H(y) \rangle dy 
\]

and the remainder term $R^c(c, \tilde{c}, d)$ has bound

\begin{align*}
R^c&(c, \tilde{c}, d)_i \leq C \Big( \\
&(|\lambda + e^{-\alpha X_m})(|\tilde{c}_{i-1}^+| + |\tilde{c}_{i}^-| + e^{-\alpha X_m}( |\tilde{c}_{i-2}^+| + |\tilde{c}_{i+1}^-|) ) \\
&+ e^{-\alpha X_i} |c_i^-| + e^{-\alpha X_m}( e^{-\alpha X_{i-1}} |c_{i-1}^-| + e^{-\alpha X_{i-2}} |c_{i-2}^-| + e^{-\alpha X_{i+1}} |c_{i+1}^-| \\
&+ e^{-\tilde{\alpha} X_m} (|\lambda| + |D|)|d|
\Big)
\end{align*}

This is horrible. But at least we know exactly which of the $c_i$ are involved where. We will write out only the matrices to solve for the $c_i^-$, since those are the ones we are worried about.\\

We (more or less) have the matrix equation

\[
(C_1 K(\lambda)+ C_2) c_i^- = 0
\]

where

\begin{align*}
K(\lambda) =  
\begin{pmatrix}
e^{-\nu(\lambda)X_1} & & & & & -e^{\nu(\lambda)X_0} \\
-e^{\nu(\lambda)X_1} & e^{-\nu(\lambda)X_2} \\
& -e^{\nu(\lambda)X_2} & e^{-\nu(\lambda)X_3} \\
\vdots & & \vdots & &&  \vdots \\
& & & & -e^{\nu(\lambda)X_{n-1}} & e^{-\nu(\lambda)X_0} 
\end{pmatrix}
\end{align*}

and

\begin{align*}
C_1 &= I + \mathcal{O}(|\lambda| + e^{-\alpha X_m}) I 
+ \mathcal{O}(e^{-\alpha X_m}( |\lambda| + e^{-\alpha X_m}))\\
C_2 &= \mathcal{O}(e^{-\alpha X_m}) I + \mathcal{O}(e^{-2 \alpha X_m})
\end{align*}

IGNORE REST OF THIS

\\

\subsubsection*{Jump in Psi direction}

For this jump, we project on $\Psi_i(0)$.

\[
\xi_i = \langle \Psi_i(0), P_i^+(0; \lambda) Z_i^+(0) - P_i^-(0; \lambda) Z_i^-(0) \rangle
\]

Recall that $Z_i^\pm(0)$ are given by

\begin{align*}
Z_i^-(0) &= \Phi^s(0, -X_{i-1}; \lambda) a_{i-1}^- + b_i^- + (P_0^u(\lambda) - P_0^u(0))b_i^- \\
&+ e^{\nu(\lambda) X_{i-1}} P_0^c(0) c_{i-1}^- + e^{\nu(\lambda) X_{i-1}} (P_0^c(\lambda) - P_0^c(0))c_{i-1}^- \\
&+ \lambda^2 d_i \int_{-X_{i-1}}^0 \Phi^s(0, y; \lambda) P_i^-(y; \lambda)^{-1} \tilde{H}_i^-(y) dy 
+ \lambda^2 d_i \int_{-X_{i-1}}^0 \Phi^c(0, y; \lambda) P_i^-(y; \lambda)^{-1} \tilde{H}_i^-(y) dy  \\ 
Z_i^+(0) &= \Phi^u(0, X_i; \lambda) a_i^+ + b_i^+ + (P_0^s(\lambda) - P_0^s(0)) b_i^+ \\
&+ e^{-\nu(\lambda)X_i} P_0^c(0) c_i^+ + e^{-\nu(\lambda)X_i} (P_0^c(\lambda) - P_0^c(0))c_i^+ \\
&+ \lambda^2 d_i \int_{X_i}^0 \Phi^u(0, y; \lambda) P_i^+(y; \lambda)^{-1} \tilde{H}_i^+(y) dy 
+ \lambda^2 d_i \int_{X_i}^0 \Phi^c(0, y; \lambda) P_i^+(y; \lambda)^{-1} \tilde{H}_i^+(y) dy \\
\end{align*}

We will compute the significant terms first. The noncenter integral will give us the Melnikov integral. For the ``minus'' piece, we have

\begin{align*}
&\langle \Psi_i(0), P_i^-(0; \lambda) \int_{-X_{i-1}}^0 \Phi^s(0, y; \lambda) P_i^-(y; \lambda)^{-1} \tilde{H}_i^-(y) dy \rangle \\
&= \int_{-X_{i-1}}^0 \langle \Psi_i(0), P_i^-(0; \lambda), \Phi^s(0, y; \lambda) P_i^-(y; \lambda)^{-1} \tilde{H}(y) \rangle dy \\
&= \int_{-X_{i-1}}^0 \langle \Psi_i(0), P_i^-(0; \lambda), \Phi^s(0, y; \lambda) P_i^-(y; \lambda)^{-1} H(y) \rangle dy + \mathcal{O}({e^{-\alpha X_m}})\\
&= \int_{-X_{i-1}}^0 \langle \Psi_i(0), \Theta_i^s(0, y; \lambda) H(y) \rangle dy + \mathcal{O}({e^{-\alpha X_m}})\\
&= \int_{-X_{i-1}}^0 \langle \Psi_i(0), \Theta_i^s(0, y; 0) H(y) \rangle dy + \mathcal{O}(|\lambda| + {e^{-\alpha X_m}})\\
&= \int_{-X_{i-1}}^0 \langle \Theta_i^s(y, 0; 0)^* \Psi_i(0), H(y) \rangle dy + \mathcal{O}(|\lambda| + {e^{-\alpha X_m}})\\
&= \int_{-X_{i-1}}^0 \langle \Psi_i(y), H(y) \rangle dy + \mathcal{O}(|\lambda| + {e^{-\alpha X_m}})\\
&= \int_{-X_{i-1}}^0 \langle \Psi(y), H(y) \rangle dy + \mathcal{O}(|\lambda| + {e^{-\alpha X_m}})\\
\end{align*}

Note that in the last step, we put this in terms of $\Psi(x)$, the solution to the adjoint variational problem for the primary pulse.\\

Next, we look at the terms involving $a$. For these, we plug in $A_4$.

\begin{align*}
\langle &\Psi_i(0), P_i^-(0; \lambda) \Phi^s(0, -X_{i-1}; \lambda) a_{i-1}^- \rangle \\
&= \langle \Psi_i(0), P_i^-(0; \lambda) \Phi^s(0, -X_{i-1}; \lambda) (- P_i^-(-X_{i-1}; \lambda)^{-1} P_0^s(\lambda) D_{i-1} d + A_4(\lambda)_i^-(\tilde{c}, d)) \rangle \\
&= -\langle \Psi_i(0), \Theta_i^s(0, -X_{i-1}; \lambda) P_0^s(\lambda) D_{i-1} d \rangle + \mathcal{O}( e^{-(\alpha + \tilde{\alpha})X_m}(|\tilde{c}| + |\lambda|^2 |d| + |D||d|) \\
&= -\langle \Psi_i(0), \Theta_i^s(0, -X_{i-1}; 0) P_0^s(0) D_{i-1} d \rangle + \mathcal{O}( e^{-(\alpha + \tilde{\alpha})X_m}(|\tilde{c}| + |\lambda||d| + |D||d|) \\
&= -\langle \Theta_i^s(-X_{i-1}, 0; 0)^* \Psi_i(0), P_0^s(0) D_{i-1} d \rangle + \mathcal{O}( e^{-(\alpha + \tilde{\alpha})X_m}(|\tilde{c}| + |\lambda||d| + |D||d|) \\
&= -\langle \Psi_i(-X_{i-1}), P_0^s(0) D_{i-1} d \rangle + \mathcal{O}( e^{-(\alpha + \tilde{\alpha})X_m}(|\tilde{c}| + (|\lambda| + |D|)|d|) \\
\end{align*}

The remaining terms will be higher order. Doing these in turn, we have

\begin{enumerate}
\item For the terms involving $b$,

\begin{align*}
\langle \Psi(0), P_i^-(0; \lambda) \Phi^u(0, 0; \lambda) b_i^- \rangle
&= \langle \Psi(0), (P_i^-(0; 0) + \mathcal{O}(|\lambda|))(I + \mathcal{O}(|\lambda|)) b_i^- \rangle \\
&= \langle \Psi(0), P_i^-(0; 0) b_i^- \rangle + \mathcal{O}(|\lambda||b_i^-|) \\
&= |\lambda| \Big(
(|\lambda| + e^{-\tilde{\alpha}X_m})|\tilde{c}|
+ (e^{-\tilde{\alpha}X_m}|D| 
+ |\lambda|^2)|d|
\Big)
\end{align*}

where we substituted in $B_1$.

\item For the terms involving $\tilde{c}$, we have

\begin{align*}
\langle \Psi(0), P_i^-(0; \lambda)e^{\nu(\lambda) X_{i-1}} P^c(\lambda) c_{i-1}^-\rangle &= 
\langle \Psi(0), (P_i^-(0; 0) + \mathcal{O}(\lambda))(I + \mathcal{O}(\lambda)) \tilde{c}_{i-1}^+ \rangle \\
&= \mathcal{O}(|\lambda|) \tilde{c}_{i-1}^+
\end{align*}

We also have to do the same substitution as above to convert $c_i^+$ to $c_i^+$, as we did above. As above, we have

\begin{align*}
e^{-\nu(\lambda)X_i} c_i^+ &= e^{-\nu(\lambda)X_i} c_i^- + \mathcal{O}\Big( e^{-\tilde{\alpha} X_m} |\tilde{c}| + e^{-\tilde{\alpha} X_m}(|\lambda| + |D| ) |d| \Big) \\
\end{align*}

which gives us

\begin{align*}
\langle \Psi(0), P_i^+(0; \lambda)e^{-\nu(\lambda) X_i} P^c(\lambda) c_i^+ \rangle 
&= \mathcal{O}(|\lambda|) \tilde{c}_i^- + \mathcal{O}\Big( |\lambda|( e^{-\tilde{\alpha} X_m} |\tilde{c}| + e^{-\tilde{\alpha} X_m}(|\lambda| + |D| ) |d| ) \Big)
\end{align*}

\item For the center integral term, we have

\begin{align*}
&\langle \Psi(0), P_i^-(0; \lambda)
\int_{-X_{i-1}}^0 \Phi^c(0, y; \lambda) P_i^-(y; \lambda)^{-1} \tilde{H}_i^-(y) dy \rangle \\
&= \int_{-X_{i-1}}^0 \langle \Psi(0), P_i^-(0; \lambda) \Phi^c(0, y; \lambda) P_i^-(y; \lambda)^{-1} \tilde{H}_i^-(y) \rangle dy \\
&= \int_{-X_{i-1}}^0 \langle \Psi(0), \Theta^c(0, y; \lambda) \tilde{H}_i^-(y) \rangle dy \\
&= \int_{-X_{i-1}}^0 \langle \Psi(0), \Theta^c(0, y; 0) \tilde{H}_i^-(y) \rangle dy + \mathcal{O}(|\lambda|) \\
&= \mathcal{O}(|\lambda|)
\end{align*}

\end{enumerate}

Putting this all together, we have

\begin{align*}
\xi_i = \langle \Psi(X_i), P_0^u(0) D_i d \rangle
+ \langle \Psi(-X_{i-1}), P_0^s(0) D_{i-1} d \rangle 
+ \lambda_2 d_i M + R(\lambda)(\tilde{c}, d)
\end{align*}

where $M$ is the higher order Melnikov integral

\[
\int_{-\infty}^\infty \langle \Psi(y), H(y) \rangle dy 
\]

and the remainder term has bound

\begin{align*}
|R(\lambda)(\tilde{c}, d)| \leq C \Big(
|\lambda||\tilde{c}| + |\lambda|(|\lambda|^2 + e^{-\alpha X_m}|\lambda| + e^{-(\alpha + \tilde{\alpha}) X_m})|d| + (e^{-\alpha X_m} |\lambda| + e^{-(\alpha + \tilde{\alpha}) X_m})|D||d|
 \Big)
\end{align*}

Using the fact that $|D| = \mathcal{O}(e^{-\alpha X_m})$, this becomes

\begin{align*}
|R(\lambda)(\tilde{c}, d)| \leq C \Big(
|\lambda||\tilde{c}| + (|\lambda|^3 + e^{-\alpha X_m}|\lambda|^2 + e^{-(\alpha + \tilde{\alpha}) X_m}|\lambda| + e^{-(2 \alpha + \tilde{\alpha}) X_m})|d|
 \Big)
\end{align*}



\end{document}