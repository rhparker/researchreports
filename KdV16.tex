% \documentclass{book}

\documentclass[12pt]{article}
\usepackage[pdfborder={0 0 0.5 [3 2]}]{hyperref}%
\usepackage[left=1in,right=1in,top=1in,bottom=1in]{geometry}%
\usepackage[shortalphabetic]{amsrefs}%
\usepackage{amsmath}
\usepackage{enumerate}
\usepackage{enumitem}
\usepackage{amssymb}                
\usepackage{amsmath}                
\usepackage{amsfonts}
\usepackage{amsthm}
\usepackage{bbm}
\usepackage[table,xcdraw]{xcolor}
\usepackage{tikz}
\usepackage{float}
\usepackage{booktabs}
\usepackage{svg}
\usepackage{mathtools}
\usepackage{cool}
\usepackage{url}
\usepackage{graphicx,epsfig}
\usepackage{makecell}
\usepackage{array}

\def\noi{\noindent}
\def\T{{\mathbb T}}
\def\R{{\mathbb R}}
\def\N{{\mathbb N}}
\def\C{{\mathbb C}}
\def\Z{{\mathbb Z}}
\def\P{{\mathbb P}}
\def\E{{\mathbb E}}
\def\Q{\mathbb{Q}}
\def\ind{{\mathbb I}}

\graphicspath{ {images16/} }

\begin{document}

\section*{23 July 2017}

\subsection*{Integrated Eigenvalue Problem, periodic BCs}

We will take another look at the eigenvalue problem for the linearized 5th order problem about a stationary solution $u^*(x)$, but this time we will assume we have periodic BCs. Our domain will be $[-L, L]$.\\

We write the linearized 5th order problem about stationary solution $u^*(x)$ as $Lv = \partial_x Hv = \lambda v$, where
\[
L = \partial_x( \partial_x^4 - \partial_x^2 + c - 2u^*)
\]
So we have 
\[
H = \partial_x^4 - \partial_x^2 + c - 2u^*
\]

So far, this is nothing new. We will do the same setup as before, i.e. split the problem into left and right halves and look for matching conditions for the center.\\

First, integrate both sides of $\partial_x Hv = \lambda v$ from $x\geq 0$ to $L$. Assume we have a solution $v^+(x)$ which is defined on $[0, L]$ and satisfies this integrated equation. Then $v^+(x)$ satisfies the following equation:
\begin{align*}
(Hv^+)(L) -(Hv^+)(x) = \lambda \int_x^L v^+(y) dy && x \in [0, L]
\end{align*}

Similary, integrate both sides of $\partial_x Hv = \lambda v$ from $-L$ to $x \leq 0$. Assume we have a solution $v^-(x)$ which is defined on $[-L,0]$ and satisfies this integrated equation. Then $v^-(x)$ satisfies the following equation:

\begin{align*}
(Hv^-)(x) - (Hv^-)(-L) = \lambda \int_{-L}^x v^-(y) dy && x \in [-L, 0]
\end{align*}

Take $x = 0$, and add the two equations together to get

\[
[(Hv^+)(L) - (Hv^-)(-L)] - [(Hv^+)(0) - (Hv^-)(0)] = \lambda \int_{-L}^L v(y) dy
\]
where $v(y)$ is defined by ``pasting together'' $v^+$ and $v^-$:

\[
v(y) = \begin{cases}
v^+(y) & y \in [0, L] \\
v^-(y) & y \in [-L, 0]
\end{cases}
\]

First, in place of our assumption from before that the eigensolutions and all derivatives decay to 0 at $\pm \infty$, we will assume that we have constructed $v^+$ and $v^-$ so that those two functions and all their derivatives are equal at $\pm L$. (We should be able do this as in Sandstede (1998) following roughly what I did in my \texttt{KdV13} report). Thus when we do the pasting to get $v(y)$, the function $v(y)$ and its derivatives will not have discontinuities when we do the periodic extension. This implies that $(Hv^+)(L) = (Hv^-)(-L)$, so the problem reduces to:

\[
[(Hv^+)(0) - (Hv^-)(0)] = \lambda \int_{-L}^L v(y) dy
\]

As before we will asuume that $v^\pm(x)$ and its first three derivatives agree at 0. This reduces to 
\[
\partial_x^4 v^-(0) - \partial_x^4 v^+(0) = \lambda \int_{-L}^L v(y) dy
\]

In other words, as long as $\lambda$ is not 0 and $v^+(x)$ and $v^-(x)$ agree at 0 up to the third derivative, the fourth derivatives of $v^+(x)$ and $v^-(x)$ agree at 0 if and only if the above integral is 0. The big question is whether we can show that the above integral is 0.\\

Restricting ourselves to funcions which are periodic on $[-L, L]$, suppose we have a function $u$ which satisfies $\partial_x H u = \lambda u$. This function (and all derivatives that exist) are periodic on $[-L, L]$. Now take the inner product on this periodic space of this with the constant function 1.
\begin{align*}
<\partial_x H u, 1> &= <\lambda u, 1> \\
\int_{-L}^L \partial_x H u dx &= \lambda \int_{-L}^L dx \\
0 &= \lambda \int_{-L}^L dx
\end{align*}
where the last line follows from the fundamental theorem of calculus. So if we assume we have a periodic eigenfunction for the eigenvalue problem $\partial_x H u = \lambda u$, the eigenfunction must have mean 0.\\

Let's see if we can use a similar idea to show that our the integral of our constructed eigenfunction $v(x)$ is 0 on $[-L, L]$. First, let's recall what we know about $v(x)$.

\begin{enumerate}
	\item $v(-L) = v(L)$, and similar for all derivatives which exist.
	\item Since $v^+(x)$ and $v^-(x)$ are smooth, and since we assume that the derivatives up to third order match at $x = 0$, the function $v(x)$ is of class $C^3$ on $[-L, L]$.
\end{enumerate}

From the numerics, we know a few more things about $v(x)$. Recall that the idea is to (eventually) show that $v(x)$ is a eigenfunction of the 5th order eigenvalue problem $\partial_x H v = \lambda v$. For real $\lambda$, numerics suggests that $v(x)$ is an even function. For purely imaginary $\lambda$, numerics suggests that we can find a unit complex number $z$ such that the real part of $z v(x)$ is even and the imaginary part of $z v(x)$ is odd. In particular, we should not be able to show that $v(x)$ is an odd function, which would make all of this easy if it were true.\\

Recall that we are looking at 

\[
\partial_x^4 v^-(0) - \partial_x^4 v^+(0) = \lambda \int_{-L}^L v(y) dy
\]

and want to show that the integral on the RHS is 0. The only chance we have is to manipulate the RHS in some way and to use the fact that $v(x)$ is periodic. The most promising thing to do is to integrate by parts, so let's try that.

\begin{align*}
\int_{-L}^L v(y) dy &= \int_{-L}^L 1 v(y) dy\\
&= v(y)y\Big|_{-L}^L - \int_{-L}^L v_y y dy\\
&= L v(L) + Lv(-L) - \int_{-L}^L v_y y dy\\
&= 2Lv(L) - \int_{-L}^L v_y y dy
\end{align*}
The function $y$ is not periodic on $[-L, L]$, so the boundary term does not cancel. In addition, the integral is not guaranteed to be 0. In fact, if $v$ is even (which we think is the case), then $v_y$ is odd, so $v_y y$ is even, and the integral is not necessarily 0. We can keep integrating by parts, but that will only give us more boundary terms and more integrals we don't know anything about. \\

So unfortunately this does not appear to lead anywhere. Here is why I think this is not as useful as we had hoped. Recall that from the numerics we expect the eigenfunction $v(x)$ to be even (real eigenvalue) or for the real part of $v(x)$ to be even (purely imaginary eigenvalue; in this case the imaginary part is odd, so it integrates to 0 always and we can ignore it). Using what we did before (see \texttt{KdV12}), the integral condition reduces to either $\int_0^L v(y) dy = 0$ (periodic case) or $\int_0^\infty v(y) dy = 0$ (unbounded case), i.e. we need to show that the integral on half the domain is 0. Periodicity unfortunately does not help with this.

\end{document}


