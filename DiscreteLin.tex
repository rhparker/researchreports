\documentclass[12pt]{article}
\usepackage[pdfborder={0 0 0.5 [3 2]}]{hyperref}%
\usepackage[left=1in,right=1in,top=1in,bottom=1in]{geometry}%
\usepackage[shortalphabetic]{amsrefs}%
\usepackage{amsmath}
\usepackage{enumerate}
% \usepackage{enumitem}
\usepackage{amssymb}                
\usepackage{amsmath}                
\usepackage{amsfonts}
\usepackage{amsthm}
\usepackage{bbm}
\usepackage[table,xcdraw]{xcolor}
\usepackage{tikz}
\usepackage{float}
\usepackage{booktabs}
\usepackage{svg}
\usepackage{mathtools}
\usepackage{cool}
\usepackage{url}
\usepackage{graphicx,epsfig}
\usepackage{makecell}
\usepackage{array}

\def\noi{\noindent}
\def\T{{\mathbb T}}
\def\R{{\mathbb R}}
\def\N{{\mathbb N}}
\def\C{{\mathbb C}}
\def\Z{{\mathbb Z}}
\def\P{{\mathbb P}}
\def\E{{\mathbb E}}
\def\Q{\mathbb{Q}}
\def\ind{{\mathbb I}}

\DeclareMathOperator{\spn}{span}
\DeclareMathOperator{\ran}{ran}

\newtheorem{lemma}{Lemma}
\newtheorem{theorem}{Theorem}
\newtheorem{corollary}{Corollary}
\newtheorem{definition}{Definition}
\newtheorem{assumption}{Assumption}
\newtheorem{hypothesis}{Hypothesis}

\newtheorem{notation}{Notation}

\graphicspath{ {discrete/} }

\begin{document}

We will extend Lin's method to prove existence and stability results for multi-pulse on lattices. The model equation here is discrete NLS, but we will do this in a general setting, for which dNLS is a special case. Most of the results for dNLS are already known, at least if we are near the anti-continuum limit (see, for example, Pelinovsky and Kevrekidis, 2005). The idea is that if we can find an appropriate generalization and setup for Lin's method, we can check it on dNLS. 

\section{Setup and Main Theorems}

\subsection{Background}
Consider the lattice PDE on $\R^s$
\begin{equation}\label{latticePDE}
\dot{u}_n = f(u_n)
\end{equation}
where $f$ is a smooth nonlinear operator which involves involves a finite number (greater than 1) of indices near $n$. We will make one of the following two assumptions regarding symmetry of \eqref{latticePDE}.

\begin{hypothesis}\label{symmetryhyp}
There exists a group $G$ for which the group action is unitary group of symmetries $t(\theta)$ on $\R^s$ such that $f(t(\theta)u) = t(\theta)f(u)$ for all $\theta \in G$ and all $u \in \R^s$. For the group $G$, we have one of the following:
\begin{enumerate}[(i)]
\item $G$ is a finite group.
\item $G = (\R, +)$.
\end{enumerate}
\end{hypothesis}

An equilibrium solution satisfies the stationary equation 
\begin{equation}\label{stationaryeq}
f(u_n) = 0
\end{equation}
We can rewrite the stationary equation as a system of first-order difference equations of the form
\begin{equation}\label{diffeq}
U(n+1) = F(U(n))
\end{equation}
where $U \in \R^d$ for some $d > 1$ which depends on how many indices near $n$ are used by $f$. Since $f$ is smooth, $F$ is as well. The group $G$ also generates a group action on $\R^d$; this action is a unitary group of symmetries $T(\theta)$ on $\R^d$ such that $F(T(\theta)U) = T(\theta)F(U)$ for all $\theta \in G$ and all $U \in \R^d$. We take the following hypothesis regarding $F$.

\begin{hypothesis}\label{initialhyp}
We assume the following regarding \eqref{diffeq}.
\begin{enumerate}[(i)]
\item 0 is a hyperbolic equilibrium for $F$, i.e. $F(0) = 0$ and $DF(0)$ has no eigenvalues on the unit circle. Thus we can find a radius $r > 1$ such that for all eigenvalues $\nu$ of $DF(0)$ we have $|\nu| \leq 1/r$ or $|\nu| > r$.
\item $\dim E^s, \dim E^u \geq 1$, where $E^s$ and $E^u$ are the stable and unstable eigenspaces of $DF(0)$.
\item There exists a symmetric homoclinic orbit (primary pulse) solution $Q(n)$ to \eqref{diffeq} which connects the equilibrium at 0 to itself.
\end{enumerate}
\end{hypothesis}

Finally, we will take the following hypothesis regarding the intersection of the stable manifold $W^s(0)$ and unstable manifold $W^u(0)$. This hypothesis depends on which symmetry hypothesis we choose.

\begin{hypothesis}\label{intersectionhyp}
If Hypothesis \ref{symmetryhyp}(i) holds, then the stable and unstable manifolds $W^s(0)$ and $W^u(0)$ intersect transversely.

If Hypothesis \ref{symmetryhyp}(ii) holds, then the tangent spaces of the stable and unstable manifolds $W^s(0)$ and $W^u(0)$ have a one-dimensional intersection at $Q(n)$ given by $S_1(n) = S Q(n)$, where $S$ is the infinitesimal generator of $T(\theta)$.
\end{hypothesis}

If follows from the symmetry relation $F(T(\theta)u) = T(\theta)F(u)$ that $U(n)$ is a solution to \eqref{diffeq} if and only if $T(\theta)U(n)$ is also a solution. We note that for Hypothesis \ref{intersectionhyp}(i), the group $G$ can be the trivial group consisting only of the identity.

\subsection{Existence Problem}
First, we will prove the existence of multi-pulse solutions to \eqref{latticePDE}. These are solutions that resemble multiple, well-separated copies of the primary pulse $Q(n)$.

Choose an integer $m > 1$ (the number of pulses), and let $N_i$ ($i = 1, \dots, m-1$) be the desired distances (in lattice points) between the $m$ copies of $Q(n)$. Let 
\begin{equation}\label{Nipm}
\begin{aligned}
N_i^+ &= \lfloor \frac{N_i}{2} \rfloor \\
N_i^- &= N_i - N_i^+
\end{aligned}
\end{equation}
and note that either $N_i^+ = N_i^-$ (if $N_i$ is even) or $N_i^+ = N_i^- - 1$ (if $N_i$ is odd). Let $N = \min\{ N_i^\pm \}$. In addition, let $N_0^- = -\infty$ and $N_m^+ = \infty$. We will look for a solution which is a piecewise perturbation of $n$ consecutive copies of $Q(n)$, i.e. a solution of the form
\begin{align}\label{Upiecewise}
U_i^-(n) &= T(\theta_i) Q(n) + W_i^-(n) && n \in [-N_{i-1}^-, 0] \\
U_i^+(n) &= T(\theta_i) Q(n) + W_i^+(n) && n \in [0, N_i^+]
\end{align}
where $\theta_i = G$.

Since we require that the pieces match at the interval endpoints, $U_i^\pm$ must satisfy the system of equations
\begin{equation}\label{Usystem}
\begin{aligned}
(U_i^\pm)(n+1) &= F(U_i^\pm(n))  \\
U_i^+(N_i^+) - U_{i+1}^-(-N_i^-) &= 0 \\
U_i^+(0) - U_i^-(0) &= 0
\end{aligned}
\end{equation}
for $i = 1, \dots, m$. Expand $F$ in a Taylor series about $T(\theta_i) Q(n)$ to get
\begin{align*}
F(U_i^\pm(n)) &= F(T(\theta_i) Q(n) + W_i^\pm(n)) \\
&= F(T(\theta_i) Q(n)) + DF_{U}(T(\theta_i) Q(n)) W_i^\pm + G(W_i^\pm(n)) \\
&= A(n; \theta_i) W_i^\pm + G(W_i^\pm(n))
\end{align*}
where
\begin{align*}
A(n; \theta_i) &= DF(T(\theta_i) Q(n))\\
G(W_i^\pm(n)) &= \mathcal{O}(|W_i^\pm|^2)
\end{align*}
with $G(0) = 0$ and $DG(0) = 0$. Finally, let 
\begin{equation}\label{defdi}
d_i = T(\theta_{i+1}) Q(-N_i^-) - T(\theta_i) Q(N_i^+)
\end{equation}
where $d_i \mathcal{O}(r^{-N_i/2})$. Substituting these into \eqref{Usystem}, we obtain the following system of equations for the piecewise function $W_i^\pm$.
\begin{align}
W_i^\pm(n+1) &= A(n; \theta_i) W_i^\pm + G(W_i^\pm(n)) \label{Wsystem1} \\
W_i^+(N_i^+) - W_{i+1}^-(-N_i^-) &= d_i \label{Wsystem2} \\
W_i^+(0) - W_i^-(0) &= 0 \label{Wsystem3}
\end{align}

The variational and adjoint variational equations associated with \eqref{diffeq} are
\begin{align}
V(n+1) &= A(n; 0) V(n) \label{vareq} \\
Z(n+1) &= [A(n; 0)^*]^{-1} Z(n) \label{adjvareq} 
\end{align}
If we take Hypothesis \ref{symmetryhyp}(i), the variational equation does not have a bounded solution, and we have the decomposition $\R^d = Y^+ \oplus Y^-$, where $Y^+ = T_{Q(0)} W^s(0)$ and $Y^- = T_{Q(0)} W^u(0)$.

If we instead take Hypothesis \ref{symmetryhyp}(ii), the variational equation \eqref{vareq} has a bounded solution $S_1(n)$, which is unique up to scalar multiples. To see this, recall that $S_1(n) = S Q(n)$, where $S$ is the infinitesimal generator for the symmetry group $T(\theta)$, which is defined by
\begin{equation}
S = \frac{d}{d \theta} T(\theta)\Big|_{\theta = 0}
\end{equation}
Differentiating $T(\theta) Q(n+1) = F( T(\theta)Q(n))$ with respect to $\theta$ gives us
\[
\frac{d}{d \theta}T(\theta) Q(n+1) = F( T(\theta)Q(n)) \frac{d}{d \theta}T(\theta) Q(n)
\]
Taking $\theta = 0$, since $F$ is smooth and $T(0) = I$, this becomes $S Q(n+1) = DF(Q(n) S Q(n)$, which is what we want. From Hypothesis \ref{intersectionhyp}(ii), the variational equation \eqref{vareq} has no other bounded solution. It follows that the adjoint variational equation \eqref{adjvareq} has a unique bounded solution $Z_1(n)$. Thus we can decompose the tangent spaces to $W^s(0)$ and $W^u(0)$ at $Q(0)$ as
\begin{align*}
T_{Q(0)} W^u(0) &= Y^- \oplus \R S_1(0)
T_{Q(0)} W^s(0) &= Y^+ \oplus \R S_1(0)
\end{align*}
By Lemma \ref{evolop} below, $Z_1(0) \perp S_1(0) \oplus Y^- \oplus Y^+$, thus we can decompose $\R^d$ as
\begin{equation}\label{nontdecomp}
\R^d = \R S_1(0) \oplus Y^+ \oplus Y^- \oplus \R Z_1(0)
\end{equation}
If Hypothesis \ref{symmetryhyp}(ii) holds, we may without loss of generality choose $W_i^\pm$ so that 
\begin{equation}\label{W0loc}
W_i^\pm(0) \in T(\theta_i) Y^+ \oplus T(\theta_i) Y^- \oplus \R T(\theta_i) Z_1(0)
\end{equation}
since we are using the parameter $\theta_i$ for the direction $T(\theta_i) S_1$ along the symmetry.

We have the following theorems regarding existence of multi-pulse solution to \eqref{diffeq}. Proofs will follow in a later section. First, if Hypothesis \ref{symmetryhyp}(i) holds, we have a unique $m-$pulse solution to \eqref{diffeq}.

\begin{theorem}\label{transversemulti}
Consider the difference equation
\begin{equation}\label{diffeqm1}
U(n+1) = F(u(n))
\end{equation}
where $U \in R^d$ for $d > 1$. Assume Hypotheses \ref{symmetryhyp}(i), \ref{initialhyp}, and \ref{intersectionhyp}. Let $Q(n)$ be the primary pulse solution to \eqref{diffeqm1}. Choose $m > 1$ and positive integers $N_1, \dots, N_{m-1}$, and define $N_i^\pm$ as in \eqref{Nipm}. Choose any $\theta_1, \dots, \theta_m \in G$. Then, for sufficiently large $N$, where $N = \min\{ N_i^\pm : i = 1, \dots, m-1 \}$, there exists a unique $m-$pulse solution $Q_m$ to \eqref{diffeq} which can be written piecewise as 
\begin{align}
U_i^-(n) &= T(\theta_i) Q(n) + W_i^-(n) && n \in [-N_{i-1}^-, 0] \\
U_i^+(n) &= T(\theta_i) Q(n) + W_i^+(n) && n \in [0, N_i^+]
\end{align}
where $N_0 = N_m = \infty$. For the remainder terms $W_i^\pm(n)$ we have the following estimates
\begin{equation}\label{Westimates}
\begin{aligned}
||W_i^\pm|| &\leq C r^{-N} \\
W_i^+(N_i^+) &= T(\theta_{i+1}) Q(-N_i^-) + \mathcal{O}(r^{-2N}) \\
W_{i+1}^-(-N_i^-) &= T(\theta_i) Q(N_i^+) + \mathcal{O}(r^{-2N})
\end{aligned}
\end{equation}
\end{theorem}

If \ref{symmetryhyp}(ii) holds, the following theorem gives a criterion for multi-pulse solutions to exist.

\begin{theorem}\label{ntmulti}
Consider the difference equation
\begin{equation}\label{diffeqm2}
U(n+1) = F(u(n))
\end{equation}
where $U \in R^d$ for $d > 1$. Hypotheses \ref{symmetryhyp}(ii), \ref{initialhyp}, and \ref{intersectionhyp} Let $Q(n)$ be the primary pulse solution to \eqref{diffeqm2}. Choose $m > 1$ and positive integers $N_1, \dots, N_{m-1}$, and define $N_i^\pm$ as in \eqref{Nipm}. Then, for sufficiently large $N$, where $N = \min\{ N_i^\pm : i = 1, \dots, m-1 \}$, there exists a unique $m-$pulse solution $Q_m$ to \eqref{diffeqm2} of the form
\begin{align}
U_i^-(n) &= T(\theta_i) Q(n) + W_i^-(n) && n \in [-N_{i-1}^-, 0] \\
U_i^+(n) &= T(\theta_i) Q(n) + W_i^+(n) && n \in [0, N_i^+]
\end{align}
if the following $m$ jump conditions are satisfied
\begin{equation}\label{jumpZ}
\begin{aligned}
\xi_1 &= \langle T(\theta_1) Z_1(N_1^+), T(\theta_2) Q(-N_1^-) \rangle + R_1  \\
\xi_i &= \langle T(\theta_i) Z_1(N_i^+), T(\theta_{i+1}) Q(-N_i^-) \rangle
- \langle T(\theta_i) Z_1(-N_{i-1}^-), T(\theta_{i-1}) Q(N_{i-1}^+) + R_i &&
i = 2, \dots, m-1 \\
\xi_m &= -\langle T(\theta_m) Z_1(-N_{m-1}^-), T(\theta_{m-1}) Q(N_{m-1}^+) + R_m
\end{aligned}
\end{equation}
where $\theta_i \in \R$ and
\[
|R_i| \leq C r^{-3N}
\]
We have the same estimates on $W_i^\pm$ as in Theorem \ref{transversemulti}.
\end{theorem}

\subsection{Stability Problem}
Assume Hypothesis \ref{initialhyp} and Hypothesis \ref{intersectionhyp}(ii). Let $Q_m(n)$ be an $m-$pulse solution to \eqref{diffeq}, constructed as in the previous section. Write $Q_m(n)$ piecewise as
\begin{align}\label{qmpiecewise}
Q_m(n) =
\begin{cases}
T(\theta_i) Q(n) + \tilde{Q}_i^-(n) & n = [-N_{i-1}^-, 0] \\
T(\theta_i) Q(n) + \tilde{Q}_i^+(n) & n = [0, N_i^+] \\
\end{cases}
\end{align}
where the interval endpoints $N_i^\pm$ are defined in \eqref{Nipm}, with $N_0^- = -\infty$ and $N_m^+ = \infty$. From $Q_m(n)$ we can obtain an equilibriunm solution $q_m(n) \in \R^s$ to \eqref{latticePDE}.

To obtain the eigenvalue problem associated with \eqref{latticePDE}, we substitute the standard linearization ansatz $u(n) = q_m(n) + \epsilon v(n) e^{\lambda t}$ and keep terms up to order $\epsilon$, which gives us the eigenvalue problem
\begin{equation}\label{EVP}
Df(q_m(n))v(n) = \lambda v(n)
\end{equation}
As we did above with the stationary equation \eqref{stationaryeq}, we write this as a system of first-order difference equations. This gives us the eigenvalue problem in $\R^d$
\begin{align}\label{latticeEVP}
V(n+1) = A(Q_m(n)) V(n) + \lambda B V(n)
\end{align}
where $A(Q_m) = DF(q_m)$ and $B$ is a bounded, $d\times d$ constant-coefficient matrix. The dimension $d$ is the same as in the existence problem. We make the following hypothesis regarding $B$ and $T(\theta)$, which holds in most applications.

\begin{hypothesis}\label{BTcommutehyp}
For all $\theta \in G$, $T(\theta) B = B T(\theta)$.
\end{hypothesis}

By the construction of $Q_m(n)$ in the previous section, $A(Q_m(n))$ is exponentially asymptotic to a constant coefficient matrix $A_\infty = DF(0)$, which is hyperbolic by Hypothesis \ref{initialhyp}. 

We will also assume one of the following Melnikov conditions.
\begin{hypothesis}\label{melnikovhyp}
One of the following holds Melnikov conditions holds.
\begin{enumerate}[(i)]
\item 
\[
M_1 = \sum_{n=-\infty}^\infty \langle Z_1(n+1), B S_1(n) \rangle \neq 0
\]
\item There exists a bounded function $T_1(n)$ such that 
\[
T_1(n+1) = A(Q(n)) T_1(n) + B S_1(n)
\]
and
\[
M_2 = \sum_{n=-\infty}^\infty \langle Z_1(n+1), B T_1(n) \rangle \neq 0 
\]
It follows from the Fredholm alternative that
\[
M_1 = \sum_{n=-\infty}^\infty \langle Z_1(n+1), B S_1(n) \rangle = 0 \\
\]
\end{enumerate}
\end{hypothesis}

We can now state the stability theorem, in which we locate the eigenvalues of \eqref{latticePDE}.

\begin{theorem}\label{stabilitytheorem}
Let $Q_m(n)$ be discrete $m-$pulse equilibrium solution to \eqref{latticePDE}, constructed according to Theorem \ref{transversemulti} or Theorem \ref{ntmulti}. Then there exists a bounded, nonzero solution $V(n)$ of the eigenvalue problem \eqref{latticeEVP} for $|\lambda| < \delta$ if and only if $\det E(\lambda) = 0$, where
\begin{equation}\label{Elambda}
E(\lambda) = \begin{cases}
\det(A - M_1 \lambda \: \text{diag}(T(\theta_1), \dots, T(\theta_m)) + R(\lambda)
& \text{Hypothesis \ref{melnikovhyp}(i) holds} \\
\det(A - M_2 \lambda^2 \text{diag}(T(\theta_1), \dots, T(\theta_m)) + R(\lambda) 
& \text{Hypothesis \ref{melnikovhyp}(ii) holds}
\end{cases}
\end{equation}
$M_1$ and $M_2$ are defined in Hypothesis \eqref{melnikovhyp}. The matrix $A$ is tridiagonal and is given by
\begin{align}\label{matrixA}
A &= \begin{pmatrix}
-a_1 & a_1 & & &  \\
-\tilde{a}_1 & \tilde{a}_1 - a_2 & a_2 \\
& -\tilde{a}_2 & \tilde{a}_2 - a_3 & a_3 \\
& \ddots & & \ddots \\
& & & -\tilde{a}_{m-1} & \tilde{a}_{m-1}  \\
\end{pmatrix}
\end{align}
where
\begin{align*}
a_i &= \langle T(\theta_i) Z_1(N_i^+), T(\theta_{i+1}) S_1(-N_i^-) \rangle \\
\tilde{a}_i &= \langle T(\theta_i) Z_1(-N_{i-1}^-), T(\theta_{i-1}) S_1(N_{i-1}^+) \rangle
\end{align*}

The remainder term has bound
\begin{align}\label{Rbound2}
|R(\lambda)| \leq C\left( (r^N + |\lambda|)^3 \right)
\end{align}
\end{theorem}

\section{Discrete NLS}

The normalized form of dNLS is
\begin{equation}\label{dNLS}
i\dot{u}_n + \epsilon(u_{n+1} - 2 u_n + u_{n-1}) + |u_n|^2 u_n = 0
\end{equation}
The parameter $\epsilon$ represents the coupling between nodes, and the anti-continuum limit is given by $\epsilon = 0$. Equation \eqref{dNLS} is Hamiltonian (Kev09x)
\[
H = -\sum_{n=-\infty}^\infty\left( \epsilon|u_n - u_{n-1}|^2
-\frac{1}{2}|u_n|^4 \right)
\]
We are interested in stationary solutions in a ``co-rotating'' frame, i.e. solutions of the form $\phi_n e^{i \omega t}$. To that end, we take $u_n \mapsto u_n e^{i \omega t}$ in \eqref{dNLS} and simplify to get
\begin{equation}\label{dNLSrot}
i\dot{u}_n + \epsilon(u_{n+1} - 2 u_n + u_{n-1}) - \omega u_n + |u_n|^2 u_n = 0
\end{equation}

\subsection{Existence of multi-pulses}

Equilibrium solutions satisfy
\begin{equation}\label{dNLSequilib}
\epsilon(u_{n+1} - 2 u_n + u_{n-1}) - \omega u_n + |u_n|^2 u_n = 0
\end{equation}
In the anti-continuum limit, \eqref{dNLSequilib} is an algebraic equation. Any function $u(n)$ with $u(n) \in \{ 0, \pm \sqrt{\omega}\}$ is a solution. We are interested in what happens as we move away from this limit.

First, we will look for real-valued solutions to \eqref{dNLSequilib}. In this case, we have a symmetry group $G = (\{1, -1\}, \cdot)$, and the group action on $l^\infty(\Z)$ is $g(\theta) = \theta$, i.e. multiplication by $\theta$. Let $\tilde{u}_n = u_{n-1}$. Thus Hypothesis \eqref{symmetryhyp}(i) is satisfied.

Real-valued, stationary solutions to \eqref{dNLSreal} satisfy the difference equation in $\R^2$
\begin{equation}\label{dnlsdiffR2}
\begin{pmatrix}
u \\ \tilde{u}
\end{pmatrix}_{n+1} =
\begin{pmatrix}
\frac{\omega}{\epsilon} + 2 & -1 \\
1 & 0
\end{pmatrix}
\begin{pmatrix}
u \\ \tilde{u}
\end{pmatrix}_n
- \frac{1}{\epsilon} 
\begin{pmatrix}
u_n^3 \\ 0
\end{pmatrix}
\end{equation}
where we have separated the RHS into linear and nonlinear parts. This is of the form of equation \eqref{diffeq}, where $U = (u, \tilde{u})$ and $F(U) = A_\infty U + G(U)$, where $F$ is smooth. The symmetry group $G$ acts on $\R^2$ by $T(\theta) = \theta I = \{ I, -I \}$.

The rest state $U(n) = 0$ is an equilibrium solution, and $DF(0) = A_\infty$ has eigenvalues
\begin{align*}
\nu_{1,2} &= 1 + \frac{\omega}{2 \epsilon} \left( 1 \pm \sqrt{1 + \frac{4 \epsilon}{\omega}} \right)
\end{align*}
For $\epsilon, \omega > 0$, these eigenvalues are real, and it straightforward to check that $\nu_1 \nu_2 = 1$. Thus 0 is a hyperbolic equilibrium point with 1-dimensional stable and unstable manifolds. Let
\begin{align*}
r &= r(\epsilon, \omega) = 1 + \frac{\omega}{2 \epsilon} \left( 1 + \sqrt{1 + \frac{4 \epsilon}{\omega}} \right) > 1
\end{align*}
It is not hard to show that $r(\epsilon, \omega) \rightarrow 1$ as $\epsilon \rightarrow 0$ or $\omega \rightarrow 0$ (with the other variable held constant). Thus as long as $\epsilon, \omega > 0$, Hypothesis \ref{initialhyp}(i) and (ii) are satisfied. For Hypothesis \ref{initialhyp}(iii), the existence of a single-pulse homoclinic orbit $Q(n)$connecting the rest state a 0 to itself (bright soliton) has been shown to exist [CITATION NEEDED]. We should also know that the stable and unstable manifolds intersect transversely, or (equivalently) that the variational equation does not have a bounded solution, thus Hypothesis \ref{intersectionhyp} is satisied. 

Using Theorem \ref{transversemulti}, for sufficiently large $N$ (which depends on $r$, thus $\omega$ and $\epsilon$) and any choice of $\theta_i = \pm 1$, there exist $m-$pulse solutions which can be written piecewise in the form  
\begin{align*}
Q_i^-(n) &= \theta_i Q(n) + W_i^-(n) && n \in [-N_{i-1}^-, 0] \\
Q_i^+(n) &= \theta_i Q(n) + W_i^+(n) && n \in [0, N_i^+]
\end{align*}

We will now show that (up to rotation), there are no other multi-pulse soliton solutions to dNLS. This time, we will allow \eqref{dNLSequilib} to have complex-valued solutions. In this case, we have a symmetry group $G = (\R, +)$, and the group action on $l^\infty(\Z)$ is $g(\theta)u(n) = e^{i \theta}u(n)$, i.e. rotation by $\theta$. Thus Hypothesis \ref{symmetryhyp}(ii) is satisfied.

Next, write dNLS as a two-dimensional system in real variables. Substituting $u = v + i w$ and equating real and imaginary parts, we obtain the equivalent system 
\begin{equation}\label{dNLSreal}
\begin{aligned}
\dot{v}_n  &+ \epsilon (w_{n+1} - 2 w_n + w_{n-1}) - \omega w_n + v_n^2 w_n + w_n^3 = 0 \\
-\dot{w}_n &+ \epsilon (v_{n+1} - 2 v_n + v_{n-1}) - \omega v_n + v_n w_n^2 + v_n^3 = 0
\end{aligned}
\end{equation}
Let $\tilde{v}_n = v_{n-1}$ and $\tilde{w}_n = w_{n-1}$. Then stationary solutions to \eqref{dNLSreal} satisfy the difference equation in $\R^4$
\begin{equation}\label{dnlsdiffR4}
\begin{pmatrix}
v \\ \tilde{v} \\ w \\ \tilde{w}
\end{pmatrix}_{n+1} =
\begin{pmatrix}
\frac{\omega}{\epsilon} + 2 & -1 & 0 & 0 \\
1 & 0 & 0 & 0 \\
0 & 0 & \frac{\omega}{\epsilon} + 2 & -1 \\
0 & 0 & 1 & 0
\end{pmatrix}
\begin{pmatrix}
v \\ \tilde{v} \\ w \\ \tilde{w}
\end{pmatrix}_n
- \frac{1}{\epsilon} 
\begin{pmatrix}
v_n w_n^2 + v_n^3 \\ 0 \\ v_n^2 w_n + w_n^3 \\ 0
\end{pmatrix}
\end{equation}
This is of the form of equation \eqref{diffeq}, where $U = (v, \tilde{v}, w, \tilde{w})$ and $F(U) = A_\infty U + G(U)$, where $F$ is smooth. The rest state $U(n) = 0$ is an equilibrium solution. The symmetry group $G$ acts on $\R^4$ via $T(\theta)$, where
\begin{align}\label{TdNLS}
T(\theta) =
\begin{pmatrix}
\cos\theta & 0 &-\sin\theta & 0 \\
0 & \cos\theta & 0 &-\sin\theta \\
\sin\theta & 0 & \cos\theta & 0 \\
0 & \sin\theta & 0 & \cos\theta 
\end{pmatrix}
\end{align}
$T(\theta)$ has infinitesimal generator $S$, which is given by
\begin{align}\label{dnlsSgen}
S =
\begin{pmatrix}
0 & 0 &-1 & 0 \\
0 & 0 & 0 &-1 \\
1 & 0 & 0 & 0 \\
0 & 1 & 0 & 0 
\end{pmatrix}
\end{align}

$DF(0) = A_\infty$ has eigenvalues $\nu_{1,2}$, which are the same as in the real-valued case. Each eigenvalue has multiplicity 2, since $A_\infty$ is block diagonal with two identical blocks. Choosing $r(\epsilon, \omega)$ as above, 0 is a hyperbolic equilibrium point with 2-dimensional stable and unstable manifolds. Let $Q(n)$ be the real-valued primary pulse solution to dNLS, which is of the form $Q(n) = (q(n), \tilde{q}(n), 0, 0)$ be this solution, where $q(n)$ is a real-valued solution to \eqref{dNLSequilib}. Thus Hypothesis \ref{initialhyp} is satisfied.

The variational and adjoint variational equations associated with \eqref{dnlsdiffR4} and $Q$ are 
\begin{align}
V(n+1) &= A(Q(n)) V(n) \label{vareqdNLS} \\
Z(n+1) &= [A(Q(n)^*]^{-1} Z(n) \label{adjvareqdNLS}
\end{align}
where
\begin{equation}\label{AQdNLS}
A(Q(n)) = 
\begin{pmatrix}
\frac{\omega}{\epsilon} + 2 - \frac{3 q(n)^2}{\epsilon} & -1 & 0 & 0 \\
1 & 0 & 0 & 0 \\
0 & 0 & \frac{\omega}{\epsilon} + 2 - \frac{q(n)^2}{\epsilon} & -1 \\
0 & 0 & 1 & 0
\end{pmatrix}
\end{equation}
and $A(Q_n)$ is exponentially asymptotic to $A_\infty$. Let $S_1(n) = S Q(n) = (0, 0, q(n), \tilde{q}(n))$. We can verify that (as expected), $S_1(n)$ is a solution to the variational equation \eqref{vareqdNLS}. Either we assume (OR KNOW FROM SOMEWHERE that there are no other bounded solutions to \eqref{vareqdNLS}, thus Hypothesis \ref{intersectionhyp} is satisfied. It follows the the adjoint variational equation \eqref{adjvareqdNLS} has a unique bounded solution $Z_1(n)$, and we can verify directly that $Z_1(n) = (0, 0, -\tilde{q}(n), q(n))$.

Using Theorem \ref{ntmulti}, for sufficiently large $N$ (which depends on $r$, thus $\omega$ and $\epsilon$) and any choice of $\theta_i = \R$, there exist $m-$pulse solutions which can be written piecewise in the form  
\begin{align*}
Q_i^-(n) &= \theta_i Q(n) + W_i^-(n) && n \in [-N_{i-1}^-, 0] \\
Q_i^+(n) &= \theta_i Q(n) + W_i^+(n) && n \in [0, N_i^+]
\end{align*}
if any only if the jump conditions
\begin{align*}
\xi_i = \langle T(\theta_i) Z_1(N_i^+), T(\theta_{i+1}) Q(-N_i^-) \rangle
- \langle T(\theta_i) Z_1(-N_{i-1}^-), T(\theta_{i-1}) Q(N_{i-1}^+) \rangle + R_i
\end{align*}
are satisfied, where
\[
|R_i| \leq C r^{-3N}
\]
and we note that the first inner product term is zero for $i = m$ and the second inner product term is 0 for $i = 1$.

Using \eqref{TdNLS} and the fact that$T(\theta)$ is unitary, we can write the $m$ jump conditions as
\begin{align*}
\xi_i = \langle T(\theta_i - \theta_{i+1}) Z_1(N_i^+), Q(-N_i^-) \rangle
- \langle T(\theta_i - \theta_{i-1}) Z_1(-N_{i-1}^-), Q(N_{i-1}^+) \rangle + R_i
\end{align*}
For $i = 1, \dots, m-1$, let $\Delta \theta_i = \theta_{i+1} - \theta_i$. Then the jump conditions can be written
\begin{align}\label{jumpDNLS}
\xi_i = \langle T(-\Delta \theta_i) Z_1(N_i^+), Q(-N_i^-) \rangle
- \langle T(\Delta \theta_{i-1}) Z_1(-N_{i-1}^-), Q(N_{i-1}^+) \rangle + R_i
\end{align}
For the first inner product term, we have
\begin{align*}
\langle T(&-\Delta\theta_i) Z_1(N_i^+), Q(-N_i^-) \rangle \\
&= \langle 
\begin{pmatrix}
\tilde{q}(N_i^+)\sin(-\Delta\theta_i) \\
-q(N_i^+)\sin(-\Delta\theta_i) \\ 
-\tilde{q}(N_i^+)\cos(-\Delta\theta_i) \\
q(N_i^+)\cos(-\Delta\theta_i)
\end{pmatrix},
\begin{pmatrix}
q(-N_i^-) \\ \tilde{q}(-N_i^-) \\ 0 \\ 0 
\end{pmatrix}
\rangle \\
&= (\tilde{q}(N_i^+)q(-N_i^-) - q(N_i^+)\tilde{q}(-N_i^-))\sin(-\Delta\theta_i) \\
&= -(q(N_i^+ - 1)q(-N_i^-) - q(N_i^+)q(-N_i^- - 1))\sin(\Delta\theta_i) \\
&= -(q(N_i^+ - 1)q(N_i^-) - q(N_i^+)q(N_i^- + 1))\sin(\Delta\theta_i) \\
&= -a_i \sin(\Delta\theta_i) 
\end{align*}
where $a_i = (q(N_i^+ - 1)q(N_i^-) - q(N_i^+)q(N_i^- + 1))$. For the second inner product term,
\begin{align*}
\langle T(&\Delta\theta_{i-1}) Z_1(-N_{i-1}^-), Q(N_{i-1}^+) \rangle \\
&= \langle 
\begin{pmatrix}
\tilde{q}(-N_{i-1}^-)\sin(\Delta\theta_{i-1}) \\
-q(-N_{i-1}^-)\sin(\Delta\theta_{i-1}) \\ 
-\tilde{q}(-N_{i-1}^-)\cos(\Delta\theta_{i-1}) \\
q(-N_{i-1}^-)\cos(\Delta\theta_{i-1})
\end{pmatrix},
\begin{pmatrix}
q(Q(N_{i-1}^+) \\ \tilde{q}(Q(N_{i-1}^+) \\ 0 \\ 0 
\end{pmatrix}
\rangle \\
&= (\tilde{q}(-N_{i-1}^-)q(N_{i-1}^+) - q(-N_{i-1}^-)\tilde{q}(N_{i-1}^+))\sin(\Delta\theta_{i-1}) \\
&= (q(-N_{i-1}^- - 1)q(N_{i-1}^+) - q(-N_{i-1}^-)q(N_{i-1}^+ - 1))\sin(\Delta\theta_{i-1}) \\
&= (q(N_{i-1}^- + 1)q(N_{i-1}^+) - q(N_{i-1}^-)q(N_{i-1}^+ - 1))\sin(\Delta\theta_{i-1}) \\
&= -a_{i-1} sin(\Delta\theta_{i-1})
\end{align*}
Substituting these into \eqref{jumpDNLS} and writing the first and last jump condition separately, we have
\begin{equation}\label{jumpDNLS2}
\begin{aligned}
\xi_1 &= -a_1 s_1 + R_1 \\
\xi_i &= a_{i-1} s_{i-1} - a_i s_i + R_i
&& i = 2, \dots, m-1 \\
\xi_m &= a_{m-1} s_{m-1} + R_m
\end{aligned}
\end{equation}
where $s_i = \sin{\theta_i}$. Since $a_i = \mathcal{O}(r^{-2N})$ and $R_i = \mathcal{O}(r^{-3N})$, the jump conditions can only be satisfied if $s_i = \mathcal{O}(r^{-N})$. Thus we only have to consider that case from now on. 


jump conditions in matrix form as
\[
A s + R = 0
\]
where $A$ is the $m \times (m-1)$ matrix
\[
A = \begin{pmatrix}
-a_1 \\
a_1 & -a_2 \\
& a_2 & -a_3 \\
&& \ddots & \ddots \\
&&& a_{m-2} & -a_{m-1} \\
&&&& a_{m-1}
\end{pmatrix}
\]
$s = (s_1, \dots, s_{m-1})$, and $R = \mathcal{O}(r^{-3N})$. Since we have a Hamiltonian system, if we solve $m-1$ jump conditions, then the final jump condition is automatically satisfied, thus we can eliminate one of the jump conditions. It does not matter which one we eliminate, but for simplicity we will eliminate the final one. Thus the matrix $A$ reduces to the $(m-1)\times(m-1)$ matrix
\[
A = \begin{pmatrix}
-a_1 \\
a_1 & -a_2 \\
& a_2 & -a_3 \\
&& \ddots & \ddots \\
&&& a_{m-2} & -a_{m-1} \\
\end{pmatrix}
\]




From the real-valued case, this has a solution if $\Delta \theta_i \in \{0, \pi\}$ for all $i$, i.e. adjacent peaks are either in phase or out of phase. We will show that no other values of $\Delta \theta_i$ are possible. 

First, we evaluate the coefficients $a_i$. For $N_i$ even, $N_i^- = N_i^+$ and for $N_i$ odd, $N_i^- = N_i^+ + 1$, thus we have
\begin{align*}
a_i &= \begin{cases}
q(N_i^+)( q(N_i^+ - 1) - q(N_i^- + 1) ) & N_i \text{ even} \\
(q(N_i^+ - 1)q(N_i^+ + 1) - q(N_i^+)q(N_i^+ + 2)) & N_i \text{ odd}
\end{cases}
\end{align*}



\subsection{Eigenvalue problem}

Let $Q_m(n) = (q_m(n), \tilde{q}_m(n), 0, 0)$ be a $m-$pulse solution to \eqref{dnlsdiffR4} with symmetry parameters $\theta_1, \dots, \theta_m$, with $\theta \{0, \pi\}$. Taking the standard linearization ansatz
\[
\begin{pmatrix}v \\ \tilde{v} \\ w \\ \tilde{w} \end{pmatrix}_n = 
\begin{pmatrix}q_m \\ \tilde{q}_m \\ 0 \\ 0 \end{pmatrix}_n + 
\eta \begin{pmatrix}a \\ \tilde{a} \\ b \\ \tilde{b} \end{pmatrix}_n e^{\lambda t}
\]
we obtain, upon substitution and simplification, the eigenvalue problem 
\begin{equation}\label{dNLSEVP}
V(n+1) = A(Q_m(n)) V_n + \lambda B V_n
\end{equation}
where $V = (a, \tilde{a}, b, \tilde{b})$, $A(Q_m(n)$ is defined in \eqref{AQdNLS}, and $B$ is the constant coefficient matrix 
\[
B = 
\begin{pmatrix}
0 & 0 & 1 & 0 \\
0 & 0 & 0 & 0 \\
-1 & 0 & 0 & 0 \\
0 & 0 & 0 & 0
\end{pmatrix}
\]

Let $S_1 = (0, 0, q, \tilde{q})^T$ and $T_1 = (-\partial_\omega q, -\partial_\omega \tilde{q}, 0, 0)^T$. Then it is straightforward to verify that
\begin{align*}
S_1(n+1) &= A(Q_1(n)) S_1(n) \\
T_1(n+1) &= A(Q_1(n)) T_1(n) + B S_1(n)
\end{align*}
and that $Z_1 = (0, 0, -\tilde{q}, q)$ is the unique bounded solution to the adjoint variational equation \eqref{adjvareqdNLS}. For the Melnikov sums, we have
\[
M_1 = \sum_{n=-\infty}^\infty \langle Z_1(n+1), B S_1(n) \rangle = 0
\]
and
\[
M_2 = \sum_{n=-\infty}^\infty \langle Z_1(n+1), B T_1(n) \rangle =
\sum_{n=-\infty}^\infty q(n) q_\omega(n)
\]
Either we assume $M_2$ is nonzero or we find a reference for that. Thus Hypothesis \ref{melnikovhyp}(ii) is satisfied. For $N$ sufficiently large, we can find the eigenvalues of \eqref{dNLSEVP} using Theorem \ref{stabilitytheorem}. 

First, we look at the terms of the matrix $A$. For $a_i$ and $\tilde{a}_i$, we have
\begin{align*}
a_i &= \langle Z_1(N_i^+), S_1(-N_i^-) \rangle \\
&= -\tilde{q}(N_i^+)q(-N_i^-) + q(N_i^+)\tilde{q}(-N_i^-) \\
&= -q(N_i^+ - 1)q(-N_i^-) + q(N_i^+)(-N_i^- - 1) \\
&= q(N_i^+)(N_i^- + 1) - q(N_i^+ - 1)q(N_i^-)\\
\end{align*}
and
\begin{align*}
\tilde{a}_i &= \langle Z(-N_i^-), S_1(N_i^+) \rangle \\
&= -\tilde{q}(-N_i^-)q(N_i^+) + q(-N_i^-)\tilde{q}(N_i^+) \\
&= -q(-N_i^- - 1)q(N_i^+) + q(-N_i^-)q(N_i^+ - 1) \\
&= -q(N_i^- + 1)q(N_i^+) + q(N_i^-)q(N_i^+ - 1) \\
&= -q(N_i^+)q(N_i^- + 1) + q(N_i^+ - 1)q(N_i^-) \\
&= -a_i
\end{align*}
For $N_i$ even, we have $N_i^+ = N_i^-$, so $a_i$ becomes
\[
a_i = q(N_i^+)( q(N_i^+ + 1) - q(N_i^+ - 1) )
\]
For $N_i$ odd, we have $N_i^- = N_i^+ + 1$, so $a_i$ becomes
\[
a_i = q(N_i^+)q(N_i^+ + 2) - q(N_i^+ - 1)q(N_i^+ + 1)
\]
Both of these are negative, since $q(n)$ is decreases as $n$ moves away from 0. (I THINK WE GET THIS FROM EXPONENTIAL DECAY AND THE FACT THAT THE EIGENVALUES OF REST STATE ARE REAL).

The matrix $A$ becomes
\begin{align}\label{AdNLS}
A &= \begin{pmatrix}
-c_2 a_1 & c_2 a_1 & & &  \\
c_1 a_1 & -c_1 a_1 - c_3 a_2 & c_3 a_2 \\
& c_2 a_2 & -c_2 a_2 - c_4 a_3 & c_4 a_3 \\
& \ddots & & \ddots \\
& & & c_{m-1} a_{m-1} & -c_{m-1} a_{m-1}  \\
\end{pmatrix}
\end{align}
By Theorem \ref{stabilitytheorem}, to leading order, we need to find $\mu$ such that equation 
\begin{align}\label{Aeq1}
\begin{pmatrix}
-c_2 a_1 - c_1 \mu & c_2 a_1 & & &  \\
c_1 a_1 & -c_1 a_1 - c_3 a_2 - c_2 \mu & c_3 a_2 \\
& c_2 a_2 & -c_2 a_2 - c_4 a_3 - c_3 \mu & c_4 a_3 \\
& \ddots & & \ddots \\
& & & c_{m-1} a_{m-1} & -c_{m-1} a_{m-1} - c_m \mu 
\end{pmatrix}
\begin{pmatrix}d_1 \\ d_2 \\ d_3 \\ \vdots \\ d_m \end{pmatrix} 
 = 0
\end{align}
We will take $\mu = M_2 \lambda^2$. For $\mu = 0$, $(1, 1, \dots, 1)^T$ is a nontrivial solution to \eqref{Aeq1}. The inner rows of \eqref{Aeq1} satisfy the difference equation
\[
c_{j-1} a_{j-1} d_{j-1} - (c_{j-1}a_{j-1} + c_{j+1}a_j + c_j \mu) d_j
+ c_{j+1} a_j d_{j+1} = 0
\]
Multiplying this by $c_j$ and noting that $c_j^2 = 1$, this becomes
\[
c_{j-1} c_j a_{j-1} d_{j-1} - c_{j-1} c_j a_{j-1} d_j - c_j c_{j+1} a_j d_j
+ c_j c_{j+1} a_j d_{j+1} = \mu d_j
\]
Let $p_j = c_j c_{j+1} a_j$. Then we can write this as
\[
p_{j-1} d_{j-1} - p_{j-1} d_j - p_j d_j
+ p_j d_{j+1} = \mu d_j
\]
which is of the form
\[
\nabla( p_j \Delta d_j ) = \mu d_j
\]
where $\Delta$ is the forward difference operator $\Delta f(j) = f(j+1) - f(j)$ and $\nabla$ is the backward difference operator $\nabla f(j) = f(j) - f(j-1)$. This is in the form of a Sturm-Liouville difference equation $\nabla( p_j \Delta d_j ) + q_j d_j = \mu w_j d_j$ with $q_j = 0$ and $w_j = 0$. The boundary conditions are obatained from the first and last lines of the matrix equation \eqref{Aeq1}. To do this, we extend the S-L difference equation to $j = 1$ and $j = m$, which will give us terms involving $d_0$ and $d_{m+1}$. We then impose conditions so that these are equivalent to the first and last lines. For the first line, we have
\[
-p_1 d_1 + p_1 d_2 = \mu d_1
\]
Since the difference equation evaluated at $j = 1$ is
\[
p_0 d_0 - p_0 d_1 - p_1 d_1 + p_1 d_2 = \mu d_1
\]
we take the Dirichlet boundary condition $d_0 = 0$. Similarly, we take $d_{m+1} = 0$. Thus we have the S-L difference equation
\begin{equation}\label{SLdiff}
\begin{aligned}
\nabla( p_j \Delta d_j ) &= \mu d_j && j = 1, \dots, m \\
d_0 &= 0 \\
d_{m+1} &= 0
\end{aligned}
\end{equation}
where $p_j = c_j c_{j+1} a_j$. It follows from [REFERENCE] that the eigenvalues of \eqref{SLdiff} are real and distinct. To find the eigenvalues of \eqref{SLdiff}, we will write \eqref{Adzero} in the form of an eigenvalue problem. WE MIGHT BE ABLE TO DO THIS INSTEAD WITH DISCRETE S-L THEORY. Multiply row $j$ of \eqref{Aeq1} by $c_j$ to get
\begin{align*}
\begin{pmatrix}
-c_1 c_2 a_1 - \mu & c_1 c_2 a_1 & & &  \\
c_1 c_2 a_1 & -c_1 c_2 a_1 - c_2 c_3 a_2 - \mu & c_2 c_3 a_2 \\
& c_2 c_3 a_2 & -c_2 c_3 a_2 - c_3 c_4 a_3 - \mu & c_3 c_4 a_3 \\
& \ddots & & \ddots \\
& & & c_{m-1} c_m a_{m-1} & -c_{m-1} c_m a_{m-1} - \mu \\
\end{pmatrix}
\begin{pmatrix}d_1 \\ d_2 \\ d_3 \\ \vdots \\ d_m \end{pmatrix} 
 = 0
\end{align*}
Let $p_j = c_j c_{j+1} a_j$. Then \eqref{Aeq1} is equivalent to $(A_0 - \mu I)d = 0$, where
\begin{align*}
A_0 = 
\begin{pmatrix}
-p_1 & p_1 & & &  \\
p_1 & -p_1 - p_2 & p_2 \\
& p_2 & -p_2 - p_3  & p_3 \\
& \ddots & & \ddots \\
& & & p_{m-1} & -p_{m-1} \\
\end{pmatrix}
\end{align*}
This matrix is symmetric and is the same matrix as we find in San98. $A_0$ has an eigenvalue at 0 with corresponding eigenvector $(1, 1, \dots, 1)^T$. For the remaining $m-1$ eigenvalues, we use Lemma 5.4 in San98 (noting that the matrix $A_0$ there is $-A_0$ here) to conclude that  
\begin{enumerate}
	\item $A_0$ has $k_+$ negative real eigenvalues (counting multiplicity), where $k_+$ is the number of positive $p_j$.
	\item $A_0$ has $k_-$ positive real eigenvalues (counting multiplicity), where $k_-$ is the number of negative $p_j$.
\end{enumerate}
We showed above that the $a_j$ are all negative. Thus we conclude that
\begin{align*}
p_j > 0 &\text{ if }c_j c_{j+1} = -1 \\
p_j < 0 &\text{ if }c_j c_{j+1} = 1
\end{align*}
which implies that $c_j c_{j+1} = -1$ corresponds to a negative real eigenvalue of $A_0$ and $c_j c_{j+1} = 1$ corresponds to a positive real eigenvalue. We conclude that $A_0$ has one eigenvalue at 0; $k$ negative eigenvalues, where $k$ is the number of sign changes in the $m-$component vector $(c_1, c_2, \dots, c_m)$; and $m - k - 1$ negative eigenvalues. Let $0, \mu_1, \dots, \mu_{m-1}$ be the eigenvalues of $A_0$. Since these are the same eigenvalues as those of the discrete S-L problem \eqref{SLdiff}, these are distinct. 

Recall that we are taking $\mu = M_2 \lambda^2$. Thus $(A_0 - \lambda^2 M_2 I) = 0$ if and only if $\lambda = 0$ or $\lambda = \lambda_j^\pm$, where 
\begin{align*}
\lambda_j^\pm &= \pm \sqrt{\mu_j / M_2} && j = 1, \dots, m-1
\end{align*}
These are pairs of real or purely imaginary eigenvalues, depending on the sign on $\mu_j$.

So far, we have found the eigenvalues $\lambda$ to leading order. Finally, we show that, when we include the remainder term, the nonzero eigenvalues are close to $\lambda_j^\pm$. We will always have an eigenvalue at 0 with algebraic multiplicity 2 and geometric multiplicity 1. The eigenfunction is $S_m(n)$ and the generalized eigenfunction is $T_m(n)$. For the nonzero eigenvalues, recall that by Theorem \ref{stabilitytheorem}, $\lambda$ is an eigenvalue if and only if 
\[
(A - M_2 \lambda^2 \text{diag}(c_1, \dots, c_m) + R(\lambda) )d = 0
\]
has a nontrivial solution, where the remainder term has bound
\begin{align*}
|R(\lambda)(d)| \leq C\left( (r^N + |\lambda|)^3 \right)
\end{align*}
Multiply row $j$ by $c_j$ to get the equivalent problem
\[
(A_0 - M_2 \lambda^2 I + R_0(\lambda))d = 0
\]
where $R_0(\lambda)$ has the same order as $R(\lambda)$. Thus it suffices to show $E_0(\lambda) = 0$, where
\[
E_0(\lambda) = \det (A_0 - M_2 \lambda^2 I + R_0(\lambda))
\]

We showed above that $\det (A_0 - M_2 \lambda^2 I) = 0$ for $\lambda = \lambda_j^\pm$ and $\lambda = 0$. Next, we rescale our equation. Let
\begin{align*}
A_0 = r^{2N} \tilde{A} \\
\lambda = r^{N} \tilde{\lambda} \\
R_0(\lambda) = r^{3N} \tilde{R}(\lambda)
\end{align*}
Dividing by $r^{-2N}$, we get the equivalent problem
\[
(\tilde{A} - M_2 \tilde{\lambda}^2 I + r^{N} \tilde{R}(\lambda))d = 0
\]

Let $\epsilon = r^{N}$. Then it suffices to show $\tilde{E}(\lambda; \epsilon) = 0$, where
\[
\tilde{E}(\lambda; \epsilon) = \det(\tilde{A} - M_2 \tilde{\lambda}^2 I + r^{-N} \tilde{R}(\lambda))
\]
For $j = 1, \dots, m-1$, let 
\[
\lambda_j^\pm = r^{-N} \tilde{\lambda}_j^\pm
\]
Then $\tilde{E}(\tilde{\lambda}_j^\pm; 0) = 0$. Since the eigenvalues of $A_0$ (hence $\tilde{A}$) are distinct,
\[
\frac{\partial}{\partial \tilde{\lambda}} \tilde{E}(\tilde{\lambda}_j^\pm; 0) \neq 0
\]
Thus, by the IFT, we can solve for $\tilde{\lambda}$ as a function of $\epsilon$ near $(\tilde{\lambda}, \epsilon) = (\tilde{\lambda}_j^\pm; 0)$. Expanding $\tilde{\lambda}(\epsilon)$ in a Taylor series about $\epsilon = 0$ and taking $\epsilon = r^{-N}$ gives us
\begin{equation*}
\tilde{\lambda}_j^\pm(N) = \tilde{\lambda}_j^\pm + \mathcal{O}(r^{-N})
\end{equation*}
Undoing the scaling, we have
\begin{align*}
\lambda^\pm_j(N) &= \pm \sqrt{\frac{\mu_j}{M_2}} + \mathcal{O}(r^{-3N}) 
\end{align*}

By Hamiltonian symmetry, eigenvalues must come in quartets $\pm a \pm i b$. Since the $\mu_j$ are distinct and these only come in pairs, the eigenvalues $\lambda^\pm(N)$ must be either real or purely imaginary. We conclude that there is an eigenvalue at $\lambda = 0$ with algebraic multiplicity 2 (with the eigenfunctions given above) and $2m - 2$ eigenvalues at $\lambda = \pm \lambda_j(N)$, where $j = 1, \dots, m-1$ and 
\begin{align*}
\lambda_j(N) &= \sqrt{\frac{\mu_j}{M_2}} + \mathcal{O}(r^{-3N}) 
\end{align*}
where $\mu_j$ is an eigenvalue of $A_0$. These are either real or purely imaginary, and remainder term cannot move these off of the real or imaginary axis. Let $k$ be the number of sign changes in the $m-$component vector $(c_1, c_2, \dots, c_m)$, with $c_j = \pm 1$. Then $k$ of these pairs are purely imaginary, and $m - k - 1$ of these pairs are real.

\section{Proof of Existence Theorems}

\subsection{Discrete Exponential Dichotomy}

In this section, we state results about exponential dichotomies in the discrete setting. First, we define the discrete evolution operator for linear difference equations.

% lemma : discrete evolution operator
\begin{lemma}[Discrete Evolution Operator]\label{evolop}
Consider the difference equation together with its adjoint
\begin{align}
V(n+1) &= A(n) V(n) \label{diffeqevol} \\
Z(n+1) &= [A(n)^{-1}]^* Z(n) \label{adjeqevol}
\end{align}
where $n \in \Z$, $V(n) \in R^d$, and the $d \times d$ matrix $A(n)$ is invertible for all $n$. Define the discrete evolution operator by
\begin{equation}\label{evol}
\Phi(m, n) = 
\begin{cases}
I & m = n \\
A(m-1) \dots A(n+1) A(n) & m > n \\
A^{-1}(m) \dots A^{-1}(n-2) A^{-1}(n-1) & m < n
\end{cases}
\end{equation}
The evolution operators of \eqref{diffeqevol} and \eqref{adjeqevol} are related by
\begin{equation}\label{adjevol}
\Psi(m, n) = \Phi(n, m)^*
\end{equation}
Finally, if $V(n)$ is a solution to \eqref{diffeqevol} and $Z(n)$ is a solution to \eqref{adjeqevol}, then the inner product $\langle V(n), Z(n) \rangle$ is constant in $n$.

\begin{proof}
Since to evolve a difference equation we iterate a map (or its inverse) a finite number of times, the definition makes sense. We can easily check that this defines a flow. Let $\Psi(m, n)$ be the evolution operator for the adjoint equation \eqref{adjeqevol}. Then since $Z(n) = A(n)^* Z(n+1)$, for $m < n$ we have
\begin{align*}
\Psi(m, n) &= A(m)^* \dots A(n-2)^* A(n-1)^* \\
&= [A(n-1) A(n-2) \dots A(m)]^* \\
&= \Phi(n, m)^*
\end{align*}
We can similarly show that this holds for $m > n$, and it holds trivially for $m = n$.

For a solution $V(n)$ to \eqref{diffeqevol} and a solution $Z(n)$ to \eqref{adjeqevol},
\begin{align*}
\langle V(n+1), Z(n+1) \rangle &= 
\langle A(n) V(n), [A(n)^{-1}]^* Z(n) \rangle \\
&= \langle A(n)^{-1} A(n) V(n), Z(n) \rangle \\
&= \langle V(n), Z(n) \rangle
\end{align*}
Similarly, we can show that $\langle V(n+1), Z(n+1) \rangle = \langle V(n), Z(n) \rangle$.
\end{proof}
\end{lemma}

In the next lemma, we give a criterion for an exponential dichotomy in the discrete case.

% lemma : exp dichotomy in discrete case
\begin{lemma}[Exponential Dichotomy]\label{dichotomy}
Consider the difference equation
\begin{equation}\label{diffeqdichot}
V(n+1) = A(n) V(n)
\end{equation}
with evolution operator $\Phi(m, n)$ as defined in Lemma \ref{evolop}. Suppose that $A(n) \rightarrow A^\pm$ exponentially as $n \rightarrow \pm \infty$, i.e. there exist constants $r_\pm < 1$ such that
\begin{align*}
|A(n) - A^\pm| \leq C r_\pm^{|n|}
\end{align*}
where $A^\pm$ are constant coefficient matrices. If $A^\pm$ are hyperbolic, then \eqref{diffeqdichot} has exponential dichotomies on $Z^\pm$. In other words, there exist projections $P_\pm^s$ and $P_\pm^u$ defined on $\Z^\pm$ such that
\begin{equation}\label{projcommute}
P_\pm^{s/u}(m) \Phi(m, n) =  \Phi(m, n) P_\pm^{s/u}(n)
\end{equation}
Letting $\Phi_\pm^{s/u}(m, n) = \Phi(m, n) P_\pm^{s/u}(n)$ for $m, n \geq 0$ and $m, n \leq 0$ (respectively), we have the estimates
\begin{align*}
|\Phi_+^s(m, n)| \leq C (r_+^s)^{m - n} && 0 \leq n \leq m \\
|\Phi_+^u(m, n)| \leq C \left( \dfrac{1}{r_+^u} \right)^{n-m} && 0 \leq m \leq n \\
|\Phi_-^s(m, n)| \leq C (r_-^s)^{m - n} && n \leq m \leq 0 \\
|\Phi_-^u(m, n)| \leq C \left( \dfrac{1}{r_-^u} \right)^{n-m} && m \leq n \leq 0\\
\end{align*}
for constants $r_\pm^s < 1$, $r_\pm^u > 1$ which are chosen so that $|\lambda_\pm| \leq r_\pm^s < 1$ or $|\lambda_\pm| \geq r_\pm^u > 1$ for all eigenvalues $\lambda_\pm$ of $A^\pm$. 
 
Finally, let $E_\pm^{s/u}$ be the stable and unstable eigenspaces of $A^\pm$ and $Q_\pm^{s/u}$ the corresponding eigenprojections. Then we have
\begin{align*}
\dim \text{range }P_\pm^s(n) &= \dim E_\pm^s \\
\dim \text{range }P_\pm^u(n) &= \dim E_\pm^u
\end{align*}
and the decay rate
\begin{align}\label{projbound}
| P_\pm^{s/u}(n) - Q_\pm^{s/u} | \leq C r_\pm^{|n|}
\end{align}

\begin{proof}
We will consider the problem on $\Z^+$. The problem on $\Z^-$ is similar. Since $A^+$ is constant coefficient and hyperbolic, the difference equation $W(n+1) = A^+ W(n)$ has an exponential dichotomy on $\R^+$. Specifically, since $A^+$ is hyperbolic, we can find radii $r_+^s < 1$ and $r_+^u > 1$ such that for all eigenvalues $\lambda$ of $A^+$ either $|\lambda| \leq r_+^s$ or $|\lambda| \geq r_+^u$. Let $E_+^{s/u}$ be the stable and unstable eigenspaces of $A^+$, and let $P_+^{s/u}$ be the corresponding eigenprojections, which commute with $A^+$. Let $\Phi_+(m, n)$ be the evolution operator for $W(n+1) = A^+ W(n)$. Then we have bounds for the evolution operator
\begin{align*}
\Phi_+(m, n) P_+^s W &= (A^+)^{m-n} P_+^s W \leq C (r_+^s)^{m-n} W  && m > n \\
\Phi_+(m, n) P_+^u W &= [(A^+)^{-1}]^{n-m} P_+^u W \leq C \left( \dfrac{1}{r_+^s} \right)^{n-m} W && m < n
\end{align*}
where we used the fact that the stable eigenspace of $(A^+)^{-1}$ is equal to the unstable eigenspace of $A^-$ (and vice versa), and the eigenvalues of $(A^+)^{-1}$ and $A^-$ are reciprocals of each other.

Since $A(n) \rightarrow A^+$ as $n \rightarrow \infty$, by Proposition 2.5 in Beyn97, \eqref{diffeqdichot} has an exponential dichotomy on $\Z^+$, as defined in Beyn97, with the same constants $r_+^s$, $r_+^s$, and $C$. Furthermore, the range of $P_+^s(n)$ has the same dimension as $E_+^s$, and the range of $P_+^u(n)$ has the same dimension as $E_+^u$.

The estimate \eqref{projbound} should hold since we $A(n)$ is exponentially asymptotic to $A^+$. We are essentially replacing the ordinary decay in (17) and (18) of Beyn97 with exponential decay. There is probably another reference we can cite for this. By the same argument we have a corresponding exponential dichotomy on $\Z^-$.
\end{proof}
\end{lemma}

The last thing we will need is a version of the variation of constants formula for the discrete setting.

\begin{lemma}[Variation of Constants Formula]\label{VOC}
Consider the initial value for the difference equation
\begin{align*}
V(n+1) &= A(n) V(n) + G(V(n), n) \\
V(n_0) &= V_{n_0}
\end{align*}
The solution is given by
\begin{equation}\label{VOCformula}
V(n) = 
\begin{cases}
V_{n_0} & n = n_0 \\
\Phi(n, n_0) V_{n_0} + \sum_{j = n_0}^{n-1} \Phi(n, j+1) G(V(j), j)) & n > n_0 \\
\Phi(n, n_0) V_{n_0} - \sum_{j = n}^{n_0-1} \Phi(n, j+1) G(V(j), j)) & n < n_0 
\end{cases}
\end{equation}
\begin{proof}
For $n = n_0 + 1$,
\[
V(n_0 + 1) = A(n_0) V(n_0) + G(V(v_0), n_0) = \Phi(n_0+1, n_0) V_{n_0} + \Phi(n_0, n_0) G(V(v_0), n_0)
\]
Iterate this to get the result for $n > n_0$. The case for $n < n_0$ is similar.
\end{proof}
\end{lemma}

\subsection{Fixed Point Formulation}

First, we derive a formula for the family of evolution operators we will need in terms of the evolution operator for the variational equation.

\begin{lemma}\label{thetaevol}
Let $\Phi(m, n; \theta)$ be the evolution operator for 
\begin{equation}\label{Vtheta}
V(n+1; \theta) = DF(T(\theta)Q(n)) V(n; \theta) 
\end{equation}
Then $\Phi(m, n; \theta) = T(\theta)\Phi(m, n; 0)T(\theta)^{-1}$.
\begin{proof}
Differentiating the symmetry $F(T(\theta)U) = T(\theta)F(U)$ with respect to $U$, we have 
\[
DF(T(\theta)U)T(\theta) = T(\theta)DF(U)
\]
from which it follows that
\begin{equation}\label{DFtheta}
DF(T(\theta)Q(n)) = T(\theta)DF(Q(n))T(\theta)^{-1}
\end{equation}
From the definition of the evolution operator from Lemma \ref{evolop},
\[
\Phi(m, n; \theta) = T(\theta)\Phi(m, n; 0)T(-\theta)
\]
\end{proof}
\end{lemma}

Since $Q(n)$ decays exponentially to 0 at $\pm \infty$ and $T(\theta)$ is an isometry, the matrix $A(n; \theta) = DF(T(\theta) Q(n))$ in \eqref{Vtheta} is exponentially asymptotic to the constant coeffient matrix $A_\infty = DF(0)$, which is hyperbolic by Hypothesis \ref{initialhyp} and does not depend on $\theta$. Using Lemma \ref{dichotomy}, let $P_{s/u}^\pm(m; \theta)$ and $\Phi_{s/u}^\pm(m, n; \theta)$ be the projections and evolutions of the exponential dichotomy for the ODE \ref{Vtheta}. We note that the estimates from this lemma do not depend on $\theta$. Furthermore, if we take Hypothesis \ref{intersectionhyp}(ii), $T(\theta)S_1(n)$ is a solution to \eqref{Vtheta}.

We write equation \eqref{Wsystem1} in fixed-point form using the discrete variation of constants formula from Lemma \ref{VOCformula}.
\begin{equation}\label{FPeqs1}
\begin{aligned}
W_i^-(n) &= 
\Phi_s^-(n, -N_{i-1}^-; \theta_i) a_{i-1}^- + \Phi_u^-(n, 0; \theta_i) b_i^-  \\
&+ \sum_{j = -N_{i-1}^-}^{n-1} \Phi_s^-(n, j+1; \theta_i) G_i^-(W_i^-(j)) - \sum_{j = n}^{-1} \Phi_u^-(n, j+1; \theta_i) G_i^-(W_i^-(j)) \\
W_i^+(n) &= \Phi_u^+(n, N_i^+; \theta_i) a_i^+ + \Phi_s^+(n, 0; \theta_i) b_i^+ \\
&+ \sum_{j = 0}^{n-1} \Phi_s^+(n, j+1; \theta_i) G_i^+(W_i^+(j)) 
- \sum_{j = n}^{N_i^+-1} \Phi_u^+(n, j+1; \theta_i) G_i^+(W_i^+(j))
\end{aligned}
\end{equation}
where 
\begin{enumerate}
\item $a_i^- \in E^s$, $a_i^+ \in E^u$, and $a_0^- = a_m^+ = 0$
\item $b_i^+ \in T(\theta_i) Y^+$ and $b_i^- \in T(\theta_i) Y^-$
\item $W_i^-(n) \in l^\infty([-N_{i-1}^-, 0])$ and $W_i^+(n) \in l^\infty([0, N_i^+])$
\end{enumerate}

Since we are constructing a homoclinic orbit to the rest state at 0, we must take the initial conditions $a_0^- = 0$ and $a_m^+ = 0$. In these cases, the fixed point equations are given by
\begin{align*}
W_1^-(n) &= \Phi_u^-(n, 0; \theta_i) b_i^- 
+ \sum_{j = -\infty}^{n-1} \Phi_s^-(n, j+1; \theta_i) G_i^-(W_i^-(j)) - \sum_{j = n}^{-1} \Phi_u^-(n, j+1; \theta_i) G_i^-(W_i^-(j)) \\
W_m^+(n) &= \Phi_s^+(n, 0; \theta_i) b_i^+ 
+ \sum_{j = 0}^{n-1} \Phi_s^+(n, j+1; \theta_i) G_i^+(W_i^+(j)) 
- \sum_{j = n}^\infty \Phi_u^+(n, j+1; \theta_i) G_i^+(W_i^+(j))
\end{align*}
where the infinite sums converge due to the exponential dichotomy. 

We note that if Hypothesis \ref{intersectionhyp}(ii) holds, we do not need to include a component in $T(\theta_i) S_1$ in $b_i^\pm$, since that direction is taken care of by the symmetry parameter $\theta_i$.

\subsection{Inversion}

As in San98, we will solve equations \eqref{Wsystem1}, \eqref{Wsystem2}, and \eqref{Wsystem3} in stages. The first two steps are the same regardless of whether we take Hypothesis \ref{intersectionhyp}(i) or \ref{intersectionhyp}(ii). In the first lemma of this section, we solve for $W_i^\pm$. 

% solve for W
\begin{lemma}\label{inv1}
There exist unique bounded functions $W_i^\pm(n)$ such that equation \eqref{Wsystem1} is satisfied. These solutions depend smoothly on the initial conditions $a_i^\pm$ and $b_i\pm$, and we have the estimates.
\begin{equation}\label{Wipmest}
\begin{aligned}
||W_i^-|| &\leq C (|a_{i-1}^-| + |b_i^-|) \\
||W_i^+|| &\leq C (|a_i^+| + |b_i^+| )
\end{aligned}
\end{equation}
For the interior pieces, we have the piecewise estimates
\begin{equation}\label{Wipiecewise}
\begin{aligned}
|W_i^-(n)| &\leq C (r^{-(N_{i-1}^- + n)}|a_{i-1}^-| + r^n|b_i^-|) && n \in [-N_{i-1}^-, 0] \\
|W_i^+(n)| &\leq C (r^{-(N_i^+ - n)}|a_i^+| + r^{-n}|b_i^+| ) && n \in [0, N_i^+] 
\end{aligned}
\end{equation}

\begin{proof}
First, we show that the RHS of the fixed point equations \eqref{FPeqs1} defines a smooth map from $l^\infty$ (on the appropriate interval) to itself. We will verify the ``minus'' pieces; the ``plus'' pieces will be similar.

For the terms not involving sums, we have
\begin{align*}
|\Phi_s^-(n, &-N_{i-1}^-; \theta_i) a_{i-1}^-| + |\Phi_u^-(n, 0; \theta_i) b_i^-| \\
&\leq C ( r^{-(n + N_{i-1}^-)} |a_{i-1}^-| +  r^{-n}|b_i^-|) \\
&\leq C ( |a_{i-1}^-| + |b_i^-|) 
\end{align*}
which is independent of $n$. For the terms involving sums, we have
\begin{align*}
&\left| \sum_{j = -N_{i-1}^-}^{n-1} \Phi_s^-(n, j+1; \theta_i) G_i^-(W_i^-(j))\right| + \left|\sum_{j = n}^{-1} \Phi_u^-(n, j+1; \theta_i) G_i^-(W_i^-(j))\right| \\
&\leq C ||W_i^-||_{l^\infty([-N_{i-1}, 0])}^2 \left( \sum_{j = -N_{i-1}^-}^{n-1} r^{-(n - j - 1)} + \sum_{j = n}^{-1} r^{-(n - j - 1)} \right) \\
&= C ||W_i^-||_{l^\infty([-N_{i-1}, 0])}^2 \sum_{j = -N_{i-1}^-}^{-1} r^{-(n - j - 1)} \\
&= C ||W_i^-||_{l^\infty([-N_{i-1}, 0])}^2 r^{-n} \sum_{j = 0}^{N_{i-1} - 1} r^{-j} \\
&\leq C ||W_i^-||_{l^\infty([-N_{i-1}, 0])}^2 
\end{align*}
which is finite and also independent of $n$. Define the map $H_i^-: l^\infty([-N_{i-1}, 0]) \times E^s \times Y^- \rightarrow l^\infty([-N_{i-1}, 0])$ by
\begin{align*}
H_i^-(W_i^-(n), &a_{i-1}^-, b_i^-) = W_i^-(n) - \Phi_s^-(n, -N_{i-1}^-; \theta_i) a_{i-1}^- - \Phi_u^-(n, 0; \theta_i) b_i^-  \\
&- \sum_{j = -N_{i-1}^-}^{n-1} \Phi_s^-(n, j+1; \theta_i) G_i^-(W_i^-(j)) + \sum_{j = n}^{-1} \Phi_u^-(n, j+1; \theta_i) G_i^-(W_i^-(j)) 
\end{align*}
Since 0 is an equilibrium, $H(0, 0, 0) = 0$. We can check that the Fr\'echet derivative of $H_i^-$ with respect to $W_i^-$ at $(W_i^-(n), a_{i-1}^-, b_i^-) = (0, 0, 0)$ is a Banach space isomorphism on $l^\infty([-N_{i-1}, 0])$. Using the implicit function theorem, we can solve for $W_i^-(x)$ in terms of $(a_{i-1}^-, b_i^-)$. This dependence is smooth, since the map $H_i^-$ is smooth. The estimate \eqref{Wipmest} on $W_i^\pm$ comes from the terms in the fixed point equations which do not involve sums, since the terms involving sums are quadratic in $W_i^\pm$. For the interior pieces, we can similarly obtain the piecewise estimates
\begin{align*}
|W_i^-(n)| &\leq C (r^{-(N_{i-1}^- + n)}|a_{i-1}^-| + r^n|b_i^-|) && n \in [-N_{i-1}^-, 0] \\
|W_i^+(n)| &\leq C (r^{-(N_i^+ - n)}|a_i^+| + r^{-n}|b_i^+| ) && n \in [0, N_i^+] 
\end{align*}
\end{proof}
\end{lemma}

Next, we use the center matching conditions at $N_i^\pm$ to solve for the initial conditions $a_i^\pm$.

% match in middles
\begin{lemma}\label{inv2}
For $i = 1, \dots m-1$ there is a unique pair of initial conditions $(a_i^+, a_i^-) \in E^u \times E^s$ such that the matching conditions \eqref{Wsystem2} are satisfied. $(a_i^+, a_i^-)$ depends smoothly on $(b_i^+, b_{i+1}^-)$, and we have the following expressions for $a_i^-$ and $a_i^+$. 
\begin{equation}\label{aipmest}
\begin{aligned}
a_i^+ &= P_0^u d_i + \tilde{a}_i^+ \\
a_i^- &= -P_0^s d_i + \tilde{a}_i^-
\end{aligned}
\end{equation}
where 
\begin{equation}\label{tildeaest}
\tilde{a}_i^\pm = \mathcal{O}(r^{-N}(|b_i^+|+|b_{i+1}^-|) + |b_i^+|^2+|b_{i+1}^-|^2) 
\end{equation}
In terms of $Q(\pm N_i^\pm)$, we can write \eqref{aipmest} as 
\begin{equation}\label{aipmexp}
\begin{aligned}
a_i^- &= T(\theta_i) Q(N_i^+) + \tilde{a}_i^- + \mathcal{O}(r^{-2N}) \\
a_i^+ &= T(\theta_{i+1}) Q(-N_i^-) + \tilde{a}_i^+ + \mathcal{O}(r^{-2N}) \\
\end{aligned}
\end{equation}

\begin{proof}
Evaluating the fixed point equations \eqref{FPeqs1} at $\pm N_i^\pm$ and subtracting, we need to solve
\begin{align*}
d_i &= W_i^+(N_i^+) - W_{i+1}^-(-N_i^-) \\
&= P_u^+(N_i^+; \theta_i) a_i^+ + \Phi_s^+(N_i^+, 0; \theta_i) b_i^+ + \sum_{j = 0}^{N_i^+-1} \Phi_s^+(N_i^+, j+1; \theta_i) G_i^+(W_i^+(j)) \\
&-P_s^-(-N_i^-; \theta_{i+1}) a_i^- - \Phi_u^-(-N_i^-, 0; \theta_{i+1}) b_{i+1}^-
- \sum_{j = -N_i^-}^{-1} \Phi_u^-(-N_i^-, j+1; \theta_{i+1}) G_i^-(W_i^-(j))
\end{align*}
Define $H_i: E^s \times E^u \times Y^+ \times Y^- \times \rightarrow \R^d$ by
\begin{align*}
H_i(a_i^+, &a_i^-, b_i^+, b_{i+1}^-) \\
&= a_i^+ - a_i^- - d_i + (P_u^+(N_i^+; \theta_i) - P_0^u) a_i^+ - (P_s^-(-N_i^-; \theta_{i+1}) - P_0^s) a_i^- \\
&+ \Phi_s^+(N_i^+, 0; \theta_i) b_i^+ - \Phi_u^-(-N_i^-, 0; \theta_{i+1}) b_{i+1}^- \\
&+ \sum_{j = 0}^{N_i^+-1} \Phi_s^+(N_i^+, j+1; \theta_i) G_i^+(W_i^+(j; a_i^+, b_i^+)) \\
&+ \sum_{j = -N_i^-}^{-1} \Phi_u^-(-N_i^-, j+1; \theta_{i+1}) G_i^-(W_{i+1}^-(j; a_i^-, b_{i+1}^-))
\end{align*}
where we substituted $W_{i+1}^-(n; a_i^-, b_{i+1}^-)$ and $W_i^+(n; a_i^+, b_i^+)$ from the previous lemma. Since $(a_i^+, a_i^-) \in E^s \oplus E^u = \R^d$, we can use the IFT. First, we note that $H_i(0,0,0,0) = 0$. Next, we differentiate $H_i$ with respect to $a_i^\pm$. When we do this, the derivatives of the terms involving sums will be 0 since $G_i^\pm$ is quadratic in $W_i^\pm$, thus quadratic order in $a_i^\pm$ by Lemma \ref{inv1}. Thus the partial derivative with respect to $a_i^\pm$ at $(a_i^+, a_i^-, b_i^+, b_{i+1}^-) = (0, 0, 0, 0)$ is 
\begin{align*}
\frac{\partial}{\partial a_i^-} H_i(0, 0, 0, 0) &= \pm 1 + \mathcal{O}(r^{-N_i^-}) \\
\frac{\partial}{\partial a_i^+} H_i(0, 0, 0, 0) &= \pm 1 + \mathcal{O}(r^{-N_i^+})
\end{align*}
For sufficiently large $N$, $D_{a_i^\pm} H(0, 0, 0, 0, 0)$ is invertible in a neighborhood of $(0, 0, 0, 0, 0)$, thus we can use the IFT to solve for $a_i^\pm$ in terms of $(b_i^+$, $b_{i+1}^-)$ for $(b_i^+, b_{i+1}^-)$ sufficiently small.

To get the estimates on and expressions for $a_i^\pm$, we note that we solved the equation $H_i(a_i^+, a_i^-, b_i^+, b_{i+1}^-) = 0$, which is of the form
\begin{align*}
0 &= a_i^+ - a_i^- - d_i + \mathcal{O}(r^{-N}(|a_i^+|+|a_i^-|+|b_i^+|+|b_{i+1}^-|)\\
&+ \mathcal{O}((|a_i^+|^2+|a_i^-|^2+|b_i^+|^2+|b_{i+1}^-|^2)
\end{align*}
Taking projections on $E^s$ and $E^u$, we have
\begin{align*}
a_i^+ &= P_0^u d_i + \mathcal{O}(r^{-N}(|b_i^+|+|b_{i+1}^-|) + |b_i^+|^2+|b_{i+1}^-|^2) \\
a_i^- &= -P_0^s d_i + \mathcal{O}(r^{-N}(|b_i^+|+|b_{i+1}^-|) + |b_i^+|^2+|b_{i+1}^-|^2)
\end{align*}
We can write this as 
\begin{align*}
a_i^+ &= P_0^u d_i + \tilde{a}_i^+ \\
a_i^- &= -P_0^s d_i + \tilde{a}_i^-
\end{align*}
where
\[
\tilde{a}_i^\pm = \mathcal{O}(r^{-N}(|b_i^+|+|b_{i+1}^-|) + |b_i^+|^2+|b_{i+1}^-|^2) 
\]

To obtain expressions in terms of $Q(\pm N_i^\pm)$, we note that
\begin{align*}
P_0^s T(\theta_i) Q(N_i^+) &= (P_0^s - P_s^+(N_i^+; \theta_i)) T(\theta_i) Q(N_i^+) + P_s^+(N_i^+; \theta_i) T(\theta_i) Q(N_i^+) \\
&= P_s^+(N_i^+; \theta_i) T(\theta_i) Q(N_i^+) + \mathcal{O}(r^{-2N}) \\
&= T(\theta_i) Q(N_i^+) + \mathcal{O}(r^{-2N}) 
\end{align*}
and 
\begin{align*}
P_0^s T(\theta_{i+1}) Q(-N_i^-) 
&= P_0^s \Big( ( P_u^-(-N_i^-; \theta_{i+1}) - P_0^u) T(\theta_{i+1}) Q(-N_i^-) + P_0^u T(\theta_{i+1}) Q(-N_i^-) \Big) \\
&= P_0^s ( P_u^-(-N_i^-; \theta_{i+1}) - P_0^u) T(\theta_{i+1}) Q(-N_i^-) + P_0^s P_0^u T(\theta_{i+1}) Q(-N_i^-) \\
&= \mathcal{O}(r^{-2N})
\end{align*}
Using these, we obtain
\begin{align*}
a_i^- &= T(\theta_i) Q(N_i^+) + \tilde{a}_i^- + \mathcal{O}(r^{-2N})
\end{align*}
Similarly, we have
\begin{align*}
a_i^+ &= T(\theta_{i+1}) Q(-N_i^-) + \tilde{a}_i^+ + \mathcal{O}(r^{-2N}) \\
\end{align*}
\end{proof}
\end{lemma}

It only remains to solve \eqref{Wsystem3}. Here, the two cases from Hypothesis \ref{intersectionhyp} are very different.

First, we consider the transverse intersection case, i.e. we assume Hypothesis \ref{intersectionhyp}(i). In this case, existence of the multi-pulse solution is a straightforward application of the implicit function theorem.

% center match, transverse intersection
\begin{lemma}\label{inv3t}
Assume Hypothesis \ref{intersectionhyp}(i). Then for $i = 1, \dots m$ there is a unique pair of initial conditions $(b_i^-, b_i^+) \in T(\theta_i) Y^- \times T(\theta_i) Y^+$ such that the matching conditions \eqref{Wsystem3} are satisfied. We have the uniform bound
\begin{equation}\label{bboundt}
b = \mathcal{O}(r^{-2N})
\end{equation}

\begin{proof}
Evaluating the fixed point equations \eqref{FPeqs1} at 0 and subtracting, we have
\begin{align*}
W_i^+(0) &- W_i^-(0) = b_i^+ - b_i^- 
+ \Phi_u^+(0, N_i^+; \theta_i) a_i^+ - \Phi_s^-(0, -N_{i-1}^-; \theta_i) a_{i-1}^- \\
&- \sum_{j = 0}^{N_i^+-1} \Phi_u^+(0, j+1; \theta_i) G_i^+(W_i^+(j)) 
- \sum_{j = -N_{i-1}^-}^{-1} \Phi_s^-(0, j+1; \theta_i) G_i^-(W_i^-(j)) \\
\end{align*}
Next, substitute $W_i^\pm$ from Lemma \ref{inv1} and $a_i^\pm$ from Lemma \ref{inv2}. Define the spaces
\begin{align}\label{spaceYt}
Y &= \bigoplus_{i=1}^m (T(\theta) Y^+ \oplus T(\theta) 2Y^-) = \bigoplus_{i=1}^m \R^d \\
Z &= \bigoplus_{i=1}^{m-1} \R^d
\end{align}
Let $b = (b_1^+, b_1^-, \dots, b_m^+, b_m^-) \in Y$ and $d = (d_1, \dots, d_{m-1}) \in Z$. Define the function $H: Y \times Z \rightarrow Y$ component-wise by
\begin{align*}
H_i(b, d) &= 
 b_i^+ - b_i^- + \Phi_u^+(0, N_i^+; \theta_i) P_0^u d_i + \Phi_s^-(0, -N_{i-1}^-; \theta_i) P_0^s d_{i-1} \\
&+ \Phi_u^+(0, N_i^+; \theta_i) \tilde{a}_i^+(b_i^+, b_{i+1}^-) 
- \Phi_s^-(0, -N_{i-1}^-; \theta_i) \tilde{a}_{i-1}^-(b_{i-1}^+, b_i^-) \\
&- \sum_{j = 0}^{N_i^+-1} \Phi_u^+(0, j+1; \theta_i) G_i^+(W_i^+(j; b_i^+, b_{i+1}^-)) \\
&- \sum_{j = -N_{i-1}^-}^{-1} \Phi_s^-(0, j+1; \theta_i) G_i^-(W_i^-(j; b_{i-1}^+, b_i^-))
\end{align*}

where $d_0 = d_m = 0$, and we have indicated the dependencies on the $b_i^\pm$. From \eqref{spaceYt}, we are set up to use the implicit function theorem. Using the estimates from Lemmas \ref{inv1} and \ref{inv2}, $H(0, 0) = 0$. For the partial derivatives with respect to $b_i^\pm$, we have
\begin{align*}
\frac{\partial}{\partial b_i^+}H_i(0) &= 1 + \mathcal{O}(r^{-N})  \\
\frac{\partial}{\partial b_i^-}H_i(0) &= -1 + \mathcal{O}(r^{-N}) \\
\frac{\partial}{\partial b_{i-1}^+}H_i(0),
\frac{\partial}{\partial b_{i+1}^-}H_i(0) &= \mathcal{O}(r^{-N}) \\
\end{align*}
and 
\[
\frac{\partial}{\partial b_j^\pm}H_i(0) = 0
\]
for all other indices. Thus, for sufficiently large $N$, the matrix $D_b H(0,0)$ is invertible. Using the implicit function theorem, we can solve for $b$ in terms of $d$ for near $(b,d) = (0, 0)$, i.e. there is a function $b: Z \rightarrow Y$ with $b(0) = 0$ such that $H(b(d),d) = 0$ for $d$ sufficiently small. Since $d = \mathcal{O}(r^{-N}$, we can do this as long as $N$ is sufficiently large.

To find a bound for $b$, recall that we use the IFT to solve $H(b(d),d) = 0$, which, componentwise, has the form
\begin{align*}
0 &= b_i^+ - b_i^- + \mathcal{O}(r^{-N_i^+} d_i + r^{-N_{i-1}^-} d_{i-1}) \\
&+ \mathcal{O}( r^{-N}(|b_i^+| + |b_i^-| + |b_{i+1}^-| + |b_{i-1}^+|)
+ (|b_i^+|^2 + |b_i^-|^2 + |b_{i+1}^-|^2 + |b_{i-1}^+|^2)
\end{align*}
Since $b$ is small and $d = \mathcal{O}(r^{-N})$, we have the bound
\[
b = \mathcal{O}(r^{-2N})
\]
\end{proof}
\end{lemma}

For the non-transverse intersection case, we will not have such a straightforward result, i.e. we will not in general be able to uniquely solve equation \eqref{Wsystem3}. Instead, we will obtain a set of jump conditions in the direction of $T(\theta) Z_1(0)$ which will depend on the symmetry parameters $\theta_i$. Recall that we have the decomposition
\[
\R^d = \R S_1(0) \oplus Y^+ \oplus Y^- \oplus \R Z_1(0)
\]
Thus, projecting in these directions, we can write \eqref{Wsystem3} as the system of equations

\begin{align}
P_{T(\theta)S_1}\left( W_i^+(0) - W_i^-(0) \right) &= 0 \label{jumpS1} \\
P_{T(\theta)Y^+ \oplus T(\theta)Y^-}\left( W_i^+(0) - W_i^-(0) \right) &= 0 \label{jumpnonZ} \\
P_{\R T(\theta)Z_1(0)} \left( W_i^+(0) - W_i^-(0) \right) &= 0 \label{jumpZ}
\end{align}

Since by \eqref{W0loc} we were able to choose $W_i^\pm$ such that $W_i^\pm(0) \in Y^+ \oplus Y^- \oplus \R Z_1(0)$, equation \ref{jumpS1} is automatically satisfied. Since $b_i^+ \in T(\theta) Y^+$ and $b_i^- \in T(\theta) Y^-$, we will be able to satisfy \eqref{jumpnonZ} by solving for the $b_i^\pm$, which we do in the following lemma.

% solve for $b_i^\pm$, nontransverse int
\begin{lemma}\label{inv3nt}
Assume Hypothesis \ref{intersectionhyp}(ii). Then for $i = 1, \dots m$ there is a unique pair of initial conditions $(b_i^-, b_i^+) \in T(\theta_i) Y^- \times T(\theta_i) Y^+$ such that the jump conditions \eqref{jumpnonZ} are satisfied. We have the uniform bound
\begin{equation}\label{bboundt}
b = \mathcal{O}(r^{-2N})
\end{equation}

\begin{proof}
Evaluating the fixed point equations \eqref{FPeqs1} at 0 and subtracting, we have
\begin{align*}
W_i^+(0) &- W_i^-(0) = b_i^+ - b_i^- 
+ \Phi_u^+(0, N_i^+; \theta_i) a_i^+ - \Phi_s^-(0, -N_{i-1}^-; \theta_i) a_{i-1}^- \\
&- \sum_{j = 0}^{N_i^+-1} \Phi_u^+(0, j+1; \theta_i) G_i^+(W_i^+(j)) 
- \sum_{j = -N_{i-1}^-}^{-1} \Phi_s^-(0, j+1; \theta_i) G_i^-(W_i^-(j)) \\
\end{align*}
Substitute $W_i^\pm$ from Lemma \ref{inv1} and $a_i^\pm$ from Lemma \ref{inv2}, and project onto $T(\theta)Y^+ \oplus T(\theta)Y^-$. The remainder of the proof follows as in the proof of Lemma \ref{inv3t}.
\end{proof}
\end{lemma}

Finally, we will look at equation \ref{jumpZ}. As mentioned above, we will in general not be able to solve this equation uniquely. Instead, we will have $m$ jump conditions in the direction of $Z_1$ which will depend on the symmetry parameters $\theta_i$. We derive these jump conditions in the next lemma.

\begin{lemma}\label{jumpZlemma}
Assume Hypothesis \ref{intersectionhyp}(ii). Then the jump conditions in the direction of $Z_1$ are given by
\begin{equation}\label{jumpZ}
\begin{aligned}
\xi_1 &= \langle T(\theta_1) Z_1(N_1^+), T(\theta_2) Q(-N_1^-) \rangle + R_1  \\
\xi_i &= \langle T(\theta_i) Z_1(N_i^+), T(\theta_{i+1}) Q(-N_i^-) \rangle
- \langle T(\theta_i) Z_1(-N_{i-1}^-), T(\theta_{i-1}) Q(N_{i-1}^+) + R_i &&
i = 2, \dots, m-1 \\
\xi_m &= -\langle T(\theta_m) Z_1(-N_{m-1}^-), T(\theta_{m-1}) Q(N_{m-1}^+) + R_m
\end{aligned}
\end{equation}
for $i = 1, \dots, m$, where 
\[
|R_i| \leq C r^{-3N}
\]
\begin{proof}
Evaluating the fixed point equations \eqref{FPeqs1} at 0 and subtracting, we have
\begin{align*}
W_i^+(0) &- W_i^-(0) = b_i^+ - b_i^- 
+ \Phi_u^+(0, N_i^+; \theta_i) a_i^+ - \Phi_s^-(0, -N_{i-1}^-; \theta_i) a_{i-1}^- \\
&- \sum_{j = 0}^{N_i^+-1} \Phi_u^+(0, j+1; \theta_i) G_i^+(W_i^+(j)) 
- \sum_{j = -N_{i-1}^-}^{-1} \Phi_s^-(0, j+1; \theta_i) G_i^-(W_i^-(j)) \\
\end{align*}
Substitute \eqref{aipmest} from Lemma \ref{inv1} to get
\begin{align*}
W_i^+(0) &- W_i^-(0) = \Phi_u^+(0, N_i^+; \theta_i) P_0^u d_i + \Phi_s^-(0, -N_{i-1}^-; \theta_i) P_0^s d_{i-1} \\
&+ b_i^+ - b_i^- 
+ \Phi_u^+(0, N_i^+; \theta_i) \tilde{a}_i^+ - \Phi_s^-(0, -N_{i-1}^-; \theta_i) \tilde{a}_{i-1}^- \\
&- \sum_{j = 0}^{N_i^+-1} \Phi_u^+(0, j+1; \theta_i) G_i^+(W_i^+(j)) 
- \sum_{j = -N_{i-1}^-}^{-1} \Phi_s^-(0, j+1; \theta_i) G_i^-(W_i^-(j)) \\
\end{align*}
Now, project on $\R T(\theta_i) Z_1(0)$ by taking the inner product with $T(\theta_i) Z_1$. Since $b_i^\pm \in T(\theta_i) Y_i^\pm$, these terms are eliminated with the projection. For the leading order terms, we have
\begin{align*}
\langle T(\theta_i) Z_1(0), &\Phi_u^+(0, N_i^+; \theta_i) P_0^u d_i \rangle
= \langle \Phi_u^+(N_i^+, 0; \theta_i)^* T(\theta_i) Z_1(0), P_0^u d_i \rangle \\
&= \langle T(\theta_i) Z_1(N_i^+), P_0^u d_i \rangle \\
&= \langle T(\theta_i) Z_1(N_i^+), P_0^u T(\theta_{i+1}) Q(-N_i^-) - P_0^u T(\theta_i) Q(N_i^+) \rangle \\
&= \langle T(\theta_i) Z_1(N_i^+), P_-^u(-N_i^-; \theta_{i+1}) T(\theta_{i+1}) Q(-N_i^-) - P_0^u P_+^s(N_i^+; \theta_i)  T(\theta_i) Q(N_i^+) \rangle + \mathcal{O}(r^{-3N}) \\
&= \langle T(\theta_i) Z_1(N_i^+), T(\theta_{i+1}) Q(-N_i^-) - P_0^u P_0^s T(\theta_i) Q(N_i^+) \rangle + \mathcal{O}(r^{-3N}) \\
&= \langle T(\theta_i) Z_1(N_i^+), T(\theta_{i+1}) Q(-N_i^-) \rangle + \mathcal{O}(r^{-3N}) 
\end{align*}
Similarly,
\begin{align*}
\langle T(\theta_i) Z_1(0), &\Phi_s^-(0, -N_{i-1}^-; \theta_i) P_0^s d_{i-1} \rangle
= -\langle T(\theta_i) Z_1(-N_{i-1}^-), T(\theta_{i-1}) Q(N_{i-1}^+) \rangle + \mathcal{O}(r^{-3N}) 
\end{align*}

For the higher order terms, substitute $W_i^\pm$ from Lemma \ref{inv1}, $\tilde{a}_i^\pm$ from Lemma \ref{inv2}, and $b_i^\pm$ from Lemma \ref{inv3nt}. This gives us the following bounds.
\begin{enumerate}
	\item For the terms involving $\tilde{a}$, 
	\begin{align*}
	|\Phi_u^+(0, N_i^+; \theta_i) \tilde{a}_i^+| 
	&\leq C r^{-N} (r^{-N}(|b_i^+|+|b_{i+1}^-|) + |b_i^+|^2+|b_{i+1}^-|^2) \\
	&\leq C r^{-N} (r^{-N}r^{-2N} + r^{-4N}) \\
	&\leq C r^{-4N}
	\end{align*}
	The other term is similar.
	\item For the terms involving sums, since $G_i^\pm(U)$ is quadratic in $U$,
	\begin{align*}
	\left| \sum_{j = 0}^{N_i^+-1} \Phi_u^+(0, j+1; \theta_i) G_i^+(W_i^+(j)) \right|
	\leq C \sum_{j = 0}^{N_i^+-1} r^{-(j+1)}|W_i^+(j)|^2
	\end{align*}
	For the interior pieces, we use the piecewise estimates \eqref{Wipiecewise} from Lemma \ref{inv1} to get
	\begin{align*}
	\left| \sum_{j = 0}^{N_i^+-1} \Phi_u^+(0, j+1; \theta_i) G_i^+(W_i^+(j)) \right|
	&\leq C \sum_{j = 0}^{N_i^+-1} r^{-(j+1)}(r^{-(N_i^+ - j)}|a_i^+| + r^{-j}|b_i^+| )^2 \\
	&\leq C \sum_{j = 0}^{N_i^+-1} r^{-(j+1)}(r^{-2(N_i^+ - j)}|a_i^+|^2 + r^{-j}|b_i^+|^2) \\
	&\leq C \left( \sum_{j = 0}^{N_i^+-1} r^{-(j+1)}r^{-2(N_i^+ - j)}|a_i^+|^2 + \sum_{j = 0}^{N_i^+-1} r^{-(j+1)} r^{-2j}|b_i^+|^2 \right) \\
	&\leq C \left( \sum_{j = 0}^{N_i^+-1} r^{-1}r^{-(N_i^+ - j)}r^{-N_i^+}r^{-2N} + \sum_{j = 0}^{N_i^+-1} r^{-(j+1)} r^{-2j}r^{-4N} \right) \\
	&\leq C r^{-3N}
	\end{align*}
	The ``negative'' piece is similar. For the end pieces, we have for $W_m^+$
	\begin{align*}
	\left| \sum_{j = 0}^{N_m^+-1} \Phi_u^+(0, j+1; \theta_m) G_m^+(W_m^+(j)) \right|
	&\leq C \sum_{j = 0}^\infty r^{-(j+1)}|b_m^+|^2 \\
	&\leq C \sum_{j = 0}^\infty r^{-(j+1)}r^{-4N} \\
	&\leq C r^{-4N}
	\end{align*}
	The other end piece involving $W_1^-$ is similar.
\end{enumerate}
Combining these, we obtain the $m$ jump conditions
\[
\xi_i = \langle T(\theta_i) Z_1(N_i^+), T(\theta_{i+1}) Q(-N_i^-) \rangle
- \langle T(\theta_i) Z_1(-N_{i-1}^-), T(\theta_{i-1}) Q(N_{i-1}^+) + R_i
\]
where 
\[
|R_i| \leq C r^{-3N}
\]
Since $N_0^- = N_m^+ = \infty$, one of the two inner product terms vanishes in $\xi_1$ and $\xi_m$, leaving us with
\begin{align*}
\xi_1 &= \langle T(\theta_1) Z_1(N_1^+), T(\theta_2) Q(-N_1^-) \rangle + R_1  \\
\xi_i &= \langle T(\theta_i) Z_1(N_i^+), T(\theta_{i+1}) Q(-N_i^-) \rangle
- \langle T(\theta_i) Z_1(-N_{i-1}^-), T(\theta_{i-1}) Q(N_{i-1}^+) + R_i &&
i = 2, \dots, m-1 \\
\xi_m &= -\langle T(\theta_m) Z_1(-N_{m-1}^-), T(\theta_{m-1}) Q(N_{m-1}^+) + R_m
\end{align*}

\end{proof}
\end{lemma}

\subsection{Proofs of Theorems \ref{transversemulti} and \ref{ntmulti}}
The existence statement from Theorem \ref{transversemulti} follows from Lemma \ref{inv3t}, and the existence statement from Theorem \ref{ntmulti} follows from Lemma \ref{jumpZlemma}. The uniform bound $||W_i^\pm|| \leq C r^{-N}$follows from Lemma \ref{inv1} together with the estimates on $a_i^\pm$ and $b_i^\pm$. For the remaining estimates, recall that we have solved equation \eqref{Wsystem2}. Substituting \eqref{defdi} into \eqref{Wsystem2}, we have solved the equation
\begin{equation}\label{Wsubdi}
W_i^+(N_i^+) - W_{i+1}^-(-N_i^-) = T(\theta_{i+1}) Q(-N_i^-) - T(\theta_i) Q(N_i^+)
\end{equation}
Apply the projection $P^u_-(-N_i^-; \theta_{i+1})$ to both sides of \eqref{Wsubdi}, noting that it acts as the identity on $T(\theta_{i+1}) Q(-N_i^-)$. We look at the three remaining terms in \eqref{Wsubdi} one at a time. For $T(\theta_i) Q(N_i^+)$, we use Lemma \ref{dichotomy} to get
\begin{align*}
P^u_-(-N_i^-; \theta_{i+1})T(\theta_i) Q(N_i^+)
&= P^u_-(-N_i^-; \theta_{i+1}) P^s_+(N_i^+ \theta_i) T(\theta_i) Q(N_i^+) \\
&= P^u_0 P^s_+(N_i^+ \theta_i) T(\theta_i) Q(N_i^+) + \mathcal{O}(r^{-2N}) \\
&= P^u_0 P^s_0 T(\theta_i) Q(N_i^+) + \mathcal{O}(r^{-2N}) \\
&= \mathcal{O}(r^{-2N})
\end{align*}
For $W_i^+(N_i^+)$, we use the fixed point equations \eqref{FPeqs1} and the uniform bound on $W_i^\pm$ to get
\begin{align*}
(I - &P^u_-(-N_i^-; \theta_{i+1})) W_i^+(N_i^+) = P^s_-(-N_i^-; \theta_{i+1}) W_i^+(N_i^+) \\
&= P^s_0 W_i^+(N_i^+)W_i^+(N_i^+) + \mathcal{O}(r^{-2N}) \\
&= P^s_+(N_i^+; \theta_i)W_i^+(N_i^+) + \mathcal{O}(r^{-2N}) \\
&= P^s_+(N_i^+; \theta_i) P^u_+(N_i^+; \theta_i) a_i^+ + P^s_+(N_i^+; \theta_i) \left( \Phi_s^+(N_i^+, 0; \theta_i) b_i^+ + \sum_{j = 0}^{N_i^+-1} \Phi_s^+(n, j+1; \theta_i) G_i^+(W_i^+(j)) \right) \\
&= \mathcal{O}(r^{-2N})
\end{align*}
Thus we conclude 
\[
P^u_-(-N_i^-; \theta_{i+1}) W_i^+(N_i^+) = W_i^+(N_i^+) + \mathcal{O}(e^{-2N})
\]
For $W_{i+1}^-(-N_i^-)$, we use the fixed point equation \eqref{FPeqs1} and, following a similar procedure, conclude that
\[
P^u_-(-N_i^-; \theta_{i+1}) W_{i+1}^-(-N_i^-) = \mathcal{O}(e^{-2N})
\]
Combining all of these, we attain the bound
\[
W_i^+(N_i^+) = T(\theta_{i+1}) Q(-N_i^-) + \mathcal{O}(r^{-2N})
\]
Following the same method, but applying the projection $P^s_+(N_i^+; \theta_i)$ to both sides of \eqref{Wsubdi}, we also have the bound
\[
W_{i+1}^-(-N_i^-) = T(\theta_i) Q(N_i^+) + \mathcal{O}(r^{-2N})
\]

\section{Proof of Stability Theorem}

\subsection{Setup}
To prove the stability theorem, we will also use Lin's method. Assume Hypothesis \ref{initialhyp} and Hypothesis \ref{intersectionhyp}(ii). Using Theorem \ref{ntmulti}, let $Q_m(n)$ be an $m-$pulse solution to \eqref{diffeq} which resembles $m$ copies of the primary pulse solution $Q_1(n)$. We write $Q_m(n)$ piecewise as in \eqref{qmpiecewise}, and we have the following bounds.
\begin{enumerate}[(i)]
\item $Q_1(n) = \mathcal{O}(r^n)$
\item $||\tilde{Q}|| \leq C r^N$
\item $|\tilde{Q}_{i+1}^-(-N_i^-) - T(\theta_i) Q_1(N_i^+)|| \leq C r^{2N}$ 
\item $|\tilde{Q}_i^+(N_i^+) - T(\theta_{i+1}) Q_1(-N_i^-)|| \leq C r^{2N}$
\end{enumerate}

Linearizing about $Q_m(n)$, we have the difference equation
\[
V(n+1) = A(Q_m(n))V(n)
\]
This has a solution $S_m(n)$ which we can write piecewise as $S_m(n) = T(\theta_i) S_1(n) + \tilde{S}_i^\pm(n)$, and the bounds above apply. Similarly, if Hypothesis \ref{melnikovhyp} holds, there exists a solution $T_m(n)$ with $T_m(n+1) = A(Q_m(n))T_m(n) + BS(n)$; we can again write $T_m(n)$ piecewise as $T_m(n) = T(\theta_i) T_1(n) + \tilde{T}_i^\pm(n)$, and the bounds above apply. I ASSUME WE CAN DO THIS USING THE METHODS IN THE PREVIOUS SECTION.

Similar to San98, we will take a piecewise ansatz for the eigenfunction $V(n)$. The form of the ansatz will depend on whether we take Hypothesis \ref{melnikovhyp}(i) or \ref{melnikovhyp}(ii). For Hypothesis \ref{melnikovhyp}(i), let
\begin{align*}
V_i^\pm(n) &= d_i ( T(\theta_i) S_1(n) + \tilde{S}_i^\pm(n) ) + W_i^\pm(n)
&& i = 1, \dots, n-1
\end{align*}
where $d_i \in \C$. Substituting this into \eqref{latticeEVP} and simplifying, we get
\[
W_i^\pm(n) = A(Q_m(n)) W_i^\pm(n) + \lambda B W_i^\pm(n) + \lambda d_i B T(\theta_i) S_1(n) + \lambda d_i B \tilde{S}_i^\pm(n)
\]

For case (ii), we make the piecewise ansatz
\[
V_i^\pm(n) = d_i [ c_i S_1(n) + \tilde{S}_i^\pm(n) + \lambda(c_i T_1(n) + \tilde{T}_i^\pm(n))] + W_i^\pm(n)
\]
Substituting this into \eqref{latticeEVP} and simplifying, we get
\[
W_i^\pm(n) = A(q_m) W_i^\pm(n) + \lambda B W_i^\pm(n) + \lambda^2 d_i B T(\theta_i) T_1(n) + \lambda^2 d_i B \tilde{T}_i^\pm(n)
\]

Since these are almost identical, we will only consider (ii), since that version shows up in dNLS. The results for (i) will be similar. We want to write this in terms of $A(T(\theta_i) Q_1(n))$. Adding and subtracting $A(T(\theta_i) Q_1(n))$, we have
\begin{equation}\label{Weq1}
W_i^\pm(n) = A(T(\theta_i) Q_1(n)) W_i^\pm(n) + \left(A(Q_m(n)) - A(T(\theta_i) Q_1(n))\right) W_i^\pm(n) + \lambda B W_i^\pm(n) + \lambda^2 d_i c_i B S_1(n) + \lambda^2 d_i B \tilde{S}_i^\pm(n)
\end{equation}
Let
\begin{align*}
G_i^\pm(n) &= A(Q_m(n)) - A(T(\theta_i) Q_1(n)) \\
\tilde{H}_i^\pm(n) &= BT_m(n) = B( T(\theta_i) T_1(n) + \tilde{T}_i^\pm(n) ) \\
H_i(n) &= B T(\theta_i) T_1(n) \\
\end{align*}
Then equation \eqref{Weq1} becomes
\begin{equation*}
W_i^\pm(n) = A(T(\theta_i) Q_1(n)) W_i^\pm(n) + (G_i^\pm(n) + \lambda B) W_i^\pm(n) + \lambda^2 d_i B \tilde{H}_i^\pm(n)
\end{equation*}

To constuct an eigenfunction, we add appropriate matching conditions at $n = \pm N_i$ and $n = 0$ to get the following system
\begin{align}
W_i^\pm(n) &= A(T(\theta_i) Q_1(n)) W_i^\pm(n) + (G_i^\pm(n) + \lambda B) W_i^\pm(n) + \lambda^2 d_i B \tilde{H}_i^\pm(n) \label{eigsystem1} \\
W_i^+(N_i^+) - W_{i+1}^-(-N_i^-) &= D_i d \label{eigsystem2} \\
W_i^\pm(0) &\in \C T(\theta_i) Z_1(0) \oplus T(\theta_i) Y^+ \oplus T(\theta_i) Y^- \label{eigsystem3a}  \\
W_i^+(0) - W_i^-(0) &\in \C T(\theta_i) Z_1(0) \label{eigsystem3b} 
\end{align}

where
\begin{align*}
D_i d &= [ T(\theta_{i+1}) S_1(-N_i^-) + \tilde{S}_{i+1}^-(-N_i^-)] d_{i+1}
- [ T(\theta_i) S_1(N_i^+) + \tilde{S}_i^+(N_i^+)] d_i \\
&+ \lambda[ T(\theta_{i+1}) T_1(-N_i^-) + \tilde{T}_{i+1}^-(-N_i^-)] d_{i+1}
- \lambda[ T(\theta_i) T_1(N_i^+) + \tilde{T}_i^+(N_i^+)] d_i 
\end{align*}

A solution to our system above will generically have $n$ jumps at $n = 0$. Thus we have an eigenfunction if and only if the $n$ jump conditions are satisfied.
\begin{equation}\label{jumpcond}
\xi_i = \langle T(\theta_i) Z_1(0), W_i^+(0) - W_i^-(0) \rangle = 0
\end{equation}

Using the bounds above, we have the estimates
\begin{align*}
|G_i^\pm| &\leq C r^N \\
|\tilde{H}_i^\pm - H| &\leq C r^N \\
D_i d &= [ c_{i+1} S_1(-N_i^-) + c_i S_1(N_i^+) ] d_{i+1}
- [ c_i S_1(N_i^+) + c_{i+1} S_1(-N_i^-) ] d_i 
+\mathcal{O}(r^N( |\lambda| + r^N))
\end{align*}

\subsection{Fixed point formulation}
As in San98, we write \eqref{eigsystem1} as a fixed point problem using the exponential dichotomy from Lemma \ref{dichotomy}. Let $\delta > 0$ be small, and choose $N$ sufficiently large so that $r^N < \delta$. Define the family of evolution operators $\Phi(m, n; \theta_i)$ as in the existence problem. Define the spaces

\begin{align*}
V_W &= l^\infty([-N_{i-1}, 0]) \oplus l^\infty([0, N_i])  \\
V_a &= \bigoplus_{i=0}^{n-1} E^u \oplus E^s \\
V_b &= \bigoplus_{i=0}^{n-1} 
\text{ range } P_-^u(0; \theta_i) \oplus \text{ range } P_+^s(0; \theta_i)\\
V_\lambda &= B_\delta(0) \subset \C \\
V_d &= \C^d
\end{align*}

The fixed point equations follow from the variation of constants formula in Lemma \ref{VOCformula} together with the projections from the exponential dichotomy in Lemma \ref{dichotomy}.
\begin{align*}
W_i^-(n) &= 
\Phi_s^-(n, -N_{i-1}^-; \theta_i) a_{i-1}^- + \sum_{j = -N_{i-1}^-}^{n-1} \Phi_s^-(n, j+1; \theta_i)
[(G_i^-(j) + \lambda B) W_i^-(j) + \lambda^2 d_i B \tilde{H}_i^-(j)]
 \\
&+ \Phi_u^-(n, 0; \theta_i) b_i^- - \sum_{j = n}^{-1} \Phi_u^-(n, j+1; \theta_i) 
[(G_i^-(j) + \lambda B) W_i^-(j) + \lambda^2 d_i B \tilde{H}_i^-(j)] \\
W_i^+(n) &= \Phi_s^+(n, 0; \theta_i) b_i^+ + \sum_{j = 0}^{n-1} \Phi_s^+(n, j+1; \theta_i) 
[(G_i^+(j) + \lambda B) W_i^+(j) + \lambda^2 d_i B \tilde{H}_i^+(j)] \\
&+ \Phi_u^+(n, N_i^+; \theta_i) a_i^+ - \sum_{j = n}^{N_i^+-1} \Phi_u^+(n, j+1; \theta_i) 
[(G_i^+(j) + \lambda B) W_i^+(j) + \lambda^2 d_i B \tilde{H}_i^+(j)]
\end{align*}
where $a_0^- = a_m^+ = 0$ and the sums are defined to be $0$ if the upper index is smaller than the lower index. Since we are taking $a_0^- = a_m^+ = 0$, the corresponding equations are
\begin{align*}
W_1^-(n) &= \sum_{j = -\infty}^{n-1} \Phi_s^-(n, j+1; \theta_1)
[(G_i^-(j) + \lambda B) W_i^-(j) + \lambda^2 d_i B \tilde{H}_i^-(j)]
 \\
&+ \Phi_u^-(n, 0; \theta_1) b_i^- - \sum_{j = n}^{-1} \Phi_u^-(n, j+1; \theta_1) 
[(G_i^-(j) + \lambda B) W_i^-(j) + \lambda^2 d_i B \tilde{H}_i^-(j)] \\
W_m^+(n) &= \Phi_s^+(n, 0; \theta_m) b_i^+ + \sum_{j = 0}^{n-1} \Phi_s^+(n, j+1; \theta_m) 
[(G_i^+(j) + \lambda B) W_i^+(j) + \lambda^2 d_i B \tilde{H}_i^+(j)] \\
&- \sum_{j = n}^{\infty} \Phi_u^+(n, j+1; \theta_m) 
[(G_i^+(j) + \lambda B) W_i^+(j) + \lambda^2 d_i B \tilde{H}_i^+(j)]
\end{align*}

As in San98, we will now solve the system in a series of lemmas.

\subsection{The inversion}
First, we solve for $W_i^\pm$. 

% Lemma : solve for W
\begin{lemma}\label{eiginv1}
There exists an operator $W_1: V_\lambda \times V_a \times V_b \times V_d \rightarrow V_W$ such that
\[
W = W_1(\lambda)(a,b,d)
\]
is a solution to \eqref{eigsystem1} for and $(a,b,d)$ and $\lambda$. The operator $W_1$ is analytic in $\lambda$, linear in $(a,b,d)$, and has bound
\begin{equation}\label{W1bound}
||W_1(\lambda)(a,b,d)|| \leq C \left( |a| + |b| + |\lambda|^2 |d| \right)
\end{equation}

\begin{proof}
Rewrite the fixed point equations as
\[
(I - L_1(\lambda))W = L_2(\lambda)(a,b,d)
\]
where $L_1(\lambda): V_W \rightarrow V_W$ is the linear operator composed of terms in the fixed point equations involving $W$
\begin{align*}
(L_1(\lambda)W)_i^-(n) &= \sum_{j = -N_{i-1}^-}^{n-1} \Phi_s^-(n, j+1; \theta_i)
(G_i^-(j) + \lambda B) W_i^-(j) \\&- \sum_{j = n}^{-1} \Phi_u^-(n, j+1; \theta_i) 
(G_i^-(j) + \lambda B) W_i^-(j)\\
(L_1(\lambda)W)_i^+(n) &= \sum_{j = 0}^{n-1} \Phi_s^+(n, j+1; \theta_i) 
(G_i^+(j) + \lambda B) W_i^+(j) \\
&-\sum_{j = n}^{N_i^+-1} \Phi_u^+(n, j+1; \theta_i) 
(G_i^+(j) + \lambda B) W_i^+(j)
\end{align*}
and $L_2(\lambda): V_\lambda \times V_a \times V_b $ is the linear operator composed of terms in the fixed point equations not involving $W$.
\begin{align*}
(L_2(\lambda)(a,b,d))_i^-(n) &= 
\Phi_s^-(n, -N_{i-1}^-; \theta_i) a_{i-1}^- + \sum_{j = -N_{i-1}^-}^{n-1} \Phi_s^-(n, j+1; \theta_i)
\lambda d_i B \tilde{H}_i^-(j)
 \\
&+ \Phi_u^-(n, 0; \theta_i) b_i^- - \sum_{j = n}^{-1} \Phi_u^-(n, j+1; \theta_i) 
\lambda d_i B \tilde{H}_i^-(j) \\
(L_2(\lambda)(a,b,d))_i^+(n) &= \Phi_s^+(n, 0; \theta_i) b_i^+ + \sum_{j = 0}^{n-1} \Phi_s^+(n, j+1; \theta_i)\lambda^2 d_i B \tilde{H}_i^+(j) \\
&+ \Phi_u^+(n, N_i^+; \theta_i) a_i^+ - \sum_{j = n}^{N_i^+-1} \Phi_u^+(n, j+1; \theta_i)\lambda^2 d_i B \tilde{H}_i^+(j)
\end{align*}
To find a bound for $L_1$, we look at the ``minus'' piece. Note that $n \leq 0$ on this piece. Using the exponential dichotomy bounds from Lemma \ref{dichotomy},
\begin{align*}
|(L_1(\lambda)W)_i^-(n)| &\leq C (||G|| + |\lambda|)\left(
\sum_{j = -N_{i-1}^-}^{n-1} |\Phi_s^-(n, j+1; \theta_i)| + \sum_{j = n}^{-1} |\Phi_u^-(n, j+1; \theta_i)| \right) ||W|| \\
&\leq C (||G|| + |\lambda|) ||W||
\left( \sum_{j = -N_{i-1}^-}^{n-1} r^{n - j - 1} + \sum_{j = n}^{-1} r^{j+1-n} \right) \\
&\leq C (||G|| + |\lambda|) ||W||
\left( \sum_{j = 0}^{N_{i-1}^- -|n| -1} r^j + \sum_{j = 1}^{|n|} r^j \right) \\
&\leq C (||G|| + |\lambda|) ||W||\sum_{j = 0}^\infty r^j \\
&\leq C (||G|| + |\lambda|)||W||
\end{align*}
The ``plus'' piece is similar. To find a bound for $L_2$, we again look at the ``minus'' piece. Since the sums involve the same evolution operators as those in $L_1$, they have the same bounds as in $L_1$. Thus we have the bound
\begin{align*}
|(L_2(\lambda)(a,b,d))_i^-(n)| \leq C\left( |a| + |b| + |\lambda|^2 |d| \right)
\end{align*}
Overall, we have uniform bounds
\begin{align*}
||L_1(\lambda)W)|| &\leq C \left(||G|| + |\lambda| \right)||W|| \leq C \delta ||W|| \\
||L_2(\lambda)(a,b,d))|| &\leq C\left( |a| + |b| + |\lambda|^2 |d| \right)
\end{align*}
For sufficiently small $\delta$, $||(L_1(\lambda)W)|| < 1$, thus $I - L_1(\lambda)$ is invertible on $V_W$. $(I - L_1(\lambda))^{-1}$ is analytic in $\lambda$, and we obtain the solution 
\[
W = W_1(\lambda)(a,b,d) = (I - L_1(\lambda))^{-1} L_2(\lambda(a,b,d)
\]
which is analytic in $\lambda$, linear in $(a, b, d)$, and for which we have estimate
\[
||W_1(\lambda)(a,b,d)|| \leq C \left( |a| + |b| + |\lambda|^2 |d| \right)
\]
\end{proof}
\end{lemma}

In the next lemma, we solve for the matching condition at the tails of the pulses, i.e.
\[
W_i^+(N_i^+) - W_{i+1}^-(-N_i^-) = D_i d
\]
for $i = 1, \dots, m-1$. 

% lemma : solve for a
\begin{lemma}\label{eiginv2}
There exist operators 
\begin{align*}
A_1 : V_\lambda \times V_b \times V_d \rightarrow V_a \\
W_2 : V_\lambda \times V_b \times V_d \rightarrow V_W
\end{align*}
such that $(a, w) = (A_1(\lambda)(b,d), W_2(\lambda)(b,d)$ solves \eqref{eigsystem1} and \eqref{eigsystem2} for any $(b, d)$ and $\lambda$. These operators are analytic in $\lambda$, linear in $(b,d)$, and have bounds 
\begin{align}
|A_1(\lambda)(b, d)| &\leq C \left( (r^N + ||G|| + |\lambda| ) |b| + (|\lambda|^2 + |D| ) |d| \right) \label{A1bound} \\
||W_2(\lambda)(b,d)|| &\leq C \left( |b| + (|\lambda|^2 + |D|) |d| \right) \label{W2bound}
\end{align}
Furthermore, we can write
\begin{align*}
a_i^+ &= P_0^u D_i d + A_2(\lambda)_i(b,d) \\
a_i^- &= -P_0^s D_i d + A_2(\lambda)_i(b,d)
\end{align*}
where $A_2$ is a bounded linear operator with bound
\begin{align}\label{A2bound}
|A_2(\lambda)(b,d)| \leq 
C\left( (r^N + ||G|| + |\lambda| )|b| + (r^N + ||G|| + |\lambda|)|D||d| + |\lambda|^2 |d|  \right)
\end{align}

\begin{proof}
At $n = \pm \N_i^\pm$, the fixed point equations become
\begin{align*}
W_{i+1}^-(-N_i^-) &= 
\Phi_s^-(-N_i^-, -N_i^-; \theta_{i+1}) a_i^- + \Phi_u^-(-N_i^-, 0; \theta_{i+1}) b_i^- \\
&- \sum_{j = -N_i^-}^{-1} \Phi_u^-(-N_i^-, j+1; \theta_{i+1}) 
[(G_i^-(j) + \lambda B) W_i^-(j) + \lambda^2 d_i B \tilde{H}_i^-(j)] \\
W_i^+(N_i^+) &= \Phi_u^+(N_i^+, N_i^+; \theta_i) a_i^+ + \Phi_s^+(N_i^+, 0; \theta_i) b_i^+ \\
&+ \sum_{j = 0}^{N_i^+-1} \Phi_s^+(N_i^+, j+1; \theta_i) 
[(G_i^+(j) + \lambda B) W_i^+(j) + \lambda^2 d_i B \tilde{H}_i^+(j)]
\end{align*}
Note that $\Phi_s^-(-N_i^-, -N_i^-; \theta_{i+1}) = P_-^s(-N_i^-; \theta_{i_1}$ and $\Phi_u^+(N_i^+, N_i^+; \theta_i) = P_+^u(N_i^+; \theta_{i_1})$. Recalling that $a_i^- \in E^s$ and $a_i^+ \in E^u$, we add and subtract $P_0^{s/u}$ to get
\begin{align*}
W_{i+1}^-(-N_i^-) &= 
a_i^- + (P_s^-(-N_i^-; \theta_{i+1}) - P_0^s) a_i^- + \Phi_u^-(-N_i^-, 0; \theta_{i+1}) b_i^- \\
&- \sum_{j = -N_i^-}^{-1} \Phi_u^-(-N_i^-, j+1; \theta_{i+1}) 
[(G_i^-(j) + \lambda B) W_i^-(j) + \lambda^2 d_i B \tilde{H}_i^-(j)] \\
W_i^+(N_i^+) &= a_i^+ + (P_u^+(N_i^+; \theta_i) - P_0^u) a_i^+ + \Phi_s^+(N_i^+, 0; \theta_i) b_i^+ \\
&+ \sum_{j = 0}^{N_i^+-1} \Phi_s^+(N_i^+, j+1; \theta_i) 
[(G_i^+(j) + \lambda B) W_i^+(j) + \lambda^2 d_i B \tilde{H}_i^+(j)]
\end{align*}
Thus the condition $W_i^+(N_i^+) - W_{i+1}^-(-N_i^-) = D_i d$ can be written
\begin{align}
D_i d &= a_i^+ - a_i^- + (P_u^+(N_i^+; \theta_i) - P_0^u) a_i^+ - (P_s^-(-N_i^-; \theta_{i+1}) - P_0^s) a_i^- \\
&+ \Phi_s^+(N_i^+, 0; \theta_i) b_i^+ - \Phi_u^-(-N_i^-, 0; \theta_{i+1}) b_i^- \nonumber \\
&+ \sum_{j = 0}^{N_i^+-1} \Phi_s^+(N_i^+, j+1; \theta_i) 
[(G_i^+(j) + \lambda B) W_i^+(j) + \lambda^2 d_i B \tilde{H}_i^+(j)] \nonumber \\
&- \sum_{j = -N_i^-}^{-1} \Phi_u^-(-N_i^-, j+1; \theta_{i+1}) 
[(G_i^-(j) + \lambda B) W_i^-(j) + \lambda^2 d_i B \tilde{H}_i^-(j)] \nonumber \\
\end{align}

Substituting $W = W_1(\lambda)(a, b, d)$ from Lemma \ref{eiginv1}, we obtain an equation of the form 
\begin{equation}\label{Dideq}
D_i d = (a_i^+ - a_i^-) + L_3(\lambda)_i(a,b,d)
\end{equation}
Using Lemma \ref{dichotomy} and the bound for $W_1$ from Lemma \ref{eiginv1}, $L_3$ has uniform bound
\begin{align*}
L_3(\lambda)(a,b,d)| &\leq C\left( r^N (|a| + |b|) + \left(\sum_{j = -N_i^-}^{-1} r^{j+1+N_i^-} + \sum_{j = 0}^{N_i^+ - 1} r^{N_i^+ - j - 1} \right)((||G|| + |\lambda|)||W|| + |\lambda|^2 |d|) \right) \\
&\leq C\left( r^N (|a| + |b|) + \left(\sum_{j = 1}^{N_i^-} r^j + \sum_{j = 1}^{N_i^+ - 1} r^j \right)((||G|| + |\lambda|)||W|| + |\lambda|^2 |d| ) \right) \\
&\leq C\left( r^N (|a| + |b|) + (||G|| + |\lambda|)||W_1(\lambda)(a,b,d)|| + |\lambda|^2 |d| ) \sum_{j = 1}^\infty r^j \right) \\
&\leq C\left( (r^N + ||G|| + |\lambda| ) (|a| + |b|) + |\lambda|^2 |d|  \right)
\end{align*}
Since $r^N, |\lambda| < \delta$, this becomes
\begin{align*}
L_3(\lambda)(a,b,d)| &\leq C \delta |a| + C\left( (r^N + ||G|| + |\lambda| ) |b| + |\lambda|^2 |d|  \right)
\end{align*}

Define the map
\[
J_1: V_a \rightarrow \bigoplus_{j=1}^{m-1} \C^d
\]
by $(J_1)_i(a_i^+, a_i^-) = a_i^+ - a_i^-$. Since $E^s \oplus E^u = \C^d$.  The map $J_1$ is a linear isomorphism. Consider the map
\[
S_1(a) = J_1 (a) + L_3(\lambda)(a, 0, 0) = J_1( I + J_1^{-1} L_3(\lambda)(a, 0 0) )
\]
For sufficiently small $\delta$, the operator norm $||J_1^{-1} L_3(\lambda)(a, 0, 0)|| < 1$, thus the operator $S_1(a)$ is invertible. We can solve for $a$ to get
\[
a = A_1(\lambda)(b, d) = S_i^{-1}(-D d - L_3(\lambda)(b, d))
\]
which has uniform bound
\begin{equation*}
|A_1(\lambda)(b, d)| \leq C \left( (r^N + ||G|| + |\lambda| ) |b| + (|\lambda|^2 + |D| ) |d|  \right)
\end{equation*}
We plug this estimate into $W_1$ to get $W_2(\lambda)(b,d)$ with bound
\begin{equation*}
||W_2(\lambda)(b,d)|| \leq C \left( |b| + (|\lambda|^2 + |D|) |d| \right)
\end{equation*}

Finally, we project \eqref{Dideq} onto $E^s$ and $E^u$ using $P_0^s$ and $P_0^u$.
\begin{align*}
a_i^+ &= P_0^u D_i d - P_0^u L_3(\lambda)_i(a,b,d) \\
a_i^- &= -P_0^s D_i d + P_0^s L_3(\lambda)_i(a,b,d)
\end{align*}
Substituting $A_1(\lambda)(b,d)$ for $a$ we obtain the equations
\begin{align*}
a_i^+ &= P_0^u D_i d + A_2(\lambda)_i(b,d) \\
a_i^- &= -P_0^s D_i d + A_2(\lambda)_i(b,d)
\end{align*}
Finally, we substitute the bound for $A_1$ into the bound for $L_3$ to obtain the uniform bound
\begin{align*}
|A_2(\lambda)(b,d)| \leq 
C\left( (r^N + ||G|| + |\lambda| )|b| + (r^N + ||G|| + |\lambda|)|D||d| + |\lambda|^2 |d|  \right)
\end{align*}
\end{proof}
\end{lemma}

Finally, we will satisfy the conditions at $n = 0$
\begin{align*}
W_i^\pm(0) &\in \C T(\theta_i) Z_1(0) \oplus T(\theta_i) Y^+ \oplus T(\theta_i) Y^- \\
W_i^+(0) - W_i^-(0) &\in \C T(\theta_i) Z_1(0)
\end{align*}
Since $C^d = \C T(\theta_i) Z_1(0) \oplus \C T(\theta_i) S_1(0) \oplus T(\theta_i) Y^+ \oplus T(\theta_i) Y^-$, these are equivalent to the projections
\begin{equation}\label{projeq}
\begin{aligned}
P(T(\theta_i) S_1(0)) W_i^- &= 0 \\
P(T(\theta_i) S_1(0)) W_i^+ &= 0 \\
P(T(\theta_i) Y^+ \oplus T(\theta_i) Y^-) (W_i^+ - W_i^-) &= 0
\end{aligned}
\end{equation}
where the kernel of each projection is the remaining elements of the direct sum decomposition of $\C^d$. Since we have eliminated any component in $T(\theta_i) S_1(0)$ in the first two projections, we do not need it in the third projection.

We decompose $b_i^\pm$ uniquely as $b_i^\pm = x_i^\pm + y_i^\pm$, where $x_i^\pm \in \C T(\theta_i) S_1(0)$ and $y_i^\pm \in T(\theta_i) Y^\pm$. We can now solve for the conditions at $n = 0$, which we do in the following lemma.

% lemma: solve at n=0
\begin{lemma}\label{eiginv3}
There exist operators 
\begin{align*}
B_1 : V_\lambda \times V_d \rightarrow V_b \\
A_3 : V_\lambda \times V_d \rightarrow V_a \\
W_3 : V_\lambda \times V_d \rightarrow V_W
\end{align*}
such that $(a, b, w) = (A_3(\lambda)(d), B_1(\lambda)(d), W_2(\lambda)(d)$ solves \eqref{eigsystem1}, \eqref{eigsystem2}, \eqref{eigsystem3a}, and \eqref{eigsystem3b} for any $d$ and $\lambda$. These operators are analytic in $\lambda$, linear in $d$, and have bounds 
\begin{align}
|B_1(\lambda)(d)| &\leq C \left( (r^{N} + ||G|| + |\lambda|)|D| |d| + |\lambda|^2 |d| \right) \label{B1bound} \\
|A_3(\lambda)(d)| &\leq C \left(|\lambda|^2 + |D|\right)|d| \label{A3bound} \\
||W_3(\lambda)(d)|| &\leq C \left(|\lambda|^2 + |D|\right)|d| \label{W3bound} \\
\end{align}
Furthermore, we can write
\begin{align*}
a_i^+ &= P_0^u D_i d + A_4(\lambda)_i(d) \\
a_i^- &= -P_0^s D_i d + A_4(\lambda)_i(d)
\end{align*}
where $A_4$ is a bounded linear operator with estimate
\begin{align}\label{A4bound}
|A_4(\lambda)(d)| &\leq 
C\left( (r^N + ||G|| + |\lambda|)|D||d| + |\lambda|^2 |d|  \right)
\end{align}

\begin{proof}
At $n = 0$, the fixed point equations become 
\begin{align*}
W_i^-(0) &= x_i^- + y_i^- +
\Phi_s^-(0, -N_{i-1}^-; \theta_i) a_{i-1}^- \\
&+ \sum_{j = -N_{i-1}^-}^{-1} \Phi_s^-(0, j+1; \theta_i)
[(G_i^-(j) + \lambda B) W_i^-(j) + \lambda^2 d_i B \tilde{H}_i^-(j)] \\
W_i^+(0) &= x_i^+ + y_i^+ + \Phi_u^+(0, N_i^+; \theta_i) a_i^+ \\
&- \sum_{j = 0}^{N_i^+-1} \Phi_u^+(0, j+1; \theta_i) 
[(G_i^+(j) + \lambda B) W_i^+(j) + \lambda^2 d_i B \tilde{H}_i^+(j)]
\end{align*}
The projection equations \eqref{projeq}can thus be written as
\begin{equation}\label{projeq2}
\begin{pmatrix}
x_i^- \\ x_i^+ \\ y_i^+ - y_i^-
\end{pmatrix}
= (L_4(\lambda)(b,d))_i
\end{equation}
We use the exponential dichotomy estimates from Lemma \ref{dichotomy} and plug in $(a, w) = (A_1(\lambda)(b,d), W_2(\lambda)(b,d))$ from Lemma \ref{eiginv2} to get the uniform bound on $L_4$.
\begin{align*}
|L_4(\lambda)(b,d)| 
&\leq C \left( r^N |a| + \left(\sum_{j = -N_i^-}^{-1} r^{-j-1} + \sum_{j = 0}^{N_i^+ - 1} r^{j+1} \right)((||G|| + |\lambda|)||W|| + |\lambda|^2 |d|) \right) \\
&\leq C \left( r^N |A_1(\lambda)(b,d)| + \left(\sum_{j = 0}^{N_i^- -1} r^j + \sum_{j = 1}^{N_i^+} r^j \right)((||G|| + |\lambda|)||W|| + |\lambda|^2 |d|) \right) \\
&\leq C \left( r^N |A_1(\lambda)(b,d)| + (||G|| + |\lambda|)||W_2(\lambda(b,d)|| + |\lambda|^2 |d| \right) \\ \\
&\leq C \left( (r^{2N} + ||G|| + |\lambda|)|b| + 
(r^{N} + ||G|| + |\lambda|)|D| |d| + |\lambda|^2 |d| )
\right)
\end{align*}
Since $r^N, |\lambda| < \delta$, we have bound
\begin{align*}
|L_4(\lambda)(b,d)| &\leq C \delta(|x| + |y|) + C \left( (r^{N} + ||G|| + |\lambda|)|D| |d| + |\lambda|^2 |d| )
\right)
\end{align*}

Define the map
\[
J_2: \left( \bigoplus_{j=1}^n \C S_1(0) \oplus \C S_1(0) \right) \oplus
\left( \bigoplus_{j=1}^n Y^- \oplus Y^+ \right) 
\rightarrow \bigoplus_{j=1}^n \C S_1(0) \oplus \C S_1(0) \oplus (Y^- \oplus Y^+)
\]
by 
\[
J_2( (x_i^+, x_i^-),(y_i^+, y_i^-))_i = ( x_i^+, x_i^-, y_i^+ - y_i^- )
\]
Since $\C^d = \C T(\theta_i) Z_1(0) \oplus \C T(\theta_i) S_1(0) \oplus T(\theta_i) Y^- \oplus T(\theta_i) Y^+)$, $J_2$ is an isomorphism. Using this and the fact that $b_i = (x_i^- + y_i^-, x_i^+ + y_i^+)$, we can write \eqref{projeq2} as
\begin{equation}\label{projxy2}
J_2( (x_i^+, x_i^-),(y_i^+, y_i^-))_i 
+ L_4(\lambda)_i(b_i, 0) + L_4(\lambda)_i(0, d) = 0
\end{equation}
Consider the map
\begin{align*}
S_2(b)_i &= J_2( (x_i^+, x_i^-),(y_i^+, y_i^-))_i 
+ L_4(\lambda)_i(b_i, 0) 
\end{align*}
Substituting this in \eqref{projxy2}, we have
\begin{align*}
S_2(b) &= -L_4(\lambda)(0, d)
\end{align*}
For sufficiently small $\delta$, the operator $S_2(b)$ is invertible. Thus we can solve for $b$ to get
\begin{equation}
b = B_1(\lambda)(d) = -S_2^{-1} L_4(\lambda)(0, d)
\end{equation}
where we have the uniform bound on $B_1$
\begin{equation}
|B_1(\lambda)(d)| \leq C \left( (r^{N} + ||G|| + |\lambda|)|D| |d| + |\lambda|^2 |d| \right) 
\end{equation}

We can plug this into $A_1$, $W_2$, and $A_2$ to get operators $A_3$, $W_3$, and $A_4$ with bounds
\begin{align*}
|A_3(\lambda)(d)| &\leq C \left(|\lambda|^2 + |D|\right)|d|\\
||W_3(\lambda)(d)|| &\leq C \left(|\lambda|^2 + |D|\right)|d| \\
|A_4(\lambda)(d)| &\leq 
C\left( (r^N + ||G|| + |\lambda|)|D||d| + |\lambda|^2 |d|  \right)
\end{align*}
\end{proof}
\end{lemma}

\subsection{Jump Conditions}
Up to this point, given $\lambda$ and $d$, we have found a unique solution equations \eqref{eigsystem1}, \eqref{eigsystem2}, \eqref{eigsystem3a}, and \eqref{eigsystem3b}, which is given by
\[
(a, b, w) = (A_3(\lambda)(d), B_1(\lambda)(d), W_3(\lambda)(d))
\]
Such a solution will have $n-1$ jumps in the direction of $T(\theta_i) Z_1(0)$, which are given by
\[
\xi_i = \langle T(\theta_i) Z_1(0), W_i^+(0) - W_i^-(0) \rangle
\]
In the next lemma, we derive formulas for these jumps.

% lemma : jump conditions
\begin{lemma}\label{jumpcond}
Assume Hypothesis \ref{BTcommutehyp}. Then the matching condition at 0 $W_i^+(0) = W_i^-(0)$ for $i = 1, \dots, m$ holds if and only if the $m$ jump conditions
\begin{equation}\label{xicond}
\xi_i = \langle T(\theta_i) Z_1(0), W_i^+(0) - W_i^-(0) \rangle = 0
\end{equation}
are satisfied. The jumps $\xi_i$ can be written as 
\begin{equation}\label{xieq}
\xi_i = \langle T(\theta_i) Z_1(N_i^+), P_0^u D_i d \rangle 
+ \langle T(\theta_i) Z_1(-N_{i-1}^-), P_0^s D_{i-1} d \rangle 
- \sum_{j = -\infty}^{\infty} \langle Z_1(j+1), B T_1(j)\rangle + R(\lambda)_i(d)
\end{equation}
where the remainder term $R(\lambda)(d)$ has bound
\begin{align}\label{xiRbound}
|R(\lambda)(d)| \leq C\left( (r^N + ||G|| + |\lambda|)( (r^N + ||G|| + |\lambda|)|D| + |\lambda|^2 \right)
\end{align}

\begin{proof}
From the previous lemma, the fixed point equations at $n = 0$ are given by 
\begin{align*}
W_i^-(0) &= b_i^- +
\Phi_s^-(0, -N_{i-1}^-; \theta_i) a_{i-1}^- \\
&+ \sum_{j = -N_{i-1}^-}^{-1} \Phi_s^-(0, j+1; \theta_i)
[(G_i^-(j) + \lambda B) W_i^-(j) + \lambda^2 d_i B \tilde{H}_i^-(j)] \\
W_i^+(0) &= b_i^+ + \Phi_u^+(0, N_i^+; \theta_i) a_i^+ \\
&- \sum_{j = 0}^{N_i^+-1} \Phi_u^+(0, j+1; \theta_i) 
[(G_i^+(j) + \lambda B) W_i^+(j) + \lambda^2 d_i B \tilde{H}_i^+(j)]
\end{align*}
When we take the inner product with $T(\theta_i)Z_1(0)$, the $b_i^\pm$ terms will vanish since they lie in spaces orthogonal to $T(\theta_i) Z_1(0)$. For the terms involving $a$, we substitute $A_4$ from Lemma \ref{eiginv3} to get
\begin{align*}
\langle T(\theta_i) &Z_1(0), \Phi_s^-(0, -N_{i-1}^-; \theta_i) a_{i-1}^- \rangle \\
&= -\langle T(\theta_i) Z_1(0), \Phi_s^-(0, -N_{i-1}^-; \theta_i) P_0^s D_{i-1} d \rangle + \langle Z_1(0), \Phi_s^-(0, -N_{i-1}^-; \theta_i) A_4(\lambda)_i(d) \rangle \\
&= -\langle \Phi_s^-(N_{i-1}^-, 0; \theta_i)^* T(\theta_i) Z_1(0), P_0^s D_{i-1} d \rangle + \mathcal{O}\left(r^N( (r^N + ||G|| + |\lambda|)|D| + |\lambda|^2 )|d| \right) \\
&= -\langle T(\theta_i) Z_1(-N_{i-1}^-), P_0^s D_{i-1} d \rangle + \mathcal{O}\left(r^N( (r^N + ||G|| + |\lambda|)|D| + |\lambda|^2 )|d| \right)
\end{align*}
Similarly,
\begin{align*}
\langle T(\theta_i) Z_1(0), \Phi_u^+(0, N_i^+; c_i) a_i^+ \rangle
&= \langle T(\theta_i) Z_1(N_i^+), P_0^u D_i d \rangle + \mathcal{O}\left(r^N( (r^N + ||G|| + |\lambda|)|D| + |\lambda|^2 )|d| \right)
\end{align*}

The sums involving $\tilde{H}$ will give us the Melnikov-like sum.
\begin{align*}
&\langle T(\theta_i) Z_1(0), \sum_{j = -N_{i-1}^-}^{-1} \Phi_s^-(0, j+1; \theta_i) B \tilde{H}_i^-(j) + \sum_{j = 0}^{N_i^+-1} \Phi_u^+(0, j+1; \theta_i) B \tilde{H}_i^+(j) \rangle \\
&= \langle T(\theta_i) Z_1(0), \sum_{j = -N_{i-1}^-}^{-1} \Phi_s^-(0, j+1; \theta_i) B T(\theta_i) T_1(j) + \sum_{j = 0}^{N_i^+-1} \Phi_u^+(0, j+1; \theta_i) B T(\theta_i) T_1(j) \rangle + \mathcal{O}(r^N) \\
&= \sum_{j = -N_{i-1}^-}^{-1} \langle T(\theta_i) Z_1(0), \Phi_s^-(0, j+1; \theta_i) B T(\theta_i) T_1(j)\rangle + \sum_{j = 0}^{N_i^+-1} \langle T(\theta_i) Z_1(0), \Phi_u^+(0, j+1; \theta_i) B T(\theta_i) T_1(j) \rangle + \mathcal{O}(r^N)\\
&= \sum_{j = -N_{i-1}^-}^{-1} \langle T(\theta_i) Z_1(j+1), B T(\theta_i) T_1(j) \rangle + \sum_{j = 0}^{N_i^+-1} \langle T(\theta_i) Z_1(j+1), B T(\theta_i) T_1(j) \rangle + \mathcal{O}(r^N)\\
&= \sum_{j = -N_{i-1}^-}^{N_i^+-1} \langle T(\theta_i)  Z_1(j+1), B T(\theta_i)  T_1(j) \rangle + \mathcal{O}(r^N)\\
&= \sum_{j = -\infty}^{\infty} \langle T(\theta_i) Z_1(j+1), B T(\theta_i) T_1(j)\rangle + \mathcal{O}(r^N)
\end{align*}
Using Hypothesis \ref{BTcommutehyp} and the fact that $T(\theta)$ is unitary, this becomes
\begin{align*}
&\langle T(\theta_i) Z_1(0), \sum_{j = -N_{i-1}^-}^{-1} \Phi_s^-(0, j+1; \theta_i) B \tilde{H}_i^-(j) + \sum_{j = 0}^{N_i^+-1} \Phi_u^+(0, j+1; \theta_i) B \tilde{H}_i^+(j) \rangle \\
&= \sum_{j = -\infty}^{\infty} \langle T(\theta_i) Z_1(j+1), T(\theta_i) B T_1(j)\rangle + \mathcal{O}(r^N) \\
&= \sum_{j = -\infty}^{\infty} \langle T(\theta_i)^{-1} T(\theta_i) Z_1(j+1), B T_1(j)\rangle + \mathcal{O}(r^N) \\
&= \sum_{j = -\infty}^{\infty} \langle Z_1(j+1), B T_1(j)\rangle + \mathcal{O}(r^N)
\end{align*}

Finally, we need to bound the term with sum involving $W$. For the ``minus'' sum, we have
\begin{align*}
\langle Z_1(0), &\sum_{j = -N_{i-1}^-}^{-1} \Phi_s^-(0, j+1; c_i)
(G_i^-(j) + \lambda B) W_i^-(j) \rangle = 
\sum_{j = -N_{i-1}^-}^{-1} c_i \langle c_i Z_1(0), \Phi_s^-(0, j+1; c_i)
(G_i^-(j) + \lambda B) W_i^-(j) \rangle \\
&= \sum_{j = -N_{i-1}^-}^{-1} \langle Z_1(j+1), 
(G_i^-(j) + \lambda B) W_i^-(j) \rangle 
\end{align*}
We will need an improved bound for $W$ (the problem is the $|D|$ term in the $W_3$ bound). From the fixed point equations, we have piecewise bounds
\begin{align*}
|W_i^-(n)| &\leq C( r^{N_{i-1}^- + n}|a_{i-1}| + |b_i^-| + 
(||G|| + |\lambda|)||W|| + |\lambda|^2|d|) \\
|W_i^-(n)| &\leq C( r^{N_i^+ - n}|a_i| + |b_i^+| + 
(||G|| + |\lambda|)||W|| + |\lambda|^2|d|) \\
\end{align*}
Plugging in the bounds for $A_3$, $W_3$, and $B_1$, these become
\begin{align*}
|W_i^-(n)| &\leq C( r^{N_{i-1}^- + n}|D| +  
(r^N + ||G|| + |\lambda|)|D| + |\lambda|^2 )|d| \\
|W_i^-(n)| &\leq C( r^{N_i^+ - n}|D| +  
(r^N + ||G|| + |\lambda|)|D| + |\lambda|^2 )|d|
\end{align*}
Recall that we have the bound $Z_1(n) \leq C r^{|n|}$. Since $A_\infty$ is hyperbolic, we can find a constant $\tilde{r} < r$ such that
\[
Z_1(n) \leq C \tilde{r}^n
\]
for this new $\tilde{r}$. The price to pay is that the constant $C$ will be larger, but that is okay. Thus we have
\begin{align*}
&\left| \sum_{j = -N_{i-1}^-}^{-1} \langle Z_1(j+1), 
(G_i^-(j) + \lambda B) W_i^-(j) \rangle \right| \\
&\leq C (||G|| + |\lambda|) \sum_{j = -N_{i-1}^-}^{-1} \tilde{r}^{|j+1|} r^{N_{i-1}^- + j}|D||d| + C (||G|| + |\lambda|)(r^N + ||G|| + |\lambda|)|D| + |\lambda|^2 )|d| \\
&\leq C |D| r^N (||G|| + |\lambda|)|d| \sum_{j = 1}^{N_{i-1}^-} \tilde{r}^{j+1} r^{-j} + C (||G|| + |\lambda|)(r^N + ||G|| + |\lambda|)|D| + |\lambda|^2 )|d| \\
&\leq C |D| r^N (||G|| + |\lambda|)|d| \sum_{j = 1}^\infty \left( \frac{\tilde{r}}{r}\right)^j + C (||G|| + |\lambda|)(r^N + ||G|| + |\lambda|)|D| + |\lambda|^2 )|d| \\
&\leq C (||G|| + |\lambda|)(r^N + ||G|| + |\lambda|)|D| + |\lambda|^2 )|d|
\end{align*}
where the infinite sum is convergent by the choice of $\tilde{r}$. We have a similar bound for the ``plus'' sum involving $W$. Putting this all together, we have
\begin{equation*}
\xi_i = \langle T(\theta_i) Z_1(N_i^+), P_0^u D_i d \rangle 
+ \langle T(\theta_i) Z_1(-N_{i-1}^-), P_0^s D_{i-1} d \rangle 
- \sum_{j = -\infty}^{\infty} \langle T(\theta_i) Z_1(j+1), B T(\theta_i) T_1(j)\rangle + R(\lambda)_i(d)
\end{equation*}
where the remainder term $R(\lambda)(d)$ has bound
\begin{align*}
|R(\lambda)(d)| \leq C\left( (r^N + ||G|| + |\lambda|)( (r^N + ||G|| + |\lambda|)|D| + |\lambda|^2 \right)
\end{align*}

\end{proof}
\end{lemma}

\subsubsection{Substitution}

In the final lemma, we substitute the expression for $D_i d$ into the jump conditions.

\begin{lemma}\label{jumpcond2}
Assume Hypothesis \ref{BTcommutehyp}. Then the jumps $\xi_i$ from Lemma \ref{jumpcond} can be written as
\begin{equation}\label{jumpxi2}
\xi_i = \langle T(\theta_i) Z_1(N_i^+), T(\theta_{i+1}) S_1(-N_i^-) \rangle (d_{i+1} - d_i)
+ \langle T(\theta_i) Z_1(-N_{i-1}^-), T(\theta_{i-1}) S_1(N_{i-1}^+) \rangle (d_i - d_{i-1})
- \sum_{j = -\infty}^{\infty} \langle Z_1(j+1), B T_1(j)\rangle + R(\lambda)_i(d) \\
\end{equation}
for $i = 1, \dots, m$, where we have remainder bound
\begin{align*}
|R(\lambda)(d)| \leq C\left( (r^N + |\lambda|)^3 \right)
\end{align*}

\begin{proof}
Recall that $D_i d$ is given by
\begin{align*}
D_i d &= [ T(\theta_{i+1}) S_1(-N_i^-) + \tilde{S}_{i+1}^-(-N_i^-)] d_{i+1}
- [ T(\theta_i) S_1(N_i^+) + \tilde{S}_i^+(N_i^+)] d_i \\
&+ \lambda[ T(\theta_{i+1}) T_1(-N_i^-) + \tilde{T}_{i+1}^-(-N_i^-)] d_{i+1}
- \lambda[ T(\theta_i) T_1(N_i^+) + \tilde{T}_i^+(N_i^+)] d_i 
\end{align*}
which simplifies to
\begin{align}\label{Did3}
D_i d &= [ T(\theta_{i+1}) S_1(-N_i^-) + \tilde{S}_{i+1}^-(-N_i^-)] d_{i+1}
- [ T(\theta_i) S_1(N_i^+) + \tilde{S}_i^+(N_i^+)] d_i + \mathcal{O}(|\lambda|r^{-N}|d|)
\end{align}
Using the estimates \eqref{Westimates}
\begin{align*}
\tilde{Q}_{i+1}^-(-N_i^-) &= T(\theta_i) Q(N_i^+) + \mathcal{O}(r^{-2N}) \\
\end{align*}
Recalling that $S_1(n) = S Q$, where $S$ is the infinitesimal generator of the symmetry group $T(\theta)$ (and is just a constant matrix) and $S_m(n) = S Q_m(n)$, this implies
\begin{align*}
\tilde{S}_{i+1}^-(-N_i^-) &= T(\theta_i) S_1(N_i^+) + \mathcal{O}(r^{-2N}) \\
\end{align*}
since the infinitesimal generator of a group commutes with the group elements. Similarly,
\begin{align*}
\tilde{S}_i^+(N_i^+) &= T(\theta_{i+1}) S_1(-N_i^-) + \mathcal{O}(r^{-2N}) \\
\end{align*}
Substituting these into \eqref{Did3}, we get
\begin{equation}\label{Did4}
D_i d = [ T(\theta_{i+1}) S_1(-N_i^-) + T(\theta_i) S_1(N_i^+) ] d_{i+1}
- [ T(\theta_i) S_1(N_i^+) + T(\theta_{i+1}) S_1(-N_i^-) ] d_i 
+\mathcal{O}(r^N( |\lambda| + r^N))
\end{equation}

We substitute this into the expression for the jump $\xi_i$ from Lemma \ref{jumpcond}. For the term $\langle T(\theta_i) Z_1(N_i^+), P_0^u D_i d \rangle$, we will compute the two inner product terms. Using the exponential dichotomy from Lemma \ref{dichotomy},
\begin{align*}
\langle T(\theta_i) Z_1(N_i^+), &P_0^u T(\theta_{i+1}) S_1(-N_i^-) \rangle
= \langle T(\theta_i) Z_1(N_i^+), P_-^u(-N_i^-; \theta_{i+1}) T(\theta_{i+1}) S_1(-N_i^-) \rangle + \mathcal{O}(r^{3N}) \\
&= \langle T(\theta_i) Z_1(N_i^+), T(\theta_{i+1}) S_1(-N_i^-) \rangle + \mathcal{O}(r^{3N}) 
\end{align*}
and
\begin{align*}
\langle T(\theta_i) Z_1(N_i^+), &P_0^u T(\theta_i) S_1(N_i^+) \rangle
= \langle T(\theta_i) Z_1(N_i^+), P_+^u(N_i^+; \theta_i) T(\theta_i) S_1(N_i^+) \rangle + \mathcal{O}(r^{3N}) \\
&= \langle T(\theta_i) Z_1(N_i^+), T(\theta_i) S_1(N_i^+) \rangle + \mathcal{O}(r^{3N}) \\
&= \langle T(-\theta_i) T(\theta_i) Z_1(N_i^+), S_1(N_i^+) \rangle + \mathcal{O}(r^{3N}) \\
&= \langle Z_1(N_i^+), S_1(N_i^+) \rangle + \mathcal{O}(r^{3N}) \\
&= \mathcal{O}(r^{-3N})
\end{align*}
since $T(\theta)$ is unitary and $\langle Z_1(n), S_1(n) \rangle = 0$ for all $n$. Substituting these expressions into $\langle T(\theta_i) Z_1(N_i^+), P_0^u D_i d \rangle$, we have
\begin{equation*}
\langle T(\theta_i) Z_1(N_i^+), P_0^u D_i d \rangle 
= \langle T(\theta_i) Z_1(N_i^+), T(\theta_{i+1}) S_1(-N_i^-) \rangle (d_{i+1} - d_i)
+ \mathcal{O}(r^{-3N})
\end{equation*}

Similarly, for $\langle T(\theta_i) Z_1(-N_{i-1}^-), P_0^s D_{i-1} d \rangle $, we have
\begin{align*}
\langle T(\theta_i) Z_1(-N_{i-1}^-), P_0^s D_{i-1} d \rangle 
&= \langle T(\theta_i) Z_1(-N_{i-1}^-), T(\theta_{i-1}) S_1(N_{i-1}^+) \rangle (d_i - d_{i-1})
\end{align*}
Substituting these into the jump expressions from Lemma \ref{jumpcond}, we have the final jump condition equations
\begin{equation*}
\xi_i = \langle T(\theta_i) Z_1(N_i^+), T(\theta_{i+1}) S_1(-N_i^-) \rangle (d_{i+1} - d_i)
+ \langle T(\theta_i) Z_1(-N_{i-1}^-), T(\theta_{i-1}) S_1(N_{i-1}^+) \rangle (d_i - d_{i-1})
- \sum_{j = -\infty}^{\infty} \langle T(\theta_i) Z_1(j+1), B T(\theta_i) T_1(j)\rangle + R(\lambda)_i(d)
\end{equation*}

For the remainder term, we substitute $|D|, ||G|| = \mathcal{O}(r^N)$. Since the remainder terms from the substitutions above are order $\mathcal{O}(r^{-3N})$, they are already included in the remainder term from Lemma \ref{jumpcond}. Thus we have remainder bound
\begin{align*}
|R(\lambda)(d)| \leq C\left( (r^N + |\lambda|)^3 \right)
\end{align*}
\end{proof}
\end{lemma}

\end{document}