\documentclass[12pt]{article}
\usepackage[pdfborder={0 0 0.5 [3 2]}]{hyperref}%
\usepackage[left=1in,right=1in,top=1in,bottom=1in]{geometry}%
\usepackage[shortalphabetic]{amsrefs}%
\usepackage{amsmath}
\usepackage{enumerate}
% \usepackage{enumitem}
\usepackage{amssymb}                
\usepackage{amsmath}                
\usepackage{amsfonts}
\usepackage{amsthm}
\usepackage{bbm}
\usepackage[table,xcdraw]{xcolor}
\usepackage{tikz}
\usepackage{float}
\usepackage{booktabs}
\usepackage{svg}
\usepackage{mathtools}
\usepackage{cool}
\usepackage{url}
\usepackage{graphicx,epsfig}
\usepackage{makecell}
\usepackage{array}

\def\noi{\noindent}
\def\T{{\mathbb T}}
\def\R{{\mathbb R}}
\def\N{{\mathbb N}}
\def\C{{\mathbb C}}
\def\Z{{\mathbb Z}}
\def\P{{\mathbb P}}
\def\E{{\mathbb E}}
\def\Q{\mathbb{Q}}
\def\ind{{\mathbb I}}

\DeclareMathOperator{\spn}{span}
\DeclareMathOperator{\ran}{range}

\graphicspath{ {periodic/} }

\newtheorem{lemma}{Lemma}
\newtheorem{theorem}{Theorem}
\newtheorem{corollary}{Corollary}
\newtheorem{definition}{Definition}
\newtheorem{assumption}{Assumption}
\newtheorem{proposition}{Proposition}
\newtheorem{hypothesis}{Hypothesis}

\newtheorem{notation}{Notation}

\begin{document}

\section{Existence Problem for 2-pulse}

\subsection{Bifurcation Equation}

Recall that from Lin's method, the (single) jump condition is given by

\begin{align}\label{jumpcond1}
s_0 e^{-2 \alpha X_0} \sin(2 \beta X_0 + \phi) - s_0 e^{-2 \alpha X_1} \sin(2 \beta X_1 + \phi) + \mathcal{O}(e^{-(2 \alpha + \gamma) X_m}) &= 0
\end{align}

where $s_0$ is a constant, and $X_m = \min\{X_0,X_1\}$. Next, we make the substitution

\begin{align}
r &= e^{-\alpha(2 X_m + \phi/\beta)} \\
b_j &= e^{-2 \alpha(X_j - X_m)} && j = 0, 1
\end{align}

so that

\begin{align*}
b_j r &= e^{-2 \alpha X_j}e^{-2\alpha \phi/\beta} \\
X_m &= -\frac{1}{2\alpha}\log r - \alpha \frac{\phi}{\beta} \\
X_j &= -\frac{1}{2\alpha}\log(b_j r) - \frac{\phi}{2 \beta} 
\end{align*}

With this substitution, the jump condition becomes

\begin{align*}
s_0 b_0 r e^{-2\alpha \phi/\beta} \sin(-\rho \log(b_0 r)) - s_0 b_1 r e^{-2\alpha \phi/\beta} \sin(-\rho \log(b_1 r)) + \mathcal{O}(r^{1 + \gamma / 2 \alpha}) &= 0
\end{align*}

where $\rho = \beta / \alpha > 0$. Divide by $r$ and the constants out front to get

\begin{align}\label{jumpcond2}
b_0 \sin(-\rho \log(b_0 r)) - b_1 \sin(-\rho \log(b_1 r)) + \mathcal{O}(r^{\gamma / 2 \alpha}) &= 0
\end{align}

Define the space

\begin{equation}\label{setR}
\mathcal{R} = \left\{ \exp\left(-\frac{m \pi}{\rho}\right) : m \in \N_0 \right\} \cup \{ 0 \}
\end{equation}

Since $\mathcal{R}$ is closed and bounded, is is compact, thus complete. In what follows, we will take $r \in \mathcal{R}$ with $r > 0$. Thus the $r$ inside the log disappears (which is really nice!), leaving us with the jump equation

\begin{align}\label{jumpcond3}
G(b_0, b_1,r) =
b_0 \sin(-\rho \log b_0 ) - b_1 \sin(-\rho \log b_1 ) + \mathcal{O}(r^{\gamma / 2 \alpha}) &= 0
\end{align}

We note that the only dependence on $r$ is in the remainder term.

\subsection{Parameterization}

Before we solve the jump equation, we introduce the following parameterization. This parameterization is motivated both by geometry and by the result we eventually want to obtain. For our periodic 2-pulse, we will use the following three parameters.

\begin{enumerate}[(i)]
\item A scaling parameter $r = \exp(-m \pi / \rho ) \in \mathcal{R}$, where $m$ is a positive integer.
\item Two length parameters $b_0^0$, $b_0^1$, where $b_j^0 = \exp(-m_j \pi / \rho )$ and $m_0, m_1$ are nonnegative integers, at least one of which must be 0. (This restriction is necessary to get uniqueness).
\item A phase parameter $\theta \in [-\arctan \rho, \pi - \arctan \rho)$.
\end{enumerate}

For convenience, let $F(b_0, b_1) = G(b_0, b_1, 0)$. We first show that this parameterization gives rise to a family of unique solutions to $F(b_0, b_1) = 0$. If that is successful, we will then show that this persists for small $r$.\\

First, we look for solutions to $F(b_0, b_1)$. We start with the following lemma.

\begin{lemma}\label{pitchforkF}
A countable family of pitchfork bifurcations in the zero set of $F$ occurs along the diagonal at 

\begin{align*}
(b_0, b_1) &= (p_n^*, p_n^*) && n \in \Z
\end{align*}

where 

\begin{equation}
p^*_n = \exp\left(-\frac{n \pi}{\rho} \right) p^*
\end{equation}

and 

\begin{equation}
p^* = \exp \left( -\frac{1}{\rho} \arctan \rho \right)
\end{equation}

Locally, the arms of the pitchfork bifurcations open upwards along the diagonal.
\begin{proof}
This has been proved. Roughly, we change variables so that the proposed pitchfork occurs along the $x$-axis. We then verify the symmetry and derivative criteria for a pitchfork bifurcation.
\end{proof}
\end{lemma}

Next, we note that we have following two classes of solutions to $F(b_0,b_1) = 0$. 

\begin{enumerate}
	\item $b_0 = b_1$. This corresponds to the middle of the pitchfork.
	\item $b_0 \neq b_1$ but $b_j = b_j^0 = \exp(-m_j \pi / \rho )$. Recall that one of the $b_j^0$ must be 1. These solutions are on the arms of the pitchfork.
\end{enumerate}

We will now use the phase $\theta$ to parameterize all (relevant) solutions to $F(b_0, b_1) = 0$. To state the lemma, we have to make a choice as to which $b_i^0$ will be 1. We arbitrarily choose $b_0^0 = 1$ and note that we if we swap $b_0$ and $b_1$, we get the same periodic solution since we are on a periodic domain.

\begin{lemma}\label{thetaparam}\[\]
\begin{itemize}
\item For $j = 0, 1$, let $b_j^0 = \exp(-m_j \pi / \rho )$, where $m_1$ is a nonnegative integer and $m_0 = 0$. 
\item Then for $\theta \in [-\arctan \rho,\pi - \arctan \rho)$, there is a smooth family of solutions $( b_0(m_0, \theta), b_1(m_1, \theta) )$ to $F(b_0, b_1) = 0$, where $(b_0(m_0, 0), b_1(m_1, 0)) = (b_0^0, b_1^0) = (1, b_1^0)$. 
\item Furthermore, these smooth families all ``connect up'', i.e.
\[
\Big( b_0(0, -\arctan \rho), b_1(j, -\arctan \rho) \Big) = \Big( b_0(0, \pi - \arctan \rho), b_1(j+1, \pi - \arctan \rho) \Big)
\]
\item We also have,
\[
\Big( b_0(0, -\arctan \rho), b_1(0, -\arctan \rho) \Big) = \Big( b_0(0, \pi - \arctan \rho), b_1(1, \pi - \arctan \rho) \Big) = (p^*, p^*)
\] 
where $(p^*, p^*)$ is the pitchfork bifurcation point from Lemma \ref{pitchforkF}
\item Finally, the parameterization is given explicitly by
\begin{align*}
b_1(\theta) &= e^{ -\frac{1}{\rho}(m_1 \pi - \theta) } \\
b_0(\theta) &= e^{-\frac{1}{\rho} \theta^*(\theta) }
\end{align*}
where $\theta^*$ depends on $\theta$, $\theta^* \in [-\arctan \rho,\pi - \arctan \rho)$, and we have bound
\[
|\theta^*| &\leq C e^{ -\frac{\pi}{\rho} m_1 }
\]
\end{itemize}

\begin{proof}
For $\theta \in [-\arctan \rho,\pi - \arctan \rho)$, let
\begin{align*}
b_1(\theta) &= e^{ -\frac{1}{\rho}(m_1 \pi - \theta) } \\
\end{align*}
We want to find an expression for $b_1(\theta)$ that gives us our desired parameterization. Since we are parameterizing two pieces of a pitchfork, we will have different expressions for the cases $b_1^0 = 1$ and $b_1^0 \neq 1$. 
\begin{enumerate}
	\item Let $b_1^0 = 1$ as well, so that $b_1(\theta) = e^{ \frac{1}{\rho}\theta }$. Then we can take $b_0(\theta) = e^{ \frac{1}{\rho}\theta }$ as well to get our solution. Thus we have have the parameterization
	\[
	( b_0(\theta), b_1(\theta) ) = ( e^{ \frac{1}{\rho}\theta }, e^{ \frac{1}{\rho}\theta })
	\]
	which parameterizes the center part of the pitchfork where $b_0 = b_1$. We see easily that $(b_0(0), b_1(0)) = (b_0^0, b_1^0) = (1,1)$. For future use, we note that
	\[
	(b_0(-\arctan \rho), b_1(-\arctan \rho)) = (p^*, p^*)
	\]
	where $p^*$ is defined in Lemma \ref{pitchforkF}. From that lemma, $(p^*, p^*)$ is the bifurcation point.

	\item Now let $b_1^0 \neq 1$ so that $m_1 > 0$. Let
	\begin{align*}
	b_0(\theta^*) &= e^{-\frac{1}{\rho} \theta^* } \\
	\end{align*}
	We will solve for $\theta^*$ in terms of $\theta$. We also need to make sure we also have $\theta^* \in [-\arctan \rho,\pi - \arctan \rho)$, otherwise the value of $b_0^0$ will change. Plugging in the expressions for $b_0$ and $b_1$ into $F$, we get
	\begin{align*}
	F(b_0, b_1) &= e^{ -\frac{1}{\rho}\theta^* } \sin\left( -\rho \log e^{ -\frac{1}{\rho}\theta^* }\right) - e^{ -\frac{1}{\rho}(m_1 \pi - \theta) }\sin \left( -\rho \log e^{ -\frac{1}{\rho}(m_1 \pi + \theta) } \right) \\
	&= e^{ -\frac{1}{\rho}\theta^* } \sin\left( \theta^* \right) - e^{ -\frac{1}{\rho} m_1 \pi} e^{ \frac{1}{\rho} \theta } \sin(m_1 \pi - \theta) \\
	&= e^{ -\frac{1}{\rho}\theta^* } \sin\left( \theta^* \right) - e^{ -\frac{1}{\rho} m_1 \pi } e^{ \frac{1}{\rho} \theta } (-1)^{m_1 + 1} \sin(\theta)
	\end{align*}
	Thus we need to solve the following equation for $\theta^*$.
	\begin{align}\label{thetastareq}
	e^{ -\frac{1}{\rho}\theta^* } \sin\left( \theta^* \right) &= \left[ e^{ -\frac{1}{\rho} m_1 \pi } (-1)^{m_1 + 1} \right] e^{ \frac{1}{\rho} \theta } \sin(\theta)
	\end{align}
	for $\theta \in [-\arctan \rho,\pi - \arctan \rho)$. We proceed in the following steps.
	\begin{enumerate}
		\item Let $g(\theta) = e^{ \frac{1}{\rho} \theta } \sin(\theta)$. First, we show that $g(\theta)$ is increasing on $[-\arctan \rho,\pi - \arctan \rho]$. The derivative is 
		\[
		g'(\theta) = e^{ \frac{1}{\rho} \theta } \left( \cos(\theta) + \frac{1}{\rho} \sin(\theta)\right)
		\]
		At the left endpoint,
		\begin{align*}
		g'(-\arctan \rho) &= e^{ -\frac{1}{\rho} \arctan \rho } \left(\cos(-\arctan \rho) + \frac{1}{\rho} \sin(-\arctan \rho)\right) \\
		&= e^{ -\frac{1}{\rho} \arctan \rho } \left(\cos\frac{1}{\sqrt{1 + \rho^2}} - \frac{1}{\rho} \frac{\rho}{\sqrt{1 + \rho^2}}\right) = 0
		\end{align*}
		The only critical point of $g'(\theta)$ on $[-\arctan \rho,\pi - \arctan \rho]$ is a local maximum at $\theta = \arctan \rho < \pi/2$, thus $g'(\theta) > 0$ on $(-\arctan \rho,\pi - \arctan \rho)$ and is zero at the endpoints, from which we conclude that $g(\theta)$ is increasing on $(-\arctan \rho,\pi - \arctan \rho)$.
		
		\item From this, we conclude that 
		\[
		g(\theta) \in \left[ -\frac{\rho}{\sqrt{1+\rho^2}}e^{-\frac{1}{\rho}\arctan \rho}, \frac{\rho}{\sqrt{1+\rho^2}}e^{\frac{1}{\rho}(\pi - \arctan \rho)}\right] = [-T, e^{\frac{1}{\rho}\pi} T]
		\]
		where 
		\[
		T = \frac{\rho}{\sqrt{1+\rho^2}}e^{-\frac{1}{\rho}\arctan \rho}
		\]
		It follows that for the RHS of \eqref{thetastareq} we have

		\begin{equation}\label{RHSbounds}
		\left( e^{ -\frac{1}{\rho} m_1 \pi } (-1)^{m_1 + 1} \right) e^{ \frac{1}{\rho} \theta } \sin(\theta) \in
		\begin{cases}
		[-e^{-\frac{1}{\rho}m_1 \pi} T, e^{-\frac{1}{\rho}(m_1 - 1) \pi} T] & m_1 \text{ odd}\\
		[-e^{-\frac{1}{\rho}(m_1 - 1) \pi} T, e^{-\frac{1}{\rho}m_1 \pi} T] & m_1 \text{ even}
		\end{cases}
		\end{equation}

		\item Let $h(\theta^*) = e^{ -\frac{1}{\rho}\theta^* }\sin{\theta^*}$ be the LHS of \eqref{thetastareq}. We can similarly show that $h(\theta^*)$ has a local minimum at $\theta^* = \arctan{\rho} - \pi$, a local maximum at $\theta^* = \arctan{\rho}$, and no critical points between these. We also have $h(0) = 0$. We conclude that $h(\theta^*)$ is strictly increasing and thus invertible on $[\arctan{\rho} - \pi, \arctan{\rho}]$. Plugging the endpoints into $h$, $h^{-1}$ is defined and increasing on 
		\[
		\left[ -\frac{\rho}{\sqrt{1+\rho^2}}e^{\frac{1}{\rho}\pi}e^{-\frac{1}{\rho}\arctan \rho}, \frac{\rho}{\sqrt{1+\rho^2}}e^{-\frac{1}{\rho}\arctan \rho}\right] = [-e^{\frac{1}{\rho}\pi} T , T]
		\] 
		with $h^{-1}(-e^{\frac{1}{\rho}\pi} T) = \arctan{\rho} - \pi$ and $h^{-1}(T) = \arctan \rho$. 

		\item From the previous two parts, we conclude that since $m_1 \geq 1$, we can invert $h$ to solve for $\theta^*$. Thus we have

		\[
		\theta^* = h^{-1}\left( e^{ -\frac{1}{\rho} m_1 \pi } (-1)^{m_1 + 1}  e^{ \frac{1}{\rho} \theta } \sin(\theta) \right)
		\]
		for $\theta \in [-\arctan \rho,\pi - \arctan \rho)$. We can then plug this into the equation for $b_0(\theta^*)$ to solve for $b_0$ in terms of $\theta$.\\

		From this, we have the simple bound $\theta^* \in (-\arctan \rho, \pi - \arctan \rho)$, which at least ensures that varying $\theta$ does not cause the integers $m_i$ to change.

		\item We now have a piecewise parameterization which depends on $b_1^0$. We will now show that the pieces all match up. First, we look at the matching not at the pitchfork bifurcation point. Let $m_1 \geq 1$, $b_1^0 = \exp(-m_1 \pi \ \rho )$ and $\tilde{b}_1^0 = \exp(-(m_1+1) \pi \ \rho )$. Then we have

		\begin{align*}
		b_1(-\arctan \rho) &= e^{ -\frac{1}{\rho}(m_1 \pi + \arctan \rho) } \\
		\tilde{b}_1(\pi - \arctan \rho) 
		&= e^{ -\frac{1}{\rho}((m_1+1) \pi - (\pi - \arctan \rho)) } \\
		&= e^{ -\frac{1}{\rho}(m_1 \pi + \arctan \rho)) } = b_1(-\arctan \rho)
		\end{align*}

		These are the same, so the $b_1$ component matches. For the $b_0$ component, we note that we obtain $\theta^*$ and $\tilde{\theta}^*$ by solving

		\begin{align*}
		\theta^* &= h^{-1} (b_1(-\arctan \rho)) \\
		\tilde{\theta}^* &= h^{-1} (\tilde{b}_1(\pi - \arctan \rho)) 
		\end{align*}

		Since the RHS of the two equations is the same, we conclude that $\theta^* = \tilde{\theta}^*$, thus $b_0(-\arctan \rho) = \tilde{b}_0(\pi - \arctan \rho)$, and so the $b_0$ components match.

		\item Next, we show that for $m_1 = 1$, we hit the pitchfork bifurcation point at $\theta = \pi - \arctan \rho$. For the $b_1$ component, 
		\begin{align*}
		b_1(\pi -\arctan \rho) &= e^{ -\frac{1}{\rho}(\pi - (\pi - \arctan \rho)) } \\
		&= e^{ -\frac{1}{\rho} \arctan \rho } = p^*
		\end{align*}

		For $b_0$, we note that for $\theta = \pi -\arctan \rho$, the RHS of \eqref{thetastareq} is exactly $L$, thus $\theta^* = h^{-1}(L) = \arctan \rho$, and so $b_0(\pi -\arctan \rho) = e^{ -\frac{1}{\rho} \arctan \rho } = p^*$ as well.

		\item Finally, we get some bounds on $\theta^*$. These will be in terms of $m_1$. From \eqref\label{RHSbounds} we have
		\[
		|e^{ -\frac{1}{\rho} m_1 \pi } (-1)^{m_1 + 1}  e^{ \frac{1}{\rho} \theta } \sin(\theta)| \leq C e^{ -\frac{1}{\rho}(m_1 - 1) \pi }
		\]
		Since $h(0) = 0$, we have $\theta^* \rightarrow 0$ as $m_1 \rightarrow \infty$. For a decay rate, recall that $h'(\theta^*) = 0$  $\theta^* = \arctan{\rho} - \pi$ and $\theta^* = \arctan{\rho}$, and no critical points between these. We also have $h(0) = 0$. For the derivative $h'(\theta^*$, we have
		\[
		h'(\theta^*) = \frac{1}{\rho}e^{\frac{1}{\rho}\theta^*}(\rho \cos \theta^* - \sin \theta^*)
		\]
		Since $h'(0) = 1$, we can find an interval $[-R_0,R_0]$ on which $h'(\theta^*) \geq 1/2$. Thus we can find an interval $[-R_1, R_1]$ on which $h^{-1}'(x) \leq L$ for a constant $L > 0$. For sufficiently large $m_1$, we will always have $\theta^* \in [-R_1, R_1]$, and $h^{-1}$ is Lipschitz with constant $L$ on that interval. Thus we conclude that
		\begin{align*}
		|\theta^*| &= | h^{-1}(e^{ -\frac{1}{\rho} m_1 \pi } (-1)^{m_1 + 1}  e^{ \frac{1}{\rho} \theta } \sin(\theta)) | \\
		&\leq L |e^{ -\frac{1}{\rho} m_1 \pi } (-1)^{m_1 + 1}  e^{ \frac{1}{\rho} \theta } \sin(\theta)| \\
		&\leq C e^{ -\frac{1}{\rho}(m_1 - 1) \pi } \\
		&\leq C e^{ -\frac{\pi}{\rho} m_1 }
		\end{align*}

	\end{enumerate}
\end{enumerate}

\end{proof}
\end{lemma}

Here is an attempt to draw the parameterization.

\begin{figure}[H]
\begin{center}
\includegraphics[width=18cm]{param1}
\end{center}
\end{figure}


\end{document}