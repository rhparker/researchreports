\documentclass[12pt]{article}
\usepackage[pdfborder={0 0 0.5 [3 2]}]{hyperref}%
\usepackage[left=1in,right=1in,top=1in,bottom=1in]{geometry}%
\usepackage[shortalphabetic]{amsrefs}%
\usepackage{amsmath}
\usepackage{enumerate}
% \usepackage{enumitem}
\usepackage{amssymb}                
\usepackage{amsmath}                
\usepackage{amsfonts}
\usepackage{amsthm}
\usepackage{bbm}
\usepackage[table,xcdraw]{xcolor}
\usepackage{tikz}
\usepackage{float}
\usepackage{booktabs}
\usepackage{svg}
\usepackage{mathtools}
\usepackage{cool}
\usepackage{url}
\usepackage{graphicx,epsfig}
\usepackage{makecell}
\usepackage{array}

\def\noi{\noindent}
\def\T{{\mathbb T}}
\def\R{{\mathbb R}}
\def\N{{\mathbb N}}
\def\C{{\mathbb C}}
\def\Z{{\mathbb Z}}
\def\P{{\mathbb P}}
\def\E{{\mathbb E}}
\def\Q{\mathbb{Q}}
\def\ind{{\mathbb I}}

\graphicspath{ {images17/} }

\newtheorem{lemma}{Lemma}
\newtheorem{corollary}{Corollary}
\newtheorem{definition}{Definition}
\newtheorem{assumption}{Assumption}
\newtheorem{hypothesis}{Hypothesis}

\begin{document}

\section*{Double Pulse}

Now we repeat the above in the case of a periodic double pulse $q_{2p}$. Here the pulse separation is $L$ and the period is $T$. The system of equations we want to solve is

\begin{enumerate}[(i)]
\item $(W_i^\pm)' = A(q; \lambda) W_i^\pm + G_i(\lambda)^\pm W_i^\pm + \lambda^2 d_i \tilde{H}_i^\pm$
\item $W_i^\pm(0) \in \C \psi(0) \oplus Y^+ \oplus Y^- \oplus Y^0$
\item $W_i^+(0) - W_i^-(0) \in \C \psi(0) $
\item $W_1^+(L) - W_2^-(-L) = D_1 d $
\item $W_2^+(T) - W_1^-(-T) = 0$
\end{enumerate}

where

\begin{align*}
G_i(\lambda)^\pm &= (A(q_{2p};\lambda) - A(q_2;\lambda)) + (A(q_{2};\lambda) - A(q;\lambda)) \\
D_1 d &= d_2(Q_{2p}'(-L) + \lambda (Q_{2p})_c(-L)) 
- d_1 ( Q_{2p}'(L) + \lambda (Q_{2p})_c(L) )
\tilde{H} &= -B(Q_{2p})_c \\
H &= -B Q_c \\
\Delta H &= \tilde{H} - H
\end{align*}

Again we do not have a $\lambda B W_i^\pm$ term since this has been absorbed into $A(q; \lambda) W_i^\pm$. 

For these we should have bounds

\begin{align*}
G_i(\lambda) &= \mathcal{O}(e^{-\alpha T} + e^{-\alpha L}) \\
\Delta H &= \mathcal{O}(e^{-\alpha T} + e^{-\alpha L}) \\
D_1 &= ( Q'(L) + Q'(-L) )(d_2 - d_1 ) + \mathcal{O} \left( e^{-\alpha L} \left( |\lambda| +  e^{-\alpha L}  \right) |d| \right)
\end{align*}

For the double pulse, we have

\[
X = (X_0, X_1, X_2) = (T, L, T)
\]

Then the fixed point equations are 

\begin{align*}
W_i^-(x) = \Phi^s_-(&x, -X_{i-1}; \lambda)a_{i-1}^- + \Phi^u_-(x, 0; \lambda)b_i^- + e^{\nu(\lambda)(x+X_{i-1})} v_-(x; \lambda) \langle v_0(\lambda), w_-(-X_{i-1}; \lambda) \rangle c_{i-1}^- \\
&+ \int_0^x \Phi^u_-(x, y; \lambda)[ G_i^-(\lambda)W_i^-(y) + \lambda^2 d_i \tilde{H}(y) ] dy \\
&+ \int_{-X_{i-1}}^x \Phi^s_-(x, y; \lambda) [ G_i^-(\lambda)W_i^-(y) + \lambda^2 d_i \tilde{H}(y) ] dy \\
&+ \int_{-X_{i-1}}^x 
e^{\nu(\lambda)(x-y)} v_-(x; \lambda) \langle G_i^-(\lambda)(y)W_i^-(y) + \lambda^2 d_i \tilde{H}(y), w_-(y; \lambda) \rangle dy \\
W_i^+(x) = \Phi^u_+(&x, X_i; \lambda)a_i^+ + \Phi^s_+(x, 0; \lambda)b_i^+ + e^{\nu(\lambda)(x - X_i)} v_+(x; \lambda) \langle v_0(\lambda), w_+(X_i; \lambda) \rangle c_i^+ \\
&+ \int_0^x \Phi^s_+(x, y; \lambda) [ G_i^+(\lambda)W_i^+(y) + \lambda^2 d_i \tilde{H}(y) ] dy \\
&+ \int_{X_i}^x \Phi^u_+(x, y; \lambda) [ G_i^+(\lambda)W_i^+(y) + \lambda^2 d_i \tilde{H}(y) ] dy \\
&+ \int_{X_i}^x e^{\nu(\lambda)(x-y)} v_+(x; \lambda) \langle G_i^+(\lambda)(y)W_i^+(y) + \lambda^2 d_i \tilde{H}(y), w_+(y; \lambda) \rangle dy
\end{align*}

So now we do the same thing we did with the single pulse. The main difference here is the presence of the $d_i$ in the $\tilde{H}$ terms. 

\begin{enumerate}

\item Solve for $W_i$ in terms of the other stuff. The estimates should be the same as the single pulse, with the addition of the $d$ term, so this should be

\[
||W_1(\lambda)(a,b,c,d)|| \leq C (|a| + |b| + e^{\nu(\lambda)T}(|c| + |\lambda|^2 |d| ))
\]

\item Solve for the joins which are not at 0, i.e. solve

\begin{align*}
W_2^+(T) - W_1^-(-T) &= 0 \\
W_1^+(L) - W_2^-(-L) &= D_1 d \\
\end{align*}

To solve these, we have two equations. The first one is the periodic matching, which we have in the single pulse case.

\begin{align*}
0 &= a_2^+ - a_0^- \\
&+ (P^u_+(T; \lambda) - P_0^u)a_2^+ - (P^s_-(-T; \lambda) - P_0^s)a_0^- \\
&+ \Phi^s_+(T, 0; \lambda)b_2^+ - \Phi^u_-(-T, 0; \lambda)b_1^- \\
&+ v_+(T; \lambda) \langle v_0(\lambda), w_+(T; \lambda) \rangle c_2^+ - v_-(-T; \lambda) \langle v_0(\lambda), w_-(-T; \lambda) \rangle c_0^- \\
&+ \int_0^T \Phi^u_+(T, y; \lambda) [ G(\lambda)W^-(y) + d_2 \lambda^2 \tilde{H}(y) ] dy \\
&- \int_0^{-T} \Phi^s_-(-T, y; \lambda) [ G(\lambda)W^+(y) + d_1 \lambda^2 \tilde{H}(y) ] dy
\end{align*}

The second one is the center matching at $\pm L$.

\begin{align*}
D_1 d &= a_1^+ - a_1^- \\
&+ (P^u_+(L; \lambda) - P_0^u)a_1^+ - (P^s_-(-L; \lambda) - P_0^s)a_1^- \\
&+ \Phi^s_+(L, 0; \lambda)b_1^+ - \Phi^u_-(-L, 0; \lambda)b_2^- \\
&+ v_+(L; \lambda) \langle v_0(\lambda), w_+(L; \lambda) \rangle c_1^+ - v_-(-L; \lambda) \langle v_0(\lambda), w_-(-L; \lambda) \rangle c_1^- \\
&+ \int_0^{L} \Phi^u_+(L, y; \lambda) [ G(\lambda)W^-(y) + d_1 \lambda^2 \tilde{H}(y) ] dy \\
&- \int_0^{-L} \Phi^s_-(-L, y; \lambda) [ G(\lambda)W^+(y) + d_2 \lambda^2 \tilde{H}(y) ] dy
\end{align*}

Since the collections of initial conditions do not overlap between these two, we can actually solve them separately. To do this, note for each one we will have somehting like

\begin{align*}
S_1^L(a) &= D_1 d - L_3(\lambda)(0, b, c, d) \\
S_1^T(a) &= - L_3^b(\lambda) (0, b, c, d)
\end{align*}

where the superscript indicates which length parameter is the key one. The idea is, at least for now, that $L$ and $T$ may be vastly different, and even in the case of multiples, the various $L_n$ will be the same order of magnitide, so it makes sense to separate out the one for the periodic BCs.\\

We should now be able to solve for $a, \Delta c$ as we did in the single pulse case, i.e. we should have 

\begin{align*}
(a_{L}, \Delta c_{L}) &= A_1^{L}(\lambda)(b_{L}, c_{L}^-,d) \\
(a_{T}, \Delta c_{T}) &= A_1^{T}(\lambda)(b_{T}, c_{T}^-,d) 
\end{align*}

for appropriate subscripts, e.g.

\begin{align*}
\Delta c^L &+ c_1^+ - c_1^- \\
\Delta c^T &= c_2^- - c_0^-
\end{align*}

 These should have bounds

\begin{align*}
|A_1^T(\lambda)(b_T, c_T^-, d)| &\leq C ( (e^{-\alpha L} + |G|)|b_T| \\
&+ ( p_2(T; \lambda) + |G|e^{\nu(\lambda)T})|c_T^-| \\
&+ e^{-|\nu(\lambda)|T} |\lambda^2||d|)
\end{align*}

and

\begin{align*}
|A_1^L(\lambda)(b_L, c_L^-, d)| &\leq C ( ( e^{-\alpha L} + |G|)|b_L| \\
&+ ( p_2(L; \lambda) + |G|e^{\nu(\lambda)L})|c_L^-| \\
&+ (e^{-|\nu(\lambda)|L}|\lambda^2| + |D_1|)|d|)
\end{align*}

where

\begin{align*}
p_2(x; \lambda) &= |\Delta v_\pm(\pm x, \lambda)| + |\Delta w_\pm(\pm x, \lambda)| 
\end{align*}

We can plug this into our expression for $W_1$ to get $W_2(\lambda)$ which has bound

\begin{align*}
||W_2(\lambda)(b,c^-, d)|| &\leq C (|b| + (e^{\nu(\lambda)T} + e^{\nu(\lambda)L})(|c^-| + |\lambda|^2))
\end{align*}

With a multipulse, we will need an analogue of (3.25) in Sandstede (1998). Starting with the $L$-match (the other one is not useful, since there is no term in $D_1 d$)

\[
D_1 d = a_1^+ - a_1^- + \Delta c_1 v_0(\lambda) + L_3(\lambda)(a, b, \Delta c, c^-)
\]

we write this as

\[
a_1^+ - a_1^- + \Delta c_1 v_0(\lambda) = -D_1 d - L_3(\lambda)(a, b, \Delta c, c^-)
\]

Then we take projections $P^s_0$ and $P^u_0$. Recalling where the $a_i^\pm$ live, this becomes 

\begin{align*}
a_1^+ &= -P^u_0 D_1 d - P^u_0 \Delta c_1 v_0(\lambda) - P^u_0 L_3(\lambda)(a, b, \Delta c, c^-) \\
a_1^- &=  P^s_0 D_1 d + P^s_0 \Delta c_1 v_0(\lambda) - P^s_0 L_3(\lambda)(a, b, \Delta c, c^-)
\end{align*}

Thus we have

\begin{align*}
a_1^+ &= -P^u_0 D_1 d + (A_2^L(\lambda)(b_L, c_L^-, d))_1^+\\
a_1^- &=  P^s_0 D_1 d + (A_2^L(\lambda)(b_L, c_L^-, d))_1^-
\end{align*}

where the bound on $A_2$ is the same as that on $A_1$ except we have two more terms to deal with. The first is the $P^{u/s}_0 \Delta c_1$ term. For $\lambda = 0$, this is wiped out by the projection. So if we write

\[
v_0(\lambda) = (v_0(\lambda) - v_0(0)) + v_0(0)
\]

$v_0(0)$ is wiped out by the two projections, and for the other term, we can use the bound

\[
p_5(\lambda) = |v_0(\lambda) - v_0(0)| 
\]

which should be order $\lambda$.

\begin{align*}
|A_2^L(\lambda)(b_L, c_L^-, d)| &\leq C ( ( e^{-\alpha L} + |G|)|b_L| \\
&+ ( p_2(L; \lambda) + p_5(\lambda) + |G|e^{\nu(\lambda)L})|c_L| \\
&+ (e^{-|\nu(\lambda)|L}|\lambda^2| + |D_1|)|d|)
\end{align*}

\item Next we use the projections 

\begin{align*}
P(\C Q'(0))W^-(0) &= 0 \\
P(\C Q'(0))W^+(0) &= 0 \\
P(Y^+ \oplus Y^- \oplus Y^0) (W^+(0) - W^-(0) ) &= 0
\end{align*}

to implement what happens at $x = 0$ (on both pieces). Note that the $T$ and $L$ terms will mix here. At $x = 0$, the fixed point equations become

\begin{align*}
W_i^-(0) = \Phi^s_-(&0, -X_{i-1}; \lambda)a_i^- + b_i^- + (P^u_-(0; \lambda) - P^u_-(0; 0))b_i^- \\
&+ e^{\nu(\lambda)(X_{i-1})} v_-(0; \lambda) \langle v_0(\lambda), w_-(-X_{i-1}; \lambda) \rangle c_i^- \\
&+ \int_{-X_{i-1}}^0 \Phi^s_-(0, y; \lambda) [ G_i^-(\lambda)W_i^-(y) + \lambda^2 d_i \tilde{H}(y) ] dy \\
&+ \int_{-X_{i-1}}^0
e^{\nu(\lambda)(y)} v_-(0; \lambda) \langle G_i^-(\lambda)(y)W_i^-(y) + \lambda^2 d_i \tilde{H}(y), w_-(y; \lambda) \rangle dy \\
W_i^+(0) = \Phi^u_+(&0, X_i; \lambda)a_i^+ + b_i^+ + (P^s_+(0; \lambda) - P^s_-(0; 0))b_i^+ \\
&+ e^{\nu(\lambda)(-X_i)} v_+(0; \lambda) \langle v_0(\lambda), w_+(X_i; \lambda) \rangle c_i^+ \\
&+ \int_{X_i}^0 \Phi^u_+(0, y; \lambda) [ G_i^+(\lambda)W_i^+(y) + \lambda^2 d_i \tilde{H}(y) ] dy \\
&+ \int_{X_i}^0 e^{\nu(\lambda)(-y)} v_+(0; \lambda) \langle G_i^+(\lambda)(y)W_i^+(y) + \lambda^2 d_i \tilde{H}(y), w_+(y; \lambda) \rangle dy
\end{align*}

Using the same setup as in the single pulse case and doing the same projections, we should have

\[
\begin{pmatrix}x^- \\ x_i^+ \\ y_i^+ - y_i^- + (e^{-\nu(\lambda)X_i} - e^{\nu(\lambda)X_{i-1}}) c_{i-1}^- y_0 \end{pmatrix} + L_4(\lambda)(b, c_{i-1}^-) = 0
\]

We should be able to play the same game we did in the single pulse case and define

\begin{align*}
f_i(\lambda) &= e^{-\nu(\lambda)X_i} - e^{\nu(\lambda)X_{i-1}}  \\
f_i^-(\lambda) &= e^{\nu(\lambda)X_{i-1}} / f_i(\lambda) \\
f_i^+(\lambda) &= e^{-\nu(\lambda)X_{i}} / f_i(\lambda)
\end{align*}

where as before we have $|f_i^\pm| \leq 1$. Then this should look like

\[
\begin{pmatrix}x_i^- \\ x_i^+ \\ y_i^+ - y_i^- + f_i(\lambda) c_{i-1}^- y_0 \end{pmatrix} + L_4(\lambda)(b_i, f_i(\lambda) c_{i-1}^-) = 0
\]

So the same thing should work that we did before, i.e. we should have a bound which looks like 

\[
|B_1(\lambda)| \leq C|\lambda|^2
\]

which makes $c_{i-1}$ be of order $|\lambda^2| / f$. This is annoying but maybe things will work out nice in the end. 


Substitutiting this into $A_1$ and $W_2$ gives us $A_3$ and $W_3$ with bounds

\begin{align*}
|A_3(\lambda)(b, c^-)| &\leq C ( (e^{-\alpha T} + |G|)|\lambda|^2 + ( p_2(T; \lambda) + |G|e^{\nu(\lambda)T})e^{-|\nu(\lambda)T|} |\lambda|^2 + e^{-|\nu(\lambda)|T} |\lambda^2|) \\
&\leq C e^{-|\nu(\lambda)T|} |\lambda|^2
\end{align*}

This implies that $c^+$ and $c^-$ are the same order, which is good. We can plug this into our expression for $W_1$ to get $W_2(\lambda)$ which has bound

\begin{align*}
||W_2(\lambda)(b,c^-)|| &\leq C (|\lambda|^2 + e^{\nu(\lambda)T}(e^{-|\nu(\lambda)T|} |\lambda|^2 + |\lambda|^2)) \\
&\leq C e^{\nu(\lambda)T} |\lambda|^2
\end{align*}





\end{enumerate}

\end{document}