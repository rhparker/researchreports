\documentclass[12pt]{article}
\usepackage[pdfborder={0 0 0.5 [3 2]}]{hyperref}%
\usepackage[left=1in,right=1in,top=1in,bottom=1in]{geometry}%
\usepackage[shortalphabetic]{amsrefs}%
\usepackage{amsmath}
\usepackage{enumerate}
% \usepackage{enumitem}
\usepackage{amssymb}                
\usepackage{amsmath}                
\usepackage{amsfonts}
\usepackage{amsthm}
\usepackage{bbm}
\usepackage[table,xcdraw]{xcolor}
\usepackage{tikz}
\usepackage{float}
\usepackage{booktabs}
\usepackage{svg}
\usepackage{mathtools}
\usepackage{cool}
\usepackage{url}
\usepackage{graphicx,epsfig}
\usepackage{makecell}
\usepackage{array}

\def\noi{\noindent}
\def\T{{\mathbb T}}
\def\R{{\mathbb R}}
\def\N{{\mathbb N}}
\def\C{{\mathbb C}}
\def\Z{{\mathbb Z}}
\def\P{{\mathbb P}}
\def\E{{\mathbb E}}
\def\Q{\mathbb{Q}}
\def\ind{{\mathbb I}}

\graphicspath{ {periodic/} }

\newtheorem{lemma}{Lemma}
\newtheorem{corollary}{Corollary}
\newtheorem{definition}{Definition}
\newtheorem{assumption}{Assumption}
\newtheorem{hypothesis}{Hypothesis}

\begin{document}

\section*{Double Pulse}

Now we repeat the above in the case of a periodic double pulse $q_{2p}$. The picture looks like

\begin{figure}[H]
\includegraphics[width=8.5cm]{dpimage}
\end{figure}

Note that because of the way this method is set up, the period of this thing is $2(L_1 + L_2)$. I am aware this notation is terrible because we also have linear operators $L_i$, but for now we will keep it since it's clear enough which thing we are dealing with. Also the existence proof (if such a thing exists) constructs this whole thing at once, rather than constructing a double pulse and then making it periodic. \\

The equations we are looking to solve are

\begin{enumerate}[(i)]
\item $(W_i^\pm)' = A(q; \lambda) W_i^\pm + G_i(\lambda)^\pm W_i^\pm + \lambda^2 d_i \tilde{H}_i^\pm$
\item $W_i^\pm(0) \in \C \psi(0) \oplus Y^+ \oplus Y^- \oplus Y^0$
\item $W_i^+(0) - W_i^-(0) \in \C \psi(0) $
\item $W_1^+(L_1) - W_2^-(-L_1) = D_1 d $
\item $W_2^+(L_2) - W_1^-(-L_2) = 0$
\end{enumerate}

where

\begin{align*}
G_i(\lambda)^\pm &= A(q_{2p};\lambda) - A(q;\lambda)) \\
D_1 d &= d_2(Q_{2p}'(-L_1) + \lambda (Q_{2p})_c(-L_1)) 
- d_1 ( Q_{2p}'(L_1) + \lambda (Q_{2p})_c(L_1) ) \\
\tilde{H} &= -B(Q_{2p})_c \\
H &= -B Q_c \\
\Delta H &= \tilde{H} - H
\end{align*}

Again we do not have a $\lambda B W_i^\pm$ term since this has been absorbed into $A(q; \lambda) W_i^\pm$. \\

Let $L_m = \min\{ L_1, L_2 \}$ and $L_M = \max\{ L_1, L_2 \}$. Then we should have bounds

\begin{align*}
G_i(\lambda) &= \mathcal{O}(e^{-\alpha L_m}) \\
\Delta H &= \mathcal{O}(e^{-\alpha L_m}) \\
D_1 &= ( Q'(L_1) + Q'(-L_1 )(d_2 - d_1 ) + \mathcal{O} \left( e^{-\alpha L_1} \left( |\lambda| +  e^{-\alpha L_1}  \right) |d| \right)
\end{align*}

The idea behind these (unproven) bounds is that deviation from the single pulse, nonperiodic case decreases exponentially with distance from the center, so by defining $Y$ to be the minimum of the two distances we are using, we are letting the ``worst offender'' dictate the behavior. The $D_1$ equation does not depend on $L_2$, so its bound is dictated by $L_1$ alone.\\

For the setup, let

\[
X = (X_0, L_1, L_2) = (L_2, L_1, L_2)
\]

The fixed point equations then become

\begin{align*}
W_i^-(x) = \Phi^s_-(&x, -X_{i-1}; \lambda)a_{i-1}^- + \Phi^u_-(x, 0; \lambda)b_i^- + e^{\nu(\lambda)(x+X_{i-1})} v_-(x; \lambda) \langle v_0(\lambda), w_-(-X_{i-1}; \lambda) \rangle c_{i-1}^- \\
&+ \int_0^x \Phi^u_-(x, y; \lambda)[ G_i^-(\lambda)W_i^-(y) + \lambda^2 d_i \tilde{H}(y) ] dy \\
&+ \int_{-X_{i-1}}^x \Phi^s_-(x, y; \lambda) [ G_i^-(\lambda)W_i^-(y) + \lambda^2 d_i \tilde{H}(y) ] dy \\
&+ \int_{-X_{i-1}}^x 
e^{\nu(\lambda)(x-y)} v_-(x; \lambda) \langle G_i^-(\lambda)(y)W_i^-(y) + \lambda^2 d_i \tilde{H}(y), w_-(y; \lambda) \rangle dy \\
W_i^+(x) = \Phi^u_+(&x, X_i; \lambda)a_i^+ + \Phi^s_+(x, 0; \lambda)b_i^+ + e^{\nu(\lambda)(x - X_i)} v_+(x; \lambda) \langle v_0(\lambda), w_+(X_i; \lambda) \rangle c_i^+ \\
&+ \int_0^x \Phi^s_+(x, y; \lambda) [ G_i^+(\lambda)W_i^+(y) + \lambda^2 d_i \tilde{H}(y) ] dy \\
&+ \int_{X_i}^x \Phi^u_+(x, y; \lambda) [ G_i^+(\lambda)W_i^+(y) + \lambda^2 d_i \tilde{H}(y) ] dy \\
&+ \int_{X_i}^x e^{\nu(\lambda)(x-y)} v_+(x; \lambda) \langle G_i^+(\lambda)(y)W_i^+(y) + \lambda^2 d_i \tilde{H}(y), w_+(y; \lambda) \rangle dy
\end{align*}

So now we do the same thing we did with the single pulse. The main difference here is the presence of the $d_i$ in the $\tilde{H}$ terms. 

\begin{enumerate}

\item Fix $\tilde{\alpha}$, with $0 < \tilde{\alpha} < \alpha$. 

\item Solve for $W_i$ in terms of the other stuff. The estimates should be the same as the single pulse, with the addition of the $d$ term, so this should be

\[
||W_1(\lambda)(a,b,c,d)|| \leq C (|a| + |b| + e^{\nu(\lambda)L_M}(|c| + |\lambda|^2 |d| ))
\]

where we used the max $L_M$ here since the growth on the ``center'' space gets worse as we go further from 0.

\item Solve for the joins which are not at 0, i.e. solve

\begin{align*}
W_2^+(L_2) - W_1^-(-L_2) &= 0 \\
W_1^+(L_1) - W_2^-(-L_1) &= D_1 d \\
\end{align*}

To solve these, we have two equations. The first one is the periodic matching, which we have in the single pulse case.

\begin{align*}
0 &= a_2^+ - a_0^- \\
&+ (P^u_+(L_2; \lambda) - P_0^u)a_2^+ - (P^s_-(-L_2; \lambda) - P_0^s)a_0^- \\
&+ \Phi^s_+(L_2, 0; \lambda)b_2^+ - \Phi^u_-(-L_2, 0; \lambda)b_1^- \\
&+ v_+(L_2; \lambda) \langle v_0(\lambda), w_+(L_2; \lambda) \rangle c_2^+ - v_-(-L_2; \lambda) \langle v_0(\lambda), w_-(-L_2; \lambda) \rangle c_0^- \\
&+ \int_0^{L_2} \Phi^u_+(L_2, y; \lambda) [ G(\lambda)W^-(y) + d_2 \lambda^2 \tilde{H}(y) ] dy \\
&- \int_0^{-L_2} \Phi^s_-(-L_2, y; \lambda) [ G(\lambda)W^+(y) + d_1 \lambda^2 \tilde{H}(y) ] dy
\end{align*}

The second one is the center matching at $\pm L$.

\begin{align*}
D_1 d &= a_1^+ - a_1^- \\
&+ (P^u_+(L_1; \lambda) - P_0^u)a_1^+ - (P^s_-(-L_1; \lambda) - P_0^s)a_1^- \\
&+ \Phi^s_+(L_1, 0; \lambda)b_1^+ - \Phi^u_-(-L_1, 0; \lambda)b_2^- \\
&+ v_+(L_1; \lambda) \langle v_0(\lambda), w_+(L_1; \lambda) \rangle c_1^+ - v_-(-L_1; \lambda) \langle v_0(\lambda), w_-(-L_1; \lambda) \rangle c_1^- \\
&+ \int_0^{L_1} \Phi^u_+(L_1, y; \lambda) [ G(\lambda)W^-(y) + d_1 \lambda^2 \tilde{H}(y) ] dy \\
&- \int_0^{-L_1} \Phi^s_-(-L_1, y; \lambda) [ G(\lambda)W^+(y) + d_2 \lambda^2 \tilde{H}(y) ] dy
\end{align*}

Since the collections of initial conditions do not overlap between these two, we can actually solve them separately. To do this, note for each one we will have somehting like

\begin{align*}
S_1^{L_1}(a) &= D_1 d - L_3^{L_1}(\lambda)(0, b, c, d) \\
S_1^{L_2}(a) &= - L_3^{L_2}(\lambda) (0, b, c, d)
\end{align*}

where the superscript indicates which length parameter is the key one. This notation is kind of terrible, but at least it indicates which one we are dealing with. The idea is, at least for now, that $L_1$ and $L_2$ may be quite different, so it makes sense to separate out the one for the periodic BCs for now.\\

We should now be able to solve for $a, \Delta c$ as we did in the single pulse case, i.e. we should have 

\begin{align*}
(a_{L_1}, \Delta c_{L_1}) &= A_1^{L_1}(\lambda)(b_{L_1}, c_{L_1}^-,d) \\
(a_{L_2}, \Delta c_{L_2}) &= A_1^{L_2}(\lambda)(b_{L_2}, c_{L_2}^-,d) 
\end{align*}

for appropriate subscripts, e.g.

\begin{align*}
\Delta c_{L_1} &= c_1^+ - c_1^- \\
\Delta c_{L_2} &= c_2^+ - c_0^-
\end{align*}

 These should have bounds

\begin{align*}
|A_1^{L_2}(\lambda)(b_{L_2}, c_{L_2}^-, d)| &\leq C ( (e^{-\alpha L} + |G|)|b_{L_2}| \\
&+ ( p_2(L_2; \lambda) + |G|e^{\nu(\lambda)L_2})|c_{L_2}^-| \\
&+ e^{-\tilde{\alpha}L_2} |\lambda^2||d|)
\end{align*}

and

\begin{align*}
|A_1^{L_1}(\lambda)(b_{L_1}, c_{L_1}^-, d)| &\leq C ( ( e^{-\alpha L_1} + |G|)|b_{L_1}| \\
&+ ( p_2(L_1; \lambda) + |G|e^{\nu(\lambda) L_1})|c_{L_1}^-| \\
&+ (e^{-\tilde{\alpha}L_1}|\lambda^2| + |D_1|)|d|)
\end{align*}

where

\begin{align*}
p_2(x; \lambda) &= |\Delta v_\pm(\pm x, \lambda)| + |\Delta w_\pm(\pm x, \lambda)| 
\end{align*}

We will need a bound which combines $L_1$ and $L_2$, so to do this we always assume the worst case scenario

\begin{align*}
|A_1(\lambda)(b, c^-, d)| &\leq C ( ( e^{-\alpha L_m} + |G|)|b|
+ ( p_2(L_m; \lambda) + |G|e^{\nu(\lambda) L_M})|c^-| 
+ (e^{-\tilde{\alpha}L_m}|\lambda^2| + |D_1|)|d|)
\end{align*}

We can plug this into our expression for $W_1$ to get $W_2(\lambda)$ which has bound

\begin{align*}
||W_2(\lambda)(b,c^-, d)|| &\leq C (|b| + e^{\nu(\lambda)L_M}(|c^-| + |\lambda|^2))
\end{align*}

With a multipulse, we will need an analogue of (3.25) in Sandstede (1998). Starting with the $L_1$-match (the other one is not useful, since there is no term in $D_1 d$)

\[
D_1 d = a_1^+ - a_1^- + \Delta c_1 v_0(\lambda) + L_3^{L_1}(\lambda)(a, b, \Delta c, c^-)
\]

we write this as

\[
a_1^+ - a_1^- + \Delta c_1 v_0(\lambda) = -D_1 d - L_3^{L_1}(\lambda)(a, b, \Delta c, c^-)
\]

Then we take projections $P^s_0$ and $P^u_0$. Recalling where the $a_i^\pm$ live, this becomes 

\begin{align*}
a_1^+ &= -P^u_0 D_1 d - P^u_0 \Delta c_1 v_0(\lambda) - P^u_0 L_3^{L_1}(\lambda)(a, b, \Delta c, c^-) \\
a_1^- &=  P^s_0 D_1 d + P^s_0 \Delta c_1 v_0(\lambda) - P^s_0 L_3^{L_1}(\lambda)(a, b, \Delta c, c^-)
\end{align*}

Thus we have

\begin{align*}
a_1^+ &= -P^u_0 D_1 d + (A_2^{L_1}(\lambda)(b_{L_1}, c_{L_1}^-, d))_1^+\\
a_1^- &=  P^s_0 D_1 d + (A_2^{L_1}(\lambda)(b_{L_1}, c_{L_1}^-, d))_1^-
\end{align*}

where the bound on $A_2$ is the same as that on $A_1$ except we have some more terms to deal with. The first is the $P^{u/s}_0 \Delta c_1$ term. For $\lambda = 0$, this is wiped out by the projection. So if we write

\[
v_0(\lambda) = (v_0(\lambda) - v_0(0)) + v_0(0)
\]

$v_0(0)$ is wiped out by the two projections, and for the other term, we can use the bound

\[
p_5(\lambda) = |v_0(\lambda) - v_0(0)| 
\]

which should be order $\lambda$. Thus we redefine $A_2(\lambda)$ as 

\begin{align*}
(A_2(\lambda)(b_{L_1}, c_{L_1}^-, d))^- &= -P^u_0 \Delta c_1 (v_0(\lambda) - v_0(0)) - P^u_0 L_3^{L_1}(\lambda)(a, b, \Delta c, c^-) \\
(A_2(\lambda)(b_{L_1}, c_{L_1}^-, d))^+ &= P^s_0 \Delta c_1 (v_0(\lambda) - v_0(0)) - P^s_0 L_3^{L_1}(\lambda)(a, b, \Delta c, c^-)
\end{align*}

where there is an $a$ on the RHS, but we will put $A_1$ in for that. We do need the bound for $L_3$, so in this case we have

\begin{align*}
|L_3^{L_1}&(\lambda)(a, b, \Delta c_{L_1}, c_{L_1}^-, d)| \leq C ( (p_1(L_1; \lambda) + |G|)|a_{L_1}| + (e^{-\alpha L_1} + |G|)|b_{L_1}| \\
&+ ( p_2(L_1; \lambda) + |G|e^{\nu(\lambda)L_1})|\Delta c_{L_1}| + ( p_2(L_1; \lambda) + |G|e^{\nu(\lambda)L_1})|c^-|+ e^{-\tilde{\alpha} L_1} |\lambda^2||d| )
\end{align*}

Plugging $A_1^{L_1}$ in for $a$ and $\Delta c_{L_1}$ we get

\begin{align*}
|A_2(\lambda)&(b_{L_1}, c_{L_1}^-, d)| \\
&\leq ( p_1(L_1; \lambda) + p_2(L_1; \lambda) + p_5(\lambda) + |G|e^{\nu(\lambda)L_1})( e^{-\alpha L_1} + |G|)|b_{L_1}  \\
&+ ( p_1(L_1; \lambda) + p_2(L_1; \lambda) + p_5(\lambda) + |G|e^{\nu(\lambda)L_1})( p_2(L_1; \lambda) + |G|e^{\nu(\lambda) L_1})|c_{L_1}^-| \\
&+ ( p_1(L_1; \lambda) + p_2(L_1; \lambda) + p_5(\lambda) + |G|e^{\nu(\lambda)L_1})(e^{-\tilde{\alpha}L_1}|\lambda^2| + |D_1|)|d|) \\  
&+ ( e^{-\alpha L_1} + |G|)|b_{L_1}| \\
&+ ( p_2(L_1; \lambda) + |G|e^{\nu(\lambda)L_1})|c_{L_1}^-|+ e^{-\tilde{\alpha} L_1} |\lambda^2||d| ) \\
&\leq C( e^{-\alpha L_1} + |G|)|b_{L_1}| + ( p_2(L_1; \lambda) + |G|e^{\nu(\lambda)L_1})|c_{L_1}^-| \\
&+ (( p_1(L_1; \lambda) + p_2(L_1; \lambda) + p_5(\lambda) + |G|e^{\nu(\lambda)L_1})|D_1| + e^{-\tilde{\alpha}L_1}|\lambda^2|)|d| ) 
\end{align*}


\item Next we use the projections 

\begin{align*}
P(\C Q'(0))W_i^-(0) &= 0 \\
P(\C Q'(0))W_i^+(0) &= 0 \\
P(Y^+ \oplus Y^- \oplus Y^0) ( W_i^+(0) - W_i^-(0) ) &= 0
\end{align*}

to implement what happens at $x = 0$ (on both pieces). Note that the $L_1$ and $L_2$ terms will mix here. At $x = 0$, the fixed point equations become

\begin{align*}
W_i^-(0) = \Phi^s_-(&0, -X_{i-1}; \lambda)a_{i-1}^- + b_i^- + (P^u_-(0; \lambda) - P^u_-(0; 0))b_i^- \\
&+ e^{\nu(\lambda)(X_{i-1})} v_-(0; \lambda) \langle v_0(\lambda), w_-(-X_{i-1}; \lambda) \rangle c_{i-1}^- \\
&+ \int_{-X_{i-1}}^0 \Phi^s_-(0, y; \lambda) [ G_i^-(\lambda)W_i^-(y) + \lambda^2 d_i \tilde{H}(y) ] dy \\
&+ \int_{-X_{i-1}}^0
e^{\nu(\lambda)(y)} v_-(0; \lambda) \langle G_i^-(\lambda)(y)W_i^-(y) + \lambda^2 d_i \tilde{H}(y), w_-(y; \lambda) \rangle dy \\
W_i^+(0) = \Phi^u_+(&0, X_i; \lambda)a_i^+ + b_i^+ + (P^s_+(0; \lambda) - P^s_-(0; 0))b_i^+ \\
&+ e^{\nu(\lambda)(-X_i)} v_+(0; \lambda) \langle v_0(\lambda), w_+(X_i; \lambda) \rangle c_i^+ \\
&+ \int_{X_i}^0 \Phi^u_+(0, y; \lambda) [ G_i^+(\lambda)W_i^+(y) + \lambda^2 d_i \tilde{H}(y) ] dy \\
&+ \int_{X_i}^0 e^{\nu(\lambda)(-y)} v_+(0; \lambda) \langle G_i^+(\lambda)(y)W_i^+(y) + \lambda^2 d_i \tilde{H}(y), w_+(y; \lambda) \rangle dy
\end{align*}

Using the same setup as in the single pulse case and doing the same projections, we would like to get something like

\[
\begin{pmatrix}x^- \\ x_i^+ \\ y_i^+ - y_i^- + (e^{-\nu(\lambda)X_i} - e^{\nu(\lambda)X_{i-1}}) c^- y_0 \end{pmatrix} + L_4(\lambda)(b, c^-,d) = 0
\]

It turns out it's not that simple, but let's see what we can get. First, let's look at the 3rd component of this. The bound on $L_4$ will be determined by this, since the same bound will hold for the first two components.

\begin{align*}
W_i^+(0) - W_i^-(0) ) &= \Phi^u_+(0, X_i; \lambda)a_i^+ - \Phi^s_-(0, -X_{i-1}; \lambda)a_i^- \\
&+ b_i^+ - b_i^- + (P^s_+(0; \lambda) - P^s_-(0; 0))b_i^+  - (P^u_-(0; \lambda) - P^u_-(0; 0))b_i^- \\
&+ e^{-\nu(\lambda)X_i} c_i^+ y_0 + e^{-\nu(\lambda)X_i} c_i^+( (v_+(0; \lambda) - y_0) + v_+(0; \lambda) \langle  v_0(\lambda), \Delta w_+(X_i; \lambda) \rangle) \\
&- e^{\nu(\lambda)X_{i-1}} c_{i-1}^- y_0 - e^{\nu(\lambda)X_{i-1}} c_{i-1}^- ( (v_-(0; \lambda) - y_0) + v_-(0; \lambda) \langle  v_0(\lambda), \Delta w_-(-X_{i-1}; \lambda) \rangle) \\
&+ \int_{-X_{i-1}}^0 \Phi^s_-(0, y; \lambda) [ G_i^-(\lambda)W_i^-(y) + \lambda^2 d_i \tilde{H}(y) ] dy \\
&+ \int_{-X_{i-1}}^0
e^{\nu(\lambda)(y)} v_-(0; \lambda) \langle G_i^-(\lambda)(y)W_i^-(y) + \lambda^2 d_i \tilde{H}(y), w_-(y; \lambda) \rangle dy \\
&+ \int_{X_i}^0 \Phi^u_+(0, y; \lambda) [ G_i^+(\lambda)W_i^+(y) + \lambda^2 d_i \tilde{H}(y) ] dy \\
&+ \int_{X_i}^0 e^{\nu(\lambda)(-y)} v_+(0; \lambda) \langle G_i^+(\lambda)(y)W_i^+(y) + \lambda^2 d_i \tilde{H}(y), w_+(y; \lambda) \rangle dy
\end{align*}

When we make the $\Delta c$ substitution, the coefficients will mix, since

\begin{align*}
c_1^+ &= c_1^- + \Delta c_{L_1} \\
c_2^+ &= c_0^- + \Delta c_{L_2} 
\end{align*}

To allow us to do this with one set of notation, let $c_2^- = c_0^-$, and define 

\begin{align*}
c_1^+ &= c_1^- + \Delta c_1 \\
c_2^+ &= c_0^- + \Delta c_0
\end{align*}

Then we have

\begin{align*}
W_i^+(0) - W_i^-(0) &= b_i^+ - b_i^- + e^{-\nu(\lambda)X_i} c_i^- y_0 - e^{\nu(\lambda)X_{i-1}} c_{i-1}^- y_0\\
&+ \Phi^u_+(0, X_i; \lambda)a_i^+ - \Phi^s_-(0, -X_{i-1}; \lambda)a_i^- \\
&+ (P^s_+(0; \lambda) - P^s_-(0; 0))b_i^+  - (P^u_-(0; \lambda) - P^u_-(0; 0))b_i^- \\
&+ e^{-\nu(\lambda)X_i} \Delta c_i y_0 \\
&+ e^{-\nu(\lambda)X_i} \Delta c_i ( (v_+(0; \lambda) - y_0) + v_+(0; \lambda) \langle  v_0(\lambda), \Delta w_+(X_i; \lambda) \rangle) \\
&+ e^{-\nu(\lambda)X_i} c_i^-( (v_+(0; \lambda) - y_0) + v_+(0; \lambda) \langle  v_0(\lambda), \Delta w_+(X_i; \lambda) \rangle) \\
&- e^{\nu(\lambda)X_{i-1}} c_{i-1}^- ( (v_-(0; \lambda) - y_0) + v_-(0; \lambda) \langle  v_0(\lambda), \Delta w_-(-X_{i-1}; \lambda) \rangle) \\
&+ \int_{-X_{i-1}}^0 \Phi^s_-(0, y; \lambda) [ G_i^-(\lambda)W_i^-(y) + \lambda^2 d_i \tilde{H}(y) ] dy \\
&+ \int_{-X_{i-1}}^0
e^{\nu(\lambda)(y)} v_-(0; \lambda) \langle G_i^-(\lambda)(y)W_i^-(y) + \lambda^2 d_i \tilde{H}(y), w_-(y; \lambda) \rangle dy \\
&+ \int_{X_i}^0 \Phi^u_+(0, y; \lambda) [ G_i^+(\lambda)W_i^+(y) + \lambda^2 d_i \tilde{H}(y) ] dy \\
&+ \int_{X_i}^0 e^{\nu(\lambda)(-y)} v_+(0; \lambda) \langle G_i^+(\lambda)(y)W_i^+(y) + \lambda^2 d_i \tilde{H}(y), w_+(y; \lambda) \rangle dy
\end{align*}

Then our matrix-equation becomes

\[
\begin{pmatrix}x_i^- \\ x_i^+ \\ y_i^+ - y_i^- + (e^{-\nu(\lambda)X_i} c_i^- - e^{\nu(\lambda)X_{i-1}} c_{i-1}^-) y_0 \end{pmatrix} + L_4(\lambda)(b_i, e^{-\nu(\lambda)X_i} c_i^-, e^{\nu(\lambda)X_{i-1}} c_{i-1}^-, d) = 0
\]

where we plugged in the approprate things from above for $W$, $a$, and $\Delta c$. Unfortunately, this is not what we want. To do the inversion, we would like this to be

\[
\begin{pmatrix}x_i^- \\ x_i^+ \\ y_i^+ - y_i^- + (e^{-\nu(\lambda)X_i} c_i^- - e^{\nu(\lambda)X_{i-1}} c_{i-1}^-) y_0 \end{pmatrix} + L_4(\lambda)(b_i, (e^{-\nu(\lambda)X_i} c_i^- + e^{\nu(\lambda)X_{i-1}} c_{i-1}^-), d) = 0
\]

so then the terms on the RHS and LHS would match. I am not sure if I see a way to get this. One thing I tried was taking the sum and the difference of our matrix equations for $i = 1, 2$. If we can solve those, then we can solve the original set of two equations. If we do that, the LHS of the two equations become

\begin{align*}
&\begin{pmatrix}x_1^- - x_2^- \\ x_1^+ - x_2^+ \\ (y_1^+ - y_1^-) - (y_2^+ - y_2^-) + (e^{-\nu(\lambda)X_1} c_1^- - e^{\nu(\lambda)X_{0}} c_{0}^-) y_0 - (e^{-\nu(\lambda)X_2} c_2^- - e^{\nu(\lambda)X_{1}} c_{1}^-)\end{pmatrix} \\
&= \begin{pmatrix}x_1^- - x_2^- \\ x_1^+ - x_2^+ \\ (y_1^+ - y_1^-) - (y_2^+ - y_2^-) + [(e^{-\nu(\lambda)L_1} + e^{\nu(\lambda)L_1}) c_1^- - (e^{-\nu(\lambda)L_2} + e^{\nu(\lambda)L_2})c_0^-]y_0 \end{pmatrix} 
\end{align*}

and 

\begin{align*}
&\begin{pmatrix}x_1^- + x_2^- \\ x_1^+ + x_2^+ \\ (y_1^+ - y_1^-) + (y_2^+ - y_2^-) + (e^{-\nu(\lambda)X_1} c_1^- - e^{\nu(\lambda)X_{0}} c_{0}^-) y_0 - (e^{-\nu(\lambda)X_2} c_2^- - e^{\nu(\lambda)X_{1}} c_{1}^-)\end{pmatrix} \\
&= \begin{pmatrix}x_1^- + x_2^- \\ x_1^+ + x_2^+ \\ (y_1^+ - y_1^-) + (y_2^+ - y_2^-) + [(e^{-\nu(\lambda)L_1} - e^{\nu(\lambda)L_1}) c_1^- + (e^{-\nu(\lambda)L_2} - e^{\nu(\lambda)L_2})c_0^-]y_0 \end{pmatrix} \\
\end{align*}

These look rather nice, but the RHS of them again is not what we need to do the inversion. I am not sure what the right thing to do here is, since it appears the ``mixing'' of the coefficients $c_i^\pm$ is causing all the problems.


\end{enumerate}

\end{document}