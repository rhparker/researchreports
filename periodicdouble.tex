\documentclass[12pt]{article}
\usepackage[pdfborder={0 0 0.5 [3 2]}]{hyperref}%
\usepackage[left=1in,right=1in,top=1in,bottom=1in]{geometry}%
\usepackage[shortalphabetic]{amsrefs}%
\usepackage{amsmath}
\usepackage{enumerate}
% \usepackage{enumitem}
\usepackage{amssymb}                
\usepackage{amsmath}                
\usepackage{amsfonts}
\usepackage{amsthm}
\usepackage{bbm}
\usepackage[table,xcdraw]{xcolor}
\usepackage{tikz}
\usepackage{float}
\usepackage{booktabs}
\usepackage{svg}
\usepackage{mathtools}
\usepackage{cool}
\usepackage{url}
\usepackage{graphicx,epsfig}
\usepackage{makecell}
\usepackage{array}

\def\noi{\noindent}
\def\T{{\mathbb T}}
\def\R{{\mathbb R}}
\def\N{{\mathbb N}}
\def\C{{\mathbb C}}
\def\Z{{\mathbb Z}}
\def\P{{\mathbb P}}
\def\E{{\mathbb E}}
\def\Q{\mathbb{Q}}
\def\ind{{\mathbb I}}

\graphicspath{ {periodic/} }

\newtheorem{lemma}{Lemma}
\newtheorem{corollary}{Corollary}
\newtheorem{definition}{Definition}
\newtheorem{assumption}{Assumption}
\newtheorem{hypothesis}{Hypothesis}

\begin{document}

\section*{Double Pulse}

Now we repeat the above in the case of a periodic double pulse $q_{2p}$. The picture looks like

\begin{figure}[H]
\includegraphics[width=8.5cm]{dpimage}
\end{figure}

Note that because of the way this method is set up, the period of this thing is $2(X_1 + X_2)$. The existence proof (if such a thing exists) constructs this whole thing at once, rather than constructing a double pulse and then making it periodic. \\

The equations we are looking to solve are

\begin{enumerate}[(i)]
\item $(W_i^\pm)' = A(q; \lambda) W_i^\pm + G_i(\lambda)^\pm W_i^\pm + \lambda^2 d_i \tilde{H}_i^\pm$
\item $W_1^+(X_1) - W_2^-(-X_1) = D_1 d$
\item $W_2^+(X_2) - W_1^-(-X_2) = D_2 d$
\end{enumerate}

where

\begin{align*}
G_i(\lambda)^\pm (x) &= A(q_{2p}(x);\lambda) - A(q(x);\lambda)) \\
D_1 d &= d_2(Q_{2p}'(-X_1) + \lambda (Q_{2p})_c(-X_1))
- d_1 ( Q_{2p}'(X_1) + \lambda (Q_{2p})_c(X_1) ) \\
D_2 d &= d_1(Q_{2p}'(-X_2) + \lambda (Q_{2p})_c(-X_2))
- d_2 ( Q_{2p}'(X_2) + \lambda (Q_{2p})_c(X_2) ) \\
\tilde{H} &= -B(Q_{2p})_c \\
H &= -B Q_c \\
\Delta H &= \tilde{H} - H
\end{align*}

NOTE TO SELF: DOUBLE CHECK THE SIGN FOR $D_2 d$ DOES NOT MATTER FOR THE BOUND BUT WE NEED TO GET IT RIGHT. We do not have a $\lambda B W_i^\pm$ term since this has been absorbed into $A(q; \lambda) W_i^\pm$. \\

We also have some conditions about which spaces various things should lie in. From prior problems (without the ``center'' space), these look like

\begin{enumerate}[(i)]
\item $W_i^\pm(0) \in \C \psi(0) \oplus Y^+ \oplus Y^-$
\item $W_i^+(0) - W_i^-(0) \in \C \psi(0) $
\end{enumerate}

These will be slightly different for the periodic wavetrain. Let's recall what the various spaces are. All these things are defined at $x = 0$, since that is the ``middle'' of the homoclinic orbit. All these are also for the unperturbed situation, i.e. when $\lambda = 0$.

\begin{enumerate}
	\item $\text{span }\{Q'(0)\}$ is the span of the derivative of the pulse at $x = 0$. This is in both the stable and unstable manifolds of 0.
	\item $Y^+$ is the remaining dimension of the 2D stable manifold.
	\item $Y^-$ is the remaining dimension of the 2D unstable manifold.
	\item $\C \Psi(0)$ is the span of the solution to the adjoint variational equation, and since it is perpendicular to the unstable and stable manifolds, it fills out $\C^4$ when combined with the three things above.
\end{enumerate}

In our case, we will have an additional ``center'' space at $x = 0$, which gives us one more dimension. For now, we will call this $Y^0$, but this will become more specific later. What we need to do now is figure out what role this plays in our conditions. Elements in the center space may blow up (not necessarily exponentially) at one of the ends. Thus if we want to look for a localized eigenfunction, it is important that the initial condition at $x = 0$ not have a component in this space. \\

This essentially tells us where our perturbation must live. Since we are specifying our eigenfunction as $V\pm = Q' - \lambda Q_c + W_i\pm$, and we know that $Q'(0)$ and $Q_c(0)$ have no component in the $Y^0$ (since they both decay exponentially at both ends), this implies that $W^i\pm(0)$ cannot have a component in $Y^0$. This also implies that $W^+(0) - W^-(0)$ cannot have a component in $Y^0$. Thus the conditions are the same as for the version without the ``center'' space, i.e. 

\begin{enumerate}[(i)]
\item $W^\pm(0) \in \C \psi(0) \oplus Y^+ \oplus Y^-$
\item $W^+(0) - W^-(0) \in \C \psi(0) $
\end{enumerate}

Let $X_m = \min\{ X_1, X_2 \}$ and $X_M = \max\{ X_1, X_2 \}$. Then we should have bounds WHICH WE SHOULD SHOW IF THIS ALL WORKS

\begin{align*}
G_i(\lambda) &= \mathcal{O}(e^{-\alpha X_m}) \\
\Delta H &= \mathcal{O}(e^{-\alpha X_m}) \\
D_i &= ( Q'(X_i) + Q'(-X_i)(d_{i+1} - d_i ) + \mathcal{O} \left( e^{-\alpha X_i} \left( |\lambda| +  e^{-\alpha X_i}  \right) |d| \right)
\end{align*}

The idea behind these (unproven) bounds is that deviation from the single pulse, nonperiodic case decreases exponentially with distance from the center, so we are letting the ``worst offender'' dictate the behavior. The $D_i$ equations only depend on $X_i$, so the bound only involves that. \\

For the setup, let

\[
X = (X_0, X_1, X_2) = (X_2, X_1, X_2)
\]

YES THIS IS AWFUL NOTATION, CAN FIX IF THIS WORKS. The fixed point equations then become

\begin{align*}
W_i^-(x) = \Phi^s_-(&x, -X_{i-1}; \lambda)a_{i-1}^- + \Phi^u_-(x, 0; \lambda)b_i^- + e^{\nu(\lambda)(x+X_{i-1})} v_-(x; \lambda) \langle v_0(\lambda), w_-(-X_{i-1}; \lambda) \rangle c_{i-1}^- \\
&+ \int_0^x \Phi^u_-(x, y; \lambda)[ G_i^-(\lambda)W_i^-(y) + \lambda^2 d_i \tilde{H}(y) ] dy \\
&+ \int_{-X_{i-1}}^x \Phi^s_-(x, y; \lambda) [ G_i^-(\lambda)W_i^-(y) + \lambda^2 d_i \tilde{H}(y) ] dy \\
&+ \int_{-X_{i-1}}^x 
e^{\nu(\lambda)(x-y)} v_-(x; \lambda) \langle G_i^-(\lambda)(y)W_i^-(y) + \lambda^2 d_i \tilde{H}(y), w_-(y; \lambda) \rangle dy \\
W_i^+(x) = \Phi^u_+(&x, X_i; \lambda)a_i^+ + \Phi^s_+(x, 0; \lambda)b_i^+ + e^{\nu(\lambda)(x - X_i)} v_+(x; \lambda) \langle v_0(\lambda), w_+(X_i; \lambda) \rangle c_i^+ \\
&+ \int_0^x \Phi^s_+(x, y; \lambda) [ G_i^+(\lambda)W_i^+(y) + \lambda^2 d_i \tilde{H}(y) ] dy \\
&+ \int_{X_i}^x \Phi^u_+(x, y; \lambda) [ G_i^+(\lambda)W_i^+(y) + \lambda^2 d_i \tilde{H}(y) ] dy \\
&+ \int_{X_i}^x e^{\nu(\lambda)(x-y)} v_+(x; \lambda) \langle G_i^+(\lambda)(y)W_i^+(y) + \lambda^2 d_i \tilde{H}(y), w_+(y; \lambda) \rangle dy
\end{align*}

where

\begin{align*}
(a^-, a^+) &\in E^s \oplus E^u\\
(b^-, b^+) &\in R^u_-(0; 0) \oplus R^s_+(0; 0)\\
\end{align*}

Note that these spaces refer to the original, unperturbed problem, i.e. with $\lambda = 0$. The projections onto $E^s$, $E^u$, and $E^c$ are given by $P_0^s$, $P_0^u$, and $P_0^c$. The initial conditions on the ``center'' subspace is given by $c^\pm v_0(\lambda)$, where here $v_0(\lambda)$ refers to the perturbed problem. This is useful since we have equations for the evolution along this subspace.\\

Now we do the same thing we did with the single pulse. The main difference here is the presence of the $d_i$ in the $\tilde{H}$ terms. \\

We take the following notational conventions, which we use for simplicity. The first one makes sense since periodic BCs basically means that things ``wrap around'', so can can consider the problem as posed on a ``loop''.

\begin{enumerate}[(i)]
\item $c_2^- = c_0^-$, $a_2^- = a_0^-$, $b_0^- = b_2^-$, $d_0 = d_2$, and $W_0 = W_2$
\item If we eliminate either a subscript or a superscript in the norm, we are taking the max of the eliminated things, e.g. $|c_i| = \max(|c_i^+|, |c_i^-|)$ or $|c^+| = \max(|c_1^+|, |c_2^+|)$. 
\end{enumerate}

Now we can do the thing.

\begin{enumerate}

\item Fix $\tilde{\alpha}$, with $0 < \tilde{\alpha} < \alpha$. 

\item For convenience, we are taking $\nu(\lambda) \geq 0$. If this works, we will redo it with the appropriate absolute value signs. The idea is that we know one of $e^{\nu(\lambda) X_1}$ and $e^{-\nu(\lambda) X_1}$ will blow up, but we do not know which one.

\item Solve for $W_i$ in terms of the other stuff. We do the same thing with the linear operators $L_1$ and $L_2$ as we did in the single pulse case.\\

Linear operator $L_1$ is stuff from the fixed point equations involving $W$.

\begin{align*}
(L_1(\lambda)W)_i^-(x) &= \int_0^x \Phi^u_-(x, y; \lambda) G_i^-(\lambda)W_i^-(y) dy + \int_{-X_{i-1}}^x \Phi^s_-(x, y; \lambda) G_i^-(\lambda)W_i^-(y) dy \\
&+ \int_{-X_{i-1}}^x 
e^{\nu(\lambda)(x-y)} v_-(x; \lambda) \langle G_i^-(\lambda)(y)W_i^-(y), w_-(y; \lambda) \rangle dy \\
(L_1(\lambda)W)_i^-(x) &= \int_0^x \Phi^s_+(x, y; \lambda) G_i^+(\lambda)W_i^+(y) dy + \int_{X_i}^x \Phi^u_+(x, y; \lambda) G_i^+(\lambda) W_i^+(y) dy \\
&+ \int_{X_i}^x e^{\nu(\lambda)(x-y)} v_+(x; \lambda) \langle G_i^+(\lambda)(y)W_i^+(y), w_+(y; \lambda) \rangle dy
\end{align*}

The first two terms on the RHS of this are like those from Sanstede (1998). For the third term we have for the negative piece,

\begin{align*}
\Big| \int_{-X_{i-1}}^x &e^{\nu(\lambda)(x-y)} v_-(x; \lambda) \langle G(\lambda)(y)W_i^-(y), w_-(y; \lambda) \rangle dy \Big| \\
&\leq \int_{-X_{i-1}}^x e^{\nu(\lambda)(x-y)} |v_-(x; \lambda)| |G(\lambda)|||W|||w_-(y; \lambda)|dy \\
&\leq |G||v||w|||W|| \int_{-X_{i-1}}^x e^{\nu(\lambda)(x-y)} dy \\
&= |G||v||w|||W|| \frac{e^{\nu(\lambda)x} - 1}{\nu(\lambda)} \\
&\leq C e^{\nu(\lambda)X_{i-1}} |G| \: ||W||
\end{align*}

where we used the fact that $x \leq 0$ on the negative piece. Since $v$ and $w$ are bounded and only depend on $\lambda$ (we pulled out the exponential growth/decay in our expressions for $\tilde{v}$ and $\tilde{w}$), we incorporate those bounds into the constant $C$, which depends on $\lambda$. The positive piece has a similar bound with $X_{i-1}$ replaced with $X_i$. Thus since this depends on the larger of the $X_i$ the overall bound is 

\[
||L_1(\lambda)W|| \leq C e^{\nu(\lambda)X_M} |G| \: ||W||
\]

Linear operator $L_2$ is the stuff from fixed point equations not involving $W$.

\begin{align*}
(L_2(\lambda)&(a,b,c,d))^-(x) = \Phi^s_-(x, -X_{i-1}; \lambda)a_{i-1}^- + \Phi^u_-(x, 0; \lambda)b_i^- \\
&+ e^{\nu(\lambda)(x+X_{i-1})} v_-(x; \lambda) \langle v_0(\lambda), w_-(-X_{i-1}; \lambda) \rangle c_{i-1}^- \\
&+ \int_0^x \Phi^u_-(x, y; \lambda)\lambda^2 d_i \tilde{H}(y) dy + \int_{-X_{i-1}}^x \Phi^s_-(x, y; \lambda) \lambda^2 d_i \tilde{H}(y) dy \\
&+ \int_{-X_{i-1}}^x 
e^{\nu(\lambda)(x-y)} v_-(x; \lambda) \langle \lambda^2 d_i \tilde{H}(y), w_-(y; \lambda) \rangle dy \\
(L_2(\lambda)&(a,b,c,d))^+(x) \Phi^u_+(x, X_i; \lambda)a_i^+ + \Phi^s_+(x, 0; \lambda)b_i^+ \\
&+ e^{\nu(\lambda)(x - X_i)} v_+(x; \lambda) \langle v_0(\lambda), w_+(X_i; \lambda) \rangle c_i^+ \\
&+ \int_0^x \Phi^s_+(x, y; \lambda) \lambda^2 d_i \tilde{H}(y) dy + \int_{X_i}^x \Phi^u_+(x, y; \lambda) \lambda^2 d_i \tilde{H}(y) dy \\
&+ \int_{X_i}^x e^{\nu(\lambda)(x-y)} v_+(x; \lambda) \langle \lambda^2 d_i \tilde{H}(y), w_+(y; \lambda) \rangle dy
\end{align*}

Most of the bounds on these terms are the same as in Sanstede (1998). For the $c$ terms, we have

\[
e^{\nu(\lambda)(x+X_{i-1})} v_-(x; \lambda) \langle v_0(\lambda), w_-(-X_{i-1}; \lambda) \rangle c_{i-1}^- \leq C e^{\nu(\lambda) X_{i-1} }|c_{i-1}^-|
\]

and similar for the $c_i^+$. In each case, the length scale $X_i$ is paired with $c_i^\pm$. We have an $e^{\nu(\lambda)X_i}$ term in the bound, which could grow exponentially, but for now there is nothing we can do about it. \\

For the third integrals in $L_2$, we use the $\tilde{\alpha}$ trick to get a better bound which does not involve a potential exponential growth term.

\begin{align*}
&\left| \int_{-X_{i-1}}^x 
e^{\nu(\lambda)(x-y)} v_-(x; \lambda) \langle \lambda^2 d_i \tilde{H}(y), w_-(y; \lambda) \rangle dy \right| \\
&\leq C |\lambda|^2 |d| e^{\tilde{\alpha}x} \int_{-X_{i-1}}^x e^{-\tilde{\alpha}x} e^{\tilde{\alpha}y} e^{\nu(\lambda)(x-y)} |e^{-\tilde{\alpha}y}\tilde{H}(y)|dy \\
&\leq C |\lambda|^2 |d| e^{\tilde{\alpha}x} \int_{-X_{i-1}}^x e^{-\tilde{\alpha}(x-y)} e^{\nu(\lambda)(x-y)} |e^{-\tilde{\alpha}y}\tilde{H}(y)|dy \\
&\leq C |\lambda|^2 |d| \int_{-X_{i-1}}^0 e^{(\tilde{\alpha}-\nu(\lambda))y} |e^{-\tilde{\alpha}y}\tilde{H}(y)|dy \\
&\leq C |\lambda|^2 |d|
\end{align*}

since $|e^{-\tilde{\alpha}y}\tilde{H}(y)|$ is bounded (decay rate of $\tilde{H}$ is known) and $|\nu(\lambda)| < \tilde{\alpha}$. Thus for $L_2$ we have the overall bound

\[
|L_2(\lambda)(a,b,c,d)| \leq C (|a| + |b| + e^{\nu(\lambda)X_1}|c_1| + e^{\nu(\lambda)X_2}|c_2| + |\lambda|^2 |d| )
\]

Now we do the inversion like we have done in the past, and as in Sandstede (1998). Omitting the details of this for now, we can invert the expression $(I - L_1(\lambda))W = L_2(\lambda)(a,b,c,d)$ to get $W = W_1(\lambda)(a,b,c,d)$, where we have the bound

\[
||W_1(\lambda)(a,b,c,d)|| \leq C (|a| + |b| + e^{\nu(\lambda)X_1}|c_1| + e^{\nu(\lambda)X_2}|c_2| + |\lambda|^2 |d| )
\]

Note that by symmetry, for $i = 1, 2$ this is always of the form

\[
||W_1(\lambda)(a,b,c,d)|| \leq C (|a| + |b| + e^{\nu(\lambda)X_i}|c_i| + e^{\nu(\lambda)X_{i-1}}|c_{i-1}| + |\lambda|^2 |d| )
\]


\item Solve for the joins which are not at 0, i.e. solve

\begin{align*}
W_2^+(X_2) - W_1^-(-X_2) &= D_2 d \\
W_1^+(X_1) - W_2^-(-X_1) &= D_1 d \\
\end{align*}

To solve these, we have two equations for $i = 1, 2$. Using our notation convention for ``wrapping around'', we start with

\begin{align*}
W_i^+(X_i) &- W_{i-1}^-(-X_i) = P^u_+(X_i; \lambda) a_i^+ - P^s_-(-X_i; \lambda) a_i^- \\
&+ \Phi^s_+(X_i, 0; \lambda)b_i^+ - \Phi^u_-(-X_i, 0; \lambda)b_{i-1}^- \\
&+ v_+(X_i; \lambda) \langle v_0(\lambda), w_+(X_i; \lambda) \rangle c_i^+ - v_-(-X_i; \lambda) \langle v_0(\lambda), w_-(-X_i; \lambda) \rangle c_i^- \\
&+ \int_0^{-X_i} \Phi^u_+(-X_i, y; \lambda) [ G(\lambda)W_i^+(y) + d_i \lambda^2 \tilde{H}(y) ] dy \\
&- \int_0^{X_i} \Phi^s_-(X_i, y; \lambda) [ G(\lambda)W_{i-1}^-(y) + d_{i-1} \lambda^2 \tilde{H}(y) ] dy
\end{align*}

First, we get the coefficients $a_i^\pm$ by themselves by adding and subtracting $P_0^u a_i^+$ and $P_0^s a_i^-$. Recalling where the various $a_i^\pm$ live and what happens when we hit them with the projections on $E^u$ and $E^s$, this becomes

\begin{align*}
D_i d &= a_i^+ - a_i^- \\
&+ (P^u_+(X_i; \lambda) - P_0^u)a_i^+ - (P^s_-(-X_i; \lambda) - P_0^s)a_i^- \\
&+ \Phi^s_+(X_i, 0; \lambda)b_i^+ - \Phi^u_-(-X_i, 0; \lambda)b_{i-1}^- \\
&+ v_+(X_i; \lambda) \langle v_0(\lambda), w_+(X_i; \lambda) \rangle c_i^+ - v_-(-X_i; \lambda) \langle v_0(\lambda), w_-(-X_i; \lambda) \rangle c_i^- \\
&+ \int_0^{-X_i} \Phi^u_+(-X_i, y; \lambda) [ G(\lambda)W_i^+(y) + d_i \lambda^2 \tilde{H}(y) ] dy \\
&- \int_0^{X_i} \Phi^s_-(X_i, y; \lambda) [ G(\lambda)W_{i-1}^-(y) + d_{i-1} \lambda^2 \tilde{H}(y) ] dy
\end{align*}

For a bound on the ``projection difference'', let

\[
p_1(X;\lambda) = \sup_{x \geq X} (|P^u(x;\lambda) - P_0^u| + |P^s(-X;\lambda) - P_0^s|)
\]

Note that this will change with both $\lambda$ and with $X$, but the effect will not be multiplicative (since this will still happen if we take $\lambda = 0$. Thus this should be order $e^{-\alpha T} + |\lambda|$. These two things should (eventually) be of the same order, so that is ok.
\\

As in the (revised) single pulse version, we will only solve for the $a_i^\pm$ at this stage. To see why this makes sense, we will rearrange the above slightly to get

\begin{align*}
a_i^- - a_i^+ &= -D_i d  \\
&+ (P^u_+(X_i; \lambda) - P_0^u)a_i^+ - (P^s_-(-X_i; \lambda) - P_0^s)a_i^- \\
&+ \Phi^s_+(X_i, 0; \lambda)b_i^+ - \Phi^u_-(-X_i, 0; \lambda)b_{i-1}^- \\
&+ v_+(X_i; \lambda) \langle v_0(\lambda), w_+(X_i; \lambda) \rangle c_i^+ - v_-(-X_i; \lambda) \langle v_0(\lambda), w_-(-X_i; \lambda) \rangle c_i^- \\
&+ \int_0^{-X_i} \Phi^u_+(-X_i, y; \lambda) [ G(\lambda)W_i^+(y) + d_i \lambda^2 \tilde{H}(y) ] dy \\
&- \int_0^{X_i} \Phi^s_-(X_i, y; \lambda) [ G(\lambda)W_{i-1}^-(y) + d_{i-1} \lambda^2 \tilde{H}(y) ] dy
\end{align*}

The idea is the following. The LHS $a_i^- - a_i^+$ lives in $E^+ \oplus E^-$, thus the RHS must as well. But we know that $C^5 = E^+ \oplus E^- \oplus E^0$, which means the RHS has no component in the (unperturbed) center space. On the RHS we have the vectors $v_\pm(\pm X_i; \lambda)$, which are approximately (for large $X_i$) $v_0(\lambda)$, which is approximately a vector $v_0(0)$ in $E^0$. The second approximation is order $\lambda$. The first approximation has no order in terms of stuff we care about, but we can take $X_i$ sufficiently large to make it order $\lambda$. Alternately we could probably ``straighten out'' things in a neighborhood of the origin so that the first approximation is exact. In any case, we will assume that for sufficiently large $X$

\[
p_7(\lambda; X) = |v_\pm(\pm X_i; \lambda) - v_0(0)| 
\]

is order $\lambda$. Then we can hit the equation above with the projection $P_0^\pm$ onto $E^s \oplus E^u$ which leaves the LHS unchanged but eliminates ``most'' of $v_\pm(\pm X_i; \lambda)$. First we write this as

\begin{align*}
a_i^- &- a_i^+ = -D_i d + (P^u_+(X_i; \lambda) - P_0^u)a_i^+ - (P^s_-(-X_i; \lambda) - P_0^s)a_i^- \\
&+ v_0(0) \langle v_0(\lambda), w_+(X_i; \lambda) \rangle c_i^+ 
- v_0(0) \langle v_0(\lambda), w_-(-X_i; \lambda) \rangle c_i^- \\
&+ \Phi^s_+(X_i, 0; \lambda)b_i^+ - \Phi^u_-(-X_i, 0; \lambda)b_{i-1}^- \\
&+ (v_0(0) - v_+(X_i; \lambda)) \langle v_0(\lambda), w_+(X_i; \lambda) \rangle c_i^+ \\
&- (v_0(0) - v_-(-X_i; \lambda)) \langle v_0(\lambda), w_-(-X_i; \lambda) \rangle c_i^- \\
&+ \int_0^{-X_i} \Phi^u_+(-X_i, y; \lambda) [ G(\lambda)W_i^+(y) + d_i \lambda^2 \tilde{H}(y) ] dy \\
&- \int_0^{X_i} \Phi^s_-(X_i, y; \lambda) [ G(\lambda)W_{i-1}^-(y) + d_{i-1} \lambda^2 \tilde{H}(y) ] dy  
\end{align*}

When we hit this with $P_0^\pm$, the $v_0(0)$ terms disappear, which leaves us with

\begin{align*}
a_i^- &- a_i^+ = P_0^\pm \Big(-D_i d \\
&+(v_+(X_i; \lambda) - v_0(0)) \langle v_0(\lambda), w_+(X_i; \lambda) \rangle c_i^+ \\
&- (v_-(-X_i; \lambda) - v_0(0)) \langle v_0(\lambda), w_-(-X_i; \lambda) \rangle c_i^- \\
&+ (P^u_+(X_i; \lambda) - P_0^u)a_i^+ - (P^s_-(-X_i; \lambda) - P_0^s)a_i^- \\
&+ \Phi^s_+(X_i, 0; \lambda)b_i^+ - \Phi^u_-(-X_i, 0; \lambda)b_{i-1}^- \\
&+ \int_0^{-X_i} \Phi^u_+(-X_i, y; \lambda) [ G(\lambda)W_i^+(y) + d_i \lambda^2 \tilde{H}(y) ] dy \\
&- \int_0^{X_i} \Phi^s_-(X_i, y; \lambda) [ G(\lambda)W_{i-1}^-(y) + d_{i-1} \lambda^2 \tilde{H}(y) ] dy \Big)
\end{align*}

Then we have

\begin{align*}
a_i^- - a_i^+ = -P_0^\pm D_i d + L_3(\lambda)_i(a, b, c, d)
\end{align*}

Where $L_3(\lambda)_i(a, b, c, d)$ is the rest of the stuff on the RHS.

\begin{align*}
L_3(\lambda)_i&(a, b, c, d) = P_0^\pm \Big( \\
&+(v_+(X_i; \lambda) - v_0(0)) \langle v_0(\lambda), w_+(X_i; \lambda) \rangle c_i^+ \\
&- (v_-(-X_i; \lambda) - v_0(0)) \langle v_0(\lambda), w_-(-X_i; \lambda) \rangle c_i^- \\
&+ (P^u_+(X_i; \lambda) - P_0^u)a_i^+ - (P^s_-(-X_i; \lambda) - P_0^s)a_i^- \\
&+ \Phi^s_+(X_i, 0; \lambda)b_i^+ - \Phi^u_-(-X_i, 0; \lambda)b_{i-1}^- \\
&+ \int_0^{-X_i} \Phi^u_+(-X_i, y; \lambda) [ G(\lambda)W_i^+(y) + d_i \lambda^2 \tilde{H}(y) ] dy \\
&- \int_0^{X_i} \Phi^s_-(X_i, y; \lambda) [ G(\lambda)W_{i-1}^-(y) + d_{i-1} \lambda^2 \tilde{H}(y) ] dy \Big)
\end{align*}

For the bound on $L_3$, we will again use the $\tilde{\alpha}$ trick to get a better bound for the term involving $\tilde{H}$. We can do this since we know the decay rate of $\tilde{H}$.

\begin{align*}
\left| \int_0^{X_i} \Phi^s_+(X_i, y; \lambda) \tilde{H}(y) dy \right| 
&\leq C \int_0^{X_i} e^{-\alpha (X_i - y)}|\tilde{H}(y)| dy \\
&= C e^{-\tilde{\alpha}X_i} \int_0^{X_i} e^{-\alpha X_i} e^{\alpha y}  e^{\tilde{\alpha}X_i} e^{-\tilde{\alpha}y} |e^{\tilde{\alpha}y} \tilde{H}(y)| \\
&= C e^{-\tilde{\alpha}X_i} \int_0^{X_i} e^{-(\alpha - \tilde{\alpha})(X_i-y)} |e^{\tilde{\alpha}y} \tilde{H}(y)|\\
&\leq C e^{-\tilde{\alpha}X_i} 
\end{align*}

where we again used the fact that $|e^{\tilde{\alpha}y} \tilde{H}(y)|$ is bounded, which holds since $\tilde{\alpha}$ is smaller than $\alpha$ and $\tilde{H}(y)$ decays with rate $\alpha$.\\

Thus we have the following bound on $L_3$.

\[
L_3(\lambda)_i(a, b, c, d) \leq C ( p_1(X_i; \lambda)|a_i|
+ e^{-\alpha X_i}|b| + p_7(X_i; \lambda)|c_i| + |G| ||W|| + e^{-\tilde{\alpha} X_i} |\lambda^2| |d| )
\]

Note that so far this involves only $a_i$, whereas it involves both $b_i$ and $b_{i-1}$. Unfortunately, things will not stay that way, since both $W_1$ and $W_2$ are involved in $L_3(\lambda)_i(a, b, c, d)$. Thus via $W$ (through the $||W||$ bound), we involve the other subscript for $a$. Plugging in the bound on $W_1$ for $W$, we have

\begin{align*}
|L_3&(\lambda)_i(a, b, c, d)| \\
&\leq C \Big( p_1(X_i; \lambda)|a_i|
+ e^{-\alpha X_i}|b| + p_7(X_i; \lambda)|c_i| + |G|\:||W_1(\lambda)(a,b,c,d)|| + e^{-\tilde{\alpha} X_i} |\lambda^2| |d| \Big) \\
& \leq C \Big( p_1(X_i; \lambda)|a_i|
+ e^{-\alpha X_i}|b| + p_7(X_i; \lambda)|c_i| + e^{-\tilde{\alpha} X_i} |\lambda^2| |d| \\
&+ |G| (|a| + |b| + e^{\nu(\lambda)X_i}|c_i| + e^{\nu(\lambda)X_{i-1}}|c_{i-1}| + |\lambda|^2 |d| \Big) \\
& \leq C \Big( (p_1(X_i; \lambda) + |G|)|a_i| + |G||a_{i-1}| + (e^{-\alpha X_i} + |G|) |b| \\
&+ ( p_7(X_i; \lambda) + e^{\nu(\lambda)X_i} |G|) |c_i| + e^{\nu(\lambda)X_{i-1}} |G| |c_{i-1}| \\
&+ (e^{-\tilde{\alpha} X_i} + |G|) |\lambda|^2 |d| \Big)
\end{align*} 

At this point we will take $X_2 \geq X_1$ without loss of generality. We can do this since it does not matter where the ``center'' of the wavetrain is since it is periodic. Thus we will have $X_M = X_2$ and $X_m = X_1$. \\

At this point, we have a small complication. Since we have (or are assuming!) that $|G|$ is order $e^{-\alpha X_m} = e^{-\alpha X_1}$, and since we have both $e^{\nu(\lambda)X_i} |G|$ and $e^{\nu(\lambda)X_{i-1}} |G|$ terms, one of these will have a term which is order $e^{\nu(\lambda)X_2} e^{-\alpha X_1}$. The simplest way to do this is to have $X_2$ depend on $X_1$. THERE MIGHT BE ANOTHER WAY TO DO THIS SINCE $G$ IS A FUNCITON OF $x$, BUT WE WILL KEEP IT SIMPLE FOR NOW. Let's do the easy thing and let

\[
X_2 = k X_1
\]

where $k \geq 1$. Then for the inversion to work, we need the following:

\begin{enumerate}[(i)]
\item Choose $\delta > 0$ small
\item Choose $0 < \tilde{\delta} \leq \delta$ such that for all $|\lambda| < \tilde{\delta}$ we also have $|\nu{\lambda}| \leq \delta$. Since $\nu(\lambda)$ has order $\lambda$, this is possible.
\item Choose $k$ so that $\delta k < \alpha$. 
\item This implies $\alpha - \nu(\lambda) k > 0$ for all $|\lambda| < \tilde{\delta}$
\item In particular, this means that all the $e^{\nu(\lambda)k X_1} |G|$ terms are uniformly bounded and decay in $X_1$
\item Choose $X_1$ sufficiently large so that $p_2(X_i; \lambda) < \delta$.
\end{enumerate}

Assume we have done this. For simplicity, we will often leave these bounds in terms of the subscript $i$, but these conditions are necessary for the bounds to make sense in the first place. Thus we have the bound 

\begin{align*}
|L_3&(\lambda)_i(a, b, c, d)| \\
&\leq C \Big( (p_1(X_i; \lambda) + |G|)|a_i| + |G||a_{i-1}| + (e^{-\alpha X_i} + |G|) |b| \\
&+ ( p_7(X_i; \lambda) + e^{\nu(\lambda)X_i} |G|) |c_i| + e^{\nu(\lambda)X_{i-1}} |G| |c_{i-1}| \\
&+ (e^{-\tilde{\alpha} X_i} + |G|) |\lambda|^2 |d| \Big)
\end{align*} 

Let $J_1: V_a \rightarrow \C^4$ be defined by $J_i(a_i) = (a_i^+ - a_i^-)$. The map $J_i$ is a linear isomorphism. Now consider the map

\[
S_i(a_i) = J_i (a_i) + L_3(\lambda)_i(a_i, 0, 0, 0) = J_i( I + J_i^{-1} L_3(\lambda)_i(a_i, 0, 0, 0))
\]

For suffiently small $\delta$, by what we have above, we can get the operator norm 

\[
J_i^{-1} L_3(\lambda)_i(\cdot, 0, 0, 0)|| < 1
\]

thus the map $a_i \rightarrow I + J_1^{-1} L_3(\lambda)_i(a_i, 0, 0, 0)$ is invertible and so the operator $S_i$ is invertible.\\

Thus we can solve for $a$ to get

\[
a_i = A_1(\lambda)_i(b, c, d) = S_i^{-1}(-D_i d - L_3(\lambda)_i(0, b, c, d))
\]

Using the bound on $L_3$ together with $|D_i|$, $A_1$ will have bound

\begin{align*}
|A_1&(\lambda)_i(b, c, d)| \\
&\leq C \Big( (e^{-\alpha X_i} + |G|) |b| 
+ ( p_7(X_i; \lambda) + e^{\nu(\lambda)X_i} |G|) |c_i| + e^{\nu(\lambda)X_{i-1}} |G| |c_{i-1}| \\
&+ (e^{-\tilde{\alpha} X_i} + |G|) |\lambda|^2 |d| + |D_i||d| \Big)
\end{align*} 

This bound is similar in form to (3.24) in Sanstede (1998) with the exceptions of: an additional $|c^-|$ since we have a ``center'' space in play; and a better bound for the $|\lambda|^2$ term, thanks to the $\tilde{\alpha}$ trick. The $|D| |d| $ term is by itself, as in Sanstede (1998). I think this bound is the best we can get at this point.

We can plug this into our expression for $W_1$ to get $W_2(\lambda)$.

\begin{align*}
||W_2&(\lambda)(b,c,d)|| \\
&\leq C \Big(|a| + |b| + e^{\nu(\lambda)X_i}|c_i| + e^{\nu(\lambda)X_{i-1}}|c_{i-1}| + |\lambda|^2 |d| \Big) \\
&\leq C \Big(|a_i| + |a_{i-1}| + |b| + e^{\nu(\lambda)X_i}|c_i| + e^{\nu(\lambda)X_{i-1}}|c_{i-1}| + |\lambda|^2 |d| \Big) \\
&\leq C \Big(|A_1(\lambda)_i(b,c,d)| + |A_1(\lambda)_{i-1}(b,c,d)| + |b| + e^{\nu(\lambda)X_i}|c_i| + e^{\nu(\lambda)X_{i-1}}|c_{i-1}| + |\lambda|^2 |d| \Big) \\
&\leq C \Big( ((e^{-\alpha X_i} + |G|) |b| 
+ ( p_7(X_i; \lambda) + e^{\nu(\lambda)X_i} |G|) |c_i| + e^{\nu(\lambda)X_{i-1}} |G| |c_{i-1}| \\
&+ (e^{-\tilde{\alpha} X_i} + |G|) |\lambda|^2 |d| + |D_i||d| )\\
&+ ((e^{-\alpha X_{i-1}} + |G|) |b| 
+ ( p_7(X_{i-1}; \lambda) + e^{\nu(\lambda)X_{i-1}} |G|) |c_{i-1}| + e^{\nu(\lambda)X_i} |G| |c_i| \\
&+ (e^{-\tilde{\alpha} X_{i-1}} + |G|) |\lambda|^2 |d| + |D_{i-1}||d| )\\
&+ |b| + e^{\nu(\lambda)X_i}|c_i| + e^{\nu(\lambda)X_{i-1}}|c_{i-1}| + |\lambda|^2 |d| \Big) \\
&\leq C \Big( (1 + e^{-\alpha X_i} + e^{-\alpha X_{i-1}} + |G|)|b|\\
&+ ( e^{\nu(\lambda)X_i} + p_7(X_i; \lambda) + e^{\nu(\lambda)X_i} |G| ) |c_i| \\
&+ ( e^{\nu(\lambda)X_{i-1}} + p_7(X_{i-1}; \lambda) + e^{\nu(\lambda)X_{i-1}} |G|) |c_{i-1}| \\ 
&+ (1 + e^{-\tilde{\alpha} X_i} + e^{-\tilde{\alpha} X_{i-1}} + |G|)|\lambda|^2 |d| \\
&+ |D_i||d| + |D_{i-1}||d|
\Big)
\end{align*}

So the final bound is

\begin{align*}
||W_2&(\lambda)(b,c,d)|| \\
&\leq C \Big( |b| + e^{\nu(\lambda)X_i} |c_i| + e^{\nu(\lambda)X_{i-1}} |c_{i-1}| 
+ |\lambda|^2 |d| + |D_i||d| + |D_{i-1}||d|)
\end{align*}

This is essentially the same as the $W_1$ bound, so it should be fine.\\

With a multipulse, we will need an analogue of (3.25) in Sandstede (1998). The idea here is that we hit our expression for $D_i d$ with projections to kill some of the terms. We start with

\begin{align*}
a_i^- &- a_i^+ = -D_i d + (P^u_+(X_i; \lambda) - P_0^u)a_i^+ - (P^s_-(-X_i; \lambda) - P_0^s)a_i^- \\
&+ v_0(0) \langle v_0(\lambda), w_+(X_i; \lambda) \rangle c_i^+ 
- v_0(0) \langle v_0(\lambda), w_-(-X_i; \lambda) \rangle c_i^- \\
&+ \Phi^s_+(X_i, 0; \lambda)b_i^+ - \Phi^u_-(-X_i, 0; \lambda)b_{i-1}^- \\
&+ (v_0(0) - v_+(X_i; \lambda)) \langle v_0(\lambda), w_+(X_i; \lambda) \rangle c_i^+ \\
&- (v_0(0) - v_-(-X_i; \lambda)) \langle v_0(\lambda), w_-(-X_i; \lambda) \rangle c_i^- \\
&+ \int_0^{-X_i} \Phi^u_+(-X_i, y; \lambda) [ G(\lambda)W_i^+(y) + d_i \lambda^2 \tilde{H}(y) ] dy \\
&- \int_0^{X_i} \Phi^s_-(X_i, y; \lambda) [ G(\lambda)W_{i-1}^-(y) + d_{i-1} \lambda^2 \tilde{H}(y) ] dy  
\end{align*}

This time, we take projections $P^s_0$ and $P^u_0$ individually. Recalling where the $a_i^\pm$ and $v_0(0)$ live and that $v_0(0)$ is wiped out by both projections, this becomes 

\begin{align*}
a_i^+ &= P^u_0 D_i d - P^u_0 L_3(\lambda)_i(a, b, c, d) \\
a_i^- &= -P^s_0 D_i d + P^s_0 L_3(\lambda)_i(a, b, c, d)
\end{align*}

Define $A_2$ to be all the stuff on the RHS other than the $D_i d$ term. Thus we have 

\begin{align*}
a_i^+ &= P^u_0 D_i d + A_2(\lambda)_i^+(b, c, d) \\
a_i^- &= -P^s_0 D_i d + A_2(\lambda)_i^-(b, c, d)
\end{align*}

We then can come up with a bound for $A_2$ using the bound for $L_3$ and the bound for $A_1$.

\begin{align*}
|A_2&(\lambda)_i(b, c, d)| \\
&\leq C |L_3(\lambda)_i(a, b, c, d)| \\
&\leq C \Big( (p_1(X_i; \lambda) + |G|)|A_1(\lambda)_i(b, c, d)| \\
&+ |G||A_1(\lambda)_{i-1}(b, c, d)| \\
&+ (e^{-\alpha X_i} + |G|) |b| \\
&+ ( p_7(X_i; \lambda) + e^{\nu(\lambda)X_i} |G|) |c_i| + e^{\nu(\lambda)X_{i-1}} |G| |c_{i-1}| \\
&+ (e^{-\tilde{\alpha} X_i} + |G|) |\lambda|^2 |d| \Big) \\
&\leq C \Big( (p_1(X_i; \lambda) + |G|) ((e^{-\alpha X_i} + |G|) |b| 
+ ( p_7(X_i; \lambda) + e^{\nu(\lambda)X_i} |G|) |c_i| + e^{\nu(\lambda)X_{i-1}} |G| |c_{i-1}| \\
&+ (e^{-\tilde{\alpha} X_i} + |G|) |\lambda|^2 |d| + |D_i||d| )\\
&+ |G|((e^{-\alpha X_{i-1}} + |G|) |b| 
+ ( p_7(X_{i-1}; \lambda) + e^{\nu(\lambda)X_{i-1}} |G|) |c_{i-1}| + e^{\nu(\lambda)X_i} |G| |c_i| \\
&+ (e^{-\tilde{\alpha} X_{i-1}} + |G|) |\lambda|^2 |d| + |D_{i-1}||d| ) \\
&+ (e^{-\alpha X_i} + |G|) |b| \\
&+ ( p_7(X_i; \lambda) + e^{\nu(\lambda)X_i} |G|) |c_i| + e^{\nu(\lambda)X_{i-1}} |G| |c_{i-1}| \\
&+ (e^{-\tilde{\alpha} X_i} + |G|) |\lambda|^2 |d| \Big)
\end{align*} 

Collecting terms and dropping higher order terms, this becomes

\begin{align*}
|A_2&(\lambda)_i(b, c, d)| \\
&\leq C \Big( (p_1(X_i; \lambda) + |G|) ((e^{-\alpha X_i} + |G|) |b| 
+ ( p_7(X_i; \lambda) + e^{\nu(\lambda)X_i} |G|) |c_i| + e^{\nu(\lambda)X_{i-1}} |G| |c_{i-1}| \\
&+ (e^{-\tilde{\alpha} X_i} + |G|) |\lambda|^2 |d| + |D_i||d| )\\
&+ |G|((e^{-\alpha X_{i-1}} + |G|) |b| 
+ ( p_7(X_{i-1}; \lambda) + e^{\nu(\lambda)X_{i-1}} |G|) |c_{i-1}| + e^{\nu(\lambda)X_i} |G| |c_i| \\
&+ (e^{-\tilde{\alpha} X_{i-1}} + |G|) |\lambda|^2 |d| + |D_{i-1}||d| ) \\
&+ (e^{-\alpha X_i} + |G|) |b| \\
&+ ( p_7(X_i; \lambda) + e^{\nu(\lambda)X_i} |G|) |c_i| + e^{\nu(\lambda)X_{i-1}} |G| |c_{i-1}| \\
&+ (e^{-\tilde{\alpha} X_i} + |G|) |\lambda|^2 |d| \Big) \\
&\leq C \Big( ((p_1(X_i; \lambda) + |G| + 1) (e^{-\alpha X_i} + |G|) +
|G|(e^{-\alpha X_{i-1}} + |G|) ) |b| \\
&+ (p_1(X_i; \lambda) + |G| + 1)(p_7(X_i; \lambda) + e^{\nu(\lambda)X_i} |G|) |c_i| \\
&+ ((p_1(X_i; \lambda) + |G| + 1)e^{\nu(\lambda)X_{i-1}} |G| |c_{i-1}| \\
&+ ((p_1(X_i; \lambda) + |G|)(e^{-\tilde{\alpha} X_i} + |G|) 
+ |G|(e^{-\tilde{\alpha} X_{i-1}} + |G|) + (e^{-\tilde{\alpha} X_i} + |G|) )|\lambda|^2 |d| \\
&+ (p_1(X_i; \lambda) + |G|) |D_i| |d| \\
&+ |G| |D_{i-1}| |d| \Big)
\end{align*} 

Thus the final bound is

\begin{align*}
|A_2&(\lambda)_i(b, c, d)| \\
&\leq \Big( (e^{-\alpha X_i} + |G| )|b| \\
&+ (p_7(X_i; \lambda) + e^{\nu(\lambda)X_i} |G|) |c_i| + e^{\nu(\lambda)X_{i-1}} |G| |c_{i-1}| \\
&+ (e^{-\tilde{\alpha} X_i} + |G|)|\lambda|^2 |d| 
+ (p_1(X_i; \lambda) + |G|) |D_i| |d| 
+ |G| |D_{i-1}| |d| \Big)
\end{align*}

\item Next we want to satisfy the conditions

\begin{align*}
P(\C Q'(0))W_i^-(0) &= 0 \\
P(\C Q'(0))W_i^+(0) &= 0 \\
P(Y^+ \oplus Y^- \oplus Y^0) ( W_i^+(0) - W_i^-(0) ) &= 0
\end{align*}

Since the stable and unstable range spaces at $\lambda = 0$ both contain $\C Q'(0)$, we can decompose $b^\pm$ uniquely as $b^\pm = x^\pm + y^\pm$, where $x^\pm \in \C Q'(0)$ and $y^\pm \in Y^\pm$. Then since

\begin{equation}\label{directsum}
\C^n = \C\Psi(0) \oplus \C Q'(0) \oplus Y^+ \oplus Y^- \oplus Y^0
\end{equation}

(each of these is 1D in this case), the conditions above are equivalent to the following projections

\begin{align*}
P(\C Q'(0) \oplus Y^0 )W^-(0) &= 0 \\
P(\C Q'(0) \oplus Y^0 )W^+(0) &= 0 \\
P(Y^+ \oplus Y^-) (W^+(0) - W^-(0) ) &= 0
\end{align*}

where the range of each projection is indicated, and the kernel of each projection is just the other elements of the direct sum \eqref{directsum}. Since the first two equations wipe out any component in $\C Q'(0) \oplus Y^0$, we don't need to put that in the range of the third projection. \\

Let $y_0 = v_\pm(0; 0)$ be a unit vector for $Y^0$. Since there is only a small order $\lambda$ perturbation when we go from $y_0$ to $v_\pm(0; \lambda)$ the following two direct sums hold as well.

\begin{equation}\label{directsum}
\C^n = \C\Psi(0) \oplus \C Q'(0) \oplus Y^+ \oplus Y^- \oplus v_\pm(0; \lambda)
\end{equation}

(It is two direct sums because we can choose either of $v_\pm(0; \lambda)$ for the last component).\\

So we should be able to use the following projections instead

\begin{align*}
P(\C Q'(0) \oplus \C v_-(0; \lambda) )W^-(0) &= 0 \\
P(\C Q'(0) \oplus \C v_+(0; \lambda) )W^+(0) &= 0 \\
P(Y^+ \oplus Y^-) (W^+(0) - W^-(0) ) &= 0
\end{align*}

We can write this as five projections
\begin{align*}
P(\C Q'(0) )W^-(0) &= 0 \\
P(\C Q'(0) )W^+(0) &= 0 \\
P(\C v_-(0; \lambda))W^-(0) &= 0 \\
P(\C v_+(0; \lambda))W^+(0) &= 0 \\
P(Y^+ \oplus Y^-) (W^+(0) - W^-(0) ) &= 0
\end{align*}

which as the advantage of separating out the various coefficients in a nice way. It also eliminates any center component of $W^\pm(0)$ in the perturbed version, and I think we want to do that since we want the eigenfunction to be localized.\\

At $x = 0$, the fixed point equations become

\begin{align*}
W_i^-(0) = \Phi^s_-(&0, -X_{i-1}; \lambda)a_{i-1}^- + b_i^- + (P^u_-(0; \lambda) - P^u_-(0; 0))b_i^- \\
&+ e^{\nu(\lambda)X_{i-1}} v_-(0; \lambda) \langle v_0(\lambda), w_-(-X_{i-1}; \lambda) \rangle c_{i-1}^- \\
&+ \int_{-X_{i-1}}^0 \Phi^s_-(0, y; \lambda) [ G_i^-(\lambda)W_i^-(y) + \lambda^2 d_i \tilde{H}(y) ] dy \\
&+ \int_{-X_{i-1}}^0
e^{\nu(\lambda)y} v_-(0; \lambda) \langle G_i^-(\lambda)(y)W_i^-(y) + \lambda^2 d_i \tilde{H}(y), w_-(y; \lambda) \rangle dy \\
W_i^+(0) = \Phi^u_+(&0, X_i; \lambda)a_i^+ + b_i^+ + (P^s_+(0; \lambda) - P^s_-(0; 0))b_i^+ \\
&+ e^{-\nu(\lambda) X_i} v_+(0; \lambda) \langle v_0(\lambda), w_+(X_i; \lambda) \rangle c_i^+ \\
&+ \int_{X_i}^0 \Phi^u_+(0, y; \lambda) [ G_i^+(\lambda)W_i^+(y) + \lambda^2 d_i \tilde{H}(y) ] dy \\
&+ \int_{X_i}^0 e^{-\nu(\lambda)y} v_+(0; \lambda) \langle G_i^+(\lambda)(y)W_i^+(y) + \lambda^2 d_i \tilde{H}(y), w_+(y; \lambda) \rangle dy
\end{align*}

where we have added and subtracted $P^s_-(0; 0))b_i^+$ and $P^u_-(0; 0))b_i^-$ since we want the $b_i$ to disappear when we take the projection. We will have a bound

\[
p_3(\lambda) = |P^u_-(0;\lambda) - P^u_-(0; 0)| + |P^s_+(0;\lambda) - P^s_+(0;0)|
\]

which should be order $|\lambda|$.\\

Before we start hitting things with all sorts of projections, we need to get the $c_i^\pm$ term into a form we can deal with. 

\begin{align*}
e^{\nu(\lambda)X_{i-1}} &v_-(0; \lambda) \langle v_0(\lambda), w_-(-X_{i-1}; \lambda) \rangle c_{i-1}^- \\
&= e^{\nu(\lambda)X_{i-1}} v_-(0; \lambda) \langle v_0(\lambda), w_0(\lambda) \rangle c_{i-1}^- + e^{\nu(\lambda)X_{i-1}} v_-(0; \lambda) \langle v_0(\lambda), \Delta w_-(-X_{i-1}; \lambda) \rangle c_{i-1}^- \\
&= e^{\nu(\lambda)X_{i-1}} v_-(0; \lambda) c_{i-1}^- + e^{\nu(\lambda)X_{i-1}} v_-(0; \lambda) \langle v_0(\lambda), \Delta w_-(-X_{i-1}; \lambda) \rangle c_{i-1}^-
\end{align*}

where 

\begin{align*}
\Delta v_\pm(x; \lambda) &= v_\pm(x; \lambda) - v_0(\lambda) \\
\Delta w_\pm(x; \lambda) &= w_\pm(x; \lambda) - w_0(\lambda)
\end{align*}

These things approach 0 as $x \rightarrow \pm \infty$, but we do not have a rate of convergence since we pulled out the exponential factor above when we defined these things.\\

Similarly, we have

\begin{align*}
e^{-\nu(\lambda)X_i} &v_+(0; \lambda) \langle v_0(\lambda), w_+(X_i; \lambda) \rangle c_i^+ \\
&= e^{-\nu(\lambda)X_i} v_-(0; \lambda) \langle v_0(\lambda), w_0(\lambda) \rangle c_i^- + e^{-\nu(\lambda)X_i} v_+(0; \lambda) \langle v_0(\lambda), \Delta w_+(X_i; \lambda) \rangle c_i^+ \\
&= e^{-\nu(\lambda)X_i} v_+(0; \lambda) c_i^- + e^{-\nu(\lambda)X_i} v_+(0; \lambda) \langle v_0(\lambda), \Delta w_+(X_i; \lambda) \rangle c_i^+
\end{align*}

Next we note that the $c_i^\pm$ have coefficients like $e^{\pm \nu(\lambda) }X_i$. One of these is potentially bad, but we do not know which one. To get around this, we will add and subtract the other version. \\

To make this less messy, we define the following functions scalar functions of $X$ and $\lambda \neq 0$. (We know that $\nu(\lambda) \neq 0$ for $\lambda \neq 0$).

\begin{align*}
f(X; \lambda) = e^{\nu(\lambda)X} + e^{-\nu(\lambda)X} &= 2 \cosh (\nu(\lambda) X) \\
f^-(X; \lambda) &= e^{-\nu(\lambda)X} / f(X; \lambda) \\
f^+(X; \lambda) &= e^{\nu(\lambda)X} / f(X; \lambda)
\end{align*}

Note that we always have $|f^\pm(X; \lambda)| < 1$.\\

Using these, we have

\begin{align*}
e^{\nu(\lambda)X_{i-1}} c_{i-1}^- &= (e^{\nu(\lambda)X_{i-1}} + e^{-\nu(\lambda)X_{i-1}})c_{i-1}^- - e^{-\nu(\lambda)X_{i-1}})c_{i-1}^- \\
&= f(X_{i-1}; \lambda) c_{i-1}^- - f^-(X_{i-1}; \lambda)f(X_{i-1}; \lambda) c_{i-1}^-
\end{align*}

Similarly,

\begin{align*}
e^{-\nu(\lambda)X_i} c_i^+ &= f(X_i; \lambda) c_i^+ - f^+(X_i; \lambda)f(X_i; \lambda) c_i^+
\end{align*}

Plugging all of these into the fixed point equation at $x = 0$, we have

\begin{align*}
W_i^-(0) &= x_i^- + y_i^- + v_-(0; \lambda) f(X_{i-1}; \lambda) c_{i-1}^-  \\
&+\Phi^s_-(0, -X_{i-1}; \lambda)a_{i-1}^- + b_i^- + (P^u_-(0; \lambda) - P^u_-(0; 0))b_i^- \\
&- f^-(X_{i-1}; \lambda) v_-(0; \lambda) f(X_{i-1}; \lambda) c_{i-1}^- \\
&+ \langle v_0(\lambda), \Delta w_-(-X_{i-1}; \lambda) \rangle 
f^+(X_{i-1}; \lambda) v_-(0; \lambda) f(X_{i-1}; \lambda) c_{i-1}^- \\
&+ \int_{-X_{i-1}}^0 \Phi^s_-(0, y; \lambda) [ G_i^-(\lambda)W_i^-(y) + \lambda^2 d_i \tilde{H}(y) ] dy \\
&+ \int_{-X_{i-1}}^0
e^{\nu(\lambda)y} v_-(0; \lambda) \langle G_i^-(\lambda)(y)W_i^-(y) + \lambda^2 d_i \tilde{H}(y), w_-(y; \lambda) \rangle dy \\
W_i^+(0) &= x_i^+ + y_i^+ + v_+(0; \lambda) f(X_i; \lambda) c_i^+ \\
&+\Phi^u_+(0, X_i; \lambda)a_i^+ + b_i^+ + (P^s_+(0; \lambda) - P^s_-(0; 0))b_i^+ \\
&- f^+(X_i; \lambda) v_+(0; \lambda) f(X_i; \lambda) c_i^+ \\
&+ \langle v_0(\lambda), \Delta w_+(X_i; \lambda) \rangle 
f^-(X_i; \lambda)f(X_i; \lambda) v_+(0; \lambda) c_i^+ \\
&+ \int_{X_i}^0 \Phi^u_+(0, y; \lambda) [ G_i^+(\lambda)W_i^+(y) + \lambda^2 d_i \tilde{H}(y) ] dy \\
&+ \int_{X_i}^0 e^{-\nu(\lambda)y} v_+(0; \lambda) \langle G_i^+(\lambda)(y)W_i^+(y) + \lambda^2 d_i \tilde{H}(y), w_+(y; \lambda) \rangle dy
\end{align*}

If we plug in the fixed point equations into the above set of projections, we get the matrix equation

\[
\begin{pmatrix}x_i^- \\ x_i^+ \\ 
f(X_{i-1}; \lambda) v_-(0; \lambda) c_{i-1}^- \\
f(X_i; \lambda) v_+(0; \lambda) c_i^+ \\
y_i^+ - y_i^- \end{pmatrix} + L_4(\lambda)(b, c) = 0
\]


where $L_4(\lambda)(b, c)$ is the rest of the terms that don't get eliminated outright by the projections.\\

We have the following bound on $L_4(\lambda)$. 

\begin{align*}
|L_4&(\lambda)_i(b, c, d)|\\ 
&\leq C \Big( e^{-\alpha X_i} |a_i^+| +  e^{-\alpha X_{i-1}} |a_{i-1}^-| + p_3(\lambda) |b_i| \\
&+ (f^+(T; \lambda) p_2(X_{i-1}; \lambda) + f^-(X_{i-1}; \lambda)) v_-(0; \lambda)  f(X_{i-1}; \lambda) |c_{i-1}^-| \\
&+ (f^-(X_i; \lambda) p_2(X_i; \lambda) + f^+(X_i; \lambda)) v_+(0; \lambda) f(X_i; \lambda)  |c_i^+| \\
&+ (e^{\nu(\lambda)X_i} + e^{\nu(\lambda)X_{i-1}}) |G| ||W|| + |\lambda^2| |d| \Big)
\end{align*}

where

\begin{align*}
p_2(X; \lambda) &= |\Delta v_\pm(\pm X, \lambda)| + |\Delta w_\pm(\pm X, \lambda)|\\
&= |v_\pm(\pm X; \lambda) - v_0(\lambda)| + |w_\pm(\pm X; \lambda) - w_0(\lambda)|
\end{align*}

To finish the bound, we need to plug in $A_1$ and $W_2$ for $|a|$ and $||W||$.

\begin{align*}
|L_4&(\lambda)_i(b, c, d)|\\ 
&\leq C \Big( e^{-\alpha X_i} |A_1(\lambda)_i(b, c, d)| +  e^{-\alpha X_{i-1}} |A_1(\lambda)_{i-1}(b, c, d)| + p_3(\lambda) |b_i| \\
&+ (f^+(T; \lambda) p_2(X_{i-1}; \lambda) + f^-(X_{i-1}; \lambda)) v_-(0; \lambda)  f(X_{i-1}; \lambda) |c_{i-1}^-| \\
&+ (f^-(X_i; \lambda) p_2(X_i; \lambda) + f^+(X_i; \lambda)) v_+(0; \lambda) f(X_i; \lambda)  |c_i^+| \\
&+ (e^{\nu(\lambda)X_i} + e^{\nu(\lambda)X_{i-1}}) |G| ||W_2(\lambda)(b,c,d)|| + |\lambda^2| |d| \Big) \\
&\leq C \Big( e^{-\alpha X_i} ( (e^{-\alpha X_i} + |G|) |b| 
+ ( p_7(X_i; \lambda) + e^{\nu(\lambda)X_i} |G|) |c_i| + e^{\nu(\lambda)X_{i-1}} |G| |c_{i-1}| \\
&+ (e^{-\tilde{\alpha} X_i} + |G|) |\lambda|^2 |d| + |D_i||d| ) \\
&+ e^{-\alpha X_{i-1}} ( (e^{-\alpha X_{i-1}} + |G|) |b| 
+ ( p_7(X_{i-1}; \lambda) + e^{\nu(\lambda)X_{i-1}} |G|) |c_i| + e^{\nu(\lambda)X_i} |G| |c_i| \\
&+ (e^{-\tilde{\alpha} X_{i-1}} + |G|) |\lambda|^2 |d| + |D_{i-1}||d| ) \\
&+ p_3(\lambda) |b_i| \\
&+ (f^+(T; \lambda) p_2(X_{i-1}; \lambda) + f^-(X_{i-1}; \lambda)) v_-(0; \lambda)  f(X_{i-1}; \lambda) |c_{i-1}| \\
&+ (f^-(X_i; \lambda) p_2(X_i; \lambda) + f^+(X_i; \lambda)) v_+(0; \lambda) f(X_i; \lambda)  |c_i| \\
&+ (e^{\nu(\lambda)X_i} + e^{\nu(\lambda)X_{i-1}}) |G| ( |b| + e^{\nu(\lambda)X_i} |c_i| + e^{\nu(\lambda)X_{i-1}} |c_{i-1}| 
+ |\lambda|^2 |d| + |D_i||d| + |D_{i-1}||d|)\\ 
&+ |\lambda^2| |d| \Big) \\
\end{align*}

Collecting terms (and dropping some higher order ones), we have

\begin{align*}
|L_4&(\lambda)_i(b, c, d)|\\ 
&\leq C\Big(( p_3(\lambda) + e^{-\alpha X_i}( e^{-\alpha X_i} + |G|) + e^{-\alpha X_{i-1}}( e^{-\alpha X_{i-1}} + |G|) + (e^{\nu(\lambda)X_i} + e^{\nu(\lambda)X_{i-1}}) |G| )|b| \\ 
&+ ((f^+(X_i; \lambda) (p_2(X_i; \lambda) + (e^{\nu(\lambda)X_i} + e^{\nu(\lambda)X_{i-1}})|G|) + f^-(X_i; \lambda) ) f(X_i; \lambda) |c_i| \\
&+ ((f^-(X_{i-1}; \lambda) (p_2(X_{i-1}; \lambda) + (e^{\nu(\lambda)X_i} + e^{\nu(\lambda)X_{i-1}})|G|) + f^+(X_{i-1}; \lambda) ) f(X_{i-1}; \lambda) |c_{i-1}| \\
&+ ( e^{-\alpha X_i}(e^{-\tilde{\alpha} X_i} + |G|) + e^{-\alpha X_{i-1}}(e^{-\tilde{\alpha} X_{i-1}} + |G|) + (e^{\nu(\lambda)X_i} + e^{\nu(\lambda)X_{i-1}}) |G| + 1) |\lambda|^2 |d| \\ 
&+ (e^{-\alpha X_i} + (e^{\nu(\lambda)X_i} + e^{\nu(\lambda)X_{i-1}}) |G| ) |D_i||d| \\
&+ (e^{-\alpha X_{i-1}} + (e^{\nu(\lambda)X_i} + e^{\nu(\lambda)X_{i-1}}) |G| ) |D_{i-1}||d|\Big)  \\
\end{align*}

Dropping some more higher order terms, this becomes

\begin{align*}
|L_4&(\lambda)_i(b, c, d)|\\ 
&\leq C\Big(( p_3(\lambda) + (e^{\nu(\lambda)X_i} + e^{\nu(\lambda)X_{i-1}}) |G| )|b| \\ 
&+ ((f^+(X_i; \lambda) (p_2(X_i; \lambda) + (e^{\nu(\lambda)X_i} + e^{\nu(\lambda)X_{i-1}})|G|) + f^-(X_i; \lambda) ) f(X_i; \lambda) |c_i| \\
&+ ((f^-(X_{i-1}; \lambda) (p_2(X_{i-1}; \lambda) + (e^{\nu(\lambda)X_i} + e^{\nu(\lambda)X_{i-1}})|G|) + f^+(X_{i-1}; \lambda) ) f(X_{i-1}; \lambda) |c_{i-1}| \\
&+ (e^{-\alpha X_i} + (e^{\nu(\lambda)X_i} + e^{\nu(\lambda)X_{i-1}}) |G| ) |D_i||d| \\
&+ (e^{-\alpha X_{i-1}} + (e^{\nu(\lambda)X_i} + e^{\nu(\lambda)X_{i-1}}) |G| ) |D_{i-1}||d| \\
&+ |\lambda|^2 |d| \Big) \\
\end{align*}

To do the inversion, we need for the coefficients of $|b|$, $|f(X_i; \lambda) c_i|$, and $|f(X_{i-1}; \lambda) c_{i-1}|$ to have magnitude less than 1. For $|b|$ we are all set, since $p_3(\lambda)$ is order $\lambda$, and we can choose $\lambda$ as small as we want; all other coefficients of $|b|$ exponentially decay in $X_i$ since $|G|$ is order $e^{-\alpha X_1}$. \\

For $|f(X_i; \lambda) c_i|$, note that 

\[
f^-(X_i; \lambda) + f^+(X_i; \lambda) = 1
\]

and both terms on the LHS are positive. Since $(p_2(X_i; \lambda) + (e^{\nu(\lambda)X_i} + e^{\nu(\lambda)X_{i-1}})|G|)$ can be made as small as we want by choosing $\lambda$ sufficiently small and $X_i, X_{i-1}$ sufficiently large, we can obtain a coefficient of $|f(X_i)c_i|$ of magnitude less than 1. The same holds for $|f(X_{i-1})c_{i-1}|$. Thus we can do the inversion. Leaving out the details for now, this will give us an operator $B_1(\lambda)$ such that

\[
(b, f(X_i; \lambda) c_i, f(X_{i-1}; \lambda) c_{i-1} ) = B_1(\lambda)
\]

with bound

\[
|B_1(\lambda)| \leq C( (e^{-\alpha X_i} + e^{-\alpha X_{i-1}}) + (e^{\nu(\lambda)X_i} + e^{\nu(\lambda)X_{i-1}}) |G|) |D||d| + |\lambda|^2 |d|)
\]

For convenience, let

\[
K(X, \lambda) = (e^{-\alpha X_i} + e^{-\alpha X_{i-1}}) + (e^{\nu(\lambda)X_i} + e^{\nu(\lambda)X_{i-1}})|G|
\]

Then we have

\[
|B_1(\lambda)| \leq C( K(X, \lambda) |D|+ |\lambda|^2 )|d|
\]


From here, we plug this into $A_1$, $W_2$, and $A_2$ to get new estimates which are only functions of $d$ and $\lambda$.\\

First, we plug this into $A_1(\lambda)(b, c, d)$ to get $A_3(\lambda)(d)$. Note that we plug $|B_1(\lambda)|$ in for $c_i$, we can cancel a factor of $e^{\nu(\lambda)}$ since $|B_1(\lambda)|$ solves for $f(X_i; \lambda) c_i$. If we do not have such a factor, there is nothing we can do since the real part of $\nu(\lambda)$ can be 0.

\begin{align*}
|A_3&(\lambda)_i(d)| \\
&\leq C \Big( (e^{-\alpha X_i} + |G|) |b|
+ ( p_7(X_i; \lambda) + e^{\nu(\lambda)X_i} |G|) |c_i| + e^{\nu(\lambda)X_{i-1}} |G| |c_{i-1}| \\
&+ (e^{-\tilde{\alpha} X_i} + |G|) |\lambda|^2 |d| + |D_i||d| \Big) \\
&\leq C \Big( (e^{-\alpha X_i} + |G| + p_7(X_i; \lambda) )|B_1(\lambda)(d)|\\
&+ (e^{-\tilde{\alpha} X_i} + |G|) |\lambda|^2 |d| + |D_i||d| \Big) \\
&\leq C \Big( (e^{-\alpha X_i} + |G| + p_7(X_i; \lambda) )( K(X, \lambda) |D|+ |\lambda|^2 )|d|\\
&+ (e^{-\tilde{\alpha} X_i} + |G|) |\lambda|^2 |d| + |D_i||d| \Big) \\
&\leq C \Big( (e^{-\tilde{\alpha} X_i} + |G| + p_7(X_i; \lambda) ) |\lambda|^2  + |D|)|d| \Big) \\
\end{align*}

Next, we plug this into $W_2(\lambda)(b, c, d)$ to get $W_3(\lambda)(d)$

\begin{align*}
||W_3&(\lambda)(b,c,d)|| \\
&\leq C \Big( |b| + e^{\nu(\lambda)X_i} |c_i| + e^{\nu(\lambda)X_{i-1}} |c_{i-1}| 
+ |\lambda|^2 |d| + |D_i||d| + |D_{i-1}||d|) \\
&\leq C( |B_1(\lambda)(d)|+ |\lambda|^2 |d| + |D_i||d| + |D_{i-1}||d|) \\
&\leq C( ( K(X, \lambda) |D|+ |\lambda|^2 )|d| + |\lambda|^2 |d| + |D_i||d| + |D_{i-1}||d|) \\
&\leq C( |\lambda|^2 + |D|)|d|
\end{align*}

Finally, we plug this into $A_2(\lambda)(b, c, d)$ to get $A_4(\lambda)(d)$

\begin{align*}
|A_4&(\lambda)_i(b, c, d)| \\
&\leq \Big( (e^{-\alpha X_i} + |G| )|b| \\
&+ (p_7(X_i; \lambda) + e^{\nu(\lambda)X_i} |G|) |c_i| + e^{\nu(\lambda)X_{i-1}} |G| |c_{i-1}| \\
&+ (e^{-\tilde{\alpha} X_i} + |G|)|\lambda|^2 |d| 
+ (p_1(X_i; \lambda) + |G|) |D_i| |d| 
+ |G| |D_{i-1}| |d| \Big) \\
&\leq C \Big( (e^{-\alpha X_i} + |G| + p_7(X_i; \lambda))|B_1(\lambda)(d)|\\
&+ (e^{-\tilde{\alpha} X_i} + |G|)|\lambda|^2 |d| 
+ (p_1(X_i; \lambda) + |G|) |D_i| |d| 
+ |G| |D_{i-1}| |d| \Big) \\
&\leq C \Big( (e^{-\alpha X_i} + |G| + p_7(X_i; \lambda))
( K(X, \lambda) |D|+ |\lambda|^2 )|d|\\
&+ (e^{-\tilde{\alpha} X_i} + |G|)|\lambda|^2 |d| 
+ (p_1(X_i; \lambda) + |G|) |D_i| |d| 
+ |G| |D_{i-1}| |d| \Big) \\
&\leq C \Big( (e^{-\tilde{\alpha} X_i} + |G| + p_7(X_i; \lambda))|\lambda|^2 |d| \\ 
&+ (p_1(X_i; \lambda) + |G| + K(X, \lambda) ) |D| |d| \Big) \\
&\leq C \Big( (e^{-\tilde{\alpha} X_i} + |G| + p_7(X_i; \lambda))|\lambda|^2 + (p_1(X_i; \lambda) + K(X, \lambda) ) |D| \Big)|d|
\end{align*}

\item Now we can estimate the jumps

\[
\langle \Psi(0), W_i^+(0) - W_{i-1}^-(0) \rangle 
\]

Recall here that $\Psi(0)$ is the adjoint solution for the unperturbed problem, i.e. when $\lambda = 0$, so it does not depend on $\lambda$. The equations for $W$ contain the evolution operator $\Phi^{(s/u)}_\pm(x, y; \lambda)$ which are for the perturbed system with $\lambda \neq 0$.\\

For the adjoint solution $\Psi(x)$, we have estimate 

\[
|\Psi(x)| \leq C e^{-\alpha|x|}
\]

which holds since we know exactly what $\Psi$ is in this case. Note as well that $\Psi(0)$ is just a fixed constant.\\

Thus the jumps are given by

\[
\langle \Psi(0), W_i^+(0) - W_{i-1}^-(0) \rangle 
\]

where (as above) we have

\begin{align*}
W_i^+(0) - W_{i-1}^-(0) &= b_i^+ - b_i^- \\
&+ \Phi^u_+(0, X_i; \lambda)a_i^+ - \Phi^s_-(0, -X_{i-1}; \lambda)a_{i-1}^- \\
&+(P^s_+(0; \lambda) - P^s_-(0; 0))b_i^+  - (P^u_-(0; \lambda) - P^u_-(0; 0))b_i^- \\
&+ e^{-\nu(\lambda)X_i} v_+(0; \lambda) \langle v_0(\lambda), w_+(X_i; \lambda) \rangle c_i^+ \\
&- e^{\nu(\lambda)X_{i-1}} v_-(0; \lambda) \langle v_0(\lambda), w_-(-X_{i-1}; \lambda) \rangle c_{i-1}^- \\
&+ \int_{-X_{i-1}}^0 \Phi^s_-(0, y; \lambda) [ G_i^-(\lambda)W_i^-(y) + \lambda^2 d_i \tilde{H}(y) ] dy \\
&+ \int_{-X_{i-1}}^0
e^{\nu(\lambda)y} v_-(0; \lambda) \langle G_i^-(\lambda)(y)W_i^-(y) + \lambda^2 d_i \tilde{H}(y), w_-(y; \lambda) \rangle dy \\
&+ \int_{X_i}^0 \Phi^u_+(0, y; \lambda) [ G_i^+(\lambda)W_i^+(y) + \lambda^2 d_i \tilde{H}(y) ] dy \\
&+ \int_{X_i}^0 e^{-\nu(\lambda)y} v_+(0; \lambda) \langle G_i^+(\lambda)(y)W_i^+(y) + \lambda^2 d_i \tilde{H}(y), w_+(y; \lambda) \rangle dy
\end{align*}

\item Terms involving $a$

For $a_i^\pm$ we have

\begin{align*}
a_i^+ &= P^u_0 D_i d + A_4(\lambda)_i^+(d))\\
a_i^- &= -P^s_0 D_i d + A_4(\lambda)_i^-(d))
\end{align*}

For the ``plus'' one we have

\begin{align*}
\langle \Psi(0), &\Phi^u_+(0, X_i; \lambda) a_i^+ \rangle = \langle \Psi(0), \Phi^u_+(0, X_i; \lambda) P^u_0 D_i d \rangle + \langle \Psi(0), \Phi^u_+(0, X_1; \lambda) A_4(\lambda)_i^+(d) \rangle \\
\end{align*} 

This is all fine and good, but the adjoint $\Psi(0)$ is unperturbed by $\lambda$, whereas the evolution $\Phi^u_+(0, X_i; \lambda)$ is perturbed by $\lambda$. Let

\[
p_6(y; \lambda) = |\Phi^s_-(0, y; \lambda) - \Phi^s_-(0, y; 0)| + |\Phi^u_-(0, -y; \lambda) - \Phi^s_-(0, -y; 0)| 
\]

We will assume for now (as in the single pulse case, BUT WE SHOULD SHOW THIS) that

\[
p_6(y; \lambda) \leq C |\lambda| e^{-\alpha y}
\]

Then we have

\begin{align*}
\langle \Psi(0), &\Phi^u_+(0, X_i; \lambda) a_i^+ \rangle \\
&= \langle \Psi(0), \Phi^u_+(0, X_i; 0) P^u_0 D_i d \rangle + \langle \Psi(0), (\Phi^u_+(0, X_i; \lambda) - \Phi^u_+(0, X_i; 0)) P^u_0 D_i d \rangle \\
&+ \langle \Psi(0), \Phi^u_+(0, X_i; \lambda) A_4(\lambda)_i^+(d) \rangle \\
&= \langle \Psi(X_i), P^u_0 D_i d \rangle + \mathcal{O}(|\lambda| e^{-\alpha X_i}|D_i||d|) + \langle \Psi(0), \Phi^u_+(0, X_i; \lambda) A_4(\lambda)_i^+(d) \rangle 
\end{align*}

Plugging in the bound for $A_4(\lambda)$, this becomes

\begin{align*}
\langle \Psi(0), &\Phi^u_+(0, X_i; \lambda) a_i^+ \rangle \\
&= \langle \Psi(X_i), P^u_0 D_i d \rangle  + \mathcal{O}(|\lambda| e^{-\alpha X_i}|D_i||d|) \\
&+ \mathcal{O}(e^{-\alpha X_i}(e^{-\tilde{\alpha} X_i} + |G| + p_7(X_i; \lambda))|\lambda|^2 + (p_1(X_i; \lambda) + K(X, \lambda) ) |D|)|d|) \\
&= \langle \Psi(X_i), P^u_0 D_i d \rangle \\
&+ \mathcal{O}(e^{-\alpha X_i}(e^{-\tilde{\alpha} X_i} + |G| + p_7(X_i; \lambda))|\lambda|^2 + e^{-\alpha X_i}(p_1(X_i; \lambda) + K(X, \lambda) + |\lambda|) |D|)|d|)
\end{align*}

Similarly we have

\begin{align*}
\langle \Psi(0), &\Phi^s_-(0, -X_{i-1}; \lambda)a_{i-1}^- \rangle \\
&= -\langle \Psi(-X_{i-1}), P^s_0 D_{i-1} d \rangle \\
&+ \mathcal{O}(e^{-\alpha X_{i-1}}(e^{-\tilde{\alpha} X_{i-1}} + |G| + p_7(X_{i-1}; \lambda))|\lambda|^2 + e^{-\alpha X_{i-1}}(p_1(X_{i-1}; \lambda) + K(X, \lambda) + |\lambda|) |D|)|d|)
\end{align*}

\item Terms involving $b$.\\

The terms involving $b^\pm$ by themselves will die since they are in the spaces $R^u_-(0; 0) \oplus R^s_+(0; 0)$ which are perpendicular to $\Psi(0)$.\\

The other terms involving $b$ look like $(P^u_-(0; \lambda) - P^u_-(0; 0))b_i^-$ (and similar for the other one). A bound on these terms looks like

\begin{align*}
|\langle \Psi(0), &(P^u_-(0; \lambda) - P^u_-(0; 0))b_i^- \rangle|
\leq |\Psi(0)| p_3(\lambda)|b_i^-| \\
&\leq |\Psi(0)| p_3(\lambda)|B_1(\lambda)(d)| \\
&\leq C p_3(\lambda) ( K(X, \lambda) |D|+ |\lambda|^2 )|d|
\end{align*}

where in the last line we substituted $B_1(\lambda)(d)$ for $b_i^\pm$.\\

\item Terms involving $c$.\\

These terms look like

\begin{align*}
&e^{\nu(\lambda)X_{i-1}} v_-(0; \lambda) \langle v_0(\lambda), w_-(-X_{i-1}; \lambda) \rangle c_{i-1}^- \\
&e^{-\nu(\lambda)X_i} v_+(0; \lambda) \langle v_0(\lambda), w_+(X_i; \lambda) \rangle c_i^+ \\
\end{align*}

Let's do the first one. Taking the inner product with $\Psi(0)$ only hits the $v_-(0; \lambda)$ term since everything else is a scalar. This gives us 

\begin{align*}
e^{\nu(\lambda)X_{i-1}} \langle \Psi(0), v_-(0; \lambda) \rangle \langle v_0(\lambda), w_-(X_{i-1}; \lambda) \rangle c_{i-1}^-
\end{align*}

Before we plug in our bound, we would like to do something useful with the second inner product, but we cannot at the moment since one term is a solution to the adjoint when $\lambda = 0$ and the other term is a solution to the perturbed eigenvalue problem when $\lambda \neq 0$. To do this, we will expand $v_-(0; \lambda)$ in a Taylor series about $\lambda = 0$. FOR NOW WE WILL ASSUME WE CAN ACTUALLY DO THIS.

\[
v_-(0; \lambda) = v_-(0; 0) + \lambda \frac{\partial}{\partial \lambda}v_-(0; \lambda)\Big|_{\lambda = 0} + \mathcal{O}(\lambda^2)
\]

I don't know if we can compute $\frac{\partial}{\partial \lambda}v_-(0; \lambda)\Big|_{\lambda = 0}$, but it is a constant which depends only on the initial setup of the problem, so we should not care what it is. Then we have

\begin{align*}
|&e^{\nu(\lambda)X_{i-1}} c_{i-1}^- \langle v_0(\lambda), w_-(-X_{i-1}; \lambda) \rangle \langle \Psi(0), v_-(0; \lambda) \rangle|\\
&\leq e^{\nu(\lambda)X_{i-1}}|c_{i-1}^-| |\langle v_0(\lambda), w_-(-X_{i-1}; \lambda) \rangle|\langle \Psi(0), v_-(0; 0) + \lambda \frac{\partial}{\partial \lambda}v_-(0; \lambda)\Big|_{\lambda = 0} + \mathcal{O}(\lambda^2) \rangle| \\
&\leq e^{\nu(\lambda)X_{i-1}}|c_{i-1}^-| |\langle v_0(\lambda), w_-(-X_{i-1}; \lambda) \rangle| \left( |\langle \Psi(0), v_-(0; 0) \rangle| +  |\langle \Psi(0), \lambda \frac{\partial}{\partial \lambda}v_-(0; \lambda)\Big|_{\lambda = 0} \rangle| + \mathcal{O}(|\lambda|^2) \right) \\
&\leq e^{\nu(\lambda)X_{i-1}}|c_{i-1}^-| |\langle v_0(\lambda), w_-(-X_{i-1}; \lambda) \rangle| \left( |\langle \Psi(0), \lambda \frac{\partial}{\partial \lambda}v_-(0; \lambda)\Big|_{\lambda = 0} \rangle| + \mathcal{O}(|\lambda|^2) \right)
\end{align*}

In the last line, we used the fact that $\langle \Psi(0), v_-(0; 0) \rangle = 0$ since $\langle \Psi(0), v_-(0; 0) \rangle = \langle \Psi(0), \tilde{v}_-(0; 0) \rangle$, the inner product of this is constant in $x$, and $\Psi(x)$ decays to 0 at $-\infty$ faster than any growth in $\tilde{v}_-(x; 0)$ at $-\infty$. Taking $\frac{\partial}{\partial \lambda}v_-(0; \lambda)\Big|_{\lambda = 0}$ as a constant, we have

\begin{align*}
|&e^{\nu(\lambda)X_{i-1}} c_{i-1}^- \langle v_0(\lambda), w_-(-X_{i-1}; \lambda) \rangle \langle \Psi(0), v_-(0; \lambda) \rangle| \\
&\leq C e^{\nu(\lambda)X_{i-1}}|c_{i-1}^-| (\langle v_0(\lambda), w_0(\lambda) \rangle + \langle v_0(\lambda), \Delta w_-(-X_{i-1}; \lambda) \rangle) (|\lambda| + \mathcal{O}(\lambda^2) ) \\
&\leq C e^{\nu(\lambda)X_{i-1}}|c_{i-1}^-| (1 + p_2(X_{i-1}; \lambda) (|\lambda| + \mathcal{O}(\lambda^2) )
\end{align*}

Since $p_2(X_{i-1}; \lambda)$ is small for sufficiently small $\lambda$ and sufficiently large $X_1$, this becomes

\begin{align*}
|&e^{\nu(\lambda)k X_1} c_2^- \langle v_0(\lambda), w_-(-k X_1; \lambda) \rangle \langle \Psi(0), v_-(0; \lambda) \rangle| \\
&\leq C e^{\nu(\lambda)k X_1}|c_2^-| (|\lambda| + \mathcal{O}(\lambda^2) )
\end{align*}

Now we plug in our estimate for $|c_{i-1}^-|$. Recall from our discussion in the $B_1$ section that this is of order $e^{-|\nu(\lambda)|X_{i-1}} B_1(\lambda)(d)$. This means we can cancel the exponential factor out front! 

\begin{align*}
|&e^{\nu(\lambda)X_{i-1}} c_{i-1}^- \langle v_0(\lambda), w_-(-X_{i-1}; \lambda) \rangle \langle \Psi(0), v_-(0; \lambda) \rangle| \\
&\leq C e^{\nu(\lambda)X_{i-1}} e^{-|\nu(\lambda)|X_{i-1}} |B_1(\lambda)(d)| (|\lambda| + \mathcal{O}(\lambda^2) ) \\
&\leq C |\lambda| ( K(X, \lambda) |D|+ |\lambda|^2 )|d|
\end{align*}

The other one is similar.\\

\item ``Noncenter'' integral terms

\begin{enumerate}[(i)]

\item Integrals not involving $H$. For these we only do the ``minus'' half. The ``plus'' half is similar

\begin{align*}
\left| \int_{-X_{i-1}}^0 \langle \Psi(0), \Phi^s_-(0, y; \lambda) G_i^-(\lambda)W_i^-(y) \rangle dy \right| 
&\leq C |G| ||W|| \\
&\leq C |G| ( |\lambda|^2 + |D|)|d|
\end{align*}

where we plugged in $W_3(\lambda)(d)$ for $W$. We need a better estimate here. We will try what we did in the exponentially weighted case, and what was done in Sanstede (1998).

\item Improved estimate for $W$

Since we have 
\[
W_i^\pm = L_1(\lambda)W_i^\pm + L_2(\lambda)W_i^\pm 
\]

we can use the estimate for $L_1(\lambda)$ together with an improved estimate for $L_2(\lambda)$. We use the same estimate for $L_1(\lambda)$ as we have above. We will do the negative piece here. The positive piece is similar.

\[
||L_1(\lambda)W|| \leq C e^{\nu(\lambda)X_M} |G| \: ||W||
\]

For $L_2(\lambda)$ we will use the following estimate for the negative piece.

\begin{align*}
| (L_2(\lambda)(a, b, c, d)_i^-)(x)| &\leq C \left( e^{-\alpha^s(x + X_{i-1})} |a^-_{i-1}| + |b| + e^{\nu(\lambda)X_i} |c_i| + e^{\nu(\lambda)X_{i-1}} |c_{i-1}| |\lambda|^2 |d_i| \right)
\end{align*}

Combining these, we have

\begin{align*}
| W_i^-(x)| &\leq C \left( e^{-\alpha^s(x + X_{i-1})} |a^-_{i-1}| + |b| + e^{\nu(\lambda)X_i} |c_i| + e^{\nu(\lambda)X_{i-1}} |c_{i-1}| |\lambda|^2 |d_i| + e^{\nu(\lambda)X_M} |G| ||W|| \right)
\end{align*}

Substituting in $A_3(\lambda)$, $B_1(\lambda)$, and $W_3(\lambda)$, and recalling that we cancel the exponential factors $e^{\nu(\lambda)X_i}$ when we substitute for the $c$, this estimate becomes

\begin{align*}
| W_i^-(x)| &\leq C \left( e^{-\alpha^s(x + X_{i-1})} |A_3(\lambda)| + |B_1(\lambda)| + |G| ||W_3(\lambda)|| + |\lambda|^2 |d_i|  \right) \\
&\leq \Big( e^{-\alpha^s(x + X_{i-1})}( (e^{-\tilde{\alpha} X_i} + |G| + p_7(X_i; \lambda) ) |\lambda|^2  + |D|)|d|) \\
&+ ( K(X; \lambda) |D|+ |\lambda|^2 )|d| \\
&+ |G| ( |\lambda|^2 + |D|)|d| + |\lambda|^2 |d_i| \Big) \\
&\leq C( (e^{-\alpha^s(x + X_{i-1})} + K(X; \lambda))|D| + |\lambda^2|)|d|
\end{align*}

Similarly

\begin{align*}
| W_i^+(x)| \leq C ( (e^{\alpha^u(x - X_i)} + K(X; \lambda))|D| + |\lambda|^2 )|d|
\end{align*}

\item Better integral bound

Now we use this to get a better integral bound. Doing the ``minus'' one, we have

\begin{align*}
&\left| \int_{-X_{i-1}}^0 \langle \Psi(0), \Phi^s_-(0, y; \lambda) G_i^-(\lambda)W_i^-(y) \rangle dy \right| \\
&\leq C |G| |d| \int_{-X_{i-1}}^0 | \langle \Psi(0), \Phi^s_-(0, y; \lambda) ( (e^{-\alpha^s(y + X_{i-1})} + K(X; \lambda))|D| + |\lambda^2|) \rangle | dy
\end{align*}

The only one which is tricky is the one involving ${-\alpha^s(y + X_{i-1})}$, so let's look at that separately. For convenience, let $\tilde{\delta} = \alpha - \tilde{\alpha} > 0$.

\begin{align*}
\int_{-X_{i-1}}^0 &| \langle \Psi(0), \Phi^s_-(0, y; \lambda) (e^{-\alpha^s(y + X_{i-1})} |D| \rangle | dy \leq C |D| \int_{-X_{i-1}}^0 e^{\alpha y} e^{-\alpha(y + X_{i-1})} dy \\
&\leq C |D| \int_{-X_{i-1}}^0 e^{-\alpha X_{i-1}} dy \\
&= C |D| \int_{-X_{i-1}}^0 e^{-\tilde{\delta} y} e^{\tilde{\delta} y}  e^{-\alpha X_{i-1}} dy \\
&\leq C |D| \int_{-X_{i-1}}^0 e^{\tilde{\delta} X_{i-1}} e^{\tilde{\delta} y}  e^{-\alpha X_{i-1}} dy \\
&\leq C |D| \int_{-X_{i-1}}^0 e^{-(\alpha - \tilde{\delta}) X_{i-1}} e^{\tilde{\delta} y} dy \\
&\leq C e^{-\tilde{\alpha} X_{i-1}} |D| \int_{-X_{i-1}}^0 e^{\tilde{\delta} y} dy \\
&\leq C e^{-\tilde{\alpha} X_{i-1}} |D|
\end{align*}

This is similar to what we have in Sandstede (1998), except the exponential factor is $\tilde{\alpha}$ instead of $\alpha$, which is fine. Now that we have done the first part of the integral bound, the rest is no problem. We have for the entire thing 

\begin{align*}
&\left| \int_{-X_{i-1}}^0 \langle \Psi(0), \Phi^s_-(0, y; \lambda) G_i^-(\lambda)W_i^-(y) \rangle dy \right| \\
&\leq C |G| ( e^{-\tilde{\alpha} X_{i-1}} + K(X; \lambda))|D| + |\lambda^2|)|d|
\end{align*}

The other ones are similar.

\item Integrals involving $H$

Recall that these integrals are multiplied by $d_i \lambda^2$. We will omit that for now when obtaining the bound. Then we have

\begin{align*}
\langle \Psi(0)&, \int_{-X_{i-1}}^0 \Phi^s_-(0, y; \lambda) \tilde{H}(y) dy \rangle \\ 
&= \int_{-X_{i-1}}^0 \langle \Psi(0), \Phi^s_-(0, y; 0) \tilde{H}(y) \rangle dy + 
\int_{-X_{i-1}}^0 \langle \Psi(0), (\Phi^s_-(0, y; \lambda) - \Phi^s_-(0, y; 0)) \tilde{H}(y) \rangle dy
\end{align*}

where we need to play this trick to replace the $\lambda$-dependent evolution with the evolution when $\lambda = 0$ in order to get the Melnikov term we want. Again, we will assume we have a bound of the form

\[
|\Phi^s_-(0, y; \lambda) - \Phi^s_-(0, y; 0)| \leq C |\lambda| e^{-\alpha y}
\]

which is the same $p_6$ above. With this bound, the second integral above is order $|\lambda|$. For the first integral, we have 

\begin{align*}
\int_{-X_{i-1}}^0 \langle \Psi(0), \Phi^s_-(0, y; 0) \tilde{H}(y) \rangle dy &= 
\int_{-X_{i-1}}^0 \langle \Psi(y), H(y) \rangle dy + \int_{-X_{i-1}}^0 \langle \Psi(y), \Delta H(y) \rangle dy \\
&= \int_{-\infty}^0 \langle \Psi(y), H(y) \rangle dy - \int_{-\infty}^{-X_{i-1}} \langle \Psi(y), H(y) \rangle dy \\
&+ \int_{-X_{i-1}}^0 \langle \Psi(y), \Delta H(y) \rangle dy 
\end{align*}

The first integral on the RHS is half of our Melnikov integral. The second is order $e^{-\alpha X_{i-1}}$. The third is order $e^{-\alpha X_1}$ (the order of $\Delta H$). When we do the ``plus'' piece, the first integral is the other half of the Melnikov integral; the second integral is order $e^{-\alpha X_i}$; and the third integral is the same order. Thus these integral terms look like

\begin{align*}
d_i \lambda^2 \int_{-\infty}^\infty \langle \Psi(y), H(y) \rangle dy + \mathcal{O}( ( |\lambda| + e^{-\alpha X_i} + e^{-\alpha X_{i-1}} ) |\lambda|^2 |d|)
\end{align*}

\end{enumerate}

\item ``Center'' integral terms

The ``minus'' integral term is

\begin{align*}
\langle \Psi(0) &, \int_{-X_{i-1}}^0
e^{\nu(\lambda)y} v_-(0; \lambda) \langle G_i^-(\lambda)(y)W_i^-(y) + \lambda^2 d_i \tilde{H}(y), w_-(y; \lambda) \rangle dy \rangle \\
&= \int_{-X_{i-1}}^0
e^{\nu(\lambda)y} \langle \Psi(0), v_-(0; \lambda) \rangle \langle G_i^-(\lambda)(y)W_i^-(y) + \lambda^2 d_i \tilde{H}(y), w_-(y; \lambda) \rangle dy 
\end{align*}

For this bound, we will use what we did above and have

\[
\langle \Psi(0), v_-(0; \lambda) \rangle = \mathcal{O}(\lambda)
\]

Next we do the standard thing and write $\tilde{H} = H + \Delta H$; then these integrals give us three terms, so we have these three integrals to deal with.

\begin{align*}
&\int_{-X_{i-1}}^0
e^{\nu(\lambda)y} \langle \Psi(0), v_-(0; \lambda) \rangle \langle G_i^-(\lambda)(y)W_i^-(y) , w_-(y; \lambda) \rangle dy \\
&+\int_{-X_{i-1}}^0
e^{\nu(\lambda)y} \langle \Psi(0), v_-(0; \lambda) \rangle \langle \lambda^2 d_i \Delta H(y), w_-(y; \lambda) \rangle dy \\
&+\int_{-X_{i-1}}^0
e^{\nu(\lambda)y} \langle \Psi(0), v_-(0; \lambda) \rangle \langle \lambda^2 d_i H(y), w_-(y; \lambda) \rangle dy 
\end{align*}

\begin{enumerate}

\item Integral involving $W$ \\

For this one, we need to use our improved bound for $||W||$. Recall that we are taking $\nu(\lambda) > 0$.

\begin{align*}
&\left| \int_{-X_{i-1}}^0
e^{\nu(\lambda)y} \langle \Psi(0), v_-(0; \lambda) \rangle \langle G_i^-(\lambda)(y)W_i^-(y), w_-(y; \lambda) \rangle dy \right| \\
&\leq C |\lambda| |G| \int_{-X_{i-1}}^0 e^{\nu(\lambda)y} ||W_i^-|| dy \\
&\leq C |\lambda| |G| \int_{-X_{i-1}}^0 e^{\nu(\lambda)y} ( (e^{-\alpha^s(y + X_{i-1})} + K(X; \lambda))|D| + |\lambda|^2 )|d| dy \\
&\leq C |\lambda| |G| e^{\nu(\lambda)X_{i-1}} (( 1 + K(X; \lambda))|D| + |\lambda|^2 )|d| \\
&\leq C |\lambda| |G| e^{\nu(\lambda)X_{i-1}} (|D| + |\lambda|^2 )|d| \\
\end{align*}

We could have used the regular bound $W_3$ here, but I have left it this way in case we need something better. The other integral has the same bound except it involves $X_i$.

\item Integrals not involving $W$ \\

The first of these (which is multiplied by $d_i \lambda^2$) has bound

\begin{align*}
\left| \int_{-X_{i-1}}^0
e^{\nu(\lambda)(y)} \langle \Psi(0), v_-(0; \lambda) \rangle \langle \Delta H(y), w_-(y; \lambda) \rangle dy \right| 
&\leq C |\lambda| e^{\nu(\lambda)X_{i-1}} |\Delta H| \\
&\leq C |\lambda| e^{\nu(\lambda)X_{i-1}} e^{-\alpha X_1}
\end{align*}

Recalling what we did above, the other one (which is also multiplied by $d_i \lambda^2$) has bound

\begin{align*}
&\left| \int_{-X_{i-1}}^0 e^{\nu(\lambda)y} \langle \Psi(0), v_-(0; \lambda) \rangle
\langle H(y), w_-(y; \lambda) \rangle dy \right| \\
&\leq C |\langle \Psi(0), v_-(0; \lambda) \rangle| \int_{-X_{i-1}}^0 e^{\tilde{\alpha}y}e^{\nu(\lambda)y} | e^{-\tilde{\alpha} y} H(y)|dy\\
&\leq C |\lambda| 
\end{align*}

where $|e^{-\tilde{\alpha} y} H(y)|$ is bounded since we know the decay properties of $H$, and the integral is uniformly bounded in $X_{i-1}$ since $|\nu(\lambda)| < \tilde{\alpha}$. \\

The ``plus'' terms are similar (although they involve $X_i)$. Thus we have an overall bound for these two integral terms which is of order

\begin{align*}
(|\lambda| e^{\nu(\lambda)X_{i-1}} e^{-\alpha X_1} + |\lambda|)|\lambda|^2 |d|
&= (1 + e^{\nu(\lambda)X_{i-1}} e^{-\alpha X_1})|\lambda|^3 |d| \\
&= |\lambda|^3 |d|
\end{align*}

\end{enumerate}


\item Put all of this together\\

Now that we have bounds on all the things, we have our expression for the jump.

\begin{align*}
\langle \Psi(0), &W_i^+(0) - W_i^-(0) \rangle = 
\langle \Psi(X_i), P^u_0 D_i d \rangle + \langle \Psi(-X_{i-1}), P^u_0 D_{i-1} d \rangle \\
&+ d_i \lambda^2 \int_{-\infty}^\infty \langle \Psi(y), H(y) \rangle dy + R(\lambda)_i(d)
\end{align*}

where we have remainder bound (which incorporates $i = 1, 2$)

\begin{align*}
R&(\lambda)(d) \leq C \Big( \\
&(e^{-\alpha X_{i-1}}(e^{-\tilde{\alpha} X_{i-1}} + |G| + p_7(X_{i-1}; \lambda))|\lambda|^2 + e^{-\alpha X_{i-1}}(p_1(X_{i-1}; \lambda) + K(X, \lambda) + |\lambda|) |D|)|d| \\
&+(e^{-\alpha X_i}(e^{-\tilde{\alpha} X_i} + |G| + p_7(X_i; \lambda))|\lambda|^2 + e^{-\alpha X_i}(p_1(X_i; \lambda) + K(X, \lambda) + |\lambda|) |D|)|d|\\
&+ p_3(\lambda) ( K(X, \lambda) |D|+ |\lambda|^2 )|d| \\
&+ |\lambda| ( K(X, \lambda) |D|+ |\lambda|^2 )|d| \\
&+ |G| ( e^{-\tilde{\alpha} X_{i-1}} + e^{-\tilde{\alpha} X_i} + K(X; \lambda))|D| + |\lambda^2|)|d| \\
&+ ( |\lambda| + e^{-\alpha X_i} + e^{-\alpha X_{i-1}} ) |\lambda|^2 |d| \\
&+ |\lambda| |G| (e^{\nu(\lambda)X_{i-1}} + e^{\nu(\lambda)X_i})(|D| + |\lambda|^2 )|d|  \\
&+ |\lambda|^3 |d| \Big)
\end{align*}

For convenience, let

\begin{align*}
K_2(X, \lambda) &= (e^{-\alpha X_i} + e^{-\alpha X_{i-1}} + |G| + p_3(\lambda) + |\lambda|) \\
K_3(X, \alpha) &= e^{-\alpha X_i} + e^{-\alpha X_{i-1}}
\end{align*}

$K_2(X, \lambda)$ has order $(e^{-\alpha X_i} + e^{-\alpha X_{i-1}} + |\lambda|$. Collecting terms above and dropping some of the higher order ones, this becomes

\begin{align*}
R&(\lambda)(d) \leq C \Big( \\
&K_2(X, \lambda)|\lambda|^2 |d| \\
&(K_2(X, \lambda)K(X, \lambda) + K_3(X, \alpha)|\lambda| + K_3(X, \tilde{\alpha})|G| + |\lambda| |G| (e^{\nu(\lambda)X_{i-1}} + e^{\nu(\lambda)X_i})\\
&+ e^{-\alpha X_i}(p_1(X_i; \lambda) + e^{-\alpha X_{i-1}}(p_1(X_{i-1}; \lambda))|D||d| \Big)
\end{align*}

\item Plug in stuff \\

First, we plug in our known (or guessed) for the components of the remainder term. If this is not good enough, we need to rethink the whole thing, so it's good to do this first.\\

Recall that we have the following bounds (and that we have $X_2 = k X_1 \geq X_1$)

\begin{align*}
|G| &\leq C e^{-\alpha X_1} \\
D_i &\leq C e^{-\alpha X_1} |d| \\
K(X, \lambda) &= (e^{-\alpha X_i} + e^{-\alpha X_{i-1}}) + (e^{\nu(\lambda)X_i} + e^{\nu(\lambda)X_{i-1}})|G| \\
&\leq C(e^{-\alpha X_1} + e^{\nu(\lambda)k X_1} e^{-\alpha X_1} ) \\
&\leq C e^{-\alpha X_1} e^{k \nu(\lambda) X_1} \\
K_2(X, \lambda) &\leq C(e^{-\alpha X_i} + e^{-\alpha X_{i-1}} + |\lambda|) \\
&\leq C(e^{-\alpha X_1} + |\lambda|) \\
K_3(X, \alpha) &= e^{-\alpha X_i} + e^{-\alpha X_{i-1}} \\
&\leq C e^{-\alpha X_i} \\
p_1(X_i; \lambda) &\leq C e^{-\alpha X_i} \\
&\leq C e^{-\alpha X_1} \\
\end{align*}

Using these, the coefficient of $|\lambda|^2 |d|$ is 

\[
e^{-\alpha X_1} + |\lambda|
\]

The $|D|$ coefficient has bound

\begin{align*}
(e&^{-\alpha X_1} + |\lambda|) e^{-\alpha X_1} e^{\nu(\lambda) k X_1}\\
&+ e^{-\alpha X_1}|\lambda| \\
&+ e^{-\tilde{\alpha} X_1}e^{-\alpha X_1} \\
&+ |\lambda| e^{-\alpha X_1} e^{\nu(\lambda) k X_1}\\
&+ e^{-\alpha X_1}e^{-\alpha X_1} \\
&= e^{-\alpha X_1}( e^{-\alpha X_1} e^{k \nu(\lambda) X_1} + e^{-\tilde{\alpha} X_1}) + e^{-\alpha X_1} e^{k \nu(\lambda) X_1} |\lambda|
\end{align*}

To get the remainder bound, we plug in the bound for $|D|$ to get

\begin{align*}
R&(\lambda)(d) \leq C \Big( (e^{-\alpha X_1} + |\lambda|)|\lambda|^2 |d| 
+ e^{-\alpha X_1}(e^{-\alpha X_1}( e^{-\alpha X_1} e^{k \nu(\lambda) X_1} + e^{-\tilde{\alpha} X_1}) + e^{-\alpha X_1} e^{k \nu(\lambda) X_1} |\lambda|) \Big)
\end{align*}

Combining terms and simplifying, we get our final remainder bound

\begin{align*}
R&(\lambda)(d) \leq C \Big( (e^{-\alpha X_1} + |\lambda|)|\lambda|^2 
+ e^{-2 \alpha X_1}( e^{-\alpha X_1} e^{k \nu(\lambda) X_1} + e^{-\tilde{\alpha} X_1}) + e^{-2\alpha X_1} e^{k \nu(\lambda) X_1} |\lambda|) \Big)
\end{align*}

As long as $|\lambda|$ is order $e^{-2 \alpha X_1}$, which is what we expect from our leading order computations, etc, we are all set!\\

\item Make the matrix \\

Finally, we plug in $D_i$ and see what we get. Recall that we have the following expression/bound for $D_i$ (which is a reasonable guess but unproven)

\[
D_i = ( Q'(X_i) + Q'(-X_i)(d_2 - d_1) + \mathcal{O} \left( e^{-\alpha X_i} \left( |\lambda| +  e^{-\alpha X_i}  \right) |d| \right)
\]

First, we substitute this into the term $\langle \Psi(X_i), P^u_0 D_i d \rangle$. Using the fact that $\Psi(\pm X_i)$ is order $e^{-\alpha X_i}$, we have

\begin{align*}
\langle \Psi(X_i), P^u_0 D_i d \rangle &= \langle \Psi(X_1), (Q'(X_1) + Q'(-X_1 )(d_{i+1} - d_i ) \rangle + \mathcal{O} \left( e^{-2 \alpha X_i} \left( |\lambda| +  e^{-\alpha X_i}  \right) |d| \right)
\end{align*}

Now we assume what was done in (3.36) in Sandstede (1998) applies here. We do need to verify this, since this was assumed in the exponentially weighted version as well. With this assumption, this becomes

\begin{align*}
\langle \Psi(X_i), P^u_0 D_i d \rangle &= \langle \Psi(X_i), Q'(-X_i) \rangle (d_{i+1} - d_i ) + \mathcal{O} \left( e^{-2 \alpha X_i} \left( |\lambda| +  e^{-\alpha X_i}  \right) |d| \right)
\end{align*}

Similarly, we have

\begin{align*}
\langle \Psi(-X_{i-1}), P^u_0 D_{i-1} d \rangle &= \langle \Psi(-X_{i-1}), Q'(X_{i-1}) \rangle (d_i - d_{i-1} ) + \mathcal{O} \left( e^{-2 \alpha X_i} \left( |\lambda| +  e^{-\alpha X_i}  \right) |d| \right)
\end{align*}

Thus, to leading order, the jump is given by

\begin{align*}
\xi_i &= \langle \Psi(0), W_i^+(0) - W_i^-(0) \rangle \\
&= \langle \Psi(X_i), Q'(-X_i) \rangle (d_{i+1} - d_i ) + \langle \Psi(-X_{i-1}), Q'(X_{i-1}) \rangle (d_i - d_{i-1} ) + \lambda^2 d_i M
\end{align*}

Taking $i = 1, 2$, this becomes

\begin{align*}
\xi_1 &= \langle \Psi(X_1), Q'(-X_1) \rangle (d_2 - d_1 ) + \langle \Psi(-X_2), Q'(X_2) \rangle (d_1 - d_2 ) + \lambda^2 d_1 M \\
\xi_2 &= \langle \Psi(X_2), Q'(-X_2) \rangle (d_1 - d_2 ) + \langle \Psi(-X_1), Q'(X_1) \rangle (d_2 - d_1 ) + \lambda^2 d_2 M
\end{align*}

If we define

\begin{align*}
a_i &= \langle \Psi (X_i), Q'(-X_i) \rangle \\
\tilde{a}_i &= \langle \Psi(-X_i), Q'(X_i) \rangle
\end{align*}

then in matrix form our jump equations say that there is a nonzero solution if and only if

\[
E(\lambda) = \det S(\lambda) = \det(A + \lambda^2 MI + R(\lambda) ) = 0
\]

where $M$ is the higher order Melnikov integral

\[
M = \int_{-\infty}^\infty \langle \Psi(y), H(y) \rangle dy
\]

and the matrix $A$ is defined by

\[
A = 
\begin{pmatrix}
-a_1 + \tilde{a}_2 & a_1 - \tilde{a}_2 \\
-\tilde{a}_1 + a_2 & \tilde{a}_1 - a_2 
\end{pmatrix}
\]

As in the exponentially weighted case, we have (using even and oddness of the functions involved)

\[
\tilde{a}_i = \langle \Psi(-X_i), Q'(X_i) \rangle = -\langle \Psi(X_i), Q'(-X_i) \rangle = -a_i
\]

Thus the matrix $A$ becomes

\[
A = 
\begin{pmatrix}
-a_1 - a_2 & a_1 + a_2 \\
a_1 + a_2 & -a_1 - a_2 
\end{pmatrix}
\]

If we let $a = a_1 + a_2$, and redefine $A$ by

\[
A = 
\begin{pmatrix}
-1 & 1 \\
1 & -1
\end{pmatrix}
\]

then we have the matrix equation

\[
E(\lambda) = \det S(\lambda) = \det(a A + \lambda^2 MI + R(\lambda) ) = 0
\]

Thus to leading order we are computing the determinant of $a A + \lambda^2 MI$, i.e. 

\[
a A + \lambda^2 MI = 
\begin{pmatrix}
-a + \lambda^2 M & a \\
a & -a + \lambda^2 M
\end{pmatrix}
\]

The characteristic polynomial for this is

\begin{align*}
0 &= (-a + \lambda^2 M)^2 - a^2 \\
&= a^2 - 2 a \lambda^2 M + \lambda^4 M^2 - a^2 \\
&= - 2 a \lambda^2 M + \lambda^4 M^2 \\
&= \lambda^2 M (-2a + M \lambda^2 )
\end{align*}

This has two roots at 0 (which are expected) as well as two roots at $\pm \sqrt{2a/M}$. This agrees with the leading-order versino for the exponentially weighted case, except the $a$ here is a little different and there is no negative sign (COULD BE A MISTAKE, SHOULD CHECK).

\end{enumerate}

\end{document}