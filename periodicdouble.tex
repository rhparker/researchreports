\documentclass[12pt]{article}
\usepackage[pdfborder={0 0 0.5 [3 2]}]{hyperref}%
\usepackage[left=1in,right=1in,top=1in,bottom=1in]{geometry}%
\usepackage[shortalphabetic]{amsrefs}%
\usepackage{amsmath}
\usepackage{enumerate}
% \usepackage{enumitem}
\usepackage{amssymb}                
\usepackage{amsmath}                
\usepackage{amsfonts}
\usepackage{amsthm}
\usepackage{bbm}
\usepackage[table,xcdraw]{xcolor}
\usepackage{tikz}
\usepackage{float}
\usepackage{booktabs}
\usepackage{svg}
\usepackage{mathtools}
\usepackage{cool}
\usepackage{url}
\usepackage{graphicx,epsfig}
\usepackage{makecell}
\usepackage{array}

\def\noi{\noindent}
\def\T{{\mathbb T}}
\def\R{{\mathbb R}}
\def\N{{\mathbb N}}
\def\C{{\mathbb C}}
\def\Z{{\mathbb Z}}
\def\P{{\mathbb P}}
\def\E{{\mathbb E}}
\def\Q{\mathbb{Q}}
\def\ind{{\mathbb I}}

\graphicspath{ {periodic/} }

\newtheorem{lemma}{Lemma}
\newtheorem{corollary}{Corollary}
\newtheorem{definition}{Definition}
\newtheorem{assumption}{Assumption}
\newtheorem{hypothesis}{Hypothesis}

\begin{document}

\section*{Double Pulse}

Now we repeat the above in the case of a periodic double pulse $q_{2p}$. The picture looks like

\begin{figure}[H]
\includegraphics[width=8.5cm]{dpimage}
\end{figure}

Note that because of the way this method is set up, the period of this thing is $2(X_1 + X_2)$. The existence proof (if such a thing exists) constructs this whole thing at once, rather than constructing a double pulse and then making it periodic. \\

The equations we are looking to solve are

\begin{enumerate}[(i)]
\item $(W_i^\pm)' = A(q; \lambda) W_i^\pm + G_i(\lambda)^\pm W_i^\pm + \lambda^2 d_i \tilde{H}_i^\pm$
\item $W_i^\pm(0) \in \C \psi(0) \oplus Y^+ \oplus Y^- \oplus Y^0$
\item $W_i^+(0) - W_i^-(0) \in \C \psi(0) $
\item $W_1^+(X_1) - W_2^-(-X_1) = D_1 d $
\item $W_2^+(X_2) - W_1^-(-X_2) = 0$
\end{enumerate}

where

\begin{align*}
G_i(\lambda)^\pm &= A(q_{2p};\lambda) - A(q;\lambda)) \\
D_1 d &= d_2(Q_{2p}'(-X_1) + \lambda (Q_{2p})_c(-X_1)) 
- d_1 ( Q_{2p}'(X_1) + \lambda (Q_{2p})_c(X_1) ) \\
\tilde{H} &= -B(Q_{2p})_c \\
H &= -B Q_c \\
\Delta H &= \tilde{H} - H
\end{align*}

Again we do not have a $\lambda B W_i^\pm$ term since this has been absorbed into $A(q; \lambda) W_i^\pm$. \\

Let $X_m = \min\{ X_1, X_2 \}$ and $X_M = \max\{ X_1, X_2 \}$. Then we should have bounds

\begin{align*}
G_i(\lambda) &= \mathcal{O}(e^{-\alpha X_m}) \\
\Delta H &= \mathcal{O}(e^{-\alpha X_m}) \\
D_1 &= ( Q'(X_1) + Q'(-X_1 )(d_2 - d_1 ) + \mathcal{O} \left( e^{-\alpha X_1} \left( |\lambda| +  e^{-\alpha X_1}  \right) |d| \right)
\end{align*}

The idea behind these (unproven) bounds is that deviation from the single pulse, nonperiodic case decreases exponentially with distance from the center, so by defining $Y$ to be the minimum of the two distances we are using, we are letting the ``worst offender'' dictate the behavior. The $D_1$ equation does not depend on $X_2$, so its bound is dictated by $X_1$ alone.\\

For the setup, let

\[
X = (X_0, X_1, X_2) = (X_2, X_1, X_2)
\]

The fixed point equations then become

\begin{align*}
W_i^-(x) = \Phi^s_-(&x, -X_{i-1}; \lambda)a_{i-1}^- + \Phi^u_-(x, 0; \lambda)b_i^- + e^{\nu(\lambda)(x+X_{i-1})} v_-(x; \lambda) \langle v_0(\lambda), w_-(-X_{i-1}; \lambda) \rangle c_{i-1}^- \\
&+ \int_0^x \Phi^u_-(x, y; \lambda)[ G_i^-(\lambda)W_i^-(y) + \lambda^2 d_i \tilde{H}(y) ] dy \\
&+ \int_{-X_{i-1}}^x \Phi^s_-(x, y; \lambda) [ G_i^-(\lambda)W_i^-(y) + \lambda^2 d_i \tilde{H}(y) ] dy \\
&+ \int_{-X_{i-1}}^x 
e^{\nu(\lambda)(x-y)} v_-(x; \lambda) \langle G_i^-(\lambda)(y)W_i^-(y) + \lambda^2 d_i \tilde{H}(y), w_-(y; \lambda) \rangle dy	 \\
W_i^+(x) = \Phi^u_+(&x, X_i; \lambda)a_i^+ + \Phi^s_+(x, 0; \lambda)b_i^+ + e^{\nu(\lambda)(x - X_i)} v_+(x; \lambda) \langle v_0(\lambda), w_+(X_i; \lambda) \rangle c_i^+ \\
&+ \int_0^x \Phi^s_+(x, y; \lambda) [ G_i^+(\lambda)W_i^+(y) + \lambda^2 d_i \tilde{H}(y) ] dy \\
&+ \int_{X_i}^x \Phi^u_+(x, y; \lambda) [ G_i^+(\lambda)W_i^+(y) + \lambda^2 d_i \tilde{H}(y) ] dy \\
&+ \int_{X_i}^x e^{\nu(\lambda)(x-y)} v_+(x; \lambda) \langle G_i^+(\lambda)(y)W_i^+(y) + \lambda^2 d_i \tilde{H}(y), w_+(y; \lambda) \rangle dy
\end{align*}

So now we do the same thing we did with the single pulse. The main difference here is the presence of the $d_i$ in the $\tilde{H}$ terms. \\

We have the following notational conventions, which we use for simplicity. The first one makes sense since periodic BCs basically means that things ``wrap around''.

\begin{enumerate}[(i)]
\item $c_2^- = c_0^-$, $a_2^- = a_0^-$, $b_0^- = b_2^-$, $d_0 = d_2$, and $W_0 = W_2$
\item $D_2 = 0$
\item If we eliminate either a subscript or a superscript in the norm, we are taking the max of the eliminated things, e.g. $|c_i| = \max(|c_i^+|, |c_i^-|)$ or $|c^+| = \max(|c_1^+|, |c_2^+|)$. 
\end{enumerate}

Now we can do the thing.

\begin{enumerate}

\item Fix $\tilde{\alpha}$, with $0 < \tilde{\alpha} < \alpha$. 

\item For convenience, we are taking $\nu(\lambda) \geq 0$. If this works, we will redo it with the appropriate absolute value signs.

\item Solve for $W_i$ in terms of the other stuff. The estimates should be the same as the single pulse, with the addition of the $d$ term, so this should be

\[
||W_1(\lambda)(a,b,c,d)|| \leq C (|a| + |b| + e^{\nu(\lambda)X_M}(|c| + |\lambda|^2 |d| ))
\]

where we used the max $X_M$ here since the growth on the ``center'' space gets worse as we go further from 0.\\

Although this bound certainly works, it would be nice to have separate bounds for the $c_i$ since the two length scales $X_1$ and $X_2$ could be very different. This will turn out to be useful. Thus we have 

\[
||W_1(\lambda)(a,b,c,d)|| \leq C (|a| + |b| + e^{\nu(\lambda)X_1}|c_1| + e^{\nu(\lambda)X_2}|c_2| + e^{\nu(\lambda)X_M}|\lambda|^2 |d| )
\]

\item Solve for the joins which are not at 0, i.e. solve

\begin{align*}
W_2^+(X_2) - W_1^-(-X_2) &= 0 \\
W_1^+(X_1) - W_2^-(-X_1) &= D_1 d \\
\end{align*}

To solve these, we have two equations. The first one is the periodic matching, which we have in the single pulse case.

\begin{align*}
0 &= a_2^+ - a_0^- \\
&+ (P^u_+(X_2; \lambda) - P_0^u)a_2^+ - (P^s_-(-X_2; \lambda) - P_0^s)a_0^- \\
&+ \Phi^s_+(X_2, 0; \lambda)b_2^+ - \Phi^u_-(-X_2, 0; \lambda)b_1^- \\
&+ v_+(X_2; \lambda) \langle v_0(\lambda), w_+(X_2; \lambda) \rangle c_2^+ - v_-(-X_2; \lambda) \langle v_0(\lambda), w_-(-X_2; \lambda) \rangle c_0^- \\
&+ \int_0^{X_2} \Phi^u_+(X_2, y; \lambda) [ G(\lambda)W_2^+(y) + d_2 \lambda^2 \tilde{H}(y) ] dy \\
&- \int_0^{-X_2} \Phi^s_-(-X_2, y; \lambda) [ G(\lambda)W_1^-(y) + d_1 \lambda^2 \tilde{H}(y) ] dy
\end{align*}

The second one is the center matching at $\pm L$.

\begin{align*}
D_1 d &= a_1^+ - a_1^- \\
&+ (P^u_+(X_1; \lambda) - P_0^u)a_1^+ - (P^s_-(-X_1; \lambda) - P_0^s)a_1^- \\
&+ \Phi^s_+(X_1, 0; \lambda)b_1^+ - \Phi^u_-(-X_1, 0; \lambda)b_2^- \\
&+ v_+(X_1; \lambda) \langle v_0(\lambda), w_+(X_1; \lambda) \rangle c_1^+ - v_-(-X_1; \lambda) \langle v_0(\lambda), w_-(-X_1; \lambda) \rangle c_1^- \\
&+ \int_0^{X_1} \Phi^u_+(X_1, y; \lambda) [ G(\lambda)W_1^+(y) + d_1 \lambda^2 \tilde{H}(y) ] dy \\
&- \int_0^{-X_1} \Phi^s_-(-X_1, y; \lambda) [ G(\lambda)W_2^-(y) + d_2 \lambda^2 \tilde{H}(y) ] dy
\end{align*}

Using our notation convention we can write this as


\begin{align*}
D_i d &= a_i^+ - a_i^- \\
&+ (P^u_+(X_i; \lambda) - P_0^u)a_i^+ - (P^s_-(-X_i; \lambda) - P_0^s)a_i^- \\
&+ \Phi^s_+(X_i, 0; \lambda)b_i^+ - \Phi^u_-(-X_i, 0; \lambda)b_{i-1}^- \\
&+ v_+(X_i; \lambda) \langle v_0(\lambda), w_+(X_i; \lambda) \rangle c_i^+ - v_-(-X_i; \lambda) \langle v_0(\lambda), w_-(-X_i; \lambda) \rangle c_i^- \\
&+ \int_0^{X_i} \Phi^u_+(X_i, y; \lambda) [ G(\lambda)W_i^+(y) + d_i \lambda^2 \tilde{H}(y) ] dy \\
&- \int_0^{-X_i} \Phi^s_-(-X_i, y; \lambda) [ G(\lambda)W_{i-1}^-(y) + d_{i-1} \lambda^2 \tilde{H}(y) ] dy
\end{align*}

We follow what we did in the single pulse case, but this time we solve for $c_i^+$ in terms of $c_i^-$ rather than doing the $\Delta c$ thing. First, after manipulating things to get a $c^+ v_0(\lambda)$ term (as in the single pulse version) we rewrite this as

\begin{align*}
D_i d &= a_i^+ - a_i^- + c_i^+ v_0(\lambda) \\
&+ (P^u_+(X_i; \lambda) - P_0^u)a_i^+ - (P^s_-(-X_i; \lambda) - P_0^s)a_i^- \\
&+ \Phi^s_+(X_i, 0; \lambda)b_i^+ - \Phi^u_-(-X_i, 0; \lambda)b_{i-1}^- \\
&+ c_i^+ \Delta v_+(X_i; \lambda) + c_i^+ v_+(X_i; \lambda) \langle v_0(\lambda), \Delta w_+(X_i; \lambda) \rangle \\
&- c_i^- v_-(-X_i; \lambda) \langle v_0(\lambda), w_-(-X_i; \lambda) \rangle \\
&+ \int_0^{X_i} \Phi^u_+(X_i, y; \lambda) [ G(\lambda)W_i^+(y) + d_i \lambda^2 \tilde{H}(y) ] dy \\
&- \int_0^{-X_i} \Phi^s_-(-X_i, y; \lambda) [ G(\lambda)W_{i-1}^-(y) + d_{i-1} \lambda^2 \tilde{H}(y) ] dy
\end{align*}

where

\begin{align*}
\Delta v_\pm(x; \lambda) &= v_\pm(x; \lambda) - v_0(\lambda) \\
\Delta w_\pm(x; \lambda) &= w_\pm(x; \lambda) - w_0(\lambda)
\end{align*}

Then we have 

\begin{align*}
0 &= a_i^+ - a_i^- + c_i^+ v_0(\lambda) + L_3(\lambda)_i(a_i, b, c_i^+, c_i^-, d)
\end{align*}

We have the bound 

\[
|L_3(\lambda)_i(a_i, b, c_i^+, c_i^-, d)| \leq C ( p_1(X_i; \lambda)|a_i|
+ e^{-\alpha X_i}|b| + p_2(X_i; \lambda)|c_i^+| + |c_i^-| + |G| ||W|| + e^{-\tilde{\alpha} X_i} |\lambda^2| |d| )
\]

where

\begin{align*}
p_1(X;\lambda) = \sup_{x \geq X} (|P^u(x;\lambda) - P_0^u| + |P^s(-x;\lambda) - P_0^s|)
\end{align*}

and

\begin{align*}
p_2(T; \lambda) &= |\Delta v_\pm(\pm T, \lambda)| + |\Delta w_\pm(\pm T, \lambda)|\\
&= |v_\pm(\pm T; \lambda) - v_0(\lambda)| + |w_\pm(\pm T; \lambda) - w_0(\lambda)|
\end{align*}

and we did the same $\tilde{\alpha}$ thing that we did for the single pulse. So far this is nice, since it looks like only the initial conditions with subscript $i$ are involved. This, unfortunately, is deceptive since both $W_1$ and $W_2$ are involved, thus via $W$ we involve the other subscript. We see this ``mixing'' when we plug in $W_1$ for $W$.

\begin{align*}
|L_3&(\lambda)_i(a, b, c^+, c^-, d)| \leq C ( p_1(X_i; \lambda)|a_i|
+ e^{-\alpha X_i}|b| + p_2(X_i; \lambda)|c_i^+| + |c_i^-| + e^{-\tilde{\alpha} X_i} |\lambda^2| |d| ) \\
&+ |G| (|a| + |b| + e^{\nu(\lambda)X_1}|c_1| + e^{\nu(\lambda)X_2}|c_2| + e^{\nu(\lambda)X_M}|\lambda|^2 |d| ) \\
&= C( (p_1(X_i; \lambda) + |G|)|a| + (e^{-\alpha X_i} + |G|) |b| + ( p_2(X_i; \lambda) + e^{\nu(\lambda)X_M} |G|) |c^+| \\
&+ (1 + e^{\nu(\lambda)X_i} |G|)|c_i^-| + e^{\nu(\lambda)X_{i-1}} |G||c_{i-1}^-| + (e^{-\tilde{\alpha} X_i} + e^{\nu(\lambda)X_M} |G|) |\lambda|^2 |d| )
\end{align*}

Note that when we plug in $W_1(\lambda)$, each half of this involves both pieces of $W$, so the coefficients mix. Since we have bounds for all of them it should not matter.\\ 

At this point, we have a small complication. Since we have (or are assuming) that $|G|$ is order $e^{-\alpha X_m}$, we will have a term which is order $e^{\nu(\lambda)X_i} e^{-\alpha X_{i-1}}$, and there is no telling what this going to do as we increase our two length parameters. The simplest way to handle this is to have $X_2$ depend on $X_1$, so we can simplify all the things.\\

Since (using the way we have set it up) it is likely that $X_2 > X_1$, we will let $X_2 = k X_1$, where $k \geq 1$. Of course, since this is periodic and we can think of this as happening on a cylinder, what really matters is the ratio of $X_2$ to $X_1$, since we can take the origin to be wherever we want. Thus we have $X_M = k X_1$ and $X_m = X_1$. Using this relationship, our equation becomes

\begin{align*}
|L_3&(\lambda)_1(a, b, c^+, c^-, d)| \leq C (p_1(X_1; \lambda) + |G|)|a| + (e^{-\alpha X_1} + |G|) |b| + ( p_2(X_1; \lambda) + e^{\nu(\lambda) k X_1 } |G|) |c^+| \\
&+ (1 + e^{\nu(\lambda)X_1} |G|)|c_1^-| + e^{\nu(\lambda)k X_1} |G||c_2^-| + (e^{-\tilde{\alpha} X_1} + e^{\nu(\lambda)k X_1} |G|) |\lambda|^2 |d| ) \\
|L_3&(\lambda)_2(a, b, c^+, c^-, d)| \leq C (p_1(k X_1; \lambda) + |G|)|a| + (e^{-\alpha k X_1} + |G|) |b| + ( p_2(k X_1; \lambda) + e^{\nu(\lambda) k X_1 } |G|) |c^+| \\
&+ (1 + e^{\nu(\lambda)k X_1} |G|)|c_2^-| + e^{\nu(\lambda)X_1} |G||c_1^-| + (e^{-\tilde{\alpha} k X_1} + e^{\nu(\lambda)k X_1} |G|) |\lambda|^2 |d| )
\end{align*}

For the inversion to work, we need the following:

\begin{enumerate}[(i)]
\item Choose $\delta > 0$ small
\item Choose $0 < \tilde{\delta} \leq \delta$ such that for all $|\lambda| < \tilde{\delta}$ we also have $|\nu{\lambda}| \leq \delta$. Since $\nu(\lambda)$ has order $\lambda$, this is possible
\item Choose $k$ so that $\delta k < \alpha$. 
\item This implies $\alpha - \nu(\lambda) k > 0$ for all $|\lambda| < \tilde{\delta}$
\item In particular, this means that all the $e^{\nu(\lambda)k X_1} |G|$ terms are uniformly bounded and decay in $X_1$
\end{enumerate}

Having done this, the bound becomes

\begin{align*}
|L_3&(\lambda)_1(a, b, c^+, c^-, d)| \leq C (p_1(X_1; \lambda) + |G|)|a| + (e^{-\alpha X_1} + |G|) |b| + ( p_2(X_1; \lambda) + e^{\nu(\lambda) k X_1 } |G|) |c^+| \\
&+ |c_1^-| + e^{\nu(\lambda)k X_1} |G||c_2^-| + (e^{-\tilde{\alpha} X_1} + e^{\nu(\lambda)k X_1} |G|) |\lambda|^2 |d| ) \\
|L_3&(\lambda)_2(a, b, c^+, c^-, d)| \leq C (p_1(k X_1; \lambda) + |G|)|a| + (e^{-\alpha k X_1} + |G|) |b| + ( p_2(k X_1; \lambda) + e^{\nu(\lambda) k X_1 } |G|) |c^+| \\
&+ |c_2^-| + e^{\nu(\lambda)X_1} |G||c_1^-| + (e^{-\tilde{\alpha} k X_1} + e^{\nu(\lambda)k X_1} |G|) |\lambda|^2 |d| )
\end{align*}

Let $J_1: V_a \times \text{span }\{v_0(\lambda)\} \rightarrow \C^n$ be defined by $J_i(a_i, c_i^+) = (a_i^+ - a_i^-, c_i^+)$. The map $J_i$ is a linear isomorphism. Now consider the map

\[
S_i(a_i, c_i^+) = J_i (a_i, c_i^+) + L_3(\lambda)_i(a_i, 0, c_i^+, 0, 0) = J_i( I + J_i^{-1} L_3(\lambda)_i(a_i, 0, c_i^+, 0, 0))
\]


For suffiently small $\delta$, by what we have above, we can get the operator norm $||J_i^{-1} L_3(\lambda)_i(\cdot, 0, \cdot, 0, 0)|| < 1$, thus the map $(a_i, c^+) \rightarrow I + J_1^{-1} L_3(\lambda)_i(a_i, 0, c^+, 0, 0)$ is invertible and so the operator $S_i$ is invertible.\\

Thus we have

\[
(a_i, c_i^+) = A_1(\lambda)_i(b, c_i, d) = S_i^{-1}(D_i d - L_3(\lambda)_i(0, b, 0, c_i, d))
\]

This will have bound

\begin{align*}
|A_1&(\lambda)_1(b, c^-, d)| \leq C ((e^{-\alpha X_1} + |G|) |b| + |c_1^-| + e^{\nu(\lambda)k X_1} |G||c_2^-| + ((e^{-\tilde{\alpha} X_1} + e^{\nu(\lambda)k X_1} |G|) |\lambda|^2 + |D_1| )|d| ) \\
|A_2&(\lambda)_2(b, c^-, d)| \leq C ((e^{-\alpha k X_1} + |G|) |b| + |c_2^-| + e^{\nu(\lambda) X_1} |G||c_1^-| + (e^{-\tilde{\alpha} k X_1} + e^{\nu(\lambda)k X_1} |G|) |\lambda|^2 |d| ) \\
\end{align*}

The overall bound (worst case scenario) is 

\begin{align*}
|A_1&(\lambda)(b, c^-, d)| \leq C( (e^{-\alpha X_1} + |G|) |b| + |c^-| + (e^{-\tilde{\alpha} X_1} + e^{\nu(\lambda)k X_1} |G|) |\lambda|^2 + |D_1|) |d| )
\end{align*}

We can plug this into our expression for $W_1$ to get $W_2(\lambda)$ which has bound

\begin{align*}
||W_2&(\lambda)(b,c^-,d)|| \leq C (|b| + e^{\nu(\lambda)k X_1}|\lambda|^2 |d|\\
&+ ( (e^{-\alpha X_1} + |G|) |b| + |c^-| + (e^{-\tilde{\alpha} X_1} + e^{\nu(\lambda)k X_1} |G|) |\lambda|^2 + |D_1|) |d| ) \\
&+ e^{\nu(\lambda)X_1} ((e^{-\alpha X_1} + |G|) |b| + |c_1^-| + e^{\nu(\lambda)k X_1} |G||c_2^-| + (e^{-\tilde{\alpha} X_1} + (e^{\nu(\lambda)k X_1} |G|) |\lambda|^2 + |D_1|)|d| ) \\
&+ e^{\nu(\lambda)k X_1} ((e^{-\alpha k X_1} + |G|) |b| + |c_2^-| + e^{\nu(\lambda) X_1} |G||c_1^-| + (e^{-\tilde{\alpha} k X_1} + e^{\nu(\lambda)k X_1} |G|) |\lambda|^2 |d| ) \\
&\leq C( |b| + e^{\nu(\lambda) X_1} |c_1^-| +  e^{\nu(\lambda) k X_1} |c_2^-| + (e^{\nu(\lambda)k X_1}|\lambda|^2 + e^{\nu(\lambda) X_1} |D_1|) |d| )
\end{align*}

where we are now taking $(k + 1)|\nu(\lambda)| < \alpha$.\\

With a multipulse, we will need an analogue of (3.25) in Sandstede (1998). Starting with the $X_1$-match (the other one is not useful, since there is no term in $D_1 d$), we have

\[
D_1 d = a_1^+ - a_1^- + c_1^+ v_0(\lambda) + L_3(\lambda)_1(a_i, b, c_i^+, c_i^-, d)
\]

we write this as

\[
a_1^+ - a_1^- + c_1^+ v_0(0) = -D_1 d - c_1^+ (v_0(\lambda) - v_0(0)) - L_3(\lambda)_1(a_1, b, c_1^+, c_1^-, d)
\]

Then we take projections $P^s_0$ and $P^u_0$. Recalling where the $a_1^\pm$ live and that $v_0(0)$ is wiped out by both projections, this becomes 

\begin{align*}
a_1^+ &= -P^u_0 D_1 d - c_1^+ P^u_0 (v_0(\lambda) - v_0(0)) - P^u_0 L_3(\lambda)_1(a_1, b, c_1^+, c_1^-, d) \\
a_1^- &=  P^s_0 D_1 d + c_1^+ P^s_0 (v_0(\lambda) + v_0(0)) - P^s_0 L_3(\lambda)_1(a_1, b, c_1^+, c_1^-, d)
\end{align*}

Defining $A_2$ to be all the stuff on the RHS other than the $D_1 d$ term, we have

\begin{align*}
a_1^+ &= -P^u_0 D_1 d + A_2(\lambda)_1^+(b, c_1^-, d))\\
a_1^- &=  P^s_0 D_1 d + A_2(\lambda)_1^-(b, c_1^-, d))
\end{align*}

where we will plug $A_1(\lambda)_1$ in for $a_1$ and $c_1^+$. Let 

\[
p_5(\lambda) = |v_0(\lambda) - v_0(0)| 
\]

Then the bound for $A_2$ is

\begin{align*}
|A_2&(\lambda)_1(b, c_1^-, d))| \\
&\leq C( p_5(\lambda) |c_1^-| + (p_1(X_1; \lambda) + |G|)|a| + (e^{-\alpha X_1} + |G|) |b| + ( p_2(X_1; \lambda) + e^{\nu(\lambda)X_M} |G|) |c^+| \\
&+ |c_1^-| + (e^{-\tilde{\alpha} X_1} + e^{\nu(\lambda)X_M} |G|) |\lambda|^2 |d| ) ) \\
&= C( (e^{-\alpha X_1} + |G|) |b| + p_5(\lambda) |c_1^-| + |c_1^-| + (e^{-\tilde{\alpha} X_1} + e^{\nu(\lambda)X_M} |G|) |\lambda|^2 |d| \\
&+ (p_1(X_1; \lambda) + p_2(X_1; \lambda) + e^{\nu(\lambda)X_M} |G|) \\
&\:\:\:\:\:( ( e^{-\alpha X_1} + |G|) |b| + |c_1^-| + (e^{-\tilde{\alpha} X_1} + e^{\nu(\lambda)X_M} |G|) |\lambda|^2 + |D_1|) |d| )\\
&\leq C( (e^{-\alpha X_1} + |G|) |b| + |c_1^-| \\
&+ ((e^{-\tilde{\alpha} X_1} + e^{\nu(\lambda)X_M} |G|) |\lambda|^2 + (p_1(X_1; \lambda) + p_2(X_1; \lambda) + e^{\nu(\lambda)X_M} |G|)|D_1|)|d| )\\
&\leq C( (e^{-\alpha X_1} + |G|) |b| + |c_1^-| \\
&+ ((e^{-\tilde{\alpha} X_1} + e^{\nu(\lambda)k X_1} |G|) |\lambda|^2 + (p_1(X_1; \lambda) + p_2(X_1; \lambda) + e^{\nu(\lambda) k X_1} |G|)|D_1|)|d| )
\end{align*}

\item Next we use the projections 

\begin{align*}
P(\C Q'(0))W_i^-(0) &= 0 \\
P(\C Q'(0))W_i^+(0) &= 0 \\
P(Y^+ \oplus Y^- \oplus Y^0) ( W_i^+(0) - W_i^-(0) ) &= 0
\end{align*}

to implement what happens at $x = 0$ (on both pieces). Note that the $X_1$ and $X_2$ terms will mix here. At $x = 0$, the fixed point equations become

\begin{align*}
W_i^-(0) = \Phi^s_-(&0, -X_{i-1}; \lambda)a_{i-1}^- + b_i^- + (P^u_-(0; \lambda) - P^u_-(0; 0))b_i^- \\
&+ e^{\nu(\lambda)(X_{i-1})} v_-(0; \lambda) \langle v_0(\lambda), w_-(-X_{i-1}; \lambda) \rangle c_{i-1}^- \\
&+ \int_{-X_{i-1}}^0 \Phi^s_-(0, y; \lambda) [ G_i^-(\lambda)W_i^-(y) + \lambda^2 d_i \tilde{H}(y) ] dy \\
&+ \int_{-X_{i-1}}^0
e^{\nu(\lambda)(y)} v_-(0; \lambda) \langle G_i^-(\lambda)(y)W_i^-(y) + \lambda^2 d_i \tilde{H}(y), w_-(y; \lambda) \rangle dy \\
W_i^+(0) = \Phi^u_+(&0, X_i; \lambda)a_i^+ + b_i^+ + (P^s_+(0; \lambda) - P^s_-(0; 0))b_i^+ \\
&+ e^{\nu(\lambda)(-X_i)} v_+(0; \lambda) \langle v_0(\lambda), w_+(X_i; \lambda) \rangle c_i^+ \\
&+ \int_{X_i}^0 \Phi^u_+(0, y; \lambda) [ G_i^+(\lambda)W_i^+(y) + \lambda^2 d_i \tilde{H}(y) ] dy \\
&+ \int_{X_i}^0 e^{\nu(\lambda)(-y)} v_+(0; \lambda) \langle G_i^+(\lambda)(y)W_i^+(y) + \lambda^2 d_i \tilde{H}(y), w_+(y; \lambda) \rangle dy
\end{align*}

Using the same setup as in the single pulse case and doing the same projections, we would like to get something like

\[
\begin{pmatrix}x^- \\ x_i^+ \\ y_i^+ - y_i^- + \text{some stuff involving $c_i^-$} \end{pmatrix}+ L_4(\lambda)(b, c_i^-,d) = 0
\]

Let's see what we can get. First, let's look at the 3rd component of this. The bound on $L_4$ will be determined by this, since the same bound will hold for the first two components.

\begin{align*}
W_i^+(0) - W_i^-(0) &= \Phi^u_+(0, X_i; \lambda)a_i^+ - \Phi^s_-(0, -X_{i-1}; \lambda)a_{i-1}^- \\
&+ b_i^+ - b_i^- + (P^s_+(0; \lambda) - P^s_-(0; 0))b_i^+  - (P^u_-(0; \lambda) - P^u_-(0; 0))b_i^- \\
&+ e^{-\nu(\lambda)X_i} c_i^+ y_0 + e^{-\nu(\lambda)X_i} c_i^+( (v_+(0; \lambda) - y_0) + v_+(0; \lambda) \langle  v_0(\lambda), \Delta w_+(X_i; \lambda) \rangle) \\
&- e^{\nu(\lambda)X_{i-1}} c_{i-1}^- y_0 - e^{\nu(\lambda)X_{i-1}} c_{i-1}^- ( (v_-(0; \lambda) - y_0) + v_-(0; \lambda) \langle  v_0(\lambda), \Delta w_-(-X_{i-1}; \lambda) \rangle) \\
&+ \int_{-X_{i-1}}^0 \Phi^s_-(0, y; \lambda) [ G_i^-(\lambda)W_i^-(y) + \lambda^2 d_i \tilde{H}(y) ] dy \\
&+ \int_{-X_{i-1}}^0
e^{\nu(\lambda)(y)} v_-(0; \lambda) \langle G_i^-(\lambda)(y)W_i^-(y) + \lambda^2 d_i \tilde{H}(y), w_-(y; \lambda) \rangle dy \\
&+ \int_{X_i}^0 \Phi^u_+(0, y; \lambda) [ G_i^+(\lambda)W_i^+(y) + \lambda^2 d_i \tilde{H}(y) ] dy \\
&+ \int_{X_i}^0 e^{\nu(\lambda)(-y)} v_+(0; \lambda) \langle G_i^+(\lambda)(y)W_i^+(y) + \lambda^2 d_i \tilde{H}(y), w_+(y; \lambda) \rangle dy
\end{align*}

Then our matrix-equation becomes

\[
\begin{pmatrix}x_i^- \\ x_i^+ \\ y_i^+ - y_i^- - e^{\nu(\lambda)X_{i-1}} c_{i-1}^- y_0 \end{pmatrix} + L_4(\lambda)_i(b_i, e^{\nu(\lambda)X_{i-1}} c_{i-1}^-, d) = 0
\]

It turns out that this will not quite work. The basic reason this is the case is that since we do not know the sign of $\nu(\lambda)$, thus we know nothing about the magnitude of $e^{\nu(\lambda)X_{i-1}}$. We can hopefully get around this by adding and subtracting the ``other-sign'' version in the ``negative'' fixed point equation at $x = 0$.

\begin{align*}
W_i^-(0) = \Phi^s_-(&0, -X_{i-1}; \lambda)a_{i-1}^- + b_i^- + (P^u_-(0; \lambda) - P^u_-(0; 0))b_i^- \\
&+ (e^{\nu(\lambda)X_{i-1}} + e^{-\nu(\lambda)X_{i-1}} - e^{-\nu(\lambda)X_{i-1}}) c_{i-1}^- y_0 \\
&+ e^{\nu(\lambda)X_{i-1}} c_{i-1}^- ( (v_-(0; \lambda) - y_0) + v_-(0; \lambda) \langle  v_0(\lambda), \Delta w_-(-X_{i-1}; \lambda) \rangle) \\
&+ \int_{-X_{i-1}}^0 \Phi^s_-(0, y; \lambda) [ G_i^-(\lambda)W_i^-(y) + \lambda^2 d_i \tilde{H}(y) ] dy \\
&+ \int_{-X_{i-1}}^0
e^{\nu(\lambda)(y)} v_-(0; \lambda) \langle G_i^-(\lambda)(y)W_i^-(y) + \lambda^2 d_i \tilde{H}(y), w_-(y; \lambda) \rangle dy
\end{align*}

Making this change, we have

\[
\begin{pmatrix}x_i^- \\ x_i^+ \\ y_i^+ - y_i^- - (e^{\nu(\lambda)X_{i-1}} + e^{-\nu(\lambda)X_{i-1}})c_{i-1}^- y_0 \end{pmatrix} + L_4(\lambda)_i(b_i, e^{\nu(\lambda)X_{i-1}} c_{i-1}^-, d) = 0
\]

Note that the RHS does not match the LHS, but we will get there. For convenience, let

\begin{align*}
f(X; \lambda) = e^{\nu(\lambda)X} + e^{-\nu(\lambda)X} &= 2 \cosh (\nu(\lambda) X) \\
f^+(X; \lambda) &= e^{\nu(\lambda)X} / f(X; \lambda) \\
f^-(X; \lambda) &= e^{-\nu(\lambda)X} / f(X; \lambda)
\end{align*}

This is nice, since $|f(X; \lambda)| \geq 1|$ and $0 < |f^\pm(X; \lambda)| < 1$. So we have

\[
\begin{pmatrix}x_i^- \\ x_i^+ \\ y_i^+ - y_i^- - f(X_{i-1}; \lambda)c_{i-1}^- y_0 \end{pmatrix} + L_4(\lambda)_i(b_i, f(X_{i-1}; \lambda) c_{i-1}^-, d) = 0
\]

where we have written RHS to match the LHS. Of course, we need to show that is possible, which is what we do now. Once again, we look at $W_i^+(0) - W_i^-(0)$.

\begin{align*}
W_i^+(0) - W_i^-(0) &= b_i^+ - b_i^- - f(X_{i-1}; \lambda) c_{i-1}^- y_0 \\
&+ \Phi^u_+(0, X_i; \lambda)a_i^+ - \Phi^s_-(0, -X_{i-1}; \lambda)a_{i-1}^- \\
&+(P^s_+(0; \lambda) - P^s_-(0; 0))b_i^+  - (P^u_-(0; \lambda) - P^u_-(0; 0))b_i^- \\
&+ c_i^+ e^{-\nu(\lambda)X_i} v_+(0; \lambda) \langle v_0(\lambda), w_+(X_i; \lambda) \rangle c_i^+ \\
&+ e^{-\nu(\lambda)X_{i-1}} c_{i-1}^- y_0 - e^{\nu(\lambda)X_{i-1}} c_{i-1}^- ( (v_-(0; \lambda) - y_0) + v_-(0; \lambda) \langle  v_0(\lambda), \Delta w_-(-X_{i-1}; \lambda) \rangle) \\
&+ \int_{-X_{i-1}}^0 \Phi^s_-(0, y; \lambda) [ G_i^-(\lambda)W_i^-(y) + \lambda^2 d_i \tilde{H}(y) ] dy \\
&+ \int_{-X_{i-1}}^0
e^{\nu(\lambda)(y)} v_-(0; \lambda) \langle G_i^-(\lambda)(y)W_i^-(y) + \lambda^2 d_i \tilde{H}(y), w_-(y; \lambda) \rangle dy \\
&+ \int_{X_i}^0 \Phi^u_+(0, y; \lambda) [ G_i^+(\lambda)W_i^+(y) + \lambda^2 d_i \tilde{H}(y) ] dy \\
&+ \int_{X_i}^0 e^{\nu(\lambda)(-y)} v_+(0; \lambda) \langle G_i^+(\lambda)(y)W_i^+(y) + \lambda^2 d_i \tilde{H}(y), w_+(y; \lambda) \rangle dy
\end{align*}

Upon substituting the various $f$ things, this becomes
 
\begin{align*}
W_i^+(0) - W_i^-(0) &= b_i^+ - b_i^- - f(X_{i-1}; \lambda) c_{i-1}^- y_0 \\
&+ \Phi^u_+(0, X_i; \lambda)a_i^+ - \Phi^s_-(0, -X_{i-1}; \lambda)a_{i-1}^- \\
&+(P^s_+(0; \lambda) - P^s_-(0; 0))b_i^+  - (P^u_-(0; \lambda) - P^u_-(0; 0))b_i^- \\
&+ c_i^+ e^{-\nu(\lambda)X_i} v_+(0; \lambda) \langle v_0(\lambda), w_+(X_i; \lambda) \rangle \\
&+ f^-(X_{i-1}; \lambda)f(X_{i-1}; \lambda) c_{i-1}^- y_0 \\
&- f^+(X_{i-1}; \lambda)f(X_{i-1}; \lambda) c_{i-1}^- ( (v_-(0; \lambda) - y_0) + v_-(0; \lambda) \langle  v_0(\lambda), \Delta w_-(-X_{i-1}; \lambda) \rangle) \\
&+ \int_{-X_{i-1}}^0 \Phi^s_-(0, y; \lambda) [ G_i^-(\lambda)W_i^-(y) + \lambda^2 d_i \tilde{H}(y) ] dy \\
&+ \int_{-X_{i-1}}^0
e^{\nu(\lambda)(y)} v_-(0; \lambda) \langle G_i^-(\lambda)(y)W_i^-(y) + \lambda^2 d_i \tilde{H}(y), w_-(y; \lambda) \rangle dy \\
&+ \int_{X_i}^0 \Phi^u_+(0, y; \lambda) [ G_i^+(\lambda)W_i^+(y) + \lambda^2 d_i \tilde{H}(y) ] dy \\
&+ \int_{X_i}^0 e^{\nu(\lambda)(-y)} v_+(0; \lambda) \langle G_i^+(\lambda)(y)W_i^+(y) + \lambda^2 d_i \tilde{H}(y), w_+(y; \lambda) \rangle dy
\end{align*}

We have the following bound for $L_4$, where we use the $\tilde{\alpha}$ trick like we did for the single pulse. Note that the bound will depend on all the things in $W_i^+(0) - W_i^-(0)$ other than the first line.

\begin{align*}
|L_4&(\lambda)_i(b_i, f(X_{i-1}; \lambda) c_{i-1}^-, d)|\\ 
&\leq C( e^{- \alpha X_i} |a_i^+| +  e^{-\alpha X_{i-1}} |a_{i-1}^-| \\
&+ p_3(\lambda) |b_i| + (f^-(X_{i-1}; \lambda) + f^+(X_{i-1}; \lambda) (p_2(\lambda; X_{i-1}) + p_4(\lambda)) | f(X_{i-1}; \lambda) c_{i-1}^-| \\
&+ e^{\nu(\lambda)X_{i}} |c_i^+| \\
&+ e^{\nu(\lambda)X_M}|G| ||W|| + |\lambda^2| |d|)
\end{align*}

where

\[
p_3(\lambda) = |P^u_-(0;\lambda) - P^u_-(0; 0)| + |P^s_+(0;\lambda) - P^s_+(0;0)|
\]

and

\[
p_4(\lambda) = |v_-(0; \lambda) - y_0| + |v_+(0; \lambda) - y_0|
\]

To finish the bound, we need to plug in $A_1$ and $W_2$. Note that because of the ``mixing'' of coefficents ($a_i$ and $c_i$), we need to plug in both $A_1(\lambda)_i$ and $A_1(\lambda)_{i-1}$. For now, since we only have $i = 1, 2$, let's do them separately. Recall that $X_0 = X_2$ and the other ``wraparounds'' we have above. For $i = 1$, we have

\begin{align*}
|L_4&(\lambda)_1(b_1, f(X_2; \lambda) c_2^-, d)|\\ 
&\leq C( e^{-\alpha X_1} |a_1^+| + e^{-\alpha X_2} |a_2^-| \\
&+ p_3(\lambda) |b_1| \\
&+ (f^-(X_2; \lambda) + f^+(X_2; \lambda) (p_2(\lambda; X_2) + p_4(\lambda)) | f(X_2; \lambda) c_2^-| \\
&+ e^{\nu(\lambda)X_1} |c_1^+| \\
&+ e^{\nu(\lambda)X_M}|G| ||W|| + |\lambda^2||d|)
\end{align*}

Making the substitutions, this becomes

\begin{align*}
|L_4&(\lambda)_1(b_1, f(X_2; \lambda) c_2^-, d)|\\ 
&\leq C( e^{-\alpha X_1}((e^{-\alpha X_1} + |G|) |b| + |c_1^-| + e^{\nu(\lambda)k X_1} |G||c_2^-| + ((e^{-\tilde{\alpha} X_1} + e^{\nu(\lambda)k X_1} |G|) |\lambda|^2 + |D_1| )|d| ) \\
&+ e^{-\alpha X_2} ((e^{-\alpha k X_1} + |G|) |b| + |c_2^-| + e^{\nu(\lambda) X_1} |G||c_1^-| + (e^{-\tilde{\alpha} k X_1} + e^{\nu(\lambda)k X_1} |G|) |\lambda|^2 |d| ) \\
&+ p_3(\lambda) |b_1| \\
&+ (f^-(X_2; \lambda) + f^+(X_2; \lambda) (p_2(\lambda; X_2) + p_4(\lambda)) | f(X_2; \lambda) c_2^-| \\
&+ e^{\nu(\lambda)X_1} ((e^{-\alpha X_1} + |G|) |b| + |c_1^-| + e^{\nu(\lambda)k X_1} |G||c_2^-| + ((e^{-\tilde{\alpha} X_1} + e^{\nu(\lambda)k X_1} |G|) |\lambda|^2 + |D_1| )|d| ) \\
&+ e^{\nu(\lambda)k X_1}|G| ( |b| + e^{\nu(\lambda) X_1} |c_1^-| +  e^{\nu(\lambda) k X_1} |c_2^-| + (e^{\nu(\lambda)k X_1}|\lambda|^2 + e^{\nu(\lambda) X_1} |D_1|) |d| )\\
&+ |\lambda^2||d|) \\
&\leq C( ( p_3(\lambda) + e^{\nu(\lambda)X_1} (e^{-\alpha X_1} + |G|)) |b| \\
&+ e^{\nu(\lambda)X_1} |c_1^-| + e^{2 \nu(\lambda)k X_1}|G| |c_2^-| \\
&+ (f^-(X_2; \lambda) + f^+(X_2; \lambda) (p_2(\lambda; X_2) + p_4(\lambda)) | f(X_2; \lambda) c_2^-| \\
&+ (1 + e^{2 \nu(\lambda)k X_1}|G|)|\lambda^2||d| \\
&+ (e^{\nu(\lambda)X_1} + e^{2 \nu(\lambda)k X_1}|G| )|D_1||d
\end{align*}

At this point, to make things simpler, we will take $k$ sufficiently small so that $2 \delta k < \alpha$. Then since $|G|$ is of order $e^{-\alpha X_1}$, all the terms that look like $e^{2 \nu(\lambda)k X_1}|G|$ are higher order, so this becomes

\begin{align*}
|L_4&(\lambda)_1(b_1, f(X_2; \lambda) c_2^-, d)|\\ 
&\leq C( ( p_3(\lambda) + e^{\nu(\lambda)X_1} (e^{-\alpha X_1} + |G|)) |b| \\
&+ e^{\nu(\lambda)X_1} |c_1^-| + e^{2 \nu(\lambda)k X_1}|G| |c_2^-| \\
&+ (f^-(k X_1; \lambda) + f^+(k X_1; \lambda) (p_2(\lambda; k X_1) + p_4(\lambda)) | f(k X_1; \lambda) c_2^-| \\
&+ (|\lambda^2| + e^{\nu(\lambda)X_1}|D_1|)|d| \\
&\leq C( ( p_3(\lambda) + e^{\nu(\lambda)X_1} (e^{-\alpha X_1} + |G|)) |b| \\
&+ f^-( X_1; \lambda) |f(X_1; \lambda) c_1^-| + e^{\nu(\lambda)k X_1}|G| f^+( k X_1; \lambda)| f(k X_1; \lambda) c_2^-| \\
&+ (f^-(k X_1; \lambda) + f^+(k X_1; \lambda) (p_2(\lambda; k X_1) + p_4(\lambda)) | f(k X_1; \lambda) c_2^-| \\
&+ (|\lambda^2| + e^{\nu(\lambda)X_1}|D_1|)|d| \\
&\leq C( ( p_3(\lambda) + e^{\nu(\lambda)X_1} (e^{-\alpha X_1} + |G|)) |b| \\
&+ f^-( X_1; \lambda) |f(X_1; \lambda) c_1^-| \\
&+ (f^-(k X_1; \lambda) + f^+(k X_1; \lambda) (p_2(\lambda; k X_1) + p_4(\lambda) + e^{\nu(\lambda)k X_1}|G| ) | f(k X_1; \lambda) c_2^-| \\
&+ (|\lambda^2| + e^{\nu(\lambda)X_1}|D_1|)|d|
\end{align*}


To do the inversion, we need for the coefficients of $b_i$, $| f(k X_1; \lambda) c_2^-| $, and $| f(X_1; \lambda) c_1^-|$ to have magnitude less than 1. For $b_i$ we are all set, since $p_3(\lambda)$ is order $\lambda$, and we can choose $\lambda$ as small as we want. For $|f(X_1; \lambda) c_1^-|$ we are all set since $f^-(X_1; \lambda) < 1$. For $|f(k X_1; \lambda) c_2^-|$, we note that 

\[
f^-(k X_1; \lambda) + f^+(k X_1; \lambda) = 1
\]

and both terms on the LHS are positive. Since $(p_2(\lambda; X_{i-1}) + p_4(\lambda)) + e^{\nu(\lambda)k X_1}|G| $ can be made as small as we want by choosing $\lambda$ sufficiently small and $X_1$ sufficiently large, we can obtain a coefficient of $| f(k X_1; \lambda) c_2^-|$ of magnitude less than 1. Thus we can do the inversion.\\

Assume that the bound for $|L_4&(\lambda)_2|$ is similar, i.e. we swap the 1s and 2s for this, or something close. Something like

\begin{align*}
|L_4&(\lambda)_2(b_2, f(X_1; \lambda) c_1^-, d)|\\ 
&\leq C( ( p_3(\lambda) + e^{\nu(\lambda)k X_1} (e^{-\alpha k X_1} + |G|)) |b| \\
&+ f^-( k X_1; \lambda) |f(k X_1; \lambda) c_2^-| \\
&+ (f^-(X_1; \lambda) + f^+(X_1; \lambda) (p_2(\lambda; X_1) + p_4(\lambda) + e^{\nu(\lambda)X_1}|G| ) | f(X_1; \lambda) c_1^-| \\
&+ (|\lambda^2| + e^{\nu(\lambda)k X_1}|D_1|)|d|
\end{align*} 

Then we can solve for $b$ and $f(X_i; \lambda) c_i^-$ to get $(b_i, f(X_i; \lambda) c_i^-) = B_1(\lambda)(d)$ which should have a bound like

\begin{align*}
|B_1(\lambda(d)|_1 \leq C(|\lambda^2| + e^{\nu(\lambda) X_1}|D_1|)|d| \\
|B_1(\lambda(d)|_2 \leq C(|\lambda^2| + e^{\nu(\lambda)k X_1}|D_1|)|d|
\end{align*}

Note that since this solves for $f(X_i; \lambda) c_i^-$, $c_i^i$ is of order $e^{-|\nu(\lambda)|X_i}$ multiplied by this, so the term in front of the $D_1$ cancels.\\

Then we can plug this into all the other things to get


\begin{align*}
|A_3&(\lambda)_1(b, c^-, d)| \leq C ((e^{-\alpha X_1} + |G|) (|\lambda^2| + e^{\nu(\lambda)k X_1}|D_1|)|d| \\
&+ e^{-|\nu(\lambda)| X_1}(|\lambda^2| + e^{\nu(\lambda) X_1}|D_1|)|d| \\
&+ e^{-|\nu(\lambda)| k X_1} e^{\nu(\lambda)k X_1} |G| (|\lambda^2| + e^{\nu(\lambda) k X_1}|D_1|)|d|\\
&+ ((e^{-\tilde{\alpha} X_1} + e^{\nu(\lambda)k X_1} |G|) |\lambda|^2 + |D_1| )|d| ) \\
&\leq C( |\lambda|^2 + |D_1|) |d| )
\end{align*}

Similarly, 

\begin{align*}
|A_2&(\lambda)_2(b, c^-, d)| \leq C( |\lambda|^2 + |D_1|) |d| )
\end{align*}

We also have

\begin{align*}
||W_3&(\lambda)(b,c^-,d)|| \leq C( (|\lambda^2| + e^{\nu(\lambda) k X_1}|D_1|)|d| \\
&+ e^{\nu(\lambda) X_1} e^{-|\nu(\lambda)| X_1} (|\lambda^2| + e^{\nu(\lambda) X_1}|D_1|)|d| \\
&+ e^{\nu(\lambda) k X_1} e^{-|\nu(\lambda)| X_1} (|\lambda^2| + e^{\nu(\lambda) k X_1}|D_1|)|d|\\
&+ (e^{\nu(\lambda)k X_1}|\lambda|^2 + e^{\nu(\lambda) X_1} |D_1|) |d| )\\
&\leq C e^{\nu(\lambda)k X_1} ( |\lambda|^2 + |D_1| )|d|
\end{align*}

\begin{align*}
|A_4&(\lambda)_1(d)| \\
&\leq C( (e^{-\alpha X_1} + |G|) (|\lambda^2| + e^{\nu(\lambda)k X_1}|D_1|)|d| \\ 
&+ e^{-|\nu(\lambda)| X_1} (|\lambda^2| + e^{\nu(\lambda)X_1}|D_1|)|d| \\
&+ ((e^{-\tilde{\alpha} X_1} + e^{\nu(\lambda)k X_1} |G|) |\lambda|^2 + (p_1(X_1; \lambda) + p_2(X_1; \lambda) + e^{\nu(\lambda) k X_1} |G|)|D_1|)|d| )\\
&\leq C(|\lambda|^2 + |D_1|) |d|
\end{align*}



\end{enumerate}

\end{document}