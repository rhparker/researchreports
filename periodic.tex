\documentclass[12pt]{article}
\usepackage[pdfborder={0 0 0.5 [3 2]}]{hyperref}%
\usepackage[left=1in,right=1in,top=1in,bottom=1in]{geometry}%
\usepackage[shortalphabetic]{amsrefs}%
\usepackage{amsmath}
\usepackage{enumerate}
% \usepackage{enumitem}
\usepackage{amssymb}                
\usepackage{amsmath}                
\usepackage{amsfonts}
\usepackage{amsthm}
\usepackage{bbm}
\usepackage[table,xcdraw]{xcolor}
\usepackage{tikz}
\usepackage{float}
\usepackage{booktabs}
\usepackage{svg}
\usepackage{mathtools}
\usepackage{cool}
\usepackage{url}
\usepackage{graphicx,epsfig}
\usepackage{makecell}
\usepackage{array}

\def\noi{\noindent}
\def\T{{\mathbb T}}
\def\R{{\mathbb R}}
\def\N{{\mathbb N}}
\def\C{{\mathbb C}}
\def\Z{{\mathbb Z}}
\def\P{{\mathbb P}}
\def\E{{\mathbb E}}
\def\Q{\mathbb{Q}}
\def\ind{{\mathbb I}}

\graphicspath{ {images17/} }

\newtheorem{lemma}{Lemma}
\newtheorem{definition}{Definition}
\newtheorem{assumption}{Assumption}
\newtheorem{hypothesis}{Hypothesis}

\begin{document}

\section*{KdV5 with Periodic Boundary Conditions}

Here we look at the case with periodic boundary conditions. Note that we cannot use an exponential weight in this case since our domain is of finite length. As before, we are looking to solve the system

\begin{enumerate}[(i)]
\item $(W_i^\pm)' = A(q) W_i^\pm + G_i^\pm W_i^\pm + \lambda B W_i^\pm + \lambda^2 d_i \tilde{H}_i^\pm$
\item $W_i^\pm(0) \in \C \psi(0) \oplus Y^+ \oplus Y^-$
\item $W_i^+(0) - W_i^-(0) \in \C \psi(0) $
\item $W_1^+(L) - W_2^-(-L) = D_1 d $
\end{enumerate}

which is the piecewise eigenvalue problem written as a first order system in the unweighted norm. The eigenfunction here is given by $V_i^\pm = (Q' - \lambda Q_c)d_i + W_i\pm$.\\

Let's first look at this for the case of a single pulse, where there is only one join at the center, so the eigenfunction is given by $V\pm = Q' - \lambda Q_c + W_i\pm$, and this becomes

\begin{enumerate}[(i)]
\item $(W^\pm)' = A(q) W_i^\pm + \lambda B W^\pm - \lambda^2 B Q_c$
\item $W^\pm(0) \in \C \psi(0) \oplus Y^+ \oplus Y^-$
\item $W^+(0) - W^-(0) \in \C \psi(0) $
\end{enumerate}

Now we would like to do this with periodic BCs on the interval $[-T, T]$. For now, we state the following as a hypothesis.

\begin{hypothesis}\label{qpexists}
For some value of $c$ (likely $> 1/4$) a periodic wavetrain solution $q_p(x)$ exists with period $2T$ for $T$ sufficiently large. We can choose the interval $[-T,T]$ so that the center of one of the peaks is at 0. The solution $q_p(x)$ is an even function on $[-T, T]$. The periodic solution is close to the single pulse solution $q(x)$ on $[-T, T]$, where we have the bound

\[
||q_p(x) - q(x)||_{[-T, T]} \leq C e^{-\alpha T}
\]

which makes sense since the larger $T$ is, the more $q_p$ should resemble the single pulse on $[-T, T]$.
\end{hypothesis}

We need one more condition to ensure the eigenfunction satisfies the BCs. By Hypothesis \ref{qpexists}, a wavetrain solution $q_p(x)$ exists, thus this function and all its derivatives must match at the endpoints of $[-T, T]$.  Thus for the periodic boundary condition for the eigenfunction we must have

\begin{align*}
Q_p'(-T) - \lambda (Q_p)c(-T) + W_i^-(-T) &= Q_p'(T) - \lambda (Q_p)_c(T) + W_i^+(T) \\
W_i^-(-T) - W_i^+(T) &= ( Q_p'(T) - Q_p'(-T) ) - \lambda( (Q_p)_c(T) - (Q_p)_c(-T) ) \\
&= 0
\end{align*}

Thus the system we are looking to solve in the periodic case is

\begin{enumerate}[(i)]
\item $(W^\pm)' = A(q_p) W_i^\pm + \lambda B W^\pm - \lambda^2 B (Q_p)_c$
\item $W^\pm(0) \in \C \psi(0) \oplus Y^+ \oplus Y^-$
\item $W^+(0) - W^-(0) \in \C \psi(0) $
\item $W_i^-(-T) - W_i^+(T) = 0 $
\end{enumerate}

Before we go to town on this, we will rewrite this in the following way. Let

\[
\tilde{A}(\lambda, q_p) = A(q_p) + \lambda B
\]

Then the first equation becomes 

\[
(W^\pm)' = \tilde{A}(\lambda, q_p) W_i^\pm - \lambda^2 B (Q_p)_c
\]

We would also like to write this in terms of the single pulse $q$ rather than the periodic solution $q_p$. To do this, let

\begin{align*}
\tilde{H} &= -B(Q_p)_c \\
H &= -BQ_c \\
G(\lambda) &= \tilde{A}(\lambda, q_p) - \tilde{A}(\lambda, q)
\end{align*}

Then the first equation becomes 

\[
(W^\pm)' = \tilde{A}(\lambda, q) W_i^\pm + G(\lambda) W_i^\pm + \lambda^2 \tilde{H}
\]

For now, we will assume we have these bounds (based on Hypothesis \ref{qpexists}).

\begin{align*}
|G(\lambda)(x)| &\leq C e^{-\alpha T} && x \in [-T, T]\\
|\tilde{H}(x) - H(x)| &\leq C e^{-\alpha T} && x \in [-T, T]
\end{align*}

Now we look at the eigenvalue problem

\begin{equation}
U' = \tilde{A}(\lambda, q) U = A(\lambda) U + R(q(x)) U
\end{equation}

where we split up $\tilde{A}(\lambda, q)$ so that all of the $x$ dependence is via the single pulse $q(x)$ and is contained in $R(x)$. Everything else (including the $\lambda$) is in the matrix $A(\lambda)$, which is a constant matrix and different from the $A$ above, but we are seriously running out of letters, so it will do for now. The matrix $\tilde{A}(\lambda, q)$ is exponentially asymptotic (due to the exponential decay properties of $q$ and its derivatives), and its limit as $|x| \rightarrow \infty$ is $A(\lambda)$.\\

For $\lambda = 0$, $A(\lambda)$ has an eigenvalue at 0. The characteristic polynomial for $A(\lambda)$ is $f(\lambda, \nu) = \lambda - c \nu + \nu^3 - \nu^5$, which to leading order is $f(\lambda, \nu) = \lambda - c \nu + \mathcal{O}(\nu^3)$. Thus, to leading order, this polynomial has a zero when $t = \lambda / c$, which is approximately the value of the spatial eigenvalue closest to 0. For small $\lambda$, numerics shows this is really close.\\

Before we continue, we will prove the following lemma, based on Exercise 29 on p. 104 of Coddington and Levinson (1955).

\begin{lemma}Consider the eigenvalue problem

\begin{equation}\label{veigproblem}
V(x)' = AV(x) + R(x)V(x)
\end{equation}

where $V(x) \in R^n$, $A$ is a constant, diagonalizable $n \times n$ matrix, and $R(x): \R \rightarrow \R^n$ is an integrable function which is globally Lipschitz continuous in $x$. Let $\nu$ be any eigenvalue of $A$ with corresponding eigenvector $p$, i.e. $A p = \nu p$. Then there is a unique solution $\phi(x)$ to \eqref{veigproblem} such that 

\[
\lim_{x\rightarrow\infty} \phi(x) e^{-\nu x} = p
\]

In other words, for large $x$, the solution $\phi(x)$ resembles that of the constant coefficient eigenvalue problem $V(x)' = AV(x)$.

\begin{proof}
Let $\sigma = \text{Re} \nu$. Let $\nu_1, \dots, \nu_n$ be the $n$ eigenvalues of $A$ with corresponding eigenvectors $p_1, \dots, p_n$. Since we are assuming that $A$ is diagonalizable, we have a complete set of $n$ of these. Order the eigenvalues by increasing real part; if more than one eigenvalue has the same real part, any order is fine, as long as we make sure that any eigenvalue besides $\nu$ with real part $\sigma$ occurs after $\nu$ in the list. Then $\nu = \nu_k$ for some $k$, $\text{Re} \nu_j < \sigma$ for $j < k$ (as long as $k \neq 1$), and $\text{Re} \nu_j \geq \sigma$ for $j \geq k$. \\

Let $e^{Ax}$ be the fundamental matrix solution for $U' = A U$, and split $e^{Ax}$ up into
\[
e^{Ax} = Y_1(x) + Y_2(x)
\]
where $Y_1$ involves only eigenvectors corresponding to eigenvalues $\nu_1, \dots, \nu_{k-1}$ and $Y_2$ involves only eigenvectors corresponding to eigenvalues $\nu_{k}, \dots, \nu_n$. Since $A$ is a constant-coefficient matrix, we can write down an explicit formula for $Y_1$ and $Y_2$. Essentially, all we need to do is change coordinates to the eigenbasis, evolve along the appropriate eigenvectors, and zero out the other ones. To be specific, let P be the $n \times n$ matrix with columns $p_1, \dots, p_n$. Since we are assuming $A$ is diagonalizable, this matrix is invertible, and $D = P^{-1}AP$ is diagonal with eigenvalues $\nu_1, \dots, \nu_n$ on the diagonal. Recall that the matrix exponential is given by $e^{Ax} = P^{-1}e^{Dx}P$. Starting with the matrix $D$, form the matrix $D_1$ by keeping only the eigenvalues $\nu_1, \dots, \nu_{k-1}$ on the diagonal, and form the matrix $D_2$ by keeping only the eigenvalues $\nu_{k}, \dots, \nu_n$ on the diagonal. Then we have

\begin{align*}
Y_1(x) &= P^{-1}e^{D_1x}P \\
Y_1(x) &= P^{-1}e^{D_2x}P \\
\end{align*}
 
Choose $\delta$ such that $0 < \delta < \sigma - \text{Re} \nu_{k-1}$, i.e. smaller than the spectral gap between $\nu$ and the eigenvalue with the next smallest real part. (If $k = 1$, $Y_1 = 0$, $Y_2 = e^{Ax}$, and we don't care about $\delta$). Then we can find a constant $C$ such that

\begin{align*} 
|Y_1(x)| &\leq Ce^{(\sigma - \delta)x} && x \geq 0 \\
|Y_2(x)| &\leq Ce^{\sigma x} && x \leq 0 
\end{align*}

Define the exponentially weighted function space with weight $\sigma$

\[
B_{\sigma, a} = \{ f \in C^0([a, \infty), \R^n) : \sup_{x \in [a, \infty)} |e^{-\sigma x} f(x)| < \infty 
\]

where $a$ will be chosen later. The norm on this space is given by

\[
||f||_{\sigma, a} = \sup_{x \in [a, \infty)} |e^{-\sigma x} f(x)|
\]

In other words, we allow functions in $B_{\sigma, a}$ to grow exponentially as $x \rightarrow \infty$ at a rate of $\sigma$ or slower. It is known that $B_{\sigma, a}$ is a Banach space. Define the operator $F$ on $B_{\sigma, a}$ by

\begin{align*}
F(\phi)(x) = e^{\nu x} p + \int_a^x Y_1(x - y)R(y)\phi(y)dy + \int_\infty^x Y_2(x - y)R(y)\phi(y)dy
\end{align*}

where the $a$ in the integral is the same as in $B_{\sigma, a}$ and will be chosen later. First we show that $F: B_{\sigma, a} \rightarrow B_{\sigma, a}$. Let $\phi \in B_{\sigma, a}$. For $x \geq a$ we have

\begin{align*}
|e^{-\sigma x} &F(\phi)(x)| \leq e^{(\nu - \sigma) x} |p| + \int_a^x |Y_1(x - y)||R(y)||\phi(y)| dy + \int_x^\infty |Y_2(x - y)||R(y)||\phi(y)|dy \\
&\leq |p| + C \left( e^{-\sigma x}  \int_a^x e^{(\sigma - \delta)(x - y)}|R(y)||\phi(y)| dy + e^{-\sigma x}  \int_x^\infty e^{\sigma(x - y)}|R(y)||\phi(y)|dy \right) \\
&\leq |p| +  C \left( \int_a^x e^{-\delta(x - y)}|R(y)||e^{-\sigma y}\phi(y)| dy + \int_x^\infty |R(y)||e^{-\sigma y} \phi(y)|dy \right) \\
&\leq |p| + C ||\phi||_{\sigma, a}\left( \int_a^x e^{-\delta(x - y)}|R(y)| dy + \int_x^\infty |R(y)|dy \right) \\
&\leq |p| + C ||\phi||_{\sigma, a} \int_a^\infty |R(y)| dy 
\end{align*}

Since $R$ is integrable, the RHS is finite, thus the map $F: B_{\sigma, a} \rightarrow B_{\sigma, a}$ is well defined. Now we show the map $F$ is a contraction. Let $\phi, \psi \in B_{\sigma, a}$. For $x \geq a$ we have

\begin{align*}
|e^{-\sigma x}( &F(\phi)(x) - F(\psi)(x))| \leq \int_a^x |Y_1(x - y)||R(y)||\phi(y) - \psi(y)| dy + \int_x^\infty |Y_2(x - y)||R(y)||\phi(y) - \psi(y)|dy \\
&\leq C \left( e^{-\sigma x}  \int_a^x e^{(\sigma - \delta)(x - y)}|R(y)||\phi(y) - \psi(y)| dy + e^{-\sigma x}  \int_x^\infty e^{\sigma(x - y)}|R(y)||\phi(y) - \psi(y)|dy \right) \\
&\leq C \left( \int_a^x e^{-\delta(x - y)}|R(y)||e^{-\sigma y}(\phi(y) - \psi(y))| dy + \int_x^\infty |R(y)||e^{-\sigma y} (\phi(y) - \psi(y))|dy \right) \\
&\leq C ||\phi - \psi ||_{\sigma, a}\left( \int_a^x e^{-\delta(x - y)}|R(y)| dy + \int_x^\infty |R(y)|dy \right) \\
&\leq C ||\phi - \psi ||_{\sigma, a} \int_a^\infty |R(y)| dy 
\end{align*}

Since $R$ is integrable, we can choose $a$ sufficiently large so that

\[
\int_a^\infty |R(y)| dy < \frac{1}{2C}
\]

from which we conclude that

\[
||F(\phi) - F(\psi) ||_{\sigma, a} \leq \frac{1}{2} ||\phi - \psi ||_{\sigma, a}
\]

Since $F$ is a contraction on the Banach space $B_{\sigma, a}$, by the Banach Fixed Point Theorem the map $F$ has a unique fixed point, i.e. a unique $\phi(x) \in B_{\sigma, a}$ such that $F(\phi) = \phi$. Note that by the fixed point theorem and definition of $B_{\sigma, a}$, we only have $\phi(x)$ defined for $\phi \geq a$. However, choosing the initial condition $\phi(a)$ at $x = a$, by the existence and uniqueness of solutions to \eqref{veigproblem} and the global Lipschitz condition placed on $R(x)$ (which guarantees global existence of solutions), we can extend $\phi(x)$ uniquely to all of $\R$. This extension of $\phi(x)$ to $\R$ is given by the same formula we have for $\phi(x)$ when $x \geq a$.

\begin{equation}\label{fpphi}
\phi(x) = e^{\nu x} p + \int_a^x Y_1(x - y)R(y)\phi(y)dy + \int_\infty^x Y_2(x - y)R(y)\phi(y)dy
\end{equation}

To see this, all we need to do is show that it satisfies the ODE \eqref{veigproblem}. Differentiating \eqref{fpphi}, we get

\begin{align*}
\phi'(x) &= \nu e^{\nu x} p + (Y_1(0) + Y_2(0))R(x)\phi(x) + \int_a^x Y_1'(x - y)R(y)\phi(y)dy + \int_\infty^x Y_2'(x - y)R(y)\phi(y)dy \\
&= e^{\nu x} A p + e^{0A}R(x)\phi(x) + \int_a^x A Y_1(x - y)R(y)\phi(y)dy + \int_\infty^x A Y_2(x - y)R(y)\phi(y)dy \\
&= A \left( e^{\nu x} p + \int_a^x Y_1(x - y)R(y)\phi(y)dy + \int_\infty^x Y_2(x - y)R(y)\phi(y)dy \right) + R(x) \phi(x) \\
&= A \phi(x) + R(x) \phi(x)
\end{align*}

Thus $\phi(x)$ defined in \eqref{fpphi} is the unique extension that we seek. All that remains is to show what happens when $x \rightarrow \infty$. Since we are interested in end behavior, we only need to consider what happens when $x \geq a$. Since $\phi(x) \in B_{\lambda, a}$, $||\phi||_{\sigma, a}$ is finite and independent of $x$. Thus for $x \geq a$, using what we did above, 

\begin{align*}
|e^{-\sigma x} &(\phi(x) - e^{\nu x} p)| \leq C ||\phi||_{\sigma, a}\left( \int_a^x e^{-\delta(x - y)}|R(y)| dy + \int_x^\infty |R(y)|dy \right) \\
&\leq C ||\phi||_{\sigma, a}\left( \int_a^{x/2} e^{-\delta(x - y)}|R(y)| dy + \int_{x/2}^x |R(y)|+ \int_x^\infty |R(y)|dy \right)\\
&\leq C ||\phi||_{\sigma, a}\left( e^{-\delta(x/2)} \int_a^{x/2} e^{-\delta(x/2 - y)}|R(y)| dy + \int_{x/2}^\infty |R(y)|dy \right)\\
&\leq C ||\phi||_{\sigma, a}\left( e^{-\delta(x/2)} \int_a^{\infty} |R(y)| dy + \int_{x/2}^\infty |R(y)|dy \right)\\
&\leq ||\phi||_{\sigma, a}\left(\frac{1}{2} e^{-\delta(x/2)} + \int_{x/2}^\infty |R(y)|dy \right)
\end{align*}

Since $\delta > 0$, $R(x)$ is integrable, and $||\phi||_{\sigma, a}$ is constant, both terms on the RHS go to 0 as $x \rightarrow \infty$. Thus we conclude that

\[
\lim_{x \rightarrow \infty} |e^{-\sigma x} (\phi(x) - e^{\nu x} p)| = 0
\]

Pulling out a factor of $e^{\nu x}$, this becomes 

\[
\lim_{x \rightarrow \infty} |e^{(\nu - \sigma) x}||\phi(x) e^{-\nu x} - p)| = 0
\]

since $\text{Re} \phi = \nu$, $|e^{(\nu - \sigma) x} = 1$ for all $x$. Thus we conclude that

\[
\lim_{x \rightarrow \infty} |\phi(x) e^{-\nu x} - p)| = 0
\]  

from which it follows that

\[
\lim_{x\rightarrow\infty} \phi(x) e^{-\nu x} = p
\]

\end{proof}
\end{lemma}


Let $\Phi(y,x; \lambda)$ be the evolution operator for $U' = \tilde{A}(\lambda, q) U$. For $\lambda = 0$, this has a center subspace which will perturb slightly for $\lambda \neq 0$ into a subspace which grows (or decays) exponentially, albeit at a slow rate since the spatial eigenvalue responsible for this is small. Thus instead of an exponential dichotomy we have an exponential trichotomy. We have this separately on $\R^+$ and $\R^-$. We will do this on $\R^+$ below.\\

Following Hale and Lin (1985), we will write this as follows. We have projections $P^s(x; \lambda)$, $P^u(x; \lambda)$ and $P^c(x; \lambda) = I - P^s(x; \lambda) - P^u(x; \lambda)$ such that

\begin{align*}
\Phi(y, x; \lambda)P^s(x; \lambda) &= P^s(y; \lambda)\Phi(y, x; \lambda) \\
\Phi(y, x; \lambda)P^u(x; \lambda) &= P^u(y; \lambda)\Phi(y, x; \lambda) \\
\Phi(y, x; \lambda)P^c(x; \lambda) &= P^c(y; \lambda)\Phi(y, x; \lambda) \\
\end{align*}

In other words, it does not matter if you project or evolve first. For $\lambda = 0$ the superscript $c$ actually represents the center subspace, and for small $\lambda$, this is the subspace that the center subspace perturbs to. Using these, we can split the evolution up into evolution on the three subspaces by defining

\begin{align*}
\Phi^s(y, x; \lambda) &= \Phi(y, x; \lambda)P^s(x; \lambda) \\
\Phi^u(y, x; \lambda) &= \Phi(y, x; \lambda)P^u(x; \lambda) \\
\Phi^c(y, x; \lambda) &= \Phi(y, x; \lambda)P^c(x; \lambda) \\
\end{align*}

For the stable and unstable subspaces, we know what the eigenvalues of $\tilde{A}(0, q)$ are. Let $\alpha$ be the smallest real part of the positive eigenvalues of this and $-\alpha$ be the largest real part of the negative eigenvalues of this. (We get the same $\alpha$ both ways by symmetry of the eigenvalues of $\tilde{A}(0, q)$. For small $\lambda$, the spatial eigenvalues will not perturb much, so for stable and unstable subspaces we will still have the bounds

\begin{align*}
\Phi^s(y, x; \lambda) \leq C e^{-\alpha(y-x)} \\
\Phi^u(x, y; \lambda) \leq C e^{-\alpha(y-x)}
\end{align*}
where $x \leq y$ and $C$ is a constant. Technically we should probably replace $\alpha$ by $\alpha - \delta$ for small $\delta$ to account for the perturbation, but it does not matter for now.\\

For the center subspace, we can use Lemma \ref{veigproblem}. For small $\lambda$, the perturbed asymptotic matrix $A(\lambda)W' = -A(x)^* W$ has a small spatial eigenvalue at $\nu \approxeq \lambda / c$. If we designate the corresponding eigenvector by $p$, then by Lemma \ref{veigproblem} we can find an eigenfunction $\phi(x)$ of $\tilde{A}(\lambda, q)$ such that $\phi(x) e^{-\nu x} \rightarrow p$ as $x \rightarrow \infty$. Thus we should have a $\lambda$-dependent bound we found to get

\begin{align*}
\Phi^c(y, x; \lambda) \leq C e^{\lambda(y-x)/c} \\
\Phi^c(x, y; \lambda) \leq C e^{\lambda(y-x)/c}
\end{align*}

So the (potential) exponential growth rate on this center space is limited by $\lambda$, which we know is small.\\

We would like an explicit form of the projection onto the center space, which we should be able to do in this case since that space is one-dimensional and is given by $\phi(x)$. Since this will involve the adjoint variational equation, we will collect what we know about that here for future reference.

\begin{lemma}Consider the eigenvalue problem $U' = A(x)U$ and the corresponding adjoint problem $W' = -A(x)^* W$, where $A$ is an $n \times n$ matrix depending on $x$. Then the following are true.
\begin{enumerate}[(i)]
\item $\frac{d}{dx}\langle U(x), W(x) \rangle = 0$, thus the inner product is constant as $x$ varies.
\item If $\Phi(y, x)$ is the evolution operator for $U' = A(x)U$, then $\Phi(x, y)^*$ is the evolution operator for the adjoint problem $W' = -A(x)^* W$.
\end{enumerate}
\end{lemma}

Let $\nu(\lambda)$ be the small eigenvalue of $A(\lambda)$ with corresponding eigenvector $v_0(\lambda)$. Then $-\overline{\nu(\lambda)}$ is the small eigenvalue of $-A(\lambda)^*$ with corresponding eigenvector $w_0(\lambda)$. Using Lemma \ref{veigproblem}, let $\tilde{v}(x; \lambda)$ and $\tilde{w}(x; \lambda)$ be solutions to the eigenvalue problem and its adjoint problem such that

\begin{align*}
\lim_{x \rightarrow \infty} e^{-\nu(\lambda) x} \tilde{v}(x) = v_0 \\
\lim_{x \rightarrow \infty} e^{\overline{\nu(\lambda)} x} \tilde{w}(x) = w_0 \\
\end{align*}

Now let 

\begin{align*}
\tilde{v}(x) &= e^{\nu(\lambda) x } v(x) \\
\tilde{w}(x) &= e^{-\overline{\nu(\lambda)} x} w(x) \\
\end{align*}

Then

\begin{align*}
\lim_{x \rightarrow \infty} v(x) = v_0 \\
\lim_{x \rightarrow \infty} w(x) = w_0 \\
\end{align*}

What we would like to do is write the projection $P^c(x)$ in terms of the adjoint solution $\tilde{w}(x)$. To do this, let $R^s(x; \lambda)$, $R^u(x; \lambda)$, and $R^c(x; \lambda)$ be the ranges of the corresponding projections (stable, unstable, and center ranges; NOT THE SAME AS THE CORRESPONDING MANIFOLDS). The dimensions of these ranges are 2, 2, and 1 (respectively). First we show that $\tilde{w}(0)$ is perpendicular to $R^s(0; \lambda)$. Let $u(0) \in R^s(0; \lambda)$. Then $u(x) = \Phi(x, 0)u(0) \in R^s(x; \lambda)$ for all $x \geq 0$. Since the inner product $\langle u(x), \tilde{w}(x) \rangle$ is constant in $x$, we let $x \rightarrow \infty$ to get

\begin{align*}
\lim_{x \rightarrow \infty} \langle u(x), \tilde{w}(x) \rangle &= \lim_{x \rightarrow \infty} e^{-\nu(\lambda) x} \langle u(x), w(x) \rangle \\
&= \langle \lim_{x \rightarrow \infty} e^{-\nu(\lambda) x} u(x), w_0 \rangle \\
&= 0
\end{align*}
since $u(x)$ decays exponentially at a faster rate than $|\nu(\lambda)|$. Thus we have $\tilde{w}(0) \perp R^s(0; \lambda)$ and by the same token, $\tilde{w}(x) \perp R^s(x; \lambda)$ for all $ \geq 0$.\\

Note that we cannot play this same game with $R^u(x; \lambda)$, since to get decay to 0 we would have to take $x \rightarrow -\infty$. Since all these only pertain to the dichotomy/trichotomy on $\R^+$, they are only valid for $x \geq 0$, so we cannot take that limit.\\

However, we really want to have $\tilde{w}(x) \perp R^u(x; \lambda)$. To attain this, let's change coordinates, like we do in the constant coeffient case. The idea is that if we do this at $x = 0$ to get $\tilde{w}(0) \perp R^u(0; \lambda)$, the invariance of the inner product in $x$ should take care of the rest. In fact, we can do this for both the stable and unstable ranges so we might as well do that. Let $\{ a^s_1(0), a^s_2(0)\}$ be a basis for $R^s(0; \lambda)$ and let $\{a^u_1(0), a^u_2(0)\}$ be a basis for $R^u(0; \lambda)$. Then the set $\{ a^s_1(0), a^s_2(0),a^u_1(0), a^u_2(0), \tilde{w}(0) \}$ is linearly independent thus spans $\R^5$ (or $\C^5$). Change coordinates so that $\tilde{w}(0)$ is perpendicular to the span of the other four. Let $a^{s/u}_i(x) = \Phi(x,0)a^{s/u}_i(0)$, i.e. evolve the basis vectors forward. These will (pretty sure) remain linearly independent after evolution, so will be a basis for the appropriate projection ranges at $x$. For simplicity, consider one of these, say $a^s_1(x)$. Since $\Phi(x,0) a^s_1(0)$ solves the eigenvalue problem, therefore by the invariance in $x$ of the inner product of solutions to the eigenvalue problem with solutions to the adjoint eigenvalue problem,

\begin{align*}
\langle a^s_1(x), \tilde{w}(x) \rangle &= \langle \Phi(x,0) a^s_1(0), \tilde{w}(x) \rangle \\
&= \langle a^s_1(0), \tilde{w}(0) \rangle = 0
\end{align*}

Thus a single change of variables at $x = 0$ accomplishes what we want. Then since $\tilde{w}(x) \in R^c(x; \lambda)$ and is perpendicular to the other two spaces, to get the center projection $P^c(x; \lambda)$, all we have to do is take the inner product with $\tilde{w}(x)$. This gives us 

\[
P^c(x; \lambda)u = \langle u, \tilde{w}(x) \rangle = e^{-\nu(\lambda)x}\langle u, w(x) \rangle
\]

Thus for an arbitrary solution $u(x)$ to the eigenvalue problem with initial condition $u(0)$, we have

\begin{align*}
P^c(x; \lambda)u(x) &= P^c(x; \lambda)\Phi(x,0)u(0) \\
&= \Phi(x,0)P^c(0; \lambda)u(0) \\
&= \Phi^c(x,0)P^c(0; \lambda)u(0) \\
&= e^{-\nu(\lambda)x} \Phi^c(x,0) \langle u(0), w(0) \rangle \\
&= e^{-\nu(\lambda)x} v(x) \langle u(0), w(0) \rangle
\end{align*}

In the last line, we have projected onto the one-dimensional center space which is invariant under the evolution $\Phi$. Since $v(x)$ also evolves in that space, we must have $\Phi(x,0) = v(x)$.\\

At this point, we write down the fixed point equations for the problem. Before we do that, we take a look at where these equations came from. From what I can tell, it is very similar to the variation of constants formula, with the primary difference being that we split the solution up into stable and unstable parts, evolve them separately (each with its own IC), and recombine them. So the fixed point equations should look like those in Sandstede (1998) with the addition of a center evolution term together with an IC in the center subspace. Also since we are no longer integrating out to $\pm \infty$ the ICs $a_i$ are no longer 0. We can choose which way to integrate on the center subspace, but I am not sure how much it matters. We also note that we have incorporated the $B \lambda$ term into $\tilde{A}(\lambda, q)$ to get the eigenvalue problem

\[
(W^\pm)' = \tilde{A}(\lambda, q) W_i^\pm - \lambda^2 B Q_c
\]

Of course, $W$ is dependent on $\lambda$, but we suppress that dependence in the notation since it is annoying. Thus we the fixed point problem looks like

\begin{align*}
W^-(x) = \Phi^s(&x, -T; \lambda)a_- + \Phi^u(x, 0; \lambda)b^- + \Phi^c(x, 0; \lambda)c^- \\
&- \lambda^2 \int_0^x \Phi^u(x, y; \lambda) B Q_c(y) ] dy \\
&- \lambda^2 \int_{-T}^x \Phi^s(x, y; \lambda) B Q_c(y) ] dy \\
&- \lambda^2 \int_0^x \Phi^c(x, y; \lambda) B Q_c(y) ] dy \\
W^+(x) = \Phi^u(&x, T; \lambda)a^+ + \Phi^s(x, 0; \lambda)b^+ + \Phi^c(x, 0; \lambda)c^+ \\
&- \lambda^2 \int_0^x \Phi^s(x, y; \lambda) B Q_c(y) ] dy \\
&- \lambda^2 \int_T^x \Phi^u(x, y; \lambda) B Q_c(y) ]dy \\
&- \lambda^2 \int_0^x \Phi^c(x, y; \lambda) B Q_c(y) ] dy
\end{align*}

\end{document}