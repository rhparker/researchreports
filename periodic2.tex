\documentclass[12pt]{article}
\usepackage[pdfborder={0 0 0.5 [3 2]}]{hyperref}%
\usepackage[left=1in,right=1in,top=1in,bottom=1in]{geometry}%
\usepackage[shortalphabetic]{amsrefs}%
\usepackage{amsmath}
\usepackage{enumerate}
% \usepackage{enumitem}
\usepackage{amssymb}                
\usepackage{amsmath}                
\usepackage{amsfonts}
\usepackage{amsthm}
\usepackage{bbm}
\usepackage[table,xcdraw]{xcolor}
\usepackage{tikz}
\usepackage{float}
\usepackage{booktabs}
\usepackage{svg}
\usepackage{mathtools}
\usepackage{cool}
\usepackage{url}
\usepackage{graphicx,epsfig}
\usepackage{makecell}
\usepackage{array}

\def\noi{\noindent}
\def\T{{\mathbb T}}
\def\R{{\mathbb R}}
\def\N{{\mathbb N}}
\def\C{{\mathbb C}}
\def\Z{{\mathbb Z}}
\def\P{{\mathbb P}}
\def\E{{\mathbb E}}
\def\Q{\mathbb{Q}}
\def\ind{{\mathbb I}}

\DeclareMathOperator{\spn}{span}
\DeclareMathOperator{\ran}{ran}
\DeclareMathOperator{\dm}{dim}

\graphicspath{ {periodic/} }

\newtheorem{lemma}{Lemma}
\newtheorem{theorem}{Theorem}
\newtheorem{corollary}{Corollary}
\newtheorem{definition}{Definition}
\newtheorem{assumption}{Assumption}
\newtheorem{hypothesis}{Hypothesis}

\begin{document}

\section{Periodic Multipulses}

\subsection{Setup}

We will cast the problem in general terms, for which KdV5 will be a specific case. Consider the PDE

\begin{equation}\label{genPDE}
u_t = \partial_x E'(u)
\end{equation}

where $u \in \R$, $E(u)$ is the energy of the system (which is conserved in $t$), and $E(0) = 0$. The operator $E'(u): H^{2m}(\R) \subset L^2(\R) \rightarrow L^2(\R)$ is self-adjoint in $L^2(\R)$, where $m \geq 2$.\\

Equilibrium solutions which decay at $\pm \infty$ satisfy $E'(u) = 0$, which we write as the ODE in $\R^{2m}$

\begin{equation}\label{genODE}
U' = F(U)
\end{equation}

where $F: \R^{2m} \rightarrow \R^{2m}$ is smooth with $F(0) = 0$. We begin by looking at the existence problem, which involves solutions to \eqref{genODE}.

\subsection{Existence of Periodic Multipulses}

We take the following hypotheses. First, we assume that \eqref{genODE} is Hamiltonian.

\begin{hypothesis}\label{Hhyp}
There exists a smooth function $H: \R^{2m} \rightarrow \R$ such that $H(0) = 0$, $\nabla H(u) = 0$ if and only if $F(u) = 0$, and for all $u \in \R^{2m}$,
\begin{equation}
\langle \nabla H(u), F(u) \rangle = 0
\end{equation}
\end{hypothesis}

From this it follows that $H$ is conserved along solutions $U(x)$ to \eqref{genODE}.\\

The second hypothesis addresses the hyperbolicity of the equilibrium of \eqref{genODE} at $U = 0$.  

\begin{hypothesis}\label{hypeq}
The spectrum of $DF(0)$ contains a quartet of simple eigenvalues $\pm \alpha \pm \beta i$, $\alpha, \beta > 0$. For any other eigenvalue $\nu$ of $DF(0)$, $|\text{Re }|\nu| > \alpha$.
\end{hypothesis}

We note that since we are assuming we have a Hamiltonian system, the existence of an eigenvalue $\alpha + \beta i$ implies the existence of the entire quartet.\\

Let $W^s(0)$ and $W^u(0)$ be the stable and unstable manifolds of the equilibrium at 0. By Hypothesis \ref{Hhyp}, $W^s(0), W^u(0) \subset H^{-1}(0)$, where $H^{-1}(0)$ is the 0-level set of the energy $H$. Since $\dm H^{-1}(0) = m-1$, a dimensionality argument shows that $\dim W^s(0) \cap W^u(0) \geq 1$. Thus there exists a homoclinic orbit $Q(x) \subset W^s(0) \cap W^u(0)$. In the next hypothesis, we asssume that this intersection is 1-dimensional. 

The next hypothesis states that there exists a homoclinic orbit connecting $W^s(0)$ and $W^u(0)$. 

\begin{hypothesis}\label{transverseQ}
Let $Q(x)$ be the homoclinic orbit in $W^s(0) \cap W^u(0) \subset H^{-1}(0)$. We assume the non-degeneracy condition
\begin{equation}
T_{Q(0)}W^s(0) \cap T_{Q(0)}W^u(0) = \R Q'(0)
\end{equation}
\end{hypothesis}

Define the variational equation and adjoint variational equation corresponding to $Q(x)$ by

\begin{align}
V' = DF(Q(x))V \label{vareq} \\
W' = -DF(Q(x))^* W \label{adjvareq}
\end{align}

It follows from Hypothesis \ref{transverseQ} that $Q'(x)$ is the unique bounded solution to \eqref{vareq} and that there exists a unique bounded solution $\Psi(x)$ to \eqref{adjvareq}. It follows from Hypotheiss \ref{Hhyp} that $\Psi(x) = \nabla H(Q(x))$. \\

Furthermore, we may decompose the tangent space at $Q(0)$ as 

\begin{equation}
\R^{2m} = \R \Psi(0) \oplus \R Q'(0) \oplus Y^+ \oplus Y^-
\end{equation}

where

\begin{align*}
T_{Q(0)}W^s(0) &= \R Q'(0) \oplus Y^+ \\
T_{Q(0)}W^u(0) &= \R Q'(0) \oplus Y^- \\
\end{align*}

and $\Psi(0) \perp \R Q'(0) \oplus Y^+ \oplus Y^-$.\\

For now, we are not concerned with the eigenvalue problem associated with \eqref{genODE}. For an equilibrium solution $U^*$ to \eqref{genODE}, this eigenvalue problem is

\begin{equation}\label{genODEeig}
V' = (DF(U^*) + \lambda \tilde{B})V
\end{equation}

where the matrix $\tilde{B}$ is constant coefficient. These eigenvalues are the eigenvalues of the Hessian $E''(u^*)$, where $u^*$ is the first component of $U^*$. If we wish to consider that problem, we will need an additional hypothesis concerning a Melnikov integral related to the problem.

\begin{hypothesis}\label{ODEMelnikov}
The following Melnikov condition holds.
\begin{equation}
M = \int_{-\infty}^\infty \langle \Psi(x), \tilde{B} Q'(x) \rangle \neq 0
\end{equation}
\end{hypothesis}

For now we can ignore this, but if we wish, we should be able to relate these eigenvalues to the interaction eigenvalues of \eqref{genPDE}.\\

We can now state the existence theorem for periodic multi-pulse solutions. A$n-$periodic solution to \eqref{genODE} is a periodic orbit $Q_{np}(x)$ which resembles $n$ copies of the primary pulse solution $Q(x)$. We will describe an $n-$periodic solution by $n$ lengths $X_0, \dots, X_{n-1}$, where the $n-1$ distances between the $n$ peaks are $2X_0, \dots 2X_{n_2}$, and $X_{n-1}$ is the distance from the left- and rightmost peaks to the periodic boundary.

\begin{theorem}[Existence of $n$-periodic solutions]\label{perexist}
Let $Q(x)$ be a transversely constructed primary pulse solution to \eqref{genODE}, according to Hypothesis \ref{transverseQ}. Assume Hypotheses \ref{hypeq} and \ref{Hhyp}.\\

Then there exists an a nonnegative integer $M$ and $K > 0$ such that the following holds. For any 
\begin{enumerate}[(i)]
\item integer $n \geq 2$ \\
\item integer $m \geq M$ \\
\item sequence of nonnegative integers $\{ m_0, \dots, m_{n-2} \}$, where at least one of them must be 0
\end{enumerate}

there exists a 1-parameter family of $n$-periodic solutions to \eqref{genODE} associated with $m$ and the set $\{ m_j \}$. This 1-parameter family is parameterized by the length $X_{n-1} \in [K, \infty)$.\\

The lengths $X_0, \dots, X_{n-2}$ are given by

\begin{equation}\label{Xi}
X_i = \frac{\pi}{2 \beta}m 
- \frac{1}{2 \alpha} \log(a_i(X_{n-1}, r_m)) + C
\end{equation}

where $C$ is a constant, $r_m = e^{-m \pi \alpha/\beta}$, and $a_i(X_{n-1}, r_m) \rightarrow e^{-m_i \pi \alpha/ \beta}$ as $(X_{n-1}, r_m) \rightarrow (\infty, 0)$. For large $X_{n-1}$ and $m$, the lengths $X_0, \dots, X_{n-2}$ are approximately

\begin{equation}\label{Xiapprox}
X_i = \frac{\pi}{2 \beta}(m + m_i) + C
\end{equation}

If $U(x)$ is such a solution, then $U(x)$ can be written piecewise for $i = 0, \dots, n-1$ as 

\begin{align}
U_i^-(x) &= Q^-(x; \beta_i^-) + V_i^-(x) && x \in [X_{i-1}, 0] \\
U_i^+(x) &= Q^+(x; \beta_i^+) + V_i^+(x) && x \in [0, X_i]
\end{align}

where the subscripts $i$ are taken $\mod n$. The functions $Q^\pm(x; \beta_i^\pm)$ are solutions to \eqref{genODE} with initial conditions $\beta_i^\pm$ in the stable/unstable manifold. For these terms, we have bounds

\begin{align*}
|Q^\pm(x; \beta_i^\pm)| \leq C (e^{-2 \alpha X_{i-1}} + e^{-2 \alpha X_i})e^{-\alpha |x|}
\end{align*} 

The remainder terms $V_i^\pm(x)$ have bounds

\begin{align}
|V_i^-(x)| &\leq C e^{-\alpha(X_{i-1} + x)}e^{-\alpha X_{i-1}} \\
|V_i^+(x)| &\leq C e^{-\alpha(X_i - x)}e^{-\alpha X_i} 
\end{align} 
\end{theorem}

For the family of $2-$periodic solutions, we actually have more information (including bifurcation structure), which we can add here.

\subsection{Stability Problem}

We will now look at the stability problem, which involves the eigenvalues of \eqref{genPDE}. We rewrite the PDE in $\R^{2m+1}$ as

\begin{equation}\label{PDEsystem}
U_t = F_2(U)
\end{equation}

where $F_2: \R^{2m+1} \rightarrow \R^{2m+1}$. For an equilibrium solution $U^*$, $F_2(U^*) = 0$. In particular, we have $F_2(0) = 0$, $F_2(Q) = 0$, and $F_2(Q_{np}) = 0$, where $Q$ the primary pulse solution and $Q_{np}$ is the $n-$periodic solution from the existence problem. (Note that all of these are now functions $\R \rightarrow \R^{2m+1}$).
\\

The eigenvalue problem of interest is the linearization about an equilibrium solution $U^*$.

\begin{equation}\label{PDEeig}
V' = ( A(U^*)V + \lambda B)V 
\end{equation}

where $A(U^*) = DF_2(U^*)$ and $B$ is constant coefficient. From Hypothesis \ref{hypeq}, it follows that the spectrum of $A(0)$ contains $\{ 0, \pm \alpha \pm \beta i\}$, and for any other eigenvalue $\nu$ of $A(0)$, $|\text{Re }|\nu| > \alpha$. Since $A(0)$ is not hyperbolic, the results of San98 do not apply.\\

Let $Q(x)$ be the primary pulse solution, whose existence comes from Hypothesis \ref{transverseQ}. We make the following nondegeneracy hypothesis regarding $Q(x)$, which is similar to Hypothesis \ref{transverseQ}.

\begin{hypothesis}\label{nondegen}
For the primary pulse solution $Q(x)$, we have the non-degeneracy condition
\begin{equation}
T_{Q(0)}W^s(0) \cap T_{Q(0)}W^u(0) = \R Q'(0)
\end{equation}
\end{hypothesis}

Let $W^s(0)$, $W^u(0)$, and $W^c(0)$ be the stable, unstable, and center manifolds of the equilibrium at 0. Define the variational equation and adjoint variational equations by

\begin{align}
V' = A(Q) V \label{vareq2} \\
W' = -A(Q)^* W \label{adjvareq2}
\end{align}

Then $Q'(x)$ a bounded solution to \eqref{vareq2}. Since $Q(x)$ decays to 0 exponentially at both ends, $A(Q)$ is exponentially asymptotic to the constant-coefficient matrix $A_\infty$, which has an eigenvalue of 0. We make the following additional hypothesis concerning solutions to \eqref{adjvareq2}.

\begin{hypothesis}\label{adjsolutions}
There exists exactly two bounded solution $\Psi(x)$ and $\Phi^c(x)$ to \eqref{adjvareq}. For $\Psi(x)$, $e^{\alpha |x|}\Phi(x)$ is bounded, and for $\Phi^c(x)$,
\begin{equation}
\Phi(x) \rightarrow W_0 \text{ as }|x| \rightarrow \infty
\end{equation}
where $W_0$ is the eigenvector of $A_\infty$ corresponding to the eigenvalue 0.
\end{hypothesis}

From this, it follows that we can decompose the tangent space at $Q(0)$ as 

\begin{equation}
\R^{2m+1} = S \oplus \R Q'(0) \oplus Y^+ \oplus Y^-
\end{equation}

where $S = \spn\{ \Psi(0), \Psi^c(0) \}$, and $S \perp \R Q'(0) \oplus Y^+ \oplus Y^-$.

Finally, we take the following hypothesis regarding Melnikov integrals.

\begin{hypothesis}\label{Melnikov2}
We have the following Melnikov-like expressions
\begin{enumerate}[(i)]
\item 
\begin{equation}
M_1 = \int_{-\infty}^\infty \langle \Psi(x), B Q'(x) \rangle = 0
\end{equation}
\item 
\begin{equation}
M_2 = \int_{-\infty}^\infty \langle \Psi(x), B T(x) \rangle = 0
\end{equation}
where $T(x)$ solves 
\begin{equation}
T' = A(Q)T + BQ'
\end{equation}
\end{enumerate}
We should be guaranteed such a solution $T$ by (i) and the Fredholm alternative.
\end{hypothesis}

We can now state the stability theorem. The proof uses Lin's method to find the eigenvalues of \eqref{PDEeig}.

% main stability theorem

\begin{theorem}\label{PDEeigtheorem}

Let $Q_{np}(x)$ be a $n-$periodic solution constructed according to Theorem \ref{perexist} with lengths $X_0, \dots, X_{n-1}$. Assume Hypotheses \ref{nondegen}, \ref{adjsolutions}, and \ref{Melnikov2}.\\

Then there exists $\delta > 0$ such that a bounded, nonzero solution $V$ of the eigenvalue problem 

\begin{equation}
V' = ( A(U^*)V + \lambda B)V 
\end{equation}

exists for $|\lambda| < \delta$ if and only if $\det S(\lambda) = 0$. $S(\lambda)$ is the block matrix

\begin{equation}\label{blockmatrix}
S(\lambda) = 
\begin{pmatrix}
K(\lambda) + C_2 & -\lambda^2 M^c I + D_1 \\
C_3 K(\lambda) + C_4 & A - \lambda^2 M_2 I + D_2
\end{pmatrix}
\end{equation}

where 

\begin{enumerate}

\item The matrix $K(\lambda)$ is given by

\begin{equation}
K(\lambda) = 
\begin{pmatrix}
e^{-\nu(\lambda)X_1} & & & & & -e^{\nu(\lambda)X_0} \\
-e^{\nu(\lambda)X_1} & e^{-\nu(\lambda)X_2} \\
& -e^{\nu(\lambda)X_2} & e^{-\nu(\lambda)X_3} \\
\vdots & & \vdots & &&  \vdots \\
& & & & -e^{\nu(\lambda)X_{n-1}} & e^{-\nu(\lambda)X_0} 
\end{pmatrix}
\end{equation}

where $\nu(\lambda)$ is the small eigenvalue of the asympotic matrix $A_\infty + B \lambda$.

\item The matrix $A$ is given by

\begin{align*}
A &= \begin{pmatrix}
-a_0 + \tilde{a}_1 & a_0 - \tilde{a}_1 \\
-\tilde{a}_0 + a_1 & \tilde{a}_0 - a_1
\end{pmatrix} && n = 2 \\
A &= \begin{pmatrix}
\tilde{a}_{n-1} - a_0 & a_0 & & & \dots & -\tilde{a}_{n-1}\\
-\tilde{a}_0 & \tilde{a}_0 - a_1 &  a_1 \\
& -\tilde{a}_1 & \tilde{a}_1 - a_2 &  a_2 \\
& & \vdots & & \vdots \\
a_{n-1} & & & & -\tilde{a}_{n-2} & \tilde{a}_{n-2} - a_{n-1} \\
\end{pmatrix} && n > 2
\end{align*}

where

\begin{align*}
a_i &= \langle \Psi(X_i), Q'(-X_i) \rangle \\
\tilde{a}_i &= \langle \Psi(-X_i), Q'(X_i) \rangle
\end{align*}

\item $M_2$ and $M^c$ are the Melnikov integrals

\begin{align*}
M_2 &= \int_{-\infty}^\infty \langle \Psi(x), B T(x) \rangle dx \\
M^c &= \int_{-\infty}^\infty \langle \Psi^c(x), B T(x) \rangle dy
\end{align*}

\item The remainder terms have bounds

\begin{align*}
C_2 &= \mathcal{O}(e^{-\alpha X_m}) \\
C_3 &= \mathcal{O}(|\lambda| + e^{-\alpha X_m}) I
+ \mathcal{O}((|\lambda| + e^{-\tilde{\alpha} X_m})( |\lambda| + e^{-\alpha X_m}))\\
C_4 &= \mathcal{O}(e^{-\alpha X_m}(|\lambda| + e^{-\tilde{\alpha} X_m})) \\
D_1 &= \mathcal{O}((|\lambda| + e^{-\tilde{\alpha} X_m})^2) \\
D_2 &= \mathcal{O}((|\lambda| + e^{-\tilde{\alpha} X_m})(|\lambda| + e^{-\alpha X_m})^2) 
\end{align*}

where $X_m = \min \{X_0, \dots, X_{n-1}\}$

\end{enumerate}

To leading order, equation \eqref{blockmatrix} is upper triangular block equation

\begin{equation}\label{blocktri}
\begin{pmatrix}
K(\lambda) & -\lambda^2 M^c I  \\
0 & A - \lambda^2 MI 
\end{pmatrix}
\begin{pmatrix}c \\ d \end{pmatrix} = 0
\end{equation}

Thus, to leading order, the eigenvalue problem has a nontrivial solution $V$ if either of the following conditions holds.

\begin{enumerate}[(i)]
\item $\nu(\lambda) = i \dfrac{n \pi}{X}, n \in \Z$ 
\item $\det(A - \lambda^2 MI) = 0$
\end{enumerate}

where $X = X_0 + \dots + X_{n-1}$ is half the length of the domain. The first condition gives us the essential spectrum, and the second condition gives us the point spectrum.

\end{theorem}

\subsection{KdV5}

For KdV5, a primary pulse solution is known to exist (Chug2007, among others), so Hypothesis \ref{transverseQ} is satisfied. For $c > 1/4$, Hypothesis \ref{hypeq} is satisfied. Finally, for KdV5, \eqref{genODE} is Hamiltonian, so Hypothesis \ref{Hhyp} is satisfied. Thus, Theorem \ref{perexist} holds, and $n-$periodic solutions exist.\\

We assume Hypothesis \ref{nondegen} holds for KdV5. The adjoint variational equation has two solutions, $\Psi(x)$ and $\Psi^c(x)$, which are given by

\begin{equation}\label{KdV5psi}
\Psi(x) = \begin{pmatrix}
q^{(4)}(x) - q''(x) + (-2q(x) + c)q(x)\\
-q^{(3)}(x) + q'(x) \\
q''(x) - q(x) \\
-q'(x) \\
q(x)
\end{pmatrix}
\end{equation}

and

\begin{align}\label{KdV5psic}
\Psi^c(x) = (c - 2 q(x), 0, -1, 0, 1)^T
\end{align}

where $q(x)$ is the primary pulse solution (as a real-valued function). These satisfy Hypothesis \ref{adjsolutions}. Finally, the Melnikov integrals are given by

\begin{align*}
M_1 &= \int_{-\infty}^\infty q(x) q'(x) dx = 0 \\
M_2 &= \int_{-\infty}^\infty q(x) q_c(x) dx \\
M^c &= \int_{-\infty}^\infty q_c(x) dx
\end{align*}

We assume that $M_2 \neq 0$, which is suggested by numerics. $M^c$ does not matter, although numerics suggests that it is also nonzero.\\

In Theorem \ref{PDEeigtheorem}, for KdV5 we have $\tilde{a}_i = a_i$, since the primary pulse $q(x)$ is an even function. Thus the matrix $A$ in \ref{PDEeigtheorem} is given by

\begin{align*}
A &= \begin{pmatrix}
-a_0 -a_1 & a_0 + a_1 \\
a_0 + a_1 & -a_0 - a_1
\end{pmatrix} && n = 2 \\
A &= \begin{pmatrix}
-a_{n-1} - a_0 & a_0 & & & \dots & a_{n-1}\\
a_0 & -a_0 - a_1 &  a_1 \\
& a_1 & -a_1 - a_2 &  a_2 \\
& & \vdots & & \vdots \\
a_{n-1} & & & & a_{n-2} & -a_{n-2} - a_{n-1} \\
\end{pmatrix} && n > 2
\end{align*}

\end{document}