\documentclass[12pt]{article}
\usepackage[pdfborder={0 0 0.5 [3 2]}]{hyperref}%
\usepackage[left=1in,right=1in,top=1in,bottom=1in]{geometry}%
\usepackage[shortalphabetic]{amsrefs}%
\usepackage{amsmath}
\usepackage{enumerate}
% \usepackage{enumitem}
\usepackage{amssymb}                
\usepackage{amsmath}                
\usepackage{amsfonts}
\usepackage{amsthm}
\usepackage{bbm}
\usepackage[table,xcdraw]{xcolor}
\usepackage{tikz}
\usepackage{float}
\usepackage{booktabs}
\usepackage{svg}
\usepackage{mathtools}
\usepackage{cool}
\usepackage{url}
\usepackage{graphicx,epsfig}
\usepackage{makecell}
\usepackage{array}

\def\noi{\noindent}
\def\T{{\mathbb T}}
\def\R{{\mathbb R}}
\def\N{{\mathbb N}}
\def\C{{\mathbb C}}
\def\Z{{\mathbb Z}}
\def\P{{\mathbb P}}
\def\E{{\mathbb E}}
\def\Q{\mathbb{Q}}
\def\ind{{\mathbb I}}

\DeclareMathOperator{\spn}{span}
\DeclareMathOperator{\ran}{ran}

\graphicspath{ {periodic/} }

\newtheorem{lemma}{Lemma}
\newtheorem{theorem}{Theorem}
\newtheorem{corollary}{Corollary}
\newtheorem{definition}{Definition}
\newtheorem{assumption}{Assumption}
\newtheorem{hypothesis}{Hypothesis}

\begin{document}

\section{Periodic Multipulses}

\subsection{Setup}

We will cast the problem in general terms, for which KdV5 will be a specific case. Consider the PDE

\begin{equation}\label{genPDE}
u_t = \partial_x E'(u)
\end{equation}

where $u \in \R$, $E(u)$ is the energy of the system (which is conserved in $t$), and $E(0) = 0$. The operator $E'(u): H^{2m}(\R) \subset L^2(\R) \rightarrow L^2(\R)$ is self-adjoint in $L^2(\R)$, where $m \geq 2$.\\

Equilibrium solutions which decay at $\pm \infty$ satisfy $E'(u) = 0$, which we write as the ODE in $\R^{2m}$

\begin{equation}\label{genODE}
U' = F(U)
\end{equation}

where $F: \R^{2m} \rightarrow \R^{2m}$ is smooth with $F(0) = 0$. We begin by looking at the existence problem, which involves solutions to \eqref{genODE}.

\subsection{Existence of Periodic Multipulses}

We take the following hypotheses. The first hypothesis states that the equilibrium of \eqref{genODE} at $U = 0$ is hyperbolic. 

\begin{hypothesis}\label{hypeq}
The spectrum of $DF(0)$ contains a quartet of simple eigenvalues $\pm \alpha \pm \beta i$, $\alpha, \beta > 0$. For any other eigenvalue $\nu$ of $DF(0)$, $|\text{Re }|\nu| > \alpha$.
\end{hypothesis}

Let $W^s(0)$ and $W^u(0)$ be the stable and unstable manifolds of this equilibrium at 0. The next hypothesis states that there exists a homoclinic orbit connecting $W^s(0)$ and $W^u(0)$.

\begin{hypothesis}\label{transverseQ}
$Q(x)$ is a transversely constructed, localized homoclinic orbit at $U = 0$ which connects $W^s(0)$ to $W^u(0)$. Furthermore, we will take the nondegeneracy condition
\begin{equation}
T_{Q(0)}W^s(0) \cap T_{Q(0)}W^u(0) = \R Q'(0)
\end{equation}
\end{hypothesis}

Define the variational equation and adjoint variational equation by

\begin{align}
V' = DF(Q(x))V \label{vareq} \\
W' = -DF(Q(x))^* W \label{adjvareq}
\end{align}

Then it follows from Hypothesis \ref{transverseQ} that $Q'(x)$ is the unique bounded solution to \eqref{vareq} and there exists a unique bounded solution $\Psi(x)$ to \eqref{adjvareq}. Furthermore, we may decompose the tangent space at $Q(0)$ as 

\begin{equation}
\R^{2m} = \R \Psi(0) \oplus \R Q'(0) \oplus Y^+ \oplus Y^-
\end{equation}

where

\begin{align*}
T_{Q(0)}W^s(0) &= \R Q'(0) \oplus Y^+ \\
T_{Q(0)}W^u(0) &= \R Q'(0) \oplus Y^- \\
\end{align*}

and $\Psi(0) \perp \R Q'(0) \oplus Y^+ \oplus Y^-$.\\

Finally, we will assume the system is Hamiltonian.

\begin{hypothesis}\label{Hhyp}
There exists a smooth function $H: \R^{2m} \rightarrow \R$ such that $\nabla H(Q(0)) \neq 0$ and for all $u \in \R^{2m}$,

\begin{equation}
\langle \nabla H(u), F(u) \rangle = 0
\end{equation}

Furthermore, we assume that
\begin{equation*}
\lim_{x \rightarrow \infty} e^{2 \alpha x}|Q(x)||Q(-x)| > 0
\end{equation*}
\end{hypothesis}

From this it follows that $H$ is conserved along solutions $U(x)$ to \eqref{genODE}. It follows from Hypothesis \ref{transverseQ} with Hypothesis \ref{Hhyp} that $\Psi(x) = \nabla H(Q(x))$.\\

For now, we have not been concerned with the eigenvalue problem associated with \eqref{genODE}, which, for a solution $U^*$ to \eqref{genODE} is

\begin{equation}\label{genODEeig}
V' = (DF(U^*) + \lambda B)V
\end{equation}

where the matrix $B$ is constant coefficient. If we wish to consider that problem, we will need an additional hypothesis concerning the Melnikov integral associated with the problem.

\begin{hypothesis}\label{ODEMelnikov}
The following Melnikov condition holds.
\begin{equation}
M = \int_{-\infty}^\infty \langle \Psi(x), B Q'(x) \rangle \neq 0
\end{equation}
\end{hypothesis}

We can now state the existence theorem for periodic multi-pulse solutions. A$n-$periodic solution to \eqref{genODE} is a periodic orbit $Q_{np}(x)$ which resembles $n$ copies of the primary pulse solution $Q(x)$. We will describe an $n-$periodic solution by $n$ lengths $X_0, \dots, X_{n-1}$, where the $n-1$ distances between the $n$ peaks are $2X_0, \dots 2X_{n_2}$, and $X_{n-1}$ is the distance from the left- and rightmost peaks to the periodic boundary.

\begin{theorem}[Existence of $n$-periodic solutions]\label{multiexist}
Let $Q(x)$ be a transversely constructed primary pulse solution to \eqref{genODE}, according to Hypothesis \ref{transverseQ}. Assume Hypotheses \ref{hypeq} and \ref{Hhyp}.\\

Then there exists an a nonnegative integer $M$ and $K > 0$ such that the following holds. For any 
\begin{enumerate}[(i)]
\item integer $n \geq 2$ \\
\item integer $m \geq M$ \\
\item sequence of nonnegative integers $\{ m_0, \dots, m_{n-2} \}$, where at least one of them must be 0
\end{enumerate}

there exists a 1-parameter family of $n$-periodic solutions to \eqref{genODE} associated with $m$ and the set $\{ m_j \}$. This 1-parameter family is paramaterized by the length $X_{n-1} \in [K, \infty)$.\\

The lengths $X_0, \dots, X_{n-2}$ are given by

\begin{equation}\label{Xi}
X_i = \frac{\pi}{2 \beta}m 
- \frac{1}{2 \alpha} \log(a_i(X_{n-1}, r_m)) + C
\end{equation}

where $C$ is a constant, $r_m = e^{-m \pi \alpha/\beta}$, and $a_i(X_{n-1}, r_m) \rightarrow e^{-m_i \pi \alpha/ \beta}$ as $(X_{n-1}, r_m) \rightarrow (\infty, 0)$. For large $X_{n-1}$ and $m$, the lengths $X_0, \dots, X_{n-2}$ are approximately

\begin{equation}\label{Xiapprox}
X_i = \frac{\pi}{2 \beta}(m + m_i) + C
\end{equation}

If $U(x)$ is such a solution, then $U(x)$ can be written piecewise for $i = 0, \dots, n-1$ as 

\begin{align}
U_i^-(x) &= Q^-(x; \beta_i^-) + V_i^-(x) && x \in [X_{i-1}, 0] \\
U_i^+(x) &= Q^+(x; \beta_i^+) + V_i^+(x) && x \in [0, X_i]
\end{align}

where the subscripts $i$ are taken $\mod n$. The functions $Q^\pm(x; \beta_i^\pm)$ are solutions to \eqref{genODE} with initial conditions $\beta_i^\pm$ in the stable/unstable manifold. For these terms, we have bounds

\begin{align*}
|Q^\pm(x; \beta_i^\pm)| \leq C (e^{-2 \alpha X_{i-1}} + e^{-2 \alpha X_i})e^{-\alpha |x|}
\end{align*} 

The remainder terms $V_i^\pm(x)$ have bounds

\begin{align}
|V_i^-(x)| &\leq C e^{-\alpha(X_{i-1} + x)}e^{-\alpha X_{i-1}} \\
|V_i^+(x)| &\leq C e^{-\alpha(X_i - x)}e^{-\alpha X_i} 
\end{align} 
\end{theorem}

\subsection{Stability Problem}

We will now look at the stability problem, which involves the eigenvalues of \eqref{genPDE}. We rewrite the PDE in $\R^{2m+1}$ as

\begin{equation}\label{PDEsystem}
U_t = F_2(U)
\end{equation}

where $F_2: \R^{2m+1} \rightarrow \R^{2m+1}$. For an equilibrium solution $U^*$, $F_2(U^*) = 0$. In particular, we have $F_2(0) = 0$, $F_2(Q) = 0$, and $F_2(Q_{np}) = 0$, where $Q$ the primary pulse solution and $Q_{np}$ is the $n-$periodic solution from the existence problem. (Note that all of these are now functions $\R \rightarrow \R^{2m+1}$).
\\

The eigenvalue problem of interest is the linearization about an equilibrium solution $U^*$.

\begin{equation}\label{PDEeig}
V' = ( A(U^*)V + \lambda B)V 
\end{equation}

where $A(U^*) = DF_2(U^*)$ and $B$ is constant coefficient. From Hypothesis \ref{hypeq}, it follows that the spectrum of $A(0)$ contains $\{ 0, \pm \alpha \pm \beta i\}$, and for any other eigenvalue $\nu$ of $A(0)$, $|\text{Re }|\nu| > \alpha$. Since $A(0)$ is not hyperbolic, the results of San98 do not apply.\\

Let $Q(x)$ be the primary pulse solution, whose existence comes from Hypothesis \ref{transverseQ}. We make the following nondegeneracy hypothesis regarding $Q(x)$, which is similar to Hypothesis \ref{transverseQ}.

\begin{hypothesis}\label{nondegen}
For the primary pulse solution $Q(x)$, we have the nondegeneracy condition
\begin{equation}
T_{Q(0)}W^s(0) \cap T_{Q(0)}W^u(0) = \R Q'(0)
\end{equation}
\end{hypothesis}

Let $W^s(0)$, $W^u(0)$, and $W^c(0)$ be the stable, unstable, and center manifolds of the equilibrium at 0. Define the variational equation and adjoint variational equations by

\begin{align}
V' = A(Q) V \label{vareq2} \\
W' = -A(Q)^* W \label{adjvareq2}
\end{align}

Then $Q'(x)$ a bounded solution to \eqref{vareq2}. Since $Q(x)$ decays to 0 exponentially at both ends, $A(Q)$ is exponentially asymptotic to the constant-coefficient matrix $A_\infty$, which has an eigenvalue of 0. We make the following additional hypothesis concerning solutions to \eqref{adjvareq2}.

\begin{hypothesis}There exists exactly two bounded solution $\Psi(x)$ and $\Phi^c(x)$ to \eqref{adjvareq}. For $\Psi(x)$, $e^{\alpha |x|}\Phi(x)$ is bounded, and for $\Phi^c(x)$,
\begin{equation}
\Phi(x) \rightarrow W_0 \text{ as }|x| \rightarrow \infty
\end{equation}
where $W_0$ is the eigenvector of $A_\infty$ corresponding to the eigenvalue 0.
\end{hypothesis}

From this, it follows that we can decompose the tangent space at $Q(0)$ as 

\begin{equation}
\R^{2m+1} = S \oplus \R Q'(0) \oplus Y^+ \oplus Y^-
\end{equation}

where $S = \spn\{ \Psi(0), \Psi^c(0) \}$, and $S \perp \R Q'(0) \oplus Y^+ \oplus Y^-$.

Finally, we take the following hypothesis regarding Melnikov integrals.

\begin{hypothesis}\label{Melnikov2}
We have the following Melnikov-like expressions
\begin{enumerate}[(i)]
\item 
\begin{equation}
\int_{-\infty}^\infty \langle \Psi(x), B Q'(x) \rangle = 0
\end{equation}
\item 
\begin{equation}
\int_{-\infty}^\infty \langle \Psi(x), B T(x) \rangle = 0
\end{equation}
where $T(x)$ solves 
\begin{equation}
T' = A(Q)T + BQ'
\end{equation}
We should be guaranteed such a solution $T$ by (i) and the Fredholm alternative.
\end{hypothesis}

We can now state the stability theorem, for which we use Lin's method to find the eigenvalues of \eqref{PDEeig}.






\end{document}