\documentclass[12pt]{article}
\usepackage[pdfborder={0 0 0.5 [3 2]}]{hyperref}%
\usepackage[left=1in,right=1in,top=1in,bottom=1in]{geometry}%
\usepackage[shortalphabetic]{amsrefs}%
\usepackage{amsmath}
\usepackage{enumerate}
% \usepackage{enumitem}
\usepackage{amssymb}                
\usepackage{amsmath}                
\usepackage{amsfonts}
\usepackage{amsthm}
\usepackage{bbm}
\usepackage[table,xcdraw]{xcolor}
\usepackage{tikz}
\usepackage{float}
\usepackage{booktabs}
\usepackage{svg}
\usepackage{mathtools}
\usepackage{cool}
\usepackage{url}
\usepackage{graphicx,epsfig}
\usepackage{makecell}
\usepackage{array}

\def\noi{\noindent}
\def\T{{\mathbb T}}
\def\R{{\mathbb R}}
\def\N{{\mathbb N}}
\def\C{{\mathbb C}}
\def\Z{{\mathbb Z}}
\def\P{{\mathbb P}}
\def\E{{\mathbb E}}
\def\Q{\mathbb{Q}}
\def\ind{{\mathbb I}}

\DeclareMathOperator{\spn}{span}
\DeclareMathOperator{\ran}{ran}
\DeclareMathOperator{\dm}{dim}

\graphicspath{ {periodic/} }

\newtheorem{lemma}{Lemma}
\newtheorem{theorem}{Theorem}
\newtheorem{corollary}{Corollary}
\newtheorem{definition}{Definition}
\newtheorem{assumption}{Assumption}
\newtheorem{hypothesis}{Hypothesis}

\begin{document}

\section{Periodic Multipulses}

\subsection{KdV5}

The 5th order KdV equation (KdV5), when written in a moving frame with speed $c$, is given by

\begin{equation}\label{KdV5}
u_t = \partial_x(u_{xxxx} - u_{xx} - u^2 + c)
\end{equation}

We can write this as $u_t = \partial_x E'(u)$, where $E(u)$ is the energy

\begin{equation}\label{energy}
E(u) = -\int_{-\infty}^{\infty} \left( \frac{1}{2}u_{xx}^2 + \frac{1}{2}u_x^2 + \frac{1}{2}cu^2 - \frac{1}{3}u^3 \right) dx
\end{equation}

and the operator $\partial_x$ is skew-symmetric. The energy $E(u)$ is conserved (in time). We also note that $E'(u)$ is also reversible, i.e. $E'(u(x)) = 0 \iff E'(u(-x)) = 0$. In particular, this means that $E'(u)$ only involves even order derivatives.\\

An equilibrium solution to \eqref{KdV5} satisfies the 5th order nonlinear ODE

\begin{equation}\label{eqODE}
u_{xxxxx} - u_{xxx} + c u_x - 2 u u_x = 0
\end{equation}

It is clear that $u = 0$ is a solution to \eqref{eqODE}. A primary pulse solution is a homoclinic orbit which connects the equilbrium state $u = 0$ to itself. Since such a solution decays (exponentially) to 0 at both ends, a primary pulse solution must also satisfy the 4th order ODE

\begin{equation}\label{eqODE4}
u_{xxxx} - u_{xx} + c u - u^2 = 0,
\end{equation}

which is obtained from \eqref{eqODE} by integrating once; we take the constant of integration to be 0, since we are seeking a solution which decays to a baseline of 0. Equation \eqref{eqODE4} is also Hamiltonian (in $x$), with energy given by

\begin{equation}\label{Hamiltonian}
H(u, u', u'', u''') = u'u''' - \frac{1}{2}(u'^2) - \frac{1}{2}(u'')^2 + \frac{c}{2}u^2 - \frac{1}{3}u^3 
\end{equation}

The Hamiltonian $H$ is conserved (in $x$).\\

The linearization of the 4th order ODE \eqref{eqODE4} about a solution $u^*(x)$ of \eqref{eqODE4} is the self-adjoint linear operator

\begin{equation}\label{defA0}
A_0(u^*) = \partial_x^4 - \partial_x^2 + c - 2 u^* 
\end{equation}

For the linearization about the rest state $u^* = 0$, the eigenvalues are the solutions to the fourth-order polynomial equation $\nu^4 - \nu^2 + c = 0$, which are

\begin{align}\label{specA0}
\nu = \pm \sqrt{ \frac{1 \pm \sqrt{1 - 4c} }{2}}
\end{align}

Since two eigenvalues have positive real part and two have negative real part, the equibrium at 0 is hyperbolic with a two-dimensional stable manifold and a two-dimensional unstable manifold. For $0 < c < 1/4$, all four eigenvalues are real. A bifurcation takes place at $c = 1/4$, and for $c > 1/4$, there is a quartet of eigenvalues of the form $\pm \alpha \pm \beta i$, where $\alpha, \beta > 0$.\\

Using a mountain-pass argument, for $c > 0$, a symmetric homoclinic orbit solution $q(x)$ exists to \eqref{eqODE4}.

\subsection{Setup}

We will write the problem in general terms, for which KdV5 will be a specific case. Consider the PDE

\begin{equation}\label{genPDE}
u_t = \partial_x E'(u)
\end{equation}

where $u \in \R$, $E'(u): H^{2m}(\R) \subset L^2(\R) \rightarrow L^2(\R)$, and $E(u)$ is the energy of the system (which is conserved in $t$), and $E(0) = 0$. We take $E'(u)$ to be of the form

\[
E'(u) = \partial_x^{2m}u + \tilde{E}(u)
\]

and we make the following hypothesis regarding $E'(u)$.

\begin{hypothesis}\label{reversiblehyp}
$E'(u)$ is reversible, i.e. $E'(u(x)) = 0 \iff E'(u(-x)) = 0$
\end{hypothesis}

This hypothesis implies that the operator $E'(u)$ only involves even-order derivatives of $u$ with respect to $x$. In addition, we note that since $E'(u)$ only depends on $x$ via $u(x)$, $E'(u)$ is translation invariant, i.e. $E'(u(x+\xi)) = E'(u(x))$ for all $\xi \in \R$.\\

Equilibrium solutions which decay to a baseline of 0 at $\pm \infty$ satisfy the ODE 

\begin{equation}\label{ODEonR}
E'(u) = \partial_x^{2m}u + \tilde{E}(u) = 0
\end{equation}

Since we are assuming that the highest order derivative $\partial_x^{2m}$ in $E'(u)$ appears only by itself with a coefficient of 1, we can write \eqref{ODEonR} as a first-order system (in the standard way) in $\R^{2m}$ as

\begin{equation}\label{genODE}
U' = F(U)
\end{equation}

where $U \in \R^{2m}$, $F: \R^{2m} \rightarrow \R^{2m}$ is smooth, and $F(0) = 0$. We begin by looking at the existence problem, which involves solutions to \eqref{genODE}.

\subsection{Existence of Periodic Multipulses}

We take the following hypotheses. First, we assume that \eqref{genODE} is Hamiltonian.

\begin{hypothesis}\label{Hhyp}
There exists a smooth function $H: \R^{2m} \rightarrow \R$ such that 
\begin{enumerate}[(i)]
\item $H(0) = 0$
\item $\nabla H(u) = 0$ if and only if $F(u) = 0$
\item For all $u \in \R^{2m}$,
\begin{equation}
\langle \nabla H(u), F(u) \rangle = 0
\end{equation}
\end{enumerate}
\end{hypothesis}

It follows from this hypothesis that $H$ is conserved along solutions $U(x)$ to \eqref{genODE}.\\

The next hypothesis addresses the hyperbolicity of the equilibrium of \eqref{genODE} at $U = 0$.  

\begin{hypothesis}\label{hypeq}
0 is a hyperbolic equilibrium of \eqref{genODE}, i.e. $DF(0)$ has no eigenvalues on the imaginary axis. Furthermore, the spectrum of $DF(0)$ contains a quartet of simple eigenvalues $\pm \alpha \pm \beta i$, $\alpha, \beta > 0$. For any other eigenvalue $\nu$ of $DF(0)$, $|\text{Re }\nu| > \alpha$.
\end{hypothesis}

We note that since we have a Hamiltonian system by Hypothesis \ref{Hhyp}, the existence of an eigenvalue $\alpha + \beta i$ implies the existence of the entire quartet $\pm \alpha \pm \beta i$.\\

Let $W^s(0)$ and $W^u(0)$ be the stable and unstable manifolds of the equilibrium at 0. By Hypothesis \ref{Hhyp}, $W^s(0), W^u(0) \subset H^{-1}(0)$, where $H^{-1}(0)$ is the 0-level set of the energy $H$. By reversibility from Hypothesis \ref{reversiblehyp}, $\dim W^s(0) = m$ and $\dim W^u(0) = m$.

In the next hypothesis, we assume that a symmetric homoclinic orbit solution exists to \eqref{genODE}.

\begin{hypothesis}\label{qexistshyp}
A homoclinic orbit solution $Q(x) \in W^s(0) \cap W^u(0) \subset H^{-1}(0)$ exists to \eqref{genODE}. Furthermore, this solution is symmetric with respect to the reverser operator $R:\R^{2m} \rightarrow \R^{2m}$, defined by
\begin{equation}\label{reverserR2m}
R(x_1, x_2, x_3, x_4, \dots, x_{2m}) = (x_1, -x_2, x_3, -x_4, \dots, x_{2m})
\end{equation}
i.e. $Q(-x) = R(Q(x))$.
\end{hypothesis}
 
Since we obtained \eqref{genODE} from putting \eqref{ODEonR} into a first order system in the standard way, $Q(x) = (q(x), q'(x), \dots, q^{(2m)}(x))^T$, where $q(x)$ is a an even function and a solution to \eqref{ODEonR}.\\

Finally, we take a standard nondegeneracy condition regarding the homoclinic orbit $Q(x)$.

\begin{hypothesis}\label{nondegen1}
Let $Q(x)$ be the symmetric homoclinic orbit from Hypothesis \ref{qexistshyp}. We take the non-degeneracy condition
\begin{equation}
T_{Q(0)}W^s(0) \cap T_{Q(0)}W^u(0) = \R Q'(0)
\end{equation}
\end{hypothesis}

Let $u^*(x)$ be a solution to \eqref{ODEonR} which exponentially decays to 0 at both ends. The linearization of \eqref{ODEonR} about $u^*(x)$ is the eigenvalue problem

\begin{equation}\label{ODEeig}
E''(u^*(x)) v = \lambda v
\end{equation}

where $E''(u^*): H^{2m}(\R) \rightarrow H^{2m}(\R)$ is the Hessian of the energy and is self-adjoint. Since $E'(u)$ is translation invariant, it follows that $E'(u^*(x)) \partial_x u^*(x) = 0$.\\

We can write the eigenvalue problem as the first-order system

\begin{equation}\label{ODEeig2}
V' = ( DF(U^*)V + \lambda B) V 
\end{equation}

where $B$ is the $2m \times 2m$ constant coefficient matrix

\begin{equation}
B = \begin{pmatrix}0 & 0 & 0 & 0 & 0 \\0 & 0 & 0 & 0 & 0 \\  & 
\vdots & 0 & \vdots & \\0 & 0 & 0 & 0 & 0 \\1 & 0 & 0 & 0 & 0 \end{pmatrix} 
\end{equation}

and 

\begin{equation}
DF(U^*) = \begin{pmatrix}
0 & 1 & 0 & \dots & 0 & 0 \\
0 & 0 & 1 & \dots & 0 & 0 \\
& & \ddots & \ddots & \ddots & & \\
0 & 0 & 0 & \dots & 0 & 1 \\
c_0 + f_0(x) & 0 & c_2 + f_2(x) &
 \dots & c_{2m-2} + f_{2m-2}(x) & 0
\end{pmatrix}
\end{equation}

where the $c_i$ are constants, and the $f_i(x)$ decay exponentially to 0 since they involve $u^*(x)$ and its derivatives. By the Hypothesis \ref{reversiblehyp}, no odd-order derivatives are involved, thus $c_j = 0$ and $f_j(x) = 0$ for $j$ odd. $DF(U^*)$ is exponentially asymptotic to $DF(0)$, which has characteristic polynomial

\begin{equation}\label{charpolyDF0}
p_1(\nu) = \nu^{2m} - c_{2m-2} \nu^{2m-2} - \dots - c_2 \nu^2 - c_0
\end{equation}

By Hypothesis \label{hypeq}, $p_1(\nu)$ has roots at $\nu = \pm \alpha \pm i \beta$, and for all other roots $\nu$ of $p_1(\nu)$ we have $|\text{Re }\nu| > \alpha$.\\

For the variational equation, we take $U^* = Q$ and $\lambda = 0$ in \eqref{ODEeig2}.

\begin{align}
\tilde{V}' = DF(Q(x))\tilde{V} \label{vareq1} \\
\tilde{W}' = -DF(Q(x))^* \tilde{W} \label{adjvareq1}
\end{align}

It follows from Hypothesis \ref{nondegen1} that $Q'(x)$ is the unique bounded solution to \eqref{vareq1} and that there exists a unique bounded solution $\Psi(x)$ to \eqref{adjvareq1}. It follows from Hypothesis \ref{Hhyp} that $\Psi(x) = \nabla H(Q(x))$. Since $E''(q)$ is self-adjoint, it is not hard to show that the $2m$-th component of $\Psi(x)$ is $q'(x)$.\\

Finally, we decompose the tangent space at $Q(0)$ as 

\begin{equation}
\R^{2m} = \R \Psi(0) \oplus \R Q'(0) \oplus Y^+ \oplus Y^-
\end{equation}

where

\begin{align*}
T_{Q(0)}W^s(0) &= \R Q'(0) \oplus Y^+ \\
T_{Q(0)}W^u(0) &= \R Q'(0) \oplus Y^- \\
\end{align*}

and $\Psi(0) \perp \R Q'(0) \oplus Y^+ \oplus Y^-$.\\

We can now state the existence theorem for periodic multi-pulse solutions. A$n-$periodic solution to \eqref{genODE} is a periodic orbit $Q_{np}(x)$ which resembles $n$ copies of the primary pulse solution $Q(x)$. We will describe an $n-$periodic solution by $n$ lengths $X_0, \dots, X_{n-1}$, where the $n-1$ distances between the $n$ peaks are $2X_0, \dots 2X_{n_2}$, and $X_{n-1}$ is the distance from the left- and rightmost peaks to the periodic boundary.

\begin{theorem}[Existence of $n$-periodic solutions]\label{perexist}
Assume Hypotheses \ref{hypeq} and \ref{Hhyp}, and let $Q(x)$ be a transversely constructed primary pulse solution to \eqref{genODE}, according to Hypothesis \ref{nondegen1}.\\

Then there exists an a nonnegative integer $M$ and $K > 0$ such that the following holds. For any 
\begin{enumerate}[(i)]
\item integer $n \geq 2$ \\
\item integer $m \geq M$ \\
\item sequence of nonnegative integers $\{ m_0, \dots, m_{n-2} \}$, where at least one of them must be 0
\end{enumerate}

there exists a 1-parameter family of $n$-periodic solutions to \eqref{genODE} associated with $m$ and the set $\{ m_j \}$. This 1-parameter family is parameterized by the length $X_{n-1} \in [K, \infty)$.\\

The lengths $X_0, \dots, X_{n-2}$ are given by

\begin{equation}\label{Xi}
X_i = \frac{\pi}{2 \beta}m 
- \frac{1}{2 \alpha} \log(a_i(X_{n-1}, r_m)) + C
\end{equation}

where $C$ is a constant, $r_m = e^{-m \pi \alpha/\beta}$, and $a_i(X_{n-1}, r_m) \rightarrow e^{-m_i \pi \alpha/ \beta}$ as $(X_{n-1}, r_m) \rightarrow (\infty, 0)$. For large $X_{n-1}$ and $m$, the lengths $X_0, \dots, X_{n-2}$ are approximately

\begin{equation}\label{Xiapprox}
X_i = \frac{\pi}{2 \beta}(m + m_i) + C
\end{equation}

If $U(x)$ is such a solution, then $U(x)$ can be written piecewise for $i = 0, \dots, n-1$ as 

\begin{align}
U_i^-(x) &= Q^-(x; \beta_i^-) + V_i^-(x) && x \in [X_{i-1}, 0] \\
U_i^+(x) &= Q^+(x; \beta_i^+) + V_i^+(x) && x \in [0, X_i]
\end{align}

where the subscripts $i$ are taken $\mod n$. The functions $Q^\pm(x; \beta_i^\pm)$ are solutions to \eqref{genODE} with initial conditions $\beta_i^\pm$ in the stable/unstable manifold. For these terms, we have bounds

\begin{align*}
|Q^\pm(x; \beta_i^\pm)| \leq C (e^{-2 \alpha X_{i-1}} + e^{-2 \alpha X_i})e^{-\alpha |x|}
\end{align*} 

The remainder terms $V_i^\pm(x)$ have bounds

\begin{align}
|V_i^-(x)| &\leq C e^{-\alpha(X_{i-1} + x)}e^{-\alpha X_{i-1}} \\
|V_i^+(x)| &\leq C e^{-\alpha(X_i - x)}e^{-\alpha X_i} 
\end{align} 
\end{theorem}

For the family of $2-$periodic solutions, we actually have more information (including bifurcation structure), which we can add here.

\subsection{Stability Problem}

Now that have shown existence of periodic multi-pulse solutions, we will now look at their stability. For linear stability, this involves finding the eigenvalues of \eqref{genPDE}. Let $u^*(x)$ be any exponantially localized equilibrium solution to \eqref{genPDE}. Substituting the standard linearization ansatz $u(x, t) = u^*(x) + \epsilon e^{\lambda t}v(x)$ into \eqref{genPDE} and keeping terms of order $\epsilon$, we obtain the PDE eigenvalue problem

\begin{equation}\label{PDEeig}
\partial_x E''(u^*(x)) v = \lambda v
\end{equation}

which is just \eqref{ODEeig} with a $\partial_x$ out front. From above, we have $[\partial_x E''(u^*(x))] \partial_x u^*(x) = 0$. Letting $U^*(x) = (u^*(x), \partial_x u^*(x), \dots, \partial_x^{2m}u^*(x))$, we can write \eqref{PDEeig} as the first order system 

\begin{equation}\label{PDEeig2}
V' = ( A(U^*)V + \lambda B)V 
\end{equation}

where $A(U^*)$ is the $(2m+1) \times (2m+1)$ matrix

\begin{equation}\label{DefA}
A(U^*) = \begin{pmatrix}
0 & 1 & 0 & 0 & \dots & 0 & 0 \\
0 & 0 & 0 & 0 & \dots & 0 & 0 \\
& \ddots & & \ddots & \ddots & & \\
0 & 0 & 0 & 0 & \dots & 0 & 1 \\
f_0'(x) & c_0 + f_0(x) & f_2'(x) & c_2 + f_2(x) & \dots & f_{2m-2}'(x) & c_{2m-2} + f_{2m-2}(x) & f_{2m}'(x)
\end{pmatrix}
\end{equation}

and $B$ is the $(2m +1) \times (2m+1)$ constant coefficient matrix

\begin{equation}\label{DefB}
B = \begin{pmatrix}0 & 0 & 0 & 0 & 0 \\0 & 0 & 0 & 0 & 0 \\  & 
\vdots & 0 & \vdots & \\0 & 0 & 0 & 0 & 0 \\1 & 0 & 0 & 0 & 0 \end{pmatrix} 
\end{equation}

(We keep the notation $B$ for the matrix since it has exactly the same form as the matrix $B$ in the previous section.) The terms $c_i$ and $f_i(x)$ in the bottom row of $A(U^*)$ are the same as in those in the bottom row of $DF(U^*)$ from the previous section. Since $u^*(x)$ is exponentially localized, $A(U^*)$ is exponentially asymptotic to $A(0)$, which has characteristic polynomial

\begin{equation}\label{charpolyDF0}
p_2(\nu) = -\nu^{2m+1} + c_{2m-2} \nu^{2m-1} + \dots + c_2 \nu^3 - c_0 \nu = \nu p_1(\nu)
\end{equation}

where $p_1(\nu)$ is the characteristic polynomial of $DF(0)$ from the previous section. Thus $A(0)$ has a simple eigenvalue at 0, and the rest of the eigenvalues of $A(0)$ are the same as those of $DF(0)$. Using Hypothesis \ref{hypeq}, it follows that the spectrum of $A(0)$ contains $\{ 0, \pm \alpha \pm \beta i\}$, and for any other eigenvalue $\nu$ of $A(0)$, $|\text{Re }|\nu| > \alpha$. Since $A(0)$ is not hyperbolic, the results of San98 do not apply.\\

Let $W^s(0)$, $W^u(0)$, and $W^c(0)$ be the stable, unstable, and center manifolds of the equilibrium at 0. By reversibility from Hypothesis \ref{reversiblehyp}, $\dim W^s(0) = m$ and $\dim W^u(0) = m$. The center manifold $W^c(0)$ is one-dimensional.\\

Let $Q(x) = (q(x), \partial_x q(x), \dots, \partial_x^{2m} q(x))$, where $q(x)$ is the symmetric primary pulse solution from Hypothesis \ref{qexistshyp}.\\

For the variational and adjoint variational equations, we take $U^* = Q$ and $\lambda = 0$ in \eqref{PDEeig2} to get

\begin{align}
V' = A(Q) V \label{vareq2} \\
W' = -A(Q)^* W \label{adjvareq2}
\end{align}

$Q'(x)$ is an expontially localized solution to \eqref{vareq2}. Since $Q(x)$ is exponentially localized, it cannot involve the center manifold $W^c(0)$, thus it must lie in the intersecion of the stable and unstable manifolds. In particular, this implies that $\R Q'(0) \subset T_{Q(0)}W^s(0) \cap T_{Q(0)}W^u(0)$. From Hypothesis \ref{nondegen1}, these are actually equal, which we state in the following lemma.

\begin{lemma}\label{nondegenlemma}
We have the nondegeneracy condition

\begin{equation}\label{nondegen2}
T_{Q(0)}W^s(0) \cap T_{Q(0)}W^u(0) = \R Q'(0)
\end{equation}

\begin{proof}
Since $Q(x)$ is a homoclinic orbit in the intersection of $W^u(0)$ and $W^s(0)$, $\R Q'(0) \subset T_{Q(0)}W^s(0) \cap T_{Q(0)}W^u(0)$. If the intersection were more than one-dimensional, there would exist another exponentially localized solution $V(x) = (v_1, \dots, v_{2m}, v_{2m+1})^T$ to \eqref{vareq2}. It follows that $\tilde{V}(x) = (v_1, \dots, v_{2m})^T$ would be an exponentially localized solution to \eqref{vareq1}, which contradicts Hypothesis \ref{nondegen1}.
\end{proof}
\end{lemma}

Let $V_0$ and $W_0$ be the eigenvectors of $A(Q)$ and $-A^*(Q)$ corresponding to the eigenvalue 0. From \eqref{DefA}, we can see that

\begin{align}
V_0 &= (-1/c_0, 0, 0, \dots, 0)^T \label{V0} \\
W_0 &= (-c_0, 0, -c_2, 0, -c_4, 0, \dots, 0, -c_{2m-2}, 0, 1)^T \label{W0}
\end{align}

where we have scaled $V_0$ so that $\langle V_0, W_0 \rangle = 1$. We have the following lemma regarding solutions to \eqref{vareq2} and \eqref{adjvareq2}.

\begin{lemma}\label{varadjsolutions}
\begin{enumerate}[(i)]
\item There exists a bounded solution $V_0(x)$ of \eqref{vareq2} such that 
\begin{equation}
V_0(x) \rightarrow V_0 \text{ as }|x| \rightarrow \infty
\end{equation}

\item There exists exactly two bounded solutions $\Psi(x)$ and $\Psi^c(x)$ to \eqref{adjvareq2}. $\Psi(x)$ is exponentially localised, i.e. $e^{\alpha |x|}\Psi(x)$ is bounded. For $\Psi^c(x)$,
	\begin{equation}
	\Psi^c(x) \rightarrow W_0 \text{ as }|x| \rightarrow \infty
	\end{equation}
\end{enumerate}

\begin{proof}

\begin{enumerate}
\item The adjoint variational equation \eqref{adjvareq2} is the following equation written as a first order system
\begin{equation}\label{adj2}
[\partial_x E''(q) ]^* v(x) = -E''(q) \partial_x v(x) = 0,
\end{equation}
which has exponentially decaying solution $q(x)$. Thus we have an exponentially localized solution $\Psi(x)$ to \eqref{adjvareq2} which has $q(x)$ as its $(m+1)$-th component.

\item Equation \eqref{adj2} also has the constant solution 1 as a solution. Thus \eqref{adjvareq2} has a solution $\Psi^c(x)$ which has $1$ as its $(m+1)$-th component. This solution has the form

\[
\Psi^c(x) = \begin{pmatrix}
-c_0 + g_1(x) \\ g_2(x) \\
-c_2 + g_3(x) \\ g_4(x) \\
\vdots \\
-c_{2m-2} + g_{2m-1}(x) \\ g_{2m}(x) \\ 1
\end{pmatrix}
\]
where the functions $g_j(x)$ are linear combinations of the functions $f_j(x)$ from \eqref{DefA} and their derivatives and are thus exponentially localized. Thus we see that $\Psi^c(x) \rightarrow W_0$ as $|x| \rightarrow \infty$, where $W_0$ is given by \eqref{W0}.

\item Recall that $\dim W^{s/u}(0) = m$ and $\dim W^c(0) = 1$, thus $\dim W^{cs}(0) = m + 1$, and $\dim W^{cu}(0) = m + 1$. $\Psi(0) \perp T_{Q(0)}W^{cs}(0) + T_{Q(0)}W^{cu}(0)$, so $\dim T_{Q(0)}W^{cs}(0) + T_{Q(0)}W^{cu}(0) \leq 2m$. This implies, by a dimension-counting argument, that $\dim T_{Q(0)}W^{cs}(0) \cap T_{Q(0)}W^{cu}(0) = 2$. Since $\dim T_{Q(0)}W^s(0) \cap T_{Q(0)}W^s(0) = 1$ by Lemma \ref{nondegenlemma}, we conclude that there exists $Y_0 \in T_{Q(0)}W^{cs}(0) \cap T_{Q(0)}W^{cu}(0)$ which is linearly independent from $Q'(0)$, i.e. $Y^0 \notin T_{Q(0)}W^s(0) \cap T_{Q(0)}W^s(0)$.

\item Using $Y_0$ as the initial condition for the variational equation \eqref{vareq2} gives us the bounded solution $V_0(x)$ in (i). For the decay to $V_0$, we might need the gap lemma (which gives this on $\R^\pm$ separately) and the symmetry of the problem.

\end{enumerate}
\end{proof}
\end{lemma}

Using Lemma \ref{nondegenlemma} and Lemma \ref{varadjsolutions}, we can decompose the tangent space at $Q(0)$ as 

\begin{equation}
\R^{2m+1} = Z \oplus \R Q'(0) \oplus Y^+ \oplus Y^-
\end{equation}

where $Z = \spn\{ \Psi(0), \Psi^c(0) \}$, and $Z \perp \R Q'(0) \oplus Y^+ \oplus Y^-$.

Finally, we will look at the Melnikov integrals relevant to the problem. For the lowest order Melnikov integral, we have

\begin{equation}\label{M1}
M_1 = \int_{-\infty}^\infty \langle \Psi(x), B Q'(x) \rangle =
\int_{-\infty}^\infty q(x) \: \partial_x q(x) = 0
\end{equation}

since $q(x)$ is an even function and the last component of $\Psi(x)$ is $q(x)$. Recall that 

\[
\ker [\partial_x E''(q)]^* = \ker(-E''(q) \partial_x) = 
\text{span}\{ q, 1 \}
\]

Since $\langle \partial_x q, q \rangle = 0$ (from above) and $\langle \partial_x q, 1 \rangle = 0$ (since $q$ is an even function), $q \perp \ker [\partial_x E''(q)]^*$. Thus, by the Fredholm alternative, there exists a function $t(x) \in H^{2m}(\R)$ such that $[ \partial_x E''(q) ]t(x) = \partial_x q(x)$. Let $T(x) = (t(x), \partial_x t(x), \dots, \partial_x^{2m}(x))^T$. Then $T(x)$ solves the equation

\begin{equation}\label{eqforT}
T' = A(Q)T + B Q'
\end{equation}

We take the following hypothesis regarding the higher order Melnikov integral.

\begin{hypothesis}\label{Melnikov2hyp}
We have the following Melnikov-like expression
\begin{equation}\label{M2}
M_2 = \int_{-\infty}^\infty \langle \Psi(x), B T(x) \rangle dx =
\int_{-\infty}^\infty q(x) t(x) dx \neq 0
\end{equation}
\end{hypothesis}

We can now state the stability theorem for periodic multi-pulse solutions. The proof uses Lin's method to find the eigenvalues of \eqref{PDEeig}.

% main stability theorem


\subsection{KdV5}

For KdV5, a primary pulse solution is known to exist (Chug2007, among others), so Hypothesis \ref{transverseQ} is satisfied. For $c > 1/4$, Hypothesis \ref{hypeq} is satisfied. Finally, for KdV5, \eqref{genODE} is Hamiltonian, so Hypothesis \ref{Hhyp} is satisfied. Thus, Theorem \ref{perexist} holds, and $n-$periodic solutions exist.\\

We assume Hypothesis \ref{nondegen} holds for KdV5. The adjoint variational equation has two solutions, $\Psi(x)$ and $\Psi^c(x)$, which are given by

\begin{equation}\label{KdV5psi}
\Psi(x) = \begin{pmatrix}
q^{(4)}(x) - q''(x) + (-2q(x) + c)q(x)\\
-q^{(3)}(x) + q'(x) \\
q''(x) - q(x) \\
-q'(x) \\
q(x)
\end{pmatrix}
\end{equation}

and

\begin{align}\label{KdV5psic}
\Psi^c(x) = (c - 2 q(x), 0, -1, 0, 1)^T
\end{align}

where $q(x)$ is the primary pulse solution (as a real-valued function). These satisfy Hypothesis \ref{adjsolutions}. Finally, the Melnikov integrals are given by

\begin{align*}
M_1 &= \int_{-\infty}^\infty q(x) q'(x) dx = 0 \\
M_2 &= \int_{-\infty}^\infty q(x) q_c(x) dx \\
M^c &= \int_{-\infty}^\infty q_c(x) dx
\end{align*}

We assume that $M_2 \neq 0$, which is suggested by numerics. $M^c$ does not matter, although numerics suggests that it is also nonzero.\\

In Theorem \ref{PDEeigtheorem}, for KdV5 we have $\tilde{a}_i = a_i$, since the primary pulse $q(x)$ is an even function. Thus the matrix $A$ in \ref{PDEeigtheorem} is given by

\begin{align*}
A &= \begin{pmatrix}
-a_0 -a_1 & a_0 + a_1 \\
a_0 + a_1 & -a_0 - a_1
\end{pmatrix} && n = 2 \\
A &= \begin{pmatrix}
-a_{n-1} - a_0 & a_0 & & & \dots & a_{n-1}\\
a_0 & -a_0 - a_1 &  a_1 \\
& a_1 & -a_1 - a_2 &  a_2 \\
& & \vdots & & \vdots \\
a_{n-1} & & & & a_{n-2} & -a_{n-2} - a_{n-1} \\
\end{pmatrix} && n > 2
\end{align*}

\end{document}