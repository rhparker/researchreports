\documentclass[12pt]{article}
\usepackage[pdfborder={0 0 0.5 [3 2]}]{hyperref}%
\usepackage[left=1in,right=1in,top=1in,bottom=1in]{geometry}%
\usepackage[shortalphabetic]{amsrefs}%
\usepackage{amsmath}
\usepackage{enumerate}
% \usepackage{enumitem}
\usepackage{amssymb}                
\usepackage{amsmath}                
\usepackage{amsfonts}
\usepackage{amsthm}
\usepackage{bbm}
\usepackage[table,xcdraw]{xcolor}
\usepackage{tikz}
\usepackage{float}
\usepackage{booktabs}
\usepackage{svg}
\usepackage{mathtools}
\usepackage{cool}
\usepackage{url}
\usepackage{graphicx,epsfig}
\usepackage{makecell}
\usepackage{array}

\def\noi{\noindent}
\def\T{{\mathbb T}}
\def\R{{\mathbb R}}
\def\N{{\mathbb N}}
\def\C{{\mathbb C}}
\def\Z{{\mathbb Z}}
\def\P{{\mathbb P}}
\def\E{{\mathbb E}}
\def\Q{\mathbb{Q}}
\def\ind{{\mathbb I}}

\graphicspath{ {periodic/} }

\newtheorem{lemma}{Lemma}
\newtheorem{corollary}{Corollary}
\newtheorem{definition}{Definition}
\newtheorem{assumption}{Assumption}
\newtheorem{hypothesis}{Hypothesis}

\begin{document}

\section*{Double Pulse}

Now we repeat the above in the case of a periodic double pulse $q_{2p}$. The picture looks like

\begin{figure}[H]
\includegraphics[width=8.5cm]{dpimage}
\end{figure}

Note that because of the way this method is set up, the period of this thing is $2(X_1 + X_2)$. The existence proof (if such a thing exists) constructs this whole thing at once, rather than constructing a double pulse and then making it periodic. \\

The equations we are looking to solve are

\begin{enumerate}[(i)]
\item $(W_i^\pm)' = A(q; \lambda) W_i^\pm + G_i(\lambda)^\pm W_i^\pm + \lambda^2 d_i \tilde{H}_i^\pm$
\item $W_i^\pm(0) \in \C \psi(0) \oplus Y^+ \oplus Y^- \oplus Y^0$
\item $W_i^+(0) - W_i^-(0) \in \C \psi(0) $
\item $W_1^+(X_1) - W_2^-(-X_1) = D_1 d$
\item $W_2^+(X_2) - W_1^-(-X_2) = D_2 d$
\end{enumerate}

where

\begin{align*}
G_i(\lambda)^\pm &= A(q_{2p};\lambda) - A(q;\lambda)) \\
D_1 d &= d_2(Q_{2p}'(-X_1) + \lambda (Q_{2p})_c(-X_1))
- d_1 ( Q_{2p}'(X_1) + \lambda (Q_{2p})_c(X_1) ) \\
D_1 d &= d_1(Q_{2p}'(-X_2) + \lambda (Q_{2p})_c(-X_2))
- d_2 ( Q_{2p}'(X_2) + \lambda (Q_{2p})_c(X_2) ) \\
\tilde{H} &= -B(Q_{2p})_c \\
H &= -B Q_c \\
\Delta H &= \tilde{H} - H
\end{align*}

NOTE TO SELF: DOUBLE CHECK THE SIGN FOR $D_2 d$. We do not have a $\lambda B W_i^\pm$ term since this has been absorbed into $A(q; \lambda) W_i^\pm$. \\

Let $X_m = \min\{ X_1, X_2 \}$ and $X_M = \max\{ X_1, X_2 \}$. Then we should have bounds WHICH WE SHOULD SHOW IF THIS ALL WORKS

\begin{align*}
G_i(\lambda) &= \mathcal{O}(e^{-\alpha X_m}) \\
\Delta H &= \mathcal{O}(e^{-\alpha X_m}) \\
D_i &= ( Q'(X_i) + Q'(-X_i)(d_2 - d_1 ) + \mathcal{O} \left( e^{-\alpha X_i} \left( |\lambda| +  e^{-\alpha X_i}  \right) |d| \right)
\end{align*}

The idea behind these (unproven) bounds is that deviation from the single pulse, nonperiodic case decreases exponentially with distance from the center, so we are letting the ``worst offender'' dictate the behavior. The $D_i$ equations only depend on $X_i$, so the bound only involves that. \\

For the setup, let

\[
X = (X_0, X_1, X_2) = (X_2, X_1, X_2)
\]

YES THIS IS AWFUL NOTATION, CAN FIX IF THIS WORKS. The fixed point equations then become

\begin{align*}
W_i^-(x) = \Phi^s_-(&x, -X_{i-1}; \lambda)a_{i-1}^- + \Phi^u_-(x, 0; \lambda)b_i^- + e^{\nu(\lambda)(x+X_{i-1})} v_-(x; \lambda) \langle v_0(\lambda), w_-(-X_{i-1}; \lambda) \rangle c_{i-1}^- \\
&+ \int_0^x \Phi^u_-(x, y; \lambda)[ G_i^-(\lambda)W_i^-(y) + \lambda^2 d_i \tilde{H}(y) ] dy \\
&+ \int_{-X_{i-1}}^x \Phi^s_-(x, y; \lambda) [ G_i^-(\lambda)W_i^-(y) + \lambda^2 d_i \tilde{H}(y) ] dy \\
&+ \int_{-X_{i-1}}^x 
e^{\nu(\lambda)(x-y)} v_-(x; \lambda) \langle G_i^-(\lambda)(y)W_i^-(y) + \lambda^2 d_i \tilde{H}(y), w_-(y; \lambda) \rangle dy \\
W_i^+(x) = \Phi^u_+(&x, X_i; \lambda)a_i^+ + \Phi^s_+(x, 0; \lambda)b_i^+ + e^{\nu(\lambda)(x - X_i)} v_+(x; \lambda) \langle v_0(\lambda), w_+(X_i; \lambda) \rangle c_i^+ \\
&+ \int_0^x \Phi^s_+(x, y; \lambda) [ G_i^+(\lambda)W_i^+(y) + \lambda^2 d_i \tilde{H}(y) ] dy \\
&+ \int_{X_i}^x \Phi^u_+(x, y; \lambda) [ G_i^+(\lambda)W_i^+(y) + \lambda^2 d_i \tilde{H}(y) ] dy \\
&+ \int_{X_i}^x e^{\nu(\lambda)(x-y)} v_+(x; \lambda) \langle G_i^+(\lambda)(y)W_i^+(y) + \lambda^2 d_i \tilde{H}(y), w_+(y; \lambda) \rangle dy
\end{align*}

where

\begin{align*}
(a^-, a^+) &\in E^s \oplus E^u\\
(b^-, b^+) &\in R^u_-(0; 0) \oplus R^s_+(0; 0)\\
\end{align*}

Note that these spaces refer to the original, unperturbed problem, i.e. with $\lambda = 0$. The projections onto $E^s$, $E^u$, and $E^c$ are given by $P_0^s$, $P_0^u$, and $P_0^c$. The initial conditions on the ``center'' subspace is given by $c^\pm v_0(\lambda)$, where here $v_0(\lambda)$ refers to the perturbed problem. This is useful since we have equations for the evolution along this subspace.\\

Now we do the same thing we did with the single pulse. The main difference here is the presence of the $d_i$ in the $\tilde{H}$ terms. \\

We take the following notational conventions, which we use for simplicity. The first one makes sense since periodic BCs basically means that things ``wrap around'', so can can consider the problem as posed on a ``loop''.

\begin{enumerate}[(i)]
\item $c_2^- = c_0^-$, $a_2^- = a_0^-$, $b_0^- = b_2^-$, $d_0 = d_2$, and $W_0 = W_2$
\item If we eliminate either a subscript or a superscript in the norm, we are taking the max of the eliminated things, e.g. $|c_i| = \max(|c_i^+|, |c_i^-|)$ or $|c^+| = \max(|c_1^+|, |c_2^+|)$. 
\end{enumerate}

Now we can do the thing.

\begin{enumerate}

\item Fix $\tilde{\alpha}$, with $0 < \tilde{\alpha} < \alpha$. 

\item For convenience, we are taking $\nu(\lambda) \geq 0$. If this works, we will redo it with the appropriate absolute value signs. The idea is that we know one of $e^{\nu(\lambda) X_1}$ and $e^{-\nu(\lambda) X_1}$ will blow up, but we do not know which one.

\item Solve for $W_i$ in terms of the other stuff. We do the same thing with the linear operators $L_1$ and $L_2$ as we did in the single pulse case.\\

Linear operator $L_1$ is stuff from the fixed point equations involving $W$.

\begin{align*}
(L_1(\lambda)W)_i^-(x) &= \int_0^x \Phi^u_-(x, y; \lambda) G_i^-(\lambda)W_i^-(y) dy + \int_{-X_{i-1}}^x \Phi^s_-(x, y; \lambda) G_i^-(\lambda)W_i^-(y) dy \\
&+ \int_{-X_{i-1}}^x 
e^{\nu(\lambda)(x-y)} v_-(x; \lambda) \langle G_i^-(\lambda)(y)W_i^-(y), w_-(y; \lambda) \rangle dy \\
(L_1(\lambda)W)_i^-(x) &= \int_0^x \Phi^s_+(x, y; \lambda) G_i^+(\lambda)W_i^+(y) dy + \int_{X_i}^x \Phi^u_+(x, y; \lambda) G_i^+(\lambda) W_i^+(y) dy \\
&+ \int_{X_i}^x e^{\nu(\lambda)(x-y)} v_+(x; \lambda) \langle G_i^+(\lambda)(y)W_i^+(y), w_+(y; \lambda) \rangle dy
\end{align*}

The first two terms on the RHS of this are like those from Sanstede (1998). For the third term we have for the negative piece,

\begin{align*}
\Big| \int_{-X_{i-1}}^x &e^{\nu(\lambda)(x-y)} v_-(x; \lambda) \langle G(\lambda)(y)W_i^-(y), w_-(y; \lambda) \rangle dy \Big| \\
&\leq \int_{-X_{i-1}}^x e^{\nu(\lambda)(x-y)} |v_-(x; \lambda)| |G(\lambda)|||W|||w_-(y; \lambda)|dy \\
&\leq |G||v||w|||W|| \int_{-X_{i-1}}^x e^{\nu(\lambda)(x-y)} dy \\
&= |G||v||w|||W|| \frac{e^{\nu(\lambda)x} - 1}{\nu(\lambda)} \\
&\leq C e^{\nu(\lambda)X_{i-1}} |G| \: ||W||
\end{align*}

where we used the fact that $x \leq 0$ on the negative piece. Since $v$ and $w$ are bounded and only depend on $\lambda$ (we pulled out the exponential growth/decay in our expressions for $\tilde{v}$ and $\tilde{w}$), we incorporate those bounds into the constant $C$, which depends on $\lambda$. The positive piece has a similar bound with $X_{i-1}$ replaced with $X_i$. Thus since this depends on the larger of the $X_i$ the overall bound is 

\[
||L_1(\lambda)W|| \leq C e^{\nu(\lambda)X_M} |G| \: ||W||
\]

Linear operator $L_2$ is the stuff from fixed point equations not involving $W$.

\begin{align*}
(L_2(\lambda)&(a,b,c,d))^-(x) = \Phi^s_-(x, -X_{i-1}; \lambda)a_{i-1}^- + \Phi^u_-(x, 0; \lambda)b_i^- \\
&+ e^{\nu(\lambda)(x+X_{i-1})} v_-(x; \lambda) \langle v_0(\lambda), w_-(-X_{i-1}; \lambda) \rangle c_{i-1}^- \\
&+ \int_0^x \Phi^u_-(x, y; \lambda)\lambda^2 d_i \tilde{H}(y) dy + \int_{-X_{i-1}}^x \Phi^s_-(x, y; \lambda) \lambda^2 d_i \tilde{H}(y) dy \\
&+ \int_{-X_{i-1}}^x 
e^{\nu(\lambda)(x-y)} v_-(x; \lambda) \langle \lambda^2 d_i \tilde{H}(y), w_-(y; \lambda) \rangle dy \\
(L_2(\lambda)&(a,b,c,d))^+(x) \Phi^u_+(x, X_i; \lambda)a_i^+ + \Phi^s_+(x, 0; \lambda)b_i^+ \\
&+ e^{\nu(\lambda)(x - X_i)} v_+(x; \lambda) \langle v_0(\lambda), w_+(X_i; \lambda) \rangle c_i^+ \\
&+ \int_0^x \Phi^s_+(x, y; \lambda) \lambda^2 d_i \tilde{H}(y) dy + \int_{X_i}^x \Phi^u_+(x, y; \lambda) \lambda^2 d_i \tilde{H}(y) dy \\
&+ \int_{X_i}^x e^{\nu(\lambda)(x-y)} v_+(x; \lambda) \langle \lambda^2 d_i \tilde{H}(y), w_+(y; \lambda) \rangle dy
\end{align*}

Most of the bounds on these terms are the same as in Sanstede (1998). For the $c$ terms, we have

\[
e^{\nu(\lambda)(x+X_{i-1})} v_-(x; \lambda) \langle v_0(\lambda), w_-(-X_{i-1}; \lambda) \rangle c_{i-1}^- \leq C e^{\nu(\lambda) X_{i-1} }|c_{i-1}^-|
\]

and similar for the $c_i^+$. In each case, the length scale $X_i$ is paired with $c_i^\pm$. We have an $e^{\nu(\lambda)X_i}$ term in the bound, which could grow exponentially, but for now there is nothing we can do about it. \\

For the third integrals in $L_2$, we use the $\tilde{\alpha}$ trick to get a better bound which does not involve a potential exponential growth term.

\begin{align*}
&\left| \int_{-X_{i-1}}^x 
e^{\nu(\lambda)(x-y)} v_-(x; \lambda) \langle \lambda^2 d_i \tilde{H}(y), w_-(y; \lambda) \rangle dy \right| \\
&\leq C |\lambda|^2 |d| e^{\tilde{\alpha}x} \int_{-X_{i-1}}^x e^{-\tilde{\alpha}x} e^{\tilde{\alpha}y} e^{\nu(\lambda)(x-y)} |e^{-\tilde{\alpha}y}\tilde{H}(y)|dy \\
&\leq C |\lambda|^2 |d| e^{\tilde{\alpha}x} \int_{-X_{i-1}}^x e^{-\tilde{\alpha}(x-y)} e^{\nu(\lambda)(x-y)} |e^{-\tilde{\alpha}y}\tilde{H}(y)|dy \\
&\leq C |\lambda|^2 |d| \int_{-X_{i-1}}^0 e^{(\tilde{\alpha}-\nu(\lambda))y} |e^{-\tilde{\alpha}y}\tilde{H}(y)|dy \\
&\leq C |\lambda|^2 |d|
\end{align*}

since $|e^{-\tilde{\alpha}y}\tilde{H}(y)|$ is bounded (decay rate of $\tilde{H}$ is known) and $|\nu(\lambda)| < \tilde{\alpha}$. Thus for $L_2$ we have the overall bound

\[
|L_2(\lambda)(a,b,c,d)| \leq C (|a| + |b| + e^{\nu(\lambda)X_1}|c_1| + e^{\nu(\lambda)X_2}|c_2| + |\lambda|^2 |d| )
\]

Now we do the inversion like we have done in the past, and as in Sandstede (1998). Omitting the details of this for now, we can invert the expression $(I - L_1(\lambda))W = L_2(\lambda)(a,b,c,d)$ to get $W = W_1(\lambda)(a,b,c,d)$, where we have the bound

\[
||W_1(\lambda)(a,b,c,d)|| \leq C (|a| + |b| + e^{\nu(\lambda)X_1}|c_1| + e^{\nu(\lambda)X_2}|c_2| + |\lambda|^2 |d| )
\]

Note that by symmetry, for $i = 1, 2$ this is always of the form

\[
||W_1(\lambda)(a,b,c,d)|| \leq C (|a| + |b| + e^{\nu(\lambda)X_i}|c_i| + e^{\nu(\lambda)X_{i-1}}|c_{i-1}| + |\lambda|^2 |d| )
\]


\item Solve for the joins which are not at 0, i.e. solve

\begin{align*}
W_2^+(X_2) - W_1^-(-X_2) &= D_2 d \\
W_1^+(X_1) - W_2^-(-X_1) &= D_1 d \\
\end{align*}

To solve these, we have two equations for $i = 1, 2$. Using our notation convention for ``wrapping around'', we start with

\begin{align*}
W_i^+(X_i) &- W_{i-1}^-(-X_i) = P^u_+(X_i; \lambda) a_i^+ - P^s_-(-X_i; \lambda) a_i^- \\
&+ \Phi^s_+(X_i, 0; \lambda)b_i^+ - \Phi^u_-(-X_i, 0; \lambda)b_{i-1}^- \\
&+ v_+(X_i; \lambda) \langle v_0(\lambda), w_+(X_i; \lambda) \rangle c_i^+ - v_-(-X_i; \lambda) \langle v_0(\lambda), w_-(-X_i; \lambda) \rangle c_i^- \\
&+ \int_0^{-X_i} \Phi^u_+(-X_i, y; \lambda) [ G(\lambda)W_i^+(y) + d_i \lambda^2 \tilde{H}(y) ] dy \\
&- \int_0^{X_i} \Phi^s_-(X_i, y; \lambda) [ G(\lambda)W_{i-1}^-(y) + d_{i-1} \lambda^2 \tilde{H}(y) ] dy
\end{align*}

First, we get the coefficients $a_i^\pm$ by themselves by adding and subtracting $P_0^u a_i^+$ and $P_0^s a_i^-$. Recalling where the various $a_i^\pm$ live and what happens when we hit them with the projections on $E^u$ and $E^s$, this becomes

\begin{align*}
D_i d &= a_i^+ - a_i^- \\
&+ (P^u_+(X_i; \lambda) - P_0^u)a_i^+ - (P^s_-(-X_i; \lambda) - P_0^s)a_i^- \\
&+ \Phi^s_+(X_i, 0; \lambda)b_i^+ - \Phi^u_-(-X_i, 0; \lambda)b_{i-1}^- \\
&+ v_+(X_i; \lambda) \langle v_0(\lambda), w_+(X_i; \lambda) \rangle c_i^+ - v_-(-X_i; \lambda) \langle v_0(\lambda), w_-(-X_i; \lambda) \rangle c_i^- \\
&+ \int_0^{-X_i} \Phi^u_+(-X_i, y; \lambda) [ G(\lambda)W_i^+(y) + d_i \lambda^2 \tilde{H}(y) ] dy \\
&- \int_0^{X_i} \Phi^s_-(X_i, y; \lambda) [ G(\lambda)W_{i-1}^-(y) + d_{i-1} \lambda^2 \tilde{H}(y) ] dy
\end{align*}

For a bound on the ``projection difference'', let

\[
p_1(X;\lambda) = \sup_{x \geq X} (|P^u(x;\lambda) - P_0^u| + |P^s(-X;\lambda) - P_0^s|)
\]

Note that this will change with both $\lambda$ and with $X$, but the effect will not be multiplicative (since this will still happen if we take $\lambda = 0$. Thus this should be order $e^{-\alpha T} + |\lambda|$. \\

We next follow what we did in the single pulse case, but this time we solve for $c_i^+$ in terms of $c_i^-$ rather than solving for the difference $\Delta c$. (I like that approach, but it does not work here since the joins at 0 and at $X_i$ involve different sets of coefficients). First, we manipulate things to get a $c^+ v_0(\lambda)$ term in front. This is the same as what was done in the single pulse case. For a generic $X$, we have

\begin{align*}
c^+ &v_+(X; \lambda)\langle v_0(\lambda), w_+(X; \lambda) \rangle \\
&= c^+ v_+(X; \lambda)\langle v_0(\lambda), w_0(\lambda) \rangle + c^+ v_+(X; \lambda)( \langle v_0(\lambda), w_+(X; \lambda) \rangle - \langle v_0(\lambda), w_0(\lambda) \rangle )\\
&= c^+ v_+(X; \lambda) + c^+ v_+(X; \lambda) \langle v_0(\lambda), w_+(X; \lambda) - w_0(\lambda) \rangle \\
&= c^+ v_0(\lambda) + c^+( v_+(X; \lambda) - v_0(\lambda)) + c^+ v_+(X; \lambda) \langle v_0(\lambda), w_+(X; \lambda) - w_0(\lambda) \rangle \\
&= c^+ v_0(\lambda) + c^+( v_+(X; \lambda) - v_0(\lambda)) + c^+ v_+(X; \lambda) \langle v_0(\lambda), w_+(X; \lambda) - w_0(\lambda) \rangle \\
&= c^+ v_0(\lambda) + c^+ \Delta v_+(X; \lambda) + c^+ v_+(X; \lambda) \langle v_0(\lambda), \Delta w_+(X; \lambda) \rangle \\
\end{align*}

where

\begin{align*}
\Delta v_\pm(x; \lambda) &= v_\pm(x; \lambda) - v_0(\lambda) \\
\Delta w_\pm(x; \lambda) &= w_\pm(x; \lambda) - w_0(\lambda)
\end{align*}

Recall that $v^-(-X)$ and $v^+(X)$ converge in the appropriate direction to $v_0(\lambda)$, and $w^-(-X)$ and $w^+(X)$ converge in the appropriate direction to $w_0(\lambda)$. Thus we define

\begin{align*}
p_2(X; \lambda) &= |\Delta v_\pm(\pm X, \lambda)| + |\Delta w_\pm(\pm X, \lambda)|\\
&= |v_\pm(\pm X; \lambda) - v_0(\lambda)| + |w_\pm(\pm X; \lambda) - w_0(\lambda)|
\end{align*}

This convergence is not exponential (since we removed the exponential part), but we can still take $X$ sufficiently large to make this as small as we need. Making this substitution, we have

\begin{align*}
D_i d &= a_i^+ - a_i^- + c_i^+ v_0(\lambda) \\
&+ (P^u_+(X_i; \lambda) - P_0^u)a_i^+ - (P^s_-(-X_i; \lambda) - P_0^s)a_i^- \\
&+ \Phi^s_+(X_i, 0; \lambda)b_i^+ - \Phi^u_-(-X_i, 0; \lambda)b_{i-1}^- \\
&+ c_i^+ \Delta v_+(X_i; \lambda) + c_i^+ v_+(X_i; \lambda) \langle v_0(\lambda), \Delta w_+(X_i; \lambda) \rangle \\
&- c_i^- v_-(-X_i; \lambda) \langle v_0(\lambda), w_-(-X_i; \lambda) \rangle \\
&+ \int_0^{-X_i} \Phi^u_+(-X_i, y; \lambda) [ G(\lambda)W_i^+(y) + d_i \lambda^2 \tilde{H}(y) ] dy \\
&- \int_0^{X_i} \Phi^s_-(X_i, y; \lambda) [ G(\lambda)W_{i-1}^-(y) + d_{i-1} \lambda^2 \tilde{H}(y) ] dy
\end{align*}

Then we have 

\begin{align*}
D_i d &= a_i^+ - a_i^- + c_i^+ v_0(\lambda) + L_3(\lambda)_i(a, b, c^+, c^-, d)
\end{align*}

Where $L_3(\lambda)_i(a, b, c^+, c^-, d)$ is the rest of the stuff on the RHS.

\begin{align*}
L_3(\lambda)_i&(a, b, c^+, c^-, d) \\ 
&= (P^u_+(X_i; \lambda) - P_0^u)a_i^+ - (P^s_-(-X_i; \lambda) - P_0^s)a_i^- \\
&+ \Phi^s_+(X_i, 0; \lambda)b_i^+ - \Phi^u_-(-X_i, 0; \lambda)b_{i-1}^- \\
&+ c_i^+ \Delta v_+(X_i; \lambda) + c_i^+ v_+(X_i; \lambda) \langle v_0(\lambda), \Delta w_+(X_i; \lambda) \rangle \\
&- c_i^- v_-(-X_i; \lambda) \langle v_0(\lambda), w_-(-X_i; \lambda) \rangle \\
&+ \int_0^{-X_i} \Phi^u_+(-X_i, y; \lambda) [ G(\lambda)W_i^+(y) + d_i \lambda^2 \tilde{H}(y) ] dy \\
&- \int_0^{X_i} \Phi^s_-(X_i, y; \lambda) [ G(\lambda)W_{i-1}^-(y) + d_{i-1} \lambda^2 \tilde{H}(y) ] dy
\end{align*}

For the bound on $L_3$, we will again use the $\tilde{\alpha}$ trick to get a better bound for the term involving $\tilde{H}$. We can do this since we know the decay rate of $\tilde{H}$.

\begin{align*}
\left| \int_0^{X_i} \Phi^s_+(X_i, y; \lambda) \tilde{H}(y) dy \right| 
&\leq C \int_0^{X_i} e^{-\alpha (X_i - y)}|\tilde{H}(y)| dy \\
&= C e^{-\tilde{\alpha}X_i} \int_0^{X_i} e^{-\alpha X_i} e^{\alpha y}  e^{\tilde{\alpha}X_i} e^{-\tilde{\alpha}y} |e^{\tilde{\alpha}y} \tilde{H}(y)| \\
&= C e^{-\tilde{\alpha}X_i} \int_0^{X_i} e^{-(\alpha - \tilde{\alpha})(X_i-y)} |e^{\tilde{\alpha}y} \tilde{H}(y)|\\
&\leq C e^{-\tilde{\alpha}X_i} 
\end{align*}

where we again used the fact that $|e^{\tilde{\alpha}y} \tilde{H}(y)|$ is bounded, which holds since $\tilde{\alpha}$ is smaller than $\alpha$ and $\tilde{H}(y)$ decays with rate $\alpha$.\\

\[
L_3(\lambda)_i(a, b, c^+, c^-, d) \leq C ( p_1(X_i; \lambda)|a_i|
+ e^{-\alpha X_i}|b| + p_2(X_i; \lambda)|c_i^+| + |c_i^-| + |G| ||W|| + e^{-\tilde{\alpha} X_i} |\lambda^2| |d| )
\]

Note that so far this involves only $a_i$ and $c_i$, whereas it involves both $b_i$ and $b_{i-1}$. Unfortunately, things will not stay that way, since both $W_1$ and $W_2$ are involved in $L_3(\lambda)_i(a, b, c^+, c^-, d)$. Thus via $W$ (through the $||W||$ bound), we involve the other subscript for $a$ and $c$. Plugging in the bound on $W_1$ for $W$, we have

\begin{align*}
|L_3&(\lambda)_i(a, b, c^+, c^-, d)| \\
&\leq C \Big( p_1(X_i; \lambda)|a_i|
+ e^{-\alpha X_i}|b| + p_2(X_i; \lambda)|c_i^+| + |c_i^-| + |G|\:||W_1(\lambda)(a,b,c,d)|| + e^{-\tilde{\alpha} X_i} |\lambda^2| |d| \Big) \\
& \leq C \Big( p_1(X_i; \lambda)|a_i|
+ e^{-\alpha X_i}|b| + p_2(X_i; \lambda)|c_i^+| + |c_i^-| + e^{-\tilde{\alpha} X_i} |\lambda^2| |d| \\
&+ |G| (|a| + |b| + e^{\nu(\lambda)X_i}|c_i| + e^{\nu(\lambda)X_{i-1}}|c_{i-1}| + |\lambda|^2 |d| \Big) \\
& \leq C \Big( (p_1(X_i; \lambda) + |G|)|a_i| + |G||a_{i-1}| + (e^{-\alpha X_i} + |G|) |b| \\
&+ ( p_2(X_i; \lambda) + e^{\nu(\lambda)X_i} |G|) |c_i^+| + e^{\nu(\lambda)X_{i-1}} |G| |c_{i-1}^+|   \\
&+ (1 + e^{\nu(\lambda)X_i} |G|)|c_i^-| + e^{\nu(\lambda)X_{i-1}} |G||c_{i-1}^-| + (e^{-\tilde{\alpha} X_i} + |G|) |\lambda|^2 |d| \Big)
\end{align*} 

At this point we will take $X_2 \geq X_1$ without loss of generality. We can do this 

At this point, we have a small complication. Since we have (or are assuming!) that $|G|$ is order $e^{-\alpha X_m} = e^{-\alpha X_1}$, one of these will have a term which is order $e^{\nu(\lambda)X_2} e^{-\alpha X_1}$, and there is no telling what this going to do as we increase our two length parameters. We can without loss of generality take $X_2 \geq X_1$, since it does not matter where we place the origin. Thus we will have $X_M = X_2$ and $X_m = X_1$. To handle the other issue, the simplest way to do this is to have $X_2$ depend on $X_1$. Let's do the easiest thing and let

\[
X_2 = k X_1
\]

where $k \geq 1$. Then for the inversion to work, we need the following:

\begin{enumerate}[(i)]
\item Choose $\delta > 0$ small
\item Choose $0 < \tilde{\delta} \leq \delta$ such that for all $|\lambda| < \tilde{\delta}$ we also have $|\nu{\lambda}| \leq \delta$. Since $\nu(\lambda)$ has order $\lambda$, this is possible.
\item Choose $k$ so that $\delta k < \alpha$. 
\item This implies $\alpha - \nu(\lambda) k > 0$ for all $|\lambda| < \tilde{\delta}$
\item In particular, this means that all the $e^{\nu(\lambda)k X_1} |G|$ terms are uniformly bounded and decay in $X_1$
\item Choose $X_1$ sufficiently large so that $p_2(X_i; \lambda) < \delta$.
\end{enumerate}

Assume we have done this. For simplicity, we will often leave these bounds in terms of the subscript $i = 1, 2$, but these conditions are necessary for the bounds to make sense to begin with. Then we have

\begin{align*}
|L_3&(\lambda)_i(a, b, c^+, c^-, d)| \\
&\leq C \Big( (p_1(X_i; \lambda) + |G|)|a_i| + |G||a_{i-1}| + (e^{-\alpha X_i} + |G|) |b| \\
&+ ( p_2(X_i; \lambda) + e^{\nu(\lambda)X_i} |G|) |c_i^+| + e^{\nu(\lambda)X_{i-1}} |G|) |c_{i-1}^+| \\
&+ |c_i^-| + e^{\nu(\lambda)X_{i-1}} |G||c_{i-1}^-| + (e^{-\tilde{\alpha} X_i} + |G|) |\lambda|^2 |d| \Big)
\end{align*} 

Written out separately for $i = 1, 2$ this is 

\begin{align*}
|L_3&(\lambda)_1(a, b, c^+, c^-, d)| \leq C (p_1(X_1; \lambda) + |G|)|a_1| + |G||a_2| + (e^{-\alpha X_1} + |G|) |b|\\
&+ ( p_2(X_1; \lambda) + e^{\nu(\lambda)X_1} |G|) |c_1^+| + e^{\nu(\lambda) k X_1} |G|) |c_2^+|\\
&+ |c_1^-| + e^{\nu(\lambda)k X_1} |G||c_2^-| + (e^{-\tilde{\alpha} X_1} + |G|) |\lambda|^2 |d| ) \\
|L_3&(\lambda)_2(a, b, c^+, c^-, d)| \leq C (p_1(k X_1; \lambda) + |G|)|a_2| + |G||a_1| + (e^{-\alpha k X_1} + |G|) |b| \\
&+ ( p_2(k X_1; \lambda) + e^{\nu(\lambda)k X_1} |G|) |c_2^+| + e^{\nu(\lambda)X_1} |G|) |c_1^+|\\
&+ |c_2^-| + e^{\nu(\lambda)X_1} |G||c_1^-| + (e^{-\tilde{\alpha} k X_1} + |G|) |\lambda|^2 |d| )
\end{align*}

Let $J_1: V_a \times \text{span }\{v_0(\lambda)\} \rightarrow \C^n$ be defined by $J_i(a_i, c_i^+) = (a_i^+ - a_i^-, c_i^+)$. The map $J_i$ is a linear isomorphism. Now consider the map

\[
S_i(a_i, c_i^+) = J_i (a_i, c_i^+) + L_3(\lambda)_i(a_i, 0, c_i^+, 0, 0) = J_i( I + J_i^{-1} L_3(\lambda)_i(a_i, 0, c_i^+, 0, 0))
\]

For suffiently small $\delta$, by what we have above, we can get the operator norm 

\[
J_i^{-1} L_3(\lambda)_i(\cdot, 0, \cdot, 0, 0)|| < 1
\]

thus the map $(a_i, c^+) \rightarrow I + J_1^{-1} L_3(\lambda)_i(a_i, 0, c^+, 0, 0)$ is invertible and so the operator $S_i$ is invertible.\\

Thus we can solve for $(a, c^+)$ to get

\[
(a_i, c_i^+) = A_1(\lambda)_i(b, c_i, d) = S_i^{-1}(D_i d - L_3(\lambda)_i(0, b, 0, c_i, d))
\]

Using the bound on $L_3$ together with $|D_i|$, $A_1$ will have bound

\begin{align*}
|A_1&(\lambda)_i(b, c^-, d)| \\
&\leq C \Big( (e^{-\alpha X_i} + |G|) |b| 
+ |c_i^-| + e^{\nu(\lambda)X_{i-1}} |G||c_{i-1}^-| + (e^{-\tilde{\alpha} X_i} + |G|) |\lambda|^2 |d| + |D_i||d| \Big)
\end{align*} 

which on each piece ($i = 1, 2$) is

\begin{align*}
|A_1&(\lambda)_1(b, c^-, d)| \leq C ((e^{-\alpha X_1} + |G|) |b| + |c_1^-| + e^{\nu(\lambda)k X_1} |G||c_2^-| + (e^{-\tilde{\alpha} X_1} + |G|) |\lambda|^2 |d| + |D_1||d| ) \\
|A_1&(\lambda)_2(b, c^-, d)| \leq C ((e^{-\alpha k X_1} + |G|) |b| + |c_2^-| + e^{\nu(\lambda) X_1} |G||c_1^-| + (e^{-\tilde{\alpha} k X_1} + |G|) |\lambda|^2 |d| + |D_2||d| ) \\
\end{align*}

The overall bound (worst case scenario) is 

\begin{align*}
|A_1&(\lambda)(b, c^-, d)| \leq C( (e^{-\alpha X_1} + |G|) |b| + |c^-| + (e^{-\tilde{\alpha} X_1} + |G|) |\lambda|^2 |d| + |D| |d| )
\end{align*}

This bound is similar in form to (3.24) in Sanstede (1998) with the exceptions of: an additional $|c^-|$ since we have a ``center'' space in play; and a better bound for the $|\lambda|^2$ term, thanks to the $\tilde{\alpha}$ trick. The $|D| |d| $ term is by itself, as in Sanstede (1998). I think this bound is the best we can get at this point.

We can plug this into our expression for $W_1$ to get $W_2(\lambda)$. Note that we plug this in for both $|a|$ and $|c^+|$ since this solved for both simultaneously. We then have

\begin{align*}
||W_2&(\lambda)(b,c^-,d)|| \\
&\leq C \Big(|a| + |b| + e^{\nu(\lambda)X_1}|c_1| + e^{\nu(\lambda)X_2}|c_2| + |\lambda|^2 |d| \Big) \\
&\leq C \Big(|a_1| + |a_2| + |b| + e^{\nu(\lambda)X_1}|c_1^+| + e^{\nu(\lambda)X_2}|c_2^+| + e^{\nu(\lambda)X_1}|c_1^-| + e^{\nu(\lambda)X_2}|c_2^-| + |\lambda|^2 |d| \Big) \\
&\leq C \Big( (1 + e^{\nu(\lambda)X_1}) |A_1(\lambda)_1(b, c^-, d)| + (1 + e^{\nu(\lambda)X_2})|A_1(\lambda)_2(b, c^-, d)| + |b| + |\lambda|^2 |d| \Big)\\
&= C \Big( (1 + e^{\nu(\lambda)X_1}) |A_1(\lambda)_1(b, c^-, d)| + (1 + e^{\nu(\lambda)k X_1 })|A_1(\lambda)_2(b, c^-, d)| + |b| + |\lambda|^2 |d| \Big)\\
&= C \Big( e^{\nu(\lambda)X_1} ((e^{-\alpha X_1} + |G|) |b| + |c_1^-| + e^{\nu(\lambda)k X_1} |G||c_2^-| + (e^{-\tilde{\alpha} X_1} + |G|) |\lambda|^2 |d| + |D_1||d| )\\
&+ e^{\nu(\lambda)k X_1} ((e^{-\alpha k X_1} + |G|) |b| + |c_2^-| + e^{\nu(\lambda) X_1} |G||c_1^-| + (e^{-\tilde{\alpha} k X_1} + |G|) |\lambda|^2 |d| + |D_2||d| )\\
&+ |b| + e^{\nu(\lambda)X_1}|c_1^-| + e^{\nu(\lambda)X_2}|c_2^-| + |\lambda|^2 |d| \Big) \\
&\leq C\Big( (1 + e^{\nu(\lambda)X_1} (e^{-\alpha X_1} + |G|) + e^{\nu(\lambda)k X_1} (e^{-\alpha k X_1} + |G|)) |b|\\
&+ ( 1 + e^{\nu(\lambda)k X_1} |G|) e^{\nu(\lambda)X_1}|c_1^-|
+ ( 1 + e^{\nu(\lambda) X_1} |G|) e^{\nu(\lambda)k X_1}|c_2^-|\\
&+( 1 + e^{\nu(\lambda)X_1}(e^{-\tilde{\alpha} X_1} + |G|)  + e^{\nu(\lambda)k X_1} (e^{-\tilde{\alpha} k X_1} + |G|)) |\lambda|^2 \\
&+ e^{\nu(\lambda)X_1}|D_1||d| + e^{\nu(\lambda)k X_1}|D_2||d| \Big)\\
&\leq C\Big( |b| + e^{\nu(\lambda) X_1} |c_1^-| +  e^{\nu(\lambda) k X_1} |c_2^-| + |\lambda|^2 |d| + e^{\nu(\lambda) X_1} |D_1| |d| + e^{\nu(\lambda) k X_1} |D_2| |d| \Big)
\end{align*}

So the final bound is

\begin{align*}
||W_2&(\lambda)(b,c^-,d)|| \\
&\leq C\Big( |b| + e^{\nu(\lambda) X_1} |c_1^-| +  e^{\nu(\lambda) k X_1} |c_2^-| + |\lambda|^2 |d| + e^{\nu(\lambda) X_1} |D_1| |d| + e^{\nu(\lambda) k X_1} |D_2| |d| \Big)
\end{align*}

Again, by symmetry, this takes the form for $i = 1, 2$

\begin{align*}
||W_2&(\lambda)(b,c^-,d)|| \\
&\leq C\Big( |b| + e^{\nu(\lambda) X_i} |c_i^-| +  e^{\nu(\lambda) x_{i-1}} |c_{i-1}^-| + |\lambda|^2 |d| + e^{\nu(\lambda) X_i} |D_i| |d| + e^{\nu(\lambda) X_{i-1}} |D_{i-1}| |d| \Big)
\end{align*}

This is essentially the same as the $W_1$ bound, which should be fine.\\

With a multipulse, we will need an analogue of (3.25) in Sandstede (1998). The idea here is that we hit our expression for $D_i d$ with projections to kill some of the terms. We start with

\[
D_i d = a_i^+ - a_i^- + c_i^+ v_0(\lambda) + L_3(\lambda)_i(a, b, c^+, c^-, d)
\]

Adding and substracting $v_0(0)$ (the unit vector for $E^c$), this becomes

\[
a_i^+ - a_i^- + c_1^+ v_0(0) = D_i d - c_1^+ (v_0(\lambda) - v_0(0)) - L_3(\lambda)_i(a, b, c^+, c^-, d)
\]

Let 

\[
p_5(\lambda) = |v_0(\lambda) - v_0(0)| 
\]

Either by Taylor expansion or geometric intuition (DETAILS LATER), this should be of order $|\lambda|$, since we are basically rotating a unit vector in a $\lambda$ dependent fashion.\\

We then take projections $P^s_0$ and $P^u_0$. Recalling where the $a_i^\pm$ and $v_0(0)$ live and that $v_0(0)$ is wiped out by both projections, this becomes 

\begin{align*}
a_i^+ &= P^u_0 D_i d - c_i^+ P^u_0 (v_0(\lambda) - v_0(0)) - P^u_0 L_3(\lambda)_i(a_i, b, c_i^+, c_i^-, d) \\
a_i^- &= -P^s_0 D_i d + c_i^+ P^s_0 (v_0(\lambda) + v_0(0)) + P^s_0 L_3(\lambda)_i(a_i, b, c_i^+, c_i^-, d)
\end{align*}

Define $A_2$ to be all the stuff on the RHS other than the $D_i d$ term. Since we will plug in $A_1$ for $a$ and $c^+$, this will be $A_2(\lambda)(b, c_i^-, d)$.

\begin{align*}
a_i^+ &= P^u_0 D_i d + A_2(\lambda)_i^+(b, c_i^-, d) \\
a_i^- &= -P^s_0 D_i d + A_2(\lambda)_i^-(b, c_i^-, d)
\end{align*}

We then can come up with a bound for $A_2$ using $p_5$, the bound for $L_3$, and the bound for $A_1$.

\begin{align*}
|A_2&(\lambda)_i(b, c_i^-, d)| \\
&\leq C \Big( p_5(\lambda)|c_i^+| + |L_3(\lambda)_i(a_i, b, c_i^+, c_i^-, d)| \Big)\\
&\leq C \Big( p_5(\lambda)|c_i^+| + (p_1(X_i; \lambda) + |G|)|a_i| + |G||a_{i-1}| + (e^{-\alpha X_i} + |G|) |b| \\
&+ ( p_2(X_i; \lambda) + e^{\nu(\lambda)X_i} |G|) |c_i^+| + e^{\nu(\lambda)X_{i-1}} |G|) |c_{i-1}^+| \\
&+ |c_i^-| + e^{\nu(\lambda)X_{i-1}} |G||c_{i-1}^-| + (e^{-\tilde{\alpha} X_i} + |G|) |\lambda|^2 |d| \Big) \\
&\leq C \Big( (p_5(\lambda) + p_1(X_i; \lambda) + |G| + p_2(X_i; \lambda) + e^{\nu(\lambda)X_i} |G|) |A_1(\lambda)_i(b, c^-, d)|  \\
&+ (|G| + e^{\nu(\lambda)X_{i-1}} |G|)|A_1(\lambda)_{i-1}(b, c^-, d)|  \\
&+ (e^{-\alpha X_i} + |G|) |b| + |c_i^-| + e^{\nu(\lambda)X_{i-1}} |G||c_{i-1}^-| 
+ (e^{-\tilde{\alpha} X_i} + |G|) |\lambda|^2 |d| \Big) \\
&\leq C \Big( (p_5(\lambda) + p_1(X_i; \lambda) + p_2(X_i; \lambda) + e^{\nu(\lambda)X_i} |G|) |A_1(\lambda)_i(b, c^-, d)|  \\
&+ e^{\nu(\lambda)X_{i-1}} |G||A_1(\lambda)_{i-1}(b, c^-, d)|  \\
&+ (e^{-\alpha X_i} + |G|) |b| + |c_i^-| + e^{\nu(\lambda)X_{i-1}} |G||c_{i-1}^-| 
+ (e^{-\tilde{\alpha} X_i} + |G|) |\lambda|^2 |d| \Big)
\end{align*} 

This is a total mess, but should hopefully simplify nicely. For convenience, let

\begin{align*}
k_i &= p_5(\lambda) + p_1(X_i; \lambda) + p_2(X_i; \lambda) + e^{\nu(\lambda)X_i} |G|\\
k_{i-1} &= e^{\nu(\lambda)X_{i-1}} |G|
\end{align*}

Both of these are small (certainly less than 1). This becomes

\begin{align*}
|A_2&(\lambda)_i(b, c_i^-, d)| \\
&\leq C \Big( k_i |A_1(\lambda)_i(b, c^-, d)| + k_{i-1} |A_1(\lambda)_{i-1}(b, c^-, d)|  \\
&+ (e^{-\alpha X_i} + |G|) |b| + |c_i^-| + e^{\nu(\lambda)X_{i-1}} |G||c_{i-1}^-| 
+ (e^{-\tilde{\alpha} X_i} + |G|) |\lambda|^2 |d| \Big)
\end{align*} 

Substituting in our estimates for $|A_1(\lambda)_i(b, c^-, d)|$ and $|A_1(\lambda)_{i-1}(b, c^-, d)|$, this becomes

\begin{align*}
|A_2&(\lambda)_i(b, c_i^-, d)| \\
&\leq C \Big( k_i ( (e^{-\alpha X_i} + |G|) |b| 
+ |c_i^-| + e^{\nu(\lambda)X_{i-1}} |G||c_{i-1}^-| + (e^{-\tilde{\alpha} X_i} + |G|) |\lambda|^2 |d| + |D_i||d| ) \\
&+ k_{i-1} ( (e^{-\alpha X_{i-1}} + |G|) |b| 
+ |c_{i-1}^-| + e^{\nu(\lambda)X_i} |G||c_i^-| + (e^{-\tilde{\alpha} X_{i-1}} + |G|) |\lambda|^2 |d| + |D_{i-1}||d| )  \\
&+ (e^{-\alpha X_i} + |G|) |b| + |c_i^-| + e^{\nu(\lambda)X_{i-1}} |G||c_{i-1}^-| 
+ (e^{-\tilde{\alpha} X_i} + |G|) |\lambda|^2 |d| \Big)\\
&\leq C \Big( ( k_i (e^{-\alpha X_i} + |G|) + k_{i-1} (e^{-\alpha X_{i-1}} + |G|) + e^{-\alpha X_i} + |G| ) |b|\\
&+ (1 + k_i + k_{i-1} e^{\nu(\lambda)X_i}|G|)|c_i^-| \\
&+ (k_{i-1} + e^{\nu(\lambda)X_{i-1}} |G| + k_i e^{\nu(\lambda)X_{i-1}} |G|)|c_{i-1}^-| \\
&+ ( k_i (e^{-\tilde{\alpha} X_i} + |G|) + k_{i-1} (e^{-\tilde{\alpha}X_{i-1}} + |G|) + e^{-\alpha X_i} + |G| ) |\lambda^2| \\
&+ k_i |D_i| + k_{i-1} |D_{i-1}||d| \Big) \\
&\leq C \Big( ( e^{-\alpha X_i} + |G| ) |b| + |c_i^-| + e^{\nu(\lambda)X_{i-1}} |G| |c_{i-1}^-| \\
&+ (e^{-\alpha X_i} + |G| ) |\lambda^2| + k_i |D_i||d| + k_{i-1} |D_{i-1}||d| \Big) \\
&\leq C \Big( ( e^{-\alpha X_i} + |G| ) |b| + |c_i^-| + e^{\nu(\lambda)X_{i-1}} |G| |c_{i-1}^-| + (e^{-\alpha X_i} + |G| ) |\lambda^2| \\
&+ (p_5(\lambda) + p_1(X_i; \lambda) + p_2(X_i; \lambda) + e^{\nu(\lambda)X_i} |G|)|D_i| |d| \\
&+ e^{\nu(\lambda)X_{i-1}} |G| |D_{i-1}||d| \Big)
\end{align*}

Thus the final bound is

\begin{align*}
|A_2&(\lambda)_i(b, c_i^-, d)| \\
&\leq C \Big( ( e^{-\alpha X_i} + |G| ) |b| + |c_i^-| + e^{\nu(\lambda)X_{i-1}} |G| |c_{i-1}^-| + (e^{-\alpha X_i} + |G| ) |\lambda^2| \\
&+ (p_5(\lambda) + p_1(X_i; \lambda) + p_2(X_i; \lambda) + e^{\nu(\lambda)X_i} |G|)|D_i| |d| \\
&+ e^{\nu(\lambda)X_{i-1}} |G| |D_{i-1}||d| \Big)
\end{align*}

The issue here (I think), is that we have the $|c_i^-|$ term by itself, rather than multiplied by a small or (better yet) a decaying factor.\\

It looks like we are missing an opportunity here, however, by not eliminating a large component of the $c^-$ term with the projection. If we do a similar trick with $c^-$ that we did with $c^+$, we have

\begin{align*}
D_i d &= a_i^+ - a_i^- + c_i^+ v_0(\lambda) - c_i^- v_0(\lambda) \\
&+ (P^u_+(X_i; \lambda) - P_0^u)a_i^+ - (P^s_-(-X_i; \lambda) - P_0^s)a_i^- \\
&+ \Phi^s_+(X_i, 0; \lambda)b_i^+ - \Phi^u_-(-X_i, 0; \lambda)b_{i-1}^- \\
&+ c_i^+ \Delta v_+(X_i; \lambda) + c_i^+ v_+(X_i; \lambda) \langle v_0(\lambda), \Delta w_+(X_i; \lambda) \rangle \\
&- c_i^- \Delta v_-(X_i; \lambda) + c_i^- v_-(-X_i; \lambda) \langle v_0(\lambda), \Delta w_-(X_i; \lambda) \rangle \\
&+ \int_0^{-X_i} \Phi^u_+(-X_i, y; \lambda) [ G(\lambda)W_i^+(y) + d_i \lambda^2 \tilde{H}(y) ] dy \\
&- \int_0^{X_i} \Phi^s_-(X_i, y; \lambda) [ G(\lambda)W_{i-1}^-(y) + d_{i-1} \lambda^2 \tilde{H}(y) ] dy
\end{align*}

Adding and subtracting $v_0(0)$, this becomes

\begin{align*}
D_i d &= a_i^+ - a_i^- + c_i^+ v_0(0) - c_i^- v_0(0) \\
&+ (P^u_+(X_i; \lambda) - P_0^u)a_i^+ - (P^s_-(-X_i; \lambda) - P_0^s)a_i^- \\
&+ \Phi^s_+(X_i, 0; \lambda)b_i^+ - \Phi^u_-(-X_i, 0; \lambda)b_{i-1}^- \\
&+ c_i^+(v_0(\lambda) - v_0(0)) + c_i^+ \Delta v_+(X_i; \lambda) + c_i^+ v_+(X_i; \lambda) \langle v_0(\lambda), \Delta w_+(X_i; \lambda) \rangle \\
&- c_i^-(v_0(\lambda) - v_0(0)) - c_i^- \Delta v_-(X_i; \lambda) + c_i^- v_-(-X_i; \lambda) \langle v_0(\lambda), \Delta w_-(X_i; \lambda) \rangle \\
&+ \int_0^{-X_i} \Phi^u_+(-X_i, y; \lambda) [ G(\lambda)W_i^+(y) + d_i \lambda^2 \tilde{H}(y) ] dy \\
&- \int_0^{X_i} \Phi^s_-(X_i, y; \lambda) [ G(\lambda)W_{i-1}^-(y) + d_{i-1} \lambda^2 \tilde{H}(y) ] dy
\end{align*}

Hitting this with the projections $P^{(s/u)}_0$, this becomes

\begin{align*}
a_i^+ &= P^u_0 D_i d - c_i^+ P^u_0 (v_0(\lambda) - v_0(0)) + c_i^- P^u_0 (v_0(\lambda) - v_0(0)) - P^u_0 \tilde{L}_3(\lambda)_i(a_i, b, c_i^+, c_i^-, d) \\
a_i^- &= -P^s_0 D_i d + c_i^+ P^s_0 (v_0(\lambda) + v_0(0)) - c_i^- P^s_0 (v_0(\lambda) + v_0(0))+ P^s_0 \tilde{L}_3(\lambda)_i(a_i, b, c_i^+, c_i^-, d)
\end{align*}

where $\tilde{L}_3(\lambda)$ is the remaining stuff on the RHS. Once again, $A_2$ is all the stuff on the RHS other than the $D_i d$ term. 

\begin{align*}
a_i^+ &= P^u_0 D_i d + A_2(\lambda)_i^+(b, c_i^-, d) \\
a_i^- &= -P^s_0 D_i d + A_2(\lambda)_i^-(b, c_i^-, d)
\end{align*}

The $\tilde{L}_3(\lambda)$ will be similar to the $L_3(\lambda)$ bound, except now the $c_i^+$ and $c_i^-$ terms will look the same.

\begin{align*}
|\tilde{L}_3&(\lambda)_i(a, b, c^+, c^-, d)| \\
&\leq C \Big( (p_1(X_i; \lambda) + |G|)|a_i| + |G||a_{i-1}| + (e^{-\alpha X_i} + |G|) |b| \\
&+ ( p_2(X_i; \lambda) + e^{\nu(\lambda)X_i} |G|) |c_i^+| + e^{\nu(\lambda)X_{i-1}} |G| |c_{i-1}^+|   \\
&+ (p_2(X_i; \lambda) + e^{\nu(\lambda)X_i} |G|)|c_i^-| + e^{\nu(\lambda)X_{i-1}} |G||c_{i-1}^-| + (e^{-\tilde{\alpha} X_i} + |G|) |\lambda|^2 |d| \Big)
\end{align*} 

We will also get an additional $p_5(\lambda) |c_i^-|$ from the $|c_i^- P^u_0 (v_0(\lambda) - v_0(0))|$ term. All together, we will have an improved bound of the form

\begin{align*}
|A_2&(\lambda)_i(b, c_i^-, d)| \\
&\leq C \Big( ( e^{-\alpha X_i} + |G| ) |b| + (p_5(\lambda) + p_2(X_i; \lambda) + e^{\nu(\lambda)X_i} |G|)|c_i^-| + e^{\nu(\lambda)X_{i-1}} |G| |c_{i-1}^-| \\
&+ (e^{-\alpha X_i} + |G| ) |\lambda^2| \\
&+ (p_5(\lambda) + p_1(X_i; \lambda) + p_2(X_i; \lambda) + e^{\nu(\lambda)X_i} |G|)|D_i| |d| \\
&+ e^{\nu(\lambda)X_{i-1}} |G| |D_{i-1}||d| \Big)
\end{align*}

\item Next we use the projections 

\begin{align*}
P(\C Q'(0))W_i^-(0) &= 0 \\
P(\C Q'(0))W_i^+(0) &= 0 \\
P(Y^+ \oplus Y^- \oplus Y^0) ( W_i^+(0) - W_i^-(0) ) &= 0
\end{align*}

to implement what happens at $x = 0$ (on both pieces). Note that the $X_1$ and $X_2$ terms will mix here. At $x = 0$, the fixed point equations become

\begin{align*}
W_i^-(0) = \Phi^s_-(&0, -X_{i-1}; \lambda)a_{i-1}^- + b_i^- + (P^u_-(0; \lambda) - P^u_-(0; 0))b_i^- \\
&+ e^{\nu(\lambda)X_{i-1}} v_-(0; \lambda) \langle v_0(\lambda), w_-(-X_{i-1}; \lambda) \rangle c_{i-1}^- \\
&+ \int_{-X_{i-1}}^0 \Phi^s_-(0, y; \lambda) [ G_i^-(\lambda)W_i^-(y) + \lambda^2 d_i \tilde{H}(y) ] dy \\
&+ \int_{-X_{i-1}}^0
e^{\nu(\lambda)y} v_-(0; \lambda) \langle G_i^-(\lambda)(y)W_i^-(y) + \lambda^2 d_i \tilde{H}(y), w_-(y; \lambda) \rangle dy \\
W_i^+(0) = \Phi^u_+(&0, X_i; \lambda)a_i^+ + b_i^+ + (P^s_+(0; \lambda) - P^s_-(0; 0))b_i^+ \\
&+ e^{-\nu(\lambda) X_i} v_+(0; \lambda) \langle v_0(\lambda), w_+(X_i; \lambda) \rangle c_i^+ \\
&+ \int_{X_i}^0 \Phi^u_+(0, y; \lambda) [ G_i^+(\lambda)W_i^+(y) + \lambda^2 d_i \tilde{H}(y) ] dy \\
&+ \int_{X_i}^0 e^{-\nu(\lambda)y} v_+(0; \lambda) \langle G_i^+(\lambda)(y)W_i^+(y) + \lambda^2 d_i \tilde{H}(y), w_+(y; \lambda) \rangle dy
\end{align*}

where we have added and subtracted $P^s_-(0; 0))b_i^+$ and $P^u_-(0; 0))b_i^-$ since we want the $b_i$ to disappear when we take the projection. Recall that we always take the projection in the $\lambda = 0$ setting. For this, we will have a bound

\[
p_3(\lambda) = |P^u_-(0;\lambda) - P^u_-(0; 0)| + |P^s_+(0;\lambda) - P^s_+(0;0)|
\]

which should be order $|\lambda|$.\\

Using the same setup as in the single pulse case and doing the same projections, we would like to get something like

\[
\begin{pmatrix}x^- \\ x_i^+ \\ y_i^+ - y_i^- + \text{\{some stuff involving $c_i^-$\} $y_0$} \end{pmatrix}+ L_4(\lambda)(b, c_i^-,d) = 0
\]

Let's see what we can get. First, let's look at the 3rd component of this. The bound on $L_4$ will be determined by this, since the same bound will hold for the first two components. Before we do that, we need to do something with the $c_i^-$ terms since we would like place the largest component of those terms in the space $Y_0$ (since we are projecting onto that space in the third projection and that space is in the kernel fo the first two projections). Looking at the $c_{i-1}^-$ term, we have

\begin{align*}
c_{i-1}^- &e^{\nu(\lambda)X_{i-1}} v_-(0; \lambda) \langle v_0(\lambda), w_-(-X_{i-1}; \lambda) \rangle \\
&= c_{i-1}^- e^{\nu(\lambda)X_{i-1}} v_-(0; \lambda) \langle v_0(\lambda), w_0(\lambda) \rangle + c_{i-1}^- e^{\nu(\lambda)X_{i-1}} v_-(0; \lambda) \langle v_0(\lambda), \Delta w_-(X_i; \lambda) \rangle \\
&= c_{i-1}^- e^{\nu(\lambda)X_{i-1}} v_-(0; \lambda) + c_{i-1}^- e^{\nu(\lambda)X_{i-1}} v_-(0; \lambda) \langle v_0(\lambda), \Delta w_-(X_i; \lambda) \rangle \\
&= c_{i-1}^- e^{\nu(\lambda)X_{i-1}} y_0 +  c_{i-1}^- e^{\nu(\lambda)X_{i-1}} (v_-(0; \lambda) - y_0) + c_{i-1}^- e^{\nu(\lambda)X_{i-1}} v_-(0; \lambda) \langle v_0(\lambda), \Delta w_-(X_i; \lambda) \rangle \\
&= c_{i-1}^- e^{\nu(\lambda)X_{i-1}} y_0 + c_{i-1}^- e^{\nu(\lambda)X_{i-1}}( (v_-(0; \lambda) - y_0) + v_-(0; \lambda) \langle v_0(\lambda), \Delta w_-(X_i; \lambda) \rangle \\
\end{align*} 

The $c_i^+$ term is similar (should we choose to do this). For this we have a bound

\[
p_4(\lambda) = |v_-(0; \lambda) - y_0| + |v_+(0; \lambda) - y_0|
\]

which should be order $|\lambda|$, again by a Taylor series argument or a geometric argument.\\

Then we have

\begin{align*}
W_i^+(0) &- W_i^-(0) = \Phi^u_+(0, X_i; \lambda)a_i^+ - \Phi^s_-(0, -X_{i-1}; \lambda)a_{i-1}^- \\
&+ b_i^+ - b_i^- + (P^s_+(0; \lambda) - P^s_-(0; 0))b_i^+  - (P^u_-(0; \lambda) - P^u_-(0; 0))b_i^- \\
&+ c_i^+ e^{-\nu(\lambda)X_i} y_0 + c_i^+ e^{-\nu(\lambda)X_i}( (v_+(0; \lambda) - y_0) + v_+(0; \lambda) \langle v_0(\lambda), \Delta w_+(X_i; \lambda) \rangle\\
&- c_{i-1}^- e^{\nu(\lambda)X_{i-1}} y_0 + c_{i-1}^- e^{\nu(\lambda)X_{i-1}}( (v_-(0; \lambda) - y_0) + v_-(0; \lambda) \langle v_0(\lambda), \Delta w_-(X_i; \lambda) \rangle \\
&+ \int_{-X_{i-1}}^0 \Phi^s_-(0, y; \lambda) [ G_i^-(\lambda)W_i^-(y) + \lambda^2 d_i \tilde{H}(y) ] dy \\
&+ \int_{-X_{i-1}}^0
e^{\nu(\lambda)y} v_-(0; \lambda) \langle G_i^-(\lambda)(y)W_i^-(y) + \lambda^2 d_i \tilde{H}(y), w_-(y; \lambda) \rangle dy \\
&+ \int_{X_i}^0 \Phi^u_+(0, y; \lambda) [ G_i^+(\lambda)W_i^+(y) + \lambda^2 d_i \tilde{H}(y) ] dy \\
&+ \int_{X_i}^0 e^{-\nu(\lambda)y} v_+(0; \lambda) \langle G_i^+(\lambda)(y)W_i^+(y) + \lambda^2 d_i \tilde{H}(y), w_+(y; \lambda) \rangle dy
\end{align*}

Our matrix-equation becomes

\[
\begin{pmatrix}x_i^- \\ x_i^+ \\ y_i^+ - y_i^- - e^{\nu(\lambda)X_{i-1}} c_{i-1}^- y_0 \end{pmatrix} + L_4(\lambda)_i(b_i, e^{\nu(\lambda)X_{i-1}} c_{i-1}^-, d) = 0
\]

It turns out that this will not quite work, since we do not know the sign of $\nu(\lambda)$, thus we know nothing about in which direction the $e^{\nu(\lambda)X_{i-1}}$ will blow up in. We can (hopefully) get around this by adding and subtracting the ``other-sign'' version in the ``negative'' fixed point equation at $x = 0$.

\begin{align*}
W_i^-(0) = \Phi^s_-(&0, -X_{i-1}; \lambda)a_{i-1}^- + b_i^- + (P^u_-(0; \lambda) - P^u_-(0; 0))b_i^- \\
&+ (e^{\nu(\lambda)X_{i-1}} + e^{-\nu(\lambda)X_{i-1}} - e^{-\nu(\lambda)X_{i-1}}) c_{i-1}^- y_0 \\
&+ e^{\nu(\lambda)X_{i-1}} c_{i-1}^- ( (v_-(0; \lambda) - y_0) + v_-(0; \lambda) \langle  v_0(\lambda), \Delta w_-(-X_{i-1}; \lambda) \rangle) \\
&+ \int_{-X_{i-1}}^0 \Phi^s_-(0, y; \lambda) [ G_i^-(\lambda)W_i^-(y) + \lambda^2 d_i \tilde{H}(y) ] dy \\
&+ \int_{-X_{i-1}}^0
e^{\nu(\lambda)y} v_-(0; \lambda) \langle G_i^-(\lambda)(y)W_i^-(y) + \lambda^2 d_i \tilde{H}(y), w_-(y; \lambda) \rangle dy
\end{align*}

Making this change, we have the revised matrix equation

\[
\begin{pmatrix}x_i^- \\ x_i^+ \\ y_i^+ - y_i^- - (e^{\nu(\lambda)X_{i-1}} + e^{-\nu(\lambda)X_{i-1}})c_{i-1}^- y_0 \end{pmatrix} + L_4(\lambda)_i(b_i, e^{\nu(\lambda)X_{i-1}} c_{i-1}^-, d) = 0
\]

Note that the RHS does not match the LHS, but we will get there. For convenience, let

\begin{align*}
f(X; \lambda) = e^{\nu(\lambda)X} + e^{-\nu(\lambda)X} &= 2 \cosh (\nu(\lambda) X) \\
f^+(X; \lambda) &= e^{\nu(\lambda)X} / f(X; \lambda) \\
f^-(X; \lambda) &= e^{-\nu(\lambda)X} / f(X; \lambda)
\end{align*}

This is nice, since $|f(X; \lambda)| \geq 1|$ and $0 < |f^\pm(X; \lambda)| < 1$. So we would like to have

\[
\begin{pmatrix}x_i^- \\ x_i^+ \\ y_i^+ - y_i^- - f(X_{i-1}; \lambda)c_{i-1}^- y_0 \end{pmatrix} + L_4(\lambda)_i(b_i, f(X_{i-1}; \lambda) c_{i-1}^-, d) = 0
\]

where we have written RHS to match the LHS. We need to show that is possible, which is what we do now. Once again, we look at $W_i^+(0) - W_i^-(0)$. This time we leave the $c_i^+$ term as $c_i^+ e^{-\nu(\lambda)X_i} v_+(0; \lambda) \langle v_0(\lambda), w_+(X_i; \lambda) \rangle$ (i.e. don't do what we did for the $c_i^-$ term).

\begin{align*}
W_i^+(0) - W_i^-(0) &= b_i^+ - b_i^- - f(X_{i-1}; \lambda) c_{i-1}^- y_0 \\
&+ \Phi^u_+(0, X_i; \lambda)a_i^+ - \Phi^s_-(0, -X_{i-1}; \lambda)a_{i-1}^- \\
&+(P^s_+(0; \lambda) - P^s_-(0; 0))b_i^+  - (P^u_-(0; \lambda) - P^u_-(0; 0))b_i^- \\
&+ c_i^+ e^{-\nu(\lambda)X_i} v_+(0; \lambda) \langle v_0(\lambda), w_+(X_i; \lambda) \rangle \\
&+ e^{-\nu(\lambda)X_{i-1}} c_{i-1}^- y_0 - e^{\nu(\lambda)X_{i-1}} c_{i-1}^- ( (v_-(0; \lambda) - y_0) + v_-(0; \lambda) \langle  v_0(\lambda), \Delta w_-(-X_{i-1}; \lambda) \rangle) \\
&+ \int_{-X_{i-1}}^0 \Phi^s_-(0, y; \lambda) [ G_i^-(\lambda)W_i^-(y) + \lambda^2 d_i \tilde{H}(y) ] dy \\
&+ \int_{-X_{i-1}}^0
e^{\nu(\lambda)(y)} v_-(0; \lambda) \langle G_i^-(\lambda)(y)W_i^-(y) + \lambda^2 d_i \tilde{H}(y), w_-(y; \lambda) \rangle dy \\
&+ \int_{X_i}^0 \Phi^u_+(0, y; \lambda) [ G_i^+(\lambda)W_i^+(y) + \lambda^2 d_i \tilde{H}(y) ] dy \\
&+ \int_{X_i}^0 e^{\nu(\lambda)(-y)} v_+(0; \lambda) \langle G_i^+(\lambda)(y)W_i^+(y) + \lambda^2 d_i \tilde{H}(y), w_+(y; \lambda) \rangle dy
\end{align*}

Upon substituting the various $f$ things, this becomes
 
\begin{align*}
W_i^+(0) - W_i^-(0) &= b_i^+ - b_i^- - f(X_{i-1}; \lambda) c_{i-1}^- y_0 \\
&+ \Phi^u_+(0, X_i; \lambda)a_i^+ - \Phi^s_-(0, -X_{i-1}; \lambda)a_{i-1}^- \\
&+(P^s_+(0; \lambda) - P^s_-(0; 0))b_i^+  - (P^u_-(0; \lambda) - P^u_-(0; 0))b_i^- \\
&+ c_i^+ e^{-\nu(\lambda)X_i} v_+(0; \lambda) \langle v_0(\lambda), w_+(X_i; \lambda) \rangle \\
&+ f^-(X_{i-1}; \lambda)f(X_{i-1}; \lambda) c_{i-1}^- y_0 \\
&- f^+(X_{i-1}; \lambda)f(X_{i-1}; \lambda) c_{i-1}^- ( (v_-(0; \lambda) - y_0) + v_-(0; \lambda) \langle  v_0(\lambda), \Delta w_-(-X_{i-1}; \lambda) \rangle) \\
&+ \int_{-X_{i-1}}^0 \Phi^s_-(0, y; \lambda) [ G_i^-(\lambda)W_i^-(y) + \lambda^2 d_i \tilde{H}(y) ] dy \\
&+ \int_{-X_{i-1}}^0
e^{\nu(\lambda)(y)} v_-(0; \lambda) \langle G_i^-(\lambda)(y)W_i^-(y) + \lambda^2 d_i \tilde{H}(y), w_-(y; \lambda) \rangle dy \\
&+ \int_{X_i}^0 \Phi^u_+(0, y; \lambda) [ G_i^+(\lambda)W_i^+(y) + \lambda^2 d_i \tilde{H}(y) ] dy \\
&+ \int_{X_i}^0 e^{-\nu(\lambda)y} v_+(0; \lambda) \langle G_i^+(\lambda)(y)W_i^+(y) + \lambda^2 d_i \tilde{H}(y), w_+(y; \lambda) \rangle dy
\end{align*}

We have the following bound for $L_4$. Note that the bound will depend on all the things in $W_i^+(0) - W_i^-(0)$ other than the first line, since the terms in the first line are not involved in $L_4$. As before, we use the $\tilde{\alpha}$ trick for the integrals involving $\tilde{H}$ to absorb the $e^{\nu(\lambda)y}$ so that we don't have an exponential growth factor in front of the $|\lambda|^2$ term.

\begin{align*}
|L_4&(\lambda)_i(b_i, f(X_{i-1}; \lambda) c_{i-1}^-, d)|\\ 
&\leq C \Big( e^{-\alpha X_i} |a_i^+| +  e^{-\alpha X_{i-1}} |a_{i-1}^-| \\
&+ p_3(\lambda) |b_i| + (f^-(X_{i-1}; \lambda) + f^+(X_{i-1}; \lambda) (p_2(\lambda; X_{i-1}) + p_4(\lambda)) | f(X_{i-1}; \lambda) c_{i-1}^-| \\
&+ e^{\nu(\lambda)X_{i}} |c_i^+| \\
&+ (e^{\nu(\lambda)X_i} + e^{\nu(\lambda)X_{i-1}}) |G| ||W|| + |\lambda^2| |d| \Big)
\end{align*}

To finish the bound, we need to plug in $A_1$ and $W_2$ for $|a|$ and $||W||$. Note that because of the ``mixing'' of coefficents ($a_i$ and $c_i$), we need to plug in both $A_1(\lambda)_i$ and $A_1(\lambda)_{i-1}$. Also note that when the ``mixing'' is done, this bound will depend on both $f(X_{i-1}; \lambda) c_{i-1}^-$ and $f(X_i; \lambda) c_i^-$. Plugging in $A_1$ and $W_2$, we have the bound

\begin{align*}
|L_4&(\lambda)_i(b, f(X_i; \lambda) c_i^-, f(X_{i-1}; \lambda) c_{i-1}^-, d)|\\ 
&\leq C\Big( e^{-\alpha X_i} |a_i^+| \\
&+  e^{-\alpha X_{i-1}} |a_{i-1}^-| \\
&+ e^{\nu(\lambda)X_{i}} |c_i^+| \\
&+ (e^{\nu(\lambda)X_i} + e^{\nu(\lambda)X_{i-1}}) |G| ||W|| \\
&+ p_3(\lambda) |b_i| + (f^-(X_{i-1}; \lambda) + f^+(X_{i-1}; \lambda) (p_2(\lambda; X_{i-1}) + p_4(\lambda)) | f(X_{i-1}; \lambda) c_{i-1}^-| \\
&+|\lambda^2| |d| \Big) \\
&\leq C\Big( e^{-\alpha X_i} |A_1(\lambda)_i(b, c^-, d)| \\
&+  e^{-\alpha X_{i-1}} |A_1(\lambda)_{i-1}(b, c^-, d)| \\
&+ e^{\nu(\lambda)X_{i}} |A_1(\lambda)_i(b, c^-, d)|  \\
&+ (e^{\nu(\lambda)X_i} + e^{\nu(\lambda)X_{i-1}}) |G| ||W_2(\lambda)(b,c^-,d)|| \\
&+ p_3(\lambda) |b_i| + (f^-(X_{i-1}; \lambda) + f^+(X_{i-1}; \lambda) (p_2(\lambda; X_{i-1}) + p_4(\lambda)) | f(X_{i-1}; \lambda) c_{i-1}^-| \\
&+|\lambda^2| |d| \Big)
\end{align*}

Note that for the $|A_1(\lambda)_i(b, c^-, d)|$ coefficient, $e^{\nu(\lambda)X_{i}}$ is much lower order than $e^{-\alpha X_i}$, so we can drop the higher order set of terms. This time we plug in the actual bounds for $A_1$ and $W_2$ to get

\begin{align*}
|L_4&(\lambda)_i(b, f(X_i; \lambda) c_i^-, f(X_{i-1}; \lambda) c_{i-1}^-, d)|\\ 
&\leq C\Big( e^{\nu(\lambda)X_{i}} |A_1(\lambda)_i(b, c^-, d)| \\
&+  e^{-\alpha X_{i-1}} |A_1(\lambda)_{i-1}(b, c^-, d)| \\
&+ (e^{\nu(\lambda)X_i} + e^{\nu(\lambda)X_{i-1}}) |G| ||W_2(\lambda)(b,c^-,d)|| \\
&+ p_3(\lambda) |b_i| + (f^-(X_{i-1}; \lambda) + f^+(X_{i-1}; \lambda) (p_2(\lambda; X_{i-1}) + p_4(\lambda)) | f(X_{i-1}; \lambda) c_{i-1}^-| \\
&+|\lambda^2| |d|) \\
&\leq C\Big( e^{\nu(\lambda)X_{i}} ((e^{-\alpha X_i} + |G|) |b| 
+ |c_i^-| + e^{\nu(\lambda)X_{i-1}} |G||c_{i-1}^-| + (e^{-\tilde{\alpha} X_i} + |G|) |\lambda|^2 |d| + |D_i||d|)\\
&+  e^{-\alpha X_{i-1}} ((e^{-\alpha X_{i-1}} + |G|) |b| 
+ |c_{i-1}^-| + e^{\nu(\lambda)X_i} |G||c_i^-| + (e^{-\tilde{\alpha} X_{i-1}} + |G|) |\lambda|^2 |d| + |D_{i-1}||d|) \\
&+ (e^{\nu(\lambda)X_i} + e^{\nu(\lambda)X_{i-1}}) |G| ( |b| + e^{\nu(\lambda) X_i} |c_i^-| +  e^{\nu(\lambda) X_{i-1}} |c_{i-1}^-| + |\lambda|^2 |d| \\
&\:\:\:\:\:\:\:+ e^{\nu(\lambda) X_i} |D_i| |d| + e^{\nu(\lambda) X_{i-1}} |D_{i-1}| |d| ) \\
&+ p_3(\lambda) |b_i| + (f^-(X_{i-1}; \lambda) + f^+(X_{i-1}; \lambda) (p_2(\lambda; X_{i-1}) + p_4(\lambda)) | f(X_{i-1}; \lambda) c_{i-1}^-| \\
&+|\lambda^2| |d|)
\end{align*}

Now we collect like terms to get

\begin{align*}
|L_4&(\lambda)_i(b, f(X_i; \lambda) c_i^-, f(X_{i-1}; \lambda) c_{i-1}^-, d)|\\ 
&\leq C\Big( e^{\nu(\lambda)X_{i}} ((e^{-\alpha X_i} + |G|) |b| 
+ |c_i^-| + e^{\nu(\lambda)X_{i-1}} |G||c_{i-1}^-| + (e^{-\tilde{\alpha} X_i} + |G|) |\lambda|^2 |d| + |D_i||d|)\\
&+  e^{-\alpha X_{i-1}} ((e^{-\alpha X_{i-1}} + |G|) |b| 
+ |c_{i-1}^-| + e^{\nu(\lambda)X_i} |G||c_i^-| + (e^{-\tilde{\alpha} X_{i-1}} + |G|) |\lambda|^2 |d| + |D_{i-1}||d|) \\
&+ (e^{\nu(\lambda)X_i} + e^{\nu(\lambda)X_{i-1}}) |G| ( |b| + e^{\nu(\lambda) X_1} |c_1^-| +  e^{\nu(\lambda) k X_1} |c_2^-| + |\lambda|^2 |d| \\
&\:\:\:\:\:\:\:+ e^{\nu(\lambda) X_1} |D_1| |d| + e^{\nu(\lambda) k X_1} |D_2| |d| ) \\
&+ p_3(\lambda) |b_i| + (f^-(X_{i-1}; \lambda) + f^+(X_{i-1}; \lambda) (p_2(\lambda; X_{i-1}) + p_4(\lambda)) | f(X_{i-1}; \lambda) c_{i-1}^-| \\
&+|\lambda^2| |d|) \\
&\leq C\Big( 
(e^{\nu(\lambda)X_{i}} (e^{-\alpha X_i} + |G|) + e^{-\alpha X_{i-1}} (e^{-\alpha X_{i-1}} + |G|) + (e^{\nu(\lambda)X_i} + e^{\nu(\lambda)X_{i-1}}) |G| + p_3(\lambda)) |b| \\
&+ (e^{\nu(\lambda)X_i} + e^{-\alpha X_{i-1}} e^{\nu(\lambda)X_i} |G|
+ e^{\nu(\lambda)X_i} (e^{\nu(\lambda)X_i} + e^{\nu(\lambda)X_{i-1}}) |G|)|c_i^-| \\
&+ (e^{\nu(\lambda)X_i} e^{\nu(\lambda)X_{i-1}} |G| +  e^{-\alpha X_{i-1}} 
+ e^{\nu(\lambda)X_{i-1}} (e^{\nu(\lambda)X_i} + e^{\nu(\lambda)X_{i-1}}) |G|) |c_{i-1}^-| \\
&+ (f^-(X_{i-1}; \lambda) + f^+(X_{i-1}; \lambda) (p_2(\lambda; X_{i-1}) + p_4(\lambda)) | f(X_{i-1}; \lambda) c_{i-1}^-| \\
&+ (1 + e^{\nu(\lambda)X_i}(e^{-\tilde{\alpha} X_i} + |G|) 
+ e^{-\alpha X_{i-1}} (e^{-\tilde{\alpha} X_{i-1}} + |G|) 
(e^{\nu(\lambda)X_i} + e^{\nu(\lambda)X_{i-1}}) |G|) )|\lambda|^2 |d| \\ 
&+ e^{\nu(\lambda)X_i}(1 + (e^{\nu(\lambda)X_i} + e^{\nu(\lambda)X_{i-1}}) |G|) |D_i||d| \\
&+ (e^{-\alpha X_{i-1}} + e^{\nu(\lambda)X_{i-1}} (e^{\nu(\lambda)X_i} + e^{\nu(\lambda)X_{i-1}}) |G|) |D_{i-1}||d| \Big)
\end{align*}

At this point we will simplify this by dropping some higher order terms.

\begin{align*}
|L_4&(\lambda)_i(b, f(X_i; \lambda) c_i^-, f(X_{i-1}; \lambda) c_{i-1}^-, d)|\\ 
&\leq C\Big( 
(e^{\nu(\lambda)X_i} |G| + e^{\nu(\lambda)X_{i-1}} |G| + p_3(\lambda)) |b| \\
&+ e^{\nu(\lambda)X_i} |c_i^-| \\
&+ e^{\nu(\lambda)X_{i-1}} (e^{\nu(\lambda)X_i} + e^{\nu(\lambda)X_{i-1}}) |G| |c_{i-1}^-| \\
&+ (f^-(X_{i-1}; \lambda) + f^+(X_{i-1}; \lambda) (p_2(\lambda; X_{i-1}) + p_4(\lambda)) | f(X_{i-1}; \lambda) c_{i-1}^-| \\
&+ |\lambda|^2 |d| \\ 
&+ e^{\nu(\lambda)X_i} |D_i||d| \\
&+ e^{\nu(\lambda)X_{i-1}} (e^{\nu(\lambda)X_i} + e^{\nu(\lambda)X_{i-1}}) |G| |D_{i-1}||d| \Big)
\end{align*}

Finally, we need to put the RHS in terms of the ``$f$-stuff''.

\begin{align*}
|L_4&(\lambda)_i(b, f(X_i; \lambda) c_i^-, f(X_{i-1}; \lambda) c_{i-1}^-, d)|\\ 
&\leq C\Big( 
(e^{\nu(\lambda)X_i} |G| + e^{\nu(\lambda)X_{i-1}} |G| + p_3(\lambda)) |b| \\
&+ f^+(X_i; \lambda) |f(X_i; \lambda) c_i^-| \\
&+ (f^-(X_{i-1}; \lambda) + f^+(X_{i-1}; \lambda)(e^{\nu(\lambda)X_i} + e^{\nu(\lambda)X_{i-1}}) |G|) |f(X_{i-1}; \lambda)c_{i-1}^-| \\
&+ f^+(X_{i-1}; \lambda) (p_2(\lambda; X_{i-1}) + p_4(\lambda)) | f(X_{i-1}; \lambda) c_{i-1}^-| \\
&+ |\lambda|^2 |d| \\ 
&+ e^{\nu(\lambda)X_i} |D_i||d| \\
&+ e^{\nu(\lambda)X_{i-1}} (e^{\nu(\lambda)X_i} + e^{\nu(\lambda)X_{i-1}}) |G| |D_{i-1}||d| \Big)\\
&\leq C\Big( 
(e^{\nu(\lambda)X_i} |G| + e^{\nu(\lambda)X_{i-1}} |G| + p_3(\lambda)) |b| \\
&+ f^+(X_i; \lambda) |f(X_i; \lambda) c_i^-| \\
&+ f^-(X_{i-1}; \lambda)|f(X_{i-1}; \lambda)c_{i-1}^-| \\
&+ f^+(X_{i-1}; \lambda) (p_2(\lambda; X_{i-1}) + p_4(\lambda) + (e^{\nu(\lambda)X_i} + e^{\nu(\lambda)X_{i-1}}) |G|) | f(X_{i-1}; \lambda) c_{i-1}^-| \\
&+ |\lambda|^2 |d| \\ 
&+ e^{\nu(\lambda)X_i} |D_i||d| \\
&+ e^{\nu(\lambda)X_{i-1}} (e^{\nu(\lambda)X_i} + e^{\nu(\lambda)X_{i-1}}) |G| |D_{i-1}||d| \Big)
\end{align*}

Simplifying this, we get 

\begin{align*}
|L_4&(\lambda)_i(b, f(X_i; \lambda) c_i^-, f(X_{i-1}; \lambda) c_{i-1}^-, d)|\\ 
&\leq C\Big( 
(e^{\nu(\lambda)X_i} |G| + e^{\nu(\lambda)X_{i-1}} |G| + p_3(\lambda)) |b| \\
&+ f^+(X_i; \lambda) |f(X_i; \lambda) c_i^-| \\
&+ ( f^-(X_{i-1}; \lambda) + 
f^+(X_{i-1}; \lambda) (p_2(\lambda; X_{i-1}) + p_4(\lambda) + (e^{\nu(\lambda)X_i} + e^{\nu(\lambda)X_{i-1}}) |G|)) | f(X_{i-1}; \lambda) c_{i-1}^-| \\
&+ |\lambda|^2 |d| \\ 
&+ e^{\nu(\lambda)X_i} |D_i||d| \\
&+ e^{\nu(\lambda)X_{i-1}} (e^{\nu(\lambda)X_i} + e^{\nu(\lambda)X_{i-1}}) |G| |D_{i-1}||d| \Big)
\end{align*}

We are finally where we want to be. Yay! At this point, we will want to take $k$ sufficiently small so that $2 \delta k < \alpha$. That way, the $D_{i-1}$ term does not grow exponentially.\\

To do the inversion, we need for the coefficients of $|b|$, $|f(X_i; \lambda) c_i^-|$, and $|f(X_{i-1}; \lambda) c_{i-1}^-|$ to have magnitude less than 1. For $|b|$ we are all set, since $p_3(\lambda)$ is order $\lambda$, and we can choose $\lambda$ as small as we want; all other coefficients of $|b|$ exponentially decay in $X_i$. For $|f(X_i; \lambda) c_i^-|$ we are all set since $f^-(X_i; \lambda) < 1$. For $|f(X_{i-1}; \lambda) c_{i-1}^-|$, we note that by our definitions above for the ``$f$-stuff'',

\[
f^-(X_{i-1}; \lambda) + f^+(X_{i-1}; \lambda) = 1
\]

and both terms on the LHS are positive. Since $p_2(\lambda; X_{i-1}) + p_4(\lambda) + (e^{\nu(\lambda)X_i} + e^{\nu(\lambda)X_{i-1}}) |G|$ can be made as small as we want by choosing $\lambda$ sufficiently small and $X_i, X_{i-1}$ sufficiently large, we can obtain a coefficient of $|f(X_{i-1}; \lambda) c_{i-1}^-|$ of magnitude less than 1. Thus we can do the inversion.\\

The details of the inversion are similar to Sandstede (1998) and are omitted for now. Thus we can solve for $b$ and $f(X_i; \lambda) c_i^-$ to get $(b_i, f(X_i; \lambda) c_i^-) = B_1(\lambda)(d)$ will have a bound based on $|L_4(\lambda)_i(0, 0,0, d)|$. This bound will look like

\begin{align}
|B_1(\lambda)_i(d)| \leq C(|\lambda|^2 |d| + e^{\nu(\lambda)X_i} |D_i||d|
+ e^{\nu(\lambda)X_{i-1}} (e^{\nu(\lambda)X_i} + e^{\nu(\lambda)X_{i-1}}) |G| |D_{i-1}||d|
\end{align}

Note that $B_1(\lambda)_i$ solves for both $b$ and for $f(X_{i-1}; \lambda) c_{i-1}^-$ (not $f(X_i; \lambda) c_{i-1}^-$) and that $f(X_{i-1}; \lambda)$ is of order $e^{|\nu(\lambda)|X_{i-1}}$. When we divide by $f(X_{i-1}; \lambda)$, we will have a problem, since dividing by $f(X_{i-1}; \lambda)$ will only cancel the $e^{\nu(\lambda)X_i}$ in front of $|D_i||d|$ in the case where $X_{i+1} \geq X_i$ (i.e. for $i = 1$).\\

I HAVE STOPPED HERE FOR NOW since I think the issue has occurred by this point. Compare to (3.31) in Sandstede (1998). I would like to have something decaying in front of both of the $D_i$. Here we only have it for $D_{i-1}$ but not for $D_i$. 

\pagebreak

From here, we plug this into $A_1$, $W_2$, and $A_2$ to get new estimates which are only functions of $d$ and $\lambda$.\\

First, we plug this $A_1(\lambda)(b, c^-, d)$ to $A_3(\lambda)(d)$.

\begin{align*}
|A_1&(\lambda)_i(b, c^-, d)| \\
&\leq C \Big( (e^{-\alpha X_i} + |G|) |b| 
+ |c_i^-| + e^{\nu(\lambda)X_{i-1}} |G||c_{i-1}^-| + (e^{-\tilde{\alpha} X_i} + |G|) |\lambda|^2 |d| + |D_i||d| \Big)
\end{align*} 

\begin{align*}
|A_3&(\lambda)_1(b, c^-, d)| \leq C ((e^{-\alpha X_1} + |G|) (|\lambda^2| + e^{\nu(\lambda) X_1}|D_1| + e^{\nu(\lambda)k X_1}|D_2|)|d| \\
&+ e^{-|\nu(\lambda)| X_1}(|\lambda^2| + e^{\nu(\lambda) X_1}|D_1| + e^{2 \nu(\lambda)k X_1}|G||D_2|)|d| \\
&+ e^{-|\nu(\lambda)| k X_1} e^{\nu(\lambda)k X_1} |G| (|\lambda^2| + e^{\nu(\lambda)k X_1}|D_2| + e^{2 \nu(\lambda)k X_1}|G||D_1|)|d|\\
&+ ((e^{-\tilde{\alpha} X_1} + e^{\nu(\lambda)k X_1} |G|) |\lambda|^2 + |D_1| )|d| ) \\
&\leq C( e^{-|\nu(\lambda)|X_1} |\lambda|^2 + |D_1| + |D_2|) |d| 
\end{align*}

The coefficient in front of the $|\lambda|^2$ term will only be useful to cancel something later.\\

Similarly, 

\begin{align*}
|A_3&(\lambda)_2(b, c^-, d)| \leq C( e^{-|\nu(\lambda)| k X_1} |\lambda|^2 + |D_1| + |D_2|) |d|
\end{align*}

This is similar to the bound (3.37) in Sandstede (1998), which is good.\\

We also have

\begin{align*}
||W_3&(\lambda)(b,c^-,d)|| \leq C( (|\lambda^2| + e^{\nu(\lambda) X_1}|D_1| + e^{\nu(\lambda)k X_1}|D_2|)|d| \\
&+ e^{\nu(\lambda) X_1} e^{-|\nu(\lambda)| X_1}(|\lambda^2| + e^{\nu(\lambda) X_1}|D_1| + e^{2 \nu(\lambda)k X_1}|G||D_2|)|d| \\
&+ e^{\nu(\lambda) k X_1} e^{-|\nu(\lambda)| k X_1} e^{\nu(\lambda)k X_1} |G| (|\lambda^2| + e^{\nu(\lambda)k X_1}|D_2| + e^{2 \nu(\lambda)k X_1}|G||D_1|)|d|\\
&+ (|\lambda|^2 + e^{\nu(\lambda) X_1} |D_1| + e^{\nu(\lambda)k X_1}|D_2|) |d| )\\
&\leq C ( |\lambda|^2 + e^{\nu(\lambda) X_1} |D_1| + e^{\nu(\lambda)k X_1} |D_2| )|d|
\end{align*}

For $A_4$ we have

\begin{align*}
|A_4&(\lambda)_1(d)| \\
&\leq C( (e^{-\alpha X_1} + |G|) (|\lambda^2| + e^{\nu(\lambda) X_1}|D_1| + e^{\nu(\lambda)k X_1}|D_2|)|d| \\
&+ e^{-|\nu(\lambda)| X_1} (|\lambda^2| + e^{\nu(\lambda) X_1}|D_1| + e^{2 \nu(\lambda)k X_1}|G||D_2|)|d|\\
&+ e^{\nu(\lambda) k X_1} |G| e^{-|\nu(\lambda)| k X_1} (|\lambda^2| + e^{\nu(\lambda)k X_1}|D_2| + e^{2 \nu(\lambda)k X_1}|G||D_1|)|d|\\
&+ ((e^{-\tilde{\alpha} X_1} + |G|) |\lambda|^2 + (p_1(X_1; \lambda) + p_2(X_1; \lambda) + e^{\nu(\lambda)k X_1} |G|)|D|)|d| ) \\
&\leq C(|\lambda|^2 + |D_1| + |D_2| )|d|
\end{align*}

If we like we can get a better bound on the $D_2$ part of this, but hopefully we don't need it. The other bound is then the same.

\begin{align*}
|A_4&(\lambda)_1(d)| \leq C(|\lambda|^2 + |D_1| + |D_2| )|d|
\end{align*}

\item Now we can estimate the jumps

\[
\langle \Psi(0), W_i^+(0) - W_{i-1}^-(0) \rangle 
\]

Recall here that $\Psi(0)$ is the adjoint solution for the unperturbed problem, i.e. when $\lambda = 0$, so it does not depend on $\lambda$. The equations for $W$ contain the evolution operator $\Phi^{(s/u)}_\pm(x, y; \lambda)$ which are for the perturbed system with $\lambda \neq 0$.\\

For the adjoint solution $\Psi(x)$, we have estimate 

\[
|\Psi(x)| \leq C e^{-\alpha|x|}
\]

which holds since we know exactly what $\Psi$ is in this case. Note as well that $\Psi(0)$ is just a fixed constant.\\

Thus the jumps are given by

\[
\langle \Psi(0), W_i^+(0) - W_{i-1}^-(0) \rangle 
\]

where (as above) we have

\begin{align*}
W_i^+(0) - W_{i-1}^-(0) &= b_i^+ - b_i^- \\
&+ \Phi^u_+(0, X_i; \lambda)a_i^+ - \Phi^s_-(0, -X_{i-1}; \lambda)a_{i-1}^- \\
&+(P^s_+(0; \lambda) - P^s_-(0; 0))b_i^+  - (P^u_-(0; \lambda) - P^u_-(0; 0))b_i^- \\
&+ e^{-\nu(\lambda)X_i} v_+(0; \lambda) \langle v_0(\lambda), w_+(X_i; \lambda) \rangle c_i^+ \\
&- e^{\nu(\lambda)X_{i-1}} v_-(0; \lambda) \langle v_0(\lambda), w_-(-X_{i-1}; \lambda) \rangle c_{i-1}^- \\
&+ \int_{-X_{i-1}}^0 \Phi^s_-(0, y; \lambda) [ G_i^-(\lambda)W_i^-(y) + \lambda^2 d_i \tilde{H}(y) ] dy \\
&+ \int_{-X_{i-1}}^0
e^{\nu(\lambda)y} v_-(0; \lambda) \langle G_i^-(\lambda)(y)W_i^-(y) + \lambda^2 d_i \tilde{H}(y), w_-(y; \lambda) \rangle dy \\
&+ \int_{X_i}^0 \Phi^u_+(0, y; \lambda) [ G_i^+(\lambda)W_i^+(y) + \lambda^2 d_i \tilde{H}(y) ] dy \\
&+ \int_{X_i}^0 e^{-\nu(\lambda)y} v_+(0; \lambda) \langle G_i^+(\lambda)(y)W_i^+(y) + \lambda^2 d_i \tilde{H}(y), w_+(y; \lambda) \rangle dy
\end{align*}

\item Terms involving $a$

For $a_i^\pm$ we have

\begin{align*}
a_i^+ &= -P^u_0 D_i d + A_4(\lambda)_i^+(d))\\
a_i^- &=  P^s_0 D_i d + A_4(\lambda)_i^-(d))
\end{align*}

For $i = 1$ we have

\begin{align*}
\langle \Psi(0), &\Phi^u_+(0, X_1; \lambda) a_1^+ \rangle = -\langle \Psi(0), \Phi^u_+(0, X_1; \lambda) P^u_0 D_1 d \rangle + \langle \Psi(0), \Phi^u_+(0, X_1; \lambda) A_4(\lambda)_1^+(d) \rangle \\
\end{align*} 

This is all fine and good, but the adjoint $\Psi(0)$ is unperturbed by $\lambda$, whereas the evolution $\Phi^u_+(0, X_i; \lambda)$ is perturbed by $\lambda$. Let

\[
p_6(y; \lambda) = |\Phi^s_-(0, y; \lambda) - \Phi^s_-(0, y; 0)| + |\Phi^u_-(0, -y; \lambda) - \Phi^s_-(0, -y; 0)| 
\]

We will assume for now (as in the single pulse case) that

\[
p_6(y; \lambda) \leq C |\lambda| e^{-\alpha y}
\]

Then we have

\begin{align*}
\langle \Psi(0), &\Phi^u_+(0, X_1; \lambda) a_1^+ \rangle \\
&= -\langle \Psi(0), \Phi^u_+(0, X_1; 0) P^u_0 D_1 d \rangle + \mathcal{O}( p_6(X_1; \lambda) e^{-\alpha X_1}|D_1||d| + e^{-\alpha X_1} (|\lambda|^2 + |D_1| + |D_2| )|d| \\
&= -\langle \Psi(X_1), P^u_0 D_1 d \rangle + \mathcal{O}( e^{-\alpha X_1}( |\lambda|^2 + |D_1| + |D_2|)|d|)
\end{align*}

Similarly we have

\begin{align*}
\langle \Psi(0), &\Phi^s_-(0, -k X_1); \lambda)a_2^- \rangle \\
&= \langle \Psi(0), \Phi^u_+(0, X_1; 0) P^u_0 D_1 d \rangle + \mathcal{O}( p_6(X_1; \lambda) e^{-\alpha k X_1}|D_1||d| + e^{-\alpha k X_1} (|\lambda|^2 + |D_1| + |D_2| )|d| \\
&= \langle \Psi(X_1), P^u_0 D_1 d \rangle + \mathcal{O}( e^{-\alpha k X_1}( |\lambda|^2 + |D_1| + |D_2|)|d|)
\end{align*}


\item Terms involving $b$.\\

The terms involving $b^\pm$ by themselves will die since they are in the spaces $R^u_-(0; 0) \oplus R^s_+(0; 0)$ which are perpendicular to $\Psi(0)$.\\

The other terms involving $b$ look like $(P^u_-(0; \lambda) - P^u_-(0; 0))b_i^-$ (and similar for the other one). A bound on these terms looks like

\begin{align*}
|\langle \Psi(0), (P^u_-(0; \lambda) - P^u_-(0; 0))b_i^- \rangle|
&\leq |\Psi(0)| p_3(\lambda)|b_i^-| \\
&\leq C p_3(\lambda) (|\lambda^2| + e^{\nu(\lambda)k X_1}|D_1|)|d|
\end{align*}

where in the last line we substituted $B_1(\lambda)(d)$ for $b_i^\pm$.\\

THIS IS NOT OKAY SINCE WE ARE MULTIPLYING $|D_1|$ by $e^{\nu(\lambda)k X_1}$. THIS PROBLEM COMES FROM THE BOUND $B_1$ AS WE DISCUSSED ABOVE.

\item Terms involving $c$.\\

These terms look like

\begin{align*}
&e^{\nu(\lambda)X_{i-1}} v_-(0; \lambda) \langle v_0(\lambda), w_-(-X_{i-1}; \lambda) \rangle c_{i-1}^- \\
&e^{-\nu(\lambda)X_i} v_+(0; \lambda) \langle v_0(\lambda), w_+(X_i; \lambda) \rangle c_i^+ \\
\end{align*}

Again take $i = 1$ (and remember the ``wrap-around''). Let's do the first one. Taking the inner product with $\Psi(0)$, which only hits the $v_-(0; \lambda)$ term since everything else is a scalar, gives us 

\begin{align*}
e^{\nu(\lambda)k X_1} \langle \Psi(0), v_-(0; \lambda) \rangle \langle v_0(\lambda), w_-(-k X_1; \lambda) \rangle c_2^- 
\end{align*}

We would like to do something useful with the second inner product, but we cannot at the moment since one term is a solution to the adjoint when $\lambda = 0$ and the other term is a solution to the perturbed eigenvalue problem when $\lambda \neq 0$. To do this, we will expand $v_-(0; \lambda)$ in a Taylor series about $\lambda = 0$. FOR NOW WE WILL ASSUME WE CAN ACTUALLY DO THIS.

\[
v_-(0; \lambda) = v_-(0; 0) + \lambda \frac{\partial}{\partial \lambda}v_-(0; \lambda)\Big|_{\lambda = 0} + \mathcal{O}(\lambda^2)
\]

I don't know if we can compute $\frac{\partial}{\partial \lambda}v_-(0; \lambda)\Big|_{\lambda = 0}$, but it is a constant which depends only on the initial setup of the problem, so we should not care what it is. Then we have

\begin{align*}
|&e^{\nu(\lambda)k X_1} c_2^- \langle v_0(\lambda), w_-(-k X_1; \lambda) \rangle \langle \Psi(0), v_-(0; \lambda) \rangle|\\
&\leq e^{\nu(\lambda)k X_1}|c_2^-| |\langle v_0(\lambda), w_-(-k X_1; \lambda) \rangle|\langle \Psi(0), v_-(0; 0) + \lambda \frac{\partial}{\partial \lambda}v_-(0; \lambda)\Big|_{\lambda = 0} + \mathcal{O}(\lambda^2) \rangle| \\
&\leq e^{\nu(\lambda)k X_1}|c_2^-| |\langle v_0(\lambda), w_-(-k X_1; \lambda) \rangle| \left( |\langle \Psi(0), v_-(0; 0) \rangle| +  |\langle \Psi(0), \lambda \frac{\partial}{\partial \lambda}v_-(0; \lambda)\Big|_{\lambda = 0} \rangle| + \mathcal{O}(|\lambda|^2) \right) \\
&\leq e^{\nu(\lambda)k X_1}|c_2^-| |\langle v_0(\lambda), w_-(-k X_1; \lambda) \rangle| \left( |\langle \Psi(0), \lambda \frac{\partial}{\partial \lambda}v_-(0; \lambda)\Big|_{\lambda = 0} \rangle| + \mathcal{O}(|\lambda|^2) \right)
\end{align*}

In the last line, we used the fact that $\langle \Psi(0), v_-(0; 0) \rangle = 0$ since $\langle \Psi(0), v_-(0; 0) \rangle = \langle \Psi(0), \tilde{v}_-(0; 0) \rangle$, the inner product of this is constant in $x$, and $\Psi(x)$ decays to 0 at $-\infty$ faster than any growth in $\tilde{v}_-(x; 0)$ at $-\infty$. Taking $\frac{\partial}{\partial \lambda}v_-(0; \lambda)\Big|_{\lambda = 0}$ as a constant, we have

\begin{align*}
|&e^{\nu(\lambda)k X_1} c_2^- \langle v_0(\lambda), w_-(-k X_1; \lambda) \rangle \langle \Psi(0), v_-(0; \lambda) \rangle| \\
&\leq C e^{\nu(\lambda)k X_1}|c_2^-| (\langle v_0(\lambda), w_0(\lambda) \rangle + \langle v_0(\lambda), \Delta w_-(-k X_1; \lambda) \rangle) (|\lambda| + \mathcal{O}(\lambda^2) ) \\
&\leq C e^{\nu(\lambda)k X_1}|c_2^-| (1 + p_2(k X_1; \lambda) (|\lambda| + \mathcal{O}(\lambda^2) )
\end{align*}

Since $p_2(k X_1; \lambda)$ is small for sufficiently small $\lambda$ and sufficiently large $X_1$, this becomes

\begin{align*}
|&e^{\nu(\lambda)k X_1} c_2^- \langle v_0(\lambda), w_-(-k X_1; \lambda) \rangle \langle \Psi(0), v_-(0; \lambda) \rangle| \\
&\leq C e^{\nu(\lambda)k X_1}|c_2^-| (|\lambda| + \mathcal{O}(\lambda^2) )
\end{align*}

Now we plug in our estimate for $|c_2^-|$. Recall from our discussion in the $B_1$ section that this is of order $e^{-|\nu(\lambda)|X_1} |\lambda|^2 + |D_1|) |d|$. The bound for $|c_1^+|$ is similar order, so this works for both. This means we for the $|\lambda|^2$ term that we can cancel the exponential factor out front. However, we cannot do this for the $|D_1|$ term. Thus we have

\begin{align*}
|&e^{\nu(\lambda)k X_1} c_2^- \langle v_0(\lambda), w_-(-k X_1; \lambda) \rangle \langle \Psi(0), v_-(0; \lambda) \rangle| \\
&\leq C e^{\nu(\lambda)k X_1} e^{-|\nu(\lambda)|k X_1} |\lambda|^2 + |D_1|) |d| (|\lambda| + \mathcal{O}(\lambda^2) ) \\
&\leq C |\lambda| (|\lambda|^2 + e^{\nu(\lambda)k X_1} |D_1|)|d|
\end{align*}

The other one is similar.\\

THIS BOUND IS NOT OKAY SINCE HERE WE HAVE THE EXPONENTIAL GROWTH TERM $e^{\nu(\lambda)k X_1}$ MULTIPLYING THE $|D_1|$ term, SO THIS WILL NOT GIVE US A UNIFORM BOUND IN $X_1$.

\item ``Noncenter'' integral terms

\begin{enumerate}[(i)]

\item Integrals not involving $H$. For these we only do the ``minus'' half. The ``plus'' half is similar

\begin{align*}
\left| \int_{-X_{i-1}}^0 \Phi^s_-(0, y; \lambda) G_i^-(\lambda)W_i^-(y) dy \right| 
&\leq C |G| ||W|| \\
&\leq C |G| e^{\nu(\lambda)k X_1} ( |\lambda|^2 + |D_1| )|d|
\end{align*}

We might need a better estimate here.

\item Integrals involving $H$

Recall that these integrals are multiplied by $d_i \lambda^2$. We will omit that for now when obtaining the bound. Then we have

\begin{align*}
\langle \Psi(0)&, \int_{-X_{i-1}}^0 \Phi^s_-(0, y; \lambda) \tilde{H}(y) dy \rangle \\ 
&= \int_{-X_{i-1}}^0 \langle \Psi(0), \Phi^s_-(0, y; 0) \tilde{H}(y) \rangle dy + 
\int_{-X_{i-1}}^0 \langle \Psi(0), (\Phi^s_-(0, y; \lambda) - \Phi^s_-(0, y; 0)) \tilde{H}(y) \rangle dy
\end{align*}

where we need to play this trick to replace the $\lambda$-dependent evolution with the evolution when $\lambda = 0$ in order to get the Melnikov term we want. I still have no idea how to bound the difference $|\Phi^s_-(0, y; \lambda) - \Phi^s_-(0, y; 0)|$. We do know that they individually decay like $e^{-\alpha y}$, but that is not sufficient. If you were to Taylor about $\lambda = 0$ you would start with an order $\lambda$ term, so we will assume we a bound of the form

\[
|\Phi^s_-(0, y; \lambda) - \Phi^s_-(0, y; 0)| \leq C |\lambda| e^{-\alpha y}
\]

which is the same $p_6$ above. It is at least reasonable, and will get us what we want, BUT WE SHOULD ACTUALLY SHOW IT AT SOME POINT. (The specific decay rate here does not really matter, just that it decays exponentially.) With this bound, the second integral above is order $|\lambda|$. For the first integral, we have 

\begin{align*}
\int_{-X_{i-1}}^0 \langle \Psi(0), \Phi^s_-(0, y; 0) \tilde{H}(y) \rangle dy &= 
\int_{-X_{i-1}}^0 \langle \Psi(y), H(y) \rangle dy + \int_{-X_{i-1}}^0 \langle \Psi(y), \Delta H(y) \rangle dy \\
&= \int_{-\infty}^0 \langle \Psi(y), H(y) \rangle dy - \int_{-\infty}^{-X_{i-1}} \langle \Psi(y), H(y) \rangle dy \\
&+ \int_{-X_{i-1}}^0 \langle \Psi(y), \Delta H(y) \rangle dy 
\end{align*}

The first integral is half of our Melnikov integral. The second is order $e^{-\alpha X_{i-1}}$. The third is order $e^{-\alpha X_1}$ (the order of $\Delta H$). When we do the ``plus'' piece, the first integral is the other half of the Melnikov integral; the second integral is order $e^{-\alpha X_i}$; and the third integral is the same order.
Thus these integral terms look like

\begin{align*}
d_i \lambda^2 \int_{-\infty}^\infty \langle \Psi(y), H(y) \rangle dy + \mathcal{O}( ( |\lambda|^3 + e^{-\alpha X_1} |\lambda|^2 )|d|)
\end{align*}

\end{enumerate}

\item ``Center'' integral terms

The ``minus'' integral term is

\begin{align*}
\langle \Psi(0) &, \int_{-X_{i-1}}^0
e^{\nu(\lambda)(y)} v_-(0; \lambda) \langle G_i^-(\lambda)(y)W_i^-(y) + \lambda^2 d_i \tilde{H}(y), w_-(y; \lambda) \rangle dy \rangle \\
&= \int_{-X_{i-1}}^0
e^{\nu(\lambda)(y)} \langle \Psi(0), v_-(0; \lambda) \rangle \langle G_i^-(\lambda)(y)W_i^-(y) + \lambda^2 d_i \tilde{H}(y), w_-(y; \lambda) \rangle dy 
\end{align*}

If we do the standard thing and write $\tilde{H} = H + \Delta H$, then these integrals give us three terms. The first one has bound

\begin{align*}
&\left| \int_{-X_{i-1}}^0
e^{\nu(\lambda)(y)} \langle \Psi(0), v_-(0; \lambda) \rangle \langle G_i^-(\lambda)(y)W_i^-(y), w_-(y; \lambda) \rangle dy \right| \\
&\leq C e^{\nu(\lambda) X_{i-1}} |G| ||W|| \\
&\leq C e^{\nu(\lambda) X_{i-1}} |G| e^{\nu(\lambda)k X_1} ( |\lambda|^2 + |D_1| )|d| \\
&\leq C e^{2 \nu(\lambda) k X_1} |G| ( |\lambda|^2 + |D_1| )|d|
\end{align*}

The second one (which is multiplied by $d_i \lambda^2$) has bound

\begin{align*}
\left| \int_{-X_{i-1}}^0
e^{\nu(\lambda)(y)} \langle \Psi(0), v_-(0; \lambda) \rangle \langle \Delta H(y), w_-(y; \lambda) \rangle dy \right| 
&\leq C e^{\nu(\lambda)X_{i-1}} |\Delta H| \\
&\leq C e^{\nu(\lambda)k X_1} e^{-\alpha X_1}
\end{align*}

Recalling what we did above, the third one (which is also multiplied by $d_i \lambda^2$) has bound

\begin{align*}
&\left| \int_{-X_{i-1}}^0 e^{\nu(\lambda)y} \langle \Psi(0), v_-(0; \lambda) \rangle
\langle H(y), w_-(y; \lambda) \rangle dy \right| \\
&\leq C |\langle \Psi(0), v_-(0; \lambda) \rangle| \int_{-X_{i-1}}^0 e^{\tilde{\alpha}y}e^{\nu(\lambda)y} | e^{-\tilde{\alpha} y} H(y)|dy\\
&\leq C |\langle \Psi(0), v_-(0; \lambda) \rangle|
\end{align*}

where $|e^{-\tilde{\alpha} y} H(y)|$ is bounded since we know the decay properties of $H$, and the integral is uniformly bounded in $X_{i-1}$ since $|\nu(\lambda)| < \tilde{\alpha}$. In order to get a better bound, we use the same trick we used above and expand $v_-(0; \lambda)$ as a Taylor series in $\lambda$ about $v_-(0; \lambda)$. Since $\langle \Psi(0), v_-(0; \lambda) \rangle = 0$ as discussed above and the coefficient of the $\lambda$ term is a constant not involving $\lambda$, this becomes (following what we did above)

\begin{align*}
\left| \int_{-X_{i-1}}^0 e^{\nu(\lambda)y} \langle \Psi(0), v_-(0; \lambda) \rangle
\langle H(y), w_-(y; \lambda) \rangle dy \right| 
&\leq C |\lambda| 
\end{align*}

The ``plus'' terms are similar.

\item Put all of this together\\

Now that we have bounds on all the things, we have our expression for the jump.

\begin{align*}
\langle \Psi(0), &W_1^-(0) - W_1^+(0) \rangle = 
-\langle \Psi(X_1), P^u_0 D_1 d \rangle + d_1 \lambda^2 \int_{-\infty}^\infty \langle \Psi(y), H(y) \rangle dy + R(\lambda)_1(d)
\end{align*}

where

\begin{align*}
R(\lambda)(d) &\leq C( e^{-\alpha X_1}( |\lambda|^2 + |D_1|)|d|) + e^{-\alpha k X_1} ( |\lambda|^2 + |D_1|) |d| \\
&+ p_3(\lambda) (|\lambda^2| + e^{\nu(\lambda)k X_1}|D_1|)|d| \\
&+ |\lambda| (|\lambda|^2 + e^{\nu(\lambda)k X_1} |D_1|)|d| \\
&+ |G| e^{\nu(\lambda)k X_1} ( |\lambda|^2 + |D_1| )|d| \\ 
&+ |\lambda|^3 + e^{-\alpha X_1} |\lambda|^2 )|d| \\
&+ e^{2 \nu(\lambda) k X_1} |G| ( |\lambda|^2 + |D_1| )|d| \\
&+ (e^{\nu(\lambda)k X_1} e^{-\alpha X_1} + |\lambda|)|\lambda|^2|d|\\
\end{align*}

BECAUSE OF WHAT WE MENTIONED ABOUT WITH THE INDIVIDUAL ELEMENTS THAT COMPRISE THIS, THIS BOUND MIGHT NOT BE NOT GOOD ENOUGH. THE $|\lambda|^2$ TERMS SHOULD BE OKAY, AS THEY ALSO APPEAR IN THE SINGLE PULSE CASE, SO THE SAME DEAL APPLIES HERE. THE $|D_1|$ TERMS MIGHT NOT BE OKAY SINCE, FOR EXAMPLE, WE HAVE $|\lambda| e^{\nu(\lambda)k X_1} |D_1|$, WHICH GROWS EXPONENTIALLY IN $X_1$.\\

Let's actually plug in $D_1$ and see if we can make anything of this. Recall that we have the following expression/bound for $D_1$ (which is a reasonable guess but unproven)

\[
D_1 = ( Q'(X_1) + Q'(-X_1 )(d_2 - d_1) + \mathcal{O} \left( e^{-\alpha X_1} \left( |\lambda| +  e^{-\alpha X_1}  \right) |d| \right)
\]

First, we substitute this into the term $\langle \Psi(X_1), P^u_0 D_1 d \rangle$. Using the fact that $\Psi(\pm X_1)$ is order $e^{-\alpha X_1}$, we have

\begin{align*}
\langle \Psi(X_1), P^u_0 D_1 d \rangle &= \langle \Psi(X_1), (Q'(X_1) + Q'(-X_1 )(d_2 - d_1 ) \rangle + \mathcal{O} \left( e^{-2 \alpha X_1} \left( |\lambda| +  e^{-\alpha X_1}  \right) |d| \right)
\end{align*}

Now we assume what was done in (3.36) in Sandstede (1998) applies here. We do need to verify this, since this was assumed in the exponentially weighted version as well. With this assumption, this becomes

\begin{align*}
\langle \Psi(X_1), P^u_0 D_1 d \rangle &= \langle \Psi(X_1), Q'(-X_1) \rangle (d_2 - d_1 ) + \mathcal{O} \left( e^{-2 \alpha X_1} \left( |\lambda| +  e^{-\alpha X_1}  \right) |d| \right)
\end{align*}

We want the first term on the RHS. The second term on the RHS will go into the remainder. For the remainder term, since we are using a bound on $D_1$ in its entirety (the bound includes the $Q'(\pm X_1)$ terms, not just the second part of the bound) we have

\[
|D_1| \leq C e^{-\alpha X_1}|d|
\]

Substituting this into the remainder, and adding in the additional term into the bound, we have

\begin{align*}
R(\lambda)(d) &\leq C( e^{-\alpha X_1}( |\lambda|^2 + e^{-\alpha X_1})|d|) + e^{-\alpha k X_1} ( |\lambda|^2 + e^{-\alpha X_1}) |d| \\
&+ p_3(\lambda) (|\lambda^2| + e^{\nu(\lambda)k X_1}e^{-\alpha X_1})|d| \\
&+ |\lambda| (|\lambda|^2 + e^{\nu(\lambda)k X_1} e^{-\alpha X_1})|d| \\
&+ |G| e^{\nu(\lambda)k X_1} ( |\lambda|^2 + e^{-\alpha X_1} )|d| \\ 
&+ (|\lambda|^3 + e^{-\alpha X_1} |\lambda|^2 )|d| \\
&+ e^{2 \nu(\lambda) k X_1} |G| ( |\lambda|^2 + e^{-\alpha X_1} )|d| \\
&+ (e^{\nu(\lambda)k X_1} e^{-\alpha X_1} + |\lambda|)|\lambda|^2|d|\\
&+ e^{-2 \alpha X_1} \left( |\lambda| +  e^{-\alpha X_1}  \right) |d|\\
&\leq C( ( e^{-(\alpha - \nu(\lambda) k)X_1 }) + p_3(\lambda) + |\lambda| + |G|e^{2 \nu(\lambda) k X_1})|\lambda|^2 \\ 
&+ e^{-2 \alpha X_1} + e^{-(\alpha - \nu(\lambda) k)X_1 }(p_3(\lambda) + |\lambda| + |G| e^{\nu(\lambda)k X_1}))|d|
\end{align*}

Thus we have 

\begin{align*}
\langle \Psi(0), &W_1^-(0) - W_1^+(0) \rangle = 
-\langle \Psi(X_1), Q'(-X_1) \rangle (d_2 - d_1 ) + d_1 \lambda^2 \int_{-\infty}^\infty \langle \Psi(y), H(y) \rangle dy + R(\lambda)_1(d)
\end{align*}

where 

\begin{align*}
|R_1(\lambda)(d)| &\leq C( ( e^{-(\alpha - \nu(\lambda) k)X_1 }) + p_3(\lambda) + |\lambda| + |G|e^{2 \nu(\lambda) k X_1})|\lambda|^2 \\ 
&+ e^{-2 \alpha X_1} + e^{-(\alpha - \nu(\lambda) k)X_1 }(p_3(\lambda) + |\lambda| + |G| e^{\nu(\lambda)k X_1}))|d|
\end{align*}

The good news is that everything in the remainder decays exponentially in $X_1$. The bad news that $\langle \Psi(X_1), Q'(-X_1) \rangle$ decays exponentially like $e^{-2 \alpha X_1}$, and there are terms in the remainder that decay at the same rate or even slower (like the $e^{-(\alpha - \nu(\lambda) k)X_1 } |\lambda|$ term).

\item Can we fix this?\\

Here is one thing we can try. Suppose we were able to show the equivalent of Lemma 2.1 in Sandstede (1998, also from 1993 thesis). THIS IS ENTIRELY WISHFUL THINKING. IN FACT, I AM NOT EVEN SURE THE IDEA MAKES SENSE SINCE $\lambda$ IS BOTH THE POTENTIAL EIGENVALUE AND THE PARAMETER.

\begin{lemma}There exists a smooth change of coordinates such that at the origin we have
\begin{enumerate}
	\item $W^s_{loc}(0; \lambda) \subset E^s$
	\item $W^u_{loc}(0; \lambda) \subset E^u$
	\item $v_0(\lambda) = v_0(0)$
\end{enumerate}
and near $q(0)$ we have for some $\delta > 0$
\begin{enumerate}
	\item $W^s(0; \lambda) \cap U_\delta(q(0) \subset q(0) + \R q'(0) \oplus Y^+ $
	\item $W^u(0; \lambda) \cap U_\delta(q(0) \subset q(0) + \R q'(0) \oplus Y^- $
	\item $v^+(0; \lambda), v^-(0; \lambda) \subset Y^0$
\end{enumerate}
\end{lemma}

The entire point of doing this is that our projections will kill many terms in their entirety instead of leaving a $\lambda$ depedent remainder. Also we may have forgotten this remainder a few times above, so how we don't have to worrk about it. In particular, we have the following.

\begin{align*}
D_i d &= a_i^+ - a_i^- + c_i^+ v_0(\lambda) - c_i^- v_0(\lambda) \\
&+ (P^u_+(X_i; \lambda) - P_0^u)a_i^+ - (P^s_-(-X_i; \lambda) - P_0^s)a_i^- \\
&+ \Phi^s_+(X_i, 0; \lambda)b_i^+ - \Phi^u_-(-X_i, 0; \lambda)b_{i-1}^- \\
&+ c_i^+ \Delta v_+(X_i; \lambda) + c_i^+ v_+(X_i; \lambda) \langle v_0(\lambda), \Delta w_+(X_i; \lambda) \rangle \\
&- c_i^- \Delta v_-(X_-; \lambda) - c_i^- v_-(-X_i; \lambda) \langle v_0(\lambda), \Delta w_-(-X_i; \lambda) \rangle \\
&+ \int_0^{X_i} \Phi^u_+(X_i, y; \lambda) [ G(\lambda)W_i^+(y) + d_i \lambda^2 \tilde{H}(y) ] dy \\
&- \int_0^{-X_i} \Phi^s_-(-X_i, y; \lambda) [ G(\lambda)W_{i-1}^-(y) + d_{i-1} \lambda^2 \tilde{H}(y) ] dy
\end{align*}

So when we do the projections and get $A_2$, both $c_1^\pm v_0(\lambda)$ terms die, so we have

\begin{align*}
a_1^+ &= -P^u_0 D_1 d + A_2(\lambda)_1^+(b, c_1^-, d))\\
a_1^- &=  P^s_0 D_1 d + A_2(\lambda)_1^-(b, c_1^-, d))
\end{align*}

where

\begin{align*}
|A_2&(\lambda)_1(b, c_1^-, d))| \\
&\leq C( (e^{-\alpha X_1} + |G|) |b| + p_2(\lambda; X_1) |c_1^-| + e^{\nu(\lambda)k X_1} |G||c_2^-| \\
&+ ((e^{-\tilde{\alpha} X_1} + |G|) |\lambda|^2 + (p_1(X_1; \lambda) + p_2(X_1; \lambda) + e^{\nu(\lambda)k X_1} |G|)|D_1|)|d| ) \\
\end{align*}

which improves the $c_1^-$ bound a little.\\

For the next step, where we solve for $B_1$, this takes $p_3$ and $p_4$ out of the picture, but since those are small and only needed to do the inversion, it only makes things simpler. The end result should be the same.\\

For the $A_4$ bound, this is now

\begin{align*}
|A_4&(\lambda)_1(d)| \\
&\leq C( (e^{-\alpha X_1} + |G|) (|\lambda^2| + e^{\nu(\lambda)k X_1}|D_1|)|d| \\ 
&+ p_2(\lambda; X_1) e^{-|\nu(\lambda)| X_1} (|\lambda^2| + e^{\nu(\lambda)X_1}|D_1|)|d| \\
&+ ((e^{-\tilde{\alpha} X_1} + e^{\nu(\lambda)k X_1} |G|) |\lambda|^2 + (p_1(X_1; \lambda) + p_2(X_1; \lambda) + e^{\nu(\lambda) k X_1} |G|)|D_1|)|d| )\\
&\leq C(|\lambda|^2 + (p_1(X_1; \lambda) + p_2(X_1; \lambda) + e^{\nu(\lambda) k X_1} |G|)|D_1|) |d|
\end{align*}

which is only slightly better than what we had before; i.e. instead of $|D_1|$ alone, it is multiplied by stuff which is order $\lambda$.\\

When we do the projection with $\Psi(0)$ to find the jump, the change of coordinates lets us get ride of all the $b_i$ terms in their entirety, which is good, so one set of nuisance terms is now gone. That leaves the $c_i$ terms to deal with. Assuming this change of coordinates works, since the $c_i$ are now in the direction of $v_\pm(0; \lambda)$ which has been coordinate-changed to be in the span of $Y^0$, this term dies as well, which is awesome. \\

Let's look at again at the terms involving $a$. For $i = 1$ we have

\begin{align*}
\langle \Psi(0), &\Phi^u_+(0, X_i; \lambda) a_i^+ \rangle \\
&= -\langle \Psi(0), \Phi^u_+(0, X_1; 0) P^u_0 D_1 d \rangle + \mathcal{O}( p_6(X_1; \lambda) e^{-\alpha X_1}|D_1||d| \\
&+ e^{-\alpha X_1} (|\lambda|^2 + (p_1(X_1; \lambda) + p_2(X_1; \lambda) + e^{\nu(\lambda) k X_1} |G|)|D_1|) |d|) \\
&= -\langle \Psi(X_1), P^u_0 D_1 d \rangle + \mathcal{O}( e^{-\alpha X_1}( |\lambda|^2 + (p_1(X_1; \lambda) + p_2(X_1; \lambda) + e^{\nu(\lambda) k X_1} |G|)|D_1|)|d|)
\end{align*}

This might not be quite good enough, since it will give us an order $e^{-\alpha X_1} |\lambda||D_1|$, but it is better than what we had before.\\

For the $a_{i-1}$ term, we just use $A_3$.

\begin{align*}
\langle \Psi(0), &\Phi^s_-(0, -X_{i-1}; \lambda)a_{i-1}^- \rangle \\
&\leq C e^{-\alpha k X_1} ( |\lambda|^2 + |D_1|) |d|
\end{align*}

This should be okay as long as $k > 1$, which we have assumed.\\

\item A better bound for $W$, maybe?.\\

Here we try to adapt a lemma from the weighted case.\\

Since we have 
\[
W_i^\pm = L_1(\lambda)W_i^\pm + L_2(\lambda)W_i^\pm 
\]

we can use the usual estimate for $L_1(\lambda)$ together with an improved estimate for $L_2(\lambda)$. This essentially involves leaving the $a$ bound in terms of $x$. We will do the negative piece here. This will give us a bound 

\begin{align*}
| W_i^-(x)| &\leq C \left( e^{-\alpha^s(x + X_{i-1})} |a^-_{i-1}| + |b_i^-| + e^{\nu(\lambda)X_1}|c_1| + e^{\nu(\lambda)X_2}|c_2| + (e^{\nu(\lambda)X_2} |G| + |\lambda|)||W|| + |\lambda|^2 |d|  \right)
\end{align*}

Substituting in $A_3(\lambda)$, $B_1(\lambda)$, and $W_3(\lambda)$, this estimate becomes

\begin{align}
|B_1(\lambda)(d)|_1 \leq C(|\lambda^2| + e^{\nu(\lambda) X_1}|D_1|)|d| \\
|B_1(\lambda)(d)|_2 \leq C(|\lambda^2| + e^{\nu(\lambda)k X_1}|D_1|)|d|
\end{align}

\begin{align*}
|A_3&(\lambda)_1(b, c^-, d)| \leq C( e^{-|\nu(\lambda)|X_1} |\lambda|^2 + |D_1|) |d| 
\end{align*}

\begin{align*}
|A_3&(\lambda)_2(b, c^-, d)| \leq C( e^{-|\nu(\lambda)| k X_1} |\lambda|^2 + |D_1|) |d|
\end{align*}

\begin{align*}
||W_3&(\lambda)(b,c^-,d)|| \leq C e^{\nu(\lambda)k X_1} ( |\lambda|^2 + |D_1| )|d|
\end{align*}

\begin{align*}
| W_i^-(x)| &\leq C ( e^{-\alpha^s(x + X_{i-1})} ( |\lambda|^2 + |D_1|) |d| \\
&+ (|\lambda^2| + e^{\nu(\lambda)k X_1}|D_1|)|d| \\
&+ ( |\lambda|^2 + e^{\nu(\lambda)k X_1} |D_1|) |d| \\
&+ (e^{\nu(\lambda)k X_1} |G| + |\lambda|) e^{\nu(\lambda)k X_1} ( |\lambda|^2 + |D_1| )|d| \\
&+ |\lambda|^2 |d| ) \\
&\leq C(|\lambda|^2 + e^{\nu(\lambda)k X_1}|D_1|)|d|  )
\end{align*}

This unfortunately is not a significant improvement.


\item This makes our remainder term have bound

\begin{align*}
R(\lambda)(d) &\leq C( e^{-\alpha X_1}( |\lambda|^2 + |D_1|)|d|) + e^{-\alpha k X_1} ( |\lambda|^2 + |D_1|) |d| \\
&+ |G| e^{\nu(\lambda)k X_1} ( |\lambda|^2 + |D_1| )|d| \\ 
&+ |\lambda|^3 + e^{-\alpha X_1} |\lambda|^2 )|d| \\
&+ e^{2 \nu(\lambda) k X_1} |G| ( |\lambda|^2 + |D_1| )|d| \\
&+ (e^{\nu(\lambda)k X_1} e^{-\alpha X_1} + |\lambda|)|\lambda|^2|d|\\
\end{align*}

\end{enumerate}

\end{document}